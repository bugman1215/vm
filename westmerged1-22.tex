  \begin{pinyinscope}{\myfontt \section{第一回}     靈根育孕源流出 心性修持大道生


  詩曰:
    混沌未分天地亂,茫茫渺渺無人見。
    自從盤古破鴻濛,開闢從茲清濁辨。
    覆載群生仰至仁,發明萬物皆成善。
    欲知造化會元功,須看西遊釋厄傳。


蓋聞天地之數,有十二萬九千六百歲為一元。將一元分為十二會,乃子、丑、寅
、卯、辰、巳、午、未、申、酉、戌、亥之十二支也。每會該一萬八百歲。且就
一日而論:子時得陽氣,而丑則雞鳴﹔寅不通光,而卯則日出﹔辰時食後,而巳
則挨排﹔日午天中,而未則西蹉﹔申時晡,而日落酉,戌黃昏,而人定亥。譬於
大數,若到戌會之終,則天地昏曚而萬物否矣。再去五千四百歲,交亥會之初,
則當黑暗,而兩間人物俱無矣,故曰混沌。又五千四百歲,亥會將終,貞下起元
,近子之會,而復逐漸開明。邵康節曰::「冬至子之半,天心無改移。一陽初
動處,萬物未生時。」到此,天始有根。再五千四百歲,正當子會,輕清上騰,
有日,有月,有星,有辰。日、月、星、辰,謂之四象。故曰,天開於子。又經
五千四百歲,子會將終,近丑之會,而逐漸堅實。《易》曰:「大哉乾元!至哉
坤元!萬物資生,乃順承天。」至此,地始凝結。再五千四百歲,正當丑會,重
濁下凝,有水,有火,有山,有石,有土。水、火、山、石、土,謂之五形。故
曰,地闢於丑。又經五千四百歲,丑會終而寅會之初,發生萬物。曆曰:「天氣
下降,地氣上升﹔天地交合,群物皆生。」至此,天清地爽,陰陽交合。再五千
四百歲,子會將終,近丑之會,而逐漸堅實。《易》曰:「大哉乾元!至哉坤元
!萬物資生,乃順承天。」至此,地始凝結。再五千四百歲,正當丑會,重濁下
凝,有水,有火,有山,有石,有土。水、火、山、石、土,謂之五形。故曰,
地闢於丑。又經五千四百歲,丑會終而寅會之初,發生萬物。曆曰:「天氣下降
,地氣上升﹔天地交合,群物皆生。」至此,天清地爽,陰陽交合。再五千四百
歲,正當寅會,生人,生獸,生禽,正謂天地人,三才定位。故曰,人生於寅。

感盤古開闢,三皇治世,五帝定倫,世界之間,遂分為四大部洲:曰東勝神洲,
曰西牛賀洲,曰南贍部洲,曰北俱蘆洲。這部書單表東勝神洲。海外有一國土,
名曰傲來國。國近大海,海中有一座名山,喚為花果山。此山乃十洲之祖脈,三
島之來龍,自開清濁而立,鴻濛判後而成。真個好山!有詞賦為證。賦曰:勢鎮
汪洋,威寧瑤海。勢鎮汪洋,潮湧銀山魚入穴﹔威寧瑤海,波翻雪浪蜃離淵。水
火方隅高積上,東海之處聳崇巔。丹崖怪石,削壁奇峰。丹崖上,彩鳳雙鳴﹔削
壁前,麒麟獨臥。峰頭時聽錦雞鳴,石窟每觀龍出入。林中有壽鹿仙狐,樹上有
靈禽玄鶴。瑤草奇花不謝,青松翠柏長春。仙桃常結果,修竹每留雲。一條澗壑
籐蘿密,四面原堤草色新。正是百川會處擎天柱,萬劫無移大地根。

那座山正當頂上,有一塊仙石。其石有三丈六尺五寸高,有二丈四尺圍圓。三丈
六尺五寸高,按周天三百六十五度﹔二丈四尺圍圓,按政曆二十四氣。上有九竅
八孔,按九宮八卦。四面更無樹木遮陰,左右倒有芝蘭相襯。

蓋自開闢以來,每受天真地秀,日精月華,感之既久,遂有靈通之意。內育仙胞
,一日迸裂,產一石卵,似圓毬樣大。因見風,化作一個石猴,五官俱備,四肢
皆全。便就學爬學走,拜了四方。目運兩道金光,射沖斗府。驚動高天上聖大慈
仁者玉皇大天尊玄穹高上帝,駕座金闕雲宮靈霄寶殿,聚集仙卿,見有金光燄燄
,即命千里眼、順風耳開南天門觀看。二將果奉旨出門外,看的真,聽的明。須
臾回報道:「臣奉旨觀聽金光之處,乃東勝神洲海東傲來小國之界,有一座花果
山,山上有一仙石,石產一卵,見風化一石猴,在那裏拜四方,眼運金光,射沖
斗府。如今服餌水食,金光將潛息矣。」玉帝垂賜恩慈曰:「下方之物,乃天地
精華所生,不足為異。」

那猴在山中,卻會行走跳躍,食草木,飲澗泉,採山花,覓樹果﹔與狼蟲為伴,
虎豹為群,獐鹿為友,獼猿為親﹔夜宿石崖之下,朝遊峰洞之中。真是:「山中
無甲子,寒盡不知年。」
  一朝天氣炎熱,與群猴避暑,都在松陰之下頑耍。你看他一個個:
跳樹攀枝,採花覓果﹔拋彈子,?麼兒﹔跑沙窩,砌寶塔﹔趕蜻蜓,撲蜡﹔參老
天,拜菩薩﹔扯葛籐,編草﹔捉虱子,咬又掐﹔理毛衣,剔指甲。挨的挨,擦的
擦﹔推的推,壓的壓﹔扯的扯,拉的拉:青松林下任他頑,綠水澗邊隨洗濯。

一群猴子耍了一會,卻去那山澗中洗澡。見那股澗水奔流,真個似滾瓜湧濺。古
云:「禽有禽言,獸有獸語。」眾猴都道:「這股水不知是那裏的水。我們今日
趕閑無事,順澗邊往上溜頭尋看源流,耍子去耶!」喊一聲,都拖男挈女,呼弟
呼兄,一齊跑來,順澗爬山,直至源流之處,乃是一股瀑布飛泉。但見那:
一派白虹起,千尋雪浪飛。
    海風吹不斷,江月照還依。
    冷氣分青嶂,餘流潤翠微。
    潺湲名瀑布,真似掛簾帷。

眾猴拍手稱揚道:「好水,好水!原來此處遠通山腳之下,直接大海之波。」又
道:「那一個有本事的,鑽進去尋個源頭出來,不傷身體者,我等即拜他為王。」
連呼了三聲,忽見叢雜中跳出一個石猴,應聲高叫道:「我進去,我進去。」好
猴!也是他:
        今日芳名顯,時來大運通。
    有緣居此地,王遣入仙宮。

你看他瞑目蹲身,將身一縱,徑跳入瀑布泉中,忽睜睛抬頭觀看,那裏邊卻無水
無波,明明朗朗的一架橋梁。他住了身,定了神,仔細再看,原來是座鐵板橋。
橋下之水,沖貫於石竅之間,倒掛流出去,遮閉了橋門。卻又欠身上橋頭,再走
再看,卻似有人家住處一般,真個好所在。但見那:

翠蘚堆藍,白雲浮玉,光搖片片煙霞。虛窗靜室,滑凳板生花。乳窟龍珠倚掛,
縈迴滿地奇葩。鍋灶傍崖存火跡,樽罍靠案見殽渣。石座石床真可愛,石盆石碗
更堪誇。又見那一竿兩竿修竹,三點五點梅花。幾樹青松常帶雨,渾然像個人家。

看罷多時,跳過橋中間,左右觀看。只見正當中有一石碣,碣上有一行楷書大字
,鐫著「花果山福地,水簾洞洞天」。

石猿喜不自勝,急抽身往外便走,復瞑目蹲身,跳出水外,打了兩個呵呵道:
「大造化!大造化!」眾猴把他圍住,問道:「裏面怎麼樣?水有多深?」石猴
道:「沒水!沒水!原來是一座鐵板橋,橋那邊是一座天造地設的家當。」眾猴
道:「怎見得是個家當?」石猴笑道:「這股水乃是橋下沖貫石橋,倒掛下來遮
閉門戶的。橋邊有花有樹,乃是一座石房。房內有石窩、石灶、石碗、石盆、石
床、石凳。中間一塊石碣上,鐫著『花果山福地,水簾洞洞天』。真個是我們安
身之處。裏面且是寬闊,容得千百口老小。
我們都進去住,也省得受老天之氣。這裏邊:
    刮風有處躲,下雨好存身。
    霜雪全無懼,雷聲永不聞。
    煙霞常照耀,祥瑞每蒸熏。
    松竹年年秀,奇花日日新。」

眾猴聽得,個個歡喜。都道:「你還先走,帶我們進去,進去。」石猴卻又瞑目
蹲身,往裏一跳,叫道:「都隨我進來,進來。」那些猴有膽大的,都跳進去了
﹔膽小的,一個個伸頭縮頸,抓耳撓腮,大聲叫喊,纏一會,也都進去了。跳過
橋頭,一個個搶盆奪碗,佔灶爭床,搬過來,移過去,正是猴性頑劣,再無一個
寧時,只搬得力倦神疲方止。石猿端坐上面道:「列位呵,『人而無信,不知其
可。』你們才說有本事進得來,出得去,不傷身體者,就拜他為王。我如今進來
又出去,出去又進來,尋了這一個洞天與列位安眠穩睡,各享成家之福,何不拜
我為王?」眾猴聽說,即拱伏無違,一個個序齒排班,朝上禮拜,都稱「千歲大
王」。自此,石猿高登王位,將「石」字兒隱了,遂稱「美猴王」。有詩為證。
詩曰:
    三陽交泰產群生,仙石胞含日月精。
    借卵化猴完大道,假他名姓配丹成。
    內觀不識因無相,外合明知作有形。
    歷代人人皆屬此,稱王稱聖任縱橫。

美猴王領一群猿猴、獼猴、馬猴等,分派了君臣佐使。朝遊花果山,暮宿水簾洞
,合契同情,不入飛鳥之叢,不從走獸之類,獨自為王,不勝歡樂。是以:
    春採百花為飲食,夏尋諸果作生涯。
    秋收芋栗延時節,冬覓黃精度歲華。

美猴王享樂天真,何期有三五百載。一日,與群猴喜宴之間,忽然憂惱,墮下淚
來。眾猴慌忙羅拜道:「大王何為煩惱?」猴王道:「我雖在歡喜之時,卻有一
點兒遠慮,故此煩惱。」眾猴又笑道:「大王好不知足。我等日日歡會,在仙山
福地,古洞神洲,不伏麒麟轄,不伏鳳凰管,又不伏人間王位所拘束,自由自在
,乃無量之福,為何遠慮而憂也?」猴王道:「今日雖不歸人王法律,不懼禽獸
威嚴,將來年老血衰,暗中有閻王老子管著,一旦身亡,可不枉生世界之中,不
得久注天人之內?」眾猴聞此言,一個個掩面悲啼,俱以無常為慮。

只見那班部中,忽跳出一個通背猿猴,厲聲高叫道:「大王若是這般遠慮,真所
謂道心開發也。如今五蟲之內,惟有三等名色不伏閻王老子所管。」猴王道:
「你知那三等人?」猿猴道:「乃是佛與仙與神聖三者,躲過輪迴,不生不滅,
與天地山川齊壽。」猴王道:「此三者居於何所?」猿猴道:「他只在閻浮世界
之中,古洞仙山之內。」猴王聞之,滿心歡喜道:「我明日就辭汝等下山,雲遊
海角,遠涉天涯,務必訪此三者,學一個不老長生,常躲過閻君之難。」噫!這
句話,頓教跳出輪迴網,致使齊天大聖成。眾猴鼓掌稱揚,都道:「善哉,善哉
!我等明日越嶺登山,廣尋些果品,大設筵宴送大王也。」
  次日,眾猴果去採仙桃,摘異果,刨山藥,斸黃精。芝蘭香蕙,瑤草奇花,
般般件件,整整齊齊,擺開石凳石桌,排列仙酒仙殽。但見那:
金丸珠彈,紅綻黃肥。金丸珠彈臘櫻桃,色真甘美﹔紅綻黃肥熟梅子,味果香酸
。鮮龍眼,肉甜皮薄﹔火荔枝,核小囊紅。林檎碧實連枝獻,枇杷緗苞帶葉擎。
兔頭梨子雞心棗,消渴除煩更解酲。香桃爛杏,美甘甘似玉液瓊漿﹔脆李楊梅,
酸蔭蔭如脂酥膏酪。紅囊黑子熟西瓜,四瓣黃皮大柿子。石榴裂破,丹砂粒現火
晶珠﹔芋栗剖開,堅硬肉團金瑪瑙。胡桃銀杏可傳茶,椰子葡萄能做酒。榛松榧
柰滿盤盛,橘蔗柑橙盈案擺。熟煨山藥,爛煮黃精。搗碎茯苓並薏苡,石鍋微火
漫炊羹。人間縱有珍饈味,怎比山猴樂更寧。

群猴尊美猴王上坐,各依齒肩排於下邊,一個個輪流上前奉酒、奉花、奉果,痛
飲了一日。

次日,美猴王早起,教:「小的們,替我折些枯松,編作?子,取個竹竿作篙,
收拾些果品之類,我將去也。」果獨自登?,儘力撐開,飄飄蕩蕩,徑向大海波
中,趁天風,來渡南贍部洲地界。這一去,正是那:
        天產仙猴道行隆,離山駕?趁天風。
    飄洋過海尋仙道,立志潛心建大功。
    有分有緣休俗願,無憂無慮會元龍。
    料應必遇知音者,說破源流萬法通。

也是他運至時來,自登木?之後,連日東南風緊,將他送到西北岸前,乃是南贍
部洲地界。持篙試水,偶得淺水,棄了?子,跳上岸來。只見海邊有人捕魚、打
雁、穵蛤、淘鹽。他走近前,弄個把戲,妝個虎,嚇得那些人丟筐棄網,四散奔
跑。將那跑不動的拿住一個,剝了他衣裳,也學人穿在身上。搖搖擺擺,穿州過
府,在市廛中學人禮,學人話。朝餐夜宿,一心裏訪問佛、仙、神聖之道,覓個
長生不老之方。見世人都是為名為利之徒,更無一個為身命者。正是那:
    爭名奪利幾時休?早起遲眠不自由!
    騎著驢騾思駿馬,官居宰相望王侯。
    只愁衣食耽勞碌,何怕閻君就取勾。
    繼子蔭孫圖富貴,更無一個肯回頭。

猴王參訪仙道,無緣得遇。在於南贍部洲,串長城,遊小縣,不覺八九年餘。忽
行至西洋大海,他想著海外必有神仙。獨自個依前作?,又飄過西海,直至西牛
賀洲地界。登岸遍訪多時,忽見一座高山秀麗,林麓幽深。他也不怕狼蟲,不懼
虎豹,登上山頂上觀看。果是好山:
千峰排戟,萬仞開屏。日映嵐光輕鎖翠,雨收黛色冷含青。瘦籐纏老樹,古渡界
幽程。奇花瑞草,修竹喬松。修竹喬松,萬載常青欺福地﹔奇花瑞草,四時不謝
賽蓬瀛。幽鳥啼聲近,源泉響溜清。重重谷壑芝蘭繞,處處巉崖苔蘚生。起伏巒
頭龍脈好,必有高人隱姓名。

正觀看間,忽聞得林深之處有人言語。急忙趨步,穿入林中,側耳而聽,原來是
歌唱之聲。歌曰:
「觀棋柯爛,伐木丁丁,雲邊谷口徐行。賣薪沽酒,狂笑自陶情。蒼逕秋高,對
月枕松根,一覺天明。認舊林,登崖過嶺,持斧斷枯籐。收來成一擔,行歌市上
,易米三升。更無些子爭競,時價平平。不會機謀巧算,沒榮辱,恬淡延生。相
逢處,非仙即道,靜坐講黃庭。」

美猴王聽得此言,滿心歡喜道:「神仙原來藏在這裏!」即忙跳入裏面,仔細再
看,乃是一個樵子,在那裏舉斧砍柴。但看他打扮非常:

頭上戴箬笠,乃是新筍初脫之籜。身上穿布衣,乃是木綿撚就之紗。腰間繫環絛
,乃是老蠶口吐之絲。足下踏草履,乃是枯莎槎就之爽。手執?鋼斧,擔挽火麻
繩。扳松劈枯樹,爭似此樵能。

猴王近前叫道:「老神仙,弟子起手。」那樵漢慌忙丟了斧,轉身答禮道:「不
當人,不當人。我拙漢衣食不全,怎敢當『神仙』二字?」猴王道:「你不是神
仙,如何說出神仙的話來?」樵夫道:「我說甚麼神仙話?」猴王道:「我才來
至林邊,只聽的你說:『相逢處,非仙即道,靜坐講《黃庭》。』《黃庭》乃道
德真言,非神仙而何?」樵夫笑道:「實不瞞你說,這個詞名做《滿庭芳》,乃
一神仙教我的。那神仙與我舍下相鄰,他見我家事勞苦,日常煩惱,教我遇煩惱
時,即把這詞兒念念,一則散心,二則解困。我才有些不足處思慮,故此念念,
不期被你聽了。」猴王道:「你家既與神仙相鄰,何不從他修行?學得個不老之
方,卻不是好?」樵夫道:「我一生命苦:自幼蒙父母養育至八九歲,才知人事
,不幸父喪,母親居孀。再無兄弟姊妹,只我一人,沒奈何,早晚侍奉。如今母
老,一發不敢拋離。卻又田園荒蕪,衣食不足,只得斫兩束柴薪,挑向市廛之間
,貨幾文錢,糴幾升米,自炊自造,安排些茶飯,供養老母。所以不能修行。」

猴王道:「據你說起來,乃是一個行孝的君子,向後必有好處。但望你指與我那
神仙住處,卻好拜訪去也。」樵夫道:「不遠,不遠。此山叫做靈臺方寸山,山
中有座斜月三星洞,那洞中有一個神仙,稱名須菩提祖師。那祖師出去的徒弟,
也不計其數,見今還有三四十人從他修行。你順那條小路兒,向南行七八里遠近
,即是他家了。」猴王用手扯住樵夫道:「老兄,你便同我去去,若還得了好處
,決不忘你指引之恩。」樵夫道:「你這漢子甚不通變,我方才這般與你說了,
你還不省?假若我與你去了,卻不誤了我的生意?老母何人奉養?我要斫柴,你
自去,自去。」

猴王聽說,只得相辭。出深林,找上路徑,過一山坡,約有七八里遠,果然望見
一座洞府。挺身觀看,真好去處!但見:
煙霞散彩,日月搖光。千株老柏,萬節修篁。千株老柏,帶雨半空青冉冉﹔萬節
修篁,含煙一壑色蒼蒼。門外奇花佈錦,橋邊瑤草噴香。石崖突兀青苔潤,懸壁
高張翠蘚長。時聞仙鶴唳,每見鳳凰翔。仙鶴唳時,聲振九皋霄漢遠﹔鳳凰翔起
,翎毛五色彩雲光。玄猿白鹿隨隱見,金獅玉象任行藏。細觀靈福地,真個賽天
堂。

又見那洞門緊閉,靜悄悄杳無人跡。忽回頭,見崖頭立一石碑,約有三丈餘高,
八尺餘闊,上有一行十個大字,乃是「靈臺方寸山,斜月三星洞」。美猴王十分
歡喜道:「此間人果是樸實,果有此山此洞。」看勾多時,不敢敲門。且去跳上
松枝梢頭,摘松子吃了頑耍。

少頃間,只聽得呀的一聲,洞門開處,裏面走出一個仙童,真個丰姿英偉,像貌
清奇,比尋常俗子不同。但見他:
    髽髻雙絲綰,寬袍兩袖風。
    貌和身自別,心與相俱空。
    物外長年客,山中永壽童。
    一塵全不染,甲子任翻騰。

那童子出得門來,高叫道:「甚麼人在此搔擾?」猴王撲的跳下樹來,上前躬身
道:「仙童,我是個訪道學仙之弟子,更不敢在此搔擾。」仙童笑道:「你是個
訪道的麼?」猴王道:「是。」童子道:「我家師父正才下榻,登壇講道,還未
說出原由,就教我出來開門。說:『外面有個修行的來了,可去接待接待。』想
必就是你了?」猴王笑道:「是我,是我。」童子道:「你跟我進來。」

這猴王整衣端肅,隨童子徑入洞天深處觀看:一層層深閣瓊樓,一進進珠宮貝闕
,說不盡那靜室幽居。直至瑤臺之下,見那菩提祖師端坐在臺上,兩邊有三十個
小仙侍立臺下。果然是:
大覺金仙沒垢姿,西方妙相祖菩提。不生不滅三三行,全氣全神萬萬慈。空寂自
然隨變化,真如本性任為之。與天同壽莊嚴體,歷劫明心大法師。

美猴王一見,倒身下拜,磕頭不計其數,口中只道:「師父,師父,我弟子志心
朝禮,志心朝禮。」祖師道:「你是那方人氏?且說個鄉貫、姓名明白,再拜。」
猴王道:「弟子乃東勝神洲傲來國花果山水簾洞人氏。」祖師喝令:「趕出去!
他本是個撒詐搗虛之徒,那裏修甚麼道果!」猴王慌忙磕頭不住道:「弟子是老
實之言,決無虛詐。」祖師道:「你既老實,怎麼說東勝神洲?那去處到我這裏
隔兩重大海,一座南贍部洲,如何就得到此?」猴王叩頭道:「弟子飄洋過海,
登界遊方,有十數個年頭,方才訪到此處。」

祖師道:「既是逐漸行來的也罷。你姓甚麼?」猴王又道:「我無性。人若罵我
,我也不惱﹔若打我,我也不嗔。只是陪個禮兒就罷了。一生無性。」祖師道:
「不是這個性。你父母原來姓甚麼?」猴王道:「我也無父母。」祖師道:「既
無父母,想是樹上生的?」猴王道:「我雖不是樹上生,卻是石裏長的。我只記
得花果山上有一塊仙石,其年石破,我便生也。」祖師聞言暗喜,道:「這等說
,卻是個天地生成的。你起來走走我看。」猴王縱身跳起,拐呀拐的走了兩遍。
祖師笑道:「你身軀雖是鄙陋,卻像個食松果的猢猻。我與你就身上取個姓氏,
意思教你姓『猢』。猢字去了個獸傍,乃是個古月。古者,老也﹔月者,陰也。
老陰不能化育,教你姓『猻』倒好。猻字去了獸傍,乃是個子系。子者,兒男也﹔
系者。嬰細也,正合嬰兒之本論。教你姓『孫』罷。」猴王聽說,滿心歡喜,朝
上叩頭道:「好!好!好!今日方知姓也。萬望師父慈悲,既然有姓,再乞賜個
名字,卻好呼喚。」祖師道:「我門中有十二個字,分派起名,到你乃第十輩之
小徒矣。」猴王道:「那十二個字?」祖師道:「乃廣、大、智、慧、真、如、
性、海、穎、悟、圓、覺十二字。排到你,正當『悟』字。與你起個法名叫做
『孫悟空』,好麼?」猴王笑道:「好!好!好!自今就叫做孫悟空也。」正是:
    鴻濛初闢原無姓,打破頑空須悟空。
  畢竟不知向後修些甚麼道果,且聽下回分解。





}  \end{pinyinscope}\switchcolumn{\myfontc \section{第 一 回} 灵 根 生 育 , 源 流 出 , 心 性 修 持 大 道 生 诗 道 : 混 沌 未 分 天 地 混 乱 , 茫 茫 渺 渺 无 人 见 。
自 从 盘 古 破 鸿 濛 , 开 辟 开 辟 从 此 清 浊 分 辨 。
覆 载 众 生 , 仰 慕 至 仁 , 发 明 万 物 皆 成 善 。
想 知 道 造 化 会 元 功 , 必 须 看 《 西 游 释 厄 传 》 。
听 说 天 地 之 数 , 有 十 二 万 九 千 六 百 年 为 一 元 。
将 一 元 分 为 十 二 会 , 就 是 子 、 丑 、 寅 、 卯 、 辰 、 巳 、 午 、 未 、 申 、 酉 、 戌 、 亥 、 亥 的 十 二 支 。
每 会 该 一 万 零 八 百 年 。
况 且 从 一 天 来 说 , 子 时 得 阳 气 , 丑 时 就 鸡 鸣 , 寅 时 不 通 光 , 卯 时 就 出 来 ; 辰 时 食 后 , 巳 时 就 挨 排 ; 午 时 在 天 中 , 而 未 时 就 向 西 斜 ; 申 时 偏 西 , 而 太 阳 落 在 酉 时 , 戌 时 黄 昏 时 人 定 定 在 亥 时 。
譬 如 大 数 , 如 果 到 了 戌 会 的 结 束 , 那 么 天 地 昏 暗 昏 暗 , 万 物 就 不 存 在 了 。
再 去 五 千 四 百 年 , 在 亥 会 的 初 度 , 则 该 是 黑 暗 , 而 两 间 的 人 物 都 没 有 了 , 所 以 称 为 混 沌 。
又 五 千 四 百 年 , 亥 会 将 要 终 结 , 贞 下 起 元 , 接 近 于 子 的 会 合 , 而 且 又 逐 渐 开 明 。
邵 康 节 说 : 冬 至 子 之 半 , 天 心 没 有 改 变 。
一 阳 初 动 的 地 方 , 万 物 未 生 的 时 候 。
到 了 这 里 , 天 才 有 根 。
再 过 五 千 四 百 年 , 正 当 子 会 , 轻 浮 上 升 , 有 日 、 月 、 有 星 、 辰 。
日 、 月 、 星 、 辰 , 称 为 四 象 。
所 以 说 , 上 天 开 创 于 子 女 。
又 经 过 五 千 四 百 年 , 子 会 将 要 结 束 , 接 近 丑 的 会 聚 , 而 且 逐 渐 坚 实 。
《 易 经 》 上 说 : 乾 元 , 至 坤 元 , 万 物 资 生 , 顺 承 天 意 。
到 了 这 里 , 地 才 凝 结 。
再 过 五 千 四 百 年 , 正 当 丑 会 , 重 浊 下 凝 , 有 水 , 有 火 , 有 山 , 有 石 , 有 土 。
水 、 火 、 山 、 石 、 土 , 称 为 五 种 形 状 。
所 以 说 , 地 辟 于 丑 。
又 经 过 五 千 四 百 年 , 丑 会 终 结 , 寅 会 的 开 始 , 发 生 万 物 。
《 历 》 上 说 : 天 气 下 降 , 地 气 上 升 , 天 地 交 合 , 万 物 都 生 长 。
到 这 里 , 天 清 地 爽 , 阴 阳 交 合 。
再 过 五 千 四 百 年 , 子 会 将 要 死 , 近 于 丑 的 会 聚 , 而 且 逐 渐 坚 实 。
《 易 经 》 上 说 : 乾 元 , 至 坤 元 , 万 物 资 生 , 顺 承 天 意 。
到 了 这 里 , 地 才 凝 结 。
再 过 五 千 四 百 年 , 正 当 丑 会 , 重 浊 下 凝 , 有 水 , 有 火 , 有 山 , 有 石 , 有 土 。
水 、 火 、 山 、 石 、 土 , 称 为 五 种 形 状 。
所 以 说 , 地 辟 于 丑 。
又 经 过 五 千 四 百 年 , 丑 会 终 结 , 寅 会 的 开 始 , 发 生 万 物 。
《 历 》 上 说 : 天 气 下 降 , 地 气 上 升 , 天 地 交 合 , 万 物 都 生 长 。
到 这 里 , 天 清 地 爽 , 阴 阳 交 合 。
再 过 五 千 四 百 年 , 正 当 寅 时 会 , 生 人 , 生 兽 , 生 禽 , 正 是 天 地 人 , 三 才 确 定 位 置 。
所 以 说 , 人 生 在 寅 。
感 念 盘 古 开 辟 , 三 皇 治 理 天 下 , 五 帝 定 伦 , 世 界 之 间 , 就 分 为 四 大 部 洲 : 东 胜 神 洲 , 西 牛 贺 洲 , 南 赡 部 洲 , 北 俱 芦 洲 。
其 书 单 表 东 胜 神 洲 。
海 外 有 一 个 国 土 , 名 叫 傲 来 国 。
国 家 靠 近 大 海 , 海 中 有 一 座 名 山 , 叫 它 花 果 山 。
此 山 是 十 洲 的 祖 脉 , 三 岛 的 来 龙 , 自 从 开 凿 出 来 清 浊 而 立 起 来 , 洪 濛 判 断 后 才 形 成 。
真 是 个 好 山 啊 有 词 赋 作 为 证 据 。
赋 说 : 威 势 镇 守 海 洋 , 威 力 安 定 瑶 海 。
威 宁 瑶 海 , 波 浪 翻 涌 , 蜃 离 渊 。
水 火 方 位 , 高 高 堆 积 在 上 面 , 东 海 的 地 方 高 耸 在 高 峰 顶 上 。
丹 崖 怪 石 , 陡 峭 的 崖 壁 和 奇 峰 。
丹 崖 上 , 彩 凤 双 双 呜 叫 , 陡 峭 的 石 壁 前 , 麒 麟 独 自 躺 在 那 里 。
峰 头 时 常 听 到 锦 鸡 鸣 叫 , 石 窟 中 常 常 看 到 龙 出 入 。
林 中 有 一 只 寿 鹿 、 仙 狐 , 树 上 有 灵 禽 、 玄 鹤 。
瑶 草 奇 花 不 凋 谢 , 青 松 翠 柏 长 春 。
仙 桃 经 常 结 果 , 修 竹 常 留 云 。
一 条 涧 壑 中 的 藤 萝 萝 草 稠 密 , 四 面 原 来 的 堤 坝 , 草 色 新 鲜 。
这 正 是 百 川 汇 合 的 地 方 , 擎 天 柱 , 万 劫 之 中 不 改 变 大 地 的 根 本 。
那 座 山 正 好 位 于 山 顶 上 , 有 一 块 仙 石 。
这 块 石 头 有 三 丈 六 尺 五 寸 高 , 有 二 丈 四 尺 , 周 围 圆 。
三 丈 六 尺 五 寸 高 , 按 周 天 三 百 六 十 五 度 ; 二 丈 四 尺 围 圆 , 按 《 政 历 》 二 十 四 节 气 。
上 面 有 九 窍 八 孔 , 按 九 宫 八 卦 。
四 面 再 没 有 树 木 遮 蔽 阴 影 , 左 右 两 侧 倒 有 芝 兰 相 映 。
自 从 开 辟 以 来 , 每 每 受 到 天 真 地 秀 , 日 精 月 华 , 感 激 他 很 久 , 于 是 产 生 了 神 灵 的 意 思 。
他 在 里 面 生 育 了 仙 胞 , 一 天 之 内 迸 裂 , 生 下 一 个 石 蛋 , 像 圆 一 样 大 。
于 是 他 看 见 风 , 变 成 了 一 个 石 猴 , 五 脏 都 具 备 了 , 四 肢 都 完 好 了 。
于 是 就 去 学 习 , 去 学 习 , 去 拜 见 四 方 。
眼 睛 运 转 两 道 金 光 , 射 冲 斗 府 。
惊 动 高 天 上 圣 大 慈 仁 者 玉 皇 大 天 尊 玄 穹 高 上 帝 , 驾 车 在 金 阙 云 宫 灵 霄 宝 殿 , 聚 集 仙 卿 , 见 有 金 光 火 焰 , 就 命 令 千 里 眼 、 顺 风 耳 打 开 南 天 门 观 看 。
两 位 将 领 果 然 奉 旨 出 门 外 , 看 得 真 , 听 得 清 楚 。
不 一 会 儿 , 他 回 来 报 告 说 : 我 奉 旨 观 听 金 光 的 地 方 , 是 东 胜 神 洲 海 东 傲 来 小 国 之 界 , 有 一 座 花 果 山 , 山 上 有 一 座 仙 石 , 石 头 生 下 一 蛋 蛋 , 见 风 化 成 一 个 猴 猴 , 在 那 里 拜 访 四 方 , 眼 睛 运 金 光 , 射 入 冲 斗 府 。
如 果 现 在 服 用 水 食 , 金 光 就 会 暗 中 熄 灭 了 。
玉 帝 垂 赐 恩 慈 说 : 下 方 之 物 , 是 天 地 精 华 所 产 生 的 , 不 足 为 怪 。
那 猴 在 山 中 , 却 会 走 跳 跃 , 吃 草 木 , 喝 涧 泉 , 采 山 花 , 寻 水 果 , 与 狼 虫 为 伴 , 虎 豹 为 伴 , 獐 鹿 为 朋 友 , 猿 猴 为 亲 , 夜 晚 住 在 石 崖 之 下 , 早 晨 游 览 山 洞 之 中 。
真 是 说 : 山 中 没 有 甲 子 , 寒 尽 不 知 年 。
一 天 早 晨 , 天 气 炎 热 , 与 群 猴 避 暑 , 都 在 松 荫 下 玩 耍 。
你 看 他 一 个 个 : 跳 树 攀 枝 , 采 摘 果 子 ; 抛 弹 子 , 是 什 么 儿 子 ; 跑 沙 窝 , 砌 宝 塔 ; 赶 蜻 蜓 , 扑 蜡 ; 参 拜 老 天 , 拜 菩 萨 ; 拽 葛 藤 , 编 草 ; 捉 虱 子 , 咬 又 咬 , 整 理 毛 衣 , 剔 指 甲 。
挨 挨 挨 , 擦 着 擦 , 推 的 推 , 压 的 压 , 拉 的 拽 , 拉 的 拽 。 青 松 林 下 任 他 顽 , 绿 水 涧 边 随 着 洗 涤 。
一 群 猴 子 玩 耍 了 一 天 , 又 去 山 涧 中 洗 澡 。
看 见 那 股 涧 水 奔 流 , 真 是 滚 瓜 涌 出 的 水 。
古 人 说 : 禽 兽 有 禽 兽 的 语 言 , 兽 兽 有 兽 的 语 言 。
众 猴 都 说 : 这 股 水 , 不 知 是 哪 里 的 水 。
我 们 今 天 赶 快 闲 暇 无 事 , 顺 着 涧 边 往 上 溜 头 去 寻 找 水 源 , 玩 子 去 啦 呀 喊 一 声 , 都 拖 着 男 人 携 着 女 儿 , 呼 叫 弟 弟 呼 叫 兄 弟 , 一 齐 奔 跑 来 , 顺 着 涧 水 爬 山 , 直 到 源 流 的 地 方 , 竟 然 是 一 股 瀑 布 飞 泉 。
只 见 那 那 道 : 一 股 白 虹 飞 起 , 千 寻 雪 浪 飞 。
海 风 吹 不 断 , 江 月 照 还 依 依 。
冷 气 分 开 青 嶂 , 余 流 润 泽 翠 微 。
潺 潺 流 水 称 为 瀑 布 , 真 像 挂 在 帘 帷 上 。
众 猴 拍 手 称 扬 道 : 好 水 , 好 水 , 好 水 , 原 来 此 处 远 远 通 到 山 脚 下 , 直 接 大 海 的 波 浪 。
又 说 : 那 一 个 有 本 事 的 人 , 钻 进 去 找 个 源 头 出 来 , 不 伤 身 体 的 人 , 我 们 立 即 封 他 为 王 。
接 连 呼 叫 了 三 声 , 忽 然 看 见 草 丛 中 跳 出 一 个 石 猴 , 应 声 高 叫 道 : 我 进 去 , 我 进 去 。
好 猴 , 也 是 他 人 : 今 日 芳 名 显 , 时 来 大 运 通 。
有 缘 居 住 在 这 个 地 方 , 王 派 他 进 入 仙 宫 。
你 看 他 闭 上 眼 睛 , 把 身 子 放 下 去 , 径 直 跳 入 瀑 布 泉 中 , 忽 然 睁 着 眼 睛 , 抬 头 观 看 , 那 里 边 没 有 水 , 没 有 波 浪 , 明 明 朗 朗 的 一 座 桥 梁 。
他 住 了 自 己 的 身 , 定 了 神 , 仔 细 再 看 , 原 来 是 一 座 铁 板 桥 。
桥 下 的 水 , 冲 贯 在 石 窍 之 间 , 倒 挂 流 出 去 , 阻 塞 了 桥 门 。
又 拖 着 身 子 上 到 桥 头 , 再 走 再 看 , 似 乎 有 人 家 住 处 一 样 , 真 是 个 好 处 所 在 。
只 见 到 那 里 , 翠 绿 的 苔 藓 堆 满 蓝 色 , 白 色 的 云 彩 飘 浮 着 玉 石 , 光 芒 闪 动 了 片 片 烟 霞 。
空 窗 静 室 , 滑 板 板 生 花 。
乳 窟 龙 珠 依 傍 挂 挂 , 绕 绕 满 地 奇 花 。
锅 灶 旁 边 的 山 崖 还 留 着 火 烧 的 痕 迹 , 桌 子 靠 着 桌 子 上 看 到 了 灰 烬 。
石 座 石 床 真 可 爱 , 石 盆 石 碗 更 值 得 夸 耀 。
又 看 见 那 一 竿 , 两 竿 是 修 竹 , 三 点 是 五 点 的 梅 花 。
几 棵 青 松 经 常 带 着 雨 , 浑 然 如 同 一 个 人 家 。
看 完 了 很 长 时 间 , 跳 过 桥 中 间 , 左 右 两 边 观 看 。
只 见 正 当 中 有 一 块 石 碣 , 石 碣 上 有 一 行 楷 书 大 字 , 上 面 写 着 花 果 山 福 地 , 水 帘 洞 洞 天 。
石 猿 喜 不 自 禁 , 急 抽 身 往 外 就 走 , 又 闭 目 蹲 身 , 跳 出 水 外 , 打 了 两 个 呵 呵 道 : 大 造 化 , 大 造 化 , 众 猴 把 它 围 住 , 问 道 : 里 面 怎 么 样 ? 石 猴 说 : 没 水 , 没 水 , 原 来 是 一 座 铁 板 桥 , 桥 那 边 是 一 座 天 造 地 设 的 家 。
众 猴 说 : 怎 么 见 到 这 个 家 当 呢 石 猴 笑 着 说 : 这 股 水 是 桥 下 冲 贯 石 桥 , 倒 挂 下 来 遮 住 门 户 的 。
桥 边 有 花 有 树 , 原 来 是 一 座 石 房 。
房 内 有 石 窝 、 石 灶 、 石 碗 、 石 盆 、 石 盆 、 石 床 、 石 凳 。
中 间 有 一 块 石 碣 , 上 面 写 着 花 果 山 福 地 , 水 帘 洞 洞 天 。
真 的 , 是 我 们 安 身 的 地 方 。
里 面 又 很 宽 敞 , 可 以 容 纳 千 百 口 老 少 。
我 们 都 进 去 住 , 省 得 受 到 老 天 的 气 。
这 里 的 地 方 : 刮 风 有 处 可 躲 , 下 雨 好 保 身 。
霜 雪 全 无 畏 惧 , 雷 声 永 远 不 闻 。
烟 霞 常 常 照 耀 , 祥 瑞 每 每 熏 蒸 。
松 竹 年 年 秀 发 , 奇 花 日 日 新 鲜 。
众 猴 听 了 , 各 个 都 欢 喜 。
郅 都 说 : 你 还 先 走 , 带 我 们 进 去 , 进 去 吧 。
石 猴 又 闭 着 眼 睛 蹲 在 地 上 , 往 里 跳 一 跳 , 叫 道 : 都 随 我 进 来 , 进 来 。
猴 子 有 胆 大 的 , 都 跳 进 去 了 ; 胆 小 的 , 一 个 个 伸 头 缩 颈 , 拍 耳 、 拍 颊 , 大 声 叫 喊 , 缠 了 一 会 儿 , 也 都 进 去 了 。
跳 过 桥 头 , 一 个 个 抢 盆 夺 碗 , 占 灶 争 床 , 搬 过 来 , 移 过 去 , 正 是 猴 性 顽 劣 , 再 没 有 一 个 安 宁 的 时 候 , 只 是 搬 得 力 疲 神 疲 才 停 下 来 。
石 猿 端 坐 在 上 面 说 : 列 位 呵 , 人 如 果 不 讲 信 用 , 不 知 道 可 行 。
君 子 之 所 以 为 之 , 不 可 以 为 之 , 不 可 以 为 之 , 不 可 以 为 之 , 不 可 以 为 之 。
我 现 在 进 来 又 出 去 , 出 去 又 进 来 , 寻 找 了 这 一 个 洞 天 和 各 位 官 员 安 眠 稳 睡 , 各 享 成 家 之 福 , 为 什 么 不 拜 我 为 王 ? 众 猴 听 了 , 立 即 拱 伏 不 违 背 , 一 个 个 排 列 排 列 , 朝 上 礼 拜 , 都 称 他 为 千 岁 大 王 。
从 此 以 后 , 石 猿 高 登 王 位 , 将 石 虎 的 字 儿 隐 瞒 起 来 , 于 是 称 他 为 美 猴 王 。
有 诗 作 证 。
《 诗 经 》 上 说 : 三 阳 交 泰 生 群 生 , 仙 石 胞 含 日 月 精 。
借 蛋 化 猴 完 成 大 道 , 假 借 他 的 名 姓 配 丹 成 。
在 内 观 察 不 认 识 , 因 为 没 有 什 么 形 象 , 在 外 合 合 明 白 的 知 识 , 就 是 有 形 体 。
历 代 人 人 都 归 属 于 此 , 称 王 称 圣 , 任 纵 横 。
美 猴 王 率 领 一 群 猿 猴 、 猕 猴 、 马 猴 等 , 分 别 派 出 君 臣 辅 佐 的 使 者 。
早 晨 游 花 果 山 , 晚 上 住 宿 水 帘 洞 , 合 心 同 情 , 不 进 入 飞 鸟 的 丛 林 , 不 跟 随 走 兽 的 同 类 , 独 自 为 王 , 不 胜 欢 快 乐 。
因 此 , 春 天 采 摘 百 花 作 为 饮 食 , 夏 天 寻 找 各 种 果 子 作 为 生 涯 。
秋 天 收 割 芋 栗 , 延 续 时 节 , 冬 天 寻 觅 黄 精 , 度 过 岁 华 。
美 猴 王 享 乐 天 真 , 何 必 希 望 有 三 五 百 年 。
一 天 , 与 群 猴 在 欢 宴 之 间 , 忽 然 忧 愁 恼 怒 , 落 下 眼 泪 来 。
众 猴 子 惊 慌 地 向 大 王 跪 拜 道 : 大 王 为 什 么 要 烦 恼 呢 猴 王 说 : 我 虽 然 在 欢 喜 的 时 候 , 却 有 一 点 远 虑 , 所 以 才 这 样 的 烦 恼 。
众 猴 又 笑 着 说 : 大 王 好 不 知 足 。
我 们 日 日 欢 乐 聚 会 , 在 仙 山 福 地 、 古 洞 神 洲 , 不 埋 伏 麒 麟 管 , 不 埋 伏 凤 凰 管 , 又 不 被 人 间 王 位 所 拘 束 , 自 由 自 在 , 是 无 量 的 福 气 , 为 什 么 还 担 心 远 虑 而 忧 虑 呢 猴 王 说 : 今 天 虽 然 不 遵 循 人 王 法 律 , 但 不 畏 惧 禽 兽 的 威 严 , 将 来 我 们 年 老 血 衰 , 暗 中 有 阎 王 老 子 管 着 , 一 旦 身 死 , 难 道 不 会 在 世 界 之 中 , 不 能 够 长 久 留 在 天 人 间 吗 ?
只 见 那 班 部 落 中 , 忽 然 跳 出 一 个 通 背 的 猿 猴 , 厉 声 大 叫 道 : 大 王 如 果 这 样 的 远 虑 , 真 是 所 谓 的 道 心 开 发 。
如 今 五 虫 之 中 , 只 有 三 等 , 名 声 和 颜 色 都 不 被 阎 王 、 老 子 所 管 辖 。
猴 王 说 : 你 知 道 那 三 个 人 ? 猿 猴 说 : 这 是 佛 与 仙 和 神 圣 三 个 人 , 逃 脱 了 世 界 , 不 生 不 灭 , 与 天 地 山 川 一 样 长 寿 。
猴 王 说 : 这 三 个 人 住 在 什 么 地 方 ? 猿 猴 说 : 他 们 只 在 阎 浮 世 界 之 中 , 古 洞 仙 山 之 内 。
猴 王 听 了 , 满 心 欢 喜 地 说 : 我 明 天 就 辞 别 你 们 下 山 , 云 游 海 角 , 远 涉 天 涯 , 一 定 要 访 求 这 三 个 , 学 一 个 不 老 长 生 , 常 常 躲 避 阎 君 的 灾 难 。
唉 , 这 个 话 , 立 刻 让 他 跳 出 轮 回 网 , 以 致 使 齐 天 大 圣 成 功 。
众 猴 击 鼓 称 扬 , 都 说 : 好 啊 , 好 啊 我 们 明 天 越 岭 登 山 , 广 泛 寻 找 些 果 品 , 大 摆 筵 席 送 大 王 。
第 二 天 , 众 猴 果 然 去 采 仙 桃 , 摘 异 果 , 劈 山 药 , 吃 黄 精 。
芝 兰 香 蕙 , 瑶 草 奇 花 , 各 种 各 种 , 整 齐 齐 齐 , 摆 开 石 凳 、 石 桌 , 排 列 着 仙 酒 、 仙 。
只 见 那 个 人 说 : 金 丸 珠 弹 , 红 色 的 花 开 , 黄 色 的 肥 胖 。
金 丸 珠 弹 腊 樱 桃 , 颜 色 真 是 甘 甜 美 味 ; 红 绽 黄 肥 的 梅 子 , 味 道 果 子 香 甜 。
鲜 龙 眼 , 肉 味 甜 , 皮 肤 薄 ; 火 荔 枝 , 核 小 袋 子 红 。
林 檎 绿 实 连 枝 献 , 枇 杷 苞 带 叶 擎 。
兔 头 梨 子 鸡 心 枣 , 消 渴 除 烦 , 更 解 渴 。
香 桃 烂 杏 , 味 道 甜 美 , 好 似 玉 液 琼 浆 ; 脆 李 、 杨 梅 , 酸 甜 的 树 荫 好 像 油 脂 膏 酪 。
红 囊 黑 子 熟 西 瓜 , 四 瓣 黄 皮 大 柿 子 。
石 榴 破 裂 , 丹 砂 粒 出 现 火 晶 珠 ; 芋 栗 剖 开 , 坚 硬 的 肉 团 金 玛 瑙 。
胡 桃 、 银 杏 可 以 传 播 茶 叶 , 荔 子 、 葡 萄 可 以 酿 酒 。
榛 、 松 、 、 、 、 、 满 盘 盛 , 橘 、 蔗 、 柑 、 橙 子 满 桌 摆 放 。
煮 成 山 药 , 煮 成 黄 精 。
捣 碎 茯 苓 和 薏 苡 , 用 石 锅 用 微 火 烧 成 的 肉 羹 。
人 间 即 使 有 珍 贵 的 肴 馔 , 哪 里 比 山 猴 乐 更 安 宁 。
群 猴 尊 美 猴 王 上 座 , 各 自 依 着 牙 齿 肩 膀 排 列 在 下 边 , 一 个 个 轮 流 上 前 , 捧 着 酒 、 捧 着 花 、 捧 着 果 , 痛 饮 了 一 天 。
第 二 天 , 美 猴 王 早 早 起 来 , 教 他 说 : 小 的 人 , 替 我 折 些 枯 松 , 编 成 竹 竿 , 拿 来 竹 竿 作 篙 , 收 拾 些 果 品 , 我 将 去 。
他 果 然 独 自 上 登 , 尽 力 撑 开 , 飘 飘 荡 荡 , 直 奔 大 海 波 中 , 乘 着 天 风 , 来 到 南 赡 部 洲 的 地 界 。
这 一 去 , 正 是 那 样 : 天 生 仙 猴 道 行 隆 , 离 山 驾 , 趁 天 风 。
飘 洋 过 海 寻 求 仙 道 , 立 志 潜 心 建 立 大 功 。
有 分 有 缘 , 能 满 足 世 俗 的 愿 望 , 无 忧 无 虑 , 会 合 元 龙 。
估 计 应 答 必 须 遇 到 了 解 音 乐 的 人 , 解 说 破 源 流 万 法 通 。
自 从 登 上 木 山 以 后 , 连 日 东 南 风 紧 , 把 它 送 到 西 北 岸 前 , 这 是 南 赡 部 洲 的 地 界 。
拿 着 篙 试 水 , 偶 然 遇 到 浅 水 , 扔 掉 了 子 , 跳 上 岸 来 。
只 见 海 边 有 人 捕 鱼 、 打 雁 、 蛤 、 淘 盐 。
他 走 近 前 , 戏 弄 一 个 小 虎 , 吓 得 那 个 人 丢 掉 筐 子 扔 掉 网 子 , 四 处 奔 走 。
把 那 些 奔 跑 不 动 的 人 拿 住 一 个 , 剥 掉 他 的 衣 服 , 也 学 人 穿 在 身 上 。
他 摇 摇 摆 动 , 穿 过 州 府 , 在 市 场 中 学 人 礼 , 学 人 语 。
早 餐 晚 宿 , 一 心 一 意 地 访 问 佛 、 仙 、 神 、 圣 之 道 , 找 到 长 生 不 老 的 方 法 。
现 在 世 上 的 人 都 是 为 了 名 声 利 益 的 人 , 更 没 有 一 个 为 了 自 己 的 性 命 。
正 是 那 些 争 名 夺 利 多 时 休 息 , 早 起 晚 睡 不 自 由 , 骑 着 驴 子 , 想 着 骏 马 , 官 居 宰 相 , 望 见 王 侯 。
只 愁 衣 食 沉 溺 于 劳 碌 , 何 必 担 心 阎 君 就 取 勾 。
继 承 儿 子 的 荫 庇 孙 子 , 图 谋 富 贵 , 更 没 有 一 个 人 肯 回 头 。
猴 王 参 访 仙 道 , 没 有 机 会 得 到 了 。
在 南 赡 部 洲 , 绕 着 长 城 , 游 览 小 县 , 不 知 不 觉 八 九 年 多 。
忽 然 走 到 西 洋 大 海 , 他 想 到 海 外 一 定 有 神 仙 。
独 自 自 己 一 个 依 照 先 前 的 作 法 , 又 漂 过 西 海 , 直 到 西 牛 贺 洲 地 界 。
登 岸 遍 访 了 许 多 时 间 , 忽 然 看 见 一 座 高 山 秀 丽 , 山 麓 幽 深 。
其 所 谓 之 。
果 然 是 好 山 , 千 峰 排 列 戟 戟 , 万 仞 开 辟 屏 风 。
太 阳 映 衬 着 山 光 , 轻 轻 地 掩 映 着 翠 色 , 雨 收 了 黛 色 , 冷 淡 含 青 。
瘦 藤 缠 绕 老 树 , 古 渡 隔 在 幽 深 的 路 程 。
奇 花 瑞 草 , 修 竹 、 高 松 。
修 竹 、 高 松 , 万 年 常 青 , 欺 骗 福 地 ; 奇 花 瑞 草 , 四 季 不 衰 , 比 蓬 瀛 。
幽 鸟 啼 声 接 近 , 泉 水 流 淌 清 。
重 重 山 谷 壑 谷 , 芝 兰 环 绕 , 处 处 山 崖 上 的 苔 藓 生 长 。
在 山 峦 之 上 , 龙 脉 好 , 必 定 有 高 人 隐 姓 名 。
正 在 观 看 的 时 候 , 忽 然 听 到 林 深 之 处 有 人 说 话 。
急 忙 快 步 走 进 树 林 中 , 侧 着 耳 朵 听 , 原 来 是 歌 唱 的 声 音 。
歌 道 : 观 看 棋 柄 烂 , 伐 木 丁 丁 , 云 边 谷 口 徐 行 。
卖 柴 卖 酒 , 狂 笑 自 己 陶 情 。
苍 茫 直 到 秋 天 的 时 候 , 对 着 月 亮 , 枕 着 松 根 , 一 觉 天 明 。
认 识 旧 林 , 登 崖 越 岭 , 手 持 斧 子 砍 断 枯 藤 。
收 回 来 成 一 担 , 在 市 场 上 行 歌 , 换 米 三 升 。
再 没 有 一 点 儿 子 竞 争 , 时 价 平 平 。
不 会 机 谋 巧 算 , 没 有 荣 辱 , 恬 淡 延 年 。
相 逢 的 地 方 , 不 是 仙 就 是 道 , 静 坐 讲 《 黄 庭 》 。
美 猴 王 听 了 这 话 , 满 心 欢 喜 地 说 : 神 仙 原 来 就 藏 在 这 里 。 于 是 就 急 忙 跳 进 里 面 , 仔 细 再 看 , 原 来 是 一 个 樵 夫 , 在 那 里 举 斧 砍 柴 。
但 看 他 的 衣 服 非 常 , 头 上 戴 笠 , 原 来 是 新 笋 刚 脱 的 鞋 子 。
他 身 上 穿 的 布 衣 , 原 来 是 用 木 绵 揉 成 的 纱 子 。
腰 间 系 着 环 , 是 老 蚕 口 吐 的 丝 。
脚 下 踏 草 鞋 , 是 枯 莎 草 鞋 的 清 爽 。
手 执 钢 斧 , 担 子 拉 着 火 麻 绳 。
攀 松 劈 断 枯 树 , 争 相 像 这 个 樵 夫 。
猴 王 近 前 喊 道 : 老 神 仙 , 弟 子 起 手 。
那 樵 夫 急 忙 丢 掉 斧 头 , 转 身 回 答 礼 节 说 : 不 当 人 , 不 当 人 。
猴 王 说 : 你 不 是 神 仙 , 为 什 么 说 出 神 仙 话 来 ? 樵 夫 说 : 我 说 什 么 神 仙 话 ? 猴 王 说 : 我 才 来 到 林 边 , 只 听 你 说 : 相 逢 处 , 非 仙 就 道 , 静 坐 讲 《 黄 庭 》 。
《 黄 庭 》 是 道 德 真 言 , 不 是 神 仙 又 是 什 么 ? 樵 夫 笑 着 说 : 实 在 不 骗 你 说 , 这 个 词 名 叫 《 满 庭 芳 》 , 是 一 位 神 仙 教 我 的 。
那 神 仙 与 我 家 里 的 人 相 邻 , 他 看 见 我 家 里 的 事 务 劳 苦 , 日 常 烦 恼 , 教 我 遇 到 烦 恼 的 时 候 , 就 用 这 个 词 句 来 念 念 , 一 是 散 发 心 , 二 是 解 除 困 难 。
我 才 有 些 不 值 得 思 虑 , 所 以 我 一 念 一 想 , 不 想 被 你 听 了 。
猴 王 说 : 你 家 既 然 和 神 仙 相 邻 , 为 什 么 不 跟 着 他 修 行 , 学 得 一 个 不 老 的 方 法 , 却 不 是 好 吗 ? 樵 夫 说 : 我 一 生 命 苦 , 自 幼 蒙 父 母 抚 养 , 到 八 九 岁 , 才 知 人 事 , 不 幸 父 亲 去 世 , 母 亲 居 寡 。
再 没 有 兄 弟 姊 妹 , 只 有 我 一 个 人 , 没 有 什 么 办 法 , 早 晚 侍 奉 。
现 在 母 亲 年 老 , 一 发 不 敢 抛 离 。
又 又 田 园 荒 芜 , 衣 食 不 足 , 只 得 砍 两 束 柴 薪 , 挑 到 市 场 之 间 , 买 几 文 钱 , 买 几 升 米 , 自 己 做 饭 , 安 排 茶 饭 , 供 养 老 母 。
所 以 不 能 修 行 。
猴 王 说 : 根 据 你 说 起 来 , 他 是 一 个 行 孝 的 君 子 , 以 后 必 定 有 好 处 。
只 希 望 你 指 给 我 那 神 仙 住 处 , 你 就 好 好 去 拜 访 。
樵 夫 说 : 不 远 , 不 远 。
此 山 名 叫 灵 台 方 寸 山 , 山 中 有 座 斜 月 三 星 洞 , 洞 中 有 一 个 神 仙 , 叫 须 菩 提 祖 师 。
祖 师 出 去 的 弟 子 , 也 不 计 其 数 , 现 在 还 有 三 四 十 人 , 跟 着 他 修 行 。
你 顺 着 那 条 小 路 , 向 南 走 七 八 里 , 就 是 他 家 了 。
猴 王 用 手 拽 住 樵 夫 说 : 老 兄 , 你 就 同 我 去 吧 如 果 还 得 到 好 处 , 我 决 不 忘 你 指 点 引 导 我 的 恩 德 。
樵 夫 说 : 你 这 个 汉 子 很 不 通 达 变 化 , 我 刚 才 这 样 跟 你 说 了 , 你 还 不 省 悟 , 如 果 我 和 你 去 了 , 却 不 误 了 我 的 生 意 , 老 母 是 什 么 人 奉 养 的 ? 我 要 砍 柴 , 你 自 己 去 , 自 己 去 。
猴 王 听 了 , 只 得 告 辞 。
走 出 深 林 , 找 上 去 的 路 , 经 过 一 个 山 坡 , 大 约 有 七 八 里 远 , 果 然 望 见 一 座 洞 府 。
挺 身 来 观 看 , 真 是 好 的 地 方 , 只 见 到 烟 霞 散 彩 , 日 月 摇 光 。
千 棵 老 柏 , 万 节 修 竹 。
千 棵 老 柏 , 带 雨 半 空 , 青 色 冉 冉 ; 万 节 修 竹 , 含 烟 一 壑 , 颜 色 苍 苍 。
门 外 奇 花 布 满 锦 绣 , 桥 边 瑶 草 喷 出 香 气 。
石 崖 突 兀 , 青 苔 润 泽 , 悬 崖 的 石 壁 高 张 , 翠 绿 的 苔 藓 长 。
有 时 听 到 仙 鹤 鸣 叫 , 常 常 看 见 凤 凰 飞 翔 。
仙 鹤 鸣 叫 时 , 声 音 震 动 九 皋 云 汉 遥 远 ; 凤 凰 飞 翔 , 羽 毛 五 彩 彩 云 光 。
玄 猿 白 鹿 随 着 隐 隐 显 现 , 金 狮 玉 象 任 凭 行 走 藏 匿 。
细 观 灵 福 地 , 真 是 赛 天 堂 。
又 看 见 那 个 洞 口 紧 闭 着 , 静 悄 悄 悄 悄 悄 悄 , 没 有 人 迹 。
忽 然 回 头 , 见 崖 头 立 着 一 块 石 碑 , 大 约 有 三 丈 多 高 , 八 尺 多 宽 , 上 面 有 一 行 十 个 大 字 , 是 灵 台 方 寸 山 , 斜 月 三 星 洞 。
美 猴 王 十 分 欢 喜 地 说 : 这 里 的 人 果 真 是 朴 实 , 果 然 有 此 山 这 个 洞 。
他 看 了 很 长 时 间 , 不 敢 敲 门 。
暂 且 去 跳 上 松 枝 梢 头 , 摘 松 子 吃 了 顽 戏 。
不 一 会 儿 , 只 听 到 一 声 呜 叫 , 洞 门 开 开 的 地 方 , 里 面 走 出 一 个 仙 童 , 真 是 个 丰 姿 英 伟 , 相 貌 清 新 奇 异 , 和 平 常 的 俗 子 不 同 。
只 见 到 他 们 说 : 髻 双 丝 , 宽 袍 两 袖 风 。
形 貌 和 身 体 自 然 分 别 , 心 与 相 同 都 是 空 的 。
物 外 长 年 客 , 山 中 永 寿 童 。
一 尘 全 不 染 , 甲 子 任 凭 翻 腾 。
那 个 童 子 出 门 来 , 高 喊 道 : 你 为 什 么 人 在 这 里 骚 扰 ? 猴 王 扑 地 跳 下 树 来 , 上 前 亲 自 说 : 仙 童 , 我 是 个 访 道 学 仙 的 弟 子 , 更 不 敢 在 这 里 干 扰 。
仙 童 笑 着 说 : 你 是 个 找 道 的 吗 ? 猴 王 说 : 是 。
童 子 说 : 我 家 的 师 父 正 刚 下 床 , 登 上 佛 坛 讲 道 , 还 没 说 出 来 的 原 因 , 就 教 我 出 来 开 门 。
回 答 说 : 外 面 有 个 修 行 的 人 来 了 , 可 以 去 接 待 。
猴 王 笑 着 说 : 是 我 , 是 我 。
童 子 说 : 你 跟 我 进 来 。
这 个 猴 王 整 理 衣 服 端 庄 严 肃 , 随 着 童 子 径 直 进 入 洞 天 深 处 观 看 , 一 层 层 深 阁 琼 楼 , 一 进 进 珠 宫 贝 阙 , 说 不 尽 那 静 室 幽 居 。
一 直 走 到 瑶 台 下 , 看 见 那 菩 提 祖 师 端 坐 在 台 上 , 两 边 有 三 十 个 小 仙 人 在 台 下 侍 立 。
果 然 是 : 大 觉 金 仙 没 垢 姿 , 西 方 妙 相 祖 菩 提 。
不 生 不 灭 , 三 三 行 , 全 气 、 神 , 万 万 慈 爱 。
空 寂 自 然 随 着 变 化 而 变 化 , 真 如 本 性 任 凭 着 它 去 做 。
与 天 同 寿 庄 严 体 , 历 劫 明 心 大 法 师 。
美 猴 王 一 见 , 就 倒 身 下 拜 , 不 计 其 数 , 口 中 只 说 : 师 父 , 师 父 , 我 弟 子 志 向 朝 拜 , 志 向 朝 拜 。
祖 师 说 : 你 是 哪 个 地 方 的 人 , 暂 且 说 一 个 乡 贯 、 姓 名 都 很 清 楚 , 再 拜 。
猴 王 说 : 弟 子 是 东 胜 神 洲 傲 来 国 花 果 山 水 帘 洞 人 氏 。
祖 师 喝 令 说 : 赶 出 去 , 他 本 来 是 个 撒 谎 捣 虚 之 徒 , 哪 里 修 得 什 么 道 果 啊 猴 王 惊 慌 地 叩 头 不 住 说 : 弟 子 是 老 实 之 言 , 决 不 会 虚 假 欺 骗 。
祖 师 说 : 你 既 然 已 经 老 了 , 怎 么 说 东 胜 神 洲 , 那 么 去 处 到 我 这 里 , 隔 着 两 重 大 海 , 一 座 南 赡 部 洲 , 怎 么 能 够 到 这 里 呢 猴 王 叩 头 说 : 弟 子 飘 洋 过 海 , 登 上 界 游 方 , 已 有 十 几 个 年 头 , 才 到 这 里 去 。
祖 师 说 : 既 然 是 逐 渐 走 来 的 。
你 姓 什 么 ? 猴 王 又 说 : 我 没 有 本 性 。
别 人 如 果 骂 我 , 我 也 不 会 恼 怒 ; 如 果 打 我 , 我 也 不 会 恼 怒 。
只 是 陪 我 一 个 礼 仪 就 罢 了 。
一 生 都 没 有 本 性 。
祖 师 说 : 不 是 这 个 性 。
你 父 母 原 来 姓 什 么 ? 猴 王 说 : 我 也 没 有 父 母 。
祖 师 说 : 既 然 没 有 父 母 , 想 是 树 上 生 的 猴 王 说 : 我 虽 然 不 是 树 上 生 的 , 却 是 石 里 长 的 。
我 只 记 得 花 果 山 上 有 一 块 仙 石 , 那 年 石 头 破 了 , 我 就 活 了 。
祖 师 听 了 , 暗 自 高 兴 , 说 : 这 些 说 法 , 却 是 天 地 生 成 的 。
你 起 来 走 , 我 看 。
猴 王 纵 身 跳 起 来 , 拐 呀 拐 子 走 了 两 遍 。
祖 师 笑 着 说 : 你 的 身 体 虽 然 鄙 陋 , 却 像 吃 松 果 的 。
吾 与 你 在 身 上 取 个 姓 氏 , 意 思 是 教 你 姓 。
字 去 掉 一 个 兽 旁 , 乃 是 个 古 月 。
古 代 的 意 思 是 老 人 , 月 亮 的 意 思 是 阴 气 。
老 阴 不 能 化 育 , 教 你 姓 倒 好 。
字 去 了 兽 旁 , 乃 是 个 子 系 。
所 谓 子 , 就 是 儿 子 和 男 孩 ;
婴 儿 细 小 , 正 合 于 婴 儿 的 本 性 。
教 你 姓 孙 罢 了 。
猴 王 听 了 , 满 心 欢 喜 , 向 皇 上 叩 头 说 : 好 , 好 , 好 , 今 天 我 才 知 道 你 的 姓 。
万 望 师 父 慈 悲 , 既 然 有 姓 , 再 请 求 赐 给 我 一 个 名 字 , 还 要 好 好 呼 唤 。
祖 师 说 : 我 门 下 有 十 二 个 字 , 分 派 起 名 , 到 你 就 是 第 十 辈 的 小 弟 子 了 。
猴 王 说 : 那 十 二 个 字 , 祖 师 说 : 是 广 、 大 、 智 、 慧 、 真 、 如 、 性 、 海 、 颖 、 悟 、 圆 、 觉 十 二 个 字 。
排 到 你 , 正 应 当 是 悟 的 字 。
给 你 起 个 法 名 叫 做 孙 悟 空 , 好 吗 ? 猴 王 笑 着 说 : 好 , 好 , 从 现 在 起 就 叫 做 孙 悟 空 了 。
这 正 是 鸿 濛 初 辟 原 无 姓 , 打 破 顽 空 须 悟 空 。
至 于 不 知 道 以 后 修 行 什 么 道 果 , 暂 且 听 下 回 分 解 。
}\switchcolumn\flushpage  \begin{pinyinscope}{\myfontt \section{第二回}     悟徹菩提真妙理 斷魔歸本合元神

話表美猴王得了姓名,怡然踴躍,對菩提前作禮啟謝。那祖師即命大眾引孫悟空
出二門外,教他灑掃應對、進退周旋之節。眾仙奉行而出。悟空到門外,又拜了
大眾師兄,就於廊廡之間安排寢處。次早,與眾師兄學言語禮貌、講經論道、習
字焚香。每日如此。閑時即掃地鋤園、養花修樹、尋柴燃火、挑水運漿。凡所用
之物,無一不備。在洞中不覺倏六七年。

一日,祖師登壇高坐,喚集諸仙,開講大道。真個是:
天花亂墜,地湧金蓮。妙演三乘教,精微萬法全。慢搖麈尾噴珠玉,響振雷霆動
九天。說一會道,講一會禪,三家配合本如然。開明一字皈誠理,指引無生了性
玄。

孫悟空在傍聞講,喜得他抓耳撓腮,眉花眼笑,忍不住手之舞之,足之蹈之。忽
被祖師看見,叫孫悟空道:「你在班中,怎麼顛狂躍舞,不聽我講?」悟空道:
「弟子誠心聽講,聽到老師父妙音處,喜不自勝,故不覺作此踴躍之狀。望師父
恕罪。」祖師道:「你既識妙音,我且問你,你到洞中多少時了?」悟空道:
「弟子本來懵懂,不知多少時節。只記得灶下無火,常去山後打柴,見一山好桃
樹,我在那裏吃了七次飽桃矣。」祖師道:「那山喚名爛桃山。你既吃七次,想
是七年了。你今要從我學些甚麼道?」悟空道:「但憑尊師教誨,只是有些道氣
兒,弟子便就學了。」

祖師道:「『道』字門中有三百六十傍門,傍門皆有正果。不知你學那一門哩?」
悟空道:「憑尊師意思,弟子傾心聽從。」祖師道:「我教你個『術』字門中之
道,如何?」悟空道:「術門之道怎麼說?」祖師道:「術字門中,乃是些請仙
、扶鸞、問卜、揲蓍,能知趨吉避凶之理。」悟空道:「似這般可得長生麼?」
祖師道:「不能,不能。」悟空道:「不學,不學。」

祖師又道:「教你『流』字門中之道,如何?」悟空又問:「流字門中是甚義理
?」祖師道:「流字門中,乃是儒家、釋家、道家、陰陽家、墨家、醫家,或看
經,或念佛,並朝真降聖之類。」悟空道:「似這般可得長生麼?」祖師道:
「若要長生,也似壁裏安柱。」悟空道:「師父,我是個老實人,不曉得打市語
。怎麼謂之『壁裏安柱』?」祖師道:「人家蓋房,欲圖堅固,將牆壁之間立一
頂柱,有日大廈將頹,他必朽矣。」悟空道:「據此說,也不長久。不學,不學
。」

祖師道:「教你『靜』字門中之道,如何?」悟空道:「靜字門中是甚正果?」
祖師道:「此是休糧守谷、清靜無為、參禪打坐、戒語持齋,或睡功,或立功,
並入定、坐關之類。」悟空道:「這般也能長生麼?」祖師道:「也似?頭土坯
。」悟空笑道:「師父果有些滴?。一行說我不會打市語。怎麼謂之『?頭土坯』
?」祖師道:「就如那?頭上造成磚瓦之坯,雖已成形,尚未經水火鍛煉,一朝
大雨滂沱,他必濫矣。」悟空道:「也不長遠。不學,不學。」

祖師道:「教你『動』字門中之道,如何?」悟空道:「動門之道卻又怎麼?」
祖師道:「此是有為有作:採陰補陽、攀弓踏弩、摩臍過氣、用方炮製、燒茅打
鼎、進紅鉛、煉秋石,並服婦乳之類。」悟空道:「似這等也得長生麼?」祖師
道:「此欲長生,亦如水中撈月。」悟空道:「師父又來了。怎麼叫做『水中撈
月』?」祖師道:「月在長空,水中有影,雖然看見,只是無撈摸處,到底只成
空耳。」悟空道:「也不學,不學。」

祖師聞言,咄的一聲,跳下高臺,手持戒尺,指定悟空道:「你這猢猻,這般不
學,那般不學,卻待怎麼?」走上前,將悟空頭上打了三下。倒背著手,走入裏
面,將中門關了,撇下大眾而去。諕得那一班聽講的人人驚懼,皆怨悟空道:
「你這潑猴,十分無狀。師父傳你道法,如何不學,卻與師父頂嘴?這番衝撞了
他,不知幾時才出來呵!」此時俱甚報怨他,又鄙賤嫌惡他。悟空一些兒也不惱
,只是滿臉陪笑。原來那猴王已打破盤中之謎,暗暗在心,所以不與眾人爭競,
只是忍耐無言。祖師打他三下者,教他三更時分存心﹔倒背著手走入裏面,將中
門關上者,教他從後門進步,秘處傳他道也。

當日悟空與眾等喜喜歡歡,在三星仙洞之前,盼望天色,急不能到晚。及黃昏時
,卻與眾就寢,假合眼,定息存神。山中又沒打更傳箭,不知時分,只自家將鼻
孔中出入之氣調定。約到子時前後,輕輕的起來,穿了衣服,偷開前門,躲離大
眾,走出外,抬頭觀看,正是那:
    月明清露冷,八極迥無塵。
    深樹幽禽宿,源頭水溜汾。
    飛螢光散影,過雁字排雲。
    正直三更候,應該訪道真。

你看他從舊路徑至後門外,只見那門兒半開半掩。悟空喜道:「老師父果然注意
與我傳道,故此開著門也。」即曳步近前,側身進得門裏,只走到祖師寢榻之下
。見祖師踡跼身軀,朝裏睡著了。悟空不敢驚動,即跪在榻前。那祖師不多時覺
來,舒開兩足,口中自吟道:


「難!難!難!道最玄,莫把金丹作等閑。不遇至人傳妙訣,空言口困舌頭乾!」

悟空應聲叫道:「師父,弟子在此跪候多時。」祖師聞得聲音是悟空,即起披衣
,盤坐喝道:「這猢猻,你不在前邊去睡,卻來我這後邊作甚?」悟空道:「師
父昨日壇前對眾相允,教弟子三更時候,從後門裏傳我道理,故此大膽徑拜老爺
榻下。」祖師聽說,十分歡喜,暗自尋思道:「這廝果然是個天地生成的,不然
,何就打破我盤中之暗謎也?」悟空道:「此間更無六耳,止只弟子一人,望師
父大捨慈悲,傳與我長生之道罷,永不忘恩。」祖師道:「你今有緣,我亦喜說
。既識得盤中暗謎,你近前來,仔細聽之,當傳與你長生之妙道也。」悟空叩頭
謝了,洗耳用心,跪於榻下。祖師云:
    顯密圓通真妙訣,惜修性命無他說。
    都來總是精氣神,謹固牢藏休漏泄。
    休漏泄,體中藏,汝受吾傳道自昌。
    口訣記來多有益,屏除邪欲得清涼。
    得清涼,光皎潔,好向丹臺賞明月。
    月藏玉兔日藏烏,自有龜蛇相盤結。
    相盤結,性命堅,卻能火裏種金蓮。
    攢簇五行顛倒用,功完隨作佛和仙。

此時說破根源,悟空心靈福至,切切記了口訣。對祖師拜謝深恩,即出後門觀看
。但見東方天色微舒白,西路金光大顯明。依舊路,轉到前門,輕輕的推開進去
,坐在原寢之處,故將床鋪搖響道:「天光了,天光了,起耶!」那大眾還正睡
哩,不知悟空已得了好事。當日起來打混,暗暗維持,子前午後,自己調息。

卻早過了三年,祖師復登寶座,與眾說法。談的是公案比語,論的是外像包皮。
忽問:「悟空何在?」悟空近前跪下:「弟子有。」祖師道:「你這一向修些甚
麼道來?」悟空道:「弟子近來法性頗通,根源亦漸堅固矣。」祖師道:「你既
通法性,會得根源,已注神體,卻只是防備著三災利害。」悟空聽說,沉吟良久
道:「師父之言謬矣。我常聞道高德隆,與天同壽﹔水火既濟,百病不生。卻怎
麼有個『三災利害』?」祖師道:「此乃非常之道:奪天地之造化,侵日月之玄
機﹔丹成之後,鬼神難容。雖駐顏益壽,但到了五百年後,天降雷災打你,須要
見性明心,預先躲避。躲得過,壽與天齊﹔躲不過,就此絕命。再五百年後,天
降火災燒你。這火不是天火,亦不是凡火,喚做『陰火』。自本身湧泉穴下燒起
,直透泥垣宮,五臟成灰,四肢皆朽,把千年苦行,俱為虛幻。再五百年,又降
風災吹你。這風不是東南西北風,不是和薰金朔風,亦不是花柳松竹風,喚做
『贔風』。自?門中吹入六腑,過丹田,穿九竅,骨肉消疏,其身自解。所以都
要躲過。」

悟空聞說,毛骨悚然,叩頭禮拜道:「萬望老爺垂憫,傳與躲避三災之法,到底
不敢忘恩。」祖師道:「此亦無難,只是你比他人不同,故傳不得。」悟空道:
「我也頭圓頂天,足方履地,一般有九竅四肢,五臟六腑,何以比人不同?」祖
師道:「你雖然像人,卻比人少腮。」原來那猴子孤拐面,凹臉尖嘴。悟空伸手
一摸,笑道:「師父沒成算。我雖少腮,卻比人多這個素袋,亦可准折過也。」
祖師說:「也罷,你要學那一般?有一般天罡數,該三十六般變化﹔有一般地煞
數,該七十二般變化。」悟空道:「弟子願多裏撈摸,學一個地煞變化罷。」祖
師道:「既如此,上前來,傳與你口訣。」遂附耳低言,不知說了些甚麼妙法。
這猴王也是他一竅通時百竅通,當時習了口訣,自修自煉,將七十二般變化都學
成了。

忽一日,祖師與眾門人在三星洞前戲玩晚景。祖師道:「悟空,事成了未曾?」
悟空道:「多蒙師父海恩,弟子功果完備,已能霞舉飛昇也。」祖師道:「你試
飛舉我看。」悟空弄本事,將身一聳,打了個連扯跟頭,跳離地有五六丈,踏雲
霞去勾有頓飯之時,返復不上三里遠近,落在面前,扠手道:「師父,這就是飛
舉騰雲了。」祖師笑道:「這個算不得騰雲,只算得爬雲而已。自古道:『神仙
朝遊北海暮蒼梧。』似你這半日,去不上三里,即爬雲也還算不得哩。」悟空道
:「怎麼為『朝遊北海暮蒼梧』?」祖師道:「凡騰雲之輩,早辰起自北海,遊
過東海、西海、南海,復轉蒼梧。蒼梧者,卻是北海零陵之語話也。將四海之外
,一日都遊遍,方算得騰雲。」悟空道:「這個卻難,卻難。」祖師道:「世上
無難事,只怕有心人。」悟空聞得此言,叩頭禮拜,啟道:「師父,為人須為徹
,索性捨個大慈悲,將此騰雲之法,一發傳與我罷,決不敢忘恩。」祖師道:
「凡諸仙騰雲,皆跌足而起,你卻不是這般。我才見你去,連扯方才跳上。我今
只就你這個勢,傳你個觔斗雲罷。」悟空又禮拜懇求,祖師卻又傳個口訣道:
「這朵雲,捻著訣,念動真言,攢緊了拳,將身一抖,跳將起來,一觔斗就有十
萬八千里路哩。」大眾聽說,一個個嘻嘻笑道:「悟空造化,若會這個法兒,與
人家當鋪兵、送文書、遞報單,不管那裏都尋了飯吃。」師徒們天昏各歸洞府。

這一夜,悟空即運神煉法,會了觔斗雲。逐日家無拘無束,自在逍遙,此亦長生
之美。

一日,春歸夏至,大眾都在松樹下會講多時。大眾道:「悟空,你是那世修來的
緣法?前日老師父附耳低言,傳與你的躲三災變化之法,可都會麼?」悟空笑道
:「不瞞諸兄長說,一則是師父傳授,二來也是我晝夜慇懃,那幾般兒都會了。」
大眾道:「趁此良時,你試演演,讓我等看看。」悟空聞說,抖搜精神,賣弄手
段道:「眾師兄請出個題目。要我變化甚麼?」大眾道:「就變棵松樹罷。」悟
空捻著訣,念動咒語,搖身一變,就變做一棵松樹。真個是:
    鬱鬱含煙貫四時,凌雲直上秀貞姿。
    全無一點妖猴像,盡是經霜耐雪枝。

大眾見了鼓掌,呵呵大笑,都道:「好猴兒,好猴兒!」不覺的嚷鬧,驚動了祖
師。祖師急拽杖出門來問道:「是何人在此喧嘩?」大眾聞呼,慌忙檢束,整衣
向前。悟空也現了本相,雜在叢中道:「啟上尊師:我等在此會講,更無外姓喧
嘩。」祖師怒喝道:「你等大呼小叫,全不像個修行的體段!修行的人,口開神
氣散,舌動是非生,如何在此嚷笑?」大眾道:「不敢瞞師父,適才孫悟空演變
化耍子。教他變棵松樹,果然是棵松樹,弟子們俱稱揚喝彩,故高聲驚冒尊師,
望乞恕罪。」

祖師道:「你等起去。」叫:「悟空過來!我問你弄甚麼精神,變甚麼松樹?這
個工夫,可好在人前賣弄?假如你見別人有,不要求他?別人見你有,必然求你
。你若畏禍,卻要傳他﹔若不傳他,必然加害:你之性命又不可保。」悟空叩頭
道:「只望師父恕罪。」祖師道:「我也不罪你,但只是你去罷。」悟空聞此言
,滿眼墮淚道:「師父,教我往那裏去?」祖師道:「你從那裏來,便從那裏去
就是了。」悟空頓然醒悟道:「我自東勝神洲傲來國花果山水簾洞來的。」祖師
道:「你快回去,全你性命﹔若在此間,斷然不可。」悟空領罪,上告尊師:
「我也離家有二十年矣,雖是回顧舊日兒孫,但念師父厚恩未報,不敢去。」祖
師道:「那裏甚麼恩義,你只是不惹禍,不牽帶我就罷了。」

悟空見沒奈何,只得拜辭,與眾相別。祖師道:「你這去,定生不良。憑你怎麼
惹禍行兇,卻不許說是我的徒弟。你說出半個字來,我就知之,把你這猢猻剝皮
剉骨,將神魂貶在九幽之處,教你萬劫不得翻身!」悟空道:「決不敢提起師父
一字,只說是我自家會的便罷。」
  
悟空謝了,即抽身,捻著訣,丟個連扯,縱起觔斗雲,徑回東勝。那裏消一個時
辰,早看見花果山水簾洞。美猴王自知快樂,暗暗的自稱道:
    去時凡骨凡胎重,得道身輕體亦輕。
    舉世無人肯立志,立志修玄玄自明。
    當時過海波難進,今日回來甚易行。
    別語叮嚀還在耳,何期頃刻見東溟。

悟空按下雲頭,直至花果山,找路而走。忽聽得鶴唳猿啼:鶴唳聲沖霄漢外,猿
啼悲切甚傷情。即開口叫道:「孩兒們,我來了也!」那崖下石坎邊,花草中,
樹木裏,若大若小之猴,跳出千千萬萬,把個美猴王圍在當中,叩頭叫道:「大
王,你好寬心,怎麼一去許久?把我們俱閃在這裏,望你誠如饑渴。近來被一妖
魔在此欺虐,強要占我們水簾洞府,是我等捨死忘生,與他爭鬥。這些時,被那
廝搶了我們家火,捉了許多子姪,教我們晝夜無眠,看守家業。幸得大王來了,
大王若再年載不來,我等連山洞盡屬他人矣。」悟空聞說,心中大怒,道:「是
甚麼妖魔,輒敢無狀?你且細細說來,待我尋他報仇。」眾猴叩頭:「告上大王
:那廝自稱混世魔王,住居在直北下。」悟空道:「此間到他那裏,有多少路程
?」眾猴道:「他來時雲,去時霧,或風或雨,或電或雷,我等不知有多少路。」
悟空道:「既如此,你們休怕,且自頑耍,等我尋他去來。」

好猴王,將身一縱,跳起去,一路觔斗,直至北下觀看,見一座高山,真是十分
險峻。好山:
筆峰挺立,曲澗深沉。筆峰挺立透空霄,曲澗深沉通地戶。兩崖花木爭奇,幾處
松篁鬥翠。左邊龍,熟熟馴馴﹔右邊虎,平平伏伏。每見鐵牛耕,常有金錢種。
幽禽睍睆聲,丹鳳朝陽立。石磷磷,波淨淨,古怪蹺蹊真惡獰。世上名山無數多
,花開花謝蘩還眾。爭如此景永長存,八節四時渾不動。誠為三界坎源山,滋養
五行水臟洞。

美猴王正默觀看景致,只聽得有人言語,徑自下山尋覓。原來那陡崖之前,乃是
那水臟洞。洞門外有幾個小妖跳舞,見了悟空就走。悟空道:「休走!借你口中
言,傳我心內事。我乃正南方花果山水簾洞洞主。你家甚麼混世鳥魔,屢次欺我
兒孫,我特尋來,要與他見個上下。」

那小妖聽說,疾忙跑入洞裏報道:「大王,禍事了!」魔王道:「有甚禍事?」
小妖道:「洞外有猴頭稱為花果山水簾洞洞主,他說你屢次欺他兒孫,特來尋你
,見個上下哩。」魔王笑道:「我常聞得那些猴精說他有個大王,出家修行去,
想是今番來了。你們見他怎生打扮?有甚器械?」小妖道:「他也沒甚麼器械,
光著個頭,穿一領紅色衣,勒一條黃絛,足下踏一對烏靴,不僧不俗,又不像道
士、神仙,赤手空拳,在門外叫哩。」魔王聞說:「取我披掛、兵器來。」那小
妖即時取出。

那魔王穿了甲冑,綽刀在手,與眾妖出得門來,即高聲叫道:「那個是水簾洞洞
主?」悟空急睜睛觀看,只見那魔王:
頭戴烏金盔,映日光明﹔身掛皂羅袍,迎風飄蕩。下穿著黑鐵甲,緊勒皮條﹔足
踏著花褶靴,雄如上將。腰廣十圍,身高三丈。手執一口刀,鋒刃多明亮。稱為
混世魔,磊落兇模樣。


猴王喝道:「這潑魔這般眼大,看不見老孫。」魔王見了,笑道:「你身不滿四
尺,年不過三旬,手內又無兵器,怎麼大膽猖狂,要尋我見甚麼上下?」悟空罵
道:「你這潑魔,原來沒眼。你量我小,要大卻也不難﹔你量我無兵器,我兩隻
手勾著天邊月哩。你不要怕,只吃老孫一拳。」縱一縱,跳上去,劈臉就打。那
魔王伸手架住道:「你這般矬矮,我這般高長﹔你要使拳,我要使刀。使刀就殺
了你,也吃人笑。待我放下刀,與你使路拳看。」悟空道:「說得是。好漢子,
走來。」那魔王丟開架子便打,這悟空鑽進去相撞相迎。他兩個拳搥腳踢,一沖
一撞。原來長拳空大,短簇堅牢。那魔王被悟空掏短脅,撞了襠,幾下觔節,把
他打重了。他閃過,拿起那板大的鋼刀,望悟空劈頭就砍。悟空急撤身,他砍了
一個空。悟空見他兇猛,即使身外身法,拔一把毫毛,丟在口中嚼碎,望空噴去
,叫一聲:「變!」即變做三、二百個小猴,週圍攢簇。

原來人得仙體,出神變化無方。不知這猴王自從了道之後,身上有八萬四千毛羽
,根根能變,應物隨心。那些小猴眼乖會跳,刀來砍不著,槍去不能傷。你看他
前踴後躍,鑽上去,把個魔王圍繞,抱的抱,扯的扯,鑽襠的鑽襠,扳腳的扳腳
,踢打撏毛,摳眼睛,捻鼻子,抬鼓弄,直打做一個攢盤。這悟空才去奪得他的
刀來,分開小猴,照頂門一下,砍為兩段。領眾殺進洞中,將那大小妖精盡皆剿
滅。卻把毫毛一抖,收上身來,又見那收不上身者,卻是那魔王在水簾洞擒去的
小猴。悟空道:「汝等何為到此?」約有三五十個,都含淚道:「我等因大王修
仙去後,這兩年被他爭吵,把我們都攝將來。那不是我們洞中的家火?石盆、石
碗都被這廝拿來也。」悟空道:「既是我們的家火,你們都搬出外去。」隨即洞
裏放起火來,把那水臟洞燒得枯乾,盡歸了一體。對眾道:「汝等跟我回去。」
眾猴道:「大王,我們來時,只聽得耳邊風響,虛飄飄到於此地,更不識路徑,
今怎得回鄉?」悟空道:「這是他弄的個術法兒,有何難也?我如今一竅通,百
竅通,我也會弄。你們都合了眼,休怕。」

好猴王,念聲咒語,駕陣狂風,雲頭落下。叫:「孩兒們睜眼。」眾猴腳屣實地
,認得是家鄉,個個歡喜,都奔洞門舊路。那在洞眾猴,都一齊簇擁同入,分班
序齒,禮拜猴王。安排酒果,接風賀喜,啟問降魔救子之事。悟空備細言了一遍
,眾猴稱揚不盡道:「大王去到那方,不意學得這般手段。」悟空又道:「我當
年別汝等,隨波逐流,飄過東洋大海,到西牛賀洲地界,徑至南贍部洲,學成人
像,著此衣,穿此履,擺擺搖搖,雲遊了八九年餘,更不曾有道。又渡西洋大海
,到西牛賀洲地界,訪問多時,幸遇一老祖,傳了我與天同壽的真功果,不死長
生的大法門。」眾猴稱賀,都道:「萬劫難逢也!」悟空又笑道:「小的們,又
喜我這一門皆有姓氏。」眾猴道:「大王姓甚?」悟空道:「我今姓孫,法名悟
空。」眾猴聞說,鼓掌忻然道:「大王是老孫,我們都是二孫、三孫、細孫、小
孫……一家孫、一國孫、一窩孫矣!」都來奉承老孫,大盆小碗的椰子酒、葡萄
酒、仙花、仙果,真個是合家歡樂。咦!

    貫通一姓身歸本,只待榮遷仙籙名。
    畢竟不知怎生結果,居此界終始如何,且聽下回分解。





}  \end{pinyinscope}\switchcolumn{\myfontc \section{第 二 回} , 悟 彻 菩 提 真 妙 理 , 断 魔 归 本 合 元 神 话 表 美 猴 王 得 到 了 姓 名 , 欣 然 得 意 , 对 着 菩 提 先 前 行 礼 表 示 谢 恩 。
那 祖 师 就 命 令 大 众 引 导 孙 慧 空 出 门 , 教 他 洒 扫 应 对 、 进 退 周 旋 的 礼 节 。
众 仙 奉 行 而 出 。
悟 空 到 了 门 外 , 又 拜 见 大 众 师 兄 , 就 在 廊 之 间 安 排 睡 觉 的 地 方 。
第 二 天 早 晨 , 与 众 位 师 兄 学 习 言 语 礼 貌 、 讲 经 论 道 、 学 字 焚 香 。
每 天 都 是 这 样 。
闲 暇 的 时 候 , 就 扫 地 锄 园 , 养 花 修 树 , 找 柴 燃 火 , 挑 水 运 浆 。
凡 所 用 的 东 西 , 没 有 一 件 不 齐 备 的 。
在 洞 中 不 觉 转 瞬 六 七 年 。
一 天 , 祖 师 登 上 坛 高 坐 , 召 集 众 位 仙 人 , 开 讲 大 道 。
真 个 是 : 天 花 乱 坠 , 地 涌 金 莲 。
妙 演 三 乘 教 , 精 微 万 法 完 备 。
慢 慢 摇 动 尾 , 喷 出 珠 玉 , 响 震 雷 霆 震 动 九 天 。
说 一 会 道 , 讲 一 会 禅 , 三 家 的 配 合 本 来 就 是 如 此 。
开 明 一 字 诚 心 诚 意 , 指 导 无 生 了 悟 性 玄 玄 。
孙 子 在 旁 边 听 到 讲 经 , 喜 欢 得 到 他 抓 着 耳 朵 、 歪 着 嘴 巴 、 眉 毛 、 眼 睛 都 笑 了 , 他 忍 不 住 手 舞 着 , 脚 踏 着 脚 踏 。
忽 然 被 祖 师 看 见 了 , 叫 孙 悟 空 说 : 你 在 班 中 , 为 什 么 狂 狂 跳 舞 , 不 听 我 讲 话 , 悟 空 说 : 弟 子 诚 心 听 讲 , 听 到 老 师 父 妙 音 的 地 方 , 喜 不 自 胜 , 所 以 不 觉 作 出 这 样 奔 腾 的 样 子 。
希 望 师 父 宽 恕 我 的 罪 过 。
祖 师 说 : 你 既 然 认 识 妙 音 , 我 又 问 你 , 你 到 洞 中 多 少 时 间 了 , 悟 空 说 : 弟 子 本 来 糊 涂 , 不 知 道 多 少 时 节 。
只 记 得 灶 下 没 有 火 , 经 常 去 山 后 打 柴 , 看 见 一 座 山 好 桃 树 , 我 在 那 里 吃 了 七 次 饱 桃 子 了 。
祖 师 说 : 那 山 叫 它 烂 桃 山 。
你 既 吃 七 次 , 想 是 七 年 了 。
你 现 在 要 跟 着 我 学 些 什 么 道 ? 悟 空 说 : 只 要 依 靠 尊 师 的 教 诲 , 只 是 有 些 道 气 , 弟 子 就 可 以 学 了 。
祖 师 说 : 道 字 门 中 有 三 百 六 十 个 旁 门 , 旁 门 都 有 正 果 。
不 知 你 学 到 哪 一 门 呢 ? 悟 空 说 : 依 你 老 师 的 意 思 , 弟 子 倾 心 听 从 。
祖 师 说 : 我 教 你 的 术 字 , 门 中 的 道 理 , 怎 么 样 ? 悟 空 说 : 术 门 之 道 , 怎 么 说 ? 祖 师 说 : 术 字 门 中 , 就 是 请 仙 、 扶 鸾 、 问 卜 、 , 能 够 知 道 趋 吉 避 凶 的 道 理 。
悟 空 说 : 像 这 样 可 以 长 生 吗 ? 祖 师 说 : 不 能 , 不 能 。
悟 空 说 : 不 学 , 不 学 。
祖 师 又 说 : 教 你 们 流 字 门 中 的 道 , 是 怎 么 样 的 ? 悟 空 又 问 : 流 字 门 中 是 什 么 义 理 ? 祖 师 说 : 流 字 门 中 , 就 是 儒 家 、 释 家 、 道 家 、 阴 阳 家 、 墨 家 、 医 家 , 或 者 看 经 , 或 者 念 佛 , 或 者 朝 真 降 圣 之 类 。
悟 空 说 : 像 这 样 可 以 长 生 吗 ? 祖 师 说 : 如 果 要 长 生 , 也 就 像 墙 壁 里 安 着 柱 子 一 样 。
悟 空 说 : 师 父 , 我 是 个 老 实 人 , 不 晓 得 打 市 语 。
祖 师 说 : 人 家 盖 房 子 , 想 要 坚 固 , 就 在 墙 壁 之 间 立 一 根 顶 柱 , 有 一 天 大 厦 将 要 塌 毁 , 其 他 必 定 会 朽 烂 了 。
悟 空 说 : 根 据 这 种 说 法 , 也 不 会 长 久 。
不 学 , 不 学 。
祖 师 说 : 教 你 们 静 字 门 中 的 道 , 是 怎 么 样 的 ? 悟 空 说 : 静 字 门 中 是 什 么 正 果 ? 祖 师 说 : 这 是 休 粮 守 谷 、 清 静 无 为 、 参 禅 打 坐 、 戒 语 持 斋 , 或 是 睡 功 , 或 是 立 功 , 或 是 入 定 、 坐 关 之 类 。
悟 空 说 : 这 样 也 能 长 生 不 老 吗 ? 祖 师 说 : 也 好 像 头 土 块 。
悟 空 笑 着 说 : 师 父 果 然 有 点 滴 吗?
一 行 说 : 我 不 会 打 市 语 。
祖 师 说 : 就 像 那 么 , 头 上 造 成 砖 瓦 的 砖 坯 , 虽 然 已 经 形 成 , 但 还 没 有 经 过 水 火 锻 炼 , 一 旦 下 大 雨 滂 沱 , 其 他 必 定 会 混 乱 。
悟 空 说 : 也 不 长 远 。
不 学 , 不 学 。
祖 师 说 : 你 教 你 动 字 门 中 之 道 , 怎 么 样 ? 悟 空 说 : 动 门 之 道 , 又 怎 么 样 ? 祖 师 说 : 这 是 有 作 的 : 采 阴 补 阳 、 攀 弓 踏 弩 、 摩 胸 过 气 、 用 方 炮 制 、 烧 茅 打 鼎 、 进 红 铅 、 炼 秋 石 , 并 服 妇 乳 之 类 。
悟 空 说 : 像 这 些 人 也 能 长 生 不 老 吗 ? 祖 师 说 : 这 些 人 想 长 生 不 老 , 也 就 像 水 中 捉 月 一 样 。
悟 空 说 : 师 父 又 来 了 。
祖 师 说 : 月 亮 在 空 中 , 水 中 有 影 子 , 虽 然 看 见 , 但 是 没 有 捉 摸 的 地 方 , 到 底 只 是 空 了 。
悟 空 说 : 也 不 学 , 不 学 。
祖 师 听 了 , 咄 呀 一 声 , 跳 下 高 台 , 手 拿 戒 尺 指 定 悟 空 说 : 你 这 样 的 , 这 样 不 学 , 那 样 不 学 , 还 等 什 么 ? 他 跑 上 前 , 把 悟 空 的 脑 袋 打 了 三 下 。
他 倒 背 着 手 , 走 进 里 面 , 将 中 门 关 上 , 把 大 众 放 下 去 。
听 到 这 一 班 听 讲 的 人 人 惊 恐 , 都 怨 恨 悟 空 说 : 你 这 泼 猴 , 十 分 无 礼 。
师 父 传 授 你 的 道 法 , 你 为 什 么 不 学 , 却 与 师 父 一 道 顶 嘴 , 这 次 冲 撞 了 他 , 不 知 何 时 才 出 来 呀 ! 这 时 都 很 报 怨 他 , 又 鄙 贱 嫌 恶 他 。
悟 空 一 点 儿 子 也 不 恼 , 只 是 满 脸 笑 。
原 来 那 猴 王 已 经 打 破 了 盘 中 的 谜 语 , 心 里 暗 暗 地 在 心 里 , 所 以 不 与 众 人 争 夺 , 只 是 忍 耐 着 没 说 话 。
祖 师 打 他 三 下 的 , 教 他 三 更 时 分 心 , 倒 背 着 手 走 进 里 面 , 将 中 门 关 上 的 , 教 他 从 后 门 进 步 , 秘 密 地 传 授 他 的 道 术 。
当 天 悟 空 和 众 人 喜 欢 欢 喜 , 在 三 星 仙 洞 的 前 面 , 望 着 天 色 , 急 忙 不 能 到 晚 上 。
到 了 黄 昏 的 时 候 , 他 又 和 众 人 一 起 睡 觉 , 假 借 眼 睛 , 定 息 保 存 精 神 。
山 中 又 没 有 打 , 更 换 传 箭 , 不 知 时 分 , 只 是 自 己 家 里 把 鼻 孔 中 的 出 入 之 气 调 定 。
大 约 到 子 时 前 后 , 轻 轻 地 起 来 , 穿 上 衣 服 , 偷 偷 打 开 前 门 , 避 开 大 众 , 走 出 外 面 , 抬 头 看 , 正 是 那 个 月 亮 清 露 冷 , 八 极 迥 无 尘 土 。
深 树 幽 禽 宿 , 源 头 水 流 注 汾 水 。
飞 萤 光 散 影 , 过 雁 字 排 云 。
正 直 三 更 时 候 , 应 该 去 拜 访 道 真 。
你 看 他 从 原 路 径 走 到 后 门 外 , 只 见 那 个 门 子 半 开 半 掩 。
悟 空 高 兴 地 说 : 老 师 父 果 然 注 意 给 我 传 授 道 教 , 所 以 我 才 开 门 。
于 是 就 拖 着 步 走 到 前 面 , 侧 着 身 子 进 到 门 里 , 只 走 到 祖 师 的 卧 榻 下 面 。
看 见 祖 师 的 身 体 , 早 晨 睡 着 了 。
悟 空 不 敢 惊 动 , 就 跪 在 床 前 。
那 位 祖 师 不 多 时 醒 来 , 伸 开 两 只 脚 , 口 中 自 吟 道 : 难 啊 难 啊 道 最 玄 , 莫 把 金 丹 作 等 闲 。
不 遇 至 人 传 授 妙 诀 , 空 说 口 困 舌 头 干 了 。 悟 空 应 声 大 叫 道 : 师 父 , 弟 子 在 这 里 跪 候 多 时 。
祖 师 听 到 了 声 音 是 悟 空 , 立 即 起 身 披 衣 , 盘 坐 着 喝 道 : 这 , 你 不 在 前 边 去 睡 , 却 来 我 这 后 边 做 什 么 ? 悟 空 说 : 师 父 昨 天 在 坛 前 面 对 众 人 说 , 教 弟 子 在 三 更 时 , 从 后 门 里 传 我 的 道 理 , 所 以 我 大 胆 直 接 拜 老 爷 的 床 下 。
祖 师 听 了 , 十 分 欢 喜 , 暗 自 寻 思 道 : 这 个 人 果 然 是 天 地 生 成 的 , 不 然 , 为 什 么 就 打 破 我 盘 中 的 暗 谜 呢 ? 悟 空 说 : 这 里 再 没 有 六 个 人 , 只 有 弟 子 一 个 人 , 希 望 师 父 大 发 慈 悲 , 传 给 我 长 生 之 道 罢 了 , 永 不 忘 恩 。
祖 师 说 : 你 现 在 有 缘 , 我 也 喜 欢 说 。
我 已 经 认 出 了 盘 中 的 暗 谜 , 你 近 来 , 仔 细 听 了 , 我 会 传 给 你 长 生 的 妙 道 。
悟 空 叩 头 谢 罪 , 洗 耳 用 心 , 跪 在 床 下 。
祖 师 说 : 显 密 圆 通 真 妙 诀 , 可 惜 修 性 命 没 有 别 的 说 法 。
都 来 总 是 精 气 神 , 谨 慎 地 保 持 牢 藏 , 不 要 泄 露 。
不 要 泄 露 , 身 体 中 藏 , 你 接 受 我 的 传 授 , 自 然 会 昌 盛 。
《 口 诀 》 记 载 下 来 , 多 有 益 处 , 除 去 邪 气 , 想 得 到 清 凉 。
得 到 清 凉 的 时 候 , 光 亮 皎 洁 , 喜 欢 去 丹 台 观 赏 明 月 。
月 藏 玉 兔 , 日 藏 乌 鸦 , 自 然 会 有 龟 蛇 互 相 盘 结 。
相 互 盘 结 , 性 命 坚 定 , 却 能 在 火 里 种 金 莲 。
聚 集 五 行 颠 倒 , 功 德 完 备 就 可 以 成 为 佛 和 仙 。
这 时 说 破 了 根 源 , 悟 空 心 灵 福 至 , 切 切 地 记 住 了 口 诀 。
对 祖 师 叩 拜 感 谢 深 深 的 恩 德 , 就 出 了 后 门 观 看 。
只 见 东 方 天 色 微 微 舒 缓 , 白 色 , 西 路 金 光 十 分 明 亮 。
仍 旧 路 , 转 到 前 门 , 轻 轻 地 把 他 推 进 去 , 坐 在 原 来 的 床 铺 上 , 故 意 把 床 铺 摇 动 着 说 : 天 光 了 , 天 光 了 , 起 来 了 , 那 大 众 还 正 在 睡 着 呀 , 不 知 道 悟 空 已 经 得 到 了 好 事 。
当 天 起 来 打 混 , 暗 暗 地 维 持 , 在 午 后 , 自 己 自 己 调 息 。
又 过 了 三 年 , 祖 师 又 登 上 宝 座 , 与 众 人 讲 说 佛 法 。
谈 论 的 是 公 案 比 语 , 评 论 的 是 外 面 的 包 皮 。
忽 然 问 道 : 悟 空 在 哪 里 ? 悟 空 走 近 前 跪 下 说 : 弟 子 有 。
祖 师 说 : 你 这 一 向 修 了 什 么 道 来 ? 悟 空 说 : 弟 子 近 来 法 性 颇 通 , 根 源 也 渐 渐 坚 固 了 。
祖 师 说 : 你 既 已 通 晓 佛 法 的 本 性 , 也 就 已 经 注 入 了 神 体 , 却 只 是 防 备 了 三 灾 利 害 。
悟 空 听 了 , 沉 吟 了 很 久 , 说 : 师 父 的 话 错 了 。
我 常 常 听 说 道 德 高 尚 , 与 上 天 同 寿 , 水 火 既 济 , 百 病 不 生 。
祖 师 说 : 这 是 非 常 之 道 , 夺 取 天 地 的 造 化 , 侵 害 日 月 的 玄 机 。
虽 然 我 已 老 , 但 是 到 了 五 百 年 以 后 , 天 降 雷 灾 打 你 , 必 须 明 白 自 己 的 本 性 , 先 要 躲 避 。
躲 避 得 过 , 寿 命 与 上 天 一 样 ; 躲 避 不 过 , 就 此 绝 命 。
再 过 五 百 年 , 天 降 火 灾 , 烧 掉 你 。
其 火 不 是 天 火 , 也 不 是 凡 火 , 叫 做 阴 火 。
从 自 己 的 自 己 的 涌 泉 洞 中 烧 起 来 , 一 直 透 过 泥 墙 宫 , 五 脏 变 成 灰 烬 , 四 肢 都 朽 烂 了 , 把 千 年 苦 行 都 变 成 了 虚 幻 。
再 过 五 百 年 , 又 降 大 风 吹 你 。
其 风 不 是 东 南 西 北 风 , 不 是 和 薰 、 金 朔 风 , 也 不 是 花 柳 、 松 竹 风 , 叫 做 风 。
从 门 中 吹 入 六 腑 , 经 过 丹 田 , 穿 过 九 窍 , 骨 肉 消 散 , 身 体 自 然 解 除 。
所 以 , 不 可 以 为 之 。
悟 空 听 了 这 话 , 毛 骨 悚 然 , 叩 头 礼 拜 说 : 希 望 老 爷 怜 悯 , 传 给 他 躲 避 三 灾 之 法 , 到 底 不 敢 忘 恩 。
祖 师 说 : 这 也 不 难 , 只 是 你 与 其 他 人 不 同 , 所 以 传 不 得 。
悟 空 说 : 我 也 头 圆 顶 天 , 脚 方 履 地 , 一 般 有 九 窍 四 肢 、 五 脏 六 腑 , 与 人 不 同 , 祖 师 说 : 你 虽 然 像 人 , 但 却 比 人 少 腮 。
原 来 那 个 猴 子 孤 拐 面 , 凹 脸 尖 嘴 。
悟 空 伸 手 一 摸 , 笑 着 说 : 师 父 没 有 成 算 。
我 虽 然 很 少 嘴 巴 , 却 比 别 人 多 这 个 白 口 袋 , 也 可 以 折 过 去 。
祖 师 说 : 也 罢 , 你 要 学 到 那 一 般 , 有 一 种 天 地 的 气 数 , 该 是 三 十 六 种 变 化 ; 有 一 个 地 狱 数 , 该 是 七 十 二 种 变 化 。
悟 空 说 : 弟 子 愿 意 多 里 摸 摸 , 学 一 个 地 狱 变 化 罢 了 。
祖 师 说 : 既 然 这 样 , 皇 上 前 来 , 传 给 你 的 口 诀 。
于 是 就 附 着 耳 朵 低 声 说 : 不 知 道 了 什 么 妙 法 。
这 个 猴 子 , 是 他 一 窍 通 、 百 窍 通 , 当 时 学 习 了 口 诀 , 自 己 修 炼 , 将 七 十 二 种 变 化 都 学 成 了 。
忽 然 有 一 天 , 祖 师 和 众 位 门 人 在 三 星 洞 前 玩 玩 晚 景 。
祖 师 说 : 悟 空 , 事 情 成 功 了 还 没 有 什 么 ? 悟 空 说 : 多 蒙 师 父 海 恩 , 弟 子 功 果 完 备 , 已 经 能 升 天 飞 升 了 。
祖 师 说 : 你 试 试 飞 举 我 看 。
悟 空 戏 弄 本 事 , 把 自 己 的 身 子 耸 起 来 , 打 了 连 扯 跟 头 , 跳 到 地 面 有 五 六 丈 , 踏 着 云 霞 去 掉 , 有 一 顿 饭 的 时 候 , 返 回 不 到 三 里 远 近 , 落 在 脸 前 , 拍 着 手 说 : 师 父 , 这 就 是 飞 举 腾 云 了 。
祖 师 笑 着 说 : 这 个 算 不 得 腾 云 , 只 算 得 上 攀 云 而 已 。
自 古 以 来 有 道 : 神 仙 早 晨 游 北 海 , 暮 暮 苍 梧 。
像 你 这 半 天 , 去 不 上 三 里 , 就 是 攀 云 也 还 算 不 到 了 。
悟 空 说 : 为 什 么 要 早 上 游 北 海 暮 上 苍 梧 呢 祖 师 说 : 凡 是 腾 云 之 类 , 早 晨 起 自 北 海 , 游 历 过 东 海 、 西 海 、 南 海 , 又 转 到 苍 梧 。
苍 梧 , 却 是 北 海 零 陵 的 语 言 。
将 四 海 之 外 , 一 天 都 游 遍 , 才 算 得 上 腾 云 。
悟 空 说 : 这 个 更 难 , 还 难 。
祖 师 说 : 世 上 没 有 难 事 , 只 怕 有 心 的 人 。
悟 空 听 到 这 话 , 叩 头 礼 拜 , 启 奏 道 : 师 父 , 为 人 必 须 彻 底 , 我 本 性 舍 弃 这 个 大 慈 悲 , 将 这 种 腾 云 的 法 术 , 一 发 传 给 我 罢 了 , 决 不 敢 忘 恩 。
祖 师 说 : 凡 是 诸 仙 腾 云 , 都 跌 脚 而 起 , 你 却 不 是 这 样 。
我 才 见 你 去 , 连 扯 方 才 跳 上 去 。
我 现 在 只 有 你 这 样 的 权 势 , 传 给 你 个 斗 云 罢 了 。
觉 空 又 礼 拜 恳 求 , 祖 师 又 传 了 一 个 口 诀 , 说 : 这 朵 云 , 捻 着 诀 , 念 动 真 言 , 将 身 一 抖 , 跳 起 来 , 一 斗 就 有 十 万 八 千 里 路 了 。
大 众 听 了 , 一 个 个 笑 笑 地 说 : 我 们 觉 得 自 己 的 造 化 , 如 果 会 了 这 个 法 术 , 与 人 家 当 铺 兵 、 送 文 书 、 递 报 书 , 不 管 哪 里 都 找 了 饭 吃 。
师 徒 们 在 天 黑 时 分 别 回 到 洞 府 。
这 一 夜 , 悟 空 就 运 神 炼 法 , 会 合 了 斗 云 。
追 逐 日 子 家 , 无 拘 无 束 , 自 在 逍 遥 , 这 也 是 长 生 不 老 的 美 德 。
有 一 天 , 春 天 回 到 夏 至 , 大 家 都 在 松 树 下 会 讲 了 很 长 时 间 。
大 众 说 : 悟 空 , 你 是 那 世 修 来 的 缘 法 , 前 日 老 师 父 附 耳 低 声 说 , 传 给 你 的 躲 避 三 灾 变 化 之 法 , 可 以 都 会 吗 ? 悟 空 笑 着 说 : 不 瞒 众 兄 长 说 , 一 则 是 师 父 传 授 , 二 来 也 是 我 昼 夜 , 那 几 个 儿 子 都 会 了 吗 ?
大 众 说 : 趁 这 个 好 时 机 , 你 试 试 演 演 , 让 我 们 看 看 。
悟 空 听 了 这 话 , 顿 时 抽 出 精 神 , 戏 弄 手 段 说 : 众 位 师 兄 , 请 让 我 出 个 题 目 。
要 我 变 化 什 么 ? 大 众 说 : 就 变 了 一 棵 松 树 罢 了 。
悟 空 念 咒 语 , 转 身 一 变 , 就 变 成 了 一 棵 松 树 。
真 的 是 : 郁 郁 含 烟 贯 四 时 , 凌 云 直 上 秀 贞 姿 。
完 全 没 有 一 点 妖 猴 的 形 象 , 全 是 经 霜 耐 雪 的 树 枝 。
众 人 见 之 , 鼓 掌 大 笑 , 都 说 : 好 猴 儿 , 好 猴 儿 。
祖 师 急 忙 拽 着 拐 杖 出 门 来 问 道 : 是 什 么 人 在 这 里 喧 闹 喧 哗 , 大 众 听 到 呼 叫 , 慌 忙 检 束 起 来 , 整 理 好 衣 服 向 前 走 。
悟 空 现 了 自 己 的 本 相 , 杂 在 草 丛 中 说 : 请 上 尊 师 说 : 我 们 在 这 里 讲 经 , 再 没 有 外 姓 喧 闹 。
祖 师 发 怒 , 喝 道 : 你 们 大 喊 小 叫 , 完 全 不 像 个 修 行 的 人 , 口 开 神 气 散 , 舌 动 是 非 , 为 什 么 在 这 里 喧 闹 笑 笑 ? 大 众 说 : 不 敢 欺 骗 师 父 , 刚 才 孙 悟 空 演 变 化 戏 子 。
教 他 变 了 一 棵 松 树 , 果 然 是 一 棵 松 树 , 弟 子 们 都 称 赞 声 誉 , 所 以 高 声 惊 吓 冒 犯 尊 师 , 希 望 饶 恕 我 的 罪 过 。
祖 师 说 : 你 们 一 起 去 吧 。
又 叫 : 我 问 你 弄 什 么 精 神 , 变 成 什 么 松 树 ? 这 个 工 夫 , 可 以 在 人 面 前 卖 弄 , 假 如 你 见 别 人 有 , 不 要 求 他 , 别 人 见 你 有 , 必 然 要 求 你 。
你 如 果 害 怕 祸 患 , 却 要 传 给 他 ; 如 果 不 传 给 他 , 必 然 要 加 害 于 你 ; 你 的 性 命 又 不 可 保 。
悟 空 叩 头 说 : 只 希 望 师 父 恕 罪 。
祖 师 说 : 我 也 不 责 怪 你 , 只 是 你 离 开 罢 了 。
悟 空 听 了 这 话 , 满 眼 泪 流 满 面 说 : 师 父 教 我 到 哪 里 去 , 祖 师 说 : 你 从 哪 里 来 , 就 从 那 里 去 就 是 了 。
悟 空 顿 时 醒 悟 , 说 : 我 是 从 东 胜 神 洲 、 傲 来 国 花 果 山 、 水 帘 洞 来 的 。
祖 师 说 : 你 快 回 去 , 保 全 你 的 性 命 , 如 果 在 这 里 , 断 然 不 可 。
悟 空 领 罪 , 皇 上 告 诉 尊 师 说 : 我 离 家 已 经 二 十 年 了 , 虽 然 是 回 想 过 去 的 儿 孙 , 但 念 念 师 父 的 厚 恩 未 报 , 不 敢 离 去 。
祖 师 说 : 那 里 有 什 么 恩 义 , 你 只 是 不 惹 祸 , 不 牵 连 我 就 罢 了 。
悟 空 见 到 没 有 办 法 , 只 得 下 拜 辞 别 , 和 众 人 告 别 。
祖 师 说 : 你 这 样 去 , 一 定 会 生 不 良 。
凭 你 为 什 么 惹 祸 行 凶 , 却 不 许 说 是 我 的 弟 弟 。
你 说 出 半 个 字 来 , 我 就 知 道 了 , 把 你 这 个 剥 皮 锯 骨 , 将 神 魂 贬 到 九 幽 之 处 , 教 你 万 劫 不 得 翻 身 ! 悟 空 说 : 决 不 敢 提 起 师 父 一 个 字 , 只 说 是 我 自 家 会 的 就 罢 了 。
悟 空 辞 谢 了 , 立 即 抽 出 身 子 , 捻 着 诀 诀 , 丢 掉 了 连 扯 , 纵 然 起 来 斗 云 , 径 直 回 到 东 胜 县 。
其 中 有 一 个 时 间 , 早 早 就 看 到 花 果 山 、 水 帘 洞 。
美 猴 王 自 知 快 乐 , 自 称 道 : 去 世 时 凡 骨 , 凡 胎 重 , 得 道 后 身 体 轻 , 身 体 也 轻 。
整 个 社 会 上 没 有 人 肯 立 志 , 立 志 修 身 养 性 , 玄 玄 自 然 明 白 。
当 时 经 过 海 波 难 以 前 进 , 今 天 回 来 很 容 易 走 。
别 语 叮 嘱 , 还 在 耳 边 , 何 期 顷 刻 见 东 海 。
悟 空 按 下 云 头 , 直 到 花 果 山 , 找 路 而 去 。
忽 然 听 到 鹤 鸣 猿 啼 : 鹤 鸣 声 冲 天 汉 之 外 , 猿 啼 悲 切 , 很 伤 情 。
随 即 开 口 大 叫 道 : 孩 子 们 , 我 来 了 。 那 崖 下 石 坎 边 , 花 草 中 , 树 木 里 , 若 大 若 小 的 猴 子 , 跳 出 千 千 万 万 , 把 那 个 美 猴 王 围 在 当 中 , 叩 头 大 叫 道 : 大 王 , 你 们 好 宽 心 , 为 何 一 去 许 久 , 把 我 们 都 闪 在 这 里 , 希 望 你 们 诚 如 饥 渴 。
近 来 被 一 个 妖 魔 在 这 里 , 强 行 占 我 们 的 水 帘 洞 府 , 是 我 们 舍 死 忘 生 , 与 他 们 争 斗 。
这 些 时 候 , 被 那 个 人 抢 了 我 们 家 的 火 , 捉 住 了 许 多 子 侄 , 教 我 们 日 夜 不 睡 , 看 守 家 业 。
幸 好 大 王 来 了 , 大 王 如 果 再 年 载 不 来 , 我 们 连 山 洞 都 归 属 于 他 人 了 。
悟 空 听 了 这 话 , 心 中 大 怒 , 说 : 是 什 么 妖 魔 , 竟 敢 没 有 表 现 , 你 暂 且 细 细 说 一 说 , 等 我 找 他 报 仇 。
众 猴 叩 头 说 : 告 诉 上 大 王 , 那 个 人 自 称 混 世 魔 王 , 住 在 直 北 下 。
悟 空 说 : 此 间 到 他 那 里 , 有 多 少 路 程 ? 众 猴 说 : 他 来 时 云 , 去 时 雾 , 有 的 风 , 有 的 雨 , 有 的 电 , 有 的 雷 , 我 们 不 知 道 有 多 少 路 。
悟 空 说 : 既 然 这 样 , 你 们 都 不 害 怕 , 暂 且 自 己 捏 傻 , 等 我 找 他 去 来 。
猴 王 , 将 身 体 放 了 一 跃 , 跳 起 去 , 一 路 上 打 斗 , 直 到 北 面 下 去 观 看 , 看 见 一 座 高 山 , 真 是 十 分 险 峻 。
好 山 : 笔 峰 挺 立 , 曲 涧 深 沉 。
笔 峰 挺 立 , 穿 透 空 中 , 弯 弯 的 沟 涧 深 沉 , 通 到 地 上 的 门 户 。
两 岸 的 山 崖 上 花 木 争 奇 , 几 处 的 松 树 和 竹 林 争 秀 。
左 边 的 龙 , 熟 熟 地 驯 服 ; 右 边 的 虎 , 平 平 地 伏 伏 。
每 次 看 见 铁 牛 耕 地 , 常 有 金 钱 的 种 子 。
幽 禽 鸣 叫 , 丹 凤 朝 阳 立 。
石 头 闪 闪 , 波 涛 清 净 , 古 怪 怪 , 小 路 上 真 是 凶 恶 凶 狠 。
世 上 名 山 没 有 多 少 , 花 开 花 谢 , 还 有 很 多 。
争 夺 如 此 , 永 远 保 存 , 八 节 四 时 浑 不 动 。
真 是 三 界 坎 源 山 , 滋 养 五 行 水 脏 洞 。
美 猴 王 正 在 默 默 地 观 看 景 致 , 只 听 到 有 人 说 话 , 径 直 下 山 寻 找 。
原 来 那 陡 崖 之 前 , 就 是 那 个 水 脏 洞 。
洞 门 外 有 几 个 小 妖 跳 舞 , 见 了 悟 空 就 走 了 。
悟 空 说 : 不 要 走 , 借 你 口 中 的 话 , 传 我 心 中 的 事 。
我 是 正 南 方 花 果 山 水 帘 洞 的 主 人 。
你 家 什 么 混 世 的 鸟 魔 , 多 次 欺 骗 我 的 儿 孙 , 我 特 意 找 来 , 要 与 他 见 一 个 上 下 。
那 个 小 妖 听 了 , 急 忙 跑 到 洞 里 报 告 说 : 大 王 , 祸 事 了 。 魔 王 说 : 有 什 么 祸 事 ? 小 妖 说 : 洞 外 有 猴 头 称 为 花 果 山 水 帘 洞 洞 主 , 他 说 你 屡 次 欺 骗 他 的 儿 孙 , 特 来 找 你 , 见 到 你 的 上 下 吗 ?
魔 王 笑 着 说 : 我 经 常 听 到 那 个 猴 子 说 他 有 个 大 王 , 出 家 修 行 , 想 是 今 天 来 了 。
小 妖 说 : 他 也 没 什 么 器 械 , 光 着 个 头 , 穿 一 领 红 色 衣 服 , 穿 一 条 黄 色 裤 子 , 脚 下 踏 一 对 乌 靴 , 不 是 僧 人 不 俗 人 , 又 不 像 道 士 、 神 仙 , 赤 手 空 拳 , 在 门 外 叫 喊 哩 。
魔 王 听 了 说 : 拿 我 的 披 挂 、 兵 器 来 。
那 个 小 妖 立 刻 取 出 来 。
那 魔 王 穿 上 铠 甲 , 挥 刀 在 手 , 和 众 妖 出 门 来 , 立 即 高 声 喊 道 : 那 个 是 水 帘 洞 的 洞 主 啊 悟 空 急 忙 睁 开 眼 睛 观 看 , 只 见 那 个 魔 王 , 头 戴 乌 金 铠 , 映 日 光 明 , 身 上 挂 着 黑 色 绫 罗 袍 , 迎 风 飘 荡 。
下 面 穿 着 黑 色 的 铁 甲 , 紧 紧 勒 住 皮 条 , 脚 踏 着 花 褶 靴 子 , 雄 健 如 上 将 。
腰 宽 十 围 , 身 高 三 丈 。
他 手 里 拿 着 一 把 口 刀 , 刀 刃 大 多 明 亮 。
被 称 为 混 世 魔 , 磊 落 凶 模 。
猴 王 喝 道 : 这 泼 魔 这 样 的 眼 睛 大 , 看 不 见 老 孙 子 。
魔 王 见 了 , 笑 着 说 : 你 身 不 满 四 尺 , 年 龄 不 过 三 十 , 手 里 又 没 有 兵 器 , 何 必 大 胆 猖 狂 , 要 寻 找 我 见 什 么 上 下 呢 ? 悟 空 骂 道 : 你 这 泼 魔 , 原 来 没 有 眼 睛 。
你 估 量 我 小 , 想 要 大 却 也 不 难 , 你 估 量 我 没 有 兵 器 , 我 两 只 手 勾 着 天 边 月 亮 啊 。
君 不 可 以 , 不 可 以 为 之 , 不 可 以 为 之 。
张 纵 一 纵 , 跳 上 去 , 劈 开 脸 就 打 。
那 个 魔 王 伸 手 搭 住 道 : 你 这 样 矮 小 , 我 这 样 高 高 长 大 ; 你 要 用 拳 头 , 我 要 用 刀 。
使 刀 就 杀 了 你 , 也 会 吃 人 笑 。
等 我 放 下 刀 , 给 你 使 路 拳 头 看 。
悟 空 说 : 说 得 对 。
好 汉 子 , 跑 来 。
那 魔 王 抛 开 架 子 就 打 , 这 个 悟 空 钻 进 去 , 互 相 撞 撞 , 互 相 迎 接 。
其 两 个 人 , 一 个 人 , 一 个 人 , 一 个 人 , 一 个 人 , 一 个 人 , 一 个 人 , 一 个 人 , 一 个 人 , 一 个 人 , 一 个 人 , 一 个 人 , 一 个 人 , 一 个 人 , 一 个 人 , 一 个 人 , 一 个 人 , 一 个 人 。
原 来 长 拳 空 大 , 短 袖 坚 硬 。
其 他 魔 王 被 悟 空 的 短 肋 , 撞 了 裤 子 , 几 下 节 , 把 它 打 得 重 了 。
他 回 来 后 , 他 就 去 了 , 拿 起 那 板 大 的 钢 刀 , 望 着 悟 空 , 劈 开 头 就 砍 。
悟 空 急 忙 撤 下 身 子 , 他 杀 了 一 个 空 子 。
悟 空 见 他 凶 猛 , 就 用 身 外 身 法 , 拔 出 一 把 毫 毛 , 扔 在 口 中 嚼 碎 , 望 着 空 中 喷 出 去 , 叫 一 声 : 变 , 立 即 变 成 三 个 二 百 个 小 猴 子 , 周 围 簇 拥 。
原 来 人 得 到 了 仙 体 , 出 了 神 , 变 化 无 方 。
不 知 道 这 个 猴 王 自 从 悟 道 以 后 , 身 上 有 八 万 四 千 毛 羽 , 根 根 能 变 化 , 随 着 事 物 随 心 。
其 他 人 之 所 以 不 知 其 所 以 不 知 其 所 以 , 不 知 其 所 以 不 知 其 所 以 不 知 其 所 以 不 知 其 所 以 不 知 其 所 以 不 知 其 所 以 不 知 其 所 以 不 知 其 所 以 不 知 其 所 以 不 知 其 所 以 不 知 其 所 以 为 之 。
你 看 他 前 踊 后 跃 , 跳 上 去 , 把 妖 王 围 绕 , 抱 着 抱 , 拽 得 扯 , 钻 着 裤 子 , 敲 着 脚 , 踢 打 毛 , 敲 打 眼 睛 , 捻 着 鼻 子 , 抬 着 鼓 弄 , 直 打 成 一 个 攒 盘 。
这 个 悟 空 刚 去 , 夺 得 他 的 刀 来 , 分 开 小 猴 , 照 着 顶 门 一 下 , 砍 成 两 段 。
他 率 领 众 人 杀 进 洞 中 , 将 那 些 大 小 妖 精 全 部 剿 灭 。
又 看 见 那 个 不 上 身 的 妖 怪 , 就 是 那 个 妖 王 在 水 帘 洞 捉 住 了 他 的 小 猴 子 。
悟 空 说 : 你 们 为 什 么 到 这 里 来 ? 大 约 有 三 五 十 个 , 都 含 着 眼 泪 说 : 我 们 因 为 大 王 修 炼 成 仙 后 , 这 两 年 被 他 们 争 吵 , 把 我 们 都 抓 来 了 。
不 是 我 们 洞 中 的 家 火 , 石 盆 、 石 碗 都 被 这 个 人 拿 来 。
悟 空 说 : 既 然 是 我 们 家 的 火 , 你 们 都 搬 出 外 去 。
随 即 洞 里 放 出 火 来 , 把 那 个 水 脏 洞 烧 得 干 枯 了 , 全 都 回 到 了 一 个 身 体 。
对 众 人 说 : 你 们 跟 我 回 去 。
众 猴 说 : 大 王 , 我 们 来 的 时 候 , 只 听 到 耳 边 风 声 , 虚 飘 飘 到 这 个 地 方 , 更 不 认 识 路 径 , 现 在 怎 么 回 乡 呢 ? 悟 空 说 : 这 是 他 弄 的 个 术 法 儿 , 有 什 么 困 难 呢 我 现 在 一 窍 通 , 百 窍 通 了 , 我 也 会 弄 。
你 们 都 合 上 眼 睛 , 不 用 害 怕 。
猴 王 念 咒 语 , 乘 风 刮 起 狂 风 , 云 头 掉 下 来 。
又 叫 道 : 孩 子 们 睁 眼 。
众 猴 子 脚 踏 着 地 , 认 得 是 家 乡 , 一 个 个 都 欢 喜 , 都 奔 往 洞 门 的 旧 路 。
在 洞 中 的 众 猴 , 都 一 齐 簇 拥 着 一 齐 进 去 , 分 班 序 齿 , 礼 拜 猴 王 。
安 排 酒 果 , 迎 风 祝 贺 喜 庆 , 启 奏 皇 帝 询 问 降 魔 救 子 的 事 情 。
悟 空 详 细 地 说 了 一 遍 , 众 猴 都 赞 扬 不 尽 说 : 大 王 去 到 那 方 , 没 想 到 我 学 到 这 样 的 手 段 。
悟 空 又 说 : 我 当 年 与 你 们 告 别 , 随 波 逐 流 , 漂 过 东 洋 大 海 , 到 西 牛 贺 洲 地 界 , 径 直 到 南 赡 部 洲 , 学 成 人 像 , 穿 着 这 种 衣 服 , 穿 着 这 种 鞋 子 , 摇 动 摇 摇 , 云 游 八 九 年 多 , 再 也 不 曾 有 道 。
又 渡 过 西 洋 大 海 , 到 西 牛 贺 洲 地 界 , 访 问 了 许 多 时 间 , 有 幸 遇 到 一 位 老 师 , 传 授 我 与 天 同 寿 的 真 功 果 , 不 死 长 生 的 大 法 门 。
众 猴 都 说 : 万 劫 难 逢 啊 ! 悟 空 又 笑 着 说 : 小 的 人 , 又 喜 欢 我 这 一 家 都 有 姓 氏 。
众 猴 说 : 大 王 姓 什 么 ? 悟 空 说 : 我 现 在 姓 孙 , 法 名 叫 悟 空 。
众 猴 听 了 , 拍 手 高 兴 地 说 : 大 王 是 老 孙 , 我 们 都 是 二 孙 、 三 孙 、 细 孙 、 小 孙 、 一 家 孙 、 一 国 孙 、 一 窝 孙 啊 都 来 奉 承 老 孙 , 大 盆 小 碗 的 槟 子 酒 、 葡 萄 酒 、 仙 花 、 仙 果 , 真 是 一 家 欢 乐 。
唉 , 贯 通 一 姓 身 归 本 , 只 须 荣 迁 仙 名 。
但 我 不 知 道 什 么 生 成 结 果 , 住 在 这 个 世 界 的 终 始 如 何 , 暂 且 听 我 下 回 的 分 解 。
}\switchcolumn\flushpage  \begin{pinyinscope}{\myfontt \section{第三回}     四海千山皆拱伏 九幽十類盡除名

卻說美猴王榮歸故里,自剿了混世魔王,奪了一口大刀。逐日操演武藝,教小猴
砍竹為標,削木為刀,治旗幡,打哨子,一進一退,安營下寨,頑耍多時。忽然
靜坐處,思想道:「我等在此,恐作耍成真,或驚動人王,或有禽王、獸王認此
犯頭,說我們操兵造反,興師來相殺,汝等都是竹竿木刀,如何對敵?須得鋒利
劍戟方可。如今奈何?」眾猴聞說,個個驚恐道:「大王所見甚長,只是無處可
取。」正說間,轉上四個老猴,兩個是赤尻馬猴,兩個是通背猿猴,走在面前道
:「大王,若要治鋒利器械,甚是容易。」悟空道:「怎見容易?」四猴道:
「我們這山向東去,有二百里水面,那廂乃傲來國界。那國界中有一王位,滿城
中軍民無數,必有金銀銅鐵等匠作。大王若去那裏,或買或造些兵器,教演我等
,守護山場,誠所謂保泰長久之機也。」悟空聞說,滿心歡喜道:「汝等在此頑
耍,待我去來。」

好猴王,急縱觔斗雲,霎時間過了二百里水面。果見那廂有座城池,六街三市,
萬戶千門,來來往往,人都在光天化日之下。悟空心中想道:「這裏定有現成的
兵器,我待下去買他幾件,還不如使個神通覓他幾件倒好。」他就捻起訣來,念
動咒語,向巽地上吸一口氣,呼的吹將去,便是一陣風,飛沙走石,好驚人也:
    炮雲起處蕩乾坤,黑霧陰霾大地昏。
    江海波翻魚蟹怕,山林樹折虎狼奔。
    諸般買賣無商旅,各樣生涯不見人。
    殿上君王歸內院,階前文武轉衙門。
    千秋寶座都吹倒,五鳳高樓幌動根。

風起處,驚散了那傲來國君王,三街六巿,都慌得關門閉戶,無人敢走。悟空才
按下雲頭,徑闖入朝門裏,直尋到兵器館、武庫中,打開門扇看時,那裏面無數
器械:刀、槍、劍、戟、斧、鉞、毛、鐮、鞭、鈀、撾、簡、弓、弩、叉、矛,
件件俱備。一見甚喜道:「我一人能拿幾何?還使個分身法搬將去罷。」好猴王
,即拔一把毫毛,入口嚼爛,噴將出去,念動咒語,叫聲:「變!」變做千百個
小猴,都亂搬亂搶,有力的拿五七件,力小的拿三二件,盡數搬個罄淨。徑踏雲
頭,弄個攝法,喚轉狂風,帶領小猴,俱回本處。

卻說那花果山大小猴兒,正在那洞門外頑耍,忽聽得風聲響處,見半空中丫丫叉
叉,無邊無岸的猴精,諕得都亂跑亂躲。少時,美猴王按落雲頭,收了雲霧,將
身一抖,收了毫毛,將兵器都亂堆在山前,叫道:「小的們,都來領兵器。」眾
猴看時,只見悟空獨立在平陽之地,俱跑來叩頭問故。悟空將前使狂風、搬兵器
,一應事說了一遍。眾猴稱謝畢,都去搶刀奪劍,撾斧爭槍,扯弓扳弩,吆吆喝
喝,耍了一日。

次日,依舊排營。悟空會聚群猴,計有四萬七千餘口。早驚動滿山怪獸,都是些
狼、蟲、虎、豹、?、麂、獐、、狐、狸、獾、?、獅、象、狻猊、猩猩、熊、
鹿、野豕、山牛、羚羊、青兕、狡兔、神獒……各樣妖王,共有七十二洞,都來
參拜猴王為尊。每年獻貢,四時點卯。也有隨班操演的,也有隨節徵糧的。齊齊
整整,把一座花果山造得似鐵桶金城。各路妖王,又有進金鼓、進彩旗、進盔甲
的,紛紛攘攘,日逐家習舞興師。

美猴王正喜間,忽對眾說道:「汝等弓弩熟諳,兵器精通,奈我這口刀著實榔?
,不遂我意,奈何?」四老猴上前啟奏道:「大王乃是仙聖,凡兵是不堪用。但
不知大王水裏可能去得?」悟空道:「我自聞道之後,有七十二般地煞變化之功
,觔斗雲有莫大的神通﹔善能隱身遯身,起法攝法。上天有路,入地有門﹔步日
月無影,入金石無礙﹔水不能溺,火不能焚。那些兒去不得?」四猴道:「大王
既有此神通,我們這鐵板橋下,水通東海龍宮。大王若肯下去,尋著老龍王,問
他要件甚麼兵器,卻不趁心?」悟空聞言,甚喜道:「等我去來。」

好猴王,跳至橋頭,使一個閉水法,捻著訣,撲的鑽入波中,分開水路,徑入東
洋海底。正行間,忽見一個巡海的夜叉,擋住問道:「那推水來的,是何神聖?
說個明白,好通報迎接。」悟空道:「吾乃花果山天生聖人孫悟空,是你老龍王
的緊鄰,為何不識?」那夜叉聽說,急轉水晶宮傳報道:「大王,外面有個花果
山天生聖人孫悟空,口稱是大王緊鄰,將到宮也。」東海龍王敖廣即忙起身,與
龍子、龍孫、蝦兵、蟹將出宮迎道:「上仙請進,請進。」直至宮裏相見,上坐
獻茶畢,問道:「上仙幾時得道?授何仙術?」悟空道:「我自生身之後,出家
修行,得一個無生無滅之體。近因教演兒孫,守護山洞,奈何沒件兵器。久聞賢
鄰享樂瑤宮貝闕,必有多餘神器,特來告求一件。」龍王見說,不好推辭,即著
鱖都司取出一把大桿刀奉上。悟空道:「老孫不會使刀,乞另賜一件。」龍王又
著?太尉領鱔力士,抬出一桿九股叉來。悟空跳下來,接在手中,使了一路,放
下道:「輕,輕,輕,又不趁手。再乞另賜一件。」龍王笑道:「上仙,你不看
看,這叉有三千六百斤重哩。」悟空道:「不趁手,不趁手。」龍王心中恐懼,
又著鯁提督、鯉總兵抬出一柄畫桿方天戟。那戟有七千二百斤重。悟空見了,跑
近前,接在手中,丟幾個架子,撒兩個解數,插在中間道:「也還輕,輕,輕。」
老龍王一發害怕道:「上仙,我宮中只有這根戟重,再沒甚麼兵器了。」悟空笑
道:「古人云:『愁海龍王沒寶』哩!你再去尋尋看,若有可意的,一一奉價。」
龍王道:「委的再無。」

正說處,後面閃過龍婆、龍女道:「大王,觀看此聖,決非小可。我們這海藏中
,那一塊天河定底的神珍鐵,這幾日霞光艷艷,瑞氣騰騰,敢莫是該出現,遇此
聖也?」龍王道:「那是大禹治水之時,定江海淺深的一個定子,是一塊神鐵,
能中何用?」龍婆道:「莫管他用不用,且送與他,憑他怎麼改造,送出宮門便
了。」老龍王依言,盡向悟空說了。悟空道:「拿出來我看。」龍王搖手道:
「扛不動,抬不動,須上仙親去看看。」悟空道:「在何處?你引我去。」

龍王果引導至海藏中間,忽見金光萬道。龍王指定道:「那放光的便是。」悟空
撩衣上前,摸了一把,乃是一根鐵柱子,約有斗來粗,二丈有餘長。他儘力兩手
撾過道:「忒粗忒長些,再短細些方可用。」說畢,那寶貝就短了幾尺,細了一
圍。悟空又顛一顛道:「再細些更好。」那寶貝真個又細了幾分。悟空十分歡喜
,拿出海藏看時,原來兩頭是兩個金箍,中間乃一段烏鐵。緊挨箍有鐫成的一行
字,喚做:「如意金箍棒,重一萬三千五百斤。」心中暗喜道:「想必這寶貝如
人意。」一邊走,一邊心思口念,手顛著道:「再短細些更妙。」拿出外面,只
有二丈長短,碗口粗細。

你看他弄神通,丟開解數,打轉水晶宮裏。諕得老龍王膽戰心驚,小龍子魂飛魄
散,龜鱉黿鼉皆縮頸,魚蝦鰲蟹盡藏頭。悟空將寶貝執在手中,坐在水晶宮殿上
,對龍王笑道:「多謝賢鄰厚意。」龍王道:「不敢,不敢。」悟空道:「這塊
鐵雖然好用,還有一說。」龍王道:「上仙還有甚說?」悟空道:「當時若無此
鐵,倒也罷了﹔如今手中既拿著他,身上更無衣服相趁,奈何?你這裏若有披掛
,索性送我一副,一總奉謝。」龍王道:「這個卻是沒有。」悟空道:「一客不
煩二主。若沒有,我也定不出此門。」龍王道:「煩上仙再轉一海,或者有之。」
悟空又道:「走三家不如坐一家。千萬告求一件。」龍王道:「委的沒有,如有
即當奉承。」悟空道:「真個沒有?就和你試試此鐵!」龍王慌了道:「上仙,
切莫動手,切莫動手,待我看舍弟處可有,當送一副。」悟空道:「令弟何在?」
龍王道:「舍弟乃南海龍王敖欽、北海龍王敖順、西海龍王敖閏是也。」悟空道
:「我老孫不去,不去。俗語謂『賒三不敵見二』,只望你隨高就低的送一副便
了。」老龍道:「不須上仙去。我這裏有一面鐵鼓、一口金鐘,凡有緊急事,擂
得鼓響,撞得鐘鳴,舍弟們就頃刻而至。」悟空道:「既是如此,快些去擂鼓撞
鐘。」真個那鼉將便去撞鐘,鱉帥即來擂鼓。

少時,鐘鼓響處,果然驚動那三海龍王,須臾來到,一齊在外面會著。敖欽道:
「大哥,有甚緊事,擂鼓撞鐘?」老龍道:「賢弟,不好說。有一個花果山甚麼
天生聖人,早間來認我做鄰居。後要求一件兵器,獻鋼叉嫌小,奉畫戟嫌輕﹔將
一塊天河定底神珍鐵,自己拿出手,丟了些解數。如今坐在宮中,又要索甚麼披
掛。我處無有,故響鐘鳴鼓,請賢弟來。你們可有甚麼披掛,送他一副,打發出
門去罷了。」敖欽聞言,大怒道:「我兄弟們點起兵拿他不是?」老龍道:「莫
說拿,莫說拿。那塊鐵,挽著些兒就死,磕著些兒就亡﹔挨挨兒皮破,擦擦兒觔
傷。」西海龍王敖閏說:「二哥不可與他動手。且只湊副披掛與他,打發他出了
門,啟表奏上上天,天自誅也。」北海龍王敖順道:「說的是。我這裏有一雙藕
絲步雲履哩。」西海龍王敖閏道:「我帶了一副鎖子黃金甲哩。」南海龍王敖欽
道:「我有一頂鳳翅紫金冠哩。」老龍大喜,引入水晶宮相見了,以此奉上。悟
空將金冠、金甲、雲履都穿戴停當,使動如意棒,一路打出去,對眾龍道:「聒
噪,聒噪。」四海龍王甚是不平,一邊商議進表上奏不題。

你看這猴王,分開水道,徑回鐵板橋頭,攛將上去。只見四個老猴領著眾猴,都
在橋邊等候。忽然見悟空跳出波外,身上更無一點水濕,金燦燦的走上橋來。諕
得眾猴一齊跪下道:「大王好華彩耶!好華彩耶!」悟空滿面春風,高登寶座,
將鐵棒豎在當中。那些猴不知好歹,都來拿那寶貝,卻便似蜻蜓撼鐵樹,分毫也
不能禁動。一個個咬指伸舌道:「爺爺呀!這般重,虧你怎的拿來也!」悟空近
前,舒開手,一把撾起,對眾笑道:「物各有主。這寶貝鎮於海藏中,也不知幾
千百年,可可的今歲放光。龍王只認做是塊黑鐵,又喚做天河鎮底神珍。那廝每
都扛抬不動,請我親去拿之。那時此寶有二丈多長,斗來粗細。被我撾他一把,
意思嫌大,他就小了許多﹔再教小些,他又小了許多﹔再教小些,他又小了許多
。急對天光看處,上有一行字,乃『如意金箍棒,一萬三千五百斤』。你都站開
,等我再叫他變一變著。」他將那寶貝顛在手中,叫:「小!小!小!」即時就
小做一個繡花針兒相似,可以揌在耳朵裏面藏下。眾猴駭然,叫道:「大王,還
拿出來耍耍。」猴王真個去耳朵裏拿出,托放掌上叫:「大!大!大!」即又大
做斗來粗細,二丈長短。他弄到歡喜處,跳上橋,走出洞外,將寶貝揝在手中,
使一個法天像地的神通,把腰一躬,叫聲:「長!」他就長的高萬丈,頭如泰山
,腰如峻嶺,眼如閃電,口似血盆,牙如劍戟﹔手中那棒,上抵三十三天,下至
十八層地獄。把些虎豹狼蟲、滿山群怪、七十二洞妖王,都諕得磕頭禮拜,戰兢
兢魄散魂飛。霎時收了法像,將寶貝還變做個繡花針兒,藏在耳內,復歸洞府。
慌得那各洞妖王,都來參賀。

此時遂大開旗鼓,響振銅鑼,廣設珍饈百味,滿斟椰液萄漿,與眾飲宴多時,卻
又依前教演。猴王將那四個老猴封為健將,將兩個赤尻馬猴喚做馬、流二元帥,
兩個通背猿猴喚做崩、芭二將軍。將那安營下寨、賞罰諸事,都付與四健將維持
。

他放下心,日逐騰雲駕霧,遨遊四海,行樂千山。施武藝,遍訪英豪﹔弄神通,
廣交賢友。此時又會了個七弟兄,乃牛魔王、蛟魔王、鵬魔王、獅狔王、獼猴王
、狨王,連自家美猴王七個。日逐講文論武,走斝傳觴,絃歌吹舞,朝去暮回,
無般兒不樂。把那萬里之遙,只當庭闈之路﹔所謂點頭徑過三千里,扭腰八百有
餘程。

一日,在本洞吩咐四健將安排筵宴,請六王赴飲,殺牛宰馬,祭天享地,著眾怪
跳舞歡歌,俱吃得酩酊大醉。送六王出去,卻又賞大小頭目。攲在鐵板橋邊松陰
之下,霎時間睡著。四健將領眾圍護,不敢高聲。只見那美猴王睡裏,見兩人拿
一張批文,上有「孫悟空」三字,走近身,不容分說,套上繩,就把美猴王的魂
靈兒索了去,踉踉蹌蹌,直帶到一座城邊。猴王漸覺酒醒,忽抬頭觀看,那城上
有一鐵牌,牌上有三個大字,乃「幽冥界」。美猴王頓然醒悟道:「幽冥界乃閻
王所居,何為到此?」那兩人道:「你今陽壽該終,我兩人領批,勾你來也。」
猴王聽說,道:「我老孫超出三界之外,不在五行之中,已不伏他管轄,怎麼朦
朧,又敢來勾我?」那兩個勾死人,只管扯扯拉拉,定要拖他進去。這猴王惱起
性來,耳朵中掣出寶貝,幌一幌,碗來粗細。略舉手,把兩個勾死人打為肉醬。
自解其索,丟開手,掄著棒,打入城中。諕得那牛頭鬼東躲西藏,馬面鬼南奔北
跑。眾鬼卒奔上森羅殿,報著:「大王,禍事!禍事!外面有一個毛臉雷公打將
來了。」

慌得那十代冥王急整衣來看,見他相貌兇惡,即排下班次,應聲高叫道:「上仙
留名!上仙留名!」猴王道:「你既認不得我,怎麼差人來勾我?」十王道:
「不敢,不敢。想是差人差了。」猴王道:「我本是花果山水簾洞天生聖人孫悟
空。你等是甚麼官位?」十王躬身道:「我等是陰間天子十代冥王。」悟空道:
「快報名來,免打。」十王道:「我等是秦廣王、初江王、宋帝王、忤官王、閻
羅王、平等王、泰山王、都市王、卞城王、轉輪王。」悟空道:「汝等既登王位
,乃靈顯感應之類,為何不知好歹?我老孫修了仙道,與天齊壽,超昇三界之外
,跳出五行之中,為何著人拘我?」十王道:「上仙息怒。普天下同名同姓者多
,敢是那勾死人錯走了也?」悟空道:「胡說!胡說!常言道:『官差吏差,來
人不差。』你快取生死簿子來我看!」十王聞言,即請上殿查看。

悟空執著如意棒,徑登森羅殿上,正中間南面坐下。十王即命掌案的判官取出文
簿來查。那判官不敢怠慢,便到司房裏捧出五六簿文書並十類簿子。逐一查看:
臝蟲、毛蟲、羽蟲、昆蟲、鱗介之屬,俱無他名。又看到猴屬之類,原來這猴似
人相,不入人名﹔似臝蟲,不居國界﹔似走獸,不伏麒麟管﹔似飛禽,不受鳳凰
轄。另有個簿子,悟空親自檢閱,直到那「魂」字一千三百五十號上,方注著孫
悟空名字,乃「天產石猴,該壽三百四十二歲,善終」。悟空道:「我也不記壽
數幾何,且只消了名字便罷。取筆過來。」那判官慌忙捧筆,飽掭濃墨。悟空拿
過簿子,把猴屬之類,但有名者,一概勾之。捽下簿子道:「了帳,了帳,今番
不伏你管了。」一路棒,打出幽冥界。那十王不敢相近,都去翠雲宮,同拜地藏
王菩薩,商量啟表,奏聞上天,不在話下。

這猴王打出城中,忽然絆著一個草紇繨,跌了個躘踵,猛的醒來,乃是南柯一夢
。才覺伸腰,只聞得四健將與眾猴高叫道:「大王,吃了多少酒,睡這一夜,還
不醒來?」悟空道:「睡還小可,我夢見兩個人來此勾我,把我帶到幽冥界城門
之外,卻才醒悟。是我顯神通,直嚷到森羅殿,與那十王爭吵,將我們的生死簿
子看了,但有我等名號,俱是我勾了,都不伏那廝所轄也。」眾猴磕頭禮謝。自
此,山猴多有不老者,以陰司無名故也。

美猴王言畢前事,四健將報知各洞妖王,都來賀喜。不幾日,六個義兄弟又來拜
賀,一聞銷名之故,又個個歡喜,每日聚樂不題。

卻表啟那個高天上聖大慈仁者玉皇大天尊玄穹高上帝,一日駕坐金闕雲宮靈霄寶
殿,聚集文武仙卿早朝之際,忽有丘弘濟真人啟奏道:「萬歲,通明殿外有東海
龍王敖廣進表,聽天尊宣詔。」玉皇傳旨:「著宣來。」敖廣宣至靈霄殿下,禮
拜畢,傍有引奏仙童接上表文。玉皇從頭看過。表曰:
「水元下界東勝神洲東海小龍臣敖廣啟奏大天聖主玄穹高上帝君:近因花果山生
、水簾洞住妖仙孫悟空者,欺虐小龍,強坐水宅,索兵器,施法施威﹔要披掛,
騁兇騁勢。驚傷水族,諕走龜鼉。南海龍戰戰兢兢,西海龍悽悽慘慘,北海龍縮
首歸降。臣敖廣舒身下拜,獻神珍之鐵棒,鳳翅之金冠,與那鎖子甲、步雲履,
以禮送出。他仍弄武藝,顯神通,但云:『聒噪!聒噪!』果然無敵,甚為難制
。臣今啟奏,伏望聖裁。懇乞天兵,收此妖孽,庶使海嶽清寧,下元安泰。奉奏
。」

聖帝覽畢,傳旨:「著龍神回海,朕即遣將擒拿。」老龍王頓首謝去。

下面又有葛仙翁天師啟奏道:「萬歲,有冥司秦廣王?奉幽冥教主地藏王菩薩表
文進上。」傍有傳言玉女接上表文。玉皇亦從頭看過。表曰:
「幽冥境界,乃地之陰司。天有神而地有鬼,陰陽輪轉﹔禽有生而獸有死,反復
雌雄。生生化化,孕女成男,此自然之數,不能易也。今有花果山水簾洞天產妖
猴孫悟空,逞惡行兇,不服拘喚。弄神通,打絕九幽鬼使﹔恃勢力,驚傷十代慈
王。大鬧森羅,強銷名號。致使猴屬之類無拘,獼猴之畜多壽﹔寂滅輪迴,各無
生死。貧僧具表,冒瀆天威。伏乞調遣神兵,收降此妖,整理陰陽,永安地府。
謹奏。」

玉皇覽畢,傳旨:「著冥君回歸地府,朕即遣將擒拿。」秦廣王亦頓首謝去。

大天尊宣眾文武仙卿,問曰:「這妖猴是幾年產育,何代出身,卻就這般有道
?」一言未已,班中閃出千里眼、順風耳道:「這猴乃三百年前天產石猴。當
時不以為然,不知這幾年在何方修煉成仙,降龍伏虎,強銷死籍也。」玉帝道
:「那路神將下界收伏?」言未已,班中閃出太白長庚星,俯伏啟奏道:「上
聖,三界中凡有九竅者,皆可修仙。奈此猴乃天地育成之體,日月孕就之身,
他也頂天履地,服露餐霞,今既修成仙道,有降龍伏虎之能,與人何以異哉?
臣啟陛下,可念生化之慈恩,降一道招安聖旨,把他宣來上界,授他一個大小
官職,與他籍名在籙,拘束此間。若受天命,後再陞賞﹔若違天命,就此擒拿
。一則不動眾勞師,二則收仙有道也。」玉帝聞言甚喜,道:「依卿所奏。」
即著文曲星官修詔,著太白金星招安。

金星領了旨,出南天門外,按下祥雲,直至花果山水簾洞,對眾小猴道:「我
乃天差天使,有聖旨在此,請你大王上界。快快報知。」洞外小猴一層層傳至
洞天深處,道:「大王,外面有一老人,背著一角文書,言是上天差來的天使
,有聖旨請你也。」美猴王聽得大喜,道:「我這兩日正思量要上天走走,卻
就有天使來請。」叫:「快請進來。」猴王急整衣冠,門外迎接。金星徑入當
中,面南立定道:「我是西方太白金星,奉玉帝招安聖旨,下界請你上天,拜
受仙籙。」悟空笑道:「多感老星降臨。」教小的們安排筵宴款待。金星道:
「聖旨在身,不敢久留。就請大王同往,待榮遷之後,再從容敘也。」悟空道
:「承光顧,空退,空退。」即喚四健將,吩咐:「謹慎教演兒孫,待我上天
去看看路,卻好帶你們上去同居住也。」四健將領諾。

這猴王與金星縱起雲頭,昇在空霄之上。正是那:
    高遷上品天仙位,名列雲班寶籙中。
    畢竟不知授個甚麼官爵,且聽下回分解。





}  \end{pinyinscope}\switchcolumn{\myfontc \section{第 三 回} , 四 海 千 山 皆 拱 伏 , 九 幽 十 类 尽 除 其 名 , 又 说 美 猴 王 荣 回 故 乡 , 自 己 剿 灭 混 世 魔 王 , 夺 了 一 口 大 刀 。
每 天 操 练 武 艺 , 教 小 猴 砍 竹 子 做 标 志 , 削 木 头 刀 , 制 旗 幡 , 打 哨 子 , 一 进 一 退 , 安 营 下 寨 , 顽 强 戏 耍 多 时 。
忽 然 静 坐 在 那 里 , 想 着 说 : 我 们 在 这 里 , 恐 怕 作 戏 弄 成 真 , 或 者 惊 动 人 王 , 或 者 有 擒 王 、 兽 王 , 认 为 这 样 的 犯 人 , 说 我 们 操 兵 造 反 , 兴 师 来 杀 我 们 , 你 们 都 是 竹 竿 木 刀 , 怎 么 能 对 付 敌 人 , 必 须 有 锋 利 剑 戟 才 可 以 。
众 猴 听 了 这 话 , 一 个 个 都 惊 恐 地 说 : 大 王 所 见 到 的 东 西 很 长 , 只 是 没 有 可 取 的 东 西 。
正 在 说 完 , 转 上 四 个 老 猴 , 两 个 是 赤 尻 马 猴 , 两 个 是 通 背 猿 猴 , 走 在 面 前 说 : 大 王 , 如 果 要 制 造 锋 利 器 械 , 很 是 容 易 。
悟 空 说 : 怎 么 见 得 上 容 易 呢 四 猴 说 : 我 们 这 座 山 向 东 去 , 有 二 百 里 的 水 面 , 那 里 是 傲 来 国 界 。
那 国 界 内 有 一 个 王 位 , 城 中 的 军 民 无 数 , 必 定 有 金 银 铜 铁 等 工 匠 制 作 。
大 王 如 果 去 那 里 , 或 买 或 造 兵 器 , 教 我 等 人 守 卫 山 场 , 确 实 是 所 谓 保 泰 长 久 之 机 。
悟 空 听 了 这 话 , 满 心 欢 喜 地 说 : 你 们 在 这 里 呆 傻 , 等 我 去 来 。
猴 王 急 忙 放 出 斗 云 , 一 会 儿 就 走 了 二 百 里 水 面 。
果 然 看 见 那 里 有 座 城 池 , 六 街 三 市 , 万 户 千 门 , 来 来 往 往 , 人 们 都 在 光 天 化 日 之 下 。
悟 空 心 里 想 着 说 : 这 里 一 定 有 现 在 的 兵 器 , 我 等 下 去 买 它 几 件 , 还 不 如 让 他 个 神 通 去 找 它 几 件 倒 好 。
其 于 是 , 不 知 其 是 , 不 知 其 是 , 不 知 其 是 , 不 知 其 是 也 。
江 海 波 涛 汹 涌 , 鱼 蟹 害 怕 , 山 林 树 木 折 断 , 虎 狼 奔 逃 。
各 种 买 卖 没 有 商 人 旅 客 , 各 种 生 涯 都 不 见 人 。
殿 上 君 王 回 内 院 , 阶 前 文 武 官 员 转 衙 门 。
千 秋 宝 座 都 吹 倒 了 , 五 凤 高 楼 飘 动 了 树 根 。
风 吹 起 来 的 地 方 , 惊 散 了 那 傲 来 国 的 君 王 , 三 街 六 市 都 惊 慌 地 关 上 门 , 闭 上 门 , 没 有 人 敢 逃 走 。
悟 空 刚 按 下 云 头 , 径 直 闯 入 朝 门 , 径 直 寻 到 兵 器 馆 、 武 库 中 , 打 开 门 扇 看 时 , 里 面 没 有 多 少 器 械 , 刀 、 刀 、 剑 、 戟 、 斧 、 、 毛 、 鞭 、 钢 、 钢 、 、 简 、 弓 、 弩 、 叉 、 矛 等 等 都 齐 备 。
一 见 到 他 非 常 高 兴 地 说 : 我 一 个 人 能 拿 得 多 少 , 还 让 我 一 个 分 身 法 把 你 搬 走 吧 。
猴 王 立 即 拔 出 一 把 毫 毛 , 进 入 嘴 里 嚼 烂 , 喷 出 去 , 念 咒 语 , 大 叫 : 变 变 成 千 百 个 小 猴 子 , 都 乱 拿 乱 抢 , 有 力 的 拿 五 七 件 , 力 小 的 拿 三 二 件 , 全 都 数 次 都 拿 完 了 。
径 踏 云 头 , 弄 个 摄 法 , 呼 唤 转 狂 风 , 带 着 小 猴 , 都 回 到 原 来 的 地 方 。
又 说 那 花 果 山 的 大 小 猴 子 , 正 在 那 洞 门 外 狂 耍 , 忽 然 听 到 风 声 响 动 的 地 方 , 看 见 半 空 中 的 丫 头 叉 叉 , 无 边 无 岸 的 猴 子 , 都 吓 得 都 乱 走 乱 躲 。
小 时 候 , 美 猴 王 按 落 在 云 头 , 收 起 了 云 雾 , 将 身 体 抖 抖 起 来 , 收 起 了 毫 毛 , 把 兵 器 都 乱 堆 在 山 前 , 喊 道 : 小 的 人 , 都 来 领 兵 器 。
众 猴 看 时 , 只 见 悟 空 独 自 站 在 平 阳 的 地 方 , 都 跑 来 叩 头 问 原 因 。
悟 空 将 先 前 使 用 狂 风 、 搬 动 兵 器 , 一 应 的 事 都 说 了 一 遍 。
众 猴 称 谢 完 毕 , 都 去 夺 刀 夺 剑 , 拔 斧 争 枪 , 扯 弓 扳 弩 , 吵 吵 吵 吵 吵 吵 吵 吵 吵 吵 吵 吵 吵 吵 吵 吵 吵 吵 吵 吵 吵 吵 吵 吵 吵 吵 吵 吵 吵 吵 吵 吵 吵 吵 吵 吵 吵 吵 吵 吵 吵 吵 吵 吵 吵 吵 吵 吵 吵 吵 吵 吵 吵 吵 吵 吵 吵 吵 吵 吵 吵 吵 吵 吵 吵 吵 吵 吵 吵 吵 吵 吵 吵 吵 吵 吵 吵 吵 吵 吵 吵 吵 吵 吵 吵 吵 吵 吵 吵 吵 吵 吵 吵 吵 吵 吵 吵 吵 吵 吵 吵 吵 吵 吵 吵 吵 吵 吵 吵 吵 吵 吵 吵 吵 吵 吵 吵 吵 吵 吵 吵 吵 吵 吵 吵 吵 吵 吵 吵 吵 吵 吵 吵 吵 吵 吵 吵 吵 吵 吵 吵 吵 吵 吵 吵 吵 吵 吵 吵 吵 吵 吵 吵 吵 吵 吵 吵 吵 吵 吵 吵 吵 吵 吵 吵 吵 吵 吵 吵 吵 吵 吵 吵 吵 吵 吵 吵 吵 吵 吵 吵 吵 吵 吵 吵 吵 吵 吵 吵 吵 吵 吵 吵 吵 吵 吵 吵 吵 吵 吵 吵 吵 吵 吵 吵 吵 吵 吵 吵 吵 吵 吵 吵 吵 吵 吵 吵 吵 吵 吵 吵 吵 吵 吵 吵 吵 吵 吵 吵 吵 吵
第 二 天 , 依 旧 排 营 。
悟 空 聚 集 了 一 群 猴 子 , 共 有 四 万 七 千 多 人 。
早 先 惊 动 满 山 的 怪 兽 , 都 是 狼 、 虫 、 虎 、 豹 、 獐 、 獐 、 獐 、 狐 、 狸 、 獐 、 獐 、 獐 、 象 、 、 猩 猩 、 熊 、 鹿 、 野 、 山 牛 、 羚 羊 、 青 、 狡 兔 、 神 狗 、 各 种 妖 王 , 共 有 七 十 二 洞 , 都 来 参 拜 猴 王 为 尊 。
每 年 献 贡 , 四 季 点 卯 。
有 的 随 班 操 演 , 有 的 随 节 征 收 粮 食 。
整 齐 整 齐 , 把 一 座 花 果 山 造 得 好 像 铁 桶 金 城 。
各 路 妖 王 , 又 有 进 金 鼓 、 进 彩 旗 、 进 铠 甲 的 , 纷 纷 攘 攘 , 每 天 都 要 逐 家 练 兵 。
美 猴 王 正 高 兴 的 时 候 , 忽 然 对 众 人 说 道 : 你 们 弓 弩 熟 悉 , 兵 器 精 通 , 可 是 我 这 把 刀 都 是 榔 子 , 不 顺 从 我 的 心 意 , 怎 么 办 呢 四 个 老 猴 上 前 启 奏 说 : 大 王 你 们 是 仙 圣 , 凡 兵 是 不 能 使 用 的 。
但 不 知 大 王 在 水 里 可 以 去 吗 ? 悟 空 说 : 我 自 从 听 了 佛 教 之 后 , 有 七 十 二 般 地 狱 变 化 之 功 , 斗 云 有 最 大 的 神 通 , 善 于 隐 身 身 , 起 法 摄 法 。
上 天 有 路 , 进 入 地 有 门 , 步 行 日 月 没 有 影 子 , 进 入 金 石 无 阻 碍 , 水 不 能 淹 没 , 火 不 能 烧 毁 。
四 猴 说 : 大 王 既 然 有 这 样 的 神 通 , 我 们 这 个 铁 板 桥 下 , 水 通 东 海 龙 宫 。
大 王 如 果 肯 下 去 , 寻 找 老 龙 王 , 问 他 要 什 么 兵 器 , 却 不 趁 着 我 , 悟 空 听 了 这 话 , 非 常 高 兴 地 说 : 等 我 去 来 。
猴 王 , 跳 到 桥 头 , 使 用 一 个 闭 水 法 , 捻 着 法 术 , 扑 进 水 波 中 , 分 开 水 路 , 径 直 进 入 东 洋 海 底 。
正 在 行 走 的 时 候 , 忽 然 看 见 一 个 巡 海 的 夜 叉 , 挡 住 他 问 道 : 那 推 水 来 的 , 是 什 么 神 圣 ?
悟 空 说 : 我 是 花 果 山 天 生 圣 人 孙 悟 空 , 是 你 老 龙 王 的 紧 邻 , 为 什 么 不 认 识 那 夜 叉 听 了 , 急 忙 转 到 水 晶 宫 传 报 说 : 大 王 , 外 面 有 个 花 果 山 天 生 圣 人 孙 悟 空 , 口 称 是 大 王 紧 邻 , 将 要 到 宫 中 去 。
东 海 龙 王 敖 广 立 即 起 身 , 和 龙 子 、 龙 孙 、 虾 兵 、 蟹 将 出 宫 迎 接 道 : 上 仙 请 进 , 请 进 。
一 直 到 宫 里 相 见 , 上 座 献 茶 后 , 问 道 : 上 仙 什 么 时 候 得 道 , 传 授 什 么 仙 术 ? 悟 空 说 : 我 自 己 生 身 之 后 , 出 家 修 行 , 得 到 了 一 个 无 生 无 灭 的 道 体 。
近 来 就 教 演 儿 孙 , 守 护 山 洞 , 为 什 么 没 有 兵 器 ?
久 闻 贤 邻 享 乐 瑶 宫 贝 阙 , 必 定 有 多 余 的 神 器 , 特 来 告 诉 我 一 件 。
龙 王 见 了 , 不 好 推 辞 , 立 即 拿 出 都 司 取 出 一 把 大 棍 刀 送 上 去 。
悟 空 说 : 老 子 不 会 使 刀 , 请 另 外 赐 给 我 一 把 。
这 时 龙 王 又 拿 起 来 , 太 尉 领 着 虾 力 士 , 抬 出 一 根 九 股 叉 来 。
悟 空 跳 下 来 , 接 在 手 中 , 让 他 走 了 一 路 , 然 后 放 下 去 说 : 轻 , 轻 , 轻 , 又 不 趁 手 。
再 次 请 求 另 赐 给 他 一 件 。
龙 王 笑 着 说 : 上 仙 , 你 不 看 , 这 个 叉 有 三 千 六 百 斤 重 吗 ?
悟 空 说 : 不 趁 手 , 不 趁 手 。
龙 王 心 中 恐 惧 , 又 拿 着 提 督 、 鲤 总 兵 , 抬 出 一 柄 画 杆 方 天 戟 。
那 戟 有 七 千 二 百 斤 重 。
悟 空 见 了 , 便 走 到 前 面 , 接 在 手 中 , 放 了 几 个 架 子 , 撒 了 两 个 解 数 , 插 在 中 间 说 : 也 还 轻 , 轻 , 轻 , 轻 。
老 龙 王 一 发 就 害 怕 道 : 上 仙 , 我 宫 中 只 有 这 根 戟 重 , 再 没 有 什 么 兵 器 了 。
悟 空 笑 着 说 : 古 人 说 : 愁 海 龙 王 没 有 宝 贝 呀 你 再 去 寻 寻 寻 找 , 如 果 有 合 乎 意 的 , 一 一 给 予 价 钱 。
龙 王 说 : 我 再 没 有 了 。
正 说 的 时 候 , 后 面 闪 过 龙 婆 、 龙 女 说 : 大 王 , 观 看 这 位 圣 人 , 决 不 是 小 可 以 的 。
我 们 这 个 海 藏 中 , 那 一 块 天 河 定 底 的 神 珍 铁 , 这 几 天 霞 光 艳 , 瑞 气 腾 腾 , 岂 敢 说 是 该 出 现 , 遇 到 这 个 圣 人 呢 ? 龙 王 说 : 那 是 大 禹 治 水 的 时 候 , 定 江 海 浅 深 的 一 个 定 子 , 是 一 块 神 铁 , 能 够 中 用 什 么 ? 龙 婆 说 : 不 管 它 用 不 用 , 暂 且 送 给 他 , 凭 他 怎 么 改 造 , 送 出 宫 门 就 可 以 了 。
老 龙 王 依 照 他 的 话 , 完 全 向 悟 空 说 了 。
悟 空 说 : 拿 出 来 我 看 。
龙 王 摇 着 手 说 : 扛 不 动 , 抬 不 动 , 必 须 上 仙 亲 自 去 看 看 。
悟 空 说 : 你 在 什 么 地 方 , 你 带 我 去 。
龙 王 果 然 引 导 他 到 海 藏 中 间 , 忽 然 看 见 金 光 万 道 。
龙 王 指 定 说 : 那 放 光 的 就 是 。
悟 空 撩 衣 服 上 前 , 摸 了 一 把 , 原 来 是 一 根 铁 柱 子 , 大 约 有 斗 来 粗 , 二 丈 多 长 。
他 尽 力 , 两 手 拍 过 去 说 : 太 粗 太 长 了 , 再 短 些 才 可 以 使 用 。
说 完 , 那 个 宝 贝 就 短 了 几 尺 , 细 了 一 围 。
悟 空 又 颠 倒 了 一 颠 , 说 : 再 细 些 更 好 。
那 个 宝 贝 真 的 , 又 细 了 几 分 。
悟 空 十 分 欢 喜 , 拿 出 海 藏 去 看 时 , 原 来 两 头 是 两 个 金 钳 , 中 间 是 一 段 黑 铁 。
紧 紧 紧 紧 紧 紧 的 锁 有 一 行 字 , 叫 做 : 如 意 金 锁 棒 , 重 一 万 三 千 五 百 斤 。
他 心 中 暗 自 高 兴 地 说 : 想 必 是 这 个 宝 贝 如 人 意 。
一 边 走 , 一 边 心 思 口 念 , 手 翻 着 说 : 再 短 些 更 妙 。
拿 出 外 面 , 只 有 两 丈 长 , 碗 口 粗 细 。
你 看 他 玩 弄 神 通 , 抛 开 解 数 , 打 转 水 晶 宫 里 。
得 到 老 龙 王 , 胆 战 心 惊 , 小 龙 子 魂 飞 魄 散 , 龟 鳖 都 缩 颈 , 鱼 虾 鳌 蟹 都 藏 在 头 上 。
悟 空 把 宝 贝 拿 在 手 中 , 坐 在 水 晶 宫 殿 上 , 对 龙 王 笑 着 说 : 多 谢 贤 邻 的 厚 意 。
龙 王 说 : 不 敢 , 不 敢 。
悟 空 说 : 这 块 铁 虽 然 好 用 , 还 有 一 种 说 法 。
龙 王 说 : 上 仙 还 有 什 么 说 法 呢 ? 悟 空 说 : 当 时 如 果 没 有 这 铁 , 倒 也 就 算 罢 了 ; 如 今 手 中 拿 着 它 , 身 上 再 没 有 衣 服 可 以 互 相 追 赶 , 怎 么 办 呢 你 这 里 如 果 有 什 么 披 挂 , 我 就 送 我 一 副 , 一 总 送 给 你 。
龙 王 说 : 这 个 最 好 是 没 有 。
悟 空 说 : 一 个 客 人 不 必 烦 劳 两 个 主 人 。
如 果 没 有 , 我 也 一 定 不 会 出 这 个 门 。
龙 王 说 : 烦 上 仙 再 转 一 海 , 或 许 有 这 种 事 。
悟 空 又 说 : 走 三 家 , 不 如 坐 一 家 。
千 万 请 求 一 件 。
龙 王 说 : 我 没 有 , 如 果 有 , 我 就 应 当 奉 承 。
悟 空 说 : 真 的 没 有 , 就 和 你 试 试 此 铁 吧 龙 王 惊 慌 地 说 : 上 仙 , 切 莫 动 手 , 等 我 看 我 弟 弟 的 地 方 可 以 有 , 我 送 一 副 。
悟 空 说 : 你 的 弟 弟 在 哪 里 ? 龙 王 说 : 我 的 弟 弟 就 是 南 海 龙 王 敖 钦 、 北 海 龙 王 敖 顺 、 西 海 龙 王 敖 闰 等 人 。
悟 空 说 : 我 的 老 孙 子 不 去 , 不 去 。
俗 语 说 : 三 不 敌 见 二 , 只 希 望 你 随 高 就 低 的 送 一 副 就 行 了 。
老 龙 说 : 不 必 要 上 仙 去 。
我 这 里 有 一 面 铁 鼓 , 一 口 金 钟 , 凡 是 有 紧 急 事 情 , 敲 得 鼓 声 , 敲 得 钟 声 , 舍 弟 们 就 会 顷 刻 就 到 来 。
悟 空 说 : 既 然 这 样 , 快 点 去 敲 鼓 撞 钟 。
真 的 那 将 去 撞 钟 , 鳖 帅 立 即 来 擂 鼓 。
不 一 会 儿 , 钟 鼓 响 响 的 地 方 , 果 然 惊 动 了 那 三 海 龙 王 。
敖 钦 说 : 大 哥 , 有 什 么 紧 急 事 情 , 敲 鼓 撞 钟 , 老 龙 说 : 贤 弟 , 不 好 说 。
有 一 个 花 果 山 , 是 什 么 天 生 的 圣 人 , 早 些 时 候 来 认 我 做 邻 居 。
后 来 要 求 一 件 兵 器 , 献 出 的 钢 叉 嫌 小 , 奉 奉 的 画 戟 嫌 轻 , 把 一 块 天 河 定 底 神 珍 铁 , 自 己 拿 出 手 , 丢 掉 了 解 数 。
今 天 我 坐 在 宫 中 , 又 要 索 要 什 么 披 挂 ?
我 的 地 方 没 有 , 所 以 敲 钟 鸣 鼓 , 请 您 的 弟 弟 来 。
你 们 可 以 有 什 么 披 挂 , 送 给 他 一 副 , 打 发 出 门 就 行 了 。
高 敖 钦 听 了 这 话 , 大 怒 说 : 我 们 兄 弟 们 点 起 兵 来 抓 他 不 是 ? 老 龙 说 : 不 要 拿 , 不 要 拿 。
那 块 铁 , 拉 着 它 的 儿 子 就 死 了 , 撞 着 它 的 儿 子 就 灭 亡 了 ; 挨 着 儿 子 的 皮 破 了 , 擦 着 儿 子 的 脑 袋 伤 了 。
西 海 龙 王 敖 闰 说 : 二 哥 不 可 与 他 们 动 手 。
况 且 只 是 凑 好 一 个 副 手 给 他 , 打 发 他 出 了 门 , 上 表 奏 上 上 天 , 上 天 自 己 诛 杀 。
北 海 龙 王 敖 顺 说 : 说 的 是 。
我 这 里 有 一 双 藕 丝 步 云 履 哩 。
西 海 龙 王 敖 闰 说 : 我 带 了 一 副 锁 子 黄 金 甲 哩 。
南 海 龙 王 敖 钦 说 : 我 有 一 顶 凤 翅 紫 金 冠 啊 。
老 龙 非 常 高 兴 , 把 他 引 进 水 晶 宫 , 相 见 了 , 就 把 这 个 东 西 奉 给 了 皇 上 。
悟 空 将 金 冠 、 金 甲 、 云 履 都 穿 好 了 , 让 他 动 起 如 意 棒 , 一 路 打 出 去 。
四 海 的 龙 王 非 常 愤 愤 不 平 , 一 边 商 议 , 进 表 上 奏 没 有 题 目 。
你 看 这 个 猴 王 , 分 开 水 道 , 径 直 回 到 铁 板 桥 头 , 快 要 上 去 。
又 见 四 个 老 猴 , 领 着 众 猴 , 都 在 桥 边 等 候 。
忽 然 见 悟 空 跳 出 波 浪 之 外 , 身 上 再 没 有 一 点 水 湿 , 金 光 灿 烂 地 走 上 桥 来 。
李 得 到 众 猴 一 齐 跪 下 说 : 大 王 你 好 华 彩 啊 , 好 华 彩 啊 ! 悟 空 满 脸 春 风 , 高 登 宝 座 , 把 铁 棒 竖 在 当 中 。
那 些 猴 子 不 知 好 坏 , 都 来 拿 那 宝 贝 , 它 就 好 像 蜻 蜓 摇 动 铁 树 一 样 , 一 点 也 不 能 禁 止 动 。
一 个 人 咬 着 手 指 伸 出 舌 头 说 : 爷 爷 啊 , 这 样 重 , 亏 你 怎 么 拿 来 ! 悟 空 走 近 前 , 伸 开 手 , 一 把 起 来 , 对 众 人 笑 道 : 物 各 有 主 。
这 个 宝 贝 镇 在 海 藏 中 , 也 不 知 道 有 多 少 千 百 年 , 可 以 今 年 放 光 。
龙 王 只 认 为 是 一 块 黑 铁 , 又 叫 做 天 河 镇 底 神 珍 。
那 个 奴 仆 每 次 都 抬 着 不 动 , 请 我 亲 自 去 拿 。
当 时 这 个 宝 物 有 两 丈 多 长 , 斗 来 的 粗 细 。
被 我 他 一 把 , 意 思 嫌 大 , 他 就 小 了 许 多 ; 再 教 小 的 , 他 又 小 了 许 多 ; 再 教 小 的 , 他 又 小 了 许 多 。
急 忙 对 着 天 光 观 看 的 地 方 , 上 面 有 一 行 字 , 是 : 如 意 金 钳 棒 , 一 万 三 千 五 百 斤 。
你 都 站 开 , 等 我 再 叫 他 变 一 变 。
他 把 宝 贝 翻 在 手 中 , 喊 道 : 小 , 小 , 小 , 小 , 当 时 就 小 做 了 一 个 绣 花 针 , 它 就 可 以 把 它 搓 在 耳 朵 里 面 藏 下 来 。
众 猴 子 惊 慌 地 喊 道 : 大 王 , 还 拿 出 来 玩 耍 。
猴 王 真 的 , 从 耳 朵 里 拿 出 来 , 托 放 在 手 掌 上 喊 道 : 大 , 大 , 大 , 又 大 作 斗 来 粗 细 , 二 丈 长 短 。
其 所 以 不 得 其 所 以 , 其 所 以 不 得 其 所 以 , 其 所 以 不 得 其 所 以 也 。
把 虎 豹 狼 虫 、 满 山 群 怪 、 七 十 二 洞 妖 王 , 都 可 以 跪 拜 , 战 战 兢 兢 , 魂 散 魂 飞 。
顷 刻 收 回 法 像 , 把 宝 贝 还 变 成 个 绣 花 针 儿 , 藏 在 耳 朵 里 , 又 回 到 洞 府 。
惊 慌 得 到 那 各 洞 的 妖 王 , 都 来 参 拜 祝 贺 。
这 时 , 就 大 开 旗 鼓 , 响 响 铜 鼓 , 广 设 珍 肴 百 味 , 满 斟 荔 液 、 葡 萄 浆 , 与 众 人 饮 宴 多 时 , 又 依 照 先 前 教 演 。
猴 王 把 四 个 老 猴 封 为 健 将 , 把 两 个 赤 尻 马 猴 叫 做 马 、 流 二 位 元 帅 , 两 个 通 背 猿 猴 叫 做 崩 、 芭 二 位 将 军 。
将 那 安 营 下 寨 、 赏 罚 等 事 , 都 交 付 给 四 位 健 将 。
他 放 下 心 , 日 逐 腾 云 驾 雾 , 遨 游 四 海 , 遨 游 千 山 。
施 展 武 艺 , 遍 访 英 雄 豪 杰 , 玩 弄 神 通 , 广 泛 交 朋 友 。
这 时 又 找 到 七 个 弟 兄 , 就 是 牛 魔 王 、 蛟 魔 王 、 鹏 魔 王 、 虎 王 、 猴 猴 王 、 王 , 连 自 己 家 的 美 猴 王 七 个 。
日 逐 讲 文 论 武 , 走 马 传 , 弦 歌 吹 舞 , 早 去 暮 回 , 没 有 一 般 儿 子 不 快 乐 。
把 那 万 里 之 遥 , 只 当 庭 院 之 路 , 所 谓 点 头 径 过 三 千 里 , 折 腰 八 百 多 里 。
有 一 天 , 在 本 洞 嘱 咐 四 位 健 将 安 排 筵 席 , 请 六 位 王 爷 去 喝 酒 , 杀 牛 宰 马 , 祭 天 享 地 , 带 着 各 种 怪 物 跳 舞 欢 歌 , 都 喝 得 酩 酊 大 醉 。
送 六 王 出 去 , 又 赏 赐 大 小 头 目 。
在 铁 板 桥 边 的 松 树 荫 荫 下 , 一 会 儿 就 睡 了 。
四 健 将 领 率 众 围 护 , 不 敢 高 声 大 叫 。
只 见 那 美 猴 王 在 睡 中 , 看 见 两 个 人 拿 着 一 张 批 文 , 上 面 有 个 叫 孙 悟 空 的 三 个 字 , 走 近 身 边 , 不 容 分 说 , 用 绳 子 系 上 绳 子 , 就 把 美 猴 王 的 魂 灵 索 去 , 踉 跄 , 一 直 带 到 了 一 座 城 边 。
猴 王 渐 渐 觉 得 酒 醒 了 , 忽 然 抬 头 看 , 那 城 上 有 一 块 铁 牌 , 铁 牌 上 有 三 个 大 字 , 是 个 幽 冥 界 。
美 猴 王 立 刻 醒 悟 , 说 : 幽 冥 界 是 阎 王 居 住 的 地 方 , 为 什 么 到 这 里 来 ? 那 两 个 人 说 : 你 今 天 的 阳 寿 该 死 , 我 们 两 个 人 领 着 批 批 , 勾 你 们 来 。
猴 王 听 了 , 说 : 我 的 老 孙 子 超 出 三 界 之 外 , 不 在 五 行 之 中 , 已 经 不 被 他 管 辖 , 怎 么 还 敢 来 勾 我 ?
这 个 猴 王 恼 怒 起 来 , 从 耳 朵 中 抽 出 一 个 宝 贝 , 一 个 锣 , 碗 子 里 的 粗 细 。
忽 然 举 手 , 把 两 个 捉 死 人 , 打 成 肉 酱 。
他 自 己 解 开 绳 索 , 抛 开 手 , 用 棒 子 打 进 城 中 。
得 到 那 个 牛 头 鬼 , 东 逃 西 藏 , 马 面 鬼 向 南 逃 跑 。
众 鬼 突 然 奔 上 森 罗 殿 , 报 告 说 : 大 王 , 是 祸 事 , 外 面 有 一 毛 毛 , 雷 公 打 将 来 了 。
他 急 忙 整 理 衣 服 来 看 , 见 他 相 貌 凶 恶 , 立 即 排 下 班 次 , 应 声 高 呼 道 : 上 仙 留 名 , 上 仙 留 名 , 猴 王 说 : 你 既 然 认 不 到 我 , 何 必 派 人 来 勾 我 ? 十 王 说 : 不 敢 , 不 敢 。
想 是 差 人 差 了 。
猴 王 说 : 我 本 来 是 花 果 山 水 帘 洞 , 天 生 圣 人 孙 悟 空 。
你 们 是 什 么 官 位 ? 十 王 亲 自 说 道 : 我 们 是 阴 间 天 子 , 十 代 冥 王 。
悟 空 说 : 快 报 名 来 , 免 打 。
十 王 说 : 我 们 是 秦 广 王 、 初 江 王 、 宋 帝 王 、 忤 官 王 、 阎 罗 王 、 平 等 王 、 泰 山 王 、 都 市 王 、 卞 城 王 、 转 轮 王 。
悟 空 说 : 你 们 既 然 登 上 王 位 , 又 是 灵 显 感 应 之 类 , 为 什 么 不 知 道 好 歹 ? 我 的 老 孙 子 修 炼 了 仙 道 , 与 上 天 同 寿 , 超 升 到 三 界 之 外 , 跳 出 五 行 之 中 , 为 什 么 会 被 人 拘 留 我 呢 ? 十 王 说 : 上 仙 息 怒 。
普 天 下 同 名 同 姓 的 人 很 多 , 怎 么 敢 说 那 些 勾 死 人 错 走 了 ? 悟 空 说 : 胡 说 , 胡 说 ! 常 言 道 : 官 差 吏 差 , 来 人 不 差 。
你 快 拿 出 生 死 簿 子 来 我 看 看 。 十 王 听 了 , 立 即 上 殿 查 看 。
悟 空 手 持 如 意 棒 , 径 直 登 上 森 罗 殿 上 , 正 中 间 面 向 南 坐 下 。
十 王 立 即 命 令 掌 管 案 卷 的 判 官 拿 出 文 簿 来 查 。
判 官 不 敢 怠 慢 , 便 到 司 房 里 , 捧 出 五 六 件 文 书 , 还 有 十 种 账 子 。
逐 一 个 查 看 , 虫 、 毛 虫 、 羽 虫 、 昆 虫 、 鳞 介 之 类 , 都 没 有 别 的 名 称 。
又 看 到 猴 类 一 类 , 原 来 这 个 猴 子 像 人 的 相 貌 , 不 入 人 的 名 字 ; 像 虫 , 不 居 国 界 ; 像 走 兽 , 不 埋 伏 麒 麟 管 ; 像 飞 禽 , 不 受 凤 凰 管 辖 。
另 外 还 有 一 个 账 子 , 悟 空 亲 自 检 阅 , 直 到 那 个 魂 字 一 千 三 百 五 十 个 字 上 , 才 注 上 孙 悟 空 的 名 字 , 是 说 : 天 生 石 猴 , 该 寿 命 三 百 四 十 二 岁 , 善 终 。
悟 空 说 : 我 也 不 记 得 寿 数 多 少 , 而 且 只 消 除 了 名 字 就 罢 了 。
拿 过 笔 来 。
那 个 判 官 急 忙 捧 笔 , 喝 得 很 浓 墨 。
悟 空 拿 过 的 簿 子 , 把 猴 属 之 类 , 只 要 有 名 字 的 , 一 概 勾 起 来 。
李 拿 出 账 子 说 : 了 账 , 了 账 , 今 天 不 服 你 管 了 。
一 路 棒 , 打 出 了 幽 冥 之 界 。
那 十 王 不 敢 接 近 , 都 去 翠 云 宫 , 一 同 拜 见 地 藏 菩 萨 , 商 量 启 奏 表 , 上 天 , 不 在 话 下 。
猴 王 打 出 城 中 , 忽 然 绊 着 一 个 草 鞋 , 跌 了 脚 脚 , 猛 然 醒 来 , 原 来 是 南 柯 一 梦 。
刚 刚 觉 得 伸 腰 , 只 听 到 四 位 健 将 和 众 猴 高 声 喊 道 : 大 王 , 喝 了 多 少 酒 , 睡 在 这 一 夜 , 还 不 能 醒 来 ? 悟 空 说 : 睡 还 小 可 , 我 梦 见 两 个 人 来 这 里 勾 我 , 把 我 带 到 幽 冥 界 的 城 门 外 。
是 我 显 神 通 , 径 直 吵 到 森 罗 殿 , 与 那 十 个 王 争 吵 , 把 我 们 的 生 死 簿 看 了 , 只 有 我 们 的 名 号 , 都 是 我 勾 了 , 都 不 服 从 那 个 奴 仆 所 管 辖 的 。
众 猴 叩 头 礼 谢 。
从 此 以 后 , 山 猴 多 有 不 老 的 , 是 因 为 阴 司 没 有 名 字 的 缘 故 。
美 猴 王 说 完 以 前 的 事 情 , 四 位 健 将 报 告 了 各 洞 的 妖 王 , 都 来 祝 贺 喜 庆 。
不 过 几 天 , 六 个 义 兄 弟 又 来 拜 贺 , 一 听 到 毁 谤 名 声 的 缘 故 , 又 一 个 个 欢 喜 , 每 天 都 聚 在 一 起 不 问 题 。
又 上 表 启 奏 说 : 万 岁 , 通 明 殿 外 有 东 海 龙 王 敖 广 进 表 , 听 天 尊 宣 读 诏 书 。
玉 皇 传 旨 说 : 著 宣 来 。
敖 广 宣 到 灵 霄 殿 下 , 礼 拜 完 毕 , 旁 边 有 人 引 奏 仙 童 接 上 表 文 。
玉 皇 从 头 看 过 。
上 表 说 : 水 元 下 界 东 胜 神 洲 、 东 海 小 龙 臣 敖 广 启 奏 大 天 圣 主 玄 穹 高 上 帝 君 : 近 来 因 为 花 果 山 生 、 水 帘 洞 住 妖 仙 孙 悟 空 , 欺 骗 小 龙 , 强 行 坐 在 水 宅 , 索 取 兵 器 , 施 行 法 令 , 施 行 威 势 , 要 他 们 披 挂 , 逞 凶 逞 势 。
惊 伤 水 族 , 走 龟 。
南 海 龙 战 战 兢 兢 , 西 海 龙 悲 哀 惨 惨 , 北 海 龙 缩 首 归 降 。
臣 敖 广 舒 身 下 拜 , 献 上 神 珍 的 铁 棒 、 凤 翅 的 金 冠 , 和 那 锁 子 甲 、 步 云 鞋 , 以 礼 送 出 。
他 还 戏 弄 武 艺 , 显 示 神 通 , 只 是 说 : 喧 哗 呀 呀 , 果 然 无 敌 , 很 难 制 服 。
臣 现 在 启 奏 , 希 望 圣 上 裁 决 。
恳 请 天 兵 , 收 捕 妖 孽 , 希 望 使 海 岳 清 宁 , 百 姓 安 泰 。
奉 上 奏 章 。
圣 帝 看 完 后 , 传 旨 说 : 着 龙 神 回 海 , 我 立 即 派 遣 将 领 擒 拿 。
老 龙 王 叩 头 谢 罪 而 去 。
下 面 又 有 葛 仙 翁 天 师 启 奏 道 : 万 岁 , 有 冥 司 秦 广 王 , 奉 上 幽 冥 教 主 地 藏 王 菩 萨 的 表 文 呈 上 。
旁 边 有 传 说 玉 女 接 上 表 文 。
玉 皇 也 从 头 看 过 。
刘 表 说 : 幽 冥 的 境 界 , 是 地 的 阴 司 。
天 有 神 , 地 有 鬼 , 阴 阳 互 转 , 禽 兽 有 生 , 兽 兽 有 死 , 反 复 有 雌 雄 。
生 生 变 化 , 怀 孕 女 孩 成 男 孩 , 这 是 自 然 的 规 律 , 不 能 改 变 的 。
现 在 有 花 果 山 水 帘 洞 天 生 出 妖 猴 孙 悟 空 , 逞 恶 行 凶 , 不 服 从 拘 捕 。
玩 弄 神 通 , 打 断 九 幽 鬼 使 , 依 仗 权 力 , 惊 吓 伤 害 了 十 代 慈 王 。
大 肆 喧 闹 , 强 行 消 除 名 声 。
以 至 于 猴 子 等 类 的 动 物 没 有 限 制 , 猕 猴 的 畜 类 多 长 寿 命 ; 寂 灭 轮 回 , 各 自 没 有 生 死 。
贫 僧 上 表 , 冒 犯 天 威 。
请 陛 下 调 遣 神 兵 , 收 降 这 些 妖 孽 , 整 顿 阴 阳 , 永 安 地 府 。
谨 上 奏 。
玉 皇 看 完 后 , 传 旨 说 : 著 冥 君 回 归 地 府 , 我 立 即 派 遣 将 领 擒 拿 。
秦 广 王 也 叩 头 谢 罪 而 去 。
大 天 尊 宣 众 文 武 仙 卿 , 问 道 : 这 个 妖 猴 是 多 年 生 育 的 , 哪 个 朝 代 出 身 , 竟 然 成 了 这 样 的 有 道 啊 一 句 话 没 说 完 , 班 中 忽 出 千 里 眼 、 顺 风 耳 说 : 这 个 猴 子 是 三 百 年 前 天 生 的 石 猴 。
当 时 不 以 为 然 , 不 知 道 这 几 年 在 什 么 地 方 修 炼 成 仙 , 降 龙 伏 虎 , 强 行 消 除 死 籍 。
玉 帝 说 : 那 路 神 将 要 下 界 收 伏 ? 说 不 完 , 班 中 闪 出 太 白 星 、 长 庚 星 , 俯 伏 启 奏 道 : 上 圣 , 三 界 中 凡 有 九 窍 的 , 都 可 以 修 炼 成 仙 。
可 是 这 个 猴 是 天 地 生 成 的 形 体 , 日 月 孕 成 的 身 体 , 他 也 是 顶 天 踏 地 , 服 露 餐 霞 , 现 在 既 然 修 成 仙 道 , 有 降 龙 伏 虎 的 能 力 , 与 人 有 什 么 不 同 呢 ? 我 启 奏 陛 下 , 可 以 念 念 生 化 的 慈 恩 , 降 下 一 道 招 安 圣 旨 , 把 它 宣 来 上 界 , 授 给 他 一 个 大 小 官 职 , 与 他 的 名 字 在 地 , 拘 束 在 这 里 。
如 果 接 受 天 命 , 以 后 再 提 升 赏 赐 ; 如 果 违 背 天 命 , 就 此 擒 拿 。
一 是 不 动 众 动 众 , 劳 师 动 众 , 二 是 得 到 成 仙 , 这 是 有 道 的 。
玉 帝 听 了 他 的 话 非 常 高 兴 , 说 : 依 你 所 奏 。
于 是 就 写 文 曲 星 官 修 诏 , 写 太 白 金 星 招 安 。
金 星 领 了 圣 旨 , 出 了 南 天 门 外 , 按 下 祥 云 , 直 到 花 果 山 水 帘 洞 , 对 众 小 猴 说 : 我 是 天 差 天 使 , 有 圣 旨 在 这 里 , 请 你 大 王 上 界 。
快 快 回 报 。
洞 外 的 小 猴 子 一 层 一 层 一 层 一 层 一 层 一 层 一 层 地 传 到 洞 天 深 处 , 说 道 : 大 王 , 外 面 有 一 个 老 人 , 背 着 一 个 角 的 文 书 , 说 是 上 天 派 来 的 天 使 , 有 圣 旨 请 你 。
美 猴 王 听 了 非 常 高 兴 , 说 道 : 我 这 两 天 正 在 思 虑 想 要 上 天 去 走 , 忽 然 有 天 使 来 请 求 。
他 大 叫 道 : 快 请 进 来 吧 。
猴 王 急 忙 整 理 衣 帽 , 在 门 外 迎 接 。
金 星 径 直 进 入 当 中 , 面 南 立 定 说 : 我 是 西 方 太 白 金 星 , 奉 玉 帝 招 安 圣 旨 , 下 界 请 你 上 天 , 拜 受 仙 丹 。
悟 空 笑 着 说 : 多 感 老 星 降 临 。
教 小 的 人 安 排 筵 席 , 款 待 。
金 星 说 : 圣 旨 在 身 , 不 敢 久 留 。
就 请 大 王 一 同 前 往 , 等 尔 朱 荣 迁 都 之 后 , 再 从 容 讲 和 。
悟 空 说 : 承 蒙 光 辉 照 顾 , 空 退 , 空 退 。
于 是 就 叫 来 四 位 健 将 , 吩 咐 说 : 谨 慎 地 教 育 儿 孙 , 等 我 上 天 去 看 看 路 , 还 好 让 你 们 上 去 同 住 。
四 位 健 将 领 都 答 应 了 。
这 个 猴 王 和 金 星 纵 起 云 头 , 升 到 天 空 之 上 。
正 是 那 么 , 高 升 上 品 天 仙 位 , 名 列 云 班 宝 之 中 。
我 最 终 不 知 授 给 你 什 么 官 爵 , 暂 且 听 下 回 分 解 。
}\switchcolumn\flushpage  \begin{pinyinscope}{\myfontt \section{第四回}     官封弼馬心何足 名注齊天意未寧

那太白金星與美猴王同出了洞天深處,一齊駕雲而起。原來悟空觔斗雲比眾不
同,十分快疾,把個金星撇在腦後,先至南天門外。正欲收雲前進,被增長天
王領著龐、劉、苟、畢、鄧、辛、張、陶一路大力天丁,槍刀劍戟,擋住天門
,不肯放進。猴王道:「這個金星老兒乃奸詐之徒,既請老孫,如何教人動刀
動槍,阻塞門路?」正嚷間,金星倏到。悟空就覿面發狠道:「你這老兒,怎
麼哄我?被你說奉玉帝招安旨意來請,卻怎麼教這些人阻住天門,不放老孫進
去?」金星笑道:「大王息怒。你自來未曾到此天堂,卻又無名,眾天丁又與
你素不相識,他怎肯放你擅入?等如今見了天尊,授了仙籙,注了官名,向後
隨你出入,誰復擋也?」悟空道:「這等說,也罷,我不進去了。」金星又用
手扯住道:「你還同我進去。」

將近天門,金星高叫道:「那天門天將、大小吏兵,放開路者。此乃下界仙人
,我奉玉帝聖旨,宣他來也。」那增長天王與眾天丁俱才斂兵退避。猴王始信
其言,同金星緩步入裏觀看。真個是:
初登上界,乍入天堂。金光萬道滾紅霓,瑞氣千條噴紫霧。只見那南天門,碧
沉沉,琉璃造就﹔明幌幌,寶玉粧成。兩邊擺數十員鎮天元帥,一員員頂梁靠
柱,持銑擁旄﹔四下列十數個金甲神人,一個個執戟懸鞭,持刀仗劍。外廂猶
可,入內驚人:裏壁廂有幾根大柱,柱上纏繞著金鱗耀日赤鬚龍﹔又有幾座長
橋,橋上盤旋著彩羽凌空丹頂鳳。明霞幌幌映天光,碧霧濛濛遮斗口。這天上
有三十三座天宮,乃遣雲宮、毘沙宮、五明宮、太陽宮、花樂宮,……一宮宮
脊吞金穩獸﹔又有七十二重寶殿,乃朝會殿、凌虛殿、寶光殿、天王殿、靈官
殿,……一殿殿柱列玉麒麟。壽星臺上,有千千年不卸的名花﹔煉藥爐邊,有
萬萬載常青的繡草。又至那朝聖樓前,絳紗衣,星辰燦爛﹔芙蓉冠,金璧輝煌
。玉簪珠履,紫綬金章。金鐘撞動,三曹神表進丹墀﹔天鼓鳴時,萬聖朝王參
玉帝。又至那靈霄寶殿,金釘攢玉戶,彩鳳舞朱門。復道迴廊,處處玲瓏剔透
﹔三簷四簇,層層龍鳳翱翔。上面有個紫巍巍,明幌幌,圓丟丟,亮灼灼,大
金葫蘆頂。下面有天妃懸掌扇,玉女捧仙巾,惡狠狠掌朝的天將,氣昂昂護駕
的仙卿。正中間,琉璃盤內,放許多重重疊疊太乙丹﹔瑪瑙瓶中,插幾枝彎彎
曲曲珊瑚樹。正是天宮異物般般有,世上如他件件無。金闕銀鑾並紫府,琪花
瑤草暨瓊葩。朝王玉兔壇邊過,參聖金烏著底飛。猴王有分來天境,不墮人間
點污泥。

太白金星領著美猴王,到於靈霄殿外,不等宣詔,直至御前,朝上禮拜。悟空
挺身在傍,且不朝禮,但側耳以聽金星啟奏。金星奏道:「臣領聖旨,已宣妖
仙到了。」玉帝垂簾問曰:「那個是妖仙?」悟空卻才躬身答應道:「老孫便
是。」仙卿們都大驚失色道:「這個野猴,怎麼不拜伏參見,輒敢這等答應道
:『老孫便是。』卻該死了,該死了。」玉帝傳旨道:「那孫悟空乃下界妖仙
,初得人身,不知朝禮,且姑恕罪。」眾仙卿叫聲:「謝恩。」猴王卻才朝上
唱個大喏。玉帝宣文選武選仙卿,看那處少甚官職,著孫悟空去除授。傍邊轉
過武曲星君啟奏道:「天宮裏各宮各殿,各方各處,都不少官,只是御馬監缺
個正堂管事。」玉帝傳旨道:「就除他做個弼馬溫罷。」眾臣叫謝恩,他也只
朝上唱個大喏。玉帝又差木德星官送他去御馬監到任。

當時猴王歡歡喜喜,與木德星官徑去到任。事畢,木德星官回宮。他在監裏,
會聚了監丞、監副、典簿、力士、大小官員人等,查明本監事務,止有天馬千
匹。乃是:
驊騮騏驥,騄駬纖離﹔龍媒紫燕,挾翼驌驦﹔駃騠銀騔,騕褭飛黃﹔騊駼翻羽
,赤兔超光﹔踰輝彌景,騰霧勝黃﹔追風絕地,飛奔霄﹔逸飄赤電,銅爵浮雲
﹔驄瓏虎,絕塵紫鱗﹔……四極大宛,八駿九逸,千里絕群。此等良馬,一個
個嘶風逐電精神壯,踏霧登雲氣力長。

這猴王查看了文簿,點明了馬數。本監中典簿管徵備草料﹔力士官管刷洗馬匹
、扎草、飲水、煮料﹔監丞、監副輔佐催辦。弼馬晝夜不睡,滋養馬匹。日間
舞弄猶可,夜間看管慇懃:但是馬睡的,趕起來吃草﹔走的,捉將來靠槽。那
些天馬見了他,泯耳攢蹄。倒養得肉肥膘滿。不覺的半月有餘。

一朝閑暇,眾監官都安排酒席,一則與他接風,二則與他賀喜。正在歡飲之間
,猴王忽停杯問曰:「我這弼馬溫是個甚麼官銜?」眾曰:「官名就是此了。」
又問:「此官是個幾品?」眾道:「沒有品從。」猴王道:「沒品,想是大之
極也?」眾道:「不大,不大,只喚做未入流。」猴王道:「怎麼叫做『未入
流』?」眾道:「末等。這樣官兒最低最小,只可與他看馬。似堂尊到任之後
,這等慇懃,喂得馬肥,只落得道聲『好』字﹔如稍有些尪羸,還要見責﹔再
十分傷損,還要罰贖問罪。」猴王聞此,不覺心頭火起,咬牙大怒道:「這般
藐視老孫!老孫在那花果山稱王稱祖,怎麼哄我來替他養馬?養馬者,乃後生
小輩下賤之役,豈是待我的?不做他,不做他,我將去也。」忽喇的一聲,把
公案推倒,耳中取出寶貝,幌一幌,碗來粗細,一路解數,直打出御馬監,徑
至南天門。眾天丁知他受了仙籙,乃是個弼馬溫,不敢阻當,讓他打出天門去
了。

須臾,按落雲頭,回至花果山上。只見那四健將與各洞妖王,在那裏操演兵卒
。這猴王厲聲高叫道:「小的們,老孫來了。」一群猴都來叩頭,迎接進洞天
深處,請猴王高登寶位,一壁廂辦酒接風。都道:「恭喜大王,上界去十數年
,想必得意榮歸也?」猴王道:「我才半月有餘,那裏有十數年?」眾猴道:
「大王,你在天上不覺時辰。天上一日,就是下界一年哩。請問大王,官居何
職?」猴王搖手道:「不好說,不好說,活活的羞殺人。那玉帝不會用人,他
見老孫這般模樣,封我做個甚麼『弼馬溫』,原來是與他養馬,未入流品之類
。我初到任時不知,只在御馬監中頑耍。及今日問我同寮,始知是這等卑賤。
老孫心中大惱,推倒席面,不受官銜,因此走下來了。」眾猴道:「來得好,
來得好。大王在這福地洞天之處為王,多少尊重快樂,怎麼肯去與他做馬夫?」
教小的們快辦酒來,與大王釋悶。

正飲酒歡會間,有人來報道:「大王,門外有兩個獨角鬼王,要見大王。」猴
王道:「教他進來。」那鬼王整衣跑入洞中,倒身下拜。美猴王問他:「你見
我何幹?」鬼王道:「久聞大王招賢,無由得見﹔今見大王授了天籙,得意榮
歸,特獻赭黃袍一件,與大王稱慶。肯不棄鄙賤,收納小人,亦得效犬馬之勞
。」猴王大喜,將赭黃袍穿起。眾等欣然排班朝拜。即將鬼王封為前部總督先
鋒。鬼王謝恩畢,復啟道:「大王在天許久,所授何職?」猴王道:「玉帝輕
賢,封我做個甚麼『弼馬溫』。」鬼王聽言,又奏道:「大王有此神通,如何
與他養馬?就做個齊天大聖,有何不可?」猴王聞說,歡喜不勝,連道幾個
「好!好!好!」教四健將:「就替我快置個旌旗,旗上寫『齊天大聖』四大
字,立竿張掛。自此以後,只稱我為齊天大聖,不許再稱大王。亦可傳與各洞
妖王,一體知悉。」此不在話下。

卻說那玉帝次日設朝,只見張天師引御馬監監丞、監副在丹墀下拜奏道:「萬
歲,新任弼馬溫孫悟空,因嫌官小,昨日反下天宮去了。」正說間,又見南天
門外增長天王領眾天丁,亦奏道:「弼馬溫不知何故,走出天門去了。」玉帝
聞言,即傳旨:「著兩路神元,各歸本職。朕遣天兵,擒拿此怪。」班部中閃
上托塔李天王與哪吒三太子,越班奏上道:「萬歲,微臣不才,請旨降此妖怪
。」玉帝大喜,即封托塔天王李靖為降魔大元帥,哪吒三太子為三壇海會大神
,即刻興師下界。

李天王與哪吒叩頭謝辭,徑至本宮,點起三軍,帥眾頭目,著巨靈神為先鋒,
魚肚將掠後,藥叉將催兵。一霎時出南天門外,徑來到花果山,選平陽處安了
營寨,傳令教巨靈神挑戰。巨靈神得令,結束整齊,掄著宣花斧,到了水簾洞
外。只見小洞門外許多妖魔,都是些狼蟲虎豹之類,丫丫叉叉,掄槍舞劍,在
那裏跳鬥咆哮。這巨靈神喝道:「那業畜!快早去報與弼馬溫知道:吾乃上天
大將,奉玉帝旨意,到此收伏﹔教他早早出來受降,免致汝等皆傷殘也。」那
些妖怪奔奔波波,傳報洞中道:「禍事了!禍事了!」猴王問:「有甚禍事?」
眾妖道:「門外有一員天將,口稱大聖官銜,道奉玉帝聖旨,來此收伏。教早
早出去受降,免傷我等性命。」猴王聽說,教:「取我披掛來。」就戴上紫金
冠,貫上黃金甲,登上步雲鞋,手執如意金箍棒,領眾出門,擺開陣勢。這巨
靈神睜睛觀看,真好猴王:
        身穿金甲亮堂堂,頭戴金冠光映映。
    手舉金箍棒一根,足踏雲鞋皆相稱。
    一雙怪眼似明星,兩耳過肩查又硬。
    挺挺身才變化多,聲音響喨如鐘磬。
    尖嘴咨牙弼馬溫,心高要做齊天聖。

巨靈神厲聲高叫道:「那潑猴!你認得我麼?」大聖聽言,急問道:「你是那
路毛神?老孫不曾會你,你快報名來。」巨靈神道:「我把你那欺心的猢猻!
你是認不得我。我乃高上神霄托塔李天王部下先鋒巨靈天將,今奉玉帝聖旨,
到此收降你。你快卸了裝束,歸順天恩,免得這滿山諸畜遭誅﹔若道半個不字
,教你頃刻化為齏粉。」猴王聽說,心中大怒道:「潑毛神!休誇大口,少弄
長舌。我本待一棒打死你,恐無人去報信。且留你性命,快早回天,對玉皇說
:他甚不用賢。老孫有無窮的本事,為何教我替他養馬?你看我這旌旗上字號
,若依此字號陞官,我就不動刀兵,自然的天地清泰﹔如若不依,時間就打上
靈霄寶殿,教他龍床定坐不成。」這巨靈神聞此言,急睜睛迎風觀看,果見門
外豎一高竿,竿上有旌旗一面,上寫著「齊天大聖」四大字。巨靈神冷笑三聲
道:「這潑猴,這等不知人事,輒敢無狀,你就要做齊天大聖。好好的吃吾一
斧。」劈頭就砍將去。那猴王正是會家不忙,將金箍棒應手相迎。這一場好殺:
棒名如意,斧號宣花。他兩個乍相逢,不知深淺,斧和棒,左右交加。一個暗
藏神妙,一個大口稱誇。使動法,噴雲曖霧﹔展開手,播土揚沙。天將神通就
有道,猴王變化實無涯。棒舉卻如龍戲水,斧來猶似鳳穿花。巨靈名望傳天下
,原來本事不如他:大聖輕輕掄鐵棒,著頭一下滿身麻。

巨靈神抵敵他不住,被猴王劈頭一棒,慌忙將斧架隔,扢扠的一聲,把個斧柄
打做兩截,急撤身敗陣逃生。猴王笑道:「膿包,膿包。我已饒了你,你快去
報信,快去報信。」

巨靈神回至營門,徑見托塔天王,忙哈哈跪下道:「弼馬溫果是神通廣大,末
將戰他不得,敗陣回來請罪。」李天王發怒道:「這廝剉吾銳氣,推出斬之!
」傍邊閃出哪吒太子拜告:「父王息怒,且恕巨靈之罪。待孩兒出師一遭,便
知深淺。」天王聽諫,且教回營待罪管事。

這哪吒太子甲冑齊整,跳出營盤,撞至水簾洞外。那悟空正來收兵,見哪吒來
的勇猛。好太子:
總角才遮?,披毛未蓋肩。神奇多敏悟,骨秀更清妍。誠為天上麒麟子,果是
煙霞彩鳳仙。龍種自然非俗相,妙齡端不類塵凡。身帶六般神器械,飛騰變化
廣無邊。今受玉皇金口詔,敕封海會號三壇。

悟空迎近前來問曰:「你是誰家小哥?闖近吾門,有何事幹?」哪吒喝道:
「潑妖猴!豈不認得我?我乃托塔天王三太子哪吒是也,今奉玉帝欽差,至此
捉你。」悟空笑道:「小太子,你的嬭牙尚未退,胎毛尚未乾,怎敢說這般大
話?我且留你的性命,不打你。你只看我旌旗上是甚麼字號,拜上玉帝:是這
般官銜,再也不須動眾,我自皈依﹔若是不遂我心,定要打上靈霄寶殿。」哪
吒抬頭看處,乃「齊天大聖」四字。哪吒道:「這妖猴能有多大神通,就敢稱
此名號?不要怕,吃吾一劍。」悟空道:「我只站下不動,任你砍幾劍罷。」
那哪吒奮怒,大喝一聲,叫:「變!」即變做三頭六臂,惡狠狠,手持著六般
兵器,乃是斬妖劍、砍妖刀、縛妖索、降妖杵、繡毬兒、火輪兒,丫丫叉叉,
撲面來打。悟空見了,心驚道:「這小哥倒也會弄些手段。莫無禮,看我神通
。」好大聖,喝聲:「變!」也變做三頭六臂﹔把金箍棒幌一幌,也變作三條
。六隻手拿著三條棒架住。這場鬥,真個是地動山搖,好殺也:
六臂哪吒太子,天生美石猴王,相逢真對手,正遇本源流。那一個蒙差來下界
,這一個欺心鬧斗牛。斬妖寶劍鋒芒快,砍妖刀狠鬼神愁﹔縛妖索子如飛蟒,
降妖大杵似狼頭﹔火輪掣電烘烘艷,往往來來滾繡毬。大聖三條如意棒,前遮
後擋運機謀。苦爭數合無高下,太子心中不肯休。把那六件兵器多教變,百千
萬億照頭丟。猴王不懼呵呵笑,鐵棒翻騰自運籌。以一化千千化萬,滿空亂舞
賽飛虯。諕得各洞妖王都閉戶,遍山鬼怪盡藏頭。神兵怒氣雲慘慘,金箍鐵棒
響颼颼。那壁廂,天丁吶喊人人怕﹔這壁廂,猴怪搖旗個個憂。發狠兩家齊鬥
勇,不知那個剛強那個柔。

三太子與悟空各騁神威,鬥了個三十回合。那太子六般兵,變做千千萬萬﹔孫
悟空金箍棒,變作萬萬千千。半空中似雨點流星,不分勝負。

原來悟空手疾眼快,正在那混亂之時,他拔下一根毫毛,叫聲:「變!」就變
做他的本相,手挺著棒,演著哪吒﹔他的真身,卻一縱,趕至哪吒腦後,著左
膊上一棒打來。哪吒正使法間,聽得棒頭風響,急躲閃時,不能措手,被他著
了一下,負痛逃走。收了法,把六件兵器依舊歸身,敗陣而回。

那陣上李天王早已看見,急欲提兵助戰,不覺太子倏至面前,戰兢兢報道:
「父王,弼馬溫真個有本事,孩兒這般法力,也戰他不過,已被他打傷膊也。」
天王大驚失色道:「這廝恁的神通,如何取勝?」太子道:「他洞門外豎一竿
旗,上寫『齊天大聖』四字。親口誇稱,教玉帝就封他做齊天大聖,萬事俱休
﹔若還不是此號,定要打上靈霄寶殿哩。」天王道:「既然如此,且不要與他
相持,且去上界,將此言回奏,再多遣天兵,圍捉這廝,未為遲也。」太子負
痛,不能復戰,故同天王回天啟奏不題。

你看那猴王得勝歸山,那七十二洞妖王與那六弟兄,俱來賀喜,在洞天福地,
飲樂無比。他卻對六弟兄說:「小弟既稱齊天大聖,你們亦可以大聖稱之。」
內有牛魔王忽然高叫道:「賢弟言之有理,我即稱做平天大聖。」蛟魔王道:
「我稱做覆海大聖。」鵬魔王道:「我稱混天大聖。」獅狔王道:「我稱移山
大聖。」獼猴王道:「我稱通風大聖。」狨王道:「我稱驅神大聖。」此時七
大聖自作自為,自稱自號,耍樂一日,各散訖。

卻說那李天王與三太子領著眾將,直至靈霄寶殿,啟奏道:「臣等奉聖旨出師
下界,收伏妖仙孫悟空,不期他神通廣大,不能取勝,仍望萬歲添兵剿除。」
玉帝道:「諒一妖猴,有多少本事,還要添兵?」太子又近前奏道:「望萬歲
赦臣死罪。那妖猴使一條鐵棒,先敗了巨靈神,又打傷臣臂膊。洞門外立一竿
旗,上書『齊天大聖』四字。道是封他這官職,即便休兵來投﹔若不是此官,
還要打上靈霄寶殿也。」玉帝聞言,驚訝道:「這妖猴何敢這般狂妄?著眾將
即刻誅之。」

正說間,班部中又閃出太白金星,奏道:「那妖猴只知出言,不知大小。欲加
兵與他爭鬥,想一時不能收伏,反又勞師。不若萬歲大捨恩慈,還降招安旨意
,就教他做個齊天大聖。只是加他個空銜,有官無祿便了。」玉帝道:「怎麼
喚做『有官無祿』?」金星道:「名是齊天大聖,只不與他事管,不與他俸祿
,且養在天壤之間,收他的邪心,使不生狂妄,庶乾坤安靖,海宇得清寧也。」
玉帝聞言道:「依卿所奏。」即命降了詔書,仍著金星領去。

金星復出南天門,直至花果山水簾洞外觀看。這番比前不同,威風凜凜,殺氣
森森,各樣妖精,無般不有。一個個都執劍拈槍,拿刀弄杖的在那裏咆哮跳躍
。一見金星,皆上前動手。金星道:「那眾頭目來,累你去報你大聖知之:吾
乃上帝遣來天使,有聖旨在此請他。」眾妖即跑入報道:「外面有一老者,他
說是上界天使,有旨意請你。」悟空道:「來得好,來得好。想是前番來的那
太白金星。那次請我上界,雖是官爵不堪,卻也天上走了一次,認得那天門內
外之路。今番又來,定有好意。」教眾頭目大開旗鼓,擺隊迎接。大聖即帶引
群猴,頂冠貫甲,甲上罩了赭黃袍,足踏雲履,急出洞門,躬身施禮,高叫道
:「老星請進,恕我失迎之罪。」

金星趨步向前,徑入洞內,面南立著道:「今告大聖:前者因大聖嫌惡官小,
躲離御馬監,當有本監中大小官員奏了玉帝。玉帝傳旨道:『凡授官職,皆由
卑而尊,為何嫌小?』即有李天王領哪吒下界取戰。不知大聖神通,故遭敗北
,回天奏道:『大聖立一竿旗,要做齊天大聖。眾武將還要支吾,是老漢力為
大聖冒罪奏聞,免興師旅,請大王授籙。玉帝准奏,因此來請。」悟空笑道:
「前番動勞,今又蒙愛,多謝,多謝!但不知上天可與我齊天大聖之官銜也?」
金星道:「老漢以此銜奏准,方敢領旨而來﹔如有不遂,只坐罪老漢便是。」

悟空大喜,懇留飲宴不肯,遂與金星縱著祥雲,到南天門外。那些天丁天將都
拱手相迎。徑入靈霄殿下。金星拜奏道:「臣奉詔宣弼馬溫孫悟空已到。」玉
帝道:「那孫悟空過來,今宣你做個齊天大聖,官品極矣,但切不可胡為。」
這猴亦止朝上唱個喏,道聲「謝恩」。玉帝即命工幹官張、魯二班,在蟠桃園
右首,起一座齊天大聖府,府內設個二司:一名安靜司,一名寧神司。司俱有
仙吏,左右扶持。又差五斗星君送悟空去到任,外賜御酒二瓶,金花十朵,著
他安心定志,再勿胡為。

那猴王信受奉行,即日與五斗星君到府,打開酒瓶,同眾盡飲。送星官回轉本
宮,他才遂心滿意,喜地歡天,在於天宮快樂,無掛無礙。正是:
    仙名永注長生籙,不墮輪迴萬古傳。
  畢竟不知向後如何,且聽下回分解。





}  \end{pinyinscope}\switchcolumn{\myfontc \section{第 四 回} 官 封 弼 、 马 心 何 足 称 注 齐 , 天 意 未 宁 , 那 太 白 金 星 与 美 猴 王 一 同 出 了 洞 天 深 处 , 一 齐 驾 云 而 起 。
原 来 悟 空 斗 云 和 众 不 同 , 十 分 快 快 , 把 金 星 放 在 脑 后 , 先 到 了 南 天 门 外 。
我 正 想 要 收 拾 云 前 进 , 被 增 长 天 王 领 着 庞 、 刘 、 苟 、 毕 、 邓 、 辛 、 张 、 陶 一 路 大 力 天 丁 , 刀 剑 戟 戟 , 阻 住 天 门 , 不 肯 放 进 。
猴 王 说 : 这 个 金 星 老 儿 是 奸 诈 之 徒 , 既 然 请 求 老 孙 , 为 什 么 让 人 动 刀 动 箭 , 堵 塞 门 路 ? 正 在 喧 哗 之 间 , 金 星 突 然 到 来 。
金 星 笑 着 说 : 你 这 个 老 子 , 为 什 么 哄 我 , 被 你 说 奉 玉 帝 招 安 的 旨 意 来 请 求 , 又 怎 么 能 让 这 些 人 阻 住 天 门 , 不 让 老 孙 进 去 ? 金 星 笑 着 说 : 大 王 息 怒 。
你 自 来 未 曾 到 过 这 个 天 堂 , 却 又 没 有 名 字 , 众 天 丁 又 与 你 一 向 不 认 识 , 他 怎 么 肯 放 你 擅 自 进 去 呢 等 如 今 见 了 天 尊 , 授 给 我 仙 , 注 了 官 名 , 以 后 随 你 出 入 , 谁 还 能 阻 挡 ? 悟 空 说 : 这 些 说 法 , 也 罢 了 , 我 不 进 去 了 。
金 星 又 用 手 扯 住 道 : 你 还 跟 我 进 去 。
将 要 走 到 天 门 , 金 星 高 声 喊 道 : 那 天 门 天 将 、 大 小 官 兵 , 放 开 路 上 的 人 。
这 是 下 界 的 仙 人 , 我 奉 玉 帝 的 圣 旨 , 宣 扬 他 来 。
那 增 长 天 王 与 众 天 丁 都 才 收 兵 退 避 。
猴 王 这 才 相 信 了 他 的 话 , 和 金 星 一 样 缓 步 进 入 里 面 观 看 。
真 个 是 : 初 登 上 界 , 忽 入 天 堂 。
金 光 万 道 滚 滚 红 霓 , 瑞 气 千 条 喷 紫 雾 。
只 见 那 南 天 门 , 碧 色 沉 沉 , 琉 璃 造 成 , 明 亮 的 模 样 , 宝 玉 装 成 。
两 边 摆 了 几 十 个 镇 天 元 帅 , 另 一 个 人 头 顶 梁 梁 靠 柱 , 手 持 戟 柄 , 四 下 排 列 着 十 几 个 金 甲 神 人 , 一 个 个 拿 戟 悬 鞭 , 持 刀 仗 剑 。
外 边 的 厢 房 还 可 以 , 进 入 内 里 惊 人 : 内 壁 的 厢 房 有 几 根 大 柱 , 柱 上 缠 绕 着 金 鳞 耀 日 赤 须 龙 , 又 有 几 座 长 桥 , 桥 上 盘 旋 着 彩 羽 凌 空 的 丹 顶 凤 凰 。
明 霞 飘 飘 映 着 天 光 , 碧 雾 茫 茫 遮 住 斗 口 。
天 上 有 三 十 三 座 天 宫 , 乃 遣 云 宫 、 毗 沙 宫 、 五 明 宫 、 太 阳 宫 、 花 乐 宫 , 一 宫 宫 脊 , 吞 金 稳 兽 ; 又 有 七 十 二 座 宝 殿 , 是 朝 会 殿 、 凌 虚 殿 、 宝 光 殿 、 天 王 殿 、 灵 官 殿 , 一 殿 殿 柱 上 有 玉 麒 麟 。
寿 星 台 上 , 有 千 千 年 不 卸 的 名 花 ; 炼 药 炉 边 , 有 万 万 年 常 青 的 绣 草 。
又 来 到 那 朝 圣 楼 前 , 穿 着 绛 纱 衣 服 , 星 辰 灿 烂 , 芙 蓉 冠 , 金 玉 辉 煌 。
玉 簪 珠 履 , 紫 绶 金 章 。
金 钟 撞 击 , 三 曹 神 表 进 献 丹 墀 ; 天 鼓 鸣 时 , 万 圣 朝 拜 玉 帝 。
又 到 那 灵 霄 宝 殿 , 金 钉 攒 玉 门 , 彩 凤 舞 朱 门 。
复 道 回 绕 的 廊 廊 , 处 处 玲 珑 剔 透 ; 三 处 屋 檐 四 处 簇 拥 , 层 层 的 龙 凤 翱 翔 。
上 面 有 紫 色 的 , 光 亮 闪 闪 , 圆 又 闪 闪 , 光 亮 闪 闪 , 大 金 葫 芦 顶 。
下 面 有 天 妃 挂 着 手 扇 , 玉 女 捧 着 仙 巾 , 恶 恶 狠 狠 掌 朝 的 天 将 , 气 势 昂 昂 , 护 驾 的 仙 卿 。
正 中 间 , 琉 璃 盘 里 放 着 许 多 重 叠 叠 的 太 乙 丹 , 玛 瑙 瓶 中 , 插 着 几 枝 弯 弯 曲 曲 的 珊 瑚 树 。
这 正 是 天 宫 奇 异 的 东 西 , 一 般 都 有 , 世 上 像 其 他 的 东 西 一 样 都 没 有 。
金 阙 银 銮 并 入 紫 府 , 琪 花 瑶 草 遍 地 琼 花 。
朝 王 玉 兔 坛 边 过 , 参 圣 金 乌 著 底 飞 。
猴 王 有 分 别 来 到 天 界 , 不 堕 入 人 间 点 污 泥 。
太 白 金 星 领 着 美 猴 王 来 到 灵 霄 殿 外 , 不 等 宣 读 诏 书 , 径 直 到 皇 帝 面 前 , 朝 见 皇 上 , 行 礼 拜 礼 。
悟 空 挺 身 在 旁 边 , 并 且 没 有 朝 见 皇 帝 的 礼 节 , 只 是 侧 着 耳 朵 听 金 星 的 启 奏 。
金 星 上 奏 说 : 臣 领 圣 旨 , 已 经 宣 告 妖 仙 到 了 。
玉 帝 垂 帘 问 道 : 那 个 是 妖 仙 ? 悟 空 才 亲 身 答 应 道 : 老 孙 就 是 。
仙 卿 大 惊 失 色 , 说 : 这 个 野 猴 , 何 不 拜 伏 参 见 , 竟 敢 这 样 回 答 说 : 老 孙 就 是 。
还 是 该 死 了 , 该 死 了 。
玉 帝 传 旨 说 : 那 个 孙 子 悟 空 是 下 界 的 妖 仙 , 刚 刚 得 到 人 身 , 不 知 朝 廷 礼 仪 , 姑 且 宽 恕 他 的 罪 过 。
众 仙 卿 大 叫 道 : 谢 恩 。
猴 王 刚 刚 上 朝 , 喊 个 大 锣 。
玉 帝 、 宣 帝 、 文 帝 选 武 帝 选 仙 卿 , 看 那 个 地 方 少 有 官 职 , 就 写 出 孙 悟 空 去 除 授 。
从 旁 边 转 过 武 曲 , 星 君 启 奏 道 : 天 宫 里 各 宫 各 殿 , 各 方 各 处 , 都 不 少 官 职 , 只 是 御 马 监 缺 一 个 正 堂 管 事 。
玉 帝 传 旨 说 : 除 他 做 个 弼 马 温 罢 。
众 臣 称 谢 恩 , 其 他 只 是 在 朝 廷 上 喊 个 大 锣 。
玉 帝 又 派 木 德 星 官 送 他 去 , 御 马 监 到 任 。
当 时 猴 王 喜 欢 喜 喜 , 与 木 德 星 官 径 直 到 任 。
事 情 结 束 , 木 德 星 官 回 宫 。
其 所 以 , 其 所 以 为 之 , 其 所 以 为 之 , 其 所 以 为 之 , 其 所 以 为 之 , 其 所 以 为 之 , 其 所 以 为 之 。
就 是 这 样 , 骅 、 骅 骥 , 、 、 、 、 、 紫 燕 , 挟 着 翅 膀 、 飞 黄 ; 、 银 、 赤 兔 翻 动 , 赤 兔 超 越 光 彩 ; 辉 辉 映 耀 , 腾 腾 的 云 气 , 腾 腾 的 雾 气 , 胜 过 黄 河 ; 追 风 绝 地 , 飞 奔 的 红 电 , 铜 爵 浮 云 ; 的 虎 , 绝 尘 紫 鳞 ; 四 极 大 宛 , 八 骏 九 逸 , 千 里 绝 群 。
这 些 良 马 , 一 个 个 嘶 风 追 电 , 精 神 壮 健 , 踏 雾 登 云 , 气 力 壮 大 。
这 个 猴 王 查 看 了 文 簿 , 点 明 了 马 的 数 字 。
本 监 中 的 典 簿 掌 管 征 收 草 料 , 力 士 掌 管 刷 洗 马 匹 、 扎 草 、 喝 水 、 煮 料 , 监 丞 、 监 副 辅 佐 催 促 办 理 。
富 弼 的 马 昼 夜 不 睡 , 养 马 匹 。
夜 间 跳 舞 玩 耍 还 可 以 , 夜 间 看 管 弦 : 只 是 马 睡 的 , 赶 起 来 吃 草 , 跑 的 , 捉 拿 着 来 靠 槽 。
那 些 天 马 见 了 他 , 就 毁 灭 了 他 的 耳 朵 和 蹄 子 。
倒 养 得 肌 肉 肥 壮 , 肌 肉 丰 满 。
不 觉 半 个 月 有 余 。
一 旦 闲 暇 , 众 监 官 都 安 排 酒 席 , 一 则 与 其 他 人 交 往 , 二 则 与 其 他 人 祝 贺 。
正 在 欢 饮 之 间 , 猴 王 忽 然 停 下 酒 杯 问 道 : 我 这 个 弼 马 温 是 什 么 官 职 ? 众 人 说 : 官 名 就 是 这 个 了 。
又 问 : 这 个 官 员 是 个 几 个 人 ? 众 人 说 : 没 有 品 从 。
猴 王 说 : 没 有 品 格 , 想 是 大 的 极 点 啊 众 人 说 : 不 大 , 不 大 , 只 叫 做 未 入 流 。
猴 王 说 : 你 怎 么 叫 做 未 入 流 ?
这 样 的 官 家 儿 子 最 低 贱 最 小 , 只 可 以 给 他 看 马 。
像 堂 尊 到 任 之 后 , 这 些 , 喂 得 马 肥 , 只 落 得 道 声 好 字 ; 如 果 稍 有 些 瘦 弱 , 还 要 受 到 责 罚 ; 再 有 十 分 伤 损 , 还 要 罚 赎 问 罪 。
猴 王 听 了 这 些 话 , 不 知 不 觉 心 头 火 起 , 咬 着 牙 大 怒 说 : 这 样 鄙 视 老 孙 , 老 孙 在 那 花 果 山 称 王 称 祖 , 为 什 么 唆 我 来 替 他 养 马 呢 ? 养 马 的 人 , 是 后 辈 卑 贱 之 辈 , 难 道 是 等 我 的 吗 ? 不 做 他 , 不 做 他 , 我 将 去 。
忽 然 发 出 一 声 , 把 公 案 推 倒 , 从 耳 里 取 出 一 个 宝 贝 , 打 一 个 账 子 , 碗 子 粗 细 , 一 路 解 开 数 字 , 直 接 打 出 御 马 监 , 径 直 到 了 南 天 门 。
众 天 丁 知 道 他 受 了 仙 丹 , 乃 是 个 弼 马 温 , 不 敢 阻 挡 , 让 他 打 出 天 门 去 了 。
不 一 会 儿 , 他 就 把 云 头 按 下 来 , 回 到 花 果 山 上 。
只 见 四 个 健 将 和 各 个 洞 中 的 妖 王 , 在 那 里 操 练 兵 士 。
猴 王 厉 声 大 叫 道 : 小 的 人 , 老 孙 来 了 。
一 群 猴 子 都 来 叩 头 , 迎 接 他 进 入 洞 天 深 处 , 请 猴 王 高 登 宝 座 , 在 一 个 墙 壁 中 设 酒 迎 风 。
都 说 : 恭 喜 大 王 , 上 界 去 十 几 年 , 想 一 定 会 得 意 荣 回 来 了 。 猴 王 说 : 我 才 半 个 月 多 , 哪 里 有 十 几 年 ? 众 猴 子 说 : 大 王 , 你 在 天 上 不 知 道 时 间 。
天 上 一 日 , 就 是 下 界 一 年 哩 。
请 问 大 王 , 你 担 任 什 么 官 职 ? 猴 王 摇 着 手 说 : 不 好 说 , 不 好 说 , 活 活 羞 耻 杀 人 。
玉 帝 不 会 用 人 , 他 见 到 老 孙 这 样 的 模 样 , 封 我 做 什 么 弼 马 温 , 原 来 是 与 他 养 马 , 没 有 进 入 品 级 之 类 。
我 刚 到 任 时 不 知 道 , 只 在 御 马 监 中 狂 耍 。
到 今 天 问 我 同 僚 , 才 知 道 是 这 样 卑 贱 。
老 子 心 中 非 常 恼 怒 , 把 他 推 倒 在 席 上 , 不 接 受 官 职 , 因 此 就 走 下 来 了 。
众 猴 说 : 来 得 好 , 来 得 好 。
大 王 在 这 个 福 地 洞 天 的 地 方 , 为 王 , 多 少 尊 重 快 乐 , 怎 么 会 去 和 他 们 做 马 夫 呢 ? 教 小 的 人 快 点 酒 来 , 和 大 王 解 闷 。
正 在 饮 酒 欢 宴 的 时 候 , 有 人 来 报 告 说 : 大 王 , 门 外 有 两 个 独 角 的 鬼 王 , 要 见 大 王 。
猴 王 说 : 让 他 进 来 。
那 个 鬼 王 整 理 衣 服 跑 进 洞 中 , 倒 身 下 拜 。
美 猴 王 问 他 : 你 看 见 我 有 什 么 样 的 ? 鬼 王 说 : 久 闻 大 王 招 贤 , 没 有 机 会 见 到 您 , 现 在 见 大 王 授 给 我 天 , 得 意 荣 归 , 特 地 献 上 赭 黄 袍 一 件 , 与 大 王 祝 贺 。
肯 不 抛 弃 鄙 贱 之 人 , 收 纳 小 人 , 也 可 以 效 犬 马 之 劳 。
猴 王 大 喜 , 把 赭 黄 袍 穿 起 来 。
众 人 欣 然 排 列 朝 拜 。
立 即 将 鬼 王 封 为 前 部 总 督 先 锋 。
鬼 王 谢 恩 后 , 又 启 奏 说 : 大 王 在 天 上 已 经 很 久 了 , 授 给 什 么 官 职 ? 猴 王 说 : 玉 帝 轻 视 贤 能 , 封 我 做 什 么 弼 马 温 ?
鬼 王 听 了 , 又 上 奏 说 : 大 王 有 这 样 的 神 通 , 怎 么 和 他 饲 养 马 匹 , 就 做 个 齐 天 大 圣 , 有 什 么 不 可 以 ? 猴 王 听 了 , 欢 喜 不 止 , 连 续 说 了 几 个 好 , 好 好 , 教 四 位 健 将 说 : 你 们 快 点 点 旌 旗 , 旗 上 写 着 齐 天 大 圣 四 个 大 字 , 立 竿 张 挂 。
从 此 以 后 , 只 称 我 为 齐 天 大 圣 , 不 许 再 称 大 王 。
也 可 以 传 给 各 洞 的 妖 王 , 一 个 人 都 知 道 了 。
说 完 就 不 在 话 下 了 。
又 说 : 那 玉 帝 第 二 天 设 朝 , 只 见 张 天 师 引 御 马 监 、 监 丞 、 监 副 在 丹 墀 下 拜 奏 道 : 万 岁 , 新 任 辅 佐 马 温 、 孙 悟 空 , 因 为 嫌 官 小 , 昨 天 反 而 下 天 宫 去 了 。
正 在 说 完 , 又 看 见 南 天 门 外 增 长 天 王 领 众 天 丁 , 也 上 奏 说 : 弼 马 温 不 知 道 什 么 缘 故 , 跑 出 天 门 去 了 。
玉 帝 听 到 这 话 , 立 即 传 旨 说 : 著 有 两 路 神 元 , 各 归 本 职 。
我 派 遣 天 兵 , 擒 拿 这 怪 物 。
班 部 中 闪 呈 上 托 塔 李 天 王 和 何 吒 三 个 太 子 , 越 班 奏 报 皇 上 说 : 万 岁 , 微 臣 没 有 才 能 , 请 圣 旨 降 此 妖 怪 。
玉 帝 大 喜 , 立 即 封 托 塔 天 王 李 靖 为 降 魔 大 元 帅 , 任 何 吒 、 三 太 子 为 三 坛 海 会 大 神 , 立 刻 兴 师 下 界 。
李 天 王 与 何 吒 叩 头 谢 罪 , 径 直 到 了 自 己 的 宫 殿 , 何 点 率 领 三 军 , 率 领 众 多 头 目 , 戴 着 巨 灵 神 为 先 锋 , 鱼 肚 将 要 抢 夺 后 面 , 药 叉 将 要 催 促 军 队 。
一 会 儿 出 了 南 天 门 外 , 径 直 来 到 花 果 山 , 选 择 平 阳 处 安 营 寨 , 传 令 让 巨 灵 神 挑 战 。
大 灵 神 得 到 命 令 , 装 好 整 齐 , 拿 着 宣 花 斧 , 来 到 水 帘 洞 外 。
只 见 小 洞 门 外 有 许 多 妖 魔 , 都 是 狼 虫 虎 豹 之 类 , 丫 叉 叉 叉 , 拿 着 刀 剑 , 在 那 里 跳 舞 , 大 声 吼 叫 。
这 个 巨 灵 神 喝 道 : 这 个 业 畜 , 快 点 去 报 告 与 弼 、 马 温 知 道 道 : 我 是 上 天 的 大 将 , 奉 玉 帝 的 旨 意 , 到 这 里 来 收 伏 , 让 他 早 早 出 来 接 受 投 降 , 以 免 让 你 们 都 伤 残 了 。
猴 王 问 : 有 什 么 祸 事 ? 众 妖 说 : 门 外 有 一 员 天 将 , 口 称 大 圣 的 官 衔 , 道 奉 玉 帝 圣 旨 , 来 此 收 伏 。
教 你 们 早 早 出 去 接 受 投 降 , 不 要 伤 害 我 们 的 性 命 。
猴 王 听 了 , 教 他 说 : 拿 我 的 披 挂 来 。
于 是 戴 上 紫 金 冠 , 穿 上 黄 金 甲 , 登 上 步 云 鞋 , 手 里 拿 着 如 意 金 铃 棒 , 领 着 众 人 出 门 , 摆 开 阵 势 。
这 个 巨 灵 神 , 睁 开 眼 睛 观 看 , 真 是 猴 王 。 他 身 穿 金 甲 光 亮 堂 堂 , 头 戴 金 冠 , 光 辉 照 耀 。
他 手 里 举 着 一 根 金 钳 棒 , 脚 踏 着 云 鞋 , 都 相 称 。
有 一 双 奇 怪 的 眼 睛 象 明 星 , 两 只 耳 朵 超 过 肩 膀 又 硬 。
挺 挺 的 身 体 才 能 , 变 化 很 多 , 声 音 响 亮 如 钟 磬 。
尖 嘴 咨 牙 弼 马 温 , 心 高 要 做 齐 天 圣 。
巨 灵 神 厉 声 高 叫 道 : 那 泼 猴 , 你 认 得 我 吗 ? 大 圣 听 了 , 急 忙 问 道 : 你 是 那 路 毛 神 , 老 孙 不 曾 会 见 你 , 你 快 报 名 来 。
巨 灵 神 说 : 我 把 你 那 个 欺 心 的 , 你 是 认 不 到 我 。
我 是 高 上 神 霄 托 塔 李 天 王 部 下 的 先 锋 巨 灵 天 将 , 现 在 奉 玉 帝 圣 旨 , 到 此 收 降 你 。
你 快 脱 下 装 束 , 归 顺 天 恩 , 免 得 这 满 山 的 牲 畜 被 杀 ; 如 果 说 半 个 不 字 , 让 你 立 刻 就 会 变 成 粉 。
猴 王 听 了 , 心 里 勃 然 大 怒 , 说 : 泼 毛 神 , 不 要 夸 大 嘴 , 稍 稍 玩 弄 长 舌 。
我 本 来 等 待 一 棒 打 死 你 , 恐 怕 没 有 人 去 报 信 。
暂 且 留 下 你 的 性 命 , 快 点 回 天 , 对 玉 皇 说 : 他 很 不 用 贤 。
你 看 我 的 旌 旗 上 的 字 号 , 如 果 凭 着 这 个 字 名 升 官 , 我 就 不 动 刀 兵 , 自 然 的 天 地 清 泰 ; 如 果 不 依 , 时 间 就 打 上 灵 霄 宝 殿 , 教 他 龙 床 定 坐 不 成 。
这 个 巨 灵 神 听 到 这 话 , 急 忙 睁 开 眼 睛 迎 风 观 看 , 果 然 看 见 门 外 竖 着 一 根 高 竿 , 竿 上 有 一 面 旌 旗 , 上 面 写 着 齐 天 大 圣 四 个 大 字 。
巨 灵 神 冷 笑 了 三 声 说 : 这 泼 猴 , 这 些 不 知 人 事 , 竟 敢 没 有 表 现 , 你 就 要 做 齐 天 大 圣 。
好 好 吃 我 一 把 斧 子 。
说 完 就 把 他 打 开 了 。
那 猴 王 正 是 会 家 不 忙 , 拿 着 金 钳 棒 应 手 迎 接 。
这 一 场 好 杀 : 棒 子 叫 如 意 , 斧 头 叫 宣 花 。
其 他 两 个 人 刚 刚 相 遇 , 不 知 道 他 们 的 深 浅 。
一 个 人 暗 藏 神 妙 , 一 个 人 大 口 夸 夸 。
让 他 动 手 , 就 会 喷 出 云 雾 , 展 开 手 , 播 土 扬 沙 。
天 将 神 通 就 有 道 , 猴 王 变 化 实 无 边 。
棒 子 举 起 来 就 像 龙 戏 水 一 样 , 斧 头 砍 开 来 就 像 凤 穿 花 一 样 。
巨 灵 的 名 望 传 遍 天 下 , 原 来 的 本 事 不 如 他 。 大 圣 轻 轻 地 踢 铁 棒 , 头 一 下 , 满 身 麻 麻 。
大 灵 神 抵 抗 不 住 , 被 猴 王 劈 开 头 一 棒 , 慌 忙 把 斧 头 架 起 来 隔 开 , 一 声 一 声 , 把 斧 柄 打 成 两 截 , 急 忙 撤 身 败 阵 逃 生 。
猴 王 笑 着 说 : 脓 包 , 脓 包 。
我 已 经 饶 了 你 , 你 快 去 报 答 我 , 快 去 报 答 我 。
大 灵 神 回 到 营 门 , 径 直 见 到 托 塔 天 王 , 忙 哈 哈 跪 下 说 : 弼 马 温 果 然 是 神 通 广 大 , 最 后 打 败 仗 回 来 请 罪 。
李 天 王 发 怒 说 : 这 个 人 挫 我 的 锐 气 , 把 他 推 出 来 斩 首 ! 旁 边 闪 出 了 何 吒 , 太 子 拜 告 说 : 父 王 息 怒 , 暂 且 宽 恕 巨 灵 的 罪 过 。
等 孩 子 出 师 一 遍 , 就 知 道 深 浅 。
天 王 听 了 劝 谏 , 并 且 教 他 回 营 待 罪 管 事 。
这 个 叫 吒 太 子 , 穿 着 铠 甲 , 跳 出 营 盘 , 撞 到 水 帘 洞 外 。
但 是 , 他 们 正 好 收 兵 , 看 见 何 吒 来 的 勇 猛 。
好 太 子 说 : 总 角 才 遮 住 了 , 披 毛 未 盖 肩 。
神 奇 多 聪 慧 , 骨 骼 秀 美 更 加 清 新 美 丽 。
果 真 是 天 上 麒 麟 子 , 果 真 是 烟 霞 彩 凤 仙 。
龙 种 自 然 不 是 世 俗 的 相 貌 , 妙 龄 端 不 像 尘 世 俗 人 。
身 带 六 种 神 器 械 , 飞 腾 变 化 广 无 边 。
现 在 接 受 玉 皇 金 口 诏 书 , 敕 封 海 会 号 为 三 坛 。
悟 空 迎 接 他 走 近 前 来 问 道 : 你 是 谁 家 的 小 哥 , 闯 近 我 家 门 , 有 什 么 事 做 ? 何 吒 喝 道 : 泼 妖 猴 , 难 道 不 认 得 我 吗 ? 我 是 托 塔 天 王 三 太 子 何 吒 , 现 在 奉 玉 帝 钦 差 , 到 这 里 捉 你 。
悟 空 笑 着 说 : 小 太 子 , 你 的 牙 齿 尚 未 脱 落 , 胎 毛 还 没 有 干 , 怎 么 敢 说 这 样 大 话 , 我 暂 且 留 你 的 性 命 , 不 打 你 。
你 只 看 我 的 旌 旗 上 是 什 么 字 号 , 拜 上 玉 帝 说 : 是 这 样 的 官 衔 , 再 也 不 必 动 众 , 我 自 己 皈 依 , 如 果 不 能 成 就 我 的 心 愿 , 一 定 要 打 上 灵 霄 宝 殿 。
何 吒 抬 头 看 的 地 方 , 是 齐 天 大 圣 四 个 字 。
何 吒 说 : 这 个 妖 猴 能 有 多 大 神 通 , 竟 敢 称 这 个 名 号 , 不 要 害 怕 , 吃 我 一 把 剑 。
悟 空 说 : 我 只 站 在 下 面 不 动 , 任 凭 你 砍 几 把 剑 罢 了 。
那 么 , 那 里 吒 得 大 怒 , 大 喝 一 声 , 大 叫 : 变 , 立 即 变 成 三 头 六 臂 , 恶 恶 凶 狠 , 手 里 拿 着 六 种 兵 器 , 是 斩 妖 剑 、 杀 妖 刀 、 缚 妖 索 、 降 妖 杵 、 绣 儿 、 火 轮 儿 , 丫 丫 叉 叉 , 扑 面 来 打 。
悟 空 见 了 , 心 惊 地 说 : 这 个 小 哥 , 倒 也 会 弄 些 手 段 。
不 要 无 礼 , 看 我 神 通 。
大 圣 , 喝 道 : 变 , 变 成 三 头 六 臂 , 拿 金 钳 棒 打 一 伞 , 也 变 成 三 条 。
六 只 手 拿 着 三 条 棒 子 。
这 一 场 战 斗 , 真 是 地 震 山 摇 , 好 杀 。 六 臂 何 吒 太 子 , 天 生 美 石 猴 王 , 相 逢 真 对 手 , 正 遇 本 源 流 。
那 一 个 蒙 差 来 下 界 , 这 一 个 欺 心 闹 斗 牛 。
斩 妖 宝 剑 锋 芒 锐 利 , 砍 妖 刀 锐 利 , 鬼 神 愁 苦 ; 缚 妖 索 子 如 飞 蟒 , 降 妖 大 杵 如 狼 头 ; 火 轮 闪 闪 闪 闪 闪 闪 闪 闪 闪 闪 闪 闪 闪 闪 闪 闪 闪 闪 闪 闪 闪 闪 闪 闪 闪 闪 闪 闪 闪 闪 闪 闪 闪 闪 闪 闪 闪 闪 闪 闪 闪 闪 闪 闪 , 往 往 来 来 跳 锦 。
大 圣 三 条 如 意 棒 , 前 面 挡 住 后 面 挡 住 , 运 用 机 谋 。
苦 苦 争 辩 几 次 , 没 有 高 下 , 太 子 心 中 不 肯 休 息 。
把 那 六 件 兵 器 多 教 化 变 化 , 百 千 万 亿 照 头 丢 掉 。
猴 王 不 害 怕 , 哈 哈 笑 , 铁 棒 翻 腾 自 己 运 筹 。
以 一 化 千 千 , 千 化 万 , 满 空 乱 舞 赛 飞 。
到 了 各 洞 的 妖 王 都 关 上 门 , 遍 山 的 鬼 怪 都 藏 在 头 上 。
神 兵 怒 气 云 气 浓 浓 , 金 铃 铁 棒 响 得 冷 冷 。
那 墙 厢 , 天 丁 嚎 叫 人 人 害 怕 ; 这 墙 厢 , 猴 怪 摇 旗 , 每 个 个 忧 愁 。
发 怒 , 两 家 同 斗 勇 , 不 知 道 那 个 刚 强 , 那 个 柔 柔 。
三 太 子 和 悟 空 各 自 发 挥 神 威 , 打 了 三 十 回 合 。
那 么 太 子 的 六 种 兵 器 , 变 成 千 千 万 万 千 千 万 万 ; 孙 子 的 金 铃 棒 , 变 成 千 万 千 千 万 千 千 。
半 空 中 好 像 雨 点 流 星 , 不 分 胜 负 。
原 来 , 悟 空 手 有 疾 , 眼 有 快 , 正 在 混 乱 的 时 候 , 他 拔 下 一 根 毫 毛 , 大 叫 道 : 变 , 就 变 成 了 他 的 本 来 相 貌 , 手 里 拿 着 棒 子 , 表 演 着 何 吒 。 他 的 真 实 身 子 , 却 一 纵 一 纵 , 赶 到 了 何 吒 的 脑 后 , 拿 着 他 的 左 臂 上 一 棒 子 打 来 。
何 吒 正 在 用 法 术 时 , 听 到 棒 头 风 声 , 急 忙 逃 走 时 , 不 能 放 手 , 被 他 拽 了 一 下 , 背 着 痛 痛 逃 走 。
收 兵 之 法 , 以 六 件 兵 器 依 旧 回 来 , 败 阵 而 回 。
那 阵 上 李 天 王 早 已 看 见 了 , 急 忙 想 带 兵 助 战 , 不 知 不 觉 太 子 突 然 来 到 面 前 , 惊 慌 地 报 告 说 : 父 王 , 弼 马 温 真 是 有 本 事 , 孩 子 这 样 的 法 力 , 也 打 不 过 , 已 被 他 打 伤 了 胳 膊 。
天 王 大 惊 失 色 , 说 : 这 个 人 你 的 神 通 , 怎 么 能 取 得 胜 利 呢 太 子 说 : 他 洞 门 外 竖 着 一 竿 旗 , 上 面 写 着 齐 天 大 圣 四 个 字 。
亲 口 夸 称 , 让 玉 帝 封 他 为 齐 天 大 圣 , 万 事 都 罢 了 , 如 果 还 不 是 这 个 称 号 , 一 定 要 打 上 灵 霄 宝 殿 了 。
天 王 说 : 既 然 如 此 , 暂 且 不 要 和 他 们 相 持 , 暂 且 离 开 上 界 , 将 此 话 回 奏 皇 上 , 再 多 派 天 兵 , 围 捉 这 个 人 , 不 算 迟 。
太 子 感 到 痛 苦 , 不 能 再 战 , 所 以 同 天 王 回 天 启 奏 没 有 题 目 。
你 看 那 猴 王 得 胜 归 山 , 那 七 十 二 洞 的 妖 王 和 那 六 个 弟 兄 , 都 来 祝 贺 喜 庆 , 在 洞 天 福 地 , 饮 酒 快 乐 无 比 。
又 对 六 弟 兄 说 : 小 弟 既 称 为 齐 天 大 圣 , 你 们 也 可 以 称 为 大 圣 。
里 面 有 个 牛 魔 王 突 然 高 声 喊 道 : 贤 弟 说 得 有 理 , 我 就 称 他 为 平 天 大 圣 。
蛟 魔 王 说 : 我 叫 做 覆 海 大 圣 。
鹏 魔 王 说 : 我 称 他 为 混 天 大 圣 。
狮 王 说 : 我 称 为 移 山 大 圣 。
侏 猴 王 说 : 我 叫 通 风 大 圣 。
王 说 : 我 称 为 驱 神 大 圣 。
这 时 七 位 大 圣 人 自 作 自 作 , 自 称 自 号 , 玩 乐 一 天 , 各 自 散 去 。
又 说 : 那 李 天 王 和 三 位 太 子 , 率 领 众 将 , 直 到 灵 霄 宝 殿 , 启 奏 道 : 臣 等 奉 圣 旨 出 师 下 界 , 收 伏 妖 仙 孙 悟 空 , 不 料 他 的 神 通 广 大 , 不 能 取 胜 , 仍 希 望 万 岁 增 兵 剿 灭 。
玉 帝 说 : 谅 是 一 个 妖 猴 , 有 多 少 本 事 , 还 要 添 兵 , 太 子 又 近 前 上 奏 说 : 希 望 万 岁 赦 免 我 的 死 罪 。
那 妖 猴 使 用 一 根 铁 棒 , 先 打 败 了 巨 灵 神 , 又 打 伤 了 我 的 胳 膊 。
洞 门 外 竖 立 一 竿 旗 , 上 面 写 着 齐 天 大 圣 四 个 字 。
如 果 不 是 这 个 官 职 , 还 要 打 上 灵 霄 宝 殿 。
玉 帝 听 了 这 话 , 惊 讶 地 说 : 这 个 妖 猴 , 怎 么 敢 这 样 狂 妄 , 现 在 众 将 立 即 杀 掉 它 。
正 在 说 完 , 班 部 中 又 闪 出 金 星 , 上 奏 说 : 那 妖 猴 只 知 道 说 话 , 不 知 道 大 小 。
想 要 增 加 兵 力 与 其 他 人 争 斗 , 想 一 时 之 间 不 能 收 服 埋 伏 , 反 而 又 使 军 队 疲 劳 。
不 如 万 岁 大 舍 恩 慈 , 还 降 招 安 的 旨 意 , 就 教 他 做 齐 天 大 圣 。
只 是 加 他 个 空 衔 , 有 官 没 有 俸 禄 就 行 了 。
玉 帝 说 : 你 怎 么 叫 做 有 官 无 禄 ? 金 星 说 : 我 的 名 字 是 齐 天 大 圣 , 只 不 给 他 的 事 , 不 给 他 的 俸 禄 , 而 且 养 在 天 地 之 间 , 收 取 他 的 邪 心 , 使 他 不 生 狂 妄 妄 想 , 希 望 天 地 安 宁 , 海 宇 得 到 清 宁 。
玉 帝 听 了 , 说 道 : 依 你 所 奏 。
即 命 下 诏 书 , 金 星 带 去 。
金 星 又 出 了 南 天 门 , 直 到 花 果 山 水 帘 洞 外 观 看 。
这 一 次 与 以 前 不 同 , 威 风 凛 凛 , 杀 气 森 森 , 各 种 妖 精 , 没 有 什 么 不 有 。
其 所 以 不 得 也 。
一 见 到 金 星 , 都 上 前 动 手 。
金 星 说 : 那 些 头 目 来 , 拖 累 你 去 报 答 你 。 大 圣 知 道 这 件 事 , 我 是 上 帝 派 来 的 天 使 , 有 圣 旨 在 此 请 求 他 。
众 妖 立 即 跑 进 去 报 告 说 : 外 面 有 一 个 老 人 , 他 说 是 上 界 的 天 使 , 有 旨 意 请 你 。
悟 空 说 : 来 得 好 , 来 得 好 。
想 是 前 番 来 的 那 太 白 金 星 。
其 次 请 我 上 界 , 虽 然 是 官 爵 不 能 胜 任 , 却 又 在 天 上 走 了 一 次 , 认 得 那 天 门 内 外 的 路 。
现 在 你 们 又 来 了 , 一 定 有 好 意 。
令 众 头 目 大 开 旗 鼓 , 摆 开 队 伍 迎 接 。
大 圣 立 即 带 领 群 猴 , 戴 上 帽 子 穿 上 铠 甲 , 铠 甲 上 罩 着 赭 黄 袍 , 脚 踏 着 云 履 , 急 忙 出 洞 门 , 亲 自 施 礼 , 高 声 大 叫 道 : 老 星 请 进 , 恕 我 失 去 迎 接 的 罪 过 。
金 星 快 步 向 前 走 , 径 直 进 入 洞 内 , 面 朝 南 站 着 说 道 : 现 在 告 诉 大 圣 说 : 以 前 因 为 大 圣 嫌 恶 官 小 , 逃 离 御 马 监 , 当 有 本 监 中 的 大 小 官 员 奏 报 了 玉 帝 。
玉 帝 传 旨 说 : 凡 授 官 职 , 都 是 由 卑 下 而 尊 , 为 什 么 嫌 小 呢 就 有 李 天 王 领 着 何 吒 下 界 打 仗 。
不 知 道 大 圣 神 通 , 所 以 遭 到 败 北 , 回 天 奏 道 : 大 圣 树 一 竿 旗 , 要 做 齐 天 大 圣 。
众 武 将 回 来 要 支 吾 , 这 是 老 汉 的 力 量 替 大 圣 冒 罪 上 奏 , 免 除 兴 师 动 众 , 请 大 王 授 予 。
玉 帝 听 从 了 他 的 奏 章 , 因 此 来 请 求 。
悟 空 笑 着 说 : 前 番 劳 苦 , 今 天 又 受 到 宠 爱 , 多 谢 , 但 不 知 道 上 天 可 给 我 齐 天 大 圣 的 官 衔 , 金 星 说 : 老 汉 以 此 官 衔 奏 准 , 我 才 敢 领 旨 而 来 , 如 果 有 不 成 功 , 只 是 坐 罪 老 汉 就 是 。
悟 空 大 喜 , 恳 请 留 他 饮 酒 喝 酒 , 不 肯 , 于 是 和 金 星 放 着 祥 云 , 来 到 南 天 门 外 。
天 下 之 人 , 天 下 之 人 , 都 拱 手 迎 接 。
径 直 进 入 灵 霄 殿 下 。
金 星 拜 奏 说 : 我 奉 诏 命 宣 布 朝 政 , 马 温 、 孙 悟 空 已 经 到 了 。
玉 帝 说 : 那 个 孙 子 悟 空 过 来 , 现 在 宣 扬 你 做 个 齐 天 大 圣 , 官 位 品 级 到 了 极 点 , 但 切 不 可 以 干 什 么 。
这 个 猴 子 也 只 在 朝 上 高 声 喊 , 喊 道 : 谢 恩 。
玉 帝 就 命 令 工 干 官 张 、 鲁 二 班 , 在 蟠 桃 园 右 边 , 建 造 一 座 齐 天 大 圣 府 , 府 内 设 置 两 个 衙 门 : 一 个 叫 安 静 司 , 一 个 叫 宁 神 司 。
司 都 有 仙 吏 , 左 右 扶 持 。
又 派 五 斗 星 君 送 悟 空 到 任 , 另 外 赐 给 御 酒 二 瓶 、 金 花 十 朵 , 表 明 他 们 安 心 定 志 , 再 也 不 要 干 什 么 。
那 猴 王 信 服 奉 行 , 当 天 与 五 斗 星 君 到 府 中 , 打 开 酒 瓶 , 和 众 人 一 起 喝 尽 酒 。
送 星 官 回 转 本 宫 , 其 他 人 就 心 满 意 满 , 喜 地 欢 天 , 在 天 宫 快 乐 , 无 挂 无 碍 。
这 正 是 : 仙 名 永 远 注 入 长 生 , 不 堕 于 世 间 万 古 传 。
最 终 不 知 道 以 后 怎 么 样 , 暂 且 听 下 回 分 解 。
}\switchcolumn\flushpage  \begin{pinyinscope}{\myfontt \section{第五回}     亂蟠桃大聖偷丹 反天宮諸神捉怪

話表齊天大聖到底是個妖猴,更不知官銜品從,也不較俸祿高低,但只註名便
了。那齊天府下二司仙吏,早晚伏侍,只知日食三餐,夜眠一榻,無事牽縈,
自由自在。閑時節會友遊宮,交朋結義。見三清稱個「老」字,逢四帝道個
「陛下」。與那九曜星、五方將、二十八宿、四大天王、十二元辰、五方五老
、普天星相、河漢群神,俱只以弟兄相待,彼此稱呼。今日東遊,明日西蕩,
雲去雲來,行蹤不定。

一日,玉帝早朝,班部中閃出許旌陽真人,頫?啟奏道:「今有齊天大聖無事
閑遊,結交天上眾星宿,不論高低,俱稱朋友,恐後閑中生事。不若與他一件
事管,庶免別生事端。」玉帝聞言,即時宣詔。那猴王欣欣然而至,道:「陛
下,詔老孫有何陞賞?」玉帝道:「朕見你身閑無事,與你件執事:你且權管
那蟠桃園,早晚好生在意。」大聖歡喜謝恩,朝上唱喏而退。

他等不得窮忙,即入蟠桃園內查勘。本園中有個土地攔住問道:「大聖何往?」
大聖道:「吾奉玉帝點差,代管蟠桃園,今來查勘也。」那土地連忙施禮,即
呼那一班鋤樹力士、運水力士、修桃力士、打掃力士都來見大聖磕頭,引他進
去。但見那:
夭夭灼灼,顆顆株株。夭夭灼灼花盈樹,顆顆株株果壓枝。果壓枝頭垂錦彈﹔
花盈樹上簇胭脂。時開時結千年熟,無夏無冬萬載遲。先熟的,酡顏醉臉﹔還
生的,帶蒂青皮。凝煙肌帶綠,映日顯丹姿。樹下奇葩並異卉,四時不謝色齊
齊﹔左右樓臺並館舍,盈空常見罩雲霓。不是玄都凡俗種,瑤池王母自栽培。

大聖看玩多時,問土地道:「此樹有多少株數?」土地道:「有三千六百株:
前面一千二百株,花微果小,三千年一熟,人吃了成仙了道,體健身輕﹔中間
一千二百株,層花甘實,六千年一熟,人吃了霞舉飛昇,長生不老﹔後面一千
二百株,紫紋緗核,九千年一熟,人吃了與天地齊壽,日月同庚。」大聖聞言
,歡喜無任。當日查明了株樹,點看了亭閣,回府。自此後,三五日一次賞玩
,也不交友,也不他遊。

一日,見那老樹枝頭,桃熟大半。他心裏要吃個嘗新,奈何本園土地、力士並
齊天府仙吏緊隨不便。忽設一計道:「汝等且出門外伺候,讓我在這亭上少憩
片時。」那眾仙果退。只見那猴王脫了冠服,爬上大樹,揀那熟透的大桃,摘
了許多,就在樹枝上自在受用。吃了一飽,卻才跳下樹來,簪冠著服,喚眾等
儀從回府。遲三二日,又去設法偷桃,儘他享用。

一朝,王母娘娘設宴,大開寶閣,瑤池中做蟠桃勝會。即著那紅衣仙女、青衣
仙女、素衣仙女、皂衣仙女、紫衣仙女、黃衣仙女、綠衣仙女各頂花籃,去蟠
桃園摘桃建會。七衣仙女直至園門首,只見蟠桃園土地、力士同齊天府二司仙
吏,都在那裏把門。仙女近前道:「我等奉王母懿旨,到此摘桃設宴。」土地
道:「仙娥且住。今歲不比往年了,玉帝點差齊天大聖在此督理,須是報大聖
得知,方敢開園。」仙女道:「大聖何在?」土地道:「大聖在園內,因困倦
,自家在亭子上睡哩。」仙女道:「既如此,尋他去來,不可遲誤。」土地即
與同進。尋至花亭不見,只有衣冠在亭,不知何往,四下裏都沒尋處。原來大
聖耍了一會,吃了幾個桃子,變做二寸長的個人兒,在那大樹梢頭濃葉之下睡
著了。七衣仙女道:「我等奉旨前來,尋不見大聖,怎敢空回?」傍有仙使道
:「仙娥既奉旨來,不必遲疑。我大聖閑遊慣了,想是出園會友去了。汝等且
去摘桃,我們替你回話便是。」

那仙女依言,入樹林之下摘桃:先在前樹摘了二籃,又在中樹摘了三籃,到後
樹上摘取,只見那樹上花果稀疏,止有幾個毛蒂青皮的。原來熟的都是猴王吃
了。七仙女張望東西,只見向南枝上止有一個半紅半白的桃子。青衣女用手扯
下枝來,紅衣女摘了,卻將枝子望上一放。

原來那大聖變化了,正睡在此枝,被他驚醒。大聖即現本相,耳朵內掣出金箍
棒,幌一幌,碗來粗細,咄的一聲道:「你是那方怪物,敢大膽偷摘我桃。」
慌得那七仙女一齊跪下道:「大聖息怒。我等不是妖怪,乃王母娘娘差來的七
衣仙女,摘取仙桃,大開寶閣,做蟠桃勝會。適至此間,先見了本園土地等神
,尋大聖不見。我等恐遲了王母懿旨,是以等不得大聖,故先在此摘桃。萬望
恕罪。」大聖聞言,回嗔作喜道:「仙娥請起。王母開閣設宴,請的是誰?」
仙女道:「上會自有舊規,請的是西天佛老、菩薩、聖僧、羅漢,南方南極觀
音,東方崇恩聖帝、十洲三島仙翁,北方北極玄靈,中央黃極黃角大仙,這個
是五方五老。還有五斗星君,上八洞三清、四帝、太乙天仙等眾,中八洞玉皇
、九壘、海嶽神仙,下八洞幽冥教主、注世地仙,各宮各殿大小尊神,俱一齊
赴蟠桃嘉會。」大聖笑道:「可請我麼?」仙女道:「不曾聽得說。」大聖道
:「我乃齊天大聖,就請我老孫做個席尊,有何不可?」仙女道:「此是上會
舊規,今會不知如何。」大聖道:「此言也是,難怪汝等。你且立下,待老孫
先去打聽個消息,看可請老孫不請。」

好大聖,捻著訣,念聲咒語,對眾仙女道:「住!住!住!」這原來是個定身
法,把那七衣仙女,一個個睖睖睜睜,白著眼,都站在桃樹之下。大聖縱朵祥
雲,跳出園內,竟奔瑤池路上而去。正行時,只見那壁廂:
一天瑞靄光搖曳,五色祥雲飛不絕。白鶴聲鳴振九皋,紫芝色秀分千葉。中間
現出一尊仙,相貌天然丰采別。神舞虹霓幌漢霄,腰懸寶籙無生滅。名稱赤腳
大羅仙,特赴蟠桃添壽節。那赤腳大仙覿面撞見大聖,大聖低頭定計,賺哄真
仙,他要暗去赴會,卻問:「老道何往?」大仙道:「蒙王母見招,去赴蟠桃
嘉會。」大聖道:「老道不知。玉帝因老孫觔斗雲疾,著老孫五路邀請列位,
先至通明殿下演禮,後方去赴宴。」大仙是個光明正大之人,就以他的誑語作
真,道:「常年就在瑤池演禮謝恩,如何先去通明殿演禮,方去瑤池赴會?」
無奈,只得撥轉祥雲,徑往通明殿去了。

大聖駕著雲,念聲咒語,搖身一變,就變做赤腳大仙模樣,前奔瑤池。不多時
,直至寶閣,按住雲頭,輕輕移步,走入裏面。只見那裏:
瓊香繚繞,瑞靄繽紛。瑤臺鋪彩結,寶閣散氤氳。鳳翥鸞翔形縹緲,金花玉萼
影浮沉。上排著九鳳丹霞扆,八寶紫霓墩,五彩描金桌,千花碧玉盆。桌上有
龍肝和鳳髓,熊掌與猩唇。珍饈百味般般美,異果嘉殽色色新。

那裏鋪設得齊齊整整,卻還未有仙來。

這大聖點看不盡,忽聞得一陣酒香撲鼻。忽轉頭,見右壁廂長廊之下,有幾個
造酒的仙官、盤糟的力士,領幾個運水的道人、燒火的童子,在那裏洗缸刷甕
,已造成了玉液瓊漿,香醪佳釀。大聖止不住口角流涎,就要去吃,奈何那些
人都在這裏。他就弄個神通,把毫毛拔下幾根,丟入口中嚼碎,噴將出去,念
聲咒語,叫:「變!」即變做幾個瞌睡蟲,奔在眾人臉上。你看那夥人,手軟
頭低,閉眉合眼,丟了執事,都去盹睡。大聖卻拿了些百味八珍,佳殽異品,
走入長廊裏面,就著缸,挨著甕,放開量,痛飲一番。吃勾了多時,酕醄醉了
。自揣自摸道:「不好,不好!再過會,請的客來,卻不怪我?一時拿住,怎
生是好?不如早回府中睡去也。」

好大聖,搖搖擺擺,仗著酒,任情亂撞。一會把路差了,不是齊天府,卻是兜
率天宮。一見了,頓然醒悟道:「兜率宮是三十三天之上,乃離恨天太上老君
之處,如何錯到此間?也罷,也罷,一向要來望此老,不曾得來,今趁此殘步
,就望他一望也好。」即整衣撞進去,那裏不見老君,四無人跡。原來那老君
與燃燈古佛在三層高閣朱陵丹臺上講道,眾仙童、仙將、仙官、仙吏都侍立左
右聽講。這大聖直至丹房裏面,尋訪不遇。但見丹灶之傍,爐中有火。爐左右
安放著五個葫蘆,葫蘆裏都是煉就的金丹。大聖喜道:「此物乃仙家之至寶。
老孫自了道以來,識破了內外相同之理,也要煉些金丹濟人,不期到家無暇。
今日有緣,卻又撞著此物。趁老子不在,等我吃他幾丸嘗新。」他就把那葫蘆
都傾出來,就都吃了,如吃炒豆相似。

一時間,丹滿酒醒。又自己揣度道:「不好,不好!這場禍比天還大,若驚動
玉帝,性命難存。走,走,走,不如下界為王去也。」他就跑出兜率宮,不行
舊路,從西天門,使個隱身法逃去。即按雲頭,回至花果山界。但見那旌旗閃
灼,戈戟光輝,原來是四健將與七十二洞妖王,在那裏演習武藝。大聖高叫道
:「小的們,我來也!」眾怪丟了器械,跪倒道:「大聖好寬心,丟下我等許
久,不來相顧。」大聖道:「沒多時,沒多時。」

且說且行,徑入洞天深處。四健將打掃安歇,叩頭禮拜畢,俱道:「大聖在天
這百十年,實受何職?」大聖笑道:「我記得才半年光景,怎麼就說百十年話
?」健將道:「在天一日,即在下方一年也。」大聖道:「且喜這番玉帝相愛
,果封做齊天大聖,起一座齊天府,又設安靜、寧神二司,司設仙吏侍衛。向
後見我無事,著我看管蟠桃園。近因王母娘娘設蟠桃大會,未曾請我,是我不
待他請,先赴瑤池,把他那仙品、仙酒,都是我偷吃了。走出瑤池,踉踉蹡蹡
誤入老君宮闕,又把他五個葫蘆金丹也偷吃了。但恐玉帝見罪,方才走出天門
來也。」

眾怪聞言大喜。即安排酒果接風,將椰酒滿斟一石碗奉上。大聖喝了一口,即
咨牙?嘴道:「不好吃,不好吃。」崩、芭二將道:「大聖在天宮吃了仙酒、
仙殽,是以椰酒不甚美口。常言道:『美不美,鄉中水。』」大聖道:「你們
就是『親不親,故鄉人。』我今早在瑤池中受用時,見那長廊之下有許多瓶罐
,都是那玉液瓊漿。你們都不曾嘗著,待我再去偷他幾瓶回來,你們各飲半杯
,一個個也長生不老。」眾猴歡喜不勝。

大聖即出洞門,又翻一觔斗,使個隱身法,徑至蟠桃會上,進瑤池宮闕,只見
那幾個造酒、盤糟、運水、燒火的還鼾睡未醒。他將大的從左右脅下挾了兩個
,兩手提了兩個,即撥轉雲頭回來,會眾猴在於洞中,就做個仙酒會,各飲了
幾杯,快樂不題。

卻說那七衣仙女自受了大聖的定身法術,一周天方能解脫。各提花籃,回奏王
母,說道:「齊天大聖使術法困住我等,故此來遲。」王母問道:「汝等摘了
多少蟠桃?」仙女道:「只有兩籃小桃,三籃中桃。至後面,大桃半個也無,
想都是大聖偷吃了。及正尋間,不期大聖走將出來,行兇拷打,又問設宴請誰
。我等把上會事說了一遍,他就定住我等,不知去向。直到如今,才得醒解回
來。」

王母聞言,即去見玉帝,備陳前事。說不了,又見那造酒的一班人,同仙官等
來奏:「不知甚麼人,攪亂了蟠桃大會,偷吃了玉液瓊漿﹔其八珍百味,亦俱
偷吃了。」又有四個大天師來奏上:「太上道祖來了。」玉帝即同王母出迎。
老君朝禮畢,道:「老道宮中煉了些九轉金丹,伺候陛下做丹元大會,不期被
賊偷去,特啟陛下知之。」玉帝見奏悚懼。少時,又有齊天府仙吏叩頭道:
「孫大聖不守執事,自昨日出遊,至今未轉,更不知去向。」玉帝又添疑思。
只見那赤腳大仙又頫?上奏道:「臣蒙王母詔,昨日赴會,偶遇齊天大聖,對
臣言萬歲有旨,著他邀臣等先赴通明殿演禮,方去赴會。臣依他言語,即返至
通明殿外,不見萬歲龍車鳳輦,又急來此俟候。」玉帝越發大驚道:「這廝假
傳旨意,賺哄賢卿。快著糾察靈官緝訪這廝蹤跡。」

靈官領旨,即出殿遍訪,盡得其詳細,回奏道:「攪亂天宮者,乃齊天大聖也
。」又將前事盡訴一番。玉帝大惱,即差四大天王,協同李天王並哪吒太子,
點二十八宿、九曜星官、十二元辰、五方揭諦、四值功曹、東西星斗、南北二
神、五岳四瀆、普天星相,共十萬天兵,佈一十八架天羅地網,下界去花果山
圍困,定捉獲那廝處治。

眾神即時興師,離了天宮。這一去,但見那:
黃風滾滾遮天暗,紫霧騰騰罩地昏。只為妖猴欺上帝,致令眾聖降凡塵。四大
天王,五方揭諦:四大天王權總制,五方揭諦調多兵。李托塔中軍掌號,惡哪
吒前部先鋒。羅猴星為頭檢點,計都星隨後崢嶸。太陰星精神抖擻,太陽星照
耀分明。五行星偏能豪傑,九曜星最喜相爭。元辰星子午卯酉,一個個都是大
力天丁。五瘟五岳東西擺,六丁六甲左右行。四瀆龍神分上下,二十八宿密層
層。角亢氐房為總領,奎婁胃昴慣翻騰。斗牛女虛危室壁,心尾箕星個個能。
井鬼柳星張翼軫,掄槍舞劍顯威靈。停雲降霧臨凡世,花果山前扎下營。
詩曰:
    天產猴王變化多,偷丹偷酒樂山窩。
    只因攪亂蟠桃會,十萬天兵佈網羅。

當時李天王傳了令,著眾天兵扎了營,把那花果山圍得水泄不通,上下佈了十
八架天羅地網,先差九曜惡星出戰。九曜即提兵徑至洞外,只見那洞外大小群
猴跳躍頑耍。星官厲聲高叫道:「那小妖,你那大聖在那裏?我等乃上界差調
的天神,到此降你這造反的大聖。教他快快來歸降﹔若道半個不字,教汝等一
概遭誅。」那小妖慌忙傳入道:「大聖,禍事了!禍事了!外面有九個兇神,
口稱上界差來的天神,收降大聖。」

那大聖正與七十二洞妖王並四健將分飲仙酒,一聞此報,公然不理道:「今朝
有酒今朝醉,莫管門前是與非。」說不了,一起小妖又跳來道:「那九個兇神
惡言潑語,在門前罵戰哩。」大聖笑道:「莫採他。詩酒且圖今日樂,功名休
問幾時成。」說猶未了,又一起小妖來報:「爺爺!那九個兇神已把門打破,
殺進來也。」大聖怒道:「這潑毛神,老大無禮。本待不與他計較,如何上門
來欺我?」即命獨角鬼王:「領帥七十二洞妖王出陣。老孫領四健將隨後。」
那鬼王疾帥妖兵出門迎敵,卻被九曜惡星一齊掩殺,抵住在鐵板橋頭,莫能得
出。

正嚷間,大聖到了,叫一聲:「開路!」掣開鐵棒,幌一幌,碗來粗細,丈二
長短,丟開架子,打將出來。九曜星那個敢抵,一時打退。那九曜星立住陣勢
道:「你這不知死活的弼馬溫,你犯了十惡之罪:先偷桃,後偷酒,攪亂了蟠
桃大會,又竊了老君仙丹,又將御酒偷來此處享樂。你罪上加罪,豈不知之?」
大聖笑道:「這幾樁事,實有,實有。但如今你怎麼?」九曜星道:「吾奉玉
帝金旨,帥眾到此收降你。快早皈依,免教這些生靈納命,不然,就屣平了此
山,掀翻了此洞也。」大聖大怒道:「量你這些毛神,有何法力,敢出浪言。
不要走,請吃老孫一棒。」這九曜星一齊踴躍﹔那美猴王不懼分毫,掄起金箍
棒,左遮右擋。把那九曜星戰得筋疲力軟,一個個倒拖器械,敗陣而走,急入
中軍帳下,對托塔天王道:「那猴王果十分驍勇,我等戰他不過,敗陣來了。」

李天王即調四大天王與二十八宿,一路出師來鬥。大聖也公然不懼,調出獨角
鬼王、七十二洞妖王與四個健將,就於洞門外列成陣勢。你看這場混戰,好驚
人也:
寒風颯颯,怪霧陰陰。那壁廂旌旗飛彩,這壁廂戈戟生輝。滾滾盔明,層層甲
亮。滾滾盔明映太陽,如撞天的銀磬﹔層層甲亮砌岩崖,似壓地的冰山。大桿
刀,飛雲掣電﹔楮白槍,度霧穿雲。方天戟,虎眼鞭,麻林擺列﹔青銅劍,四
明鏟,密樹排陣。彎弓硬弩雕翎箭,短棍蛇矛挾了魂。大聖一條如意棒,翻來
覆去戰天神。殺得那空中無鳥過,山內虎狼奔﹔揚砂走石乾坤黑,播土飛塵宇
宙昏。只聽兵兵撲撲驚天地,煞煞威威振鬼神。

這一場自辰時佈陣,混殺到日落西山。那獨角鬼王與七十二洞妖怪,盡被眾天
神捉拿去了。止走了四健將與那群猴,深藏在水簾洞底。

這大聖一條棒,抵住了四大天神與李托塔、哪吒太子,俱在半空中,殺勾多時
,大聖見天色將晚,即拔毫毛一把,丟在口中,嚼碎了,噴將出去,叫聲:
「變!」就變了千百個大聖,都使的是金箍棒,打退了哪吒太子,戰敗了五個
天王。

大聖得勝,收了毫毛,急轉身回洞,早又見鐵板橋頭,四個健將領眾叩迎,那
大眾,哽哽咽咽大哭三聲,又唏唏哈哈大笑三聲。大聖道:「汝等見了我,又
哭又笑,何也?」四健將道:「今早帥眾將與天王交戰,把七十二洞妖王與獨
角鬼王盡被眾神捉了,我等逃生,故此該哭。這見大聖得勝回來,未曾傷損,
故此該笑。」大聖道:「勝負乃兵家之常。古人云:『殺人一萬,自損三千。』
況捉了去的頭目乃是虎豹狼蟲、獾獐狐?之類,我同類者未傷一個,何須煩惱
?他雖被我使個分身法殺退,他還要安營在我山腳下。我等且緊緊防守,飽食
一頓,安心睡覺,養養精神。天明看我使個大神通,拿這些天將,與眾報仇。」
四將與眾猴將椰酒吃了幾碗,安心睡覺不題。

那四大天王收兵罷戰,眾各報功:有拿住虎豹的,有拿住獅象的,有拿住狼蟲
狐?的。更不曾捉著一個猴精。當時果又安轅營,下大寨,賞了得功之將,吩
咐了天羅地網之兵,各各提鈴喝號,圍困了花果山,專待明早大戰。各人得令
,一處處謹守。此正是:
       妖猴作亂驚天地,佈網張羅晝夜看。
       畢竟天曉後如何處治,且聽下回分解。





}  \end{pinyinscope}\switchcolumn{\myfontc \section{第 五 回} 乱 蟠 桃 大 圣 苟 丹 反 天 宫 诸 神 捉 怪 话 表 齐 天 大 圣 到 底 是 个 妖 猴 , 更 不 知 道 官 衔 品 级 , 也 不 计 较 俸 禄 的 高 低 , 只 是 只 注 上 名 字 就 可 以 了 。
那 齐 天 府 下 的 两 司 仙 吏 , 早 晚 都 在 地 上 侍 奉 , 只 知 道 日 食 三 餐 , 夜 睡 一 榻 , 无 事 牵 绕 , 自 由 自 在 。
闲 暇 时 节 , 会 友 游 宫 , 交 朋 结 义 。
他 看 见 三 清 人 称 他 为 老 字 , 遇 上 四 帝 称 他 为 陛 下 。
与 那 九 曜 星 、 五 方 将 、 二 十 八 宿 、 四 大 天 王 、 十 二 元 辰 、 五 方 五 老 、 普 天 星 相 、 河 汉 群 神 , 都 只 以 弟 兄 相 对 待 , 彼 此 称 呼 。
今 日 东 游 , 明 日 西 荡 , 云 去 云 来 , 行 踪 不 定 。
有 一 天 , 玉 帝 早 上 朝 见 , 班 部 中 出 现 了 许 旌 阳 真 人 , 赵 玮 启 奏 道 : 现 在 有 个 齐 天 大 圣 , 无 事 闲 游 , 结 交 天 上 的 众 星 宿 , 不 论 高 低 , 都 称 为 朋 友 , 恐 怕 以 后 会 生 事 。
不 如 给 他 一 件 事 管 理 , 可 以 免 除 另 外 产 生 事 端 。
玉 帝 听 了 这 话 , 立 即 宣 读 诏 书 。
那 猴 王 欣 然 而 至 , 说 : 陛 下 , 诏 令 我 老 子 有 什 么 升 迁 奖 赏 ? 玉 帝 说 : 我 见 你 身 闲 无 事 , 给 你 一 些 执 事 , 你 暂 且 暂 且 管 理 那 蟠 桃 园 , 早 晚 好 生 在 意 。
大 圣 欢 喜 谢 恩 , 在 朝 廷 上 高 声 喊 叫 着 退 下 。
其 他 人 不 得 穷 得 无 路 , 就 进 入 蟠 桃 园 里 查 看 。
本 园 中 有 个 土 地 拦 住 他 问 道 : 大 圣 去 什 么 ? 大 圣 说 : 我 奉 玉 帝 点 差 , 代 管 蟠 桃 园 , 现 在 来 查 验 。
又 呼 唤 那 些 人 , 锄 树 力 士 、 修 桃 力 士 、 打 扫 力 士 都 来 见 大 圣 的 叩 头 , 引 他 进 去 。
只 见 那 个 人 说 : 夭 夭 灼 灼 , 琐 碎 繁 琐 。
夭 夭 灼 灼 的 花 朵 满 树 , 颗 颗 的 果 子 压 在 树 枝 上 。
果 子 压 在 枝 头 垂 下 锦 弹 子 , 花 满 树 上 簇 拥 胭 脂 。
时 开 时 结 , 千 年 成 熟 , 无 夏 无 冬 , 万 年 迟 。
先 熟 的 , 脸 色 变 红 , 脸 色 变 红 ; 还 生 的 , 带 着 青 色 的 皮 肤 。
凝 聚 着 浓 烟 凝 聚 着 肌 肤 , 带 着 绿 色 , 映 衬 着 太 阳 显 出 红 色 的 姿 态 。
树 下 的 奇 花 都 是 异 草 , 四 季 不 凋 谢 , 颜 色 齐 齐 ; 左 右 楼 台 和 馆 舍 , 盈 满 空 中 , 常 常 见 到 云 雾 和 霓 虹 。
不 是 玄 都 凡 俗 种 , 瑶 池 王 母 自 己 栽 培 。
大 圣 观 赏 了 很 长 时 间 , 问 土 地 道 : 这 棵 树 有 多 少 棵 ? 土 地 说 : 有 三 千 六 百 棵 , 前 面 一 千 二 百 棵 , 花 瓣 细 小 , 三 千 年 一 熟 , 人 吃 了 成 仙 了 道 , 身 体 健 健 , 身 体 轻 健 ; 中 间 一 千 二 百 株 , 层 层 花 朵 甜 实 , 六 千 年 一 熟 , 人 吃 了 霞 举 飞 升 , 长 生 不 老 ; 后 面 一 千 二 百 棵 紫 色 的 花 朵 紫 色 斑 斓 核 , 九 千 年 一 熟 , 人 吃 了 它 与 天 地 一 同 寿 。
大 圣 听 了 这 话 , 高 兴 得 没 有 任 何 重 任 。
当 天 查 清 了 一 棵 树 , 点 看 了 亭 阁 , 回 到 府 衙 。
从 此 以 后 , 每 隔 三 五 天 一 次 赏 玩 , 也 不 结 交 朋 友 , 也 不 游 玩 其 他 地 方 。
一 天 , 他 看 见 那 棵 老 树 的 枝 头 , 桃 子 成 熟 了 一 大 半 。
他 心 里 想 吃 个 尝 新 , 怎 么 本 园 的 土 地 、 力 士 都 是 齐 天 府 的 仙 吏 , 紧 紧 跟 着 不 便 。
忽 然 设 计 说 : 你 们 暂 且 出 门 等 候 , 让 我 在 这 亭 子 上 稍 稍 休 息 一 段 时 间 。
那 些 众 仙 果 然 退 了 下 来 。
只 见 猴 王 脱 去 帽 子 , 爬 上 大 树 , 挑 出 一 个 熟 的 桃 子 , 摘 了 很 多 桃 子 , 就 在 树 枝 上 自 己 受 用 。
吃 了 一 顿 饭 , 忽 然 跳 下 树 来 , 头 上 戴 着 帽 子 穿 着 衣 服 , 叫 众 人 仪 仗 随 从 回 府 。
过 了 三 二 天 , 又 去 设 法 偷 桃 , 尽 其 他 享 用 。
一 天 早 晨 , 王 母 娘 娘 设 宴 , 大 开 宝 阁 , 在 瑶 池 中 举 行 蟠 桃 胜 会 。
就 穿 上 那 些 红 衣 仙 女 、 青 衣 仙 女 、 素 衣 仙 女 、 皂 衣 仙 女 、 紫 衣 仙 女 、 黄 衣 仙 女 、 绿 衣 仙 女 , 各 自 顶 上 花 篮 , 去 蟠 桃 园 去 摘 桃 树 。
七 衣 仙 女 一 直 来 到 园 门 口 , 只 见 蟠 桃 园 的 土 地 、 力 士 和 齐 天 府 的 二 司 的 仙 吏 , 都 在 那 里 把 门 。
仙 女 走 近 前 来 说 : 我 们 奉 王 母 的 旨 意 , 到 这 里 摘 桃 设 宴 。
土 地 说 : 仙 娥 暂 且 住 下 吧 。
今 年 不 及 往 年 了 , 玉 帝 点 差 齐 天 大 圣 在 此 监 督 治 理 , 必 须 报 答 大 圣 得 知 , 才 敢 开 园 。
仙 女 说 : 大 圣 在 什 么 地 方 呢 ? 土 地 说 : 大 圣 在 园 子 里 , 因 为 困 倦 了 , 自 己 家 在 亭 子 上 睡 了 。
仙 女 说 : 既 然 这 样 , 寻 找 他 的 去 来 , 不 可 迟 误 。
土 地 就 和 他 一 同 前 进 。
寻 到 花 亭 不 见 了 , 只 有 衣 冠 在 亭 子 上 , 不 知 去 哪 里 去 了 , 四 下 里 都 没 有 找 到 的 地 方 。
原 来 大 圣 戏 了 一 会 , 吃 了 几 个 桃 子 , 变 成 了 两 寸 长 的 人 , 在 大 树 梢 头 浓 叶 下 睡 着 了 。
七 衣 仙 女 说 : 我 们 奉 圣 旨 前 来 , 不 久 不 见 大 圣 , 怎 么 敢 空 回 来 呢 旁 边 有 个 仙 使 说 : 仙 娥 既 然 奉 旨 前 来 , 不 必 迟 疑 。
我 在 大 圣 寺 闲 游 惯 了 , 想 是 出 园 会 友 去 了 。
你 们 暂 且 去 摘 桃 , 我 们 替 你 们 回 话 就 是 。
那 仙 女 按 照 她 的 话 , 进 入 树 林 之 下 摘 桃 子 , 先 在 前 面 的 树 上 摘 了 两 个 , 又 在 中 间 的 树 上 摘 了 三 个 , 到 后 边 的 树 上 摘 取 , 只 见 那 树 上 的 花 果 稀 疏 , 只 有 几 个 毛 蒂 青 皮 的 。
原 来 熟 的 都 是 猴 王 吃 了 。
七 仙 女 张 开 望 东 西 , 只 见 向 南 的 枝 上 只 有 一 个 半 红 半 白 的 桃 子 。
青 衣 女 用 手 扯 下 枝 来 , 红 衣 女 摘 了 , 又 把 枝 子 望 着 上 去 一 放 。
原 来 大 圣 变 化 了 , 正 睡 在 这 个 树 枝 上 , 被 他 惊 醒 了 。
大 圣 立 即 出 现 了 原 来 的 相 貌 , 从 耳 朵 里 抽 出 一 个 金 钳 棒 , 一 个 碗 子 , 碗 子 粗 细 , 呵 斥 一 声 道 : 你 是 那 方 的 怪 物 , 怎 敢 大 胆 偷 摘 我 桃 子 ?
这 时 , 那 七 位 仙 女 一 齐 跪 下 说 : 大 圣 王 息 怒 。
我 们 不 是 妖 怪 , 是 王 母 娘 娘 派 来 的 七 衣 仙 女 , 摘 取 仙 桃 , 大 开 宝 阁 , 作 蟠 桃 胜 会 。
刚 刚 来 到 这 里 , 先 见 到 了 本 园 土 地 等 神 , 寻 找 大 圣 不 见 了 。
我 们 恐 怕 迟 了 王 母 的 美 意 , 因 此 我 们 没 有 得 到 圣 明 的 君 主 , 所 以 先 在 这 里 摘 桃 子 。
万 望 宽 恕 他 的 罪 过 。
大 圣 听 了 这 话 , 回 头 大 笑 道 : 仙 娥 请 你 起 来 吧 。
王 母 开 阁 设 宴 , 请 求 的 是 谁 ? 仙 女 说 : 上 会 自 有 旧 规 , 请 求 的 是 西 天 佛 老 、 菩 萨 、 圣 僧 、 罗 汉 , 南 方 南 极 观 音 , 东 方 崇 恩 圣 帝 、 十 洲 三 岛 仙 翁 , 北 方 北 极 玄 灵 , 中 央 黄 极 黄 角 大 仙 , 这 个 是 五 方 五 老 。
还 有 五 斗 星 君 , 上 八 洞 三 清 、 四 帝 、 太 乙 天 仙 等 众 , 中 八 洞 玉 皇 、 九 垒 、 海 岳 神 仙 , 下 八 洞 幽 冥 教 主 、 注 世 地 仙 , 各 宫 各 殿 大 小 尊 神 , 都 一 齐 赴 蟠 桃 嘉 会 。
大 圣 笑 着 说 : 你 可 以 请 我 吗 ? 仙 女 说 : 我 不 曾 听 我 说 。
大 圣 说 : 我 是 齐 天 大 圣 , 就 请 我 老 孙 做 个 席 尊 , 有 什 么 不 可 以 的 ? 仙 女 说 : 这 是 上 会 的 旧 规 , 今 天 不 知 怎 么 办 。
大 圣 说 : 这 话 说 得 对 , 难 怪 你 们 。
你 暂 且 立 下 , 等 老 孙 先 去 打 听 消 息 , 看 可 以 请 老 孙 不 请 。
大 圣 , 握 着 诀 语 , 念 诵 咒 语 , 对 众 位 仙 女 说 : 住 住 , 住 住 在 这 个 定 身 法 , 把 那 七 个 衣 服 的 仙 女 , 一 个 个 都 睁 着 眼 睛 , 白 着 眼 睛 , 都 站 在 桃 树 下 面 。
大 圣 放 开 一 朵 祥 云 , 跳 出 园 子 里 , 竟 然 逃 到 瑶 池 的 路 上 而 去 。
正 走 的 时 候 , 只 见 那 墙 壁 的 厢 房 里 有 : 一 天 祥 瑞 的 雾 气 光 芒 闪 闪 , 五 色 祥 云 飞 舞 不 断 。
白 鹤 的 声 音 响 动 九 皋 , 紫 芝 的 颜 色 秀 丽 分 出 千 叶 。
中 间 出 现 了 一 尊 仙 人 , 相 貌 天 然 , 丰 采 异 常 。
神 舞 虹 霓 飘 荡 汉 天 , 腰 悬 宝 , 无 生 灭 。
他 的 名 字 叫 赤 脚 大 罗 仙 , 特 赴 蟠 桃 添 寿 节 。
那 个 赤 脚 大 仙 拍 着 脸 撞 见 大 圣 , 大 圣 低 头 定 计 , 哄 骗 真 仙 , 他 要 暗 中 去 赴 会 , 又 问 : 老 道 去 哪 里 ? 大 仙 说 : 承 蒙 王 母 的 招 唤 , 去 参 加 蟠 桃 嘉 会 。
大 圣 说 : 老 子 不 知 道 。
玉 帝 因 为 老 孙 朱 斗 云 有 病 , 著 老 孙 五 路 邀 请 列 位 , 先 到 通 明 殿 下 演 礼 , 后 来 才 去 参 加 宴 会 。
大 仙 是 个 光 明 正 大 的 人 , 就 用 他 的 语 作 真 , 说 : 我 常 年 就 在 瑶 池 演 礼 谢 恩 , 为 什 么 先 去 通 明 殿 演 礼 , 然 后 再 去 瑶 池 赴 会 ?
大 圣 驾 着 云 , 念 咒 语 , 动 身 一 变 , 就 变 成 赤 脚 大 仙 , 向 前 奔 赴 瑶 池 。
不 多 时 , 直 到 宝 阁 , 按 住 云 头 , 轻 轻 移 步 , 走 进 里 面 。
只 见 那 里 , 琼 香 缭 绕 , 瑞 气 缤 纷 。
瑶 台 铺 彩 结 , 宝 阁 散 发 氤 氲 。
凤 凰 飞 翔 的 凤 凰 翱 翔 , 形 态 翩 翩 , 金 花 玉 萼 的 影 子 浮 沉 。
上 面 排 列 着 九 凤 丹 霞 , 八 宝 紫 虹 墩 , 五 彩 描 金 的 桌 子 , 千 花 碧 玉 盆 。
桌 子 上 有 龙 肝 和 凤 髓 , 熊 掌 和 猩 唇 。
珍 肴 百 味 一 样 美 , 奇 异 的 果 子 和 嘉 肴 , 色 彩 新 鲜 。
其 所 以 不 得 也 。
这 位 大 圣 人 一 点 也 看 不 完 , 忽 然 闻 到 一 阵 酒 香 扑 鼻 。
忽 然 转 头 , 见 右 边 的 长 廊 下 , 有 几 个 酿 酒 的 仙 官 、 盘 糟 的 壮 士 , 带 着 几 个 运 水 的 道 人 、 烧 火 的 童 子 , 在 那 里 洗 盆 、 洗 瓮 , 已 经 酿 成 了 玉 液 琼 浆 、 美 酒 。
大 圣 不 得 , 口 角 流 涎 , 就 要 去 吃 , 怎 么 样 的 人 都 在 这 里 呢 ?
其 他 人 之 所 以 不 得 , 而 不 得 不 得 不 得 也 , 不 得 不 得 也 , 不 可 得 也 。
你 看 那 些 人 , 手 软 头 低 , 闭 眉 合 眼 , 丢 了 执 事 , 都 去 睡 。
大 圣 于 是 拿 出 百 味 八 珍 , 佳 肴 异 味 , 走 进 长 廊 里 , 就 着 缸 , 靠 着 瓮 , 放 开 酒 量 , 痛 喝 一 番 。
吃 了 很 多 时 间 , 醉 了 。
他 自 己 揣 摸 着 自 己 说 : 不 好 , 不 好 , 再 过 一 会 儿 , 请 来 的 客 人 来 了 , 却 不 怪 我 , 一 时 拿 住 , 怎 么 能 生 出 这 样 好 , 不 如 早 点 回 府 中 睡 去 。
喜 好 大 圣 人 , 摇 摇 摆 动 , 依 仗 着 酒 , 任 意 乱 撞 。
一 会 儿 把 路 差 误 了 , 不 是 齐 天 府 , 却 是 兜 率 天 宫 。
一 见 到 , 顿 时 醒 悟 道 : 兜 率 宫 是 三 十 三 天 之 上 , 是 离 恨 天 太 上 老 君 的 地 方 , 为 什 么 错 到 这 里 呢 ? 也 罢 , 也 罢 , 一 向 要 来 望 此 老 人 , 不 曾 来 , 现 在 趁 着 这 残 余 的 步 子 , 就 望 他 一 望 也 好 。
说 完 就 整 理 衣 服 撞 进 去 , 那 里 不 见 老 君 , 四 处 也 没 有 人 迹 。
原 来 那 老 君 与 燃 灯 古 佛 在 三 层 高 阁 朱 陵 丹 台 上 讲 道 , 众 仙 童 、 仙 将 、 仙 官 、 仙 吏 都 侍 立 在 左 右 听 讲 。
这 位 大 圣 人 直 到 丹 房 里 面 , 寻 找 找 不 到 。
只 见 丹 灶 旁 边 , 炉 中 有 火 。
炉 左 右 安 放 着 五 个 葫 芦 , 葫 芦 里 都 是 炼 成 的 金 丹 。
大 圣 高 兴 地 说 : 这 个 东 西 是 仙 家 最 珍 贵 的 宝 物 。
老 孙 自 从 悟 道 以 来 , 就 知 道 了 内 外 相 同 的 道 理 , 也 要 炼 金 丹 救 人 , 不 料 到 家 没 有 空 闲 。
今 天 有 缘 缘 , 却 又 撞 着 这 个 东 西 。
趁 老 子 不 在 , 等 我 吃 它 几 丸 尝 新 。
其 人 皆 不 知 其 意 , 不 知 其 意 。
一 时 间 , 丹 砂 满 酒 醒 了 。
又 自 己 揣 度 道 : 不 好 , 不 好 , 这 场 祸 祸 比 天 还 大 , 如 果 惊 动 玉 帝 , 性 命 难 保 。
走 , 走 , 走 , 走 , 不 如 下 界 为 大 王 离 开 。
又 从 西 天 门 出 去 , 使 我 有 一 个 隐 身 的 方 法 逃 走 。
随 即 按 着 云 头 , 回 到 花 果 山 边 界 。
只 见 那 些 旌 旗 闪 闪 , 戈 戟 闪 闪 , 原 来 是 四 个 健 将 和 七 十 二 个 洞 的 妖 王 , 在 那 里 演 习 武 艺 。
大 圣 高 声 大 叫 道 : 小 的 人 , 我 来 呀 众 人 怪 怪 丢 掉 武 器 , 跪 下 说 : 大 圣 好 宽 心 , 丢 下 我 们 许 久 , 不 来 看 你 们 。
大 圣 说 : 没 有 多 时 间 , 没 有 多 时 间 。
一 边 说 一 边 走 , 径 直 进 入 洞 天 深 处 。
四 位 健 将 要 打 扫 安 歇 , 叩 头 礼 拜 完 毕 , 都 说 : 大 圣 在 天 上 这 百 十 年 , 实 在 是 承 受 什 么 职 责 ? 大 圣 笑 着 说 : 我 记 得 才 半 年 的 光 景 , 怎 么 就 说 一 百 十 年 的 话 呢 健 将 说 : 在 天 上 一 日 , 就 在 下 方 一 年 。
大 圣 说 : 且 喜 此 番 玉 帝 相 爱 , 果 然 封 他 为 齐 天 大 圣 , 建 造 一 座 齐 天 府 , 又 设 置 安 静 、 宁 神 二 司 , 司 设 置 仙 吏 侍 卫 。
以 后 看 到 我 没 事 , 就 让 我 看 管 蟠 桃 园 。
近 来 因 为 王 母 娘 娘 设 蟠 桃 大 会 , 没 有 请 我 , 是 我 不 等 他 请 求 , 先 赴 瑶 池 , 把 那 些 仙 品 、 仙 酒 都 是 我 偷 吃 了 。
走 出 瑶 池 , 踉 跄 跌 踢 , 误 入 老 君 的 宫 殿 , 又 把 五 个 葫 芦 金 丹 也 偷 吃 了 。
只 怕 玉 帝 被 判 罪 , 才 走 出 天 门 来 。
众 人 感 到 奇 怪 , 听 了 他 的 话 非 常 高 兴 。
随 即 安 排 酒 果 迎 风 , 将 荔 酒 满 斟 一 石 碗 奉 上 。
大 圣 喝 了 一 口 , 马 上 就 咬 牙 巴 嘴 说 : 不 好 吃 , 不 好 吃 。
李 崩 、 李 芭 二 位 将 领 说 : 大 圣 在 天 宫 吃 了 仙 酒 、 仙 , 所 以 荔 酒 不 很 好 吃 。
他 常 常 说 道 : 美 不 美 , 乡 中 水 。
大 圣 说 : 你 们 就 是 亲 不 亲 , 故 乡 人 。
我 今 早 在 瑶 池 中 受 用 的 时 候 , 看 见 那 长 廊 下 有 许 多 瓶 罐 , 都 是 那 玉 液 琼 浆 。
你 们 都 不 曾 尝 过 , 等 我 再 去 偷 几 瓶 回 来 , 你 们 各 饮 半 杯 , 一 个 个 都 长 生 不 老 。
众 猴 欢 喜 不 胜 。
大 圣 立 即 出 了 洞 门 , 又 翻 了 一 个 斗 , 使 用 一 个 隐 身 法 , 径 直 到 蟠 桃 会 上 , 进 入 瑶 池 宫 阙 , 只 见 那 些 造 酒 、 盘 糟 、 运 水 、 烧 火 的 还 是 睡 着 不 醒 。
他 将 大 的 狮 子 从 左 右 胁 下 挟 住 了 两 个 , 两 手 提 着 两 个 , 立 即 拨 转 了 云 头 回 来 , 会 合 众 猴 在 洞 中 , 就 作 了 一 个 仙 酒 会 。
又 说 : 七 衣 仙 女 自 己 受 了 大 圣 的 定 身 法 术 , 一 周 天 才 能 解 脱 。
他 们 各 自 提 着 花 篮 , 回 来 向 王 母 报 告 , 说 道 : 齐 天 大 圣 使 术 法 困 住 我 们 , 所 以 我 们 这 次 来 得 晚 。
王 母 问 道 : 你 们 摘 了 多 少 蟠 桃 ? 仙 女 说 : 只 有 两 篮 小 桃 , 三 箩 中 桃 。
到 了 后 面 , 大 桃 半 个 也 没 有 , 想 都 是 大 圣 偷 吃 了 。
到 了 正 寻 间 , 没 想 到 大 圣 逃 走 将 要 出 来 , 行 凶 拷 打 , 又 问 : 设 宴 请 谁 ?
吾 等 人 把 上 面 的 事 说 了 , 他 就 定 住 在 我 们 的 地 方 , 不 知 道 去 向 。
直 到 今 天 , 才 得 以 醒 解 回 来 。
王 母 听 了 这 话 , 立 即 去 见 玉 帝 , 详 细 陈 述 以 前 的 事 情 。
说 不 了 , 又 见 那 些 酿 酒 的 一 班 人 , 和 仙 官 等 来 上 奏 说 : 不 知 道 是 什 么 人 , 搅 乱 了 蟠 桃 大 会 , 偷 吃 了 玉 液 琼 浆 ; 那 些 八 珍 百 味 , 也 都 偷 吃 了 。
又 有 四 个 大 天 师 来 报 告 皇 上 : 太 上 道 祖 来 了 。
玉 帝 就 和 王 母 一 起 出 来 迎 接 。
老 君 上 朝 礼 完 毕 , 说 道 : 老 道 在 宫 中 炼 了 九 转 金 丹 , 等 候 陛 下 做 丹 元 大 会 , 不 料 被 贼 偷 去 , 特 地 启 奏 陛 下 知 道 这 件 事 。
玉 帝 看 了 奏 章 , 十 分 恐 惧 。
不 久 , 又 有 个 齐 天 府 的 仙 吏 叩 头 说 : 孙 大 圣 不 守 执 事 , 从 昨 天 出 去 游 玩 , 至 今 没 有 转 换 , 更 不 知 去 向 。
玉 帝 又 增 添 了 疑 虑 。
只 见 那 个 赤 脚 大 仙 又 捏 啦 上 奏 说 : 我 承 蒙 王 母 的 诏 令 , 昨 天 赴 会 , 偶 然 遇 到 齐 天 大 圣 , 对 我 说 万 岁 有 旨 , 其 他 邀 请 我 等 先 到 通 明 殿 演 礼 , 然 后 去 赴 会 。
我 依 照 他 的 话 , 立 即 返 回 到 通 明 殿 外 , 看 不 见 万 岁 的 龙 车 凤 , 又 急 忙 来 到 这 里 等 候 。
玉 帝 越 发 大 惊 说 : 这 个 人 假 传 圣 旨 , 欺 骗 贤 卿 。
快 去 纠 察 灵 官 , 搜 寻 这 个 人 的 踪 迹 。
灵 官 领 旨 , 就 出 殿 遍 访 , 全 部 了 解 了 其 中 的 详 细 情 况 , 回 奏 道 : 搅 乱 天 宫 的 , 是 齐 天 大 圣 。
又 将 以 前 的 事 全 部 诉 说 了 一 番 。
玉 帝 大 为 恼 怒 , 立 即 派 四 大 天 王 , 协 同 李 天 王 和 何 吒 太 子 , 点 点 二 十 八 宿 、 九 曜 星 官 、 十 二 元 辰 、 五 方 揭 谛 、 四 遇 功 曹 、 东 西 星 斗 、 南 北 二 神 、 五 岳 四 渎 、 普 天 星 相 , 共 十 万 天 兵 , 布 置 十 八 架 天 罗 地 网 , 下 界 离 开 果 山 围 困 。
众 神 立 即 起 兵 , 离 开 了 天 宫 。
这 一 去 , 只 见 到 那 里 : 黄 风 浩 荡 遮 天 黑 暗 , 紫 雾 腾 腾 覆 地 昏 暗 。
只 是 妖 猴 欺 骗 上 帝 , 以 致 让 众 位 圣 人 投 降 尘 世 。
四 大 天 王 , 五 方 揭 谛 : 四 大 天 王 权 总 制 , 五 方 揭 谛 调 多 兵 。
李 托 塔 是 中 军 掌 军 的 首 领 , 恶 何 吒 是 前 部 先 锋 。
罗 猴 星 是 头 检 点 , 计 都 星 随 后 。
太 阴 星 精 神 抖 擞 , 太 阳 星 照 耀 分 明 。
五 行 星 偏 能 豪 杰 , 九 曜 星 最 喜 欢 互 相 争 斗 。
元 辰 星 是 子 午 、 卯 酉 , 一 个 个 都 是 大 力 天 丁 。
五 脏 五 脏 , 五 岳 东 西 排 列 , 六 丁 六 甲 左 右 行 。
四 渎 龙 神 分 为 上 下 , 二 十 八 宿 密 密 。
角 、 亢 、 氐 、 房 作 为 总 领 , 奎 、 娄 、 胃 、 昴 宿 常 常 翻 腾 。
斗 牛 、 女 宿 、 虚 宿 、 危 宿 、 室 宿 、 壁 宿 、 心 宿 、 尾 宿 、 箕 宿 、 星 宿 各 个 都 能 干 。
井 鬼 、 柳 星 张 开 翅 膀 , 枪 舞 剑 显 示 威 灵 。
停 云 降 雾 临 尘 世 , 花 果 山 前 扎 下 营 。
诗 说 : 上 天 生 猴 王 变 化 多 , 偷 丹 偷 酒 乐 山 窝 。
只 因 搅 乱 蟠 桃 会 , 十 万 天 兵 布 网 罗 。
当 时 李 天 王 传 令 , 命 令 众 多 天 兵 扎 营 , 把 那 花 果 山 围 住 , 水 泄 不 通 , 上 下 布 置 十 八 架 天 罗 地 网 , 先 派 九 曜 恶 星 出 战 。
张 九 曜 就 带 着 兵 马 径 直 来 到 洞 外 , 只 见 那 洞 外 的 大 小 群 猴 子 跳 跃 着 嬉 戏 。
星 官 厉 声 大 叫 道 : 那 小 妖 , 你 那 个 大 圣 在 哪 里 , 我 等 是 上 界 差 调 的 天 神 , 到 这 里 降 你 这 个 造 反 大 圣 。
教 他 快 来 投 降 , 如 果 说 半 个 不 字 , 就 教 你 们 一 概 被 杀 。
那 个 小 妖 , 慌 忙 传 进 道 : 大 圣 , 祸 事 了 , 外 面 有 九 个 凶 神 , 口 称 上 界 派 来 的 天 神 , 收 降 了 大 圣 。
那 大 圣 正 和 七 十 二 洞 的 妖 王 和 四 位 健 将 分 别 饮 用 仙 酒 , 一 听 到 这 个 消 息 , 就 公 然 不 理 说 : 今 朝 有 酒 今 朝 醉 , 不 管 门 前 是 与 非 。
说 不 了 , 一 个 小 妖 又 跳 来 说 : 那 九 个 凶 神 , 恶 言 狂 语 , 在 门 前 骂 战 哩 。
大 圣 笑 着 说 : 不 要 采 他 。
诗 酒 且 图 今 日 快 乐 , 功 名 休 问 何 时 成 ?
说 还 没 说 完 , 又 有 一 个 小 妖 来 报 告 说 : 爷 爷 , 那 九 个 凶 神 已 经 把 门 打 破 , 杀 进 来 。
大 圣 生 气 地 说 : 这 泼 毛 神 , 老 大 无 礼 。
我 本 来 就 不 与 他 计 较 , 为 什 么 上 门 来 欺 骗 我 呢 于 是 命 令 独 角 鬼 王 说 : 我 率 领 七 十 二 洞 的 妖 王 出 阵 。
老 孙 领 着 四 位 健 将 跟 在 后 面 。
那 鬼 王 急 忙 率 领 妖 兵 出 门 迎 敌 , 却 被 九 曜 恶 星 一 齐 掩 杀 , 抵 住 在 铁 板 桥 头 , 没 有 人 能 够 出 来 。
大 圣 到 了 , 大 喊 一 声 , 叫 道 : 开 路 ! 拿 开 铁 棒 , 打 一 个 账 子 , 碗 子 粗 细 , 一 丈 二 长 , 丢 开 架 子 , 打 出 来 。
九 曜 星 哪 个 胆 敢 抵 抗 , 一 时 间 打 退 。
那 九 曜 星 站 在 阵 势 上 说 : 你 这 个 不 知 道 死 活 的 弼 马 温 , 你 犯 了 十 恶 罪 , 先 偷 桃 , 后 偷 酒 , 搅 乱 了 蟠 桃 大 会 , 又 偷 了 老 君 仙 丹 , 又 将 御 酒 偷 到 这 里 来 享 乐 。
大 圣 笑 着 说 : 这 几 件 事 , 实 在 有 , 实 在 有 。
但 现 在 你 怎 么 样 ? 九 曜 星 说 : 我 奉 玉 帝 的 旨 意 , 率 领 部 众 到 这 里 来 收 降 你 。
快 点 皈 依 , 免 教 这 些 生 灵 纳 命 , 不 然 的 话 , 就 铲 平 了 此 山 , 翻 翻 了 此 洞 。
大 圣 大 怒 道 : 估 量 你 这 些 毛 神 , 有 什 么 法 力 , 敢 说 出 浪 言 。
不 要 走 , 请 让 我 给 老 孙 吃 一 棒 。
那 美 猴 王 不 怕 一 丝 一 毫 , 拔 起 金 钳 棒 , 左 边 遮 住 右 边 挡 住 。
把 那 九 曜 星 打 得 筋 骨 疲 惫 , 一 个 个 倒 着 武 器 , 败 阵 逃 走 , 急 忙 进 入 中 军 帐 下 , 对 托 塔 天 王 说 : 那 猴 王 果 然 十 分 骁 勇 , 我 们 作 战 不 过 , 败 阵 就 来 了 。
李 天 王 就 调 动 四 大 天 王 和 二 十 八 宿 , 一 路 出 师 来 战 斗 。
大 圣 公 然 不 害 怕 , 调 出 独 角 鬼 王 、 七 十 二 洞 妖 王 和 四 个 健 将 , 在 洞 门 外 列 成 阵 势 。
你 看 这 场 混 战 , 好 让 人 惊 讶 。 寒 风 飒 飒 , 怪 雾 阴 阴 。
那 墙 壁 边 的 旌 旗 飞 扬 着 彩 色 , 这 墙 壁 边 的 戈 戟 生 辉 。
滚 滚 的 盔 盔 明 亮 , 层 层 的 铠 甲 光 亮 。
滚 滚 的 盔 盔 闪 闪 光 辉 映 照 太 阳 , 好 像 撞 击 天 空 的 银 磬 ; 层 层 的 盔 甲 光 芒 , 堆 在 岩 崖 上 , 好 像 压 在 地 上 的 冰 山 。
大 杆 刀 , 飞 云 掣 电 , 楮 白 枪 , 渡 过 云 雾 穿 越 云 霄 。
方 天 戟 , 虎 眼 鞭 , 麻 林 摆 开 阵 势 ; 青 铜 剑 , 四 明 , 密 树 排 列 。
弯 弓 硬 弩 , 雕 翎 箭 , 短 棍 蛇 矛 挟 住 了 魂 魄 。
大 圣 一 支 如 意 棒 , 翻 来 覆 去 , 战 天 神 。
杀 得 那 空 中 没 有 鸟 飞 过 , 山 内 虎 狼 狼 奔 逃 ; 扬 砂 走 石 , 乾 坤 黑 , 播 土 飞 尘 宇 宙 昏 暗 。
只 听 到 兵 兵 扑 扑 惊 天 动 地 , 威 震 鬼 神 。
此 一 场 , 从 辰 时 布 阵 , 混 杀 到 日 落 西 山 。
那 独 角 鬼 王 和 七 十 二 洞 的 妖 怪 , 全 都 被 众 位 天 神 捉 拿 去 了 。
只 走 了 四 个 健 将 和 那 群 猴 子 , 深 藏 在 水 帘 洞 底 。
大 圣 一 支 棒 , 抵 住 了 四 大 天 神 和 李 托 塔 、 何 吒 太 子 , 都 在 半 空 中 杀 掉 了 很 多 时 间 。 大 圣 见 天 色 将 晚 , 就 拔 出 一 把 毫 毛 , 扔 在 口 中 , 嚼 碎 了 , 喷 出 去 , 叫 道 : 变 了 千 百 个 大 圣 , 都 使 用 的 是 金 铃 棒 , 打 退 了 何 吒 太 子 , 打 败 了 五 个 天 王 。
大 圣 得 胜 , 收 敛 了 毫 毛 , 急 忙 转 身 回 到 洞 中 , 早 晨 又 见 到 铁 板 桥 头 , 四 个 健 将 领 着 众 人 前 去 迎 接 , 那 些 大 众 , 哽 咽 咽 咽 , 大 哭 三 声 , 又 叹 息 哈 哈 哈 哈 哈 哈 哈 哈 哈 大 笑 三 声 。
大 圣 说 : 你 们 见 了 我 , 又 哭 又 笑 , 是 为 什 么 呢 四 位 健 将 说 : 今 早 我 率 领 众 将 与 天 王 交 战 , 把 七 十 二 洞 妖 王 和 独 角 鬼 王 都 被 众 神 捉 住 了 , 我 们 逃 生 , 所 以 该 哭 。
这 是 见 到 大 圣 得 胜 回 来 , 没 有 受 到 损 伤 , 所 以 应 该 笑 。
大 圣 说 : 胜 负 是 兵 家 的 常 规 。
古 人 说 : 杀 人 一 万 人 , 自 己 损 失 三 千 人 。
况 且 捉 了 去 的 头 目 , 就 是 虎 豹 狼 虫 、 獐 獐 狐 狸 之 类 , 我 同 类 的 人 没 有 伤 害 一 个 , 何 必 烦 恼 他 , 他 虽 然 被 我 用 一 个 分 身 法 杀 掉 , 他 还 要 安 营 在 我 山 脚 下 。
我 们 暂 且 紧 紧 防 守 , 饱 食 一 顿 , 安 心 睡 觉 , 养 精 养 神 。
天 明 看 我 使 你 这 个 大 神 通 , 拿 这 些 天 将 , 与 众 人 报 仇 。
四 个 将 领 和 众 猴 子 拿 着 茶 酒 吃 了 几 碗 , 安 心 睡 觉 不 再 问 题 。
四 大 天 王 收 兵 罢 战 , 众 人 各 自 报 功 : 有 拿 住 虎 豹 的 , 有 拿 住 虎 、 象 的 , 有 拿 住 狼 、 狐 等 的 。
更 不 曾 捉 着 一 个 猴 精 。
当 时 果 然 又 安 营 , 攻 下 大 寨 , 奖 赏 得 功 的 将 领 , 指 点 天 罗 地 网 的 士 兵 , 各 自 提 起 铃 号 , 包 围 了 花 果 山 , 专 门 等 待 明 早 大 战 。
各 人 得 到 命 令 , 一 个 处 处 谨 慎 地 遵 守 。
这 正 是 妖 猴 作 乱 惊 天 动 地 , 布 网 张 网 昼 夜 看 。
再 说 , 天 亮 后 如 何 处 治 , 暂 且 听 下 回 分 析 。
}\switchcolumn\flushpage  \begin{pinyinscope}{\myfontt \section{第六回}     觀音赴會問原因 小聖施威降大聖

且不言天神圍繞,大聖安歇。話表南海普陀落伽山大慈大悲救苦救難靈感觀世
音菩薩,自王母娘娘請赴蟠桃大會,與大徒弟惠岸行者,同登寶閣瑤池,見那
裏荒荒涼涼,席面殘亂﹔雖有幾位天仙,俱不就座,都在那裏亂紛紛講論。菩
薩與眾仙相見畢,眾仙備言前事。菩薩道:「既無盛會,又不傳杯,汝等可跟
貧僧去見玉帝。」眾仙怡然隨往。至通明殿前,早有四大天師、赤腳大仙等眾
俱在此,迎著菩薩,即道玉帝煩惱,調遣天兵,擒怪未回等因。菩薩道:「我
要見見玉帝,煩為轉奏。」天師丘弘濟即入靈霄寶殿,啟知宣入。時有太上老
君在上,王母娘娘在後。

菩薩引眾同入裏面,與玉帝禮畢,又與老君、王母相見,各坐下。便問:「蟠
桃盛會如何?」玉帝道:「每年請會,喜喜歡歡﹔今年被妖猴作亂,甚是虛邀
也。」菩薩道:「妖猴是何出處?」玉帝道:「妖猴乃東勝神洲傲來國花果山
石卵化生的。當時生出,即目運金光,射沖斗府。始不介意,繼而成精,降龍
伏虎,自削死籍。當有龍王、閻王啟奏。朕欲擒拿,是長庚星啟奏道:『三界
之間,凡有九竅者,可以成仙。』朕即施教育賢,宣他上界,封為御馬監弼馬
溫官。那廝嫌惡官小,反了天宮。即差李天王與哪吒太子收降,又降詔撫安,
宣至上界,就封他做個齊天大聖,只是有官無祿。他因沒事幹管理,東遊西蕩
。朕又恐別生事端,著他代管蟠桃園。他又不遵法律,將老樹大桃,盡行偷吃
。及至設會,他乃無祿人員,不曾請他。他就設計賺哄赤腳大仙,卻自變他相
貌入會,將仙殽仙酒盡偷吃了,又偷老君仙丹,又偷御酒若干,去與本山眾猴
享樂。朕心為此煩惱,故調十萬天兵,天羅地網收伏。這一日不見回報,不知
勝負如何。」菩薩聞言,即命惠岸行者道:「你可快下天宮,到花果山,打探
軍情如何。如遇相敵,可就相助一功,務必的實回話。」

惠岸行者整整衣裙,執一條鐵棍,駕雲離闕,徑至山前。見那天羅地網,密密
層層,各營門提鈴喝號,將那山圍繞的水泄不通。惠岸立住叫:「把營門的天
丁,煩你傳報:我乃李天王二太子木吒──南海觀音大徒弟惠岸,特來打探軍
情。」那營裏五岳神兵,即傳入轅門之內。早有虛日鼠、昴日雞、星日馬、房
日兔,將言傳到中軍帳下。李天王發下令旗,教開天羅地網,放他進來。此時
東方才亮,惠岸隨旗進入,見四大天王與李天王下拜。拜訖,李天王道:「孩
兒,你自那廂來者?」惠岸道:「愚男隨菩薩赴蟠桃會,菩薩見勝會荒涼,瑤
池寂寞,引眾仙並愚男去見玉帝。玉帝備言父王等下界收伏妖猴,一日不見回
報,勝負未知,菩薩因命愚男到此打聽虛實。」李天王道:「昨日到此安營下
寨,著九曜星挑戰,被這廝大弄神通,九曜星俱敗走而回。後我等親自提兵,
那廝也排開陣勢。我等十萬天兵,與他混戰至晚,他使個分身法戰退。及收兵
查勘時,止捉得些狼蟲虎豹之類,不曾捉得他半個妖猴。今日還未出戰。」

說不了,只見轅門外有人來報道:「那大聖引一群猴精,在外面叫戰。」四大
天王與李天王並太子正議出兵,木叉道:「父王,愚男蒙菩薩吩咐,下來打探
消息,就說若遇戰時,可助一功。今不才願往,看他怎麼個大聖。」天王道:
「孩兒,你隨菩薩修行這幾年,想必也有些神通,切須在意。」

好太子,雙手掄著鐵棍,束一束繡衣,跳出轅門,高叫:「那個是齊天大聖?」
大聖挺如意棒,應聲道:「老孫便是。你是甚人,輒敢問我?」木叉道:「吾
乃李天王第二太子木叉,今在觀音菩薩寶座前為徒弟護教,法名惠岸是也。」
大聖道:「你不在南海修行,卻來此見我做甚?」木叉道:「我蒙師父差來打
探軍情,見你這般猖獗,特來擒你。」大聖道:「你敢說那等大話,且休走,
吃老孫這一棒。」木叉全然不懼,使鐵棒劈手相迎。他兩個立那半山中,轅門
外,這場好鬥:
棍雖對棍鐵各異,兵縱交兵人不同。一個是太乙散仙呼大聖,一個是觀音徒弟
正元龍。渾鐵棍乃千鎚打,六丁六甲運神功﹔如意棒是天河定,鎮海神珍法力
洪。兩個相逢真對手,往來解數實無窮。這個的陰手棍萬千兇,繞腰貫索疾如
風﹔那個的夾槍棒不放空,左遮右擋怎相容。那陣上旌旗閃閃,這陣上鼉鼓鼕
鼕。萬員天將團團繞,一洞妖猴簇簇叢。怪霧愁雲漫地府,狼煙煞氣射天宮。
昨朝混戰還猶可,今日爭持更又兇。堪羨猴王真本事,木叉復敗又逃生。

這大聖與惠岸戰經五六十合,惠岸臂膊酸麻,不能迎敵,虛幌一幌,敗陣而走
。大聖也收了猴兵,安扎在洞門之外。只見天王營門外,大小天兵接住了太子
,讓開大路,徑入轅門,對四天王、李托塔、哪吒,氣哈哈的喘息未定:「好
大聖,好大聖!著實神通廣大,孩兒戰不過,又敗陣而來也!」李天王見了心
驚,即命寫表求助,便差大力鬼王與木叉太子上天啟奏。

二人當時不敢停留,闖出天羅地網,駕起瑞靄祥雲。須臾,徑至通明殿下,見
了四大天師,引至靈霄寶殿,呈上表章。惠岸又見菩薩施禮。菩薩道:「你打
探的如何?」惠岸道:「始領命到花果山,叫開天羅地網門,見了父親,道師
父差命之意。父王道:『昨日與那猴王戰了一場,止捉得他虎豹獅象之類,更
未捉他一個猴精。』正講間,他又索戰,是弟子使鐵棍與他戰經五六十合,不
能取勝,敗走回營。父親因此差大力鬼王同弟子上界求助。」菩薩低頭思忖。

卻說玉帝拆開表章,見有求助之言,笑道:「叵耐這個猴精,能有多大手段,
就敢敵過十萬天兵?李天王又來求助,卻將那路神兵助之?」言未畢,觀音合
掌啟奏:「陛下寬心,貧僧舉一神,可擒這猴。」玉帝道:「所舉者何神?」
菩薩道:「乃陛下令甥顯聖二郎真君,見居灌洲灌江口,享受下方香火。他昔
日曾力誅六怪,又有梅山兄弟與帳前一千二百草頭神,神通廣大。奈他只是聽
調不聽宣,陛下可降一道調兵旨意,著他助力,便可擒也。」玉帝聞言,即傳
調兵的旨意,就差大力鬼王?調。

那鬼王領了旨,即駕起雲,徑至灌江口,不消半個時辰,直至真君之廟。早有
把門的鬼判傳報至裏道:「外有天使,捧旨而至。」二郎即與眾弟兄出門迎接
旨意,焚香開讀。旨意上云:
花果山妖猴齊天大聖作亂:因在宮偷桃、偷酒、偷丹,攪亂蟠桃大會,見著十
萬天兵、一十八架天羅地網,圍山收伏,未曾得勝。今特調賢甥同義兄弟即赴
花果山助力剿除。成功之後,高陞重賞。

真君大喜道:「天使請回,吾當就去拔刀相助也。」鬼王回奏不題。

這真君即喚梅山六兄弟乃康、張、姚、李四太尉,郭申、直健二將軍,聚集殿
前道:「適才玉帝調遣我等往花果山收降妖猴,同去去來。」眾兄弟俱忻然願
往。即點本部神兵,駕鷹牽犬,搭弩張弓,縱狂風,霎時過了東洋大海,徑至
花果山。見那天羅地網密密層層,不能前進,因叫道:「把天羅地網的神將聽
著:吾乃二郎顯聖真君,蒙玉帝調來,擒拿妖猴者,快開營門放行。」一時,
各神一層層傳入。四大天王與李天王俱出轅門迎接。相見畢,問及勝敗之事,
天王將上項事備陳一遍。真君笑道:「小聖來此,必須與他鬥個變化。列公將
天羅地網不要幔了頂上,只四圍緊密,讓我賭鬥。若我輸與他,不必列公相助
,我自有兄弟扶持﹔若贏了他,也不必列公綁縛,我自有兄弟動手。只請托塔
天王與我使個照妖鏡,住立空中。恐他一時敗陣,逃竄他方,切須與我照耀明
白,勿走了他。」天王各居四維,眾天兵各挨排列陣去訖。

這真君領著四太尉、二將軍,連本身七兄弟,出營挑戰﹔吩咐眾將緊守營盤,
收全了鷹犬。眾草頭神得令。真君只到那水簾洞外,見那一群猴齊齊整整,排
作個蟠龍陣勢。中軍裏立一竿旗,上書「齊天大聖」四字。真君道:「那潑妖
,怎麼稱得起齊天之職?」梅山六弟道:「且休讚嘆,叫戰去來。」那營口小
猴見了真君,急走去報知。那猴王即掣金箍棒,整黃金甲,登步雲履,按一按
紫金冠,騰出營門,急睜睛觀看,那真君的相貌果是清奇,打扮得又秀氣。真
個是:
儀容清俊貌堂堂,兩耳垂肩目有光。
    頭戴三山飛鳳帽,身穿一領淡鵝黃。
    縷金靴襯盤龍襪,玉帶團花八寶粧。
    腰挎彈弓新月樣,手執三尖兩刃槍。
    斧劈桃山曾救母,彈打棕羅雙鳳凰。
    力誅八怪聲名遠,義結梅山七聖行。
    心高不認天家眷,性傲歸神住灌江。
    赤城昭惠英靈聖,顯化無邊號二郎。

大聖見了,笑嘻嘻的將金箍棒掣起,高叫道:「你是何方小將,輒敢大膽到此
挑戰?」真君喝道:「你這廝有眼無珠,認不得我麼?吾乃玉帝外甥、敕封昭
惠靈顯王二郎是也。今蒙上命,到此擒你這造反天宮的弼馬溫猢猻,你還不知
死活。」大聖道:「我記得當年玉帝妹子思凡下界,配合楊君,生一男子,曾
使斧劈桃山的,是你麼?我行要罵你幾聲,曾奈無甚冤仇﹔待要打你一棒,可
惜了你的性命。你這郎君小輩,可急急回去,喚你四大天王出來。」真君聞言
,心中大怒道:「潑猴!休得無禮,吃吾一刃。」大聖側身躲過,疾舉金箍棒
,劈手相還。他兩個這場好殺:
昭惠二郎神,齊天孫大聖。這個心高欺敵美猴王,那個面生壓伏真梁棟。兩個
乍相逢,各人皆賭興。從來未識淺和深,今日方知輕與重。鐵棒賽飛龍,神鋒
如舞鳳。左擋右攻,前迎後映。這陣上梅山六弟助威風,那陣上馬流四將傳軍
令。搖旗擂鼓各齊心,吶喊篩鑼都助興。兩個鋼刀有見機,一來一往無絲縫。
金箍棒是海中珍,變化飛騰能取勝。若還身慢命該休,但要差池為蹭蹬。

真君與大聖鬥經三百餘合,不知勝負。那真君抖搜神威,搖身一變,變得身高
萬丈,兩隻手舉著三尖兩刃神鋒,好便似華山頂上之峰,青臉獠牙,朱紅頭髮
,惡狠狠,望大聖著頭就砍。這大聖也使神通,變得與二郎身軀一樣,嘴臉一
般,舉一條如意金箍棒,卻就是崑崙頂上擎天之柱,抵住二郎神。諕得那馬、
流元帥戰兢兢,搖不得旌旗﹔崩、芭二將虛怯怯,使不得刀劍。這陣上,康、
張、姚、李、郭申、直健傳號令,撒放草頭神,向他那水簾洞外縱著鷹犬,搭
弩張弓,一齊掩殺。可憐沖散妖猴四健將,捉拿靈怪二三千。那些猴拋戈棄甲
,撇劍丟槍,跑的跑,喊的喊,上山的上山,歸洞的歸洞。好似夜貓驚宿鳥,
飛灑滿天星。眾兄弟得勝不題。

卻說真君與大聖變做法天象地的規模,正鬥時,大聖忽見本營中妖猴驚散,自
覺心慌,收了法象,掣棒抽身就走。真君見他敗走,大步趕上道:「那裏走?
趁早歸降,饒你性命。」大聖不戀戰,只情跑起。將近洞口,正撞著康、張、
姚、李四太尉,郭申、直健二將軍,一齊帥眾擋住道:「潑猴!那裏走?」大
聖慌了手腳,就把金箍棒捏做繡花針,藏在耳內。搖身一變,變作個麻雀兒,
飛在樹梢頭釘住。那六兄弟慌慌張張,前後尋覓不見,一齊吆喝道:「走了這
猴精也!走了這猴精也!」

正嚷處,真君到了,問:「兄弟們,趕到那廂不見了?」眾神道:「才在這裏
圍住,就不見了。」二郎圓睜鳳目觀看,見大聖變了麻雀兒,釘在樹上。就收
了法象,撇了神鋒,卸下彈弓。搖身一變,變作個鷂鷹兒,抖開翅,飛將去撲
打。大聖見了,颼的一翅飛起去,變作一只大鶿老,沖天而去。二郎見了,急
抖翎毛,搖身一變,變作一隻大海鶴,鑽上雲霄來嗛。大聖又將身按下,入澗
中,變作一個魚兒,淬入水內。二郎趕至澗邊,不見蹤跡。心中暗想道:「這
猢猻必然下水去也,定變作魚蝦之類。等我再變變拿他。」果一變,變作個魚
鷹兒,飄蕩在下溜頭波面上,等待片時。那大聖變魚兒,順水正游,忽見一隻
飛禽:似青鷂,毛片不青﹔似鷺鷥,頂上無纓﹔似老鸛,腿又不紅:「想是二
郎變化了等我哩!」急轉頭,打個花就走。二郎看見道:「打花的魚兒:似鯉
魚,尾巴不紅﹔似鱖魚,花鱗不見﹔似黑魚,頭上無星﹔似魴魚,鰓上無針。
他怎麼見了我就回去了?必然是那猴變的。」趕上來,刷的啄一嘴。那大聖就
攛出水中,一變,變作一條水蛇,游近岸,鑽入草中。二郎因嗛他不著,他見
水響中,見一條蛇攛出去,認得是大聖。急轉身,又變了一隻朱繡頂的灰鶴,
伸著一個長嘴,與一把尖頭鐵鉗子相似,徑來吃這水蛇。水蛇跳一跳,又變做
一隻花鴇,木木樗樗的,立在蓼汀之上。二郎見他變得低賤,(花鴇乃鳥中至
賤至淫之物,不拘鸞、鳳、鷹、鴉,都與交群)故此不去攏傍。即現原身,走
將去,取過彈弓,拽滿,一彈子把他打個躘踵。

那大聖趁著機會,滾下山崖,伏在那裏又變,變一座土地廟兒:大張著口,似
個廟門﹔牙齒變做門扇﹔舌頭變做菩薩﹔眼睛變做窗櫺﹔只有尾巴不好收拾,
豎在後面,變做一根旗竿。真君趕到崖下,不見打倒的鴇鳥,只有一間小廟。
急睜鳳眼,仔細看之,見旗竿立在後面,笑道:「是這猢猻了,他今又在那裏
哄我。我也曾見廟宇,更不曾見一個旗竿豎在後面的。斷是這畜生弄諠。他若
哄我進去,他便一口咬住。我怎肯進去?等我掣拳先搗窗櫺,後踢門扇。」大
聖聽得,心驚道:「好狠,好狠!門扇是我牙齒,窗櫺是我眼睛,若打了牙,
搗了眼,卻怎麼是好?」撲的一個虎跳,又冒在空中不見。

真君前前後後亂趕,只見四太尉、二將軍一齊擁至,道:「兄長,拿住大聖了
麼?」真君笑道:「那猴兒才自變座廟宇哄我。我正要搗他窗櫺,踢他門扇,
他就縱一縱,又渺無蹤跡。可怪,可怪!」眾皆愕然,四望更無形影。真君道
:「兄弟們在此看守巡邏,等我上去尋他。」急縱身駕雲,起在半空。見那李
天王高擎照妖鏡,與哪吒住立雲端,真君道:「天王,曾見那猴王麼?」天王
道:「不曾上來,我這裏照著他哩。」真君把那賭變化,弄神通,拿群猴一事
說畢。卻道:「他變廟宇,正打處,就走了。」李天王聞言,又把照妖鏡四方
一照,呵呵的笑道:「真君,快去,快去。那猴使了個隱身法,走出營圍,往
你那灌江口去也。」二郎聽說,即取神鋒,回灌江口來趕。

卻說那大聖已至灌江口,搖身一變,變作二郎爺爺的模樣,按下雲頭,徑入廟
裏。鬼判不能相認,一個個磕頭迎接。他坐中間,點查香火:見李虎拜還的三
牲,張龍許下的保福,趙甲求子的文書,錢丙告病的良願。正看處,有人報:
「又一個爺爺來了。」眾鬼判急急觀看,無不驚心。真君卻道:「有個甚麼齊
天大聖,才來這裏否?」眾鬼判道:「不曾見甚麼大聖,只有一個爺爺在裏面
查點哩。」真君撞進門,大聖見了,現出本相道:「郎君不消嚷,廟宇已姓孫
了。」這真君即舉三尖兩刃神鋒,劈臉就砍。那猴王使個身法,讓過神鋒。掣
出那繡花針兒,幌一幌,碗來粗細,趕到前,對面相還。兩個嚷嚷鬧鬧,打出
廟門,半霧半雲,且行且戰,復打到花果山。慌得那四大天王等眾,隄防愈緊
。這康、張太尉等迎著真君,合心努力,把那美猴王圍繞不題。

話表大力鬼王既調了真君與六兄弟提兵擒魔去後,卻上界回奏。玉帝與觀音菩
薩、王母並眾仙卿,正在靈霄殿講話,道:「既是二郎已去赴戰,這一日還不
見回報。」觀音合掌道:「貧僧請陛下同道祖出南天門外,親去看看虛實如何
?」玉帝道:「言之有理。」即擺駕,同道祖、觀音、王母與眾仙卿至南天門
。早有些天丁、力士接著,開門遙觀。只見眾天丁佈羅網,圍住四面﹔李天王
與哪吒擎照妖鏡,立在空中﹔真君把大聖圍繞中間,紛紛賭鬥哩。

菩薩開口對老君說:「貧僧所舉二郎神如何?果有神通,已把那大聖圍困,只
是未得擒拿。我如今助他一功,決拿住他也。」老君道:「菩薩將甚兵器?怎
麼助他?」菩薩道:「我將那淨瓶楊柳拋下去,打那猴頭,即不能打死,也打
個一跌,教二郎小聖好去拿他。」老君道:「你這瓶是個磁器,能打著他便好
,如打不著他的頭,或撞著他的鐵棒,卻不打碎了?你且莫動手,等我老君助
他一功。」菩薩道:「你有甚麼兵器?」老君道:「有,有,有。」捋起衣袖
,左膊上取下一個圈子,說道:「這件兵器,乃錕鋼摶煉的,被我將還丹點成
,養就一身靈氣,善能變化,水火不侵,又能套諸物。一名『金鋼琢』,又名
『金鋼套』。當年過函關,化胡為佛,甚是虧他。早晚最可防身。等我丟下去
打他一下。」

話畢,自天門上往下一摜,滴流流,徑落花果山營盤裏,可可的著猴王頭上一
下。猴王只顧苦戰七聖,卻不知天上墜下這兵器,打中了天靈,立不穩腳,跌
了一跤,爬將起來就跑。被二郎爺爺的細犬趕上,照腿肚子上一口,又扯了一
跌。他睡倒在地,罵道:「這個亡人!你不去妨家長,卻來咬老孫!」急翻身
爬不起來,被七聖一擁按住,即將繩索捆綁,使勾刀穿了琵琶骨,再不能變化。

那老君收了金鋼琢,請玉帝同觀音、王母、眾仙等,俱回靈霄殿。這下面四大
天王與李天王諸神,俱收兵拔寨,近前向小聖賀喜,都道:「此小聖之功也。」
小聖道:「此乃天尊洪福,眾神威權,我何功之有?」康、張、姚、李道:
「兄長不必多敘,且押這廝去上界見玉帝,請旨發落去也。」真君道:「賢弟
,汝等未受天籙,不得面見玉帝。教天甲神兵押著,我同天王等上界回旨。你
們帥眾在此搜山,搜淨之後,仍回灌口。待我請了賞,討了功,回來同樂。」
四太尉、二將軍依言領諾。這真君與眾即駕雲頭,唱凱歌,得勝朝天。不多時
,到通明殿外。天師啟奏道:「四大天王等眾已捉了妖猴齊天大聖了,來此聽
宣。」玉帝傳旨,即命大力鬼王與天丁等眾,押至斬妖臺,將這廝碎剁其屍。
咦!正是:
    欺誑今遭刑憲苦,英雄氣概等時休。
  畢竟不知那猴王性命何如,且聽下回分解。





}  \end{pinyinscope}\switchcolumn{\myfontc \section{第 六 回} 观 音 王 赴 会 , 问 原 因 , 小 圣 施 威 降 大 圣 , 而 且 不 说 天 神 围 绕 , 大 圣 安 歇 。
有 人 说 , 南 海 普 陀 落 伽 山 大 慈 大 悲 救 苦 救 难 , 灵 感 观 世 音 菩 萨 , 自 从 王 母 娘 娘 请 求 参 加 蟠 桃 大 会 , 与 大 弟 弟 惠 岸 行 的 人 , 一 同 登 上 宝 阁 瑶 池 , 见 那 里 荒 凉 荒 凉 , 席 面 残 破 混 乱 , 即 使 有 几 位 天 仙 , 都 不 上 座 , 都 在 那 里 乱 纷 纷 地 讲 论 。
菩 萨 与 众 仙 相 见 后 , 众 仙 都 详 细 地 说 了 以 前 的 事 。
菩 萨 说 : 既 没 有 盛 大 的 宴 会 , 又 没 有 传 杯 , 你 们 可 以 跟 贫 僧 去 见 玉 帝 。
众 仙 都 很 高 兴 地 跟 着 他 去 。
到 通 明 殿 前 , 早 就 有 四 大 天 师 、 赤 脚 大 仙 等 众 人 都 在 这 里 , 迎 接 着 菩 萨 , 就 说 玉 帝 烦 恼 , 调 遣 天 兵 , 擒 捉 怪 物 未 回 等 人 的 缘 故 。
菩 萨 说 : 我 想 见 玉 帝 , 请 你 转 奏 。
天 师 丘 弘 济 立 即 进 入 灵 霄 宝 殿 , 启 奏 知 道 周 宣 进 去 。
当 时 有 太 上 老 君 在 上 , 王 母 娘 娘 在 后 。
菩 萨 引 导 众 人 一 同 进 入 里 面 , 与 玉 帝 行 礼 完 毕 , 又 与 老 君 、 王 母 相 见 , 各 自 坐 下 。
玉 帝 说 : 每 年 举 行 聚 会 , 都 是 喜 欢 欢 喜 , 今 年 被 妖 猴 作 乱 , 实 在 是 虚 设 邀 请 。
玉 帝 说 : 妖 猴 是 东 胜 神 洲 傲 来 国 花 果 山 的 石 卵 变 化 出 来 的 。
当 时 他 出 生 , 就 眼 睛 运 着 金 光 , 射 进 斗 府 。
开 始 时 不 介 意 , 接 着 就 成 了 精 神 , 降 龙 伏 虎 , 自 己 被 削 去 死 籍 。
应 当 有 龙 王 、 阎 王 启 奏 。
我 想 捉 拿 , 是 长 庚 星 启 奏 道 : 三 界 之 间 , 凡 有 九 窍 的 人 , 可 以 成 仙 。
我 立 即 施 行 教 化 , 教 育 贤 才 , 宣 扬 其 他 上 界 , 封 他 为 御 马 监 弼 马 温 官 。
那 个 人 嫌 恶 官 小 , 反 了 天 宫 。
即 遣 李 天 王 与 何 吒 太 子 收 降 , 又 降 诏 安 抚 , 宣 布 到 上 界 , 就 封 他 做 齐 天 大 圣 , 只 是 有 官 没 有 俸 禄 。
其 所 以 不 得 也 。
我 又 担 心 另 外 发 生 事 端 , 让 他 代 管 蟠 桃 园 。
他 又 不 遵 守 法 律 , 把 老 树 大 桃 全 都 偷 吃 。
等 到 设 置 宴 会 , 他 是 没 有 俸 禄 的 人 , 不 曾 请 求 他 。
其 所 以 不 得 也 。
我 心 里 为 此 烦 恼 , 所 以 调 动 十 万 天 兵 , 天 罗 地 网 , 收 伏 。
这 一 天 不 见 回 报 , 不 知 胜 负 如 何 。
菩 萨 听 了 , 立 即 命 令 惠 岸 行 人 说 : 你 可 以 快 点 下 天 宫 , 到 花 果 山 , 打 探 军 情 怎 么 样 ?
如 果 遇 上 相 互 对 峙 , 可 以 互 相 帮 助 一 次 功 劳 , 必 须 真 实 回 话 。
惠 岸 的 行 人 整 整 衣 裙 , 手 里 拿 着 一 根 铁 棍 , 驾 着 云 离 开 宫 阙 , 径 直 来 到 山 前 。
见 其 天 罗 地 网 , 密 密 层 层 , 各 营 门 都 提 着 铃 号 , 把 那 山 围 绕 得 不 通 。
惠 岸 立 即 大 叫 : 把 营 门 的 天 丁 , 麻 烦 你 传 报 : 我 是 李 天 王 二 太 子 木 吒 , 南 海 观 音 大 弟 惠 岸 , 特 来 打 探 军 情 。
那 军 营 里 的 五 岳 神 兵 , 立 刻 传 入 辕 门 之 内 。
早 有 虚 日 鼠 、 昴 星 、 鸡 、 星 星 、 马 、 房 星 、 兔 子 , 将 话 传 到 中 军 帐 下 。
天 王 下 令 旗 , 教 他 开 天 罗 地 网 , 放 他 进 去 。
这 时 东 方 刚 亮 , 惠 岸 随 着 旗 帜 进 入 , 看 见 四 大 天 王 和 李 天 王 下 拜 。
拜 完 , 李 天 王 说 : 孩 子 , 你 从 哪 里 来 的 ? 惠 岸 说 : 愚 男 随 菩 萨 去 蟠 桃 会 , 菩 萨 见 胜 会 荒 凉 , 瑶 池 寂 寞 , 带 着 众 仙 和 愚 男 去 见 玉 帝 。
玉 帝 详 细 地 说 , 父 王 等 人 到 下 界 收 伏 妖 猴 , 一 天 不 见 回 报 , 胜 负 不 知 , 菩 萨 于 是 命 令 愚 男 到 这 里 去 打 听 虚 实 。
李 天 王 说 : 昨 天 到 这 里 安 营 下 寨 , 穿 着 九 曜 星 挑 战 , 被 这 个 人 大 弄 神 通 , 九 曜 星 都 败 走 而 回 。
后 来 我 们 亲 自 率 兵 , 那 个 人 也 排 开 阵 势 。
我 们 十 万 天 兵 , 与 他 们 混 战 至 晚 , 他 们 使 我 们 分 身 法 战 退 。
到 收 兵 查 验 时 , 只 捉 到 一 些 狼 虫 虎 豹 之 类 , 不 曾 捉 到 他 半 个 妖 猴 。
今 天 还 没 有 出 战 。
说 不 完 , 只 见 门 外 有 人 来 报 告 说 : 那 大 圣 率 领 一 群 猴 子 , 在 外 面 喊 战 。
四 大 天 王 与 李 天 王 同 太 子 商 议 出 兵 , 木 叉 说 : 父 王 , 我 的 儿 子 承 蒙 菩 萨 的 吩 咐 , 下 来 打 探 消 息 , 就 说 如 果 遇 上 交 战 的 时 候 , 可 以 帮 助 一 个 功 劳 。
现 在 我 不 愿 去 , 看 他 是 什 么 大 圣 ?
天 王 说 : 孩 子 , 你 跟 随 菩 萨 修 行 了 几 年 , 想 必 也 有 些 神 通 , 切 须 注 意 。
好 太 子 , 双 手 握 着 铁 棍 , 捆 上 一 束 绣 衣 , 跳 出 辕 门 , 高 喊 : 那 个 是 齐 天 大 圣 , 拔 起 如 意 棒 , 应 声 说 : 老 孙 就 是 。
木 叉 说 : 我 是 李 天 王 的 第 二 个 太 子 木 叉 , 现 在 在 观 音 菩 萨 的 宝 座 前 为 弟 弟 护 教 , 法 名 叫 惠 岸 。
大 圣 说 : 你 不 在 南 海 修 行 , 却 来 此 见 我 干 什 么 ? 木 叉 说 : 我 承 蒙 师 父 派 来 打 探 军 情 , 见 你 这 样 猖 獗 , 特 来 抓 你 。
大 圣 说 : 你 敢 说 这 些 大 话 , 暂 且 不 要 走 , 吃 老 孙 这 一 棒 。
木 叉 完 全 不 害 怕 , 派 铁 棒 劈 开 手 来 迎 接 。
其 他 两 个 人 站 在 半 山 之 中 , 辕 门 之 外 。
一 个 是 太 乙 散 仙 呼 大 圣 , 一 个 是 观 音 的 弟 弟 正 元 龙 。
浑 铁 棍 是 千 头 铁 棒 打 , 六 丁 六 甲 运 神 功 ; 如 意 棒 是 天 河 定 , 镇 海 神 珍 法 力 洪 。
两 个 相 逢 真 是 对 手 , 往 来 解 释 数 字 实 在 没 有 穷 尽 。
这 个 阴 手 棍 万 千 凶 恶 , 绕 腰 贯 索 , 快 如 风 ; 那 个 的 夹 枪 棒 不 放 空 , 左 遮 右 挡 , 怎 么 能 容 忍 ?
那 阵 上 旌 旗 闪 闪 , 这 阵 上 鼓 。
万 员 天 将 团 团 绕 绕 , 一 个 洞 中 的 妖 魔 猴 子 簇 拥 成 群 。
怪 雾 愁 云 遍 地 府 , 狼 烟 恶 气 射 天 宫 。
昨 朝 混 战 还 是 可 以 , 今 日 争 持 更 是 凶 险 。
堪 羡 猴 王 真 的 本 事 , 木 叉 又 失 败 , 又 逃 生 。
这 个 大 圣 与 惠 岸 作 战 , 经 过 五 六 十 回 合 , 惠 岸 手 臂 麻 麻 , 不 能 迎 击 敌 人 , 虚 晃 一 帐 , 败 阵 而 逃 。
大 圣 也 收 集 了 猴 子 , 安 排 在 洞 门 之 外 。
天 王 营 门 外 , 大 小 天 兵 接 住 太 子 , 让 他 打 开 大 路 , 径 直 进 入 辕 门 , 对 着 四 天 王 、 李 托 塔 、 何 吒 , 气 息 不 定 , 说 : 好 大 圣 , 好 大 圣 , 实 在 是 神 通 广 大 , 小 孩 子 作 战 不 过 , 又 败 阵 而 来 。
二 人 当 时 不 敢 停 留 , 闯 出 天 罗 地 网 , 驾 起 祥 云 。
不 一 会 儿 , 径 直 来 到 通 明 殿 下 , 见 到 四 大 天 师 , 把 他 领 到 灵 霄 宝 殿 , 呈 上 表 章 。
惠 岸 又 见 菩 萨 施 礼 。
惠 岸 说 : 当 初 领 命 到 花 果 山 , 叫 开 天 罗 地 网 门 , 见 了 父 亲 , 说 出 师 父 遣 命 的 意 思 。
父 王 说 : 昨 天 与 那 猴 王 作 战 了 一 场 , 只 捉 到 了 他 的 虎 豹 虎 象 之 类 , 还 没 有 捉 到 他 的 一 个 猴 精 。
正 在 讲 经 的 时 候 , 他 又 索 要 战 斗 , 是 弟 子 让 铁 棍 与 他 交 战 , 经 过 五 六 十 回 合 , 不 能 取 得 胜 利 , 败 走 回 营 。
他 的 父 亲 因 此 派 大 力 鬼 王 和 弟 子 上 界 请 求 救 助 。
菩 萨 低 头 思 虑 。
又 说 玉 帝 拆 开 表 章 , 见 有 求 救 的 话 , 笑 着 说 : 我 不 耐 这 个 猴 精 , 能 有 多 大 的 手 段 , 竟 敢 抵 得 上 十 万 天 兵 , 李 天 王 又 来 求 救 , 却 带 着 那 个 神 兵 帮 助 他 呢 ?
玉 帝 说 : 你 所 举 荐 的 是 什 么 神 ? 菩 萨 说 : 是 陛 下 的 外 甥 显 圣 二 郎 真 君 , 现 在 住 在 灌 洲 、 灌 江 口 , 享 受 下 方 香 火 。
他 昔 日 曾 经 尽 力 诛 杀 六 种 怪 物 , 还 有 梅 山 兄 弟 和 帐 前 一 千 二 百 个 草 头 神 , 神 通 广 大 。
但 他 只 是 听 从 调 遣 而 不 听 从 宣 旨 , 陛 下 可 降 一 道 调 兵 的 旨 意 , 给 他 帮 助 , 就 可 以 擒 获 。
玉 帝 听 了 , 立 即 传 达 调 兵 的 旨 意 , 就 派 大 力 鬼 王 来 调 动 。
那 鬼 王 领 了 旨 意 , 立 即 驾 车 出 云 , 径 直 来 到 灌 江 口 , 不 到 半 个 时 间 , 直 到 真 君 庙 。
早 晨 有 个 把 门 的 鬼 判 传 来 报 告 到 里 边 说 : 外 面 有 天 使 , 捧 着 圣 旨 来 到 。
二 郎 立 即 和 众 弟 兄 出 门 迎 接 旨 意 , 焚 香 开 读 。
皇 上 的 旨 意 是 : 花 果 山 妖 猴 齐 天 大 圣 作 乱 , 因 为 在 宫 中 偷 桃 、 偷 酒 、 偷 丹 , 扰 乱 蟠 桃 大 会 , 现 有 十 万 天 兵 、 十 八 架 天 罗 地 网 , 围 山 收 伏 , 未 曾 取 得 胜 利 。
现 在 特 地 调 遣 贤 甥 、 同 义 兄 弟 , 马 上 前 往 花 果 山 协 助 剿 灭 。
成 功 之 后 , 高 升 重 赏 。
真 君 非 常 高 兴 地 说 : 天 使 请 我 回 去 , 我 就 去 拔 刀 相 助 。
鬼 王 回 答 说 没 有 问 题 。
这 位 真 君 就 叫 来 梅 山 的 六 兄 弟 , 康 、 张 、 姚 、 李 四 位 太 尉 , 郭 申 、 直 健 两 位 将 军 , 聚 集 在 殿 前 说 : 刚 才 玉 帝 调 我 们 前 往 花 果 山 收 降 妖 猴 , 一 同 去 去 来 。
众 兄 弟 都 高 兴 地 想 去 。
立 即 点 点 本 部 的 神 兵 , 驾 着 老 鹰 拉 着 狗 , 拉 着 弓 箭 , 放 开 狂 风 , 顷 刻 就 越 过 了 东 洋 大 海 , 径 直 到 了 花 果 山 。
他 看 见 那 天 罗 地 网 , 层 层 叠 叠 , 不 能 前 进 , 于 是 大 叫 道 : 把 天 罗 地 网 的 神 将 听 你 说 : 我 是 二 郎 显 圣 真 君 , 承 蒙 玉 帝 调 来 , 捉 拿 妖 猴 , 快 打 开 营 门 放 我 走 。
一 会 儿 , 各 位 神 人 一 层 一 层 一 层 一 层 一 层 一 层 地 传 入 。
四 大 天 王 与 李 天 王 都 出 了 辕 门 迎 接 。
相 见 完 毕 , 问 起 了 胜 负 的 情 况 , 天 王 将 上 项 的 事 全 都 陈 述 了 一 遍 。
真 君 笑 着 说 : 小 圣 来 到 这 里 , 必 须 和 他 打 一 个 变 化 。
列 公 将 天 罗 地 网 , 不 要 幔 子 在 顶 上 , 只 要 四 周 紧 密 , 让 我 赌 斗 。
如 果 我 输 给 他 , 不 必 列 公 帮 助 , 我 自 有 兄 弟 扶 持 ; 如 果 打 败 了 他 , 也 不 必 列 公 捆 绑 , 我 自 有 兄 弟 动 手 。
只 请 托 塔 天 王 给 我 拿 个 照 妖 镜 , 住 在 空 中 。
恐 怕 他 一 时 失 败 , 逃 窜 到 别 的 地 方 , 必 须 给 我 明 白 , 不 要 让 他 逃 走 。
天 王 各 自 居 住 在 四 方 , 众 天 兵 各 自 挨 着 排 列 阵 势 离 开 。
这 位 真 君 领 着 四 个 太 尉 、 二 位 将 军 , 连 同 自 己 的 七 个 兄 弟 , 出 营 挑 战 。
众 草 头 神 得 到 命 令 。
真 君 只 到 水 帘 洞 外 , 看 见 那 一 群 猴 子 , 齐 整 齐 整 齐 , 排 列 成 一 个 蟠 龙 阵 势 。
中 军 里 竖 立 一 竿 旗 , 上 面 写 着 齐 天 大 圣 四 个 字 。
真 君 说 : 那 泼 妖 , 怎 么 能 称 得 起 齐 天 之 职 ? 梅 山 六 弟 说 : 暂 且 不 要 赞 叹 , 叫 作 战 去 来 。
那 个 营 口 的 小 猴 子 见 了 真 君 , 急 忙 跑 去 报 告 了 真 君 。
猴 王 立 即 拿 着 金 铃 棒 , 整 理 着 黄 金 甲 , 登 上 云 履 , 敲 一 下 紫 金 冠 , 跳 出 营 门 , 急 忙 睁 眼 观 看 , 真 君 的 相 貌 果 然 是 清 奇 奇 异 的 , 打 扮 得 又 秀 丽 。
真 个 是 : 仪 容 清 俊 , 仪 表 堂 堂 , 两 耳 垂 肩 , 眼 睛 有 光 。
头 戴 三 山 飞 凤 帽 , 身 穿 一 领 淡 鹅 黄 。
缕 金 靴 映 盘 龙 袜 , 玉 带 团 花 八 宝 妆 。
腰 里 拿 着 弹 弓 新 月 样 , 手 里 拿 着 三 尖 两 刃 枪 。
用 斧 劈 桃 山 曾 救 母 亲 , 用 木 棒 打 棕 罗 双 凤 凰 。
尽 力 诛 杀 八 怪 , 声 名 远 播 , 义 气 结 交 梅 山 七 圣 行 。
心 高 不 认 识 天 家 眷 顾 , 性 格 傲 慢 归 神 , 住 在 灌 江 。
赤 城 昭 惠 英 灵 圣 明 , 显 示 教 化 无 边 , 号 称 二 郎 。
大 圣 见 之 , 笑 笑 嘻 嘻 , 把 金 铃 棒 拿 起 来 , 高 声 叫 道 : 你 是 什 么 地 方 的 小 将 , 竟 敢 大 胆 到 这 里 去 挑 战 , 真 君 喝 道 : 你 这 个 人 有 眼 无 珠 , 认 不 到 我 吗 ? 我 是 玉 帝 的 外 甥 , 敕 封 昭 惠 灵 显 王 二 郎 。
现 在 承 蒙 皇 上 的 命 令 , 到 此 擒 拿 你 这 样 造 反 天 宫 的 弼 马 温 , 你 还 不 知 道 生 死 。
大 圣 说 : 我 记 得 当 年 玉 帝 妹 子 思 凡 下 界 , 与 杨 君 结 合 , 生 下 一 男 子 , 曾 经 使 用 斧 劈 桃 山 的 , 是 你 吗 ? 我 走 路 要 骂 你 几 声 , 可 是 没 有 什 么 冤 仇 , 等 到 要 打 你 一 棒 , 可 惜 你 的 性 命 。
你 这 个 郎 君 小 辈 , 可 急 忙 回 去 , 叫 你 四 大 天 王 出 来 。
真 君 听 了 这 话 , 心 中 大 怒 说 : 泼 猴 , 不 要 无 礼 , 吃 我 一 把 刀 。
大 圣 侧 身 避 过 , 急 忙 举 起 金 钳 棒 , 劈 开 手 来 回 来 。
其 他 两 个 , 好 杀 : 昭 惠 二 郎 神 , 齐 天 孙 大 圣 。
这 个 心 高 欺 敌 美 猴 王 , 那 个 面 生 压 伏 真 梁 栋 。
两 个 人 忽 然 相 遇 , 各 个 人 都 赌 博 起 来 。
从 来 不 知 道 浅 和 深 , 今 天 才 知 道 轻 与 重 。
铁 棒 赛 飞 龙 , 神 锋 如 舞 凤 。
左 拒 右 攻 , 前 迎 后 映 。
这 阵 上 , 梅 山 六 弟 助 威 风 , 那 阵 上 马 流 四 将 传 军 令 。
摇 旗 擂 鼓 , 各 齐 心 协 力 , 呐 喊 敲 锣 都 助 兴 。
两 个 钢 刀 有 见 机 , 一 来 一 往 , 没 有 丝 缝 。
金 钏 棒 是 海 中 珍 贵 的 东 西 , 变 化 飞 腾 能 取 得 胜 利 。
如 果 还 身 慢 , 命 该 休 , 只 要 差 池 池 就 会 被 踢 。
真 君 与 大 圣 打 了 三 百 多 回 合 , 不 知 谁 胜 谁 负 。
那 真 君 振 奋 地 搜 寻 神 威 , 转 身 一 变 , 变 得 身 高 万 丈 , 两 只 手 举 着 三 尖 两 刃 神 锋 , 好 像 华 山 顶 上 的 山 峰 , 青 脸 獠 牙 , 红 色 头 发 , 恶 恶 狠 狠 , 望 大 圣 着 头 就 去 砍 。
大 圣 也 使 神 通 , 变 得 与 二 郎 身 体 一 样 , 嘴 脸 一 样 , 举 起 如 意 金 钳 棒 , 就 是 昆 仑 山 顶 上 擎 天 的 柱 子 , 抵 住 二 郎 的 神 灵 。
得 那 马 、 流 元 帅 战 战 兢 兢 , 摇 不 得 旌 旗 ; 崩 、 芭 二 将 虚 弱 胆 怯 , 使 得 得 不 到 刀 剑 。
在 这 阵 上 , 康 、 张 、 姚 、 李 、 郭 申 、 直 健 传 达 号 令 , 撒 放 草 头 神 , 向 他 那 水 帘 洞 外 放 着 鹰 犬 , 搭 着 弓 箭 , 一 齐 扑 杀 。
可 怜 冲 散 妖 猴 四 健 将 , 捉 拿 灵 怪 二 三 千 。
猴 子 抛 弃 戈 甲 , 抛 弃 剑 枪 , 跑 得 跑 , 叫 的 叫 , 上 山 的 上 山 , 归 洞 的 归 洞 。
好 似 夜 猫 惊 宿 鸟 , 飞 洒 满 天 星 。
众 兄 弟 得 胜 不 愁 。
又 说 : 真 君 与 大 圣 变 成 了 法 天 象 地 的 规 模 。 正 在 打 斗 时 , 大 圣 忽 然 看 见 本 营 中 的 妖 猴 惊 散 了 , 自 己 觉 得 心 惊 , 收 捕 了 法 象 , 拿 着 木 棒 抽 出 身 子 就 走 了 。
真 君 见 他 败 走 , 大 步 赶 上 道 : 从 哪 里 走 , 趁 早 归 降 , 饶 你 的 性 命 。
大 圣 不 恋 战 争 , 只 想 跑 起 来 。
将 近 洞 口 , 正 撞 着 康 、 张 、 姚 、 李 四 位 太 尉 , 郭 申 、 直 健 两 位 将 军 , 一 齐 率 领 众 人 挡 住 道 : 泼 猴 , 往 哪 里 走 ? 大 圣 惊 慌 了 , 就 把 金 钳 棒 捏 成 绣 花 针 , 藏 在 耳 里 。
转 身 一 变 , 变 成 一 个 麻 雀 儿 , 飞 到 树 梢 头 , 钉 住 了 。
那 六 兄 弟 惊 慌 张 张 , 前 后 寻 找 不 见 了 , 一 齐 喝 道 : 走 了 这 猴 精 , 走 了 这 猴 精 啊 正 吵 闹 的 地 方 , 真 君 来 了 , 问 道 : 兄 弟 们 , 赶 到 那 厢 不 见 了 。 众 神 说 : 刚 在 这 里 围 住 , 就 不 见 了 。
二 郎 圆 睁 着 凤 凰 的 眼 睛 观 看 , 看 见 大 圣 变 了 麻 雀 儿 , 钉 在 树 上 。
即 收 其 法 象 , 除 其 神 锋 , 撤 下 弹 弓 。
摇 动 身 体 一 变 , 变 成 一 个 鹰 儿 , 抖 开 翅 膀 , 飞 去 扑 打 。
大 圣 见 了 , 一 只 翅 膀 飞 去 , 变 成 一 只 大 老 , 冲 天 而 去 。
二 郎 见 了 , 急 忙 抽 动 羽 毛 , 动 身 一 变 , 变 成 一 只 大 海 鹤 , 钻 上 云 霄 来 。
大 圣 又 把 自 己 的 身 子 按 下 来 , 进 入 涧 中 , 变 成 一 个 鱼 儿 , 淬 进 水 里 。
二 郎 赶 到 涧 边 , 不 见 踪 迹 。
心 里 暗 想 道 : 这 一 定 会 下 水 去 了 , 一 定 会 变 成 鱼 虾 之 类 。
等 我 再 改 变 , 再 改 拿 他 。
果 然 一 变 , 变 成 一 个 鱼 鹰 儿 , 飘 荡 在 下 流 的 水 波 面 上 , 等 待 一 个 时 间 。
那 个 大 圣 变 成 鱼 儿 , 顺 着 水 流 正 好 游 玩 , 忽 然 看 见 一 只 飞 禽 , 似 青 , 毛 片 不 青 ; 似 鹭 , 头 上 没 有 头 毛 , 似 老 , 腿 又 不 红 。
二 郎 看 见 后 说 : 打 花 的 鱼 儿 , 象 鲤 鱼 , 尾 巴 不 红 ; 象 鱼 , 花 纹 的 鳞 甲 不 见 ; 象 黑 鱼 , 头 上 没 有 星 星 ; 象 鱼 , 壳 上 没 有 针 。
何 以 见 之 ?
他 赶 快 上 来 , 被 刷 得 一 只 嘴 啄 了 。
那 个 大 圣 人 就 把 它 扔 到 水 里 , 一 变 , 变 成 一 条 水 蛇 , 游 到 岸 边 , 钻 进 草 中 。
二 郎 于 是 骂 他 不 着 , 他 发 现 水 声 响 亮 , 发 现 一 条 蛇 咬 了 出 去 , 认 出 来 是 大 圣 。
急 忙 转 身 , 又 变 了 一 只 红 绣 顶 的 灰 鹤 , 伸 着 一 个 长 嘴 , 与 一 把 尖 头 铁 钳 子 相 似 , 径 直 来 吃 这 个 水 蛇 。
水 蛇 跳 一 跳 , 又 变 成 一 只 花 , 树 木 , 立 在 蓼 汀 上 。
二 郎 见 其 变 得 卑 贱 , 所 以 不 去 附 近 的 鸟 , 不 拘 于 鸾 、 凤 、 老 鹰 、 老 鹰 、 老 鹰 、 老 鹰 、 老 鹰 , 都 与 它 们 交 配 , 所 以 不 去 掉 它 。
立 即 出 现 原 来 的 身 子 , 快 走 了 将 要 离 开 , 拿 过 去 的 弹 弓 , 拽 满 了 , 一 个 弹 子 把 它 打 了 一 个 脚 脚 。
大 圣 乘 此 机 会 , 翻 下 山 崖 , 伏 在 山 崖 上 , 又 变 成 一 座 土 地 庙 , 大 张 着 口 , 好 像 是 个 庙 门 , 牙 齿 变 成 门 扇 , 舌 头 变 成 菩 萨 , 眼 睛 变 成 窗 , 只 有 尾 巴 不 好 收 拾 , 竖 在 后 面 , 变 成 一 根 旗 竿 。
真 君 赶 到 崖 下 , 看 不 见 打 倒 的 鸟 , 只 有 一 间 小 庙 。
急 忙 睁 开 凤 眼 , 仔 细 看 , 只 见 旗 竿 立 在 后 面 , 笑 着 说 : 是 这 了 , 他 现 在 又 在 那 里 哄 我 。
我 曾 经 见 过 庙 宇 , 更 不 曾 见 过 一 个 旗 竿 竖 在 后 面 的 。
断 绝 是 这 畜 生 弄 。
他 如 果 哄 我 进 去 , 他 就 一 口 咬 住 。
我 怎 么 肯 进 去 , 等 我 抽 拳 先 捣 坏 窗 , 然 后 踢 门 扇 。
大 圣 听 了 , 心 惊 地 说 : 好 恶 , 好 恶 , 好 恶 , 门 扇 是 我 的 牙 齿 , 窗 是 我 的 眼 睛 , 如 果 打 了 牙 , 捣 了 眼 , 又 怎 么 样 好 呢 ?
真 君 前 后 后 后 纷 纷 追 赶 , 只 见 四 位 太 尉 、 二 位 将 军 一 齐 拥 拥 着 来 , 说 : 兄 长 , 拿 住 大 圣 了 吗 ? 真 君 笑 着 说 : 那 猴 儿 刚 自 己 改 变 座 位 , 喧 闹 我 。
我 正 要 捣 他 的 窗 户 , 踢 他 的 门 扇 , 他 就 放 了 一 纵 , 又 渺 无 踪 迹 。
可 怪 , 可 怪 呀 众 人 都 愕 然 , 四 处 望 去 , 再 也 没 有 形 影 。
真 君 说 : 兄 弟 们 在 这 里 看 守 巡 逻 , 等 我 上 去 找 他 们 。
他 急 忙 放 开 身 子 驾 着 云 , 从 半 空 中 飞 起 。
他 看 见 那 个 李 天 王 高 擎 着 照 妖 镜 , 和 何 吒 站 在 云 端 , 真 君 说 : 天 王 , 曾 见 过 那 个 猴 王 吗 ? 天 王 说 : 不 曾 上 来 , 我 这 里 照 着 它 吗 ?
真 君 把 它 赌 变 化 , 戏 弄 神 通 , 拿 群 猴 一 事 说 完 。
又 说 : 他 变 了 庙 宇 , 正 打 的 地 方 , 就 走 了 。
李 天 王 听 了 话 , 又 拿 着 照 妖 镜 四 方 一 照 , 哈 哈 笑 着 说 : 真 君 , 快 去 吧 , 快 去 吧 。
那 猴 使 你 用 隐 身 法 , 跑 出 营 寨 , 到 你 那 里 灌 江 口 去 。
二 郎 听 了 , 立 即 取 出 神 锋 , 回 到 江 口 来 追 赶 。
又 说 : 那 大 圣 已 经 到 了 灌 江 口 , 转 身 一 变 , 变 成 了 二 郎 爷 爷 的 模 样 , 敲 下 云 头 , 径 直 进 入 庙 里 。
鬼 判 不 能 互 相 认 识 , 一 个 个 跪 着 头 迎 接 。
他 坐 在 中 间 , 点 点 点 点 点 点 点 点 点 香 火 , 看 见 李 虎 拜 还 的 三 牲 , 张 龙 许 下 的 保 福 , 赵 甲 求 子 的 文 书 , 钱 丙 告 病 的 良 愿 。
正 在 看 那 地 方 , 有 人 报 告 说 : 又 有 一 个 老 爷 来 了 。
众 鬼 判 断 急 忙 地 观 看 , 没 有 不 惊 心 的 。
真 君 又 说 : 有 个 什 么 齐 天 大 圣 , 刚 到 这 里 吗 ? 众 鬼 判 道 : 不 曾 见 过 什 么 大 圣 , 只 有 一 个 老 爷 爷 在 里 面 查 点 哩 。
真 君 撞 进 门 , 大 圣 看 见 了 , 拿 出 原 来 的 相 貌 说 : 郎 君 不 必 喧 哗 , 庙 宇 已 姓 孙 了 。
这 位 真 君 就 举 起 三 尖 两 刃 神 锋 , 劈 开 脸 就 砍 。
那 猴 王 使 你 身 法 , 让 你 超 过 神 锋 。
拿 出 那 个 绣 花 针 儿 , 打 开 一 个 账 子 , 碗 子 里 的 粗 细 , 赶 到 前 面 , 面 对 面 回 来 。
两 个 人 喧 哗 闹 闹 , 打 出 庙 门 , 半 路 雾 气 , 半 路 云 彩 , 一 边 走 一 边 作 战 , 又 打 到 花 果 山 。
惊 慌 得 到 那 四 大 天 王 等 众 , 更 加 紧 密 防 卫 。
康 、 张 太 尉 等 人 迎 接 真 君 , 合 心 协 力 , 把 美 猴 王 围 绕 不 再 问 题 。
说 : 大 力 鬼 王 既 已 调 了 真 君 和 六 兄 弟 , 率 兵 擒 拿 魔 鬼 去 后 , 又 上 界 回 来 报 告 。
玉 帝 和 观 音 菩 萨 、 王 母 和 众 仙 卿 , 正 在 灵 霄 殿 讲 话 , 说 : 既 然 是 二 郎 已 经 去 参 战 , 这 一 天 还 没 有 回 报 。
观 音 合 掌 说 : 我 请 陛 下 同 道 祖 出 南 天 门 外 , 亲 自 去 看 看 虚 实 如 何 ? 玉 帝 说 : 这 话 有 理 。
于 是 就 摆 开 车 驾 , 和 道 祖 、 观 音 、 王 母 一 起 来 到 南 天 门 。
早 晨 有 个 天 丁 、 力 士 接 着 , 开 门 远 远 地 观 看 。
只 见 诸 天 丁 布 置 罗 网 , 围 住 四 面 , 李 天 王 与 何 吒 擎 着 照 妖 镜 , 立 在 空 中 , 真 君 把 大 圣 围 绕 在 中 间 , 纷 纷 争 斗 啊 。
菩 萨 开 口 对 老 君 说 : 贫 僧 所 推 举 的 二 郎 神 如 何 , 果 真 有 神 通 , 已 经 把 那 大 圣 围 困 了 , 只 是 没 有 得 到 擒 拿 。
我 现 在 帮 助 他 一 功 , 决 定 要 拿 住 他 。
老 君 说 : 菩 萨 带 着 什 么 兵 器 , 怎 么 能 帮 助 他 呢 ? 菩 萨 说 : 我 把 那 净 瓶 、 杨 柳 扔 下 去 , 打 那 猴 头 , 即 使 不 能 打 死 , 也 打 一 跌 , 让 二 郎 小 圣 好 去 拿 它 。
老 君 说 : 你 的 瓶 是 个 磁 器 , 能 打 它 就 好 , 如 果 打 不 着 他 的 头 , 或 撞 着 他 的 铁 棒 , 却 不 打 碎 了 , 你 暂 且 不 要 动 手 , 等 我 老 君 帮 助 他 一 功 。
菩 萨 说 : 你 有 什 么 兵 器 ? 老 君 说 : 有 , 有 , 有 。
他 捋 起 衣 袖 , 从 左 臂 上 取 出 一 个 圈 子 , 说 道 : 这 件 兵 器 , 是 用 铁 炼 的 , 被 我 用 还 丹 点 成 , 养 成 了 一 身 灵 气 , 善 于 变 化 , 水 火 不 侵 , 又 能 包 藏 各 种 东 西 。
又 叫 金 钢 雕 , 又 叫 金 钢 套 。
当 年 我 经 过 函 关 时 , 化 胡 为 佛 , 非 常 惭 愧 他 。
早 晚 才 可 以 防 备 自 身 。
等 我 丢 下 去 , 打 他 一 下 。
说 完 , 从 天 门 上 往 下 一 , 滴 水 流 , 直 到 花 果 山 的 营 盘 里 。
猴 王 只 顾 苦 战 七 圣 , 却 不 知 道 天 上 坠 下 这 种 兵 器 , 打 中 了 天 灵 , 立 不 稳 脚 , 跌 了 一 下 , 爬 起 来 就 跑 。
被 二 郎 爷 爷 的 小 狗 赶 上 来 , 看 着 腿 肚 子 上 的 一 口 , 又 拽 了 一 跌 。
他 睡 倒 在 地 上 , 骂 道 : 这 个 亡 人 , 你 不 去 妨 碍 家 长 , 却 来 咬 老 孙 子 , 急 忙 翻 身 爬 不 起 来 , 被 七 圣 一 拥 拢 住 , 立 即 用 绳 子 绳 子 绑 住 , 使 用 钩 刀 把 你 的 骨 头 穿 穿 , 再 也 不 能 变 化 。
那 老 君 收 拾 了 金 钢 , 请 玉 帝 和 观 音 、 王 母 、 众 仙 等 , 一 起 回 到 灵 霄 殿 。
下 面 四 大 天 王 和 李 天 王 诸 神 都 收 兵 拔 寨 , 近 前 向 小 圣 祝 贺 喜 喜 , 都 说 : 这 是 小 圣 的 功 劳 。
小 圣 说 : 这 是 上 天 的 大 福 , 众 神 的 威 权 , 我 有 什 么 功 劳 ? 康 、 张 、 姚 、 李 说 : 兄 长 不 必 多 说 , 暂 且 押 这 个 奴 仆 去 上 界 见 玉 帝 , 请 圣 旨 发 送 去 。
真 君 说 : 贤 弟 , 你 们 没 有 接 受 上 天 的 赏 赐 , 不 能 当 面 见 玉 帝 。
让 天 甲 神 兵 押 着 , 我 同 天 王 等 上 界 回 来 。
你 们 率 领 众 人 在 这 里 搜 山 , 搜 刮 干 净 之 后 , 仍 然 回 到 灌 口 。
等 我 请 求 赏 赐 , 讨 伐 功 勋 , 回 来 共 同 欢 乐 。
四 位 太 尉 、 二 位 将 军 依 照 他 的 话 答 应 。
这 位 真 君 与 众 人 马 上 驾 着 云 头 , 唱 凯 歌 , 得 胜 朝 天 。
不 多 时 , 到 通 明 殿 外 面 。
天 师 启 奏 道 : 四 大 天 王 等 人 已 经 捉 住 妖 猴 齐 天 大 圣 了 , 来 此 听 宣 讲 。
玉 帝 传 旨 , 立 即 命 大 力 鬼 王 与 天 丁 等 人 , 押 到 斩 妖 台 , 将 这 个 人 的 尸 体 碎 割 。
唉 , 正 是 欺 现 在 遭 受 刑 法 法 令 的 痛 苦 , 英 雄 气 概 等 待 时 机 休 息 。
最 终 不 知 那 猴 王 的 性 命 如 何 , 暂 且 听 下 回 的 分 析 解 释 。
}\switchcolumn\flushpage  \begin{pinyinscope}{\myfontt \section{第七回}     八卦爐中逃大聖 五行山下定心猿

富貴功名,前緣分定,為人切莫欺心。正大光明,忠良善果彌深。些些狂妄天
加譴,眼前不遇待時臨。問東君,因甚如今禍害相侵?只為心高圖罔極,不分
上下亂規箴。

話表齊天大聖被眾天兵押去斬妖臺下,綁在降妖柱上,刀砍斧剁,槍刺劍刳,
莫想傷及其身。南斗星奮令火部眾神放火煨燒,亦不能燒著。又著雷部眾神以
雷屑釘打,越發不能傷損一毫。那大力鬼王與眾啟奏道:「萬歲,這大聖不知
是何處學得這護身之法,臣等用刀砍斧剁,雷打火燒,一毫不能傷損,卻如之
何?」玉帝聞言道:「這廝這等這等,如何處治?」太上老君即奏道:「那猴
吃了蟠桃,飲了御酒,又盜了仙丹。我那五壺丹,有生有熟,被他都吃在肚裏
。運用三昧火,鍛成一塊,所以渾做金鋼之軀,急不能傷。不若與老道領去,
放在八卦爐中,以文武火鍛煉,煉出我的丹來,他身自為灰燼矣。」玉帝聞言
,即教六丁、六甲將他解下,付與老君。老君領旨去訖。一壁廂宣二郎顯聖,
賞賜金花百朵、御酒百瓶、還丹百粒、異寶明珠、錦繡等件,教與義兄弟分享
。真君謝恩,回灌江口不題。

那老君到兜率宮,將大聖解去繩索,放了穿琵琶骨之器,推入八卦爐中,命看
爐的道人、架火的童子,將火搧起鍛煉。原來那爐是乾、坎、艮、震、巽、離
、坤、兌八卦。他即將身鑽在巽宮位下。巽乃風也,有風則無火。只是風攪得
煙來,把一雙眼火?紅了,弄做個老害病眼,故喚作「火眼金睛」。

真個光陰迅速,不覺七七四十九日,老君的火候俱全。忽一日,開爐取丹。那
大聖雙手侮著眼,正自揉搓流涕,只聽得爐頭聲響。猛睜睛看見光明,他就忍
不住,將身一縱,跳出丹爐,?喇一聲,蹬倒八卦爐,往外就走。慌得那架火
、看爐與丁甲一班人來扯,被他一個個都放倒,好似癲癇的白額虎,風狂的獨
角龍。老君趕上抓一把,被他一捽,捽了個倒栽蔥,脫身走了。即去耳中掣出
如意棒,迎風幌一幌,碗來粗細,依然拿在手中,不分好歹,卻又大亂天宮,
打得那九曜星閉門閉戶,四天王無影無形。好猴精,有詩為證。詩曰:
混元體正合先天,萬劫千番只自然。
    渺渺無為渾太乙,如如不動號初玄。
    爐中久煉非鉛汞,物外長生是本仙。
    變化無窮還變化,三皈五戒總休言。
又詩:
    一點靈光徹太虛,那條拄杖亦如之。
    或長或短隨人用,橫豎橫排任卷舒。
  又詩:
    猿猴道體配人心,心即猿猴意思深。
    大聖齊天非假論,官封弼馬豈知音。
    馬猿合作心和意,緊縛牢拴莫外尋。
    萬相歸真從一理,如來同契住雙林。

這一番,那猴王不分上下,使鐵棒東打西敵,更無一神可擋,只打到通明殿裏
,靈霄殿外。幸有佑聖真君的佐使王靈官執殿,他看大聖縱橫,掣金鞭近前擋
住道:「潑猴何往?有吾在此,切莫猖狂。」這大聖不由分說,舉棒就打﹔那
靈官鞭起相迎。兩個在靈霄殿前廝渾一處,好殺:
赤膽忠良名譽大,欺天誑上聲名壞。一低一好幸相持,豪傑英雄同賭賽。鐵棒
兇,金鞭快,正直無私怎忍耐?這個是太乙雷聲應化尊,那個是齊天大聖猿猴
怪。金鞭鐵棒兩家能,都是神宮仙器械。今日在靈霄寶殿弄威風,各展雄才真
可愛。一個欺心要奪斗牛宮,一個竭力匡扶玄聖界。苦爭不讓顯神通,鞭棒往
來無勝敗。

他兩個鬥在一處,勝敗未分。早有佑聖真君又差將佐發文到雷府,調三十六員
雷將齊來,把大聖圍在垓心,各騁兇惡鏖戰。那大聖全無一毫懼色,使一條如
意棒,左遮右擋,後架前迎。一時見那眾雷將的刀槍劍戟、鞭簡撾鎚、鉞斧金
瓜、旄鐮月鏟來的甚緊,他即搖身一變:變做三頭六臂﹔把如意棒幌一幌,變
作三條﹔六隻手使開三條棒,好便似紡車兒一般,滴流流,在那垓心裏飛舞。
眾雷神莫能相近。真個是:
圓陀陀,光灼灼,亙古常存人怎學?入火不能焚,入水何曾溺?光明一顆摩尼
珠,劍戟刀槍傷不著。也能善,也能惡,眼前善惡憑他作。善時成佛與成仙,
惡處披毛並帶角。無窮變化鬧天宮,雷將神兵不可捉。

當時眾神把大聖攢在一處,卻不能近身,亂嚷亂鬥。早驚動玉帝,遂傳旨著遊
奕靈官同翊聖真君上西方請佛老降伏。

那二聖得了旨,徑到靈山勝境雷音寶剎之前,對四金剛、八菩薩禮畢,即煩轉
達。眾神隨至寶蓮臺下啟知,如來召請。二聖禮佛三匝,侍立臺下。如來問:
「玉帝何事,煩二聖下臨?」二聖即啟道:「向時花果山產一猴,在那裏弄神
通,聚眾猴攪亂世界。玉帝降招安旨,封為弼馬溫,他嫌官小反去。當遣李天
王、哪吒太子擒拿未獲,復招安他,封做齊天大聖,先有官無祿。著他代管蟠
桃園,他即偷桃﹔又走至瑤池,偷殽、偷酒,攪亂大會﹔仗酒又暗入兜率宮,
偷老君仙丹,反出天宮。玉帝復遣十萬天兵,亦不能收伏。後觀世音舉二郎真
君同他義兄弟追殺,他變化多端,虧老君拋金鋼琢打中,二郎方得拿住。解赴
御前,即命斬之,刀砍斧剁,火燒雷打,俱不能傷。老君奏准領去,以火鍛煉
。四十九日開鼎,他卻又跳出八卦爐,打退天丁,徑入通明殿裏,靈霄殿外。
被佑聖真君的佐使王靈官擋住苦戰,又調三十六員雷將把他困在垓心,終不能
相近。事在緊急,因此玉帝特請如來救駕。」如來聞說,即對眾菩薩道:「汝
等在此穩坐法堂,休得亂了禪位,待我煉魔救駕去來。」

如來即喚阿儺、迦葉二尊者相隨,離了雷音,徑至靈霄門外。忽聽得喊聲振耳
,乃三十六員雷將圍困著大聖哩。佛祖傳法旨:「教雷將停息干戈,放開營所
,叫那大聖出來,等我問他有何法力。」眾將果退。大聖也收了法象,現出原
身近前,怒氣昂昂,厲聲高叫道:「你是那方善士,敢來止住刀兵問我?」如
來笑道:「我是西方極樂世界釋迦牟尼尊者。南無阿彌陀佛!今聞你猖狂村野
,屢反天宮,不知是何方生長,何年得道,為何這等暴橫?」大聖道:「我本:
    天地生成靈混仙,花果山中一老猿。
    水簾洞裏為家業,拜友尋師悟太玄。
    煉就長生多少法,學來變化廣無邊。
    因在凡間嫌地窄,立心端要住瑤天。
    靈霄寶殿非他久,歷代人王有分傳。
    強者為尊該讓我,英雄只此敢爭先。」

佛祖聽言,呵呵冷笑道:「你那廝乃是個猴子成精,焉敢欺心,要奪玉皇上帝
尊位?他自幼修持,苦歷過一千七百五十劫。每劫該十二萬九千六百年,你算
他該多少年數,方能享受此無極大道?你那個初世為人的畜生,如何出此大言
?不當人子,不當人子,折了你的壽算。趁早皈依,切莫胡說。但恐遭了毒手
,性命頃刻而休,可惜了你的本來面目。」大聖道:「他雖年幼修長,也不應
久占在此。常言道:『皇帝輪流做,明年到我家。』只教他搬出去,將天宮讓
與我,便罷了﹔若還不讓,定要攪攘,永不清平。」佛祖道:「你除了長生變
化之法,再有何能,敢占天宮勝境?」大聖道:「我的手段多哩:我有七十二
般變化,萬劫不老長生﹔會駕觔斗雲,一縱十萬八千里。如何坐不得天位?」
佛祖道:「我與你打個賭賽:你若有本事,一觔斗打出我這右手掌中,算你贏
,再不用動刀兵,苦爭戰,就請玉帝到西方居住,把天宮讓你﹔若不能打出手
掌,你還下界為妖,再修幾劫,卻來爭吵。」那大聖聞言,暗笑道:「這如來
十分好獃。我老孫一觔斗去十萬八千里,他那手掌方圓不滿一尺,如何跳不出
去?」急發聲道:「既如此說,你可做得主張?」佛祖道:「做得,做得。」
伸開右手,卻似個荷葉大小。

那大聖收了如意棒,抖擻神威,將身一縱,站在佛祖手心裏,卻道聲:「我出
去也。」你看他一路雲光,無形無影去了。佛祖慧眼觀看,見那猴王風車子一
般相似不住,只管前進。大聖行時,忽見有五根肉紅柱子,撐著一股青氣。他
道:「此間乃盡頭路了。這番回去,如來作證,靈霄宮定是我坐也。」又思量
說:「且住,等我留下些記號,方好與如來說話。」拔下一根毫毛,吹口仙氣
,叫:「變!」變作一管濃墨雙毫筆,在那中間柱子上寫一行大字云:「齊天
大聖,到此一遊。」寫畢,收了毫毛。又不莊尊,卻在第一根柱子根下撒了一
泡猴尿。翻轉觔斗雲,徑回本處,站在如來掌內道:「我已去,今來了。你教
玉帝讓天宮與我。」

如來罵道:「我把你這個尿精猴子,你正好不曾離了我掌哩。」大聖道:「你
是不知。我去到天盡頭,見五根肉紅柱,撐著一股青氣,我留個記在那裏,你
敢和我同去看麼?」如來道:「不消去,你只自低頭看看。」那大聖睜圓火眼
金睛,低頭看時,原來佛祖右手中指寫著「齊天大聖,到此一遊」。大指丫裏
,還有些猴尿臊氣。大聖吃了一驚道:「有這等事?有這等事?我將此字寫在
撐天柱子上,如何卻在他手指上?莫非有個未卜先知的法術?我決不信,不信
。等我再去來。」

好大聖,急縱身又要跳出。被佛祖翻掌一撲,把這猴王推出西天門外,將五指
化作金、木、水、火、土五座聯山,喚名「五行山」,輕輕的把他壓住。眾雷
神與阿儺、迦葉一個個合掌稱揚道:「善哉,善哉!
    當年卵化學為人,立志修行果道真。
    萬劫無移居勝境,一朝有變散精神。
    欺天罔上思高位,凌聖偷丹亂大倫。
    惡貫滿盈今有報,不知何日得翻身。」

如來佛祖殄滅了妖猴,即喚阿儺、迦葉同轉西方極樂世界。時有天蓬、天佑急
出靈霄寶殿道:「請如來少待,我主大駕來也。」佛祖聞言,回首瞻仰。須臾
,果見八景鸞輿,九光寶蓋,聲奏玄歌妙樂,詠哦無量神章,散寶花,噴真香
,直至佛前謝曰:「多蒙大法收殄妖邪,望如來少停一日,請諸仙做一會筵奉
謝。」如來不敢違悖,即合掌謝道:「老僧承大天尊宣命來此,有何法力?還
是天尊與眾神洪福。敢勞致謝?」玉帝傳旨,即著雷部眾神,分頭請三清、四
御、五老、六司、七元、八極、九曜、十都、千真、萬聖來此赴會,同謝佛恩
。又命四大天師、九天仙女,大開玉京金闕、太玄寶宮、洞陽玉館,請如來高
座七寶靈臺,調設各班坐位,安排龍肝鳳髓,玉液蟠桃。

不一時,那玉清元始天尊、上清靈寶天尊、太清道德天尊、五?真君、五斗星
君、三官四聖、九曜真君、左輔、右弼、天王、哪吒,玄虛一應靈通,對對旌
旗,雙雙幡蓋,都捧著明珠異寶,壽果奇花,向佛前拜獻曰:「感如來無量法
力,收伏妖猴。蒙大天尊設宴,呼喚我等皆來陳謝。請如來將此會立一名如何
?」如來領眾神之託曰:「今欲立名,可作個安天大會。」各仙老異口同聲,
俱道:「好個『安天大會』!好個『安天大會』!」言訖,各坐座位,走斝傳
觴,簪花鼓瑟,果好會也。有詩為證。詩曰:
    宴設蟠桃猴攪亂,安天大會勝蟠桃。
    龍旗鸞輅祥光藹,寶節幢幡瑞氣飄。
    仙樂玄歌音韻美,鳳簫玉管響聲高。
    瓊香繚繞群仙集,宇宙清平賀聖朝。

眾皆暢然喜會,只見王母娘娘引一班仙子、仙娥、美姬、美女飄飄蕩蕩舞向佛
前,施禮曰:「前被妖猴攪亂蟠桃一會,請眾仙眾佛俱成功。今蒙如來大法鍊
鎖頑猴,喜慶『安天大會』,無物可謝,今是我淨手親摘大株蟠桃數顆奉獻。」
真個是:
    半紅半綠噴甘香,艷麗仙根萬載長。
    堪笑武陵源上種,爭如天府更奇強。
    紫紋嬌嫩寰中少,緗核清甜世莫雙。
    延壽延年能易體,有緣食者自非常。

佛祖合掌向王母謝訖。王母又著仙姬、仙子唱的唱,舞的舞。滿會群仙又皆賞
讚。正是:
縹緲天香滿座,繽紛仙蕊仙花。玉京金闕大榮華。異品奇珍無價。對對與天齊
壽,雙雙萬劫增加。桑田滄海任更差。他自無驚無訝。

王母正著仙姬、仙子歌舞,觥籌交錯,不多時,忽又聞得:
    一陣異香來鼻噢,驚動滿堂星與宿。
    天仙佛祖把杯停,各各抬頭迎目候。
    霄漢中間現老人,手捧靈芝飛藹繡。
    葫蘆藏蓄萬年丹,寶籙名書千紀壽。
    洞裏乾坤任自由,壺中日月隨成就。
    遨遊四海樂清閑,散淡十洲容輻輳。
    曾赴蟠桃醉幾遭,醒時明月還依舊。
    長頭大耳短身軀,南極之方稱老壽。

壽星又到。見玉帝禮畢,又見如來,申謝曰:「始聞那妖猴被老君引至兜率宮鍛
煉,以為必致平安,不期他又反出。幸如來善伏此怪,設宴奉謝,故此聞風而來
。更無他物可獻,特具紫芝瑤草、碧藕金丹奉上。」詩曰:
    碧藕金丹奉釋迦,如來萬壽若恆沙。
    清平永樂三乘錦,康泰長生九品花。
    無相門中真法主,色空天上是仙家。
    乾坤大地皆稱祖,丈六金身福壽賒。

如來忻然領謝。壽星得座,依然走斝傳觴。只見赤腳大仙又至,向玉帝前頫?禮
畢,又對佛祖謝道:「深感法力,降伏妖猴。無物可以表敬,特具交梨二顆、火
棗數枚奉獻。」詩曰:
    大仙赤腳棗梨香,敬獻彌陀壽算長。
    七寶蓮臺山樣穩,千金花座錦般粧。
    壽同天地言非謬,福比洪波話豈狂。
    福壽如期真個是,清閑極樂那西方。

如來又稱謝了,叫阿儺、迦葉將各所獻之物,一一收起,方向玉帝前謝宴。眾各
酩酊。只見個巡視靈官來報道:「那大聖伸出頭來了。」佛祖道:「不妨,不妨
。」袖中只取出一張帖子,上有六個金字:「唵嘛呢叭吽」。遞與阿儺,叫貼在
那山頂上。這尊者即領帖子,拿出天門,到那五行山頂上,緊緊的貼在一塊四方
石上,那座山即生根合縫。可運用呼吸之氣,手兒爬出,可以搖掙搖掙。阿儺回
報道:「已將帖子貼了。」

如來即辭了玉帝眾神,與二尊者出天門之外。又發一個慈悲心,念動真言咒語,
將五行山召一尊土地神祗,會同五方揭諦,居住此山監押。但他饑時,與他鐵丸
子吃﹔渴時,與他溶化的銅汁飲。待他災愆滿日,自有人救他。正是:
    妖猴大膽反天宮,卻被如來伏手降。
    渴飲溶銅捱歲月,饑餐鐵彈度時光。
    天災苦困遭磨折,人事淒涼喜命長。
    若得英雄重展掙,他年奉佛上西方。
  又詩曰:
    伏逞豪強大勢興,降龍伏虎弄乖能。
    偷桃偷酒遊天府,受籙承恩在玉京。
    惡貫滿盈身受困,善根不絕氣還昇。
    果然脫得如來手,且待唐朝出聖僧。

    畢竟不知向後何年何月方滿災殃,且聽下回分解。





}  \end{pinyinscope}\switchcolumn{\myfontc \section{第 七 回} 《 八 卦 炉 中 逃 大 圣 五 行 山 下 定 心 猿 富 贵 功 名 , 前 缘 分 定 , 为 人 切 莫 欺 心 。
正 大 光 明 , 忠 良 善 果 更 深 。
这 些 狂 妄 妄 妄 , 上 天 谴 责 他 , 眼 前 不 遇 , 等 待 时 机 到 来 。
问 东 君 , 因 为 如 今 祸 害 相 互 侵 害 , 只 是 为 了 心 思 高 远 谋 划 没 有 尽 头 , 不 分 上 下 乱 规 劝 。
说 : 齐 天 大 圣 被 众 天 兵 把 他 押 到 斩 妖 台 下 , 把 他 绑 在 降 妖 柱 上 , 用 刀 砍 斧 砍 , 用 刀 刺 剑 , 没 想 到 他 的 身 体 受 伤 。
南 斗 星 奋 力 命 令 火 部 的 众 神 放 火 烧 , 也 不 能 烧 着 。
又 著 雷 部 众 神 用 雷 屑 钉 打 , 越 发 不 能 损 伤 一 毫 。
那 大 力 鬼 王 和 众 人 启 奏 道 : 万 岁 , 这 个 大 圣 不 知 是 什 么 地 方 学 到 这 样 的 保 身 法 术 , 我 们 用 刀 、 斧 、 刀 、 刀 、 刀 、 刀 、 刀 、 刀 、 刀 、 刀 、 刀 、 刀 、 打 、 火 烧 , 一 点 也 不 能 伤 害 , 又 怎 么 办 呢 ? 玉 帝 听 了 后 说 道 : 这 些 这 些 人 , 怎 么 治 呢 ? 太 上 老 君 立 即 上 奏 说 : 那 猴 吃 了 蟠 桃 , 喝 了 御 酒 , 又 偷 了 仙 丹 ?
我 的 五 壶 丹 , 有 生 有 熟 , 被 他 都 吃 在 肚 里 。
用 三 昧 火 , 锻 成 一 块 , 所 以 浑 成 金 钢 的 形 体 , 紧 急 不 能 伤 害 。
不 如 把 老 道 带 去 , 放 在 八 卦 炉 里 , 用 文 武 之 火 炼 出 我 的 丹 来 , 他 自 己 就 成 了 灰 烬 了 。
玉 帝 听 了 这 话 , 就 教 六 丁 、 六 甲 把 它 解 下 来 , 交 给 老 君 。
老 君 领 旨 离 去 。
一 壁 厢 宣 二 郎 显 圣 , 赏 赐 金 花 百 朵 、 御 酒 百 瓶 、 还 丹 百 斛 、 异 宝 明 珠 、 锦 绣 等 物 , 教 他 与 义 兄 弟 分 享 。
真 君 谢 恩 , 回 去 灌 溉 江 口 没 有 题 目 。
又 将 大 圣 解 去 的 绳 索 , 放 下 了 解 琵 琶 骨 的 器 具 , 推 进 八 卦 炉 里 , 让 看 火 炉 的 童 子 , 把 火 烧 起 来 。
原 来 的 炉 是 乾 、 坎 、 艮 、 震 、 巽 、 离 、 坤 、 兑 八 卦 。
即 将 自 己 的 身 子 钻 到 了 巽 宫 的 位 子 下 面 。
巽 是 风 , 有 风 就 没 有 火 。
只 是 风 搅 得 烟 来 , 把 一 双 眼 火 烧 红 了 , 弄 作 个 老 害 病 眼 , 所 以 叫 它 火 眼 金 睛 。
真 的 光 阴 迅 速 , 不 知 不 觉 有 七 七 四 十 九 天 , 老 君 的 火 候 全 都 保 全 了 。
忽 然 有 一 天 , 开 炉 取 丹 。
那 大 圣 双 手 侮 辱 着 眼 睛 , 正 自 己 擦 着 眼 泪 , 只 听 到 炉 头 的 声 音 。
俄 然 , 忽 然 不 见 , 忽 然 不 见 , 忽 然 不 见 , 忽 然 不 见 。
忽 然 有 一 个 架 火 、 看 炉 和 丁 甲 一 班 人 来 搅 , 被 他 一 个 个 都 放 倒 , 好 像 是 癫 狂 的 白 额 虎 , 风 狂 的 独 角 龙 。
老 君 赶 上 去 抓 住 一 把 , 被 他 咬 了 一 个 倒 地 上 的 葱 , 脱 身 逃 走 了 。
随 即 从 耳 中 拿 出 如 意 棒 , 迎 风 打 一 个 扇 子 , 碗 里 的 粗 细 , 仍 然 拿 在 手 中 , 不 分 好 歹 , 却 又 大 乱 天 宫 , 打 得 那 九 曜 星 闭 门 闭 户 , 四 天 王 无 影 无 形 。
喜 欢 猴 精 , 有 诗 作 证 据 。
《 诗 经 》 说 : 混 元 体 正 合 先 天 , 万 劫 千 番 只 自 然 。
茫 茫 无 为 浑 太 乙 , 如 不 动 号 初 玄 。
炉 中 长 久 炼 炼 不 是 铅 汞 , 物 外 长 生 就 是 仙 。
变 化 无 穷 还 有 变 化 , 三 念 五 戒 总 不 说 话 。
又 有 诗 : 一 点 灵 光 彻 太 虚 , 那 支 拄 杖 也 是 这 样 。
或 长 或 短 , 随 随 人 使 用 , 横 、 横 、 排 , 任 凭 卷 卷 舒 展 。
又 有 一 首 诗 : 猿 猴 道 体 配 人 心 , 心 就 猿 猴 意 思 深 。
大 圣 齐 天 非 假 论 , 官 封 弼 马 岂 知 音 。
马 猿 合 作 心 和 意 , 紧 锁 牢 锁 莫 外 寻 。
万 相 归 真 从 一 理 , 如 来 同 契 住 双 林 。
这 一 次 , 那 猴 王 不 分 上 下 , 使 铁 棒 东 击 西 敌 , 更 没 有 一 个 神 可 以 抵 挡 , 只 打 到 通 明 殿 里 , 灵 霄 殿 外 。
幸 有 佑 圣 真 君 的 佐 使 王 灵 官 执 殿 , 他 看 大 圣 纵 横 , 拉 着 金 鞭 , 在 前 面 挡 住 道 : 泼 猴 去 哪 里 , 有 我 在 这 里 , 切 莫 狂 妄 。
这 位 大 圣 不 由 分 说 , 举 棒 就 打 , 那 灵 官 鞭 子 起 来 迎 接 。
两 个 人 在 灵 霄 殿 前 一 个 混 沌 , 好 杀 : 赤 胆 忠 良 , 名 誉 大 , 欺 骗 皇 上 , 声 名 败 坏 。
一 个 地 位 低 下 , 一 个 人 喜 好 , 侥 幸 相 持 , 豪 杰 英 雄 同 赌 赛 。
铁 棒 凶 , 金 鞭 快 , 正 直 无 私 , 怎 么 忍 耐 ? 这 是 太 乙 雷 声 应 化 尊 , 那 是 齐 天 大 圣 猿 猴 怪 。
金 鞭 铁 棒 两 家 能 , 都 是 神 宫 的 仙 器 。
今 日 在 灵 霄 宝 殿 玩 弄 威 风 , 各 展 雄 才 , 真 可 爱 。
一 个 欺 心 夺 取 斗 牛 宫 , 一 个 竭 力 匡 扶 玄 圣 界 。
苦 苦 争 辩 不 让 , 显 示 神 通 , 鞭 打 往 来 没 有 胜 负 。
其 两 个 人 争 斗 在 一 个 地 方 , 胜 负 还 没 有 分 。
早 有 佑 圣 真 君 , 又 派 将 佐 发 文 到 雷 府 , 调 三 十 六 员 雷 将 一 齐 来 , 把 大 圣 包 围 在 心 , 各 自 逞 凶 作 恶 。
那 大 圣 完 全 没 有 一 丝 一 毫 害 怕 的 神 色 , 让 一 条 如 意 棒 , 左 边 遮 挡 右 边 挡 住 , 后 架 前 边 迎 接 。
一 时 间 , 他 看 见 那 些 雷 将 的 刀 枪 剑 戟 、 鞭 简 锤 、 斧 金 瓜 、 镰 、 月 、 镰 、 镰 、 月 、 箕 、 、 、 、 月 、 、 、 、 簸 箕 、 簸 箕 等 , 变 成 三 头 六 臂 , 又 把 如 意 棒 打 起 来 , 变 成 三 条 棍 子 ; 六 只 手 让 他 打 开 三 条 棒 子 , 就 好 像 纺 车 儿 一 样 , 流 淌 着 , 在 那 的 心 里 飞 舞 。
众 多 的 雷 神 都 不 能 相 近 。
真 的 是 : 圆 圆 的 珠 子 , 光 辉 灿 烂 , 千 古 永 远 存 在 , 人 怎 么 能 学 习 , 进 入 火 中 不 能 烧 毁 , 进 入 水 里 怎 么 能 淹 死 呢 ? 光 明 的 一 颗 摩 尼 珠 , 剑 戟 刀 枪 都 伤 不 着 。
也 能 做 到 善 事 , 也 能 做 到 恶 事 , 眼 前 的 善 恶 都 凭 借 别 人 作 出 来 。
善 的 时 候 是 成 佛 和 成 仙 , 恶 的 地 方 都 是 披 毛 并 带 角 。
无 穷 变 化 闹 天 宫 , 雷 将 神 兵 不 可 捉 。
当 时 众 神 把 大 圣 藏 在 一 个 地 方 , 却 不 能 靠 近 自 己 的 身 体 , 乱 吵 乱 斗 。
早 就 惊 动 了 玉 帝 , 于 是 传 圣 旨 著 游 奕 灵 官 , 同 翊 圣 真 君 上 西 方 , 请 佛 老 降 服 。
那 两 位 圣 人 得 到 了 圣 旨 , 径 直 到 灵 山 胜 境 雷 音 宝 刹 的 前 面 , 对 四 位 金 刚 、 八 位 菩 萨 礼 拜 完 毕 , 就 请 转 送 给 他 们 。
众 神 随 从 到 宝 莲 台 下 禀 报 , 如 来 召 请 。
二 位 圣 人 拜 佛 三 匝 , 在 台 下 侍 立 。
如 来 问 道 : 玉 帝 是 什 么 事 , 麻 烦 二 位 圣 人 来 到 我 这 里 呢 ? 二 位 圣 人 就 启 奏 道 : 刚 才 花 果 山 出 产 一 个 猴 子 , 在 那 里 弄 神 通 , 聚 集 众 猴 子 搅 乱 世 界 。
玉 帝 降 下 招 安 的 旨 意 , 封 他 为 弼 马 温 , 其 他 嫌 官 小 反 而 离 去 。
应 当 派 遣 李 天 王 、 何 吒 太 子 擒 拿 不 到 , 又 招 安 他 , 封 他 为 齐 天 大 圣 , 先 前 有 官 没 有 俸 禄 。
著 他 代 管 蟠 桃 园 , 他 就 偷 桃 ; 又 跑 到 瑶 池 , 偷 酒 , 搅 乱 大 会 ; 仗 着 酒 又 暗 中 进 入 兜 率 宫 , 偷 老 君 仙 丹 , 反 而 出 了 天 宫 。
玉 帝 又 派 遣 十 万 天 兵 , 也 不 能 收 伏 。
后 来 观 世 音 推 举 二 郎 真 君 与 他 义 兄 弟 一 起 追 杀 , 他 变 化 多 端 , 亏 欠 老 君 抛 金 钢 琢 打 中 , 二 郎 才 得 以 捉 住 。
郭 解 赶 到 皇 帝 面 前 , 立 即 命 令 把 他 斩 首 , 刀 砍 斧 砍 , 火 烧 雷 击 , 都 不 能 伤 害 。
老 君 上 奏 让 王 准 带 走 , 用 火 炼 炼 。
四 十 九 日 打 开 鼎 , 他 又 跳 出 八 卦 炉 , 打 退 了 天 丁 , 径 直 进 入 通 明 殿 里 , 灵 霄 殿 外 。
其 佐 使 者 王 灵 官 阻 止 苦 战 , 又 调 三 十 六 员 雷 将 , 囚 禁 在 心 , 始 终 不 能 接 近 。
事 情 紧 急 , 因 此 玉 帝 特 意 请 求 如 来 救 驾 。
如 来 听 了 这 话 , 就 对 众 位 菩 萨 说 : 你 们 在 这 里 稳 稳 地 坐 在 法 堂 上 , 不 要 乱 了 禅 位 , 等 我 炼 魔 救 驾 去 来 。
如 来 就 叫 来 阿 、 迦 叶 二 位 尊 者 跟 随 , 离 开 雷 音 , 径 直 到 灵 霄 门 外 。
忽 然 听 到 喊 声 震 耳 , 原 来 是 三 十 六 员 雷 将 包 围 着 大 圣 哩 。
佛 祖 传 授 法 旨 说 : 教 雷 将 停 止 干 戈 , 放 开 营 地 , 叫 那 大 圣 出 来 , 等 我 问 他 有 什 么 法 力 ?
众 将 果 然 撤 退 。
大 圣 收 藏 了 法 象 , 出 现 在 原 身 近 前 , 怒 气 昂 昂 , 厉 声 大 叫 道 : 你 是 那 方 的 善 士 , 怎 么 敢 来 阻 止 刀 兵 问 我 ? 如 来 笑 着 说 : 我 是 西 方 极 乐 世 界 的 释 迦 牟 尼 尊 者 。
南 无 阿 弥 陀 佛 , 现 在 听 说 你 在 村 野 猖 狂 , 屡 次 回 到 天 宫 , 不 知 道 你 是 什 么 地 方 生 长 , 何 年 得 道 , 为 什 么 这 样 的 暴 虐 横 行 ? 大 圣 说 : 我 本 来 是 天 地 生 成 灵 混 仙 , 花 果 山 中 的 一 只 老 猿 。
在 水 帘 洞 里 做 家 业 , 拜 友 寻 师 , 悟 得 了 《 太 玄 》 。
炼 就 长 生 多 少 法 则 , 学 习 变 化 广 无 边 。
因 为 在 凡 俗 之 中 , 嫌 地 狭 , 立 心 端 要 住 瑶 天 。
灵 霄 宝 殿 不 是 其 他 的 时 间 已 久 , 历 代 的 人 王 都 有 分 别 传 授 。
强 者 为 尊 , 应 该 让 我 , 英 雄 只 有 这 个 人 , 敢 争 先 。
佛 祖 听 了 , 呵 呵 冷 笑 道 : 你 那 个 是 个 猴 子 成 精 , 怎 么 敢 欺 心 , 要 夺 玉 皇 上 帝 的 尊 位 呢 他 自 幼 修 行 , 苦 苦 经 历 了 一 千 七 百 五 十 劫 。
每 劫 应 当 十 二 万 九 千 六 百 年 , 你 估 计 他 应 该 多 少 年 数 , 才 能 享 受 这 无 极 大 道 , 你 那 个 初 生 为 人 的 畜 生 , 为 什 么 说 出 这 样 的 大 话 呢 ? 不 当 人 的 儿 子 , 不 当 人 的 儿 子 , 折 断 了 你 的 寿 命 。
趁 早 皈 依 佛 法 , 切 莫 胡 说 。
但 恐 怕 遭 到 毒 手 , 性 命 顷 刻 就 会 停 止 , 可 惜 你 的 本 来 面 目 。
大 圣 说 : 他 虽 然 年 幼 修 养 , 也 不 应 该 长 久 占 卜 在 这 里 。
他 常 说 : 皇 帝 轮 流 做 , 明 年 到 我 家 。
只 教 他 搬 出 去 , 把 天 宫 让 给 我 , 就 算 罢 了 ; 如 果 还 不 让 , 一 定 要 搅 乱 , 永 不 清 平 。
佛 祖 说 : 你 除 了 长 生 变 化 之 法 , 再 有 什 么 能 力 , 敢 占 卜 天 宫 胜 境 ? 大 圣 说 : 我 的 手 段 多 哩 , 我 有 七 十 二 种 变 化 , 万 劫 不 老 长 生 ; 会 驾 斗 云 , 一 纵 十 万 八 千 里 。
佛 祖 说 : 我 与 你 打 个 赌 赛 , 你 若 有 本 事 , 一 斗 打 出 我 的 右 手 掌 中 , 算 你 赢 , 再 不 用 刀 兵 , 苦 苦 争 战 , 就 请 玉 帝 到 西 方 居 住 , 把 天 宫 让 给 你 ; 如 果 不 能 打 出 手 掌 , 你 还 在 下 界 成 妖 , 再 修 几 劫 , 还 来 争 吵 。
那 大 圣 听 了 这 话 , 暗 笑 着 说 : 这 位 如 来 十 分 好 吗 ?
我 的 老 孙 子 一 斗 距 十 万 八 千 里 , 他 的 手 掌 方 圆 不 满 一 尺 , 为 什 么 跳 不 出 去 , 急 忙 发 声 说 : 既 然 这 样 说 , 你 可 以 做 得 主 张 ? 佛 祖 说 : 做 得 , 做 得 , 做 得 。
他 伸 开 右 手 , 却 像 一 个 荷 叶 的 大 小 。
那 大 圣 收 了 如 意 棒 , 振 奋 神 威 , 将 身 体 放 出 去 , 站 在 佛 祖 的 手 心 里 , 又 说 道 : 我 出 去 。
你 看 他 一 路 上 的 云 光 , 无 形 无 影 去 了 。
佛 祖 慧 眼 观 看 , 见 那 猴 王 风 车 子 一 样 , 相 似 不 住 , 只 管 往 前 走 。
大 圣 出 行 时 , 忽 然 看 见 有 五 根 肉 红 色 的 柱 子 , 撑 着 一 股 青 色 的 云 气 。
他 说 : 这 里 是 尽 头 路 了 。
这 次 回 去 , 如 来 作 证 , 灵 霄 宫 一 定 是 我 坐 的 。
又 思 量 说 : 暂 且 住 , 等 我 留 下 记 号 , 才 好 和 如 来 说 话 。
他 拔 下 一 根 毫 毛 , 吹 着 一 口 仙 气 , 喊 道 : 变 变 了 一 支 浓 墨 双 毫 笔 , 在 那 中 间 柱 子 上 写 着 一 行 大 字 : 齐 天 大 圣 , 到 此 一 游 。
书 写 完 后 , 收 起 了 毫 毛 。
又 不 庄 尊 , 却 在 第 一 根 柱 子 根 下 撒 了 一 泡 猴 尿 。
翻 转 着 斗 云 , 径 回 到 原 来 的 地 方 , 站 在 如 来 掌 里 说 : 我 已 经 离 开 , 现 在 来 了 。
你 教 玉 帝 让 天 宫 给 我 。
如 来 骂 道 : 我 把 你 这 个 尿 精 猴 子 , 你 正 好 不 曾 离 开 我 的 手 掌 哩 !
大 圣 说 : 你 是 不 知 道 的 。
我 走 到 天 尽 头 , 看 见 五 根 肉 红 柱 , 撑 着 一 股 青 气 , 我 留 下 记 在 那 里 , 你 敢 和 我 一 同 去 看 吗 ? 如 来 说 : 不 消 去 , 你 只 自 己 低 头 看 看 。
那 大 圣 眼 睛 睁 着 火 眼 金 睛 , 低 头 看 时 , 原 来 佛 祖 右 手 中 指 着 : 齐 天 大 圣 , 到 此 一 游 。
大 指 梢 里 , 还 有 猴 尿 臊 气 。
大 圣 吃 了 一 惊 , 说 : 有 这 样 的 事 , 有 这 样 的 事 , 我 把 这 个 字 写 在 撑 天 柱 子 上 , 为 什 么 还 在 他 手 指 上 呢 ? 莫 不 是 有 未 卜 先 知 的 法 术 , 我 决 不 信 , 不 信 。
等 我 再 去 吧 。
大 圣 人 , 急 忙 放 开 身 子 , 又 要 跳 出 去 。
又 命 其 人 , 以 其 人 为 之 , 以 其 人 为 之 , 以 其 人 为 之 , 以 其 人 为 之 。
众 位 雷 神 和 阿 、 迦 叶 一 个 个 合 掌 称 赞 道 : 好 啊 , 好 啊 , 好 啊 当 年 蛋 化 为 人 , 立 志 修 行 果 真 。
万 劫 之 中 不 能 移 居 于 胜 境 , 一 旦 之 间 发 生 变 化 , 散 发 精 神 。
欺 天 罔 上 , 思 高 位 , 欺 圣 苟 丹 扰 乱 大 伦 。
恶 贯 满 盈 现 在 有 报 应 , 不 知 什 么 时 候 能 够 翻 身 。
如 来 佛 祖 消 灭 了 妖 猴 , 就 召 唤 阿 、 迦 叶 一 同 到 西 方 极 乐 世 界 。
当 时 有 个 叫 天 蓬 、 天 佑 急 忙 出 来 到 灵 霄 宝 殿 说 : 请 如 来 稍 稍 等 一 点 , 我 的 主 人 您 的 大 驾 来 了 。
佛 祖 听 了 这 话 , 回 头 瞻 仰 。
不 一 会 儿 , 果 然 看 见 八 景 鸾 舆 , 九 光 宝 盖 , 声 音 演 奏 玄 歌 妙 乐 , 歌 唱 无 量 神 章 , 散 发 宝 花 , 喷 出 真 香 , 直 到 佛 面 前 谢 罪 说 : 多 蒙 大 法 收 除 妖 邪 , 希 望 如 来 稍 停 一 日 , 请 诸 位 仙 人 做 一 个 会 席 奉 谢 。
如 来 不 敢 违 背 , 就 合 掌 谢 道 : 老 僧 承 蒙 大 天 尊 的 命 令 来 到 这 里 , 有 什 么 法 力 , 还 是 天 尊 与 众 神 的 洪 福 。
玉 帝 传 达 圣 旨 , 就 带 着 雷 部 众 神 , 分 头 请 求 三 清 、 四 御 、 五 老 、 六 司 、 七 元 、 八 极 、 九 曜 、 十 都 、 千 真 、 万 圣 前 来 参 加 会 合 , 共 同 感 谢 佛 恩 。
又 命 令 四 大 天 师 、 九 天 仙 女 , 大 开 玉 京 金 阙 、 太 玄 宝 宫 、 洞 阳 玉 馆 , 请 如 来 高 座 七 宝 灵 台 , 调 整 各 班 坐 位 , 安 排 龙 肝 凤 髓 , 玉 液 蟠 桃 。
不 一 会 儿 , 那 玉 清 元 始 天 尊 、 上 清 灵 宝 天 尊 、 太 清 道 德 天 尊 、 五 真 君 、 五 斗 星 君 、 三 官 四 圣 、 九 曜 真 君 、 左 辅 、 右 弼 、 天 王 、 何 吒 , 玄 虚 一 应 灵 通 , 对 面 旌 旗 , 双 双 幡 盖 , 都 捧 着 明 珠 异 宝 , 寿 果 奇 花 , 向 佛 前 拜 献 道 : 感 应 如 来 无 量 法 力 , 收 伏 妖 猴 。
承 蒙 大 天 尊 设 宴 , 呼 唤 我 们 都 来 谢 恩 。
请 如 来 将 此 会 立 一 个 名 字 叫 什 么 ? 如 来 领 着 众 神 的 嘱 托 说 : 现 在 想 立 一 个 名 字 , 可 以 作 个 安 天 大 会 。
各 位 仙 老 异 口 同 声 , 都 说 道 : 好 个 安 天 大 会 , 好 个 安 天 大 会 , 说 完 , 各 自 坐 在 座 位 上 , 走 着 传 递 的 车 辆 , 插 花 鼓 瑟 , 果 然 是 好 会 。
有 诗 作 证 。
《 诗 经 》 说 : 宴 设 蟠 桃 猴 搅 乱 , 安 天 大 会 胜 蟠 桃 。
龙 旗 鸾 祥 光 照 耀 , 宝 节 幢 幡 瑞 气 飘 飘 。
《 仙 乐 》 《 玄 歌 》 , 音 韵 美 , 凤 箫 玉 管 声 高 。
琼 香 缭 绕 , 群 仙 聚 集 , 宇 宙 清 平 , 祝 贺 圣 朝 。
众 人 都 高 兴 地 聚 会 , 只 见 王 母 娘 娘 带 领 一 班 仙 子 、 仙 娥 、 美 女 , 飘 飘 荡 荡 地 跳 到 佛 面 前 , 施 礼 说 : 以 前 被 妖 猴 搅 乱 , 蟠 桃 一 会 , 请 众 仙 、 众 佛 都 成 功 。
现 在 承 蒙 如 来 的 大 法 , 锁 住 顽 猴 , 喜 庆 安 天 大 会 , 没 有 什 么 东 西 可 以 感 谢 , 现 在 是 我 净 手 亲 自 摘 出 几 棵 大 株 蟠 桃 奉 献 。
真 个 是 : 半 红 半 绿 , 喷 甘 香 , 艳 丽 仙 根 万 年 长 。
可 笑 武 陵 源 上 种 , 争 如 天 府 更 奇 强 。
紫 色 花 纹 娇 嫩 , 寰 中 少 人 , 核 清 甜 世 无 双 。
甘 延 寿 、 甘 延 年 能 改 变 身 体 , 有 缘 缘 吃 饭 的 人 自 然 不 同 寻 常 。
佛 祖 合 掌 向 王 母 谢 罪 。
王 母 又 着 仙 姬 、 仙 子 唱 的 歌 , 舞 的 舞 。
聚 集 群 仙 , 又 都 赞 赏 赞 赏 。
正 是 天 香 满 座 , 缤 纷 仙 蕊 仙 花 。
玉 京 金 阙 大 荣 华 。
奇 珍 异 宝 无 价 。
对 答 与 上 天 同 寿 , 双 双 万 劫 增 添 。
桑 田 沧 海 任 更 差 。
其 他 人 自 然 没 有 惊 慌 , 也 没 有 惊 讶 。
王 母 正 穿 着 仙 姬 、 仙 子 唱 歌 跳 舞 , 酒 杯 交 错 , 不 多 时 , 忽 然 又 闻 到 : 一 阵 奇 异 的 香 味 来 鼻 子 , 惊 动 满 堂 的 星 宿 和 星 宿 。
天 仙 佛 祖 把 杯 停 下 , 各 自 抬 头 迎 目 等 候 。
在 霄 汉 中 间 出 现 一 个 老 人 , 手 捧 灵 芝 飞 舞 着 彩 衣 。
葫 芦 藏 蓄 万 年 丹 , 宝 名 书 千 纪 寿 。
洞 里 乾 坤 任 自 由 , 壶 中 日 月 随 着 成 就 。
遨 游 四 海 , 喜 欢 清 闲 , 散 淡 十 洲 , 容 纳 辐 。
曾 赴 蟠 桃 醉 几 遭 , 醒 时 明 月 还 依 旧 。
长 头 大 耳 , 短 身 体 , 南 极 的 地 方 称 为 老 寿 。
寿 星 又 到 。
见 到 玉 帝 行 礼 完 毕 , 又 见 到 如 来 , 申 谢 说 : 我 刚 听 说 那 妖 猴 被 老 君 引 到 兜 率 宫 炼 炼 , 以 为 必 定 能 使 天 下 平 安 , 没 想 到 他 又 反 而 出 来 。
希 望 如 来 善 于 敬 佩 这 种 怪 物 , 设 宴 奉 谢 , 所 以 我 闻 风 而 来 。
再 没 有 别 的 东 西 可 以 献 上 , 特 地 准 备 了 紫 芝 瑶 草 、 碧 藕 金 丹 奉 上 。
《 诗 经 》 说 : 碧 藕 金 丹 奉 释 迦 , 如 来 万 寿 如 恒 沙 。
清 平 永 乐 三 乘 锦 , 康 泰 长 生 九 品 花 。
无 相 门 中 的 真 法 主 , 色 空 天 上 是 仙 家 。
乾 坤 大 地 都 称 为 祖 , 一 丈 六 寸 金 身 福 寿 。
如 来 欣 然 领 谢 。
寿 星 得 到 了 座 位 , 仍 然 跑 开 马 车 传 送 。
只 见 赤 脚 大 仙 又 来 到 , 向 玉 帝 面 前 恭 敬 礼 毕 , 又 对 佛 祖 谢 罪 说 : 我 深 深 感 激 佛 祖 的 法 力 , 降 服 妖 猴 。
没 有 什 么 东 西 可 以 表 示 敬 意 , 特 准 备 了 交 梨 二 颗 、 火 枣 数 枚 奉 献 。
《 诗 经 》 说 : 大 仙 赤 脚 枣 梨 香 , 敬 献 弥 陀 寿 算 长 。
七 宝 莲 台 山 样 安 稳 , 千 金 花 座 锦 绣 一 般 。
寿 同 天 地 , 言 不 悖 , 福 比 洪 波 , 话 岂 狂 。
福 寿 如 期 , 真 的 是 , 清 闲 极 乐 那 西 方 。
如 来 又 称 谢 了 , 叫 阿 、 迦 叶 将 各 自 献 上 的 东 西 , 一 一 收 起 来 , 正 向 玉 帝 面 前 谢 酒 。
众 人 都 喝 醉 了 。
只 见 一 个 巡 视 灵 官 来 报 告 说 : 那 大 圣 伸 出 头 来 了 。
佛 祖 说 : 不 妨 , 不 妨 。
从 袖 子 里 取 出 一 张 帖 子 , 上 面 有 六 个 金 字 , 名 叫 喇 嘛 。
送 给 阿 , 叫 贴 在 那 山 顶 上 。
尊 者 立 即 领 着 帖 子 , 拿 出 天 门 , 到 了 那 五 行 山 顶 上 , 紧 紧 贴 在 一 块 四 方 石 头 上 。
可 以 使 用 呼 吸 之 气 , 可 以 使 手 儿 爬 出 来 , 可 以 摇 动 、 摇 动 、 捉 拿 。
阿 回 报 说 : 已 经 把 帖 子 贴 了 。
如 来 就 辞 谢 了 玉 帝 众 神 , 和 两 位 尊 者 出 了 天 门 之 外 。
又 发 出 一 个 慈 悲 心 , 念 念 念 动 念 真 言 咒 语 , 将 五 行 山 召 来 一 尊 土 地 神 祗 , 会 同 五 方 揭 谛 , 住 在 此 山 监 押 。
他 饥 饿 的 时 候 , 给 他 钢 丸 子 吃 ; 口 渴 的 时 候 , 给 他 溶 化 的 铜 汁 喝 。
等 到 其 他 灾 祸 过 失 满 天 , 自 然 会 有 人 救 他 。
正 是 妖 猴 大 胆 反 天 宫 , 却 被 如 来 伏 手 投 降 。
渴 饮 熔 化 钢 铁 , 忍 耐 岁 月 , 饥 餐 铁 弹 度 时 光 。
天 灾 苦 难 遭 受 挫 折 , 人 事 悲 凉 喜 好 命 运 长 久 。
如 果 得 到 英 雄 重 展 , 将 来 奉 佛 上 西 方 。
又 有 诗 道 : 伏 逞 豪 强 大 势 兴 , 降 龙 伏 虎 弄 乖 能 。
偷 桃 偷 酒 游 天 府 , 受 贿 承 恩 在 玉 京 。
恶 贯 满 盈 , 身 受 困 难 , 善 根 不 绝 , 气 又 升 。
果 然 脱 脱 了 如 来 的 手 , 暂 且 等 待 唐 朝 出 圣 僧 。
最 后 不 知 道 以 后 什 么 年 月 , 才 满 满 灾 殃 , 暂 且 听 下 回 的 分 析 解 释 。
}\switchcolumn\flushpage  \begin{pinyinscope}{\myfontt \section{第八回}     我佛造經傳極樂 觀音奉旨上長安

試問禪關,參求無數,往往到頭虛老。磨磚作鏡,積雪為糧,迷了幾多年少。毛
吞大海,芥納須彌,金色頭陀微笑。悟時超十地三乘,凝滯了四生六道。誰聽得
,絕想崖前,無陰樹下,杜宇一聲春曉。曹溪路險,鷲嶺雲深,此處故人音杳。
千丈冰崖,五葉蓮開,古殿簾垂香裊。那時節,識破源流,便見龍王三寶。

這一篇詞,名《蘇武慢》。話表我佛如來辭別了玉帝,回至雷音寶剎。但見那三
千諸佛、五百阿羅、八大金剛、無邊菩薩,一個個都執著幢幡寶蓋、異寶仙花,
擺列在靈山仙境娑羅雙林之下接迎。如來駕住祥雲,對眾道:「我以甚深般若,
遍觀三界。根本性原,畢竟寂滅。同虛空相,一無所有。殄伏乖猴,是事莫識。
名生死始,法相如是。」說罷,放舍利之光,滿空有白虹四十二道,南北通連。
大眾見了,皈身禮拜。少頃間,聚慶雲彩霧,登上品蓮臺,端然坐下。那三千諸
佛、五百羅漢、八金剛、四菩薩,合掌近前禮畢,問曰:「鬧天宮攪亂蟠桃者,
何也?」如來道:「那廝乃花果山產的一妖猴,罪惡滔天,不可名狀。概天神將
,俱莫能降伏﹔雖二郎捉獲,老君用火鍛煉,亦莫能傷損。我去時,正在雷將中
間揚威耀武,賣弄精神。被我止住兵戈,問他來歷。他言有神通,會變化,又駕
觔斗雲,一去十萬八千里。我與他打了個賭賽,他出不得我手,卻將他一把抓住
,指化五行山,封壓他在那裏。玉帝大開金闕瑤宮,請我坐了首席,立安天大會
謝我,卻方辭駕而回。」大眾聽言喜悅,極口稱揚。

謝罷,各分班而退,各執乃事,共樂天真。果然是:
瑞靄漫天竺,虹光擁世尊。西方稱第一,無相法王門。常見玄猿獻果,麋鹿啣花
﹔青鸞舞,彩鳳鳴﹔靈龜捧壽,仙鶴噙芝。安享淨土祗園,受用龍宮法界。日日
花開,時時果熟。習靜歸真,參禪果正。不滅不生,不增不減。煙霞縹緲隨來往
,寒暑無侵不記年。
  詩曰:
    去來自在任優游,也無恐怖也無愁。
    極樂場中俱坦蕩,大千之處沒春秋。

佛祖居於靈山大雷音寶剎之間。一日,喚聚諸佛、阿羅、揭諦、菩薩、金剛、比
丘僧尼等眾曰:「自伏乖猿安天之後,我處不知年月,料凡間有半千年矣。今值
孟秋望日,我有一寶盆,盆中具設百樣奇花、千般異果等物,與汝等享此盂蘭盆
會,如何?」概眾一個個合掌,禮佛三匝領會。如來卻將寶盆中花果品物,著阿
儺捧定,著迦葉佈散。大眾感激,各獻詩伸謝。
  福詩曰:
    福星光耀世尊前,福納彌深遠更綿。
    福德無疆同地久,福緣有慶與天連。
    福田廣種年年盛,福海洪深歲歲堅。
    福滿乾坤多福蔭,福增無量永周全。
  祿詩曰:
    祿重如山彩鳳鳴,祿隨時泰祝長庚。
    祿添萬斛身康健,祿享千鍾世太平。
    祿俸齊天還永固,祿名似海更澄清。
    祿恩遠繼多瞻仰,祿爵無邊萬國榮。
  壽詩曰:
    壽星獻彩對如來,壽域光華自此開。
    壽果滿盤生瑞靄,壽花新採插蓮臺。
    壽詩清雅多奇妙,壽曲調音按美才。
    壽命延長同日月,壽如山海更悠哉。

眾菩薩獻畢,因請如來明示根本,指解源流。那如來微開善口,敷演大法,宣揚
正果,講的是三乘妙典,五蘊楞嚴。但見那天龍圍繞,花雨繽紛。正是:
禪心朗照千江月,真性清涵萬里天。

如來講罷,對眾言曰:「我觀四大部洲,眾生善惡,各方不一:東勝神洲者,敬
天禮地,心爽氣平﹔北俱盧洲者,雖好殺生,只因糊口,性拙情疏,無多作踐﹔
我西牛賀洲者,不貪不殺,養氣潛靈,雖無上真,人人固壽﹔但那南贍部洲者,
貪淫樂禍,多殺多爭,正所謂口舌兇場,是非惡海。我今有三藏真經,可以勸人
為善。」諸菩薩聞言,合掌皈依,向佛前問曰:「如來有那三藏真經?」如來曰
:「我有法一藏,談天﹔論一藏,說地﹔經一藏,度鬼。三藏共計三十五部,該
一萬五千一百四十四卷,乃是修真之經,正善之門。我待要送上東土,叵耐那方
眾生愚蠢,毀謗真言,不識我法門之旨要,怠慢了瑜迦之正宗。怎麼得一個有法
力的,去東土尋一個善信,教他苦歷千山,詢經萬水,到我處求取真經,永傳東
土,勸化眾生,卻乃是個山大的福緣,海深的善慶。誰肯去走一遭來?」當有觀
音菩薩行近蓮臺,禮佛三匝道:「弟子不才,願上東土尋一個取經人來也。」諸
眾抬頭觀看,那菩薩:
理圓四德,智滿金身。纓絡垂珠翠,香環結寶明。烏雲巧疊盤龍髻,繡帶輕飄彩
鳳翎。碧玉紐,素羅袍,祥光籠罩﹔錦絨裙,金落索,瑞氣遮迎。眉如小月,眼
似雙星。玉面天生喜,朱脣一點紅。淨瓶甘露年年盛,斜插垂楊歲歲青。解八難
,度群生,大慈憫:故鎮太山,居南海,救苦尋聲,萬稱萬應,千聖千靈。蘭心
欣紫竹,蕙性愛香藤。他是落伽山上慈悲主,潮音洞裏活觀音。

如來見了,心中大喜道:「別個是也去不得。須是觀音尊者,神通廣大,方可去
得。」菩薩道:「弟子此去東土,有甚言語吩咐?」如來道:「這一去,要踏看
路道,不許在霄漢中行。須是要半雲半霧,目過山水,謹記程途遠近之數,叮嚀
那取經人。但恐善信難行,我與你五件寶貝。」即命阿儺、迦葉取出錦襴袈裟一
領。九環錫杖一根,對菩薩言曰:「這袈裟、錫杖,可與那取經人親用。若肯堅
心來此,穿我的袈裟,免墮輪迴﹔持我的錫杖,不遭毒害。」這菩薩皈依拜領。
如來又取出三個箍兒,遞與菩薩道:「此寶喚做緊箍兒,雖是一樣三個,但只是
用各不同。我有金緊禁的咒語三篇。假若路上撞見神通廣大的妖魔,你須是勸他
學好,跟那取經人做個徒弟。他若不伏使喚,可將此箍兒與他戴在頭上,自然見
肉生根。各依所用的咒語念一念,眼脹頭痛,腦門皆裂,管教他入我門來。」

那菩薩聞言,踴躍作禮而退。即喚惠岸行者隨行。那惠岸使一條渾鐵棍,重有千
斤,只在菩薩左右作一個降魔的大力士。菩薩遂將錦襴袈裟,作一個包裹,令他
背了。菩薩將金箍藏了,執了錫杖,徑下靈山。這一去,有分教:
佛子還來歸本願,金蟬長老裹栴檀。

那菩薩到山腳下,有玉真觀金頂大仙在觀門首接住,請菩薩獻茶。菩薩不敢久停
,曰:「今領如來法旨,上東土尋取經人去。」大仙道:「取經人幾時方到?」
菩薩道:「未定,約摸二三年間,或可至此。」遂辭了大仙,半雲半霧,約記程
途。有詩為證。詩曰:
    萬里相尋自不言,卻云誰得意難全。
    求人忽若渾如此,是我平生豈偶然。
    傳道有方成妄說,說明無信也虛傳。
    願傾肝膽尋相識,料想前頭必有緣。

師徒二人正走間,忽然見弱水三千,乃是流沙河界。菩薩道:「徒弟呀,此處卻
是難行。取經人濁骨凡胎,如何得渡?」惠岸道:「師父,你看河有多遠?」那
菩薩停立雲步看時,只見:
東連沙磧,西抵諸番,南達烏戈,北通韃靼。徑過有八百里遙,上下有千萬里遠
。水流一似地翻身,浪滾卻如山聳背。洋洋浩浩,漠漠茫茫,十里遙聞萬丈洪。
仙槎難到此,蓮葉莫能浮。衰草斜陽流曲浦,黃雲影日暗長堤。那裏得客商來往
?何曾有漁叟依棲?平沙無雁落,遠岸有猿啼。只是紅蓼花蘩知景色,白蘋香細
任依依。

菩薩正然點看,只見那河中潑剌一聲響喨,水波裏跳出一個妖魔來,十分醜惡。
他生得:
青不青,黑不黑,晦氣色臉﹔長不長,短不短,赤腳筋軀。眼光閃爍,好似灶底
雙燈﹔口角丫叉,就如屠家火缽。獠牙撐劍刃,紅髮亂蓬鬆。一聲叱?如雷吼,
兩腳奔波似滾風。

那怪物手執一根寶杖,走上岸就捉菩薩,卻被惠岸掣渾鐵棒擋住,喝聲:「休走
!」那怪物就持寶杖來迎。兩個在流沙河邊這一場惡殺,真個驚人:
木叉渾鐵棒,護法顯神通﹔怪物降妖杖,努力逞英雄。雙條銀蟒河邊舞,一對神
僧岸上沖。那一個威鎮流沙施本事,這一個力保觀音建大功。那一個翻波躍浪,
這一個吐霧噴風。翻波躍浪乾坤暗,吐霧噴風日月昏。那個降妖杖,好便似出山
的白虎﹔這個渾鐵棒,卻就如臥道的黃龍。那個使將來,尋蛇撥草﹔這個丟開去
,撲鷂分松。只殺得昏漠漠,星辰燦爛﹔霧騰騰,天地朦朧。那個久住弱水惟他
狠,這個初出靈山第一功。

他兩個來來往往,戰上數十合,不分勝負。那怪物架住了鐵棒道:「你是那裏和
尚,敢來與我抵敵?」木叉道:「我是托塔天王二太子木叉惠岸行者,今保我師
父往東土尋取經人去。你是何怪,敢大膽阻路?」那怪方才醒悟道:「我記得你
跟南海觀音在紫竹林中修行,你為何來此?」木叉道:「那岸上不是我師父?」

怪物聞言,連聲喏喏,收了寶杖。讓木叉揪了去見觀音,納頭下拜,告道:「菩
薩,恕我之罪,待我訴告:我不是妖邪,我是靈霄殿下侍鑾輿的捲簾大將。只因
在蟠桃會上失手打碎了玻璃盞,玉帝把我打了八百,貶下界來,變得這般模樣。
又叫七日一次,將飛劍來穿我胸脅百餘下方回。故此這般苦惱。沒奈何,饑寒難
忍,三二日間,出波濤尋一個行人食用。不期今日無知,沖撞了大慈菩薩。」菩
薩道:「你在天有罪,既貶下來,今又這等傷生,正所謂罪上加罪。我今領了佛
旨,上東土尋取經人。你何不入我門來,皈依善果,跟那取經人做個徒弟,上西
天拜佛求經?我叫飛劍不來穿你。那時節功成免罪,復你本職,心下如何?」那
怪道:「我願皈正果。」又向前道:「菩薩,我在此間吃人無數,向來有幾次取
經人來,都被我吃了。凡吃的人頭,拋落流沙,竟沉水底。這個水,鵝毛也不能
浮。惟有九個取經人的骷髏浮在水面,再不能沉。我以為異物,將索兒穿在一處
,閑時拿來頑耍。這去,但恐取經人不得到此,卻不是反誤了我的前程也?」菩
薩曰:「豈有不到之理?你可將骷髏兒掛在頭項下,等候取經人,自有用處。」
怪物道:「既然如此,願領教誨。」菩薩方與他摩頂受戒,指沙為姓,就姓了沙
﹔起個法名,叫做個沙悟淨。當時入了沙門,送菩薩過了河,他洗心滌慮,再不
傷生,專等取經人。

菩薩與他別了,同木叉徑奔東土。行了多時,又見一座高山,山上有惡氣遮漫,
不能步上。正欲駕雲過山,不覺狂風起處,又閃上一個妖魔。他生得又甚兇險,
但見他:
    捲臟蓮蓬吊搭嘴,耳如蒲扇顯金睛。
    獠牙鋒利如鋼剉,長嘴張開似火盆。
    金盔緊繫腮邊帶,勒甲絲絛蟒退鱗。
    手執釘鈀龍探爪,腰挎彎弓月半輪。
    糾糾威風欺太歲,昂昂志氣壓天神。

他撞上來,不分好歹,望菩薩舉釘鈀就築。被木叉行者擋住,大喝一聲道:「那
潑怪,休得無禮,看棒。」妖魔道:「這和尚不知死活。看鈀。」兩個在山底下
一沖一撞,賭鬥輸贏,真個好殺:
妖魔兇猛,惠岸威能。鐵棒分心搗,釘鈀劈面迎。播土揚塵天地暗,飛砂走石鬼
神驚。九齒鈀,光耀耀,雙環響喨﹔一條棒,黑悠悠,兩手飛騰。這個是天王太
子,那個是元帥精靈。一個在普陀為護法,一個在山洞作妖精。這場相遇爭高下
,不知那個虧輸那個贏。

他兩個正殺到好處,觀世音在半空中拋下蓮花,隔開鈀、杖。怪物見了心驚,便
問:「你是那裏和尚,敢弄甚麼眼前花兒哄我?」木叉道:「我把你個肉眼凡胎
的潑物!我是南海菩薩的徒弟。這是我師父拋來的蓮花,你也不認得哩!」那怪
道:「南海菩薩,可是掃三災救八難的觀世音麼?」木叉道:「不是他是誰?」
怪物撇了釘鈀,納頭下禮道:「老兄,菩薩在那裏?累煩你引見一引見。」木叉
仰面指道:「那不是?」怪物朝上磕頭,厲聲高叫道:「菩薩,恕罪,恕罪。」

觀音按下雲頭,前來問道:「你是那裏成精的野豕,何方作怪的老彘,敢在此間
擋我?」那怪道:「我不是野豕,亦不是老彘,我本是天河裏天蓬元帥。只因帶
酒戲弄嫦娥,玉帝把我打了二千鎚,貶下塵凡。一靈真性,徑來奪舍投胎,不期
錯了道路,投在個母豬胎裏,變得這般模樣。是我咬殺母豬,打死群彘,在此處
占了山場,吃人度日。不期撞著菩薩,萬望拔救拔救。」菩薩道:「此山叫做甚
麼山?」怪物道:「叫做福陵山。山中有一洞,叫做雲棧洞。洞裏原有個卵二姐
,他見我有些武藝,招我做了家長,又喚做倒蹅門。不上一年,他死了,將一洞
的家當,盡歸我受用。在此日久年深,沒有贍身的勾當,只是依本等吃人度日。
萬望菩薩恕罪。」菩薩道:「古人云,『若要有前程,莫做沒前程。』你既上界
違法,今又不改兇心,傷生造孽,卻不是二罪俱罰?」那怪道:「前程,前程,
若依你,教我喝風?常言道:『依著官法打殺,依著佛法餓殺。』去也,去也,
還不如捉個行人,肥膩膩的吃他家娘,管甚麼二罪三罪,千罪萬罪!」菩薩道:
「『人有善願,天必從之。』汝若肯歸依正果,自有養身之處。世有五穀,可以
濟饑,為何吃人度日?」

怪物聞言,似夢方覺,向菩薩道:「我欲從正,奈何『獲罪於天,無所禱也』。」
菩薩道:「我領了佛旨,上東土尋取經人。你可跟他做個徒弟,往西天走一遭來
,將功折罪,管教你脫離災瘴。」那怪滿口道:「願隨,願隨。」菩薩才與他摩
頂受戒,指身為姓,就姓了豬﹔替他起了法名,就叫做豬悟能。遂此領命歸真,
持齋把素,斷絕了五葷三厭,專候那取經人。

菩薩卻與木叉辭了悟能,半興雲霧前來。正走處,只見空中有一條玉龍叫喚。菩
薩近前問曰:「你是何龍,在此受罪?」那龍道:「我是西海龍王敖閏之子,因
縱火燒了殿上明珠,我父王表奏天庭,告了忤逆。玉帝把我吊在空中,打了三百
,不日遭誅。望菩薩搭救搭救。」

觀音聞言,即與木叉撞上南天門裏,早有丘、張二天師接著,問道:「何往?」
菩薩道:「貧僧要見玉帝一面。」二天師即忙上奏。玉帝遂下殿迎接。菩薩上前
禮畢道:「貧僧領佛旨上東土尋取經人,路遇孽龍懸吊,特來啟奏,饒他性命,
賜與貧僧,教他與取經人做個腳力。」玉帝聞言,即傳旨赦宥,差天將解放,送
與菩薩。菩薩謝恩而出。這小龍叩頭謝活命之恩,聽從菩薩使喚。菩薩把他送在
深澗之中,只等取經人來,變做白馬,上西方立功。小龍領命潛身不題。

菩薩帶引木叉行者過了此山,又奔東土。行不多時,忽見金光萬道,瑞氣千條。
木叉道:「師父,那放光之處,乃是五行山了,見有如來的壓帖在那裏。」菩薩
道:「此卻是那攪亂蟠桃會、大鬧天宮的齊天大聖,今乃壓在此也。」木叉道:
「正是,正是。」師徒俱上山來,觀看帖子,乃是「唵嘛呢叭吽」六字真言。菩
薩看罷,嘆惜不已,作詩一首。詩曰:
    堪嘆妖猴不奉公,當年狂妄逞英雄。
    欺心攪亂蟠桃會,大膽私行兜率宮。
    十萬軍中無敵手,九重天上有威風。
    自遭我佛如來困,何日舒伸再顯功?

師徒們正說話處,早驚動了那大聖。大聖在山根下高叫道:「是那個在山上吟詩
,揭我的短哩?」菩薩聞言,徑下山來尋看。只見那石崖之下,有土地、山神、
監押大聖的天將,都來拜接了菩薩,引至那大聖面前。看時,他原來壓於石匣之
中,口能言,身不能動。菩薩道:「姓孫的,你認得我麼?」大聖睜開火眼金睛
,點著頭兒高叫道:「我怎麼不認得你,你好的是那南海普陀落伽山救苦救難大
慈大悲南無觀世音菩薩。承看顧,承看顧。我在此度日如年,更無一個相知的來
看我一看。你從那裏來也?」菩薩道:「我奉佛旨,上東土尋取經人去,從此經
過,特留殘步看你。」大聖道:「如來哄了我,把我壓在此山,五百餘年了,不
能展掙。萬望菩薩方便一二,救我老孫一救。」菩薩道:「你這廝罪業彌深,救
你出來,恐你又生禍害,反為不美。」大聖道:「我已知悔了,但願大慈悲指條
門路,情願修行。」這才是:
    人心生一念,天地盡皆知。
    善惡若無報,乾坤必有私。

那菩薩聞得此言,滿心歡喜,對大聖道:「聖經云:『出其言善,則千里之外應
之﹔出其言不善,則千里之外違之。』你既有此心,待我到了東土大唐國尋一個
取經的人來,教他救你。你可跟他做個徒弟,秉教迦持,入我佛門,再修正果,
如何?」大聖聲聲道:「願去,願去。」菩薩道:「既有善果,我與你起個法名
。」大聖道:「我已有名了,叫做孫悟空。」菩薩又喜道:「我前面也有二人歸
降,正是『悟』字排行,你今也是『悟』字,卻與他相合,甚好,甚好。這等也
不消叮囑,我去也。」那大聖見性明心歸佛教,這菩薩留情在意訪神僧。

他與木叉離了此處,一直東來,不一日就到了長安大唐國。斂霧收雲,師徒們變
作兩個疥癩遊僧,入長安城裏,早不覺天晚。行至大市街傍,見一座土地廟祠,
二人徑入。諕得那土地心慌,鬼兵膽戰,知是菩薩,叩頭接入。那土地又急跑報
與城隍、社令,及滿長安各廟神祗,都知是菩薩,參見告道:「菩薩,恕眾神接
遲之罪。」菩薩道:「汝等切不可走漏一毫消息。我奉佛旨,特來此處尋訪取經
人。借你廟宇,權住幾日,待訪著真僧即回。」眾神各歸本處,把個土地趕在城
隍廟裏暫住,他師徒們隱遁真形。

    畢竟不知尋出那個取經人來,且聽下回分解。





}  \end{pinyinscope}\switchcolumn{\myfontc \section{第 八 回} 我 佛 造 经 传 《 极 乐 观 音 》 , 奉 旨 上 长 安 , 试 问 禅 关 , 参 求 无 数 , 往 往 到 头 虚 老 。
磨 砖 做 镜 子 , 积 雪 成 为 粮 食 , 迷 了 多 少 年 少 。
毛 吞 食 大 海 , 芥 纳 须 弥 , 金 色 头 陀 微 笑 。
悟 时 超 出 十 地 三 乘 , 凝 滞 了 四 生 六 道 。
谁 能 听 到 , 绝 想 崖 前 , 无 阴 树 下 , 杜 宇 一 声 春 晓 。
曹 溪 道 路 险 要 , 猿 岭 云 雾 深 邃 , 此 处 所 以 人 声 杳 杳 。
千 丈 冰 崖 , 五 叶 莲 花 开 , 古 殿 帘 垂 香 雾 缭 绕 。
那 时 候 , 能 够 识 破 源 流 , 就 能 见 到 龙 王 三 宝 。
这 一 篇 词 , 名 叫 《 苏 武 慢 》 。
话 表 说 : 我 佛 如 来 辞 别 了 玉 帝 , 回 到 雷 音 宝 刹 。
又 见 三 千 诸 佛 、 五 百 阿 罗 、 八 大 金 刚 、 无 边 菩 萨 , 一 个 个 都 拿 着 幢 幡 、 宝 盖 、 异 宝 、 仙 花 , 排 列 在 灵 山 仙 境 、 娑 罗 双 林 之 下 。
如 来 驾 车 停 住 祥 云 , 对 众 人 说 : 我 以 甚 深 般 若 的 身 份 , 遍 观 三 界 。
根 本 本 性 本 性 , 最 终 寂 灭 。
如 同 虚 空 的 相 貌 , 一 无 所 有 。
伏 乖 猴 , 此 事 没 有 人 知 道 。
生 死 开 始 , 法 相 就 是 这 样 。
说 完 , 放 出 舍 利 的 光 亮 , 满 空 有 四 十 二 道 白 虹 , 南 北 相 连 。
大 众 见 到 了 , 就 恭 敬 地 行 礼 拜 礼 。
不 一 会 儿 , 聚 集 庆 云 彩 雾 , 登 上 品 莲 台 , 端 正 地 坐 下 。
那 三 千 诸 佛 、 五 百 罗 汉 、 八 金 刚 、 四 菩 萨 , 合 掌 在 近 前 行 礼 , 问 道 : 闹 天 宫 搅 乱 蟠 桃 , 是 什 么 原 因 ? 如 来 说 : 那 这 个 是 花 果 山 出 产 的 一 个 妖 猴 , 罪 恶 滔 天 , 不 可 以 说 明 。
大 概 是 天 神 的 将 领 , 都 不 能 降 服 , 即 使 是 二 郎 捉 住 了 , 老 君 用 火 炼 炼 , 也 不 能 伤 害 。
我 去 的 时 候 , 正 在 雷 将 之 间 , 扬 威 耀 武 , 炫 耀 精 神 。
被 我 止 住 兵 戈 , 问 他 的 来 历 。
他 说 有 神 通 , 会 变 化 , 又 驾 着 斗 云 , 一 去 十 万 八 千 里 。
吾 与 其 打 一 个 赌 酒 , 他 出 不 到 我 的 手 , 却 被 他 一 把 握 住 , 指 化 五 行 山 , 封 压 他 在 那 里 。
玉 帝 大 开 金 阙 瑶 宫 , 请 我 坐 在 首 席 上 , 立 在 安 天 大 会 上 向 我 谢 罪 。
大 众 听 了 他 的 话 非 常 高 兴 , 极 口 称 扬 。
谢 恩 辞 谢 完 毕 , 各 自 分 班 退 下 , 各 自 坚 持 自 己 的 事 情 , 共 同 欢 乐 天 真 。
果 然 是 这 样 , 祥 瑞 的 云 气 漫 天 竺 , 虹 光 拥 世 尊 。
西 方 称 为 第 一 , 无 相 法 王 门 。
常 常 看 到 玄 猿 献 果 , 麋 鹿 献 花 , 青 鸾 舞 , 彩 凤 鸣 叫 ; 灵 龟 捧 寿 , 仙 鹤 衔 芝 。
安 享 净 土 祗 园 , 受 用 龙 宫 法 界 。
日 日 花 开 , 时 时 果 熟 。
学 习 静 静 归 真 , 参 禅 果 然 端 正 。
不 灭 不 生 , 不 增 不 减 。
烟 霞 缭 绕 随 着 来 往 , 寒 暑 无 所 侵 袭 , 不 记 得 年 龄 。
《 诗 经 》 上 说 : 去 来 自 在 , 任 凭 着 优 游 , 也 没 有 恐 惧 , 也 没 有 愁 愁 。
极 乐 场 中 都 是 坦 荡 荡 荡 的 , 大 千 的 地 方 没 有 春 秋 。
佛 祖 住 在 灵 山 大 雷 音 宝 刹 之 间 。
一 天 , 他 召 集 诸 佛 、 阿 罗 、 揭 谛 、 菩 萨 、 金 刚 、 比 丘 、 僧 尼 等 众 人 说 : 自 从 伏 乖 猿 、 安 天 以 后 , 我 住 在 世 间 不 知 年 月 , 估 计 凡 间 已 有 半 千 年 了 。
现 在 正 值 孟 秋 望 日 , 我 有 一 个 宝 盆 , 盆 里 具 备 了 百 种 奇 花 、 千 种 异 果 等 物 品 , 与 你 们 享 受 这 个 盂 兰 盆 会 , 怎 么 样 ?
如 来 却 把 宝 盆 中 的 花 果 等 物 , 放 在 阿 捧 定 , 放 在 迦 叶 布 散 。
大 家 都 很 感 激 , 各 自 献 诗 表 示 谢 恩 。
《 福 诗 》 说 : 福 星 光 耀 世 尊 前 , 福 纳 更 深 远 更 绵 延 。
福 德 无 疆 , 同 地 长 久 , 福 缘 有 福 , 与 上 天 相 连 。
福 田 广 种 , 年 年 兴 盛 , 福 海 洪 深 , 年 年 坚 固 。
福 满 乾 坤 多 福 荫 , 福 增 无 量 , 永 保 周 全 。
《 禄 诗 》 说 : 禄 重 如 山 彩 凤 鸣 , 福 随 时 泰 , 祝 祝 长 庚 。
俸 禄 增 加 万 斛 , 身 体 康 健 , 俸 禄 享 有 千 钟 , 世 代 太 平 。
俸 禄 齐 天 回 永 固 , 禄 名 似 海 更 澄 清 。
禄 位 恩 德 远 传 , 多 瞻 仰 , 爵 禄 无 边 万 国 荣 。
寿 诗 说 : 寿 星 献 彩 对 如 来 , 寿 域 光 华 自 此 开 。
寿 果 满 盘 生 瑞 霭 , 寿 花 新 采 插 在 莲 台 。
李 寿 的 诗 清 雅 , 多 有 奇 妙 的 作 品 , 李 寿 的 曲 调 音 调 和 谐 , 都 是 美 好 的 才 华 。
寿 命 延 长 如 同 日 月 , 寿 命 如 同 山 海 更 悠 悠 。
众 菩 萨 敬 献 完 毕 , 于 是 请 求 如 来 明 示 根 本 , 指 明 佛 法 的 源 流 。
但 是 如 来 稍 微 开 启 善 口 , 阐 述 大 法 , 宣 扬 正 果 , 讲 的 是 三 乘 妙 典 、 五 蕴 楞 严 。
只 见 那 天 龙 围 绕 , 花 雨 缤 纷 。
正 是 禅 心 朗 照 千 江 月 , 真 性 清 涵 万 里 天 。
如 来 讲 完 , 对 众 人 说 : 我 观 察 四 大 部 洲 , 众 生 的 善 恶 , 各 方 面 都 不 一 样 。 东 胜 神 洲 的 人 , 敬 天 礼 地 , 心 神 爽 朗 , 喜 欢 杀 生 , 只 是 因 为 糊 口 , 性 格 拙 劣 , 性 情 疏 远 , 没 有 多 多 作 恶 ; 我 西 牛 贺 洲 的 人 , 不 贪 婪 不 杀 , 养 精 养 气 , 虽 然 没 有 上 真 , 人 人 都 能 保 持 长 寿 ; 但 那 南 赡 部 洲 的 人 , 贪 淫 喜 怒 祸 福 , 多 杀 人 多 争 , 正 是 所 谓 口 舌 凶 场 , 是 非 恶 。
我 现 在 有 《 三 藏 真 经 》 , 可 以 劝 人 修 行 善 道 。
诸 位 菩 萨 听 了 这 话 , 合 掌 皈 依 , 向 佛 面 向 佛 面 前 问 道 : 如 来 有 什 么 三 藏 真 经 ? 如 来 说 : 我 有 一 藏 菩 萨 , 谈 论 天 , 论 一 藏 , 说 地 , 经 一 藏 , 度 鬼 。
三 藏 共 有 三 十 五 部 , 总 共 一 万 五 千 一 百 四 十 四 卷 , 是 修 真 的 经 , 正 善 的 门 。
我 要 送 上 东 土 , 不 耐 那 方 众 生 愚 蠢 , 毁 谤 真 言 , 不 知 道 我 法 门 的 旨 要 , 懒 慢 了 解 了 瑜 迦 的 正 宗 。
何 以 得 到 一 个 有 法 力 的 人 , 去 东 方 寻 找 一 个 善 信 , 教 他 苦 苦 经 历 千 山 万 水 , 到 我 的 地 方 去 求 取 真 经 , 永 远 传 播 东 土 , 劝 导 众 生 , 这 就 是 山 大 海 深 的 善 庆 。
有 个 观 音 菩 萨 走 近 莲 台 , 拜 佛 三 匝 说 : 弟 子 没 有 才 能 , 愿 上 东 方 寻 找 一 个 读 经 的 人 来 。
众 众 抬 头 观 看 , 那 菩 萨 说 : 道 理 圆 满 四 德 , 智 慧 充 满 金 身 。
缨 络 垂 珠 翡 翠 , 香 环 结 宝 明 。
乌 云 巧 叠 盘 龙 髻 , 绣 带 轻 飘 彩 凤 翎 。
碧 玉 带 纽 , 素 罗 袍 , 祥 光 笼 罩 , 锦 绣 裙 , 金 落 索 , 祥 瑞 之 气 遮 住 迎 接 。
眉 如 小 月 , 眼 如 双 星 。
玉 脸 天 生 喜 喜 , 红 唇 一 点 红 。
净 瓶 甘 露 年 年 盛 , 斜 插 垂 杨 岁 年 青 。
解 除 八 难 , 解 除 众 生 , 大 慈 悲 哀 : 所 以 我 镇 守 泰 山 , 住 在 南 海 , 救 救 苦 难 , 寻 求 声 誉 , 万 人 万 人 的 响 应 , 千 圣 千 灵 。
兰 心 欣 赏 紫 竹 , 蕙 性 喜 爱 香 藤 。
他 是 落 伽 山 上 的 慈 悲 主 , 潮 音 洞 里 的 活 观 音 。
如 来 见 了 , 心 里 非 常 高 兴 地 说 : 别 的 是 , 去 不 得 。
必 须 是 观 音 尊 者 , 神 通 广 大 , 才 可 以 去 掉 。
菩 萨 说 : 弟 子 此 次 去 东 土 , 有 什 么 言 语 , 如 来 说 : 这 一 去 , 要 踏 看 路 路 , 不 许 在 霄 汉 中 走 。
必 须 有 半 云 半 雾 , 眼 睛 看 过 山 水 , 谨 慎 地 记 住 路 程 远 近 的 数 目 , 叮 嘱 哪 里 取 得 经 过 的 人 。
但 恐 怕 善 信 难 以 实 行 , 我 给 你 五 件 宝 贝 。
于 是 命 令 阿 、 迦 叶 取 出 锦 袈 裟 一 领 。
九 环 锡 杖 一 根 , 他 对 菩 萨 说 : 这 袈 裟 锡 杖 , 可 以 给 那 些 拿 经 书 的 人 亲 自 使 用 。
如 果 愿 意 坚 持 心 意 来 到 这 里 , 穿 上 我 的 袈 裟 , 免 于 堕 入 苦 难 之 中 , 持 着 我 的 锡 杖 , 不 会 遭 受 毒 害 。
这 位 菩 萨 , 虔 诚 敬 敬 地 叩 拜 。
如 来 又 取 出 三 个 钳 子 , 递 给 菩 萨 说 : 这 个 宝 物 叫 做 紧 钳 儿 , 虽 然 是 一 样 的 三 个 , 但 只 是 使 用 各 不 相 同 。
吾 有 《 金 紧 禁 》 三 篇 。
即 使 在 路 上 碰 见 神 通 广 大 的 妖 魔 , 你 必 须 是 劝 他 学 好 , 又 让 他 取 经 的 人 做 个 弟 子 。
他 如 果 不 服 从 使 唤 , 可 以 把 这 个 钳 子 和 他 戴 在 头 上 , 自 然 会 见 肉 生 根 。
各 依 其 所 用 的 咒 语 , 念 一 念 , 眼 睛 发 肿 , 头 痛 , 脑 门 都 裂 了 , 管 教 他 进 入 我 家 门 来 。
那 菩 萨 听 了 这 话 , 踊 跃 起 来 , 行 礼 退 下 。
立 即 叫 惠 岸 走 的 人 随 行 。
惠 岸 使 用 一 根 浑 铁 棍 , 重 有 千 斤 , 只 在 菩 萨 身 边 做 一 个 降 魔 的 大 力 士 。
菩 萨 就 把 锦 袈 裟 做 了 一 个 包 裹 , 让 他 背 了 。
菩 萨 把 金 钳 藏 起 来 , 拿 着 锡 杖 , 径 直 下 到 灵 山 。
这 一 去 , 有 分 别 教 导 : 佛 子 还 来 归 本 愿 , 金 蝉 长 老 裹 檀 。
那 菩 萨 到 山 脚 下 , 有 玉 真 观 金 顶 大 仙 在 观 门 前 接 住 , 请 菩 萨 进 献 茶 。
菩 萨 不 敢 久 停 , 说 : 现 在 领 着 如 来 法 旨 , 上 东 土 寻 找 经 书 的 人 去 。
大 仙 说 : 取 经 的 人 多 少 时 间 才 到 这 里 吧 。 菩 萨 说 : 还 没 有 确 定 , 大 约 摸 二 三 年 的 时 间 , 或 许 可 以 到 这 里 。
于 是 就 辞 别 了 大 仙 , 半 路 云 霞 半 路 雾 气 , 约 好 记 住 了 路 程 。
有 诗 作 证 。
《 诗 经 》 说 : 万 里 相 寻 自 不 言 , 却 说 谁 得 意 难 全 。
寻 求 别 人 忽 然 如 此 浑 浑 , 这 是 我 平 生 难 道 是 偶 然 的 事 吗 ?
传 授 道 术 有 方 , 成 了 妄 说 , 说 明 了 没 有 信 用 , 是 虚 传 的 。
愿 倾 尽 肝 胆 寻 求 相 识 , 估 计 想 想 前 头 一 定 有 缘 。
师 徒 二 人 正 在 奔 走 之 间 , 忽 然 看 见 三 千 条 弱 水 , 原 来 是 流 沙 河 的 边 界 。
菩 萨 说 : 徒 弟 啊 , 此 处 难 以 行 走 。
惠 岸 说 : 师 父 , 你 看 河 有 多 远 , 那 菩 萨 停 立 在 云 间 步 行 观 看 时 , 只 见 : 东 连 沙 , 西 抵 诸 番 , 南 到 乌 戈 , 北 通 译 。
路 过 八 百 里 远 , 上 下 有 千 万 里 远 。
水 流 一 股 好 像 地 翻 身 一 样 , 波 浪 翻 腾 如 山 耸 背 。
浩 浩 荡 荡 , 浩 浩 茫 茫 , 十 里 远 远 听 到 万 丈 洪 。
仙 梯 难 到 此 地 , 莲 叶 不 能 浮 。
衰 草 斜 阳 在 曲 浦 流 淌 , 黄 云 影 日 暗 长 堤 。
哪 里 能 有 客 商 来 往 , 哪 里 曾 经 有 渔 叟 依 傍 栖 息 呢 ? 平 沙 上 没 有 雁 落 , 远 岸 上 有 猿 啼 。
只 是 红 蓼 花 知 道 景 色 , 白 苹 香 细 , 任 凭 依 依 。
菩 萨 正 好 点 头 看 , 只 见 那 河 里 泼 剌 一 声 , 水 波 里 跳 出 一 个 妖 魔 来 , 十 分 丑 恶 。
其 他 生 的 人 是 : 青 不 青 , 黑 不 黑 , 黑 不 黑 , 黑 色 的 颜 色 , 长 不 长 , 短 不 短 , 赤 脚 的 筋 骨 。
眼 光 闪 闪 , 好 似 灶 底 的 双 灯 , 口 角 叉 叉 , 就 像 屠 家 的 火 钵 。
獠 牙 撑 着 剑 刃 , 红 头 发 乱 蓬 松 。
一 声 叱 骂 , 就 像 雷 吼 一 样 , 两 脚 奔 波 , 好 像 滚 滚 的 风 。
怪 物 手 持 一 根 宝 杖 , 走 上 岸 去 捉 菩 萨 , 却 被 惠 岸 拿 着 浑 铁 棒 子 , 喝 道 : 不 要 走 , 那 怪 物 拿 着 宝 杖 来 迎 接 。
两 个 人 在 流 沙 河 边 , 这 一 场 恶 杀 , 真 是 惊 人 。 木 叉 浑 铁 棒 , 护 法 显 神 通 , 怪 物 降 妖 杖 , 努 力 逞 英 雄 。
双 条 银 蟒 河 边 跳 舞 , 一 对 神 僧 在 岸 上 冲 。
何 一 个 威 镇 流 沙 施 本 事 , 这 一 个 力 保 观 音 建 立 大 功 。
一 个 翻 波 跃 浪 , 一 个 吐 雾 喷 风 。
翻 波 跃 浪 , 天 地 昏 暗 , 吐 雾 喷 风 , 日 月 昏 暗 。
那 个 降 妖 杖 , 好 像 出 山 的 白 虎 ; 那 个 浑 铁 棒 , 就 像 卧 在 道 上 的 黄 龙 。
那 个 使 者 将 来 , 寻 蛇 拔 草 ; 这 个 使 者 放 开 去 , 扑 分 松 。
只 杀 得 昏 暗 漠 漠 , 星 辰 灿 烂 , 雾 气 腾 腾 , 天 地 糊 涂 。
那 个 长 久 住 在 弱 水 中 , 唯 其 凶 狠 , 这 个 刚 出 灵 山 , 是 第 一 功 。
其 他 两 个 人 来 来 往 往 , 交 战 上 数 十 回 合 , 不 分 胜 负 。
那 怪 物 架 住 铁 棒 说 : 你 是 那 里 的 和 尚 , 竟 敢 来 与 我 抵 抗 , 木 叉 说 : 我 是 托 塔 天 王 二 太 子 木 叉 惠 岸 行 的 人 , 现 在 保 护 我 师 父 去 东 土 寻 找 经 书 的 人 去 。
那 怪 这 才 醒 悟 , 说 : 我 记 得 你 跟 南 海 观 音 在 紫 竹 林 中 修 行 , 你 为 什 么 来 到 这 里 ? 木 叉 说 : 那 岸 上 不 是 我 的 师 父 ? 怪 物 听 了 , 连 声 呐 喊 , 收 了 宝 杖 。
让 木 叉 扯 去 见 观 音 , 纳 头 下 拜 , 告 诉 说 : 菩 萨 , 饶 恕 我 的 罪 过 , 等 我 告 诉 说 : 我 不 是 妖 邪 , 我 是 灵 霄 殿 下 侍 奉 皇 上 的 卷 帘 大 将 。
因 为 在 蟠 桃 会 上 失 手 打 碎 了 琉 璃 盏 , 玉 帝 把 我 打 了 八 百 , 贬 下 界 来 , 变 成 了 这 样 的 模 样 。
又 叫 七 天 一 次 , 拿 着 飞 剑 来 穿 我 的 胸 胁 , 一 百 多 下 才 回 来 。
所 以 这 样 的 苦 恼 。
没 有 办 法 , 饥 寒 难 忍 , 三 二 天 之 内 , 从 波 涛 中 寻 找 一 个 行 人 吃 。
不 料 今 日 无 知 , 冲 撞 了 大 慈 菩 萨 。
菩 萨 说 : 你 在 天 上 有 罪 , 既 被 贬 下 来 , 现 在 又 这 样 伤 生 , 正 是 所 谓 的 罪 上 加 罪 。
我 现 在 领 了 佛 教 的 旨 意 , 到 东 土 寻 找 经 书 的 人 。
你 为 什 么 不 进 入 我 门 来 , 去 投 靠 善 果 , 与 那 取 经 人 做 个 徒 弟 , 上 西 天 去 拜 佛 求 经 , 我 叫 飞 剑 不 来 穿 你 。
那 时 节 功 成 免 罪 , 恢 复 你 本 来 的 职 务 , 你 的 心 下 怎 么 样 ? 那 怪 道 : 我 愿 意 皈 依 正 果 。
又 向 前 说 : 菩 萨 , 我 在 这 里 吃 人 无 数 , 以 前 有 几 次 取 经 的 人 来 , 都 被 我 吃 了 。
凡 是 吃 的 人 头 , 都 抛 落 流 沙 , 终 于 沉 入 水 底 。
其 水 , 鹅 毛 也 不 能 浮 。
只 有 九 个 , 取 经 人 的 骷 髅 浮 在 水 面 上 , 再 也 不 能 沉 入 水 中 。
我 以 为 是 奇 异 的 东 西 , 把 索 儿 穿 在 一 个 地 方 , 闲 时 拿 来 顽 戏 。
这 样 去 , 只 怕 取 经 人 不 能 到 这 里 去 , 还 不 是 反 而 误 了 我 的 前 程 吗 ? 菩 萨 说 : 岂 有 不 到 的 道 理 , 你 可 以 把 骷 髅 儿 挂 在 头 脖 子 下 , 等 候 取 经 人 , 自 有 用 处 。
怪 物 说 : 既 然 这 样 , 我 愿 意 听 你 的 教 诲 。
菩 萨 正 和 他 摩 顶 受 戒 , 指 沙 为 姓 , 就 姓 了 沙 , 起 个 法 名 叫 做 沙 悟 净 。
当 时 我 已 经 入 了 沙 门 , 送 菩 萨 过 了 黄 河 , 他 洗 心 洗 虑 , 再 也 不 伤 害 生 灵 , 专 门 等 于 取 经 的 人 。
菩 萨 与 他 告 别 了 , 同 木 叉 一 起 直 奔 东 土 。
走 了 很 长 时 间 , 又 看 见 一 座 高 山 , 山 上 有 恶 气 遮 蔽 , 不 能 走 上 去 。
正 想 驾 云 过 山 , 不 知 不 觉 狂 风 刮 起 来 的 地 方 , 又 闪 上 一 个 妖 魔 。
他 生 下 来 又 非 常 凶 险 , 只 看 见 他 说 : 卷 起 来 的 纹 纹 、 莲 蓬 、 挂 着 嘴 , 耳 朵 像 蒲 扇 一 样 显 出 金 睛 。
獠 牙 锋 利 如 钢 钉 , 长 嘴 张 开 像 火 盆 。
金 盔 紧 系 腮 边 带 , 勒 上 甲 丝 , 蟒 退 鳞 。
手 里 拿 着 钉 铃 , 龙 爪 , 腰 上 拿 着 弯 弓 , 月 亮 半 轮 。
纷 纷 威 风 欺 负 太 岁 , 昂 昂 的 志 气 压 倒 了 天 神 。
其 他 人 , 不 分 好 坏 , 望 着 菩 萨 举 起 钉 子 , 硬 硬 的 硬 硬 的 硬 硬 的 硬 硬 。
被 木 叉 行 者 挡 住 , 大 喝 一 声 说 道 : 那 泼 怪 , 不 要 无 礼 , 看 棒 。
妖 魔 说 : 这 个 和 尚 不 知 道 死 活 。
看 砒 。
两 个 人 在 山 底 下 , 一 冲 一 撞 , 赌 斗 赢 赢 , 真 是 好 杀 。 妖 魔 凶 猛 , 惠 岸 威 武 。
铁 棒 分 心 捣 , 钉 硬 硬 硬 的 钢 子 劈 面 迎 接 。
播 土 扬 尘 天 地 昏 暗 , 飞 砂 走 石 , 鬼 神 惊 恐 。
九 齿 钢 , 光 辉 闪 耀 , 双 环 响 响 , 一 根 木 棒 , 黑 色 悠 扬 , 两 手 飞 腾 。
其 所 以 是 , 其 所 以 是 也 , 其 所 以 是 也 。
一 个 在 普 陀 山 做 护 法 , 一 个 在 山 洞 里 做 妖 精 。
此 场 相 遇 争 高 下 , 不 知 道 那 个 亏 负 , 那 个 赢 。
其 两 个 人 正 好 杀 , 观 世 音 在 半 空 中 抛 下 莲 花 , 隔 开 钢 杖 。
怪 物 见 了 心 惊 , 便 问 道 : 你 是 那 里 的 和 尚 , 竟 敢 拿 什 么 眼 前 花 儿 哄 我 ? 木 叉 说 : 我 把 你 的 肉 眼 凡 胎 的 泼 物 , 我 是 南 海 菩 萨 的 弟 弟 。
这 是 我 师 父 抛 下 来 的 莲 花 , 你 也 不 认 得 啊 ! 那 怪 道 : 南 海 菩 萨 , 可 是 是 扫 除 三 灾 救 八 难 的 观 世 音 吗 ? 木 叉 说 : 不 是 他 是 谁 ? 怪 物 摘 下 钉 头 , 伸 出 头 礼 说 : 老 兄 , 菩 萨 在 哪 里 , 多 烦 你 引 见 一 引 见 。
木 叉 仰 面 指 着 说 : 那 不 是 ? 怪 物 朝 上 磕 头 , 厉 声 大 叫 道 : 菩 萨 , 恕 罪 , 恕 罪 。
观 音 磕 下 云 头 , 上 前 来 问 道 : 你 是 那 里 成 精 的 野 猪 , 什 么 地 方 作 怪 的 老 虎 , 敢 在 这 里 阻 挡 我 ? 那 怪 说 : 我 不 是 野 猪 , 也 不 是 老 , 我 本 来 是 天 河 里 的 天 蓬 元 帅 。
只 因 为 带 酒 戏 弄 嫦 娥 , 玉 帝 把 我 打 了 二 千 槌 , 贬 下 尘 世 。
一 个 灵 性 , 径 直 来 夺 舍 投 胎 , 不 料 错 了 道 路 , 投 入 一 个 母 猪 胎 里 , 变 成 了 这 样 的 模 样 。
是 我 咬 死 母 猪 , 打 死 群 猪 , 在 此 地 占 了 山 场 , 吃 人 度 日 。
不 期 撞 着 菩 萨 , 万 希 望 救 救 , 救 救 救 。
菩 萨 说 : 此 山 叫 什 么 山 ? 怪 物 说 : 叫 福 陵 山 。
山 中 有 一 个 洞 , 名 叫 云 栈 洞 。
洞 里 原 来 有 个 蛋 二 姐 , 他 见 我 有 些 武 艺 , 招 我 做 家 长 , 又 叫 他 做 倒 门 。
不 上 一 年 , 他 死 了 , 将 一 个 洞 的 家 产 , 全 都 归 我 所 用 。
在 这 里 时 间 久 了 , 年 岁 已 深 , 没 有 供 养 自 身 的 办 法 , 只 是 依 照 本 来 的 吃 人 度 日 。
万 万 希 望 菩 萨 饶 恕 罪 过 。
菩 萨 说 : 古 人 说 : 如 果 要 有 前 程 , 不 要 做 没 有 前 程 。
你 既 然 在 上 界 违 犯 法 律 , 现 在 又 不 改 恶 心 , 伤 生 造 孽 , 又 不 是 二 罪 俱 受 惩 罚 , 那 怪 道 : 前 程 , 前 程 , 如 果 依 你 , 教 我 喝 风 ? 常 说 : 依 照 官 法 打 死 , 依 照 佛 法 饿 死 。
去 掉 , 去 掉 , 还 不 如 抓 个 行 人 , 肥 滑 滑 的 吃 他 家 娘 , 何 必 二 罪 三 罪 , 千 罪 万 罪 ! 菩 萨 说 : 人 有 善 愿 , 上 天 必 定 会 顺 从 。
你 如 果 肯 归 依 正 果 , 自 有 养 身 之 处 。
世 上 有 五 谷 , 可 以 救 济 饥 饿 , 为 什 么 要 吃 人 度 日 呢 ? 怪 物 听 了 这 话 , 好 像 是 做 梦 , 才 醒 过 来 , 对 菩 萨 说 : 我 想 从 正 路 上 走 , 为 什 么 得 罪 于 上 天 , 没 有 什 么 祈 祷 的 呢 ?
菩 萨 说 : 我 领 了 佛 教 的 旨 意 , 到 东 土 寻 找 经 书 的 人 。
你 可 以 让 他 做 个 个 徒 弟 , 去 西 天 走 一 遭 , 将 功 败 罪 , 管 教 你 们 脱 离 灾 荒 瘴 气 。
那 怪 物 满 口 说 : 愿 意 跟 随 , 愿 意 跟 随 。
菩 萨 才 与 他 摩 顶 受 戒 , 以 身 为 姓 , 就 姓 为 猪 , 替 他 起 个 法 名 , 就 叫 做 猪 悟 能 。
于 是 领 命 归 真 , 持 斋 持 素 , 断 绝 五 荤 三 厌 , 专 门 等 那 取 经 人 。
菩 萨 又 和 木 叉 辞 别 了 解 自 己 的 能 力 , 一 半 兴 起 云 雾 前 来 。
正 走 的 地 方 , 只 见 空 中 有 一 条 玉 龙 叫 叫 。
那 龙 说 : 我 是 西 海 龙 王 敖 闰 的 儿 子 , 因 为 放 火 烧 了 殿 上 的 明 珠 , 我 父 亲 王 上 奏 朝 廷 , 告 发 了 忤 逆 。
玉 帝 把 我 吊 在 空 中 , 打 了 三 百 个 , 不 久 就 被 杀 。
希 望 菩 萨 拉 救 , 拉 救 。
观 音 听 了 这 话 , 就 和 木 叉 撞 上 南 天 门 里 , 早 早 有 丘 、 张 二 位 天 师 接 着 , 问 道 : 你 去 哪 里 ? 菩 萨 说 : 我 要 见 玉 帝 一 面 。
二 位 天 师 立 即 忙 忙 上 奏 。
玉 帝 于 是 下 殿 迎 接 。
菩 萨 上 前 礼 拜 完 毕 , 说 : 贫 僧 领 佛 旨 到 东 土 去 寻 找 经 书 的 人 , 路 上 遇 到 孽 龙 悬 吊 , 特 来 启 奏 , 饶 他 的 性 命 , 赐 给 贫 僧 , 让 他 和 取 经 人 做 个 脚 力 。
玉 帝 听 了 这 话 , 立 即 传 旨 赦 免 了 他 , 等 到 天 将 解 除 他 , 送 给 了 菩 萨 。
菩 萨 感 谢 恩 德 而 出 来 。
小 龙 叩 头 谢 恩 , 听 从 菩 萨 的 使 唤 。
菩 萨 把 他 送 到 深 涧 中 , 只 等 拿 经 书 的 人 来 , 变 成 白 马 , 上 到 西 方 去 立 功 。
小 龙 领 命 , 潜 身 不 问 题 。
菩 萨 带 着 木 叉 走 的 人 过 了 此 山 , 又 奔 向 东 土 。
走 了 不 多 时 , 忽 然 看 见 金 光 万 道 , 瑞 气 千 条 。
木 叉 说 : 师 父 , 那 放 光 的 地 方 , 就 是 五 行 山 了 , 现 在 有 如 来 的 压 贴 在 那 里 。
菩 萨 说 : 这 是 那 搅 乱 蟠 桃 会 、 大 闹 天 宫 的 齐 天 大 圣 , 现 在 却 被 压 在 这 里 。
木 叉 说 : 正 是 , 正 是 。
大 师 和 弟 子 都 上 了 山 来 , 观 看 帖 子 , 原 来 是 哇 么 喇 嘛 六 个 字 的 真 言 。
菩 萨 看 完 后 , 叹 惜 不 已 , 作 了 一 首 诗 。
诗 说 : 堪 叹 妖 猴 不 奉 公 , 当 年 狂 妄 逞 英 雄 。
欺 心 搅 乱 蟠 桃 会 , 大 胆 私 行 兜 率 宫 。
十 万 军 队 中 没 有 敌 手 , 九 重 天 上 有 威 风 。
自 从 遭 受 了 我 佛 如 来 的 困 难 , 何 日 舒 展 再 次 显 示 功 德 , 师 徒 们 正 在 说 话 的 地 方 , 早 就 惊 动 了 那 大 圣 。
大 圣 在 山 根 下 高 喊 道 : 是 那 个 在 山 上 吟 诗 , 揭 我 的 短 句 啊 。
只 见 石 崖 之 下 , 有 土 地 、 山 神 、 监 押 大 圣 的 天 将 , 都 来 拜 迎 菩 萨 , 引 导 到 那 大 圣 的 面 前 。
看 时 , 他 原 来 是 被 压 在 石 匣 里 , 口 能 说 话 , 身 不 能 动 。
大 圣 睁 开 火 眼 金 睛 , 点 着 头 儿 高 喊 道 : 我 何 不 认 得 你 , 你 好 的 是 南 海 普 陀 落 伽 山 救 苦 救 难 大 慈 大 悲 南 无 观 世 音 菩 萨 。
承 看 顾 , 承 看 顾 。
我 在 此 度 日 如 年 , 更 没 有 一 个 相 知 的 人 来 看 我 一 看 。
你 从 哪 里 来 呢 ? 菩 萨 说 : 我 奉 佛 旨 , 到 东 土 寻 找 经 人 去 , 从 这 里 经 过 , 特 地 留 下 残 余 的 步 子 看 你 。
大 圣 说 : 如 来 , 哄 骗 了 我 , 把 我 压 在 这 座 山 上 , 五 百 多 年 了 , 不 能 展 开 。
万 希 望 菩 萨 的 方 便 一 二 个 , 救 我 的 老 孙 子 , 救 一 个 。
菩 萨 说 : 你 这 个 奴 子 的 罪 孽 更 深 , 救 你 出 来 , 恐 怕 你 又 生 祸 害 , 反 而 算 不 好 。
大 圣 说 : 我 已 经 知 道 悔 恨 了 , 只 希 望 大 慈 悲 指 点 门 路 , 情 愿 修 行 。
夫 人 心 生 一 念 , 天 地 都 知 道 。
善 恶 如 果 没 有 报 应 , 天 地 一 定 有 私 心 。
那 菩 萨 听 了 这 话 , 满 心 欢 喜 , 对 大 圣 说 : 《 圣 经 》 上 说 : 《 圣 经 》 上 说 : 如 果 说 出 自 己 的 话 好 , 那 么 千 里 之 外 的 人 就 应 验 他 ; 如 果 说 出 自 己 的 话 不 好 , 那 么 千 里 之 外 的 人 就 违 背 他 。
卿 既 有 此 心 , 待 我 到 东 方 大 唐 国 去 寻 找 一 个 读 经 的 人 来 , 教 他 救 你 。
你 可 以 和 他 做 个 弟 弟 , 秉 持 教 诲 迦 持 , 进 入 我 佛 门 , 再 修 正 果 , 怎 么 样 ? 大 圣 声 声 说 道 : 愿 去 , 愿 去 , 愿 去 , 愿 去 。
菩 萨 说 : 既 然 有 善 果 , 我 给 你 起 个 法 名 。
大 圣 说 : 我 已 经 有 个 名 字 了 , 叫 做 孙 悟 空 。
菩 萨 又 高 兴 地 说 : 我 前 面 也 有 两 个 人 归 降 , 正 是 因 为 你 们 的 悟 字 排 列 , 你 现 在 也 是 悟 字 , 却 与 他 们 相 合 , 很 好 , 很 好 , 很 好 。
这 样 的 话 也 不 会 再 去 , 我 就 去 了 。
大 圣 见 性 明 心 归 于 佛 教 , 这 位 菩 萨 留 心 在 意 , 访 求 神 僧 。
他 与 木 叉 离 开 此 地 , 一 直 往 东 来 , 不 一 天 就 到 了 长 安 、 大 唐 国 。
收 敛 雾 气 收 敛 云 雾 , 师 徒 们 变 成 了 两 个 疥 疮 的 游 僧 , 进 入 长 安 城 里 , 很 早 就 不 知 道 天 晚 了 。
走 到 大 市 街 旁 , 看 见 一 座 土 地 庙 的 祠 堂 , 两 个 人 径 直 进 去 。
到 了 那 个 土 地 , 心 里 惊 慌 , 鬼 兵 胆 战 , 知 道 是 菩 萨 , 叩 头 接 进 去 。
那 土 地 又 急 忙 跑 去 报 告 给 城 隍 、 社 令 , 以 及 满 了 长 安 各 庙 的 神 祗 , 都 知 道 是 菩 萨 。
菩 萨 说 : 你 们 切 不 可 漏 掉 一 点 消 息 。
我 奉 佛 旨 , 特 来 此 处 寻 访 佛 经 的 人 。
借 你 的 庙 宇 , 暂 且 住 几 天 , 等 到 访 访 着 真 的 僧 人 就 回 去 。
众 神 各 自 回 到 自 己 的 住 处 , 把 他 的 土 地 赶 到 城 隍 庙 里 暂 住 , 他 的 师 徒 也 就 隐 藏 起 来 。
最 终 不 知 道 寻 出 那 个 取 经 的 人 来 , 暂 且 听 下 回 的 分 析 解 释 。
}\switchcolumn\flushpage  \begin{pinyinscope}{\myfontt \section{第九回}     陳光蕊赴任逢災 江流僧復讎報本

話表陝西大國長安城,乃歷代帝王建都之地。自周、秦、漢以來,三州花似錦,
八水繞城流,真個是名勝之邦。彼時是大唐太宗皇帝登基,改元貞觀,已登極十
三年,歲在己巳,天下太平,八方進貢,四海稱臣。

忽一日,太宗登位,聚集文武眾官,朝拜禮畢,有魏徵丞相出班奏道:「方今天
下太平,八方寧靜,應依古法,開立選場,招取賢士,擢用人材,以資化理。」
太宗道:「賢卿所奏有理。」就傳招賢文榜,頒布天下:各府州縣,不拘軍民人
等,但有讀書儒流,文義明暢,三場精通者,前赴長安應試。

此榜行至海州地方,有一人,姓陳名萼,表字光蕊,見了此榜,即時回家,對母
張氏道:「朝廷頒下黃榜,詔開南省,考取賢才,孩兒意欲前去應試。倘得一官
半職,顯親揚名,封妻蔭子,光耀門閭,乃兒之志也。特此稟告母親前去。」張
氏道:「我兒讀書人,『幼而學,壯而行』,正該如此。但去赴舉,路上須要小
心,得了官,早早回來。」

光蕊便吩咐家僮收拾行李,即拜辭母親,趲程前進。到了長安,正值大開選場,
光蕊就進場。考畢,中選。及廷試三策,唐王御筆親賜狀元,跨馬遊街三日。

不期遊到丞相殷開山門首,有丞相所生一女,名喚溫嬌,又名滿堂嬌,未曾婚配
,正高結彩樓,拋打繡毬卜婿。適值陳光蕊在樓下經過。小姐一見光蕊人材出眾
,知是新科狀元,心內十分歡喜,就將繡毬拋下,恰打著光蕊的烏紗帽。猛聽得
一派笙簫細樂,十數個婢妾走下樓來,把光蕊馬頭挽住,迎狀元入相府成婚。那
丞相和夫人即時出堂,喚賓人贊禮,將小姐配與光蕊。拜了天地,夫妻交拜畢,
又拜了岳丈、岳母。丞相吩咐安排酒席,歡飲一宵。二人同攜素手,共入蘭房。
  次日五更三點,太宗駕坐金鑾寶殿,文武眾臣趨朝。太宗問道:「新科狀元
陳光蕊應授何官?」魏徵丞相奏道:「臣查所屬州郡,有江州缺官,乞我主授他
此職。」太宗就命為江州州主,即令收拾起身,勿誤限期。光蕊謝恩出朝,回到
相府,與妻商議,拜辭岳丈、岳母,同妻前赴江州之任。離了長安登途。

正是暮春天氣,和風吹柳綠,細雨點花紅。光蕊便道回家,同妻交拜母親張氏。
張氏道:「恭喜我兒,且又娶親回來。」光蕊道:「孩兒叨賴母親福庇,忝中狀
元,欽賜遊街,經過丞相殷府門前,遇拋打繡毬適中,蒙丞相即將小姐招孩兒為
婿。朝廷除孩兒為江州州主,今來接取母親,同去赴任。」張氏大喜,收拾行程。

在路數日,前至萬花店劉小二家安下。張氏身體忽然染病,與光蕊道:「我身上
不安,且在店中調養兩日再去。」光蕊遵命。至次日早晨,見店門前有一人提著
個金色鯉魚叫賣,光蕊即將一貫錢買了。欲待烹與母親吃,只見鯉魚閃閃?眼。
光蕊驚異道:「聞說魚蛇?眼,必不是等閑之物。」遂問漁人道:「這魚那裏打
來的?」漁人道:「離府十五里洪江內打來的。」光蕊就把魚送在洪江裏去放了
生,回店對母親道知此事。張氏道:「放生好事,我心甚喜。」光蕊道:「此店
已住三日了,欽限緊急,孩兒意欲明日起身,不知母親身體好否?」張氏道:
「我身子不快,此時路上炎熱,恐添疾病。你可這裏賃間房屋,與我暫住,付些
盤纏在此。你兩口兒先上任去,候秋涼卻來接我。」光蕊與妻商議,就租了屋宇
,付了盤纏與母親,同妻拜辭前去。

途路艱苦,曉行夜宿,不覺已到洪江渡口。只見稍水劉洪、李彪二人,撐船到岸
迎接。也是光蕊前生合當有此災難,撞著這冤家。光蕊令家僮將行李搬上船去,
夫妻正齊齊上船,那劉洪睜眼看見殷小姐面如滿月,眼似秋波,櫻桃小口,綠柳
蠻腰,真個有沉魚落雁之容,閉月羞花之貌,陡起狼心。遂與李彪設計,將船撐
至沒人煙處。候至夜靜三更,先將家僮殺死,次將光蕊打死,把尸首都推在水裏
去了。小姐見他打死了丈夫,也便將身赴水。劉洪一把抱住道:「你若從我,萬
事皆休﹔若不從時,一刀兩斷。」那小姐尋思無計,只得權時應承,順了劉洪。
那賊把船渡到南岸,將船付與李彪自管,他就穿了光蕊衣冠,帶了官憑,同小姐
往江州上任去了。

卻說劉洪殺死的家僮屍首,順水流去。惟有陳光蕊的屍首,沉在水底不動。有洪
江口巡海夜叉見了,星飛報入龍宮,正值龍王升殿,夜叉報道:「今洪江口不知
甚人把一個讀書士子打死,將屍撇在水底。」龍王叫將屍抬來,放在面前,仔細
一看道:「此人正是救我的恩人,如何被人謀死?常言道:『恩將恩報。』我今
日須索救他性命,以報日前之恩。」即寫下牒文一道,差夜叉徑往洪州城隍、土
地處投下,要取秀才魂魄來,救他的性命。城隍、土地遂喚小鬼把陳光蕊的魂魄
交付與夜叉去。夜叉帶了魂魄到水晶宮,稟見了龍王。

龍王問道:「你這秀才姓甚名誰?何方人氏?因甚到此,被人打死?」光蕊施禮
道:「小生陳萼,表字光蕊,係海州弘農縣人。忝中新科狀元,叨授江州州主,
同妻赴任。行至江邊上船,不料稍子劉洪貪謀我妻,將我打死拋屍。乞大王救我
一救。」龍王聞言道:「原來如此。先生,你前者所放金色鯉魚,即我也。你是
救我的恩人,你今有難,我豈有不救你之理?」就把光蕊屍身安置一壁,口內含
一顆定顏珠,休教損壞了,日後好還魂報仇。又道:「汝今真魂,權且在我水府
中做個都領。」光蕊叩頭拜謝,龍王設宴相待不題。

卻說殷小姐痛恨劉賊,恨不食肉寢皮。只因身懷有孕,未知男女,萬不得已,權
且勉強相從。轉盼之間,不覺已到江州。吏書門皂,俱來迎接。所屬官員,公堂
設宴相敘。劉洪道:「學生到此,全賴諸公大力匡持。」屬官答道:「堂尊大魁
高才,自然視民如子,訟簡刑清。我等合屬有賴,何必過謙?」公宴已罷,眾人
各散。

光陰迅速。一日,劉洪公事遠出。小姐在衙思念婆婆、丈夫,在花亭上感嘆。忽
然身體困倦,腹內疼痛,暈悶在地,不覺生下一子。耳邊有人囑曰:「滿堂嬌,
聽吾叮囑:吾乃南極星君,奉觀音菩薩法旨,特送此子與你。異日聲名遠大,非
比等閑。劉賊若回,必害此子,汝可用心保護。汝夫已得龍王相救,日後夫妻相
會,子母團圓,雪冤報仇有日也。謹記吾言。快醒,快醒。」言訖而去。

小姐醒來,句句記得,將子抱定,無計可施。忽然劉洪回來,一見此子,便要淹
殺。小姐道:「今日天色已晚,容待明日拋去江中。」幸喜次早劉洪忽有緊急公
事遠出。小姐暗思:「此子若待賊人回來,性命休矣。不如及早拋棄江中,聽其
生死。倘或皇天見憐,有人救得,收養此子,他日還得相逢。」但恐難以識認,
即咬破手指,寫下血書一紙,將父母姓名、跟腳緣由,備細開載﹔又將此子左腳
上一個小指,用口咬下,以為記驗。取貼身汗衫一件,包裹此子,乘空抱出衙門
。幸喜官衙離江不遠。小姐到了江邊,大哭一場。正欲拋棄,忽見江岸岸側飄起
一片木板,小姐即朝天拜禱,將此子安在板上,用帶縛住,血書繫在胸前,推放
江中,聽其所之。小姐含淚回衙不題。

卻說此子在木板上順水流去,一直流到金山寺腳下停住。那金山寺長老叫做法明
和尚,修真悟道,已得無生妙訣。正當打坐參禪,忽聞得小兒啼哭之聲,一時心
動,急到江邊觀看,只見涯邊一片木板上,睡著一個嬰兒。長老慌忙救起,見了
懷中血書,方知來歷。取個乳名,叫做江流,託人撫養。血書緊緊收藏。

光陰似箭,日月如梭。不覺江流年長一十八歲。長老就叫他削髮修行,取法名為
玄奘,摩頂受戒,堅心修道。

一日,暮春天氣,眾人同在松陰之下講經參禪,談說奧妙,那酒肉和尚恰被玄奘
難倒。和尚大怒,罵道:「你這業畜,姓名也不知,父母也不識,還在此搗甚麼
鬼?」玄奘被他罵出這般言語,入寺跪告師父,眼淚雙流道:「人生於天地之間
,稟陰陽而資五行,盡由父生母養,豈有為人在世而無父母者乎?」再三哀告,
求問父母姓名。長老道:「你真個要尋父母,可隨我到方丈裏來。」玄奘就跟到
方丈。長老到重梁之上,取下一個小匣兒,打開來,取出血書一紙、汗衫一件,
付與玄奘。玄奘將血書拆開讀之,才備細曉得父母姓名,並冤仇事跡。

玄奘讀罷,不覺哭倒在地道:「父母之仇,不能報復,何以為人?十八年來,不
識生身父母,至今日方知有母親。此身若非師父撈救撫養,安有今日?容弟子去
尋見母親,然後頭頂香盆,重建殿宇,報答師父之深恩也。」師父道:「你要去
尋母,可帶這血書與汗衫前去。只做化緣,徑往江州私衙,才得你母親相見。」

玄奘領了師父言語,就做化緣的和尚,徑至江州。適值劉洪有事出外,也是天叫
他母子相會,玄奘就直至私衙門口抄化。那殷小姐原來夜間得了一夢,夢見月缺
再圓,暗想道:「我婆婆不知音信﹔我丈夫被這賊謀殺﹔我的兒子拋在江中,倘
若有人收養,算來有十八歲矣,或今日天教相會,亦未可知。」正沉吟間,忽聽
私衙前有人念經,連叫「抄化」,小姐又乘便出來問道:「你是何處來的?」玄
奘答道:「貧僧乃是金山寺法明長老的徒弟。」小姐道:「你既是金山寺長老的
徒弟……」叫進衙來,將齋飯與玄奘吃。仔細看他舉止言談,好似與丈夫一般。

小姐將從婢打發開去,問道:「你這小師父,還是自幼出家的,還是中年出家的
?姓甚名誰?可有父母否?」玄奘答道:「我也不是自幼出家,我也不是中年出
家,我說起來,冤有天來大,仇有海樣深:我父被人謀死,我母卻被賊人占了。
我師父法明長老教我在江州衙內尋取母親。」小姐問道:「你母姓甚?」玄奘道
:「我母姓殷,名喚溫嬌。我父姓陳,名光蕊。我小名叫做江流,法名取為玄奘
。」小姐道:「溫嬌就是我。但你今有何憑據?」玄奘聽說是他母親,雙膝跪下
,哀哀大哭:「我娘若不信,見有血書、汗衫為證。」溫嬌取過一看,果然是真
,母子相抱而哭。就叫:「我兒快去。」玄奘道:「十八年不識生身父母,今朝
才見母親,教孩兒如何割捨?」小姐道:「我兒,你火速抽身前去。劉賊若回,
他必害你性命。我明日假裝一病,只說先年曾許捨百雙僧鞋,來你寺中還願。那
時節,我有話與你說。」玄奘依言拜別。

卻說小姐自見兒子之後,心內一憂一喜。忽一日推病,茶飯不吃,臥於床上。劉
洪歸衙,問其原故。小姐道:「我幼時曾許下一願,許捨僧鞋一百雙。昨五日之
前,夢見個和尚手執利刃,要索僧鞋,便覺身子不快。」劉洪道:「這些小事,
何不早說?」隨升堂,吩咐王左衙、李右衙:江州城內百姓,每家要辦僧鞋一雙
,限五日內完納。百姓俱依派完納訖。小姐對劉洪道:「僧鞋做完,這裏有甚麼
寺院,好去還願?」劉洪道:「這江州有個金山寺、焦山寺,聽你在那個寺裏去
。」小姐道:「久聞金山寺好個寺院,我就往金山寺去。」劉洪即喚王、李二衙
辦下船隻。小姐帶了心腹人,同上了船,稍水將船撐開,就投金山寺去。

卻說玄奘回寺,見法明長老,把前項說了一遍。長老甚喜。次日,只見一個丫鬟
先到,說夫人來寺還願。眾僧都出寺迎接。小姐徑進寺門,參了菩薩,大設齋襯
。喚丫鬟將僧鞋暑襪托於盤內,來到法堂,小姐復拈心香禮拜,就教法明長老分
俵與眾僧去訖。玄奘見眾僧散了,法堂上更無一人,他卻近前跪下。小姐叫他脫
了鞋襪看時,那左腳上果然少了一個小指頭。當時兩個又抱住而哭,拜謝長老養
育之恩。法明道:「汝今母子相會,恐奸賊知之,可速速抽身回去,庶免其禍。」
小姐道:「我兒,我與你一隻香環,你徑到洪州西北地方,約有一千五百里之程
,那裏有個萬花店,當時留下婆婆張氏在那裏,是你父親生身之母。我再寫一封
書與你,徑到唐王皇城之內,金殿左邊,殷開山丞相家,是你母生身之父母。你
將我的書遞與外公,叫外公奏上唐王,統領人馬,擒殺此賊,與父報仇,那時才
救得老娘的身子出來。我今不敢久停,誠恐賊漢怪我歸遲。」便出寺登舟而去。

玄奘哭回寺中,告過師父,即時拜別,徑往洪州。來到萬花店,問那店主劉小二
道:「昔年江州陳客官有一母親住在你店中,如今好麼?」劉小二道:「他原在
我店中。後來昏了眼,三四年並無店租還我。如今在南門頭一個破瓦?裏,每日
上街叫化度日。那客官一去許久,到如今杳無信息,不知為何。」玄奘聽罷,即
時問到南門頭破瓦?,尋著婆婆。婆婆道:「你聲音好似我兒陳光蕊。」玄奘道
:「我不是陳光蕊,我是陳光蕊的兒子。溫嬌小姐是我的娘。」婆婆道:「你爹
娘怎麼不來?」玄奘道:「我爹爹被強盜打死了,我娘被強盜霸占為妻。」婆婆
道:「你怎麼曉得來尋我?」玄奘道:「是我娘著我來尋婆婆。我娘有書在此,
又有香環一隻。」那婆婆接了書並香環,放聲痛哭道:「我兒為功名到此,我只
道他背義忘恩,那知他被人謀死。且喜得皇天憐念,不絕我兒之後,今日還有孫
子來尋我。」玄奘問:「婆婆的眼,如何都昏了?」婆婆道:「我因思量你父親
,終日懸望,不見他來,因此上哭得兩眼都昏了。」

玄奘便跪倒向天禱告道:「今玄奘一十八歲,父母之仇不能報復。今日領母命來
尋婆婆,天若憐鑒弟子誠意,保我婆婆雙眼復明。」祝罷,就將舌尖與婆婆舔眼
。須臾之間,雙眼舔開,仍復如初。婆婆覷了小和尚道:「你果是我的孫子,恰
和我兒子光蕊形容無二。」婆婆又喜又悲。玄奘就領婆婆出了?門,還到劉小二
店內。將些房錢賃屋一間,與婆婆棲身。又將盤纏與婆婆道:「我此去,只月餘
就回。」

隨即辭了婆婆,徑往京城。尋到皇城東街殷丞相府上,與門上人道:「小僧是親
戚,來探相公。」門上人稟知丞相,丞相道:「我與和尚並無親眷。」夫人道:
「我昨夜夢見我女兒滿堂嬌來家,莫不是女婿有書信回來也?」丞相便教請小和
尚來到廳上。小和尚見了丞相與夫人,哭拜在地,就懷中取出一封書來,遞與丞
相。丞相拆開,從頭讀罷,放聲痛哭。夫人問道:「相公,有何事故?」丞相道
:「這和尚是我與你的外孫。女婿陳光蕊被賊謀死,滿堂嬌被賊強占為妻。」夫
人聽罷,亦痛哭不止。丞相道:「夫人休得煩惱,來朝奏知主上,親自統兵,定
要與女婿報仇。」

次日,丞相入朝,啟奏唐王曰:「今有臣婿狀元陳光蕊,帶領家小江州赴任,被
稍水劉洪打死,占女為妻﹔假冒臣婿,為官多年。事屬異變,乞陛下立發人馬,
剿除賊寇。」唐王見奏大怒,就發御林軍六萬,著殷丞相督兵前去。丞相領旨出
朝,即往教場內點了兵,徑往江州進發。曉行夜宿,星落鳥飛,不覺已到江州,
殷丞相兵馬俱在北岸下了營寨。星夜令金牌下戶喚到江州同知、州判二人,丞相
對他說知此事,叫他提兵相助,一同過江而去。天尚未明,就把劉洪衙門圍了。
劉洪正在夢中,聽得火炮一響,金鼓齊鳴,眾兵殺進私衙,劉洪措手不及,早被
擒住。丞相傳下軍令,將劉洪一干人犯綁赴法場,令眾軍俱在城外安營去了。

丞相直入衙內正廳坐下,請小姐出來相見。小姐欲待要出,羞見父親,就要自縊
。玄奘聞知,急急將母解救,雙膝跪下,對母道:「兒與外公統兵至此,與父報
仇。今日賊已擒捉,母親何故反要尋死?母親若死,孩兒豈能存乎?」丞相亦進
衙勸解。小姐道:「吾聞『婦人從一而終』。痛夫已被賊人所殺,豈可靦顏從賊
?止因遺腹在身,只得忍恥偷生。今幸兒已長大,又見老父提兵報仇,為女兒者
,有何面目相見?惟有一死以報丈夫耳。」丞相道:「此非我兒以盛衰改節,皆
因出乎不得已,何得為恥?」父子相抱而哭,玄奘亦哀哀不止。丞相拭淚道:
「你二人且休煩惱﹔我今已擒捉仇賊,且去發落去來。」即起身到法場。恰好江
州同知亦差哨兵拿獲水賊李彪解到。丞相大喜,就令軍牢押過劉洪、李彪,每人
痛打一百大棍,取了供狀,招了先年不合謀死陳光蕊情由,先將李彪釘在木驢上
,推去市曹,剮了千刀,梟首示眾訖。把劉洪拿到洪江渡口,先年打死陳光蕊處
。丞相與小姐、玄奘三人親到江邊,望空祭奠,活剜取劉洪心肝,祭了光蕊,燒
了祭文一道。

三人望江痛哭,早已驚動水府,有巡海夜叉將祭文呈與龍王。龍王看罷,就差鱉
元帥去請光蕊來到,道:「先生,恭喜,恭喜。今有先生夫人、公子同岳丈俱在
江邊祭你。我今送你還魂去也。再有如意珠一顆、走盤珠二顆、絞綃十端、明珠
玉帶一條奉送。你今日便可夫妻子母相會也。」光蕊再三拜謝。龍王就令夜叉將
光蕊身屍送出江口還魂。夜叉領命而去。

卻說殷小姐哭奠丈夫一番,又欲將身赴水而死,慌得玄奘拚命扯住。正在倉皇之
際,忽見水面上一個死屍浮來,靠近江岸之傍。小姐忙向前認看,認得是丈夫的
屍首,一發嚎啕大哭不已。眾人俱來觀看,只見光蕊舒拳伸腳,身子漸漸展動,
忽地爬將起來坐下。眾人不勝驚駭。光蕊睜開眼,早見殷小姐與丈人殷丞相同著
小和尚俱在身邊啼哭。光蕊道:「你們為何在此?」小姐道:「因汝被賊人打死
,後來妾身生下此子,幸遇金山寺長老撫養長大,尋我相會,我教他去尋外公。
父親得知,奏聞朝廷,統兵到此,拿住賊人,適才生取心肝,望空祭奠我夫。不
知我夫怎生又得還魂?」光蕊道:「皆因我與你昔年在萬花店時,買放了那尾金
色鯉魚,誰知那鯉魚就是此處龍王。後來逆賊把我推在水中,全虧得他救我。方
才又賜我還魂,送我寶物,俱在身上。更不想你生下這兒子,又得岳丈為我報仇
。真是苦盡甘來,莫大之喜。」

眾官聞知,都來賀喜。丞相就令安排酒席,答謝所屬官員。即日軍馬回程。來到
萬花店,那丞相傳令安營。光蕊便同玄奘到劉家店尋婆婆。那婆婆當夜得了一夢
,夢見枯木開花,屋後喜鵲頻頻喧噪,想道:「莫不是我孫兒來也?」說猶未了
,只見店門外,光蕊父子齊到。小和尚指道:「這不是俺婆婆?」光蕊見了老母
,連忙拜倒。母子抱頭痛哭一場,把上項事說了一遍。算還了小二店錢,起程回
到京城。進了相府,光蕊同小姐與婆婆、玄奘都來見了夫人。夫人不勝之喜,吩
咐家僮,大排筵宴慶賀。丞相道:「今日此宴,可取名為團圓會。」真正合家歡
樂。

次日早朝,唐王登殿。殷丞相出班,將前後事情備細啟奏,並薦光蕊才可大用。
唐王准奏,即命陞陳萼為學士之職,隨朝理政。玄奘立意安禪,送在洪福寺內修
行。後來,殷小姐畢竟從容自盡。玄奘自到金山寺中報答法明長老。

    不知後來事體若何,且聽下回分解。





}  \end{pinyinscope}\switchcolumn{\myfontc \section{第 九 回} 陈 光 蕊 赴 任 , 遇 灾 江 流 , 僧 复 报 本 话 , 上 表 说 : 陕 西 大 国 长 安 城 , 是 历 代 帝 王 建 都 的 地 方 。
自 从 周 、 秦 、 汉 以 来 , 三 州 花 似 锦 缎 , 八 水 绕 城 流 , 真 是 名 胜 之 邦 。
当 时 是 大 唐 太 宗 皇 帝 登 基 , 改 年 号 为 贞 观 , 已 经 登 上 帝 位 十 三 年 , 正 值 己 巳 , 天 下 太 平 , 八 方 进 贡 , 四 海 称 臣 。
忽 然 有 一 天 , 太 宗 登 上 皇 位 , 召 集 文 武 百 官 , 朝 拜 礼 毕 , 有 魏 徵 、 丞 相 出 班 奏 道 : 现 在 天 下 太 平 , 八 方 安 宁 , 应 该 依 照 古 代 的 法 则 , 开 设 选 拔 官 员 , 招 纳 贤 士 , 提 拔 人 才 , 以 资 助 教 化 。
太 宗 说 : 贤 卿 的 奏 章 有 理 。
于 是 就 地 传 布 招 贤 文 榜 , 颁 布 天 下 : 各 府 州 县 , 不 拘 军 民 人 等 , 只 有 读 书 儒 生 , 文 章 义 理 明 畅 , 三 场 精 通 的 , 前 赴 长 安 参 加 考 试 。
这 个 榜 子 送 到 海 州 地 方 , 有 一 个 姓 陈 名 萼 , 表 字 光 蕊 , 见 了 这 个 榜 子 , 立 即 回 家 , 对 母 亲 张 氏 说 : 朝 廷 颁 下 黄 榜 , 诏 令 开 设 南 省 , 考 取 贤 才 , 孩 子 想 前 去 应 试 。
如 果 能 得 到 一 官 半 职 , 显 赫 亲 人 , 扬 名 , 封 他 的 妻 子 , 荫 子 , 光 耀 门 , 这 是 儿 子 的 志 向 。
特 此 禀 告 母 亲 前 去 。
张 氏 说 : 我 儿 子 是 读 书 人 , 年 幼 时 就 学 习 , 壮 年 时 就 行 动 , 正 应 该 这 样 。
但 去 参 加 科 举 考 试 , 路 上 必 须 小 心 , 得 到 官 职 , 早 早 回 来 。
光 蕊 就 指 点 家 僮 收 拾 行 李 , 立 即 拜 谢 母 亲 , 并 且 快 步 前 进 。
及 之 。
考 核 完 毕 , 中 选 。
等 到 廷 试 三 策 , 唐 王 亲 自 赐 给 状 元 , 跨 马 游 街 三 天 。
不 料 游 到 丞 相 殷 开 山 的 门 前 , 有 个 丞 相 生 了 一 个 女 儿 , 名 叫 温 娇 , 又 名 叫 满 堂 娇 , 还 没 有 结 婚 , 正 在 高 楼 结 婚 , 打 开 绣 找 女 婿 。
正 好 遇 上 陈 光 蕊 在 楼 下 经 过 。
小 姐 一 见 光 蕊 人 才 出 众 , 知 道 是 新 科 的 状 元 , 心 里 十 分 欢 喜 , 就 把 绣 扔 下 去 , 恰 好 打 着 光 蕊 的 乌 纱 帽 。
王 猛 听 到 一 股 笙 箫 细 乐 , 十 几 个 婢 妾 走 下 楼 来 , 把 光 蕊 马 头 挽 住 , 把 状 元 迎 到 相 府 结 婚 。
那 丞 相 和 夫 人 立 即 出 了 堂 , 叫 来 宾 客 赞 礼 , 把 小 姐 配 给 光 蕊 。
夫 妻 交 相 拜 , 又 拜 岳 父 、 岳 母 。
丞 相 吩 咐 安 排 酒 席 , 欢 饮 一 夜 。
二 人 一 同 携 带 着 素 手 , 一 同 进 入 兰 房 。
第 二 天 五 更 三 点 , 太 宗 驾 临 金 銮 宝 殿 , 文 武 大 臣 趋 朝 。
太 宗 问 道 : 新 科 的 状 元 陈 光 蕊 应 该 授 予 什 么 官 职 ? 魏 徵 、 丞 相 上 奏 说 : 我 查 阅 所 属 州 郡 , 有 江 州 缺 少 官 职 , 请 我 主 管 授 予 他 这 个 职 务 。
太 宗 就 任 命 他 为 江 州 州 主 , 立 即 命 令 他 收 拾 起 身 , 不 要 耽 误 期 限 。
光 蕊 谢 恩 出 朝 , 回 到 相 府 , 与 妻 子 商 议 , 拜 辞 岳 父 、 岳 母 , 同 妻 子 前 往 江 州 赴 任 。
离 开 长 安 , 登 上 路 途 。
正 是 暮 春 天 气 , 和 风 吹 柳 绿 , 细 雨 点 花 红 。
光 蕊 便 道 回 家 , 和 妻 子 一 起 拜 见 母 亲 张 氏 。
张 氏 说 : 恭 喜 我 儿 子 , 况 且 又 娶 亲 回 来 。
光 蕊 说 : 孩 子 叨 叨 依 赖 母 亲 的 福 祐 庇 护 , 中 状 元 , 钦 赐 到 街 上 游 览 , 经 过 丞 相 殷 府 门 前 , 遇 到 抛 打 绣 恰 好 打 中 , 承 蒙 丞 相 马 上 带 着 小 姐 招 唤 小 姐 为 婿 。
朝 廷 任 命 孩 儿 为 江 州 州 主 , 现 在 来 接 取 他 的 母 亲 , 一 同 去 赴 任 。
张 氏 大 喜 , 收 拾 行 程 。
在 路 上 走 了 几 天 , 先 前 到 万 花 店 刘 小 二 家 安 下 。
张 氏 的 身 体 忽 然 染 病 , 对 光 蕊 说 : 我 身 上 不 安 , 暂 且 在 店 里 调 养 , 两 天 后 再 去 。
光 蕊 听 从 了 命 令 。
到 第 二 天 早 晨 , 看 见 店 门 前 有 一 个 人 提 着 一 个 金 色 的 鲤 鱼 叫 卖 , 张 光 蕊 就 用 一 贯 钱 买 了 。
想 等 烹 煮 和 母 亲 吃 , 只 见 鲤 鱼 闪 闪 着 眼 睛 。
光 蕊 惊 异 地 说 : 听 说 鱼 、 蛇 的 眼 睛 , 一 定 不 是 等 闲 的 东 西 。
于 是 问 渔 人 说 : 这 鱼 是 从 哪 里 打 来 的 ? 渔 人 说 : 离 府 城 十 五 里 洪 江 里 打 来 的 。
光 蕊 就 把 鱼 送 到 洪 江 里 去 , 放 了 生 。
张 氏 说 : 放 生 好 事 , 我 心 里 很 高 兴 。
张 光 蕊 说 : 此 店 已 住 三 天 了 , 钦 限 紧 急 , 孩 子 想 明 天 起 身 , 不 知 你 母 亲 的 身 体 好 吗 ? 张 氏 说 : 我 的 身 体 不 快 , 此 时 路 上 炎 热 , 恐 怕 添 了 疾 病 。
你 可 在 这 里 借 一 间 房 屋 , 给 我 暂 住 , 给 你 们 盘 缠 在 这 里 。
你 们 两 个 儿 子 先 上 任 去 , 等 到 秋 天 凉 爽 时 再 来 接 我 。
光 蕊 与 妻 子 商 议 , 就 把 房 屋 租 给 了 他 , 给 了 他 的 盘 子 和 母 亲 , 同 妻 子 拜 辞 前 去 。
路 途 艰 苦 , 晓 行 夜 宿 , 不 知 不 觉 已 到 达 洪 江 渡 口 。
只 见 稍 微 有 水 , 刘 洪 、 李 彪 二 人 , 撑 着 船 到 岸 边 迎 接 。
也 是 光 蕊 前 生 应 当 有 这 样 的 灾 难 , 撞 着 这 个 冤 家 。
光 蕊 叫 家 僮 把 行 李 搬 到 船 上 去 , 夫 妻 正 一 齐 上 船 , 那 刘 洪 抬 着 眼 睛 看 见 殷 小 姐 的 脸 如 满 月 , 眼 睛 似 秋 波 , 桃 花 小 口 , 绿 柳 蛮 腰 , 真 是 有 沉 鱼 落 雁 的 容 貌 , 闭 月 羞 花 的 容 貌 , 突 然 产 生 狼 心 。
于 是 与 李 彪 设 计 , 把 船 撑 到 没 人 烟 的 地 方 。
等 到 夜 静 三 更 时 , 先 把 家 僮 杀 死 , 再 把 光 蕊 打 死 , 把 尸 首 都 推 到 水 里 去 了 。
小 姐 见 他 打 死 了 男 子 , 就 带 着 自 己 跳 进 水 里 去 。
刘 洪 一 把 握 住 他 说 : 你 如 果 跟 随 我 , 万 事 都 可 以 完 成 ; 如 果 不 顺 从 你 的 话 , 一 刀 两 断 。
这 个 小 姐 没 有 办 法 , 只 得 暂 时 应 承 , 顺 应 了 刘 洪 。
其 贼 乘 船 渡 到 南 岸 , 将 船 交 给 李 彪 , 他 就 穿 上 光 蕊 的 衣 服 , 带 上 官 府 的 凭 证 , 同 小 妹 去 江 州 上 任 去 。
又 说 : 刘 洪 杀 死 家 僮 的 尸 首 , 顺 水 流 去 。
唯 有 陈 光 蕊 的 尸 首 , 沉 在 水 底 不 动 。
有 洪 江 口 巡 海 夜 叉 见 了 , 星 飞 报 告 进 入 龙 宫 , 正 赶 上 龙 王 登 上 殿 堂 , 夜 叉 报 告 说 : 现 在 洪 江 口 不 知 有 什 么 人 把 一 个 读 书 的 士 子 打 死 了 , 把 尸 体 扔 在 水 底 。
龙 王 叫 他 把 尸 体 抬 起 来 , 放 在 脸 前 , 仔 细 一 看 , 说 : 这 个 人 正 是 救 我 的 恩 人 , 为 什 么 要 被 人 打 死 ?
我 今 天 要 求 救 他 的 性 命 , 以 报 答 我 以 前 的 恩 德 。
于 是 就 写 了 一 道 文 书 , 让 夜 叉 径 直 到 洪 州 城 隍 、 土 地 的 地 方 投 下 去 , 要 取 出 秀 才 的 魂 魄 来 , 救 他 的 性 命 。
城 隍 、 土 地 就 叫 小 鬼 , 把 陈 光 蕊 的 魂 魄 交 给 夜 叉 去 。
夜 叉 带 着 魂 魄 到 水 晶 宫 , 禀 报 了 龙 王 。
龙 王 问 道 : 你 这 个 秀 才 姓 什 么 名 叫 什 么 , 是 什 么 地 方 人 , 因 为 这 么 大 , 被 人 打 死 了 光 蕊 施 礼 说 : 小 生 陈 萼 , 表 字 光 蕊 , 是 海 州 弘 农 县 人 。
李 考 中 新 科 的 状 元 , 被 授 予 江 州 州 主 , 与 妻 子 一 起 赴 任 。
走 到 江 边 上 了 船 , 不 料 张 稍 的 儿 子 刘 洪 贪 图 谋 害 我 的 妻 子 , 把 我 打 死 抛 尸 。
请 大 王 救 我 一 次 。
龙 王 听 了 后 说 : 原 来 就 是 这 样 。
先 生 , 你 先 前 所 放 的 金 色 鲤 鱼 , 就 是 我 。
你 现 在 有 灾 难 , 我 难 道 不 救 你 的 道 理 , 就 把 光 蕊 的 尸 体 安 放 在 一 个 墙 壁 里 , 口 里 含 着 一 颗 定 颜 珠 。
又 说 : 你 现 在 真 正 的 魂 魄 , 暂 且 在 我 的 水 府 中 做 个 都 领 。
光 蕊 叩 头 拜 谢 , 龙 王 设 宴 相 待 , 没 有 问 题 。
又 说 殷 小 姐 痛 恨 刘 贼 , 恨 不 吃 肉 睡 皮 。
只 因 为 我 身 怀 有 孕 , 不 知 道 男 女 , 万 不 得 已 , 暂 且 勉 强 跟 从 。
转 眼 之 间 , 不 知 不 觉 已 到 了 江 州 。
官 吏 书 信 门 吏 , 都 来 迎 接 。
所 属 官 员 , 在 公 堂 设 宴 互 相 叙 述 。
刘 洪 道 说 : 学 生 到 这 里 , 全 靠 诸 公 大 力 匡 扶 。
他 的 属 官 回 答 说 : 堂 尊 大 魁 有 才 能 , 自 然 就 把 百 姓 看 作 儿 子 , 诉 讼 简 明 刑 罚 清 明 。
我 们 都 属 于 有 赖 的 人 , 何 必 过 分 谦 逊 呢 公 宴 已 经 结 束 , 众 人 各 自 散 去 。
光 阴 迅 速 。
一 天 , 刘 洪 公 事 远 出 。
小 姐 在 衙 门 里 思 念 婆 婆 、 丈 夫 , 在 花 亭 上 感 叹 。
忽 然 身 体 困 倦 , 腹 中 疼 痛 , 头 晕 肿 在 地 上 , 不 知 不 觉 生 下 一 个 孩 子 。
耳 边 有 人 嘱 咐 他 说 : 满 堂 娇 儿 , 听 我 嘱 咐 说 : 我 是 南 极 星 君 , 奉 观 音 菩 萨 的 法 旨 , 特 地 送 这 个 孩 子 给 你 。
他 日 他 的 声 名 远 大 , 不 能 和 他 同 等 。
刘 贼 如 果 回 来 , 必 定 杀 害 这 个 人 , 你 可 以 用 心 保 护 。
你 的 丈 夫 已 经 得 到 了 龙 王 的 救 助 , 以 后 夫 妻 会 合 , 子 母 团 团 , 洗 雪 冤 仇 , 报 仇 就 有 日 子 了 。
恭 敬 地 记 住 我 的 话 。
快 醒 , 快 醒 。
说 完 就 离 去 了 。
小 姐 醒 来 , 一 句 话 都 记 得 , 把 孩 子 抱 住 , 没 有 办 法 可 施 。
忽 然 刘 洪 回 来 , 一 见 到 这 个 人 , 就 要 把 他 淹 死 。
小 姐 说 : 今 天 天 色 已 晚 , 等 到 明 天 把 我 扔 到 江 中 去 吧 。
幸 好 第 二 天 早 晨 , 刘 洪 忽 然 有 紧 急 公 事 远 出 。
小 姐 暗 想 : 这 个 孩 子 如 果 等 贼 人 回 来 , 性 命 就 好 了 。
不 如 及 早 抛 弃 江 中 , 听 任 他 的 生 死 。
倘 若 皇 天 怜 悯 我 , 有 人 救 他 , 收 养 这 个 孩 子 , 以 后 还 能 相 遇 。
只 恐 怕 难 以 认 认 , 就 咬 破 手 指 , 写 下 一 页 血 书 , 把 父 母 的 姓 名 和 脚 跟 脚 的 缘 由 , 细 细 地 打 开 , 又 把 这 孩 子 的 左 脚 上 的 一 个 小 指 , 用 口 咬 下 来 , 作 为 记 录 。
他 拿 出 一 件 贴 身 汗 衫 , 包 裹 着 这 个 儿 子 , 乘 空 抱 出 衙 门 。
幸 好 官 衙 离 江 不 远 。
小 妹 到 了 江 边 , 大 哭 一 场 。
正 要 抛 弃 , 忽 然 看 见 江 岸 边 有 一 块 木 板 , 小 姐 就 上 朝 祈 祷 , 把 这 个 孩 子 安 放 在 木 板 上 , 用 带 子 捆 住 , 用 血 书 系 在 胸 前 , 推 放 到 江 中 , 听 他 到 哪 里 去 。
小 姐 含 着 眼 泪 回 到 衙 门 , 没 有 题 字 。
又 说 : 这 个 人 在 木 板 上 顺 水 流 去 , 一 直 流 到 金 山 寺 脚 下 停 住 。
金 山 寺 的 长 老 叫 法 明 和 尚 , 修 炼 真 道 , 已 经 得 到 了 无 生 妙 诀 。
正 当 打 坐 参 禅 , 忽 然 听 到 小 孩 啼 哭 的 声 音 , 一 时 之 间 心 动 , 急 忙 到 江 边 去 观 看 , 只 见 岸 边 一 片 木 板 上 , 睡 着 着 一 个 婴 儿 。
长 老 急 忙 把 他 救 了 起 来 , 看 见 了 怀 中 的 血 书 , 才 知 道 他 的 来 历 。
取 个 乳 名 , 叫 做 江 流 , 托 人 抚 养 。
血 书 紧 紧 收 藏 。
光 阴 如 箭 , 日 月 如 梭 。
不 知 不 觉 江 水 流 了 一 十 八 岁 。
长 者 之 所 以 为 之 , 其 所 以 为 之 所 以 为 之 。
有 一 天 , 暮 春 天 气 , 众 人 一 同 在 松 树 阴 下 讲 经 参 禅 , 谈 论 玄 妙 , 那 酒 肉 和 尚 恰 被 玄 奘 所 欺 负 。
和 尚 大 怒 , 骂 道 : 你 这 个 业 畜 , 姓 名 也 不 知 道 , 父 母 也 不 认 识 , 还 在 这 里 捣 捣 什 么 鬼 ? 玄 奘 被 他 骂 出 这 样 的 话 语 , 进 入 寺 庙 跪 着 告 诉 师 父 , 眼 泪 双 流 说 : 人 生 于 天 地 之 间 , 禀 受 阴 阳 而 资 养 五 行 , 全 都 由 父 亲 生 育 母 亲 养 育 , 难 道 有 人 在 世 上 没 有 父 母 的 吗 ?
长 老 说 : 你 真 的 要 寻 找 父 母 , 可 随 我 到 方 丈 里 来 。
玄 奘 于 是 。
长 老 来 到 重 重 的 桥 上 , 取 出 一 个 小 匣 子 , 打 开 来 , 取 出 一 张 血 书 、 一 件 汗 衫 , 送 给 了 玄 奘 。
玄 奘 把 血 书 撕 开 阅 读 , 才 知 道 父 母 的 姓 名 , 还 有 冤 仇 的 事 迹 。
玄 奘 读 完 后 , 不 知 不 觉 哭 倒 在 地 上 说 : 父 母 的 仇 恨 , 不 能 报 复 , 怎 么 能 做 人 呢 十 八 年 来 , 不 认 识 自 己 的 父 母 , 到 现 在 才 知 道 有 母 亲 。
这 个 身 子 如 果 不 是 师 父 救 救 抚 养 的 , 哪 有 今 天 呢 容 弟 子 去 寻 见 母 亲 , 然 后 头 顶 香 盆 , 重 建 殿 宇 , 报 答 师 父 的 深 恩 。
师 父 说 : 你 要 去 寻 找 母 亲 , 可 带 着 这 个 血 书 和 汗 衫 前 去 。
只 做 化 缘 , 径 直 前 往 江 州 私 衙 , 才 能 得 到 你 的 母 亲 相 见 。
玄 奘 领 了 师 父 的 言 语 , 就 做 了 化 缘 的 和 尚 , 径 直 到 了 江 州 。
恰 逢 刘 洪 有 事 出 外 , 也 是 上 天 叫 他 母 子 相 会 , 玄 奘 就 径 直 到 自 己 的 衙 门 口 抄 化 。
那 个 殷 小 姐 原 来 夜 里 做 了 一 个 梦 , 梦 见 月 亮 又 圆 了 , 暗 想 道 : 我 婆 婆 不 知 道 消 息 , 我 的 丈 夫 被 这 个 贼 人 杀 害 , 我 的 儿 子 被 抛 到 江 中 , 如 果 有 人 收 养 , 我 算 来 有 十 八 岁 了 , 或 许 今 天 上 天 让 我 相 会 , 也 不 知 道 。
正 在 沉 默 不 语 , 忽 然 听 到 私 人 衙 门 前 有 人 念 经 , 连 喊 抄 化 , 小 姐 又 乘 机 出 来 问 道 : 你 是 从 哪 里 来 的 ? 玄 奘 回 答 说 : 贫 僧 是 金 山 寺 法 明 长 老 的 弟 弟 。
小 姐 说 : 你 既 然 是 金 山 寺 长 老 的 徒 弟 啊 叫 进 衙 来 , 拿 斋 饭 给 玄 奘 吃 。
仔 细 观 察 , 言 语 , 好 像 和 男 子 一 样 。
小 姐 将 从 婢 女 打 开 去 , 问 道 : 你 这 个 小 师 父 , 还 是 自 幼 出 家 的 还 是 中 年 出 家 的 , 还 是 中 年 出 家 的 , 还 是 姓 什 么 名 , 还 有 父 母 吗 ? 玄 奘 回 答 说 : 我 也 不 是 自 幼 出 家 , 我 也 不 是 中 年 出 家 , 我 说 起 来 , 冤 有 天 来 大 , 仇 有 海 样 深 : 我 父 亲 被 人 谋 杀 , 我 母 亲 却 被 贼 人 占 了 。
我 师 父 法 明 长 老 教 我 在 江 州 府 衙 里 寻 找 母 亲 。
小 姐 问 道 : 你 母 亲 姓 什 么 ? 玄 奘 说 : 我 母 亲 姓 殷 , 名 叫 温 娇 。
我 父 亲 姓 陈 , 名 光 蕊 。
吾 小 名 叫 江 流 , 他 的 法 名 取 名 叫 玄 奘 。
小 姐 说 : 温 柔 就 是 我 。
但 你 现 在 有 什 么 凭 据 , 玄 奘 听 说 是 他 的 母 亲 , 双 膝 跪 下 , 哀 哀 地 哭 着 说 : 我 娘 如 果 不 相 信 , 看 见 有 血 书 、 汗 衫 作 为 证 据 。
温 柔 取 过 一 看 , 果 然 是 真 的 , 母 子 相 抱 而 哭 。
就 大 叫 : 我 儿 快 去 。
玄 奘 说 : 我 十 八 年 不 认 识 自 己 的 父 母 , 今 天 才 见 到 母 亲 , 教 孩 子 怎 么 割 舍 呢 小 姐 说 : 我 的 儿 子 , 你 火 快 抽 身 前 去 。
刘 贼 如 果 回 去 , 他 一 定 会 危 害 你 的 性 命 。
我 第 二 天 假 装 有 病 , 只 说 先 前 曾 经 答 应 舍 弃 一 百 双 僧 鞋 , 来 到 你 寺 中 还 愿 。
那 时 节 , 我 有 话 与 你 说 。
玄 奘 依 照 他 的 话 拜 别 。
又 说 : 小 姐 自 从 见 到 儿 子 以 后 , 心 里 一 忧 一 喜 。
忽 然 有 一 天 他 推 病 , 茶 饭 不 吃 , 躺 在 床 上 。
刘 洪 回 到 衙 门 , 问 他 原 因 。
小 姐 说 : 我 小 时 候 曾 经 许 下 一 个 愿 愿 , 答 应 舍 弃 僧 人 的 鞋 子 一 百 双 。
昨 天 五 天 之 前 , 梦 见 一 个 和 尚 手 拿 利 刃 , 想 要 和 尚 的 鞋 子 , 就 觉 得 身 子 不 快 。
刘 洪 说 : 这 些 小 事 , 何 不 早 说 ? 随 即 上 堂 , 告 诉 王 左 衙 、 李 右 衙 说 : 江 州 城 内 的 百 姓 , 每 家 要 准 备 僧 鞋 一 双 , 限 五 天 内 完 成 。
百 姓 都 依 照 他 们 的 赋 税 完 毕 收 入 。
小 姐 对 刘 洪 说 : 僧 鞋 完 了 , 这 里 有 什 么 寺 院 , 好 去 还 愿 呢 刘 洪 说 : 这 江 州 有 个 金 山 寺 、 焦 山 寺 , 听 你 到 那 个 寺 里 去 。
小 姐 说 : 久 闻 金 山 寺 好 个 寺 院 , 我 就 去 金 山 寺 去 。
刘 洪 立 即 叫 来 王 、 李 两 个 衙 门 , 准 备 下 船 。
小 姐 带 着 心 腹 人 , 一 同 上 了 船 , 稍 稍 用 水 把 船 撑 开 , 就 投 奔 金 山 寺 去 了 。
又 说 : 玄 奘 回 到 寺 中 , 见 到 法 明 长 老 , 把 前 面 的 事 说 了 一 遍 。
长 老 非 常 高 兴 。
第 二 天 , 只 见 一 个 丫 鬟 先 到 , 说 夫 人 来 寺 里 还 愿 。
众 僧 都 出 寺 迎 接 。
小 姐 径 直 走 进 寺 门 , 参 拜 了 菩 萨 , 大 摆 斋 钵 。
又 叫 来 丫 鬟 把 僧 人 的 鞋 子 、 暑 天 的 鞋 子 放 在 盘 子 里 , 来 到 法 堂 , 小 姐 又 摘 着 心 香 礼 拜 , 就 教 法 明 长 老 分 别 和 众 僧 一 起 走 完 。
玄 奘 见 众 僧 人 散 去 了 , 法 堂 上 再 没 有 一 个 人 , 他 就 走 近 前 面 跪 下 。
小 姐 叫 他 脱 掉 鞋 子 和 鞋 子 , 看 时 , 他 的 左 脚 上 果 然 缺 了 一 个 小 指 头 。
当 时 两 个 人 又 抱 住 他 哭 , 拜 谢 长 老 抚 养 他 的 恩 德 。
法 明 说 : 你 现 在 母 子 相 会 , 恐 怕 奸 贼 知 道 了 , 可 以 迅 速 抽 身 回 去 , 希 望 免 除 灾 祸 。
小 姐 说 : 我 儿 子 , 我 给 你 一 只 香 环 , 你 径 直 到 洪 州 西 北 地 方 , 大 约 有 一 千 五 百 里 的 路 程 , 那 里 有 个 万 花 店 , 当 时 留 下 婆 婆 张 氏 在 那 里 , 是 你 父 亲 生 身 的 母 亲 。
我 再 写 一 封 信 给 你 , 径 直 到 唐 王 皇 城 内 , 金 殿 左 边 , 殷 开 山 丞 相 家 , 是 你 母 亲 生 身 的 父 母 。
你 将 我 的 书 信 递 给 外 公 , 叫 外 公 奏 上 唐 王 , 统 领 人 马 , 擒 杀 此 贼 , 与 父 亲 报 仇 , 那 时 才 救 得 老 娘 的 身 子 出 来 。
我 现 在 不 敢 长 久 停 留 , 实 在 是 恐 怕 贼 寇 汉 人 责 怪 我 回 来 太 晚 了 。
于 是 出 寺 , 登 船 而 去 。
玄 奘 哭 着 回 到 寺 中 , 告 诉 了 师 父 , 当 时 就 行 礼 告 别 , 径 直 前 往 洪 州 。
来 到 万 花 店 , 问 那 个 店 主 刘 小 二 说 : 昔 年 江 州 陈 客 官 有 一 母 亲 住 在 你 的 店 里 , 现 在 好 吗 ? 刘 小 二 说 : 他 原 来 在 我 的 店 中 。
后 来 昏 了 眼 睛 , 三 四 年 都 没 有 店 租 还 给 我 。
现 在 我 在 南 门 头 一 个 破 瓦 , 每 天 上 街 叫 化 度 日 。
那 个 客 官 一 去 许 久 , 到 现 在 却 没 有 信 息 , 不 知 道 是 什 么 原 因 。
玄 奘 听 完 , 立 刻 问 到 南 门 头 破 瓦 的 事 , 找 到 婆 婆 。
婆 婆 说 : 你 的 声 音 好 像 我 儿 子 陈 光 蕊 。
玄 奘 说 : 我 不 是 陈 光 蕊 , 我 是 陈 光 蕊 的 儿 子 。
温 柔 小 姐 , 是 我 的 娘 。
婆 婆 说 : 你 的 父 母 怎 么 不 来 ? 玄 奘 说 : 我 的 父 亲 被 强 盗 打 死 了 , 我 的 娘 子 被 强 盗 占 据 成 为 妻 子 。
婆 婆 说 : 你 怎 么 知 道 来 找 我 ? 玄 奘 说 : 是 我 娘 子 带 着 我 来 找 婆 婆 。
我 娘 有 书 在 这 里 , 又 有 一 只 香 环 。
那 婆 婆 接 了 信 和 香 环 , 放 声 痛 哭 道 : 我 儿 子 为 了 功 名 到 这 种 地 步 , 我 只 说 他 背 义 忘 恩 , 哪 知 道 他 被 人 打 算 死 。
而 且 喜 得 皇 天 怜 悯 , 不 绝 我 儿 子 之 后 , 今 天 还 有 孙 子 来 寻 找 我 。
玄 奘 问 : 婆 婆 的 眼 睛 , 为 什 么 都 昏 了 ? 婆 婆 说 : 我 因 为 思 量 你 的 父 亲 , 终 日 悬 望 , 不 见 他 来 , 因 此 上 哭 , 两 眼 都 昏 了 。
玄 奘 便 跪 下 向 天 祈 祷 说 : 现 在 玄 奘 十 八 岁 , 父 母 的 仇 恨 不 能 报 复 。
今 天 我 带 着 母 亲 的 命 令 来 寻 找 婆 婆 , 上 天 如 果 怜 悯 我 弟 子 的 诚 意 , 保 我 婆 婆 双 眼 重 新 明 亮 。
祝 祷 完 毕 , 就 拿 出 舌 头 给 婆 婆 擦 眼 。
不 一 会 儿 , 他 的 双 眼 被 咬 开 了 , 仍 然 像 当 初 一 样 。
婆 婆 看 了 小 和 尚 说 : 你 果 然 是 我 的 孙 子 , 恰 好 和 我 的 儿 子 光 蕊 没 有 两 样 。
婆 婆 又 喜 又 悲 。
玄 奘 领 婆 婆 出 门 , 回 到 刘 小 二 店 里 。
拿 了 些 房 钱 租 了 一 间 房 子 , 与 婆 婆 住 在 一 起 。
又 将 盘 缠 给 婆 婆 说 : 我 此 去 , 只 有 一 个 多 月 就 回 来 了 。
随 即 辞 别 了 婆 婆 , 径 直 前 往 京 城 。
不 久 到 了 皇 城 东 街 殷 丞 相 府 上 , 对 门 上 的 人 说 : 小 僧 是 亲 戚 , 来 探 望 相 公 。
门 上 人 告 诉 了 丞 相 , 丞 相 说 : 我 与 和 尚 并 没 有 亲 戚 关 系 。
夫 人 说 : 我 昨 夜 梦 见 我 的 女 儿 满 堂 娇 娘 来 到 家 中 , 莫 不 是 女 婿 有 信 回 来 了 , 丞 相 就 教 我 请 小 和 尚 来 到 厅 上 。
小 和 尚 见 了 丞 相 和 夫 人 , 在 地 上 哭 泣 , 从 怀 中 取 出 一 封 信 , 递 给 丞 相 。
丞 相 拆 开 , 从 头 读 完 , 放 声 痛 哭 。
夫 人 问 道 : 相 公 , 有 什 么 事 ? 丞 相 说 : 这 和 尚 是 我 和 你 的 外 孙 。
他 的 女 婿 陈 光 蕊 被 贼 人 谋 杀 , 满 堂 娇 被 贼 人 强 占 为 妻 。
夫 人 听 完 , 也 痛 哭 不 止 。
丞 相 说 : 夫 人 不 要 烦 恼 , 来 朝 上 奏 知 道 皇 上 , 亲 自 统 领 军 队 , 一 定 要 和 女 婿 报 仇 。
第 二 天 , 丞 相 入 朝 , 启 奏 唐 王 说 : 现 在 我 的 女 婿 状 元 陈 光 蕊 , 带 领 家 乡 小 江 州 赴 任 , 被 稍 水 刘 洪 打 死 , 占 据 女 儿 为 妻 , 假 冒 我 的 女 婿 , 做 官 多 年 。
事 情 发 生 异 常 变 化 , 请 陛 下 立 即 征 发 人 马 , 剿 灭 贼 寇 。
唐 王 看 到 奏 章 后 大 怒 , 就 派 出 御 林 军 六 万 人 , 让 殷 丞 相 带 领 军 队 前 去 。
丞 相 领 旨 出 朝 , 立 即 前 往 教 场 内 点 点 兵 器 , 径 直 前 往 江 州 进 发 。
早 晨 走 到 夜 晚 住 宿 , 星 落 鸟 飞 , 不 知 不 觉 已 到 了 江 州 , 殷 丞 相 的 兵 马 都 在 北 岸 攻 下 营 寨 。
夜 晚 命 令 金 牌 下 户 把 他 们 叫 到 江 州 同 知 、 州 判 二 人 , 丞 相 对 他 说 知 道 此 事 , 叫 他 们 带 兵 相 助 , 一 同 过 江 而 去 。
天 还 未 明 , 就 把 刘 洪 的 衙 门 包 围 了 。
刘 洪 正 在 梦 中 , 听 到 一 声 火 炮 , 金 鼓 齐 鸣 , 众 兵 杀 进 私 衙 , 刘 洪 措 手 不 及 , 早 被 擒 住 。
丞 相 传 下 军 令 , 将 刘 洪 一 个 人 犯 绑 到 法 场 , 命 令 众 军 都 在 城 外 安 营 去 了 。
丞 相 径 直 进 入 衙 内 正 厅 坐 下 , 请 小 姐 出 来 相 见 。
小 姐 想 要 出 去 , 羞 于 见 父 亲 , 就 要 自 刎 。
玄 奘 听 到 消 息 , 急 忙 带 着 母 亲 解 救 , 双 膝 跪 下 , 对 母 亲 说 : 儿 子 与 外 公 统 兵 来 到 这 里 , 与 父 亲 报 仇 。
现 在 敌 人 已 经 捉 住 了 , 母 亲 为 什 么 反 而 要 寻 找 死 罪 , 母 亲 如 果 死 了 , 孩 子 怎 么 能 活 下 来 呢 ? 丞 相 也 进 入 衙 门 劝 解 。
小 姐 说 : 我 听 说 妇 人 从 一 个 人 死 了 。
悲 痛 丈 夫 已 经 被 贼 人 杀 死 , 岂 可 以 脸 色 羞 愧 地 跟 随 贼 人 , 只 是 因 为 我 的 遗 腹 在 我 身 上 , 只 能 忍 受 耻 辱 苟 且 偷 生 。
现 在 我 的 儿 子 已 经 长 大 了 , 又 看 见 老 父 带 着 兵 器 报 仇 , 作 为 你 的 女 儿 , 有 什 么 脸 面 相 见 , 只 有 一 死 来 报 答 丈 夫 罢 了 。
丞 相 说 : 这 不 是 我 儿 子 因 为 盛 衰 而 改 变 节 操 , 都 是 因 为 出 于 不 得 已 , 怎 么 能 为 耻 辱 呢 父 子 相 抱 而 哭 , 玄 奘 也 哀 伤 不 止 。
丞 相 拭 泪 说 : 你 们 两 个 人 暂 且 不 要 烦 恼 , 我 现 在 已 经 捉 拿 仇 贼 , 暂 且 去 掉 你 们 的 头 发 。
于 是 就 起 身 来 到 法 场 。
恰 好 江 州 同 知 也 派 遣 哨 兵 捉 住 水 贼 李 彪 的 解 送 到 了 。
丞 相 大 喜 , 就 命 令 军 牢 押 过 刘 洪 、 李 彪 , 每 人 痛 打 一 百 大 棍 子 , 拿 出 供 状 , 把 先 前 不 合 谋 杀 陈 光 蕊 的 情 况 , 先 将 李 彪 钉 在 木 驴 上 , 把 他 推 到 市 场 , 杀 了 一 千 刀 , 把 他 们 的 头 示 众 。
先 前 打 死 陈 光 蕊 的 地 方 。
丞 相 和 小 姐 、 玄 奘 三 人 亲 自 到 江 边 , 望 着 空 中 祭 奠 , 活 捉 了 刘 洪 的 心 肝 , 祭 了 光 蕊 , 烧 了 一 道 祭 文 。
三 个 人 望 着 江 边 痛 哭 , 早 已 惊 动 了 水 府 , 有 巡 海 夜 叉 将 祭 文 呈 送 给 龙 王 。
龙 王 看 完 , 就 派 鳖 元 帅 去 , 请 光 蕊 来 , 说 : 先 生 , 恭 喜 , 恭 喜 。
现 在 有 先 生 、 夫 人 、 公 子 和 岳 父 都 在 江 边 祭 祀 你 。
我 现 在 送 你 还 魂 去 了 。
又 有 如 意 珠 一 枚 、 走 盘 珠 二 枚 、 绞 十 端 、 明 珠 玉 带 一 条 送 给 他 。
你 今 天 就 可 以 夫 妻 子 母 相 会 。
光 蕊 再 三 拜 谢 。
龙 王 就 命 令 夜 叉 将 光 蕊 的 尸 体 送 出 江 口 还 魂 。
夜 叉 领 着 命 令 离 去 。
又 说 殷 小 姐 哭 奠 丈 夫 一 番 , 又 想 要 自 己 投 水 而 死 。
正 在 仓 惶 之 时 , 忽 然 看 见 水 面 上 有 一 个 死 尸 浮 来 , 靠 近 江 岸 的 旁 边 。
小 姐 急 忙 向 前 认 看 , 认 得 是 男 人 的 尸 首 , 一 发 就 大 哭 不 已 。
众 人 都 来 观 看 , 只 见 光 蕊 伸 拳 伸 脚 , 身 子 渐 渐 伸 展 起 来 , 忽 然 爬 起 来 坐 下 来 。
众 人 不 胜 惊 骇 。
光 蕊 睁 开 眼 睛 , 早 就 看 见 殷 小 姐 和 老 人 殷 丞 相 一 同 穿 着 小 和 尚 , 都 在 身 边 啼 哭 。
光 蕊 说 : 你 们 为 什 么 在 这 里 呢 ? 小 姐 说 : 因 为 你 们 被 贼 人 打 死 , 后 来 我 生 下 这 个 孩 子 , 幸 好 遇 到 金 山 寺 长 老 抚 养 长 大 , 找 我 相 会 , 我 叫 他 去 找 外 公 。
父 亲 得 知 此 事 , 上 奏 朝 廷 , 统 兵 到 此 , 捉 住 贼 人 , 刚 才 活 捉 我 的 心 肝 , 希 望 空 自 祭 奠 我 丈 夫 。
不 知 我 丈 夫 怎 么 活 着 又 能 还 魂 呢 ? 光 蕊 说 : 都 是 因 为 我 和 你 昔 年 在 万 花 店 时 , 买 了 放 了 那 尾 金 色 的 鲤 鱼 , 谁 知 那 鲤 鱼 就 是 此 处 的 龙 王 。
后 来 贼 人 把 我 推 到 水 里 , 全 亏 了 他 救 我 。
方 才 又 赐 我 还 魂 , 送 我 的 宝 物 , 都 在 身 上 。
更 不 想 你 生 下 这 个 儿 子 , 又 得 到 岳 父 替 我 报 仇 。
真 是 苦 尽 甘 来 , 没 有 比 这 更 大 的 喜 事 。
众 官 员 听 到 消 息 , 都 来 祝 贺 高 兴 。
丞 相 就 令 安 排 酒 席 , 答 谢 所 属 官 员 。
当 天 军 马 回 去 。
来 到 万 花 店 , 那 丞 相 传 令 安 营 。
光 蕊 便 和 玄 奘 到 刘 家 店 去 找 婆 婆 。
那 婆 婆 当 天 晚 上 做 了 一 个 梦 , 梦 见 枯 木 开 花 , 屋 后 喜 鹊 纷 纷 喧 哗 , 想 道 : 莫 不 是 我 的 孙 子 来 了 , 还 没 说 完 , 只 见 店 门 外 , 光 蕊 父 子 一 齐 来 到 。
小 和 尚 指 着 她 说 : 这 不 是 俺 婆 婆 吗 ? 光 蕊 见 了 老 母 , 连 忙 拜 倒 。
母 子 抱 着 头 痛 哭 一 场 , 把 上 项 的 事 说 了 一 遍 。
回 来 后 还 了 小 二 店 的 钱 , 起 程 回 到 京 城 。
进 入 相 府 , 光 蕊 和 小 姐 与 婆 婆 、 玄 奘 都 来 见 夫 人 。
夫 人 非 常 高 兴 , 吩 咐 家 僮 , 大 摆 筵 席 庆 贺 。
丞 相 说 : 今 天 这 次 宴 会 , 可 取 名 为 团 圆 会 。
这 样 , 真 正 是 全 家 欢 乐 。
第 二 天 早 朝 , 唐 王 登 殿 。
丞 相 殷 光 蕊 出 朝 , 将 前 后 事 情 详 细 启 奏 , 并 且 推 荐 光 蕊 的 才 能 可 以 重 用 。
唐 王 准 上 奏 , 就 命 令 提 升 陈 萼 为 学 士 之 职 , 随 朝 处 理 政 务 。
玄 奘 立 意 安 禅 , 送 他 到 洪 福 寺 内 修 行 。
后 来 , 殷 小 姐 竟 然 从 容 自 尽 。
玄 奘 亲 自 到 金 山 寺 中 去 报 答 法 明 长 老 。
不 知 以 后 的 事 情 的 体 制 如 何 , 暂 且 听 下 回 分 析 。
}\switchcolumn\flushpage  \begin{pinyinscope}{\myfontt \section{第一○回}     老龍王拙計犯天條 魏丞相遺書託冥吏

且不題光蕊盡職,玄奘修行。卻說長安城外涇河岸邊,有兩個賢人:一個是漁翁
,名喚張稍﹔一個是樵子,名喚李定。他兩個是不登科的進士,能識字的山人。
一日,在長安城裏賣了肩上柴,貨了籃中鯉,同入酒館之中,吃了半酣,各攜一
瓶,順涇河岸邊,徐步而回。張稍道:「李兄,我想那爭名的,因名喪體﹔奪利
的,為利亡身﹔受爵的,抱虎而眠﹔承恩的,袖蛇而走。算起來,還不如我們水
秀山青,逍遙自在,甘淡薄,隨緣而過。」李定道:「張兄說得有理。但只是你
那水秀,不如我的山青。」張稍道:「你山青不如我的水秀。有一《蝶戀花》詞
為證。詞曰:
煙波萬里扁舟小,靜依孤篷,西施聲音遶。滌慮洗心名利少,閑攀蓼穗蒹葭草。
    數點沙鷗堪樂道,柳岸蘆灣,妻子同歡笑。一覺安眠風浪消,無榮無辱
        無煩惱。」
李定道:「你的水秀,不如我的山青。也有個《蝶戀花》詞為證。詞曰:
    雲林一段松花滿,默聽鶯啼,巧舌如調管。紅瘦綠肥春正暖,倏然夏至
        光陰轉。
    又值秋來容易換,黃花香,堪供玩。迅速嚴冬如指撚,逍遙四季無人管。」
漁翁道:「你山青不如我水秀,受用些好物。有一《鷓鴣天》為證:
    仙鄉雲水足生涯,擺櫓橫舟便是家。活剖鮮鱗烹綠鱉,旋蒸紫蟹煮紅蝦。
    青蘆筍,水荇芽,菱角雞頭更可誇。嬌藕老蓮芹葉嫩,慈菇茭白鳥英花。」
樵夫道:「你水秀不如我山青,受用些好物。亦有一《鷓鴣天》為證:
    崔巍峻嶺接天涯,草舍茅庵是我家。醃臘雞鵝強蟹鱉,獐兔鹿勝魚蝦。
    香椿葉,黃楝芽,竹筍山茶更可誇。紫李紅桃梅杏熟,甜梨酸棗木樨花。」
漁翁道:「你山青真個不如我的水秀。又有《天仙子》一首:
    一葉小舟隨所寓,萬疊煙波無恐懼。垂鉤撒網捉鮮鱗,沒醬膩,偏有味
       ,老妻稚子團圓會。
    魚多又貨長安市,換得香醪吃個醉。簑衣當被臥秋江,鼾鼾睡,無憂慮
       ,不戀人間榮與貴。」
樵子道:「你水秀還不如我的山青。也有《天仙子》一首:
    茆舍數椽山下蓋,松竹梅蘭真可愛。穿林越嶺覓乾柴,沒人怪,從我賣
       ,或少或多憑世界。
    將錢沽酒隨心快,瓦缽磁甌殊自在。酕醄醉了臥松陰,無掛礙,無利害
       ,不管人間興與敗。」
漁翁道:「李兄,你山中不如我水上生意快活。有一《西江月》為證:
    紅蓼花繁映月,黃蘆葉亂搖風。碧天清遠楚江空,牽攪一潭星動。
    入網大魚作隊,吞鉤小鱖成叢。得來烹煮味偏濃,笑傲江湖打鬨。」
樵夫道:「張兄,你水上還不如我山中的生意快活。亦有《西江月》為證:
    敗葉枯藤滿路,破梢老竹盈山。女蘿乾葛亂牽攀,折取收繩殺擔。
    蟲蛀空心榆柳,風吹斷頭松柟。採來堆積備冬寒,換酒換錢從俺。」
漁翁道:「你山中雖可比過,還不如我水秀的幽雅。有一《臨江仙》為證:
    潮落旋移孤艇去,夜深罷棹歌來。簑衣殘月甚幽哉,宿鷗驚不起,天際
彩雲開。
    困臥蘆洲無個事,三竿日上還捱。隨心儘意自安排,朝臣寒待漏,怎似
        我寬懷。」
樵夫道:「你水秀的幽雅,還不如我山青更幽雅。亦有《臨江仙》可證:
    蒼徑秋高拽斧去,晚涼抬擔回來。野花插鬢更奇哉,撥雲尋路出,待月
        叫門開。
    稚子山妻欣笑接,草床木枕攲捱。蒸梨炊黍旋鋪排,甕中新釀熟,真個
        壯幽懷。」
漁翁道:「這都是我兩個生意,贍身的勾當,你卻沒有我閑時節的好處。有詩為
        證。詩曰:
    閑看蒼天白鶴飛,停舟溪畔掩蒼扉。
    倚篷教子搓?線,罷棹同妻晒網圍。
    性定果然如浪靜,身安自是覺風微。
    綠簑青笠隨時著,勝掛朝中紫綬衣。」
樵夫道:「你那閑時又不如我的閑時好也。亦有詩為證。詩曰:
    閑觀縹緲白雲飛,獨坐茅庵掩竹扉。
    無事訓兒開卷讀,有時對客把棋圍。
    喜來策杖歌芳徑,興到攜琴上翠微。
    草履麻絛粗布被,心寬強似著羅衣。」

張稍道:「李定,我兩個真是微吟可相狎,不須檀板共金樽。但散道詞章,不為
        稀罕。且各聯幾句,看我們漁樵攀話何如?」李定道:「張兄言之最妙
       。請兄先吟。」
    「舟停綠水煙波內,家住深山曠野中。
    偏愛溪橋春水漲,最憐岩岫曉雲蒙。
    龍門鮮鯉時烹煮,蟲蛀乾柴日燎烘。
    釣網多般堪贍老,擔繩二事可容終。
    小舟仰臥觀飛雁,草徑斜欹聽唳鴻。
    口舌場中無我分,是非海內少吾蹤。
    溪邊掛晒繒如錦,石上重磨斧似鋒。
    秋月暉暉常獨釣,春山寂寂沒人逢。
    魚多換酒同妻飲,柴剩沽壺共子叢。
    自唱自斟隨放蕩,長歌長嘆任顛風。
    呼兄喚弟邀船夥,挈友攜朋聚野翁。
    行令猜拳頻遞盞,拆牌道字漫傳鐘。
    烹蝦煮蟹朝朝樂,炒鴨爊雞日日豐。
    愚婦煎茶情散淡,山妻造飯意從容。
    曉來舉杖淘輕浪,日出擔柴過大衝。
    雨後披簑擒活鯉,風前弄斧伐枯松。
    潛蹤避世妝痴蠢,隱姓埋名作啞聾。」

張稍道:「李兄,我才僭先起句,今到我兄,也先起一聯,小弟亦當續之。」
    「風月佯狂山野漢,江湖寄傲老餘丁。
    清閑有分隨瀟灑,口舌無聞喜太平。
    月夜身眠茅屋穩,天昏體蓋箬簑輕。
    忘情結識松梅友,樂意相交鷗鷺盟。
    名利心頭無算計,干戈耳畔不聞聲。
    隨時一酌香醪酒,度日三餐野菜羹。
    兩束柴薪為活計,一竿鉤線是營生。
    閑呼稚子磨鋼斧,靜喚憨兒補舊繒。
    春到愛觀楊柳綠,時融喜看荻蘆青。
    夏天避暑修新竹,六月乘涼摘嫩菱。
    霜降雞肥常日宰,重陽蟹壯及時烹。
    冬來日上還沉睡,數九天高自不寒。
    八節山中隨放性,四時湖裏任陶情。
    採薪自有仙家興,垂釣全無世俗形。
    門外野花香豔豔,船頭綠水浪平平。
    身安不說三公位,性定強如十里城。
    十里城高防閫令,三公位顯聽宣聲。
    樂山樂水真是罕,謝天謝地謝神明。」

他二人既各道詞章,又相聯詩句。行到那分路去處,躬身作別。張稍道:「李兄
呵,途中保重,上山仔細看虎。假若有些凶險,正是『明日街頭少故人』。」李
定聞言,大怒道:「你這廝憊懶!好朋友也替得生死,你怎麼咒我?我若遇虎遭
害,你必遇浪翻江。」張稍道:「我永世也不得翻江。」李定道:「『天有不測
風雲,人有暫時禍福。』你怎麼就保得無事?」張稍道:「李兄,你雖這等說,
你還沒捉摸﹔不若我的生意有捉摸,定不遭此等事。」李定道:「你那水面上營
生,極凶極險,隱隱暗暗,有甚麼捉摸?」張稍道:「你是不曉得。這長安城裏
,西門街上,有一個賣卦的先生。我每日送他一尾金色鯉,他就與我袖傳一課,
依方位,百下百著。今日我又去買卦,他教我在涇河灣頭東邊下網,西岸拋鉤,
定獲滿載魚蝦而歸。明日上城來,賣錢沽酒,再與老兄相敘。」二人從此敘別。

這正是:「路上說話,草裏有人。」原來這涇河水府有一個巡水的夜叉,聽見了
百下百著之言,急轉水晶宮,慌忙報與龍王道:「禍事了!禍事了!」龍王問:
「有甚禍事?」夜叉道:「臣巡水去到河邊,只聽得兩個漁、樵攀話,相別時,
言語甚是利害。那漁翁說:長安城裏,西門街上,有個賣卦先生,算得最准。他
每日送他鯉魚一尾,他就袖傳一課,教他百下百著。若依此等算准,卻不將水族
盡情打了?何以壯觀水府,何以躍浪翻波,輔助大王威力?」龍王甚怒,急提了
劍,就要上長安城,誅滅這賣卦的。旁邊閃過龍子、龍孫、蝦臣、蟹士、鰣軍師
、鱖少卿、鯉太宰,一齊啟奏道:「大王且息怒。常言道:『過耳之言,不可聽
信。』大王此去,必有雲從,必有雨助,恐驚了長安黎庶,上天見責。大王隱顯
莫測,變化無方,但只變一秀士,到長安城內訪問一番。果有此輩,容加誅滅不
遲﹔若無此輩,可不是妄害他人也?」

龍王依奏,遂棄寶劍,也不興雲雨,出岸上,搖身一變,變作一個白衣秀士,真
個:
丰姿英偉,聳壑昂霄。步履端祥,循規蹈矩。語言遵孔孟,禮貌體周文。身穿玉
色羅襴服,頭戴逍遙一字巾。

上路來,拽開雲步,徑到長安城西門大街上。只見一簇人,擠擠雜雜,鬧鬧哄哄
。內有高談闊論的道:「屬龍的本命,屬虎的相沖。寅辰巳亥,雖稱合局,但怕
的是日犯歲君。」龍王聞言,情知是賣卜之處。走上前,分開眾人,望裏觀看。
只見:
四壁珠璣,滿堂綺繡。寶鴨香無斷,磁瓶水恁清。兩邊羅列王維畫,座上高懸鬼
谷形。端溪硯,金煙墨,相襯著霜毫大筆﹔火珠林,郭璞數,謹對了臺政新經。
六爻熟諳,八卦精通。能知天地理,善曉鬼神情。一槃子午安排定,滿腹星辰佈
列清。真個那未來事,過去事,觀如月鏡﹔幾家興,幾家敗,鑑若神明。知凶定
吉,斷死言生。開談風雨迅,下筆鬼神驚。招牌有字書名姓,神課先生袁守誠。

此人是誰?原來是當朝欽天監臺正先生袁天罡的叔父,袁守誠是也。那先生果然
相貌稀奇,儀容秀麗﹔名揚大國,術冠長安。龍王入門來,與先生相見。禮畢,
請龍上坐,童子獻茶。先生問曰:「公來問何事?」龍王曰:「請卜天上陰晴事
如何。」先生即袖傳一課,斷曰:「雲迷山頂,霧罩林梢。若占雨澤,准在明朝
。」龍王曰:「明日甚時下雨?雨有多少尺寸?」先生道:「明日辰時布雲,巳
時發雷,午時下雨,未時雨足,共得水三尺三寸零四十八點。」龍王笑曰:「此
言不可作戲。如是明日有雨,依你斷的時辰、數目,我送課金五十兩奉謝﹔若無
雨,或不按時辰、數目,我與你實說:定要打壞你的門面,扯碎你的招牌,即時
趕出長安,不許在此惑眾。」先生忻然而答:「這個一定任你。請了,請了。明
朝雨後來會。」

龍王辭別,出長安,回水府。大小水神接著,問曰:「大王訪那賣卦的如何?」
龍王道:「有,有,有。但是一個掉嘴口討春的先生。我問他幾時下雨,他就說
明日下雨。問他甚麼時辰,甚麼雨數,他就說辰時布雲,巳時發雷,午時下雨,
未時雨足,得水三尺三寸零四十八點。我與他打了個賭賽:若果如他言,送他謝
金五十兩﹔如略差些,就打破他門面,趕他起身,不許在長安惑眾。」眾水族笑
曰:「大王是八河都總管,司雨大龍神,有雨無雨,惟大王知之。他怎敢這等胡
言?那賣卦的定是輸了,定是輸了。」

此時龍子、龍孫與那魚卿、蟹士正歡笑談此事未畢,只聽得半空中叫:「涇河龍
王接旨。」眾抬頭上看,是一個金衣力士,手擎玉帝敕旨,徑投水府而來。慌得
龍王整衣端肅,焚香接了旨。金衣力士回空而去。龍王謝恩,拆封看時,上寫著:
    敕命八河總,驅雷掣電行:
    明朝施雨澤,普濟長安城。

旨意上時辰、數目,與那先生判斷者毫髮不差。諕得那龍王魂飛魄散。少頃甦醒
,對眾水族曰:「塵世上有此靈人,真個是能通天地理,卻不輸與他呵!」鰣軍
師奏云:「大王放心。要贏他有何難處?臣有小計,管教滅那廝的口嘴。」龍王
問計,軍師道:「行雨差了時辰,少些點數,就是那廝斷卦不准,怕不贏他?那
時捽碎招牌,趕他跑路,果何難也?」龍王依他所奏,果不擔憂。

至次日,點札風伯、雷公、雲童、電母,直至長安城九霄空上。他挨到那巳時方
布雲,午時發雷,未時落雨,申時雨止,卻只得三尺零四十點。改了他一個時辰
,剋了他三寸八點。雨後發放眾將班師。他又按落雲頭,還變作白衣秀士,到那
西門裏大街上,撞入袁守誠卦舖,不容分說,就把他招牌、筆、硯等一齊捽碎。
那先生坐在椅上,公然不動。這龍王又掄起門板便打,罵道:「這妄言禍福的妖
人,擅惑眾心的潑漢!你卦又不靈,言又狂謬。說今日下雨的時辰、點數俱不相
對。你還危然高坐,趁早去,饒你死罪!」守誠猶公然不懼分毫,仰面朝天冷笑
道:「我不怕,我不怕。我無死罪,只怕你倒有個死罪哩。別人好瞞,只是難瞞
我也。我認得你,你不是秀士,乃是涇河龍王。你違了玉帝敕旨,改了時辰,剋
了點數,犯了天條。你在那剮龍臺上,恐難免一刀,你還在此罵我?」龍王見說
,心驚膽戰,毛骨悚然。急丟了門板,整衣伏禮,向先生跪下道:「先生休怪。
前言戲之耳,豈知弄假成真,果然違犯天條,奈何?望先生救我一救﹔不然,我
死也不放你。」守誠曰:「我救你不得,只是指條生路與你投生便了。」龍曰:
「願求指教。」先生曰:「你明日午時三刻,該赴人曹官魏徵處聽斬。你果要性
命,須當急急去告當今唐太宗皇帝方好。那魏徵是唐王駕下的丞相,若是討他個
人情,方保無事。」

龍王聞言,拜辭含淚而去。不覺紅日西沉,太陰星上。但見:
煙凝山紫歸鴉倦,遠路行人投旅店。渡頭新雁宿汀沙,銀河現,催更籌,孤村燈
火光無焰。風裊爐煙清道院,蝴蝶夢中人不見。月移花影上欄杆,星光亂,漏聲
換,不覺深沉夜已半。

這涇河龍王也不回水府,只在空中。等到子時前後,收了雲頭,斂了霧角,徑來
皇宮門首。此時唐王正夢出宮門之外,步月花陰。忽然龍王變作人相,上前跪拜
,口叫:「陛下,救我,救我。」太宗云:「你是何人?朕當救你。」龍王云:
「陛下是真龍,臣是業龍。臣因犯了天條,該陛下賢臣人曹官魏徵處斬,故來拜
求,望陛下救我一救。」太宗曰:「既是魏徵處斬,朕可以救你,你放心前去。」
龍王歡喜,叩謝而去。

卻說那太宗夢醒後,念念在心。早已至五鼓三點,太宗設朝,聚集兩班文武官員
。但見那:
煙籠鳳闕,香藹龍樓。光搖丹扆動,雲拂翠華流。君臣相契同堯舜,禮樂威嚴近
漢周。侍臣燈,宮女扇,雙雙映彩﹔孔雀屏,麒麟殿,處處光浮。山呼萬歲,華
祝千秋。靜鞭三下響,衣冠拜冕旒。宮花燦爛天香襲,堤柳輕柔御樂謳。珍珠簾
,翡翠簾,金鉤高控﹔龍鳳扇,山河扇,寶輦停留。文官英秀,武將抖搜。御道
分高下,丹墀列品流。金章紫綬乘三象,地久天長萬萬秋。

眾官朝賀已畢,各各分班。唐王閃鳳目龍睛,一一從頭觀看,只見那文官內是房
玄齡、杜如晦、徐世勣、許敬宗、王珪等,武官內是馬三寶、段志玄、殷開山、
程咬金、劉洪紀、胡敬德、秦叔寶等,一個個威儀端肅,卻不見魏徵丞相。唐王
召徐世勣上殿道:「朕夜間得一怪夢:夢見一人,迎面拜謁,口稱是涇河龍王,
犯了天條,該人曹官魏徵處斬,拜告寡人救他,朕已許諾。今日班前獨不見魏徵
,何也?」世勣對曰:「此夢告准。須喚魏徵來朝,陛下不要放他出門,過此一
日,可救夢中之龍。」唐王大喜,即傳旨,著當駕官宣魏徵入朝。

卻說魏徵丞相在府,夜觀乾象,正爇寶香,只聞得九霄鶴唳,卻是天差仙使,捧
玉帝金旨一道,著他午時三刻,夢斬涇河老龍。這丞相謝了天恩,齋戒沐浴,在
府中試慧劍,運元神,故此不曾入朝。一見當駕官齎?來宣,惶懼無任﹔又不敢
違遲君命,只得急急整衣束帶,同旨入朝,在御前叩頭請罪。唐王道:「赦卿無
罪。」那時諸臣尚未退朝,至此,卻命捲簾散朝。獨留魏徵,宣上金鑾,召入便
殿,先議論安邦之策,定國之謀。將近巳末午初時候,卻命宮人:「取過大棋來
,朕與賢卿對弈一局。」眾嬪妃隨取棋枰,鋪設御案。魏徵謝了恩,即與唐王對
弈,一遞一著,擺開陣勢。正合《爛柯經》云:
博弈之道,貴乎嚴謹。高者在腹,下者在邊,中者在角,此棋家之常法。法曰:
「寧輸一子,不失一先。」擊左則視右,攻後則瞻前。有先而後,有後而先。兩
生勿斷,皆活勿連。闊不可太疏,密不可太促。與其戀子以求生,不若棄之而取
勝﹔與其無事而獨行,不若固之而自補。彼眾我寡,先謀其生﹔我眾彼寡,務張
其勢。善勝者不爭,善陣者不戰﹔善戰者不敗,善敗者不亂。夫棋始以正合,終
以奇勝。凡敵無事而自補者,有侵絕之意﹔棄小而不救者,有圖大之心﹔隨手而
下者,無謀之人﹔不思而應者,取敗之道。《詩》云:「惴惴小心,如臨于谷。」
此之謂也。
  詩曰:
    棋盤為地子為天,色按陰陽造化全。
    下到玄微通變處,笑誇當日爛柯仙。

君臣兩個對弈,此棋正下到午時三刻,一盤殘局未終,魏徵忽然俯伏在案邊,鼾
鼾盹睡。太宗笑曰:「賢卿真是匡扶社稷之心勞,創立江山之力倦,所以不覺盹
睡。」太宗任他睡著,更不呼喚。不多時,魏徵醒來,俯伏在地道:「臣該萬死
,臣該萬死!卻才倦困,不知所為,望陛下赦臣慢君之罪。」太宗道:「卿有何
慢罪?且起來,拂退殘棋,與卿從新更著。」

魏徵謝了恩,卻才撚子在手,忽聽得朝門外大呼小叫。原來是秦叔寶、徐茂公等
,將著一個血淋的龍頭,擲在帝前,啟奏道:「陛下,海淺河枯曾有見,這般異
事卻無聞。」太宗與魏徵起身道:「此物何來?」叔寶、茂公道:「千步廊南,
十字街上,雲端裏落下這顆龍頭,微臣不敢不奏。」唐王驚問魏徵:「此是何說
?」魏徵轉身叩頭道:「是臣才一夢斬的。」唐王聞言,大驚道:「賢卿盹睡之
時,又不曾見動身動手,又無刀劍,如何卻斬此龍?」魏徵奏道:「主公,臣的
身在君前,夢離陛下。身在君前對殘局,合眼朦朧﹔夢離陛下乘瑞雲,出神抖搜
。那條龍在剮龍臺上,被天兵將綁縛其中。是臣道:『你犯天條,合當死罪。我
奉天命,斬汝殘生。』龍聞哀苦,臣抖精神。龍聞哀苦,伏爪收鱗甘受死﹔臣抖
精神,撩衣進步舉霜鋒。扢扠一聲刀過處,龍頭因此落虛空。」

太宗聞言,心中悲喜不一。喜者,誇獎魏徵好臣,朝中有此豪傑,愁甚江山不穩
?悲者,謂夢中曾許救龍,不期竟致遭誅。只得強打精神,傳旨著叔寶將龍頭懸
掛市曹,曉諭長安黎庶。一壁廂賞了魏徵,眾官散訖。

當晚回宮,心中只是憂悶。想那夢中之龍,哭啼啼哀告求生,豈知無常,難免此
患。思念多時,漸覺神魂倦怠,身體不安。當夜二更時分,只聽得宮門外有號泣
之聲,太宗愈加驚恐。正朦朧睡間,又見那涇河龍王手提著一顆血淋淋的首級,
高叫:「唐太宗,還我命來!還我命來!你昨夜滿口許諾救我,怎麼天明時反宣
人曹官來斬我?你出來,你出來,我與你到閻君處折辨折辨。」他扯住太宗,再
三嚷鬧不放。太宗箝口難言,只掙得汗流遍體。

正在那難分難解之時,只見正南上香雲繚繞,彩霧飄飄,有一個女真人上前,將
楊柳枝用手一擺,那沒頭的龍悲悲啼啼,徑往西北而去。原來這是觀音菩薩領佛
旨,上東土尋取經人,住此長安城都土地廟裏,夜聞鬼泣神號,特來喝退業龍,
救脫皇帝。那龍徑到陰司地獄具告不題。

卻說太宗甦醒回來,只叫:「有鬼!有鬼!」慌得那三宮皇后、六院嬪妃,與近
侍太監,戰兢兢,一夜無眠。

不覺五更三點,那滿朝文武多官,都在朝門外候朝。等到天明,猶不見臨朝,諕
得一個個驚懼躊躇。及日上三竿,方有旨意出來道:「朕心不快,眾官免朝。」
不覺倏五七日,眾官憂惶,都正要撞門見駕問安,只見太后有旨,召醫官入宮用
藥。眾人在朝門外等候討信。少時,醫官出來,眾問何疾。醫官道:「皇上脈氣
不正,虛而又數,狂言見鬼。又診得十動一代,五臟無氣,恐不諱只在七日之內
矣。」眾官聞言,大驚失色。

正愴惶間,又聽得太后有旨宣徐茂公、護國公、尉遲恭見駕。三公奉旨,急入到
分宮樓下。拜畢,太宗正色強言道:「賢卿,寡人十九歲領兵,南征北伐,東擋
西除,苦歷數載,更不曾見半點邪祟,今日卻反見鬼。」尉遲恭道:「創立江山
,殺人無數,何怕鬼乎?」太宗道:「卿是不信。朕這寢宮門外,入夜就拋磚弄
瓦,鬼魅呼號,著然難處。白日猶可,昏夜難禁。」叔寶道:「陛下寬心,今晚
臣與敬德把守宮門,看有甚麼鬼祟。」太宗准奏。茂公謝恩而出。

當日天晚,各取披掛,他兩個介冑整齊,執金瓜、鉞斧,在宮門外把守。好將軍
!你看他怎生打扮:
頭戴金盔光爍爍,身披鎧甲龍鱗。護心寶鏡幌祥雲,獅蠻收緊扣,繡帶彩霞新。
這一個鳳眼朝天星斗怕,那一個環睛映電月光浮。他本是英雄豪傑舊勳臣,只落
得千年稱戶尉,萬古作門神。

二將軍侍立門傍,一夜天曉,更不曾見一點邪祟。是夜,太宗在宮,安寢無事。
曉來宣二將軍,重重賞勞道:「朕自得疾,數日不能得睡,今夜仗二將軍威勢甚
安。卿且請出安息安息,待晚間再一護衛。」二將謝恩而出。

遂此二三夜把守俱安。只是御膳減損,病轉覺重。太宗又不忍二將辛苦,又宣叔
寶、敬德與杜、房諸公入宮,吩咐道:「這兩日朕雖得安,卻只難為秦、胡二將
軍徹夜辛苦。朕欲召巧手丹青,傳二將軍真容,貼於門上,免得勞他。如何?」
眾臣即依旨,選兩個會寫真的,著胡、秦二公依前披掛,照樣畫了,貼在門上。
夜間也即無事。

如此二三日,又聽得後宰門乒乓乒乓,磚瓦亂響。曉來即宣眾臣曰:「連日前門
幸喜無事,今夜後門又響,卻不又驚殺寡人也。」茂公進前奏道:「前門不安,
是敬德、叔寶護衛﹔後門不安,該著魏徵護衛。」太宗准奏,又宣魏徵今夜把守
後門。徵領旨,當夜結束整齊,提著那誅龍的寶劍,侍立在後宰門前,真個的好
英雄也。他怎生打扮:
熟絹青巾抹額,錦袍玉帶垂腰。兜風氅袖采霜飄,壓賽壘荼神貌。腳踏烏靴坐折
,手持利刃兇驍。圓睜兩眼四邊瞧,那個邪神敢到?

一夜通明,也無鬼魅。雖是前後門無事,只是身體漸重。

一日,太后又傳旨,召眾臣商議殯殮後事。太宗又宣徐茂公,吩咐國家大事,叮
囑倣劉蜀主託孤之意。言畢,沐浴更衣,待時而已。傍閃魏徵,手扯龍衣,奏道
:「陛下寬心,臣有一事,管保陛下長生。」太宗道:「病勢已入膏肓,命將危
矣,如何保得?」徵云:「臣有書一封,進與陛下,捎去到陰司,付酆都判官崔
?。」太宗道:「崔?是誰?」徵云:「崔?乃是太上先皇帝駕前之臣,先受茲
洲令,後陞禮部侍郎。在日與臣八拜為交,相知甚厚。他如今已死,現在陰司做
掌生死文簿的酆都判官,夢中常與臣相會。此去若將此書付與他,他念微臣薄分
,必然放陛下回來。管教魂魄還陽世,定取龍顏轉帝都。」太宗聞言,接在手中
,籠入袖裏,遂瞑目而亡。那三宮六院、皇后嬪妃、侍長儲君及兩班文武,俱舉
哀戴孝。又在白虎殿上,停著梓宮不題。

畢竟不知太宗如何還魂,且聽下回分解。





}  \end{pinyinscope}\switchcolumn{\myfontc \section{第 一 回} 老 龙 王 拙 计 犯 了 天 条 , 魏 丞 相 送 信 给 冥 吏 , 并 且 不 题 光 蕊 尽 职 , 玄 奘 修 行 。
又 说 长 安 城 外 泾 河 岸 边 有 两 个 贤 人 : 一 个 是 渔 翁 , 名 叫 张 稍 ; 一 个 是 樵 夫 , 名 叫 李 定 。
其 他 两 个 , 是 不 登 科 的 进 士 , 是 能 识 字 的 山 人 。
有 一 天 , 在 长 安 城 里 卖 了 肩 上 的 柴 , 买 了 一 只 筐 中 的 鲤 鱼 , 一 起 进 入 酒 馆 中 , 吃 了 半 醉 , 各 带 一 瓶 , 顺 着 泾 河 岸 边 , 慢 慢 地 走 回 来 。
张 某 慢 慢 地 说 : 李 哥 , 我 想 那 争 名 的 人 , 因 为 名 声 而 丧 身 ; 夺 利 的 人 , 为 了 利 而 亡 身 ; 接 受 爵 位 的 人 , 抱 着 老 虎 睡 觉 , 承 蒙 恩 宠 的 人 , 袖 着 蛇 逃 走 。
估 计 起 来 , 还 不 如 我 们 水 秀 山 青 , 逍 遥 自 在 , 甘 甜 淡 薄 , 随 缘 而 过 。
李 定 说 : 张 兄 说 得 有 理 。
只 是 你 的 水 秀 , 不 如 我 的 山 青 。
张 某 稍 稍 说 道 : 你 的 山 青 不 如 我 的 水 秀 。
有 一 首 《 蝶 恋 花 词 》 作 为 证 据 。
歌 词 是 : 烟 波 万 里 扁 舟 小 , 静 依 孤 篷 , 西 施 声 音 绕 。
洗 涤 思 虑 洗 心 洗 心 , 名 利 很 少 , 闲 适 地 攀 援 蓼 穗 芦 芦 草 。
数 点 沙 鸥 可 以 在 道 上 游 玩 , 柳 岸 芦 洲 , 妻 子 儿 女 同 欢 笑 。
一 旦 醒 来 就 睡 觉 了 , 风 浪 消 除 了 , 没 有 荣 耀 , 没 有 辱 辱 , 没 有 烦 恼 。
李 定 说 : 你 的 水 秀 , 不 如 我 的 山 青 。
也 有 一 个 《 蝶 恋 花 词 》 作 为 证 据 。
歌 词 是 : 云 林 一 段 松 花 满 , 默 听 莺 啼 , 巧 舌 如 调 管 。
红 瘦 绿 肥 春 正 暖 , 忽 然 夏 至 , 光 阴 转 转 。
又 遇 到 秋 来 容 易 换 , 黄 花 香 , 可 以 供 玩 。
迅 速 严 冬 如 手 指 抽 , 逍 遥 四 季 无 人 管 。
渔 翁 说 : 你 的 山 青 不 如 我 的 水 秀 , 接 受 些 好 东 西 。
有 一 篇 《 天 》 作 为 证 据 : 仙 乡 云 水 足 生 涯 , 摆 横 舟 便 是 家 。
活 着 剖 开 鲜 鳞 烹 煮 绿 蟹 , 旋 即 蒸 紫 蟹 煮 红 虾 。
青 芦 笋 , 水 芽 , 菱 角 鸡 头 更 值 得 夸 耀 。
娇 嫩 的 藕 子 、 老 色 的 莲 子 、 芹 菜 的 叶 子 和 叶 子 都 很 嫩 , 慈 香 的 豆 子 、 茭 白 的 鸟 英 花 。
樵 夫 说 : 你 的 水 秀 不 如 我 的 山 青 , 我 受 用 些 好 东 西 。
也 有 一 篇 《 天 》 作 为 证 据 : 崔 巍 峻 岭 接 天 涯 , 草 舍 茅 庵 是 我 家 。
煮 腊 鸡 、 鹅 、 蟹 、 鳖 , 獐 、 兔 、 鹿 胜 过 鱼 、 虾 。
香 椿 叶 , 黄 椒 芽 , 竹 笋 山 茶 更 值 得 夸 耀 。
紫 李 、 红 桃 、 梅 、 杏 都 熟 了 , 甜 梨 、 酸 枣 、 木 开 花 。
渔 翁 说 : 你 的 山 青 真 是 不 如 我 的 水 秀 。
又 有 《 天 仙 子 》 一 首 《 天 仙 子 》 : 一 叶 小 舟 随 所 寓 所 , 万 叠 烟 波 无 恐 惧 。
垂 钩 撒 网 捉 鲜 鱼 , 没 有 盐 醋 , 偏 有 滋 味 , 老 妻 稚 子 团 团 聚 会 。
鱼 多 又 买 到 长 安 市 上 , 换 得 香 酒 , 吃 了 喝 醉 。
穿 着 粗 布 衣 服 , 躺 在 秋 江 上 , 一 边 呼 叫 一 边 睡 觉 , 没 有 什 么 忧 虑 , 不 恋 恋 人 间 的 荣 华 和 富 贵 。
樵 夫 说 : 你 的 水 秀 还 不 如 我 的 山 青 。
还 有 《 天 仙 子 》 一 首 《 天 仙 子 》 : 房 屋 数 椽 山 下 盖 , 松 竹 、 梅 花 、 兰 花 真 可 爱 。
穿 林 越 岭 寻 找 干 柴 , 没 人 怪 , 跟 着 我 卖 , 或 少 或 多 , 凭 借 世 界 。
拿 钱 买 酒 随 心 快 乐 , 瓦 钵 磁 瓯 特 自 在 。
醉 了 就 躺 在 松 树 阴 里 , 没 有 什 么 阻 碍 , 没 有 什 么 利 与 害 , 不 管 人 间 的 兴 与 败 。
渔 翁 说 : 李 哥 , 你 在 山 中 不 如 我 在 水 上 生 意 快 活 。
有 一 篇 《 西 江 月 》 作 为 证 据 : 红 蓼 花 繁 映 月 , 黄 芦 叶 乱 摇 风 。
碧 天 清 远 楚 江 空 , 牵 动 一 潭 星 流 动 。
进 入 网 中 , 大 鱼 成 队 , 吞 钩 小 鱼 成 丛 。
得 来 煮 煮 味 道 偏 浓 , 笑 傲 江 湖 打 。
樵 夫 说 : 张 兄 , 你 在 水 上 还 不 如 我 在 山 中 的 生 意 快 活 。
也 有 《 西 江 月 》 作 为 证 据 : 破 叶 枯 藤 满 路 , 破 梢 老 竹 满 山 。
女 萝 干 葛 乱 牵 攀 , 折 下 来 收 起 的 绳 子 杀 掉 了 担 子 。
虫 咬 空 心 榆 柳 , 风 吹 断 头 松 树 。
采 来 堆 积 , 备 备 冬 天 寒 冷 , 换 酒 换 钱 从 俺 。
渔 翁 说 : 你 的 山 中 虽 然 可 以 比 得 上 去 , 还 不 如 我 的 水 秀 的 幽 雅 。
有 一 首 《 临 江 仙 》 作 为 证 明 : 潮 落 旋 移 孤 舟 去 , 夜 深 罢 棹 歌 来 。
衣 残 月 , 非 常 幽 暗 啊 , 宿 鸟 惊 不 起 , 天 际 彩 云 散 开 。
困 卧 芦 洲 没 有 什 么 事 , 三 竿 竿 日 上 还 能 忍 耐 。
随 心 尽 意 安 排 , 朝 臣 寒 待 漏 , 何 似 我 宽 怀 。
樵 夫 说 : 你 水 秀 的 幽 雅 , 还 不 如 我 山 青 更 幽 雅 。
也 有 《 临 江 仙 》 可 以 证 明 : 苍 径 秋 高 拽 斧 去 , 晚 凉 抬 担 回 来 。
野 花 插 在 鬓 上 更 奇 妙 了 , 拨 开 云 彩 寻 找 道 路 出 来 , 等 到 月 亮 开 门 。
小 儿 子 山 的 妻 子 高 兴 地 笑 着 接 他 , 草 床 木 枕 头 枕 头 都 很 难 看 。
蒸 梨 煮 黍 , 旋 即 铺 排 , 瓮 中 新 酿 熟 , 真 是 壮 怀 幽 怀 。
渔 翁 说 : 这 都 是 我 的 两 个 生 意 , 给 我 的 干 当 , 你 却 没 有 我 闲 时 节 的 好 处 。
有 诗 作 证 。
诗 说 : 闲 看 苍 天 白 鹤 飞 , 停 舟 溪 畔 掩 苍 门 。
他 靠 着 船 头 教 儿 子 弄 线 , 停 了 船 上 的 船 和 妻 子 一 同 晒 网 围 。
性 情 稳 定 , 果 然 如 同 浪 静 , 身 体 安 定 , 自 然 觉 得 风 微 。
绿 青 笠 随 时 戴 , 胜 挂 朝 中 紫 绶 衣 。
樵 夫 说 : 你 那 闲 时 又 不 如 我 的 闲 时 好 。
也 有 诗 作 为 证 据 。
诗 说 : 闲 观 白 云 飞 , 独 坐 茅 庵 掩 竹 门 。
无 事 教 儿 子 开 卷 读 书 , 有 时 对 客 人 拿 棋 围 。
喜 来 拄 着 拐 杖 唱 歌 芳 径 , 兴 到 携 琴 上 翠 微 。
草 鞋 麻 鞋 粗 布 被 , 心 胸 宽 宏 强 壮 , 好 像 穿 着 罗 衣 。
李 定 说 : 李 定 , 我 两 个 真 是 微 吟 可 以 相 狎 , 不 必 檀 板 共 金 樽 。
只 是 散 布 文 章 , 不 是 罕 见 的 。
而 且 各 联 几 句 , 看 我 们 渔 樵 攀 话 如 何 ? 李 定 道 说 : 张 兄 的 话 最 妙 。
请 哥 哥 先 吟 。
船 停 在 绿 水 的 烟 波 里 , 家 住 在 深 山 旷 野 中 。
偏 爱 溪 桥 春 水 涨 , 最 怜 岩 石 晓 云 蒙 。
龙 门 的 鲜 鲤 有 时 被 烹 煮 , 虫 蛀 干 柴 日 烤 烤 。
钓 鱼 的 鱼 多 种 , 可 以 赡 养 老 人 , 挑 着 绳 子 的 两 件 事 可 以 完 成 。
小 船 仰 卧 观 看 飞 雁 , 草 径 斜 听 鸣 雁 的 声 音 。
口 舌 场 中 没 有 我 的 分 别 , 是 非 海 内 没 有 我 的 踪 迹 。
溪 边 挂 着 晒 的 像 锦 缎 一 样 , 石 头 上 重 又 磨 的 斧 头 像 刀 刃 一 样 锋 利 。
秋 月 辉 辉 照 耀 , 常 独 自 钓 鱼 , 春 山 寂 寂 无 人 相 遇 。
鱼 多 换 酒 和 妻 子 一 起 饮 酒 , 柴 剩 买 酒 和 子 丛 一 起 喝 。
自 唱 自 斟 随 放 荡 , 长 歌 长 叹 任 颠 风 。
呼 唤 兄 弟 , 呼 唤 弟 弟 邀 请 船 伙 , 携 带 朋 友 聚 集 野 翁 。
行 令 猜 疑 , 频 频 递 递 灯 烛 , 拆 牌 道 字 漫 传 钟 。
烹 虾 煮 蟹 , 朝 朝 欢 乐 , 炒 鸭 鸡 日 日 丰 富 。
愚 妇 煎 茶 情 意 淡 泊 , 山 妻 做 饭 , 心 情 从 容 。
天 亮 时 , 举 起 木 杖 淘 轻 浪 , 日 出 时 担 着 柴 草 过 大 冲 。
雨 后 打 开 捕 活 的 鲤 鱼 , 在 风 前 弄 斧 砍 伐 枯 松 。
隐 居 避 世 , 变 成 痴 呆 愚 蠢 , 隐 姓 埋 名 做 哑 子 。
张 某 稍 微 说 道 : 李 兄 , 我 才 僭 先 起 句 , 今 天 到 我 哥 哥 , 也 先 起 一 联 , 小 弟 也 应 当 接 着 。
风 月 假 装 狂 奔 山 野 的 汉 人 , 江 湖 寄 托 傲 视 老 余 丁 。
清 闲 有 分 , 随 着 潇 洒 , 口 舌 无 闻 , 喜 欢 太 平 。
月 夜 自 己 睡 觉 , 茅 屋 稳 稳 , 天 色 昏 暗 , 身 体 轻 薄 。
忘 情 结 交 松 梅 友 , 乐 意 相 交 , 鸥 鹭 结 盟 。
名 利 心 头 没 有 计 算 , 战 争 耳 朵 听 不 到 声 音 。
随 时 斟 一 杯 香 酒 , 度 日 吃 野 菜 羹 。
两 捆 柴 薪 为 生 计 , 一 根 竿 钩 线 是 营 生 。
闲 呼 小 孩 磨 钢 斧 , 静 呼 憨 儿 补 旧 。
春 天 到 了 喜 欢 观 赏 杨 柳 绿 , 时 融 喜 欢 看 看 芦 荻 青 。
夏 天 避 暑 修 新 竹 , 六 月 乘 凉 摘 嫩 菱 。
霜 降 时 鸡 肥 , 常 常 宰 杀 , 重 阳 时 蟹 壮 , 及 时 烹 煮 。
冬 至 日 上 还 沉 睡 , 几 九 天 高 自 然 不 寒 。
八 节 山 中 随 着 放 纵 性 情 , 四 时 湖 里 任 凭 陶 冶 情 性 。
采 薪 自 有 仙 家 兴 起 , 垂 钓 钓 鱼 完 全 没 有 世 俗 的 形 态 。
门 外 野 花 香 艳 艳 , 船 头 绿 水 波 浪 平 平 。
身 体 安 稳 不 喜 欢 三 公 的 地 位 , 性 情 稳 定 , 就 像 十 里 城 一 样 坚 定 。
十 里 城 高 防 守 法 令 , 三 公 的 地 位 显 赫 , 听 到 宣 布 的 声 音 。
乐 山 乐 水 真 是 罕 , 谢 天 谢 地 谢 神 明 。
其 他 二 人 既 然 各 自 谈 论 词 章 , 又 互 相 联 缀 诗 句 。
走 到 那 分 路 的 地 方 , 亲 自 告 别 。
张 稍 微 说 道 : 李 哥 呵 , 路 上 保 重 , 上 山 仔 细 看 虎 。
假 如 有 些 凶 险 , 正 好 是 明 天 街 头 少 老 朋 友 。
李 定 听 了 这 话 , 大 怒 说 : 你 这 个 人 疲 惫 懒 惰 , 好 朋 友 也 替 我 生 死 , 你 为 什 么 诅 咒 我 , 我 如 果 遇 到 虎 遭 害 , 你 一 定 会 遇 到 浪 翻 江 。
张 稍 微 说 道 : 我 是 永 世 不 得 翻 江 。
李 定 道 说 : 天 有 不 测 的 风 云 , 人 有 暂 时 的 祸 福 。
张 稍 说 : 李 哥 , 你 虽 然 这 样 说 , 但 你 还 没 有 捉 摸 , 不 如 我 的 生 意 有 所 摸 摸 , 一 定 不 会 遭 到 这 样 的 事 。
李 定 说 : 你 那 水 面 上 的 营 生 , 极 凶 极 险 , 隐 隐 暗 暗 , 有 什 么 可 以 摸 摸 ? 张 稍 说 : 你 是 不 晓 得 的 。
长 安 城 里 , 西 门 街 上 有 个 卖 卦 的 先 生 。
我 每 天 送 给 他 一 只 金 色 的 鲤 鱼 , 他 就 给 我 一 个 袖 子 传 授 , 按 照 方 位 , 百 下 百 著 。
今 天 我 又 去 买 卦 , 他 教 我 在 泾 河 湾 头 东 边 放 网 , 西 岸 扔 钩 , 一 定 能 得 到 满 载 的 鱼 虾 回 去 。
第 二 天 上 城 来 , 卖 钱 买 酒 , 再 次 与 老 兄 相 叙 。
二 人 从 此 分 别 。
这 正 是 说 : 路 上 说 话 , 草 里 有 人 。
原 来 泾 河 水 府 有 个 叫 夜 叉 的 夜 叉 , 听 到 百 下 百 著 的 话 , 急 忙 转 到 水 晶 宫 , 急 忙 报 告 给 龙 王 说 : 祸 事 了 , 祸 事 了 。 龙 王 问 道 : 有 什 么 祸 事 ? 夜 叉 说 : 我 巡 视 水 边 , 到 河 边 , 只 听 到 两 个 渔 夫 、 樵 夫 攀 着 话 , 相 别 时 , 言 语 很 是 利 害 。
那 个 渔 翁 说 : 长 安 城 里 , 西 门 街 上 有 个 卖 卦 的 先 生 , 算 得 最 准 。
他 日 送 给 他 一 只 鲤 鱼 , 他 就 传 授 给 他 一 个 , 教 他 百 下 百 著 。
如 果 按 照 这 样 的 计 策 , 却 不 把 水 族 全 都 打 了 , 何 以 壮 观 水 府 , 何 以 跃 浪 翻 波 , 辅 助 大 王 的 威 力 ? 龙 王 非 常 恼 怒 , 急 忙 提 剑 , 就 要 上 长 安 城 , 杀 掉 这 个 卖 卦 的 人 。
旁 边 闪 过 龙 子 、 龙 孙 、 虾 臣 、 螃 蟹 士 、 军 师 、 少 卿 、 鲤 太 宰 , 一 齐 启 奏 道 : 大 王 暂 且 息 怒 。
常 说 道 : 过 耳 之 言 , 不 可 信 。
大 王 此 去 , 必 定 有 云 从 , 必 定 有 雨 来 帮 助 , 恐 怕 惊 吓 了 长 安 百 姓 , 上 天 受 到 谴 责 。
大 王 隐 藏 不 测 , 变 化 无 方 , 只 变 成 一 个 秀 士 , 到 长 安 城 内 访 问 一 番 。
如 果 没 有 这 样 的 人 , 难 道 不 是 胡 乱 杀 害 别 人 吗 ? 龙 王 依 照 他 的 意 见 , 就 放 弃 宝 剑 , 也 不 兴 云 雨 , 出 了 岸 上 , 转 身 一 变 , 变 成 一 个 白 衣 秀 士 , 真 的 是 个 : 丰 姿 英 伟 , 高 耸 云 霄 。
步 履 端 正 , 遵 循 规 矩 。
语 言 遵 循 孔 子 、 孟 子 , 礼 仪 表 现 周 文 王 。
他 身 穿 玉 色 罗 服 , 头 戴 逍 遥 一 字 头 巾 。
上 路 来 , 拽 开 云 步 , 径 直 走 到 长 安 城 西 门 大 街 上 。
只 见 一 群 人 , 喧 闹 闹 闹 闹 闹 闹 闹 闹 闹 闹 闹 闹 闹 闹 闹 闹 闹 闹 闹 闹 闹 闹 闹 闹 。
其 中 有 高 谈 阔 论 的 道 : 属 龙 的 本 命 , 属 虎 的 相 冲 。
寅 、 辰 、 巳 、 亥 , 虽 然 称 为 合 局 , 但 害 怕 的 是 太 阳 犯 岁 星 。
龙 王 听 了 这 话 , 心 里 知 道 这 是 卖 卜 的 地 方 。
走 上 前 , 分 开 众 人 , 望 着 里 边 观 看 。
只 见 四 壁 珠 , 满 堂 绮 绣 。
宝 鸭 香 无 断 , 磁 瓶 水 更 清 。
两 边 罗 列 着 王 维 画 的 图 画 , 座 上 高 悬 着 鬼 谷 的 形 状 。
端 溪 砚 , 金 烟 墨 , 互 相 依 映 着 霜 毫 大 笔 ; 火 珠 林 , 郭 璞 数 字 , 谨 对 了 《 台 政 新 经 》 。
六 爻 熟 悉 , 八 卦 精 通 。
他 能 够 知 道 天 地 之 理 , 善 于 晓 得 鬼 神 之 情 。
一 盘 子 午 安 排 定 , 满 腹 星 辰 排 列 清 清 。
真 的 那 些 未 来 的 事 情 , 过 去 的 事 情 , 看 来 就 像 月 镜 一 样 明 亮 , 几 家 兴 盛 , 几 家 败 败 , 就 像 神 明 一 样 明 察 。
知 凶 定 吉 , 断 死 言 生 。
开 谈 风 雨 迅 猛 , 下 笔 鬼 神 惊 骇 。
招 牌 上 有 字 写 姓 名 , 是 神 课 先 生 袁 守 诚 。
原 来 是 当 朝 钦 天 监 、 台 正 先 生 袁 天 的 叔 父 , 是 袁 守 诚 。
先 生 果 然 相 貌 稀 奇 , 仪 容 秀 美 , 名 扬 大 国 , 术 冠 长 安 。
龙 王 进 门 来 , 与 先 生 相 见 。
礼 仪 结 束 , 请 龙 上 座 , 童 子 献 茶 。
先 生 问 道 : 你 来 问 什 么 事 ? 龙 王 说 : 请 你 占 卜 天 上 的 阴 晴 情 况 怎 么 样 ?
先 生 就 袖 子 传 授 了 一 篇 课 , 断 了 一 句 话 : 云 迷 山 顶 , 雾 罩 林 梢 。
如 果 占 卜 雨 泽 , 应 验 在 明 朝 。
龙 王 问 : 明 天 什 么 时 候 下 雨 , 雨 有 多 少 尺 寸 ? 先 生 说 : 明 天 辰 时 布 云 , 巳 时 发 雷 , 午 时 下 雨 , 未 时 雨 足 , 共 得 到 三 尺 三 寸 零 四 十 八 点 。
龙 王 笑 着 说 : 这 话 不 能 作 戏 。
如 果 没 有 下 雨 , 我 给 你 断 的 时 辰 、 数 目 , 我 送 给 你 五 十 两 黄 金 来 奉 谢 。 如 果 没 有 下 雨 , 或 者 不 按 照 时 辰 、 数 目 , 我 就 和 你 说 : 一 定 要 打 坏 你 的 门 面 , 扯 碎 你 的 招 牌 , 立 即 赶 出 长 安 , 不 许 在 这 里 迷 惑 众 人 。
先 生 高 兴 地 回 答 说 : 这 个 一 定 要 任 你 。
请 教 , 请 教 。
第 二 天 , 雨 后 来 会 合 。
龙 王 辞 别 , 从 长 安 出 发 , 回 到 水 府 。
大 小 水 神 接 着 问 道 : 大 王 问 那 个 卖 卦 的 人 怎 么 样 ? 龙 王 说 : 有 , 有 , 有 。
但 是 一 个 掉 嘴 去 讨 春 的 先 生 。
吾 问 他 何 时 下 雨 , 他 就 说 明 天 下 雨 ?
又 问 他 什 么 时 辰 , 什 么 雨 数 , 他 就 说 : 辰 时 布 云 , 巳 时 下 雨 , 午 时 下 雨 , 未 时 雨 足 , 得 水 三 尺 三 寸 零 四 十 八 点 。
我 与 他 打 了 赌 赛 , 如 果 真 像 他 说 的 , 送 给 他 谢 金 五 十 两 , 如 果 略 略 差 点 , 就 打 破 他 的 门 面 , 赶 他 起 身 , 不 许 在 长 安 迷 惑 众 人 。
众 水 族 笑 着 说 : 大 王 你 是 八 河 都 总 管 , 是 司 雨 大 龙 神 , 有 雨 无 雨 , 只 有 大 王 知 道 。
他 怎 么 敢 这 样 胡 言 , 那 卖 卦 的 定 是 输 了 , 定 是 输 了 。
这 时 龙 子 、 龙 孙 和 那 鱼 卿 、 螃 蟹 的 人 正 在 欢 笑 , 谈 论 此 事 还 没 有 结 束 , 只 听 到 半 空 中 的 叫 道 : 泾 河 龙 王 接 旨 。
众 人 抬 头 上 去 看 , 是 一 个 金 衣 力 士 , 手 里 拿 着 玉 帝 的 敕 令 , 直 奔 水 府 而 来 。
忽 然 龙 王 整 理 衣 服 端 庄 肃 穆 , 焚 香 接 旨 。
穿 着 金 衣 的 力 士 回 到 空 中 去 了 。
龙 王 谢 恩 , 拆 封 看 时 , 上 面 写 道 : 敕 命 八 河 总 , 驱 雷 掣 电 行 , 明 朝 施 雨 泽 , 普 济 长 安 城 。
圣 旨 上 呈 时 辰 、 数 目 , 与 那 先 生 判 断 , 毫 发 不 差 。
如 果 得 到 那 龙 王 , 魂 飞 魄 散 。
不 一 会 儿 , 他 醒 过 来 , 对 众 水 族 说 : 尘 世 上 有 这 样 的 灵 人 , 真 的 是 能 通 天 地 理 , 却 不 让 给 他 呀 ! 军 师 上 奏 说 : 大 王 放 心 。
要 打 败 他 又 有 什 么 难 处 呢 ? 我 有 小 计 , 管 教 毁 掉 那 个 人 的 嘴 嘴 。
龙 王 问 计 , 军 师 说 : 行 雨 误 了 时 间 , 少 点 数 , 就 是 那 个 人 断 卦 不 准 , 恐 怕 不 胜 他 , 那 时 打 碎 招 牌 , 赶 他 跑 路 , 果 然 有 什 么 困 难 呢 ? 龙 王 依 照 他 的 奏 请 , 果 然 不 担 忧 。
到 了 第 二 天 , 他 点 着 书 信 给 风 伯 、 雷 公 、 云 童 、 电 母 , 直 到 长 安 城 九 霄 空 上 。
及 至 于 此 。
改 其 一 个 时 辰 , 减 其 三 寸 八 点 。
雨 后 发 放 , 众 将 班 师 。
又 有 一 个 白 衣 秀 士 , 到 了 西 门 里 的 大 街 上 , 撞 进 了 袁 守 诚 的 卦 铺 , 不 容 分 说 , 就 把 他 的 招 牌 、 笔 、 砚 等 一 齐 打 碎 。
先 生 坐 在 椅 上 , 公 然 不 动 。
这 个 龙 王 又 拿 起 门 板 就 打 , 骂 道 : 这 胡 说 祸 福 的 妖 人 , 擅 自 蛊 惑 众 心 的 泼 汉 , 你 卦 又 不 灵 , 说 话 又 狂 妄 荒 谬 。
又 说 : 今 天 下 雨 的 时 辰 、 点 数 都 不 相 对 。
你 还 是 危 坐 高 坐 , 趁 早 去 , 饶 你 死 罪 ! 守 诚 还 是 公 然 不 害 怕 , 仰 面 朝 天 , 冷 笑 着 说 : 我 不 怕 , 我 不 怕 , 我 不 怕 。
我 没 有 死 罪 , 只 怕 你 倒 有 死 罪 吗 ?
别 人 喜 欢 欺 骗 , 只 是 难 以 欺 骗 我 。
我 认 得 你 , 你 不 是 秀 士 , 而 是 泾 河 龙 王 。
卿 之 所 以 违 背 玉 帝 的 旨 意 , 改 变 了 时 辰 , 改 变 了 点 数 , 违 犯 了 天 命 。
你 在 那 龙 台 上 , 恐 怕 难 免 一 刀 , 你 还 在 这 里 骂 我 吗 ? 龙 王 见 了 , 心 惊 胆 战 , 毛 骨 惊 恐 。
急 忙 丢 下 门 板 , 整 理 衣 服 伏 礼 , 向 先 生 跪 下 说 : 先 生 不 要 怪 怪 。
以 前 的 话 是 戏 弄 的 , 怎 么 知 道 弄 假 成 真 , 果 然 违 背 了 天 命 , 怎 么 办 呢 ? 希 望 先 生 救 我 一 救 , 不 然 , 我 死 也 不 放 你 。
守 诚 说 : 我 救 你 不 得 , 只 是 给 你 一 条 生 路 , 与 你 投 生 就 是 了 。
刘 龙 说 : 请 求 您 指 点 教 导 。
先 生 说 : 你 明 天 午 时 三 刻 , 应 该 到 人 曹 官 魏 徵 那 里 听 任 斩 首 。
你 如 果 要 生 命 , 必 须 急 急 去 告 诉 我 , 当 今 唐 太 宗 皇 帝 正 好 。
魏 徵 是 唐 王 驾 驭 之 下 的 丞 相 , 如 果 是 讨 伐 他 人 的 意 思 , 才 能 保 证 无 事 。
龙 王 听 了 这 话 , 叩 拜 辞 别 , 含 着 眼 泪 离 开 了 。
不 觉 红 色 的 太 阳 西 沉 , 太 阴 星 上 升 。
只 见 烟 雾 凝 聚 山 紫 , 归 鸦 疲 倦 , 远 路 行 人 投 奔 旅 店 。
渡 头 新 雁 停 宿 汀 沙 , 银 河 出 现 , 催 促 更 筹 , 孤 村 灯 火 没 有 火 光 。
风 吹 拂 炉 烟 清 道 院 , 蝴 蝶 梦 中 的 人 不 见 了 。
月 亮 移 动 花 影 上 栏 杆 , 星 光 混 乱 , 漏 声 更 换 , 不 觉 已 经 深 沉 , 夜 已 半 夜 。
这 样 泾 河 龙 王 也 不 回 到 水 府 , 只 在 空 中 。
等 到 子 时 前 后 , 收 起 了 云 头 , 收 起 了 雾 角 , 径 直 来 到 皇 宫 门 前 。
这 时 唐 王 正 在 梦 见 从 宫 门 外 出 来 , 在 花 阴 里 走 出 。
忽 然 , 龙 王 变 成 人 相 , 上 前 跪 拜 , 口 里 大 叫 : 陛 下 , 救 我 , 救 我 。
太 宗 说 : 你 是 什 么 人 , 我 应 当 救 你 。
龙 王 说 : 陛 下 是 真 龙 , 我 是 业 龙 。
臣 因 犯 了 天 命 , 应 当 陛 下 的 贤 臣 人 曹 官 魏 徵 处 斩 , 所 以 来 拜 求 , 希 望 陛 下 救 我 一 救 。
太 宗 说 : 既 然 被 魏 徵 处 斩 , 朕 可 以 救 你 , 你 放 心 前 去 。
龙 王 十 分 高 兴 , 叩 谢 而 去 。
又 说 : 那 太 宗 梦 醒 之 后 , 念 念 在 心 。
早 已 到 五 更 三 点 , 太 宗 设 朝 , 聚 集 两 班 文 武 官 员 。
只 见 那 里 烟 雾 笼 罩 凤 阙 , 香 气 缭 绕 龙 楼 。
光 辉 摇 动 丹 动 , 云 彩 拂 动 翠 华 流 。
君 臣 之 间 的 关 系 和 尧 、 舜 相 同 , 礼 乐 威 严 接 近 汉 、 周 。
侍 臣 的 灯 , 宫 女 的 扇 子 , 双 双 映 着 彩 色 ; 孔 雀 屏 风 , 麒 麟 殿 , 处 处 光 彩 浮 动 。
山 呼 万 岁 , 华 祝 千 秋 。
静 鞭 三 下 响 , 衣 冠 拜 谢 冕 。
宫 花 灿 烂 , 天 香 袭 袭 , 堤 柳 轻 柔 御 乐 。
珍 珠 帘 , 翡 翠 帘 , 金 钩 高 悬 , 龙 凤 扇 , 山 河 扇 , 宝 停 留 。
文 官 英 秀 , 武 将 精 神 抖 擞 。
御 道 分 高 下 , 丹 墀 列 品 流 。
金 章 紫 绶 乘 三 象 , 地 久 天 长 万 万 秋 。
众 官 员 朝 贺 已 毕 , 各 自 分 班 。
唐 王 闪 凤 目 龙 睛 , 一 一 从 头 观 看 , 只 见 文 官 内 是 房 玄 龄 、 杜 如 晦 、 徐 世 、 许 敬 宗 、 王 王 圭 等 , 武 官 内 是 马 三 宝 、 段 志 玄 、 殷 开 山 、 程 咬 金 、 刘 洪 纪 、 胡 敬 德 、 秦 叔 宝 等 人 , 一 个 个 威 仪 端 庄 严 肃 , 却 不 见 魏 徵 、 丞 相 。
唐 王 召 徐 世 上 殿 说 : 我 晚 上 做 了 一 个 怪 梦 , 梦 见 一 个 人 , 迎 面 拜 见 , 口 称 是 泾 河 龙 王 , 犯 了 天 命 , 这 个 人 曹 官 魏 征 被 处 斩 , 拜 告 我 救 他 , 我 已 经 答 应 了 。
今 天 我 单 单 没 有 见 到 魏 徵 , 这 是 为 什 么 呢 ? 世 回 答 说 : 这 个 梦 告 诉 了 准 。
必 须 叫 魏 徵 来 朝 见 , 陛 下 不 要 放 他 出 门 , 过 了 这 一 天 , 可 以 救 助 梦 中 的 龙 。
唐 王 大 喜 , 立 即 传 旨 , 让 当 驾 官 宣 布 魏 征 入 朝 。
又 说 魏 徵 丞 相 在 府 中 , 夜 里 观 看 天 象 , 正 在 喷 着 宝 香 , 只 闻 到 九 霄 鹤 鸣 , 却 是 天 派 仙 使 , 捧 玉 帝 的 金 旨 一 道 , 写 着 午 时 三 刻 , 梦 见 斩 了 泾 河 老 龙 。
丞 相 谢 了 天 恩 , 斋 戒 沐 浴 , 在 府 中 试 试 慧 剑 , 运 转 元 神 , 所 以 不 曾 入 朝 。
一 见 到 当 驾 官 前 来 宣 布 , 惶 恐 没 有 任 用 , 又 不 敢 违 背 您 的 命 令 , 只 得 急 忙 整 顿 衣 服 束 带 , 同 旨 入 朝 , 在 皇 帝 面 前 叩 头 请 罪 。
唐 王 说 : 赦 免 你 无 罪 。
当 时 诸 臣 还 没 有 退 朝 , 到 了 这 里 , 又 命 令 他 们 卷 起 帘 子 散 开 朝 拜 。
只 留 下 魏 徵 , 宣 上 金 銮 , 召 入 便 殿 , 先 议 论 安 定 国 家 的 策 略 , 安 定 国 家 的 计 谋 。
将 近 巳 末 午 初 时 候 , 又 命 令 宫 人 说 : 取 过 大 棋 来 , 朕 与 贤 卿 对 弈 一 局 。
众 嫔 妃 随 即 取 下 棋 盘 , 铺 设 御 案 。
魏 徵 谢 罪 , 立 即 与 唐 王 对 弈 , 一 递 一 响 , 摆 开 阵 势 。
正 合 《 烂 柯 经 》 上 说 : 博 弈 之 道 , 贵 在 严 谨 。
高 的 在 腹 部 , 低 的 在 边 部 , 中 等 在 角 宿 , 这 是 下 棋 家 的 常 法 。
诚 如 兵 法 所 说 : 宁 愿 输 一 子 , 不 失 一 先 。
攻 击 左 边 的 敌 人 就 看 到 右 边 的 敌 人 , 攻 击 后 面 的 敌 人 就 看 到 前 面 的 敌 人 。
有 先 有 后 , 有 后 有 先 。
两 生 不 断 , 都 活 不 连 。
宽 广 不 可 过 于 疏 远 , 密 密 不 可 过 于 短 促 。
与 其 恋 恋 儿 子 而 求 生 , 不 如 抛 弃 儿 子 而 取 得 胜 利 ; 与 其 无 事 而 独 自 行 走 , 不 如 坚 守 自 己 。
敌 众 我 寡 , 首 先 谋 求 生 存 ; 我 众 敌 寡 , 务 必 扩 张 势 力 。
善 于 取 胜 的 人 不 去 争 斗 , 善 于 布 阵 的 人 不 去 作 战 ; 善 于 作 战 的 人 不 会 失 败 , 善 于 打 败 的 人 不 会 混 乱 。
下 棋 , 开 始 以 正 确 合 , 最 终 以 奇 计 取 胜 。
凡 是 敌 人 没 有 战 事 而 自 我 补 救 , 有 侵 袭 、 断 绝 敌 人 的 想 法 ; 抛 弃 小 国 而 不 去 救 援 , 是 有 图 谋 大 国 的 想 法 ; 随 手 而 下 的 , 是 没 有 谋 略 的 人 ; 不 思 考 而 应 战 的 , 是 取 败 的 方 法 。
《 诗 经 》 上 说 : 惴 惴 小 心 , 好 像 面 临 山 谷 。
说 的 就 是 这 个 意 思 。
《 诗 经 》 上 说 : 棋 盘 为 地 , 子 子 为 天 , 颜 色 依 照 阴 阳 , 造 化 完 备 。
下 到 玄 微 通 变 的 地 方 , 笑 着 称 赞 当 日 烂 柯 仙 。
君 臣 二 人 对 弈 , 此 棋 正 下 到 午 时 三 刻 , 一 盘 残 局 还 没 有 完 , 魏 徵 忽 然 俯 伏 在 桌 子 旁 边 , 大 声 呼 叫 , 睡 着 。
太 宗 笑 着 说 : 贤 卿 真 是 匡 扶 社 稷 的 心 劳 , 创 立 江 山 的 力 量 疲 倦 , 所 以 不 知 不 觉 睡 觉 。
太 宗 让 他 睡 觉 , 更 不 呼 唤 。
不 多 时 , 魏 徵 醒 来 , 俯 伏 在 地 上 说 : 我 该 万 死 , 我 该 万 死 ! 我 才 疲 倦 , 不 知 所 措 , 希 望 陛 下 赦 免 我 怠 慢 您 的 罪 过 。
太 宗 说 : 你 有 什 么 怠 慢 的 罪 过 , 暂 且 起 来 , 拂 退 残 下 的 棋 子 , 与 你 从 新 更 换 。
魏 徵 谢 了 恩 , 却 还 在 手 上 , 忽 然 听 到 他 在 门 外 大 呼 小 叫 。
原 来 是 秦 叔 宝 、 徐 茂 公 等 人 , 带 着 一 个 血 淋 的 龙 头 , 扔 在 皇 帝 面 前 , 启 奏 道 : 陛 下 , 海 浅 河 枯 , 曾 有 见 过 , 这 样 奇 异 的 事 情 却 没 有 听 到 。
太 宗 和 魏 徵 起 身 说 : 这 些 东 西 从 哪 里 来 ? 叔 宝 、 茂 公 说 : 千 步 廊 南 , 十 字 街 上 , 云 端 里 落 下 这 个 龙 头 , 微 臣 不 敢 不 奏 报 。
唐 王 惊 讶 地 问 魏 徵 : 这 是 什 么 话 ? 魏 徵 转 身 叩 头 说 : 是 我 刚 刚 一 梦 就 杀 了 的 。
唐 王 听 了 这 话 , 大 吃 一 惊 说 : 您 睡 着 的 时 候 , 又 不 曾 见 过 动 身 动 手 , 又 没 有 刀 剑 , 为 什 么 却 杀 了 这 条 龙 呢 ? 魏 徵 上 奏 说 : 主 公 , 我 的 身 子 在 您 面 前 , 梦 见 我 离 开 陛 下 。
我 在 您 面 前 面 对 着 残 局 , 眼 睛 一 片 糊 涂 , 梦 中 离 开 陛 下 乘 着 祥 云 , 出 神 抖 擞 。
其 中 有 一 条 龙 在 龙 台 上 , 被 天 兵 将 要 绑 在 里 面 。
我 说 : 你 犯 了 天 命 , 应 该 判 死 罪 。
我 奉 天 命 , 斩 杀 你 残 生 。
龙 听 到 哀 伤 痛 苦 , 我 抖 动 精 神 。
龙 听 到 哀 痛 之 声 , 伏 爪 收 鳞 甘 愿 受 死 ; 臣 振 奋 精 神 , 撩 衣 进 步 , 举 起 霜 刃 。
一 声 刀 过 去 的 地 方 , 龙 头 因 此 落 在 空 中 。
太 宗 听 了 这 话 , 心 中 悲 喜 不 一 。
高 兴 的 人 , 是 夸 奖 魏 徵 的 好 臣 , 朝 中 有 这 样 的 豪 杰 , 愁 苦 江 山 不 稳 , 悲 伤 的 是 说 梦 中 曾 经 答 应 救 龙 , 不 料 竟 然 遭 到 杀 害 。
只 得 勉 强 打 精 神 , 传 旨 说 : 叔 宝 将 龙 头 悬 挂 在 市 场 上 , 晓 谕 长 安 百 姓 。
一 个 墙 上 的 厢 房 赏 赐 了 魏 徵 , 众 官 员 都 散 去 了 。
当 晚 回 宫 , 心 中 只 是 忧 愁 。
想 想 那 些 梦 中 的 龙 , 哭 啼 哀 哀 地 祈 祷 求 生 , 哪 里 知 道 没 有 常 规 , 难 以 免 除 这 种 灾 祸 。
思 念 多 时 , 渐 渐 觉 得 神 魂 疲 倦 , 身 体 不 安 。
当 夜 二 更 时 分 , 只 听 到 宫 门 外 有 哭 泣 的 声 音 , 太 宗 更 加 惊 恐 。
正 在 朦 胧 睡 觉 之 中 , 又 看 见 泾 河 龙 王 手 里 拿 着 一 颗 血 淋 淋 的 首 级 , 高 声 大 叫 : 唐 太 宗 , 还 我 命 来 , 还 我 命 来 , 你 昨 夜 满 口 答 应 救 我 , 为 何 天 亮 时 反 叛 宣 人 曹 官 来 杀 我 ? 你 出 来 , 你 出 来 , 我 和 你 到 阎 君 那 里 审 判 。
太 宗 曰 : 不 可 以 为 之 , 不 可 以 为 之 。
太 宗 闭 口 不 说 话 , 只 是 摸 得 汗 流 遍 身 。
正 在 那 些 难 分 难 解 的 时 候 , 只 见 正 南 上 香 云 缭 绕 , 彩 雾 飘 飘 , 有 一 个 女 真 人 上 前 , 把 杨 柳 枝 用 手 一 挥 , 那 没 头 的 龙 悲 悲 哭 泣 , 径 直 往 西 北 走 去 。
原 来 是 观 音 菩 萨 领 着 佛 旨 , 上 到 东 土 寻 找 佛 经 的 人 , 住 在 这 里 长 安 城 都 土 地 庙 里 , 夜 里 听 到 鬼 哭 神 号 , 特 来 喝 退 业 龙 , 救 救 脱 离 皇 帝 。
那 龙 径 直 到 阴 司 地 狱 , 详 细 地 告 诉 了 没 有 问 题 。
又 说 : 有 鬼 , 有 鬼 , 惊 得 三 宫 皇 后 、 六 院 嫔 妃 , 与 近 侍 太 监 , 战 战 兢 兢 , 一 夜 没 有 睡 觉 。
不 觉 五 更 三 点 , 满 朝 文 武 官 员 多 , 都 在 朝 门 外 等 候 朝 见 。
等 到 天 亮 , 还 看 不 见 临 朝 , 何 况 一 个 人 都 惊 恐 不 安 。
等 到 太 阳 升 起 三 竿 时 , 才 有 旨 意 出 来 说 : 朕 心 里 不 快 , 众 官 免 朝 。
不 知 不 觉 过 了 五 七 天 , 众 官 都 忧 虑 惶 恐 , 都 正 要 撞 门 见 皇 帝 问 安 , 只 见 太 后 有 旨 , 召 医 官 入 宫 用 药 。
众 人 在 朝 门 外 等 候 讨 伐 韩 信 。
不 一 会 儿 , 医 官 出 来 了 , 大 家 问 他 什 么 病 。
医 官 说 : 皇 上 脉 气 不 正 , 虚 而 又 数 , 狂 言 见 鬼 。
又 诊 得 十 动 一 代 , 五 脏 无 气 , 恐 怕 不 避 忌 , 只 在 七 天 之 内 。
众 官 听 了 , 大 惊 失 色 。
正 在 惊 恐 之 间 , 又 听 到 太 后 有 旨 , 宣 读 徐 茂 公 、 护 国 公 、 尉 迟 恭 前 来 朝 见 。
三 公 奉 旨 , 急 忙 进 入 分 宫 楼 下 。
拜 完 之 后 , 太 宗 严 肃 地 勉 强 说 道 : 贤 卿 , 我 十 九 岁 领 兵 , 南 征 北 伐 , 东 击 西 除 , 苦 苦 经 历 数 年 , 更 不 曾 见 到 半 点 邪 魔 , 今 天 反 而 见 到 鬼 怪 。
尉 迟 恭 说 : 创 立 江 山 , 杀 人 无 数 , 为 什 么 怕 鬼 呢 ? 太 宗 说 : 你 是 不 相 信 的 。
我 的 寝 宫 门 外 , 入 夜 就 抛 砖 弄 瓦 , 鬼 魅 呼 号 , 显 然 难 以 处 理 。
白 天 还 可 以 , 昏 夜 难 以 禁 止 。
王 叔 宝 说 : 陛 下 宽 心 , 今 晚 我 和 尉 迟 敬 德 把 守 宫 门 , 看 看 有 什 么 鬼 怪 ?
太 宗 听 从 了 他 的 奏 章 。
茂 公 谢 恩 而 出 。
当 时 天 色 已 晚 , 各 自 取 来 披 挂 , 其 他 两 个 人 都 很 整 齐 , 拿 着 金 瓜 、 斧 , 在 宫 门 外 把 守 。
好 将 军 , 你 看 他 怎 样 打 扮 ? 头 戴 金 盔 , 光 闪 闪 , 身 披 铠 甲 龙 鳞 。
护 心 宝 镜 飘 动 祥 云 , 狮 子 蛮 人 收 紧 紧 紧 , 绣 带 彩 霞 新 。
这 一 个 凤 眼 朝 天 , 星 斗 恐 惧 , 那 一 个 环 睛 闪 闪 闪 闪 闪 闪 闪 闪 闪 闪 闪 闪 闪 , 月 光 浮 动 。
他 本 来 是 英 雄 豪 杰 的 勋 臣 , 只 落 得 千 年 称 户 尉 , 万 古 作 门 神 。
两 位 将 军 侍 立 在 门 旁 , 一 夜 天 亮 了 , 再 也 没 有 发 现 一 点 邪 魔 。
当 夜 , 太 宗 在 宫 中 , 安 寝 无 事 。
第 二 天 早 晨 来 宣 告 二 位 将 军 , 重 重 地 奖 赏 慰 劳 他 们 说 : 我 自 从 得 病 , 好 几 天 不 能 睡 觉 , 今 天 夜 里 依 仗 二 位 将 军 的 威 势 , 很 安 定 。
你 暂 且 请 求 出 去 安 息 , 安 息 , 等 到 晚 上 再 一 次 护 卫 。
二 位 将 领 谢 恩 而 出 。
于 是 二 三 夜 , 把 守 都 安 定 了 。
只 是 御 膳 减 少 , 病 转 觉 得 更 重 。
太 宗 又 不 忍 心 两 位 将 领 辛 苦 , 又 宣 叔 宝 、 尉 迟 敬 德 和 杜 、 房 诸 位 公 卿 入 宫 , 便 嘱 咐 他 们 说 : 这 两 天 我 虽 然 得 到 安 全 , 但 只 是 难 以 做 秦 、 胡 二 位 将 军 彻 夜 辛 苦 。
我 想 召 来 一 位 巧 妙 的 手 笔 , 传 给 二 位 将 军 的 真 相 , 贴 在 门 上 , 免 得 劳 累 其 他 人 。
众 臣 随 即 按 照 旨 意 , 选 了 两 个 会 写 真 的 , 让 胡 、 秦 二 公 依 前 披 挂 , 照 样 画 了 , 贴 在 门 上 。
夜 间 , 就 没 有 事 情 。
这 样 过 了 二 三 天 , 又 听 到 后 宰 门 的 声 音 嘈 嘈 嘈 嘈 嘈 嘈 嘈 嘈 嘈 , 砖 瓦 乱 响 。
第 二 天 早 晨 前 来 就 宣 告 大 臣 们 说 : 连 日 前 门 , 幸 好 没 事 , 今 夜 后 门 又 响 , 不 是 又 惊 杀 了 我 吗 ?
王 茂 公 上 前 奏 道 : 前 门 不 安 , 是 敬 德 、 叔 宝 护 卫 , 后 门 不 安 , 应 该 让 魏 徵 护 卫 。
太 宗 听 从 了 他 的 奏 章 , 又 宣 布 魏 徵 今 晚 把 守 后 门 。
魏 徵 领 旨 , 当 夜 束 束 整 齐 , 提 着 杀 龙 的 宝 剑 , 侍 立 在 后 宰 门 前 , 真 是 个 好 英 雄 。
他 怎 么 能 生 出 这 样 的 样 子 呢 ? 熟 绢 青 巾 抹 额 , 锦 袍 玉 带 垂 腰 。
兜 风 掀 起 袖 子 , 彩 霜 飘 动 , 压 住 赛 垒 荼 的 神 态 。
他 脚 踏 着 黑 色 靴 子 , 坐 断 了 , 手 里 拿 着 利 刃 凶 猛 。
圆 睁 两 眼 , 四 边 看 , 那 个 邪 神 胆 敢 来 到 , 一 夜 之 间 都 明 亮 了 , 也 没 有 鬼 魅 。
虽 然 前 后 门 无 事 , 只 是 身 体 渐 渐 沉 重 。
一 天 , 太 后 又 传 旨 , 召 集 众 臣 商 议 丧 葬 后 事 。
太 宗 又 宣 告 徐 茂 公 , 嘱 咐 国 家 大 事 , 并 且 嘱 咐 刘 蜀 主 托 孤 的 意 思 。
说 完 , 沐 浴 更 衣 , 等 待 时 机 罢 了 。
旁 闪 魏 徵 , 手 扯 龙 衣 , 上 奏 说 : 陛 下 宽 心 , 我 有 一 件 事 , 只 管 保 陛 下 长 生 。
魏 徵 说 : 我 的 病 势 已 经 进 入 膏 肓 , 命 将 危 险 了 , 怎 么 能 保 证 得 到 魏 徵 说 : 我 有 一 封 信 , 送 给 陛 下 , 送 去 到 阴 司 , 交 给 都 判 官 崔 。
魏 徵 说 : 崔 是 太 上 先 皇 帝 驾 车 前 的 大 臣 , 先 前 受 此 洲 令 , 后 来 升 为 礼 部 侍 郎 。
在 这 一 天 与 我 拜 八 拜 为 朋 友 , 彼 此 交 情 深 厚 。
其 今 已 经 死 , 今 在 阴 司 任 掌 管 生 死 文 簿 的 都 判 官 , 梦 中 常 常 与 我 相 会 。
此 去 如 果 把 这 封 信 交 给 他 , 他 念 微 臣 薄 恩 , 必 然 放 陛 下 回 来 。
管 教 魂 魄 回 到 阳 世 , 定 取 龙 颜 转 帝 都 。
太 宗 听 到 这 话 , 就 把 他 接 在 手 中 , 把 他 笼 进 袖 子 里 , 于 是 闭 上 眼 睛 就 死 了 。
那 么 三 宫 六 院 、 皇 后 、 嫔 妃 、 侍 长 、 储 君 以 及 两 班 文 武 官 员 , 都 举 哀 戴 孝 。
又 在 白 虎 殿 上 , 停 留 在 梓 宫 不 题 。
我 最 终 不 知 道 太 宗 如 何 还 魂 , 暂 且 听 下 回 分 解 。
}\switchcolumn\flushpage  \begin{pinyinscope}{\myfontt \section{第一一回}     遊地府太宗還魂 進瓜果劉全續配

詩曰:
    百歲光陰似水流,一生事業等浮漚。
    昨朝面上桃花色,今日頭邊雪片浮。
    白蟻陣殘方是幻,子規聲切早回頭。
    古來陰騭能延壽,善不求憐天自周。

卻說太宗渺渺茫茫,魂靈徑出五鳳樓前,只見那御林軍馬,請大駕出朝採獵。太
宗忻然從之,縹渺而去。行了多時,人馬俱無。獨自一個,散步荒郊草野之間。
正驚惶難尋道路,只見那一邊,有一人高聲大叫道:「大唐皇帝,往這裏來,往
這裏來。」太宗聞言,抬頭觀看,只見那人:
頭頂烏紗,腰圍犀角。頭頂烏紗飄軟帶,腰圍犀角顯金廂。手擎牙笏凝祥靄,身
著羅袍隱瑞光。腳踏一雙粉底靴,登雲促霧﹔懷揣一本生死簿,注定存亡。鬢髮
蓬鬆飄耳上,鬍鬚飛舞繞腮旁。昔日曾為唐國相,如今掌案侍閻王。

太宗行到那邊,只見他跪拜路旁,口稱:「陛下,赦臣失誤遠迎之罪。」太宗問
曰:「你是何人?因甚事前來接拜?」那人道:「微臣半月前在森羅殿上,見涇
河鬼龍告陛下許救反誅之故,第一殿秦廣大王即差鬼使催請陛下,要三曹對案。
臣已知之,故來此間候接。不期今日來遲,望乞恕罪,恕罪。」太宗道:「你姓
甚名誰?是何官職?」那人道:「微臣存日,在陽曹侍先君駕前,為茲洲令,後
拜禮部侍郎,姓崔名?。今在陰司,得受酆都掌案判官。」太宗大喜,即近前,
御手忙攙道:「先生遠勞。朕駕前魏徵有書一封,正寄與先生,卻好相遇。」判
官謝恩,問書在何處。太宗即向袖中取出遞與。崔?拜接了,拆封而看。其書曰:
辱愛弟魏徵頓首書拜大都案契兄崔老先生臺下:憶昔交遊,音容如在。倏爾數載,
不聞清教。常只是遇節令,設蔬品奉祭,未卜享否?又承不棄,夢中臨示,始知
我兄長大人高遷。奈何陰陽兩隔,天各一方,不能面覿。今因我太宗文皇帝倏然
而故,料是對案三曹,必然得與兄長相會。萬祈俯念生日交情,方便一二,放我
陛下回陽,殊為愛也。容再修謝。不盡。

那判官看了書,滿心歡喜道:「魏人曹前日夢斬老龍一事,臣已早知,甚是誇獎
不盡。又蒙他早晚看顧臣的子孫,今日既有書來,陛下寬心,微臣管送陛下還陽
,重登玉闕。」太宗稱謝了。

二人正說間,只見那邊有一對青衣童子執幢幡、寶蓋,高叫道:「閻王有請,有
請。」太宗遂與崔判官並二童子舉步前進。忽見一座城,城門上掛著一面大牌,
上寫著「幽冥地府鬼門關」七個大金字。那青衣將幢幡搖動,引太宗徑入城中,
順街而走。只見那街傍邊有先主李淵、先兄建成、故弟元吉,上前道:「世民來
了,世民來了。」那建成、元吉就來揪打索命。太宗躲閃不及,被他扯住。幸有
崔判官喚一青面獠牙鬼使,喝退了建成、元吉,太宗方得脫身而去。行不數里,
見一座碧瓦樓臺,真個壯麗。但見:
    飄飄萬疊彩霞堆,隱隱千條紅霧現。
    耿耿簷飛怪獸頭,輝輝五疊鴛鴦片。
    門鑽幾路赤金釘,檻設一橫白玉段。
    牖近光放曉煙,簾櫳幌亮穿紅電。
    樓臺高聳接青霄,廊廡平排連寶院。
    獸鼎香雲襲御衣,絳紗燈火明宮扇。
    左邊猛烈擺牛頭,右下崢嶸羅馬面。
    接亡送鬼轉金牌,引魄招魂垂素練。
    喚作陰司總會門,下方閻老森羅殿。

太宗正在外面觀看,只見那壁廂環珮叮噹,仙香奇異,外有兩對提燭,後面卻是
十代閻王降階而至,是那十代閻君:秦廣王、初江王、宋帝王、仵官王、閻羅王
、平等王、泰山王、都市王、卞城王、轉輪王。十王出在森羅寶殿,控背躬身,
迎迓太宗。太宗謙下,不敢前行。十王道:「陛下是陽間人王,我等是陰間鬼王
,分所當然,何須過讓?」太宗道:「朕得罪麾下,豈敢論陰陽人鬼之道?」遜
之不已。太宗前行,徑入森羅殿上,與十王禮畢,分賓主坐定。

約有片時,秦廣王拱手而進言曰:「涇河鬼龍告陛下許救而反殺之,何也?」太
宗道:「朕曾夜夢老龍求救,實是允他無事。不期他犯罪當刑,該我那人曹官魏
徵處斬。朕宣魏徵在殿著棋,不知他一夢而斬。這是那人曹官出沒神機,又是那
龍王犯罪當死,豈是朕之過也?」十王聞言,伏禮道:「自那龍未生之前,南斗
星死簿上已註定該遭殺於人曹之手,我等早已知之。但只是他在此折辨,定要陛
下來此,三曹對案。是我等將他送入輪藏,轉生去了。今又有勞陛下降臨,望乞
恕我催促之罪。」言畢,命掌生死簿判官急取簿子來,看陛下陽壽天祿該有幾何
。崔判官急轉司房,將天下萬國國王天祿總簿,先逐一檢閱,只見南贍部洲大唐
太宗皇帝註定貞觀一十三年。崔判官吃了一驚,急取濃墨大筆,將「一」字上添
了兩畫,卻將簿子呈上。十王從頭一看,見太宗名下註定三十三年,閻王驚問:
「陛下登基多少年了?」太宗道:「朕即位,今一十三年了。」閻王道:「陛下
寬心勿慮,還有二十年陽壽。此一來已是對案明白,請返本還陽。」太宗聞言,
躬身稱謝。十閻王差崔判官、朱太尉二人,送太宗還魂。太宗出森羅殿,又起手
問十王道:「朕宮中老少安否如何?」十王道:「俱安,但恐御妹壽似不永。」
太宗又再拜啟謝:「朕回陽世,無物可酬謝,惟答瓜果而已。」十王喜曰:「我
處頗有東瓜、西瓜,只少南瓜。」太宗道:「朕回去即送來,即送來。」從此遂
相揖而別。

那太尉執一首引魂旛,在前引路。崔判官隨後保著太宗,徑出幽司。太宗舉目而
看,不是舊路,問判官曰:「此路差矣?」判官道:「不差。陰司裏是這般,有
去路,無來路。如今送陛下自轉輪藏出身,一則請陛下遊觀地府,一則教陛下轉
托超生。」太宗只得隨他兩個引路前來。

徑行數里,忽見一座高山,陰雲垂地,黑霧迷空。太宗道:「崔先生,那廂是甚
麼山?」判官道:「乃幽冥背陰山。」太宗悚懼道:「朕如何去得?」判官道:
「陛下寬心,有臣等引領。」太宗戰戰兢兢,相隨二人,上得山岩,抬頭觀看,
只見:形多凸凹,勢更崎嶇。峻如蜀嶺,高似廬巖。非陽世之名山,實陰司之險
地。荊棘叢叢藏鬼怪,石崖磷磷隱邪魔。耳畔不聞獸鳥噪,眼前惟見鬼妖行。陰
風颯颯,黑霧漫漫。陰風颯颯,是神兵口內哨來煙﹔黑霧漫漫,是鬼祟暗中噴出
氣。一望高低無景色,相看左右盡猖亡。那裏山也有,峰也有,嶺也有,洞也有
,澗也有﹔只是山不生草,峰不插天,嶺不行客,洞不納雲,澗不流水。岸前皆
魍魎,嶺下盡神魔,洞中收野鬼,澗底隱邪魂。山前山後,牛頭馬面亂喧呼﹔半
掩半藏,餓鬼窮魂時對泣。催命的判官,急急忙忙傳信票﹔追魂的太尉,吆吆喝
喝趲公文。急腳子,旋風滾滾﹔勾司人,黑霧紛紛。

太宗全靠著那判官保護,過了陰山。

前進又歷了許多衙門,一處處俱是悲聲振耳,惡怪驚心。太宗又道:「此是何處
?」判官道:「此是陰山背後一十八層地獄。」太宗道:「是那十八層?」判官
道:「你聽我說:
吊筋獄、幽枉獄、火坑獄,寂寂寥寥,煩煩惱惱,盡皆是生前作下千般業,死後
通來受罪名。酆都獄、拔舌獄、剝皮獄,哭哭啼啼,悽悽慘慘,只因不忠不孝傷
天理,佛口蛇心墮此門。磨捱獄、碓搗獄、車崩獄,皮開肉綻,抹嘴咨牙,乃瞞
心昧己不公道,巧語花言暗損人。寒冰獄、脫殼獄、抽腸獄,垢面蓬頭,愁眉皺
眼,都是大斗小秤欺痴蠢,致使災屯累自身。油鍋獄、黑暗獄、刀山獄,戰戰兢
兢,悲悲切切,皆因強暴欺良善,藏頭縮頸苦伶仃。血池獄、阿鼻獄、秤杆獄,
脫皮露骨,折臂斷筋,也只為謀財害命,宰畜屠生,墮落千年難解釋,沉淪永世
不翻身。一個個緊縛牢拴,繩纏索綁。差些赤髮鬼、黑臉鬼,長槍短劍﹔牛頭鬼
、馬面鬼,鐵簡銅鎚:只打得皺眉苦面血淋淋,叫地叫天無救應。正是:

人生卻莫把心欺,神鬼昭彰放過誰?善惡到頭終有報,只爭來早與來遲。」

太宗聽說,心中驚慘。

進前又走不多時,見一夥鬼卒各執幢幡,路傍跪下道:「橋梁使者來接。」判官
喝令起去,上前引著太宗,從金橋而過。太宗又見那一邊有一座銀橋,橋上行幾
個忠孝賢良之輩,公平正大之人,亦有幢幡接引﹔那壁廂又有一橋,寒風滾滾,
血浪滔滔,號泣之聲不絕。太宗問道:「那座橋是何名色?」判官道:「陛下,
那叫做奈河橋。若到陽間,切須傳記。那橋下都是些:
奔流浩浩之水,險峻窄窄之路。儼如疋練搭長江,卻似火坑浮上界。陰氣逼人寒
透骨,腥風撲鼻味鑽心。波翻浪滾,往來並沒渡人船﹔赤腳蓬頭,出入盡皆作業
鬼。橋長數里,闊只三?,高有百尺,深卻千重。上無扶手欄杆,下有搶人惡怪。
枷杻纏身,打上奈河險路。你看那橋邊神將甚兇頑,河內孽魂真苦惱。枒杈樹上
,掛的是青紅黃紫色絲衣﹔壁斗崖前,蹲的是毀罵公婆淫潑婦。銅蛇鐵狗任爭餐
,永墮奈河無出路。」
  詩曰:
    時聞鬼哭與神號,血水渾波萬丈高。
    無數牛頭並馬面,猙獰把守奈河橋。

正說間,那幾個橋梁使者早已回去了。太宗心又驚惶,點頭暗嘆,默默悲傷。相
隨著判官、太尉,早過了奈河惡水,血盆苦界。前又到枉死城,只聽哄哄人嚷,
分明說:「李世民來了,李世民來了。」太宗聽叫,心驚膽戰。見一夥拖腰折臂
、有足無頭的鬼魅,上前攔住﹔都叫道:「還我命來!還我命來!」慌得那太宗
藏藏躲躲,只叫:「崔先生救我!崔先生救我!」判官道:「陛下,那些人都是
那六十四處煙塵、七十二處草寇眾王子、眾頭目的鬼魂,盡是枉死的冤業,無收
無管,不得超生,又無錢鈔盤纏,都是孤寒餓鬼。陛下得些錢鈔與他,我才救得
哩。」太宗道:「寡人空身到此,卻那裏得有錢鈔?」判官道:「陛下,陽間有
一人,金銀若干,在我這陰司裏寄放。陛下可出名立一約,小判可作保,且借他
一庫,給散這些餓鬼,方得過去。」太宗問曰:「此人是誰?」判官道:「他是
河南開封府人氏,姓相名良,他有十三庫金銀在此。陛下若借用過他的,到陽間
還他便了。」太宗甚喜,情願出名借用。遂立了文書與判官,借他金銀一庫,著
太尉盡行給散。判官復吩咐道:「這些金銀,汝等可均分用度,放你大唐爺爺過
去,他的陽壽還早哩。我領了十王鈞語,送他還魂,教他到陽間做一個水陸大會
,度汝等超生,再休生事。」眾鬼聞言,得了金銀,俱唯唯而退。判官令太尉搖
動引魂幡,領太宗出離了枉死城中,奔上平陽大路,飄飄蕩蕩而去。

前進多時,卻來到六道輪迴之所。又見那騰雲的,身披霞帔﹔受籙的,腰掛金魚。
僧尼道俗,走獸飛禽,魑魅魍魎,滔滔都奔走那輪迴之下,各進其道。唐王問曰
:「此意何如?」判官道:「陛下明心見性,是必記了,傳與陽間人知。這喚做
六道輪迴:那行善的,昇化仙道﹔盡忠的,超生貴道﹔行孝的,再生福道﹔公平
的,還生人道﹔積德的,轉生富道﹔惡毒的,沉淪鬼道。」唐王聽說,點頭嘆曰
:「
    善哉真善哉,作善果無災。
    善心常切切,善道大開開。
    莫教興惡念,是必少刁乖。
    休言不報應,神鬼有安排。」

判官送唐王直至那超生貴道門,拜呼唐王道:「陛下呵,此間乃出頭之處,小判
告回,著朱太尉再送一程。」唐王謝道:「有勞先生遠踄。」判官道:「陛下到
陽間,千萬做個水陸大會,超度那無主的冤魂,切勿忘了。若是陰司裏無報怨之
聲,陽世間方得享太平之慶。凡百不善之處,俱可一一改過。普諭世人為善,管
教你後代綿長,江山永固。」

唐王一一准奏,辭了崔判官,隨著朱太尉,同入門來。那太尉見門裏有一匹海騮
馬,鞍?齊備,急請唐王上馬,太尉左右扶持。馬行如箭,早到了渭水河邊。只
見那水面上有一對金色鯉魚,在河裏翻波跳鬥。唐王見了心喜,兜馬貪看不捨。
太尉道:「陛下,趲動些,趁早趕時辰進城去也。」那唐王只管貪看,不肯前行
。被太尉撮著腳,高呼道:「還不走,等甚?」撲的一聲,望那渭河推下馬去。
卻就脫了陰司,徑回陽世。

卻說那唐朝駕下有徐茂公、秦叔寶、胡敬德、段志玄、馬三寶、程咬金、高士廉
、李世勣、房玄齡、杜如晦、蕭瑀、傅奕、張道源、張士衡、王珪等兩班文武,
俱保著那東宮太子與皇后、嬪妃、宮娥、侍長,都在那白虎殿上舉哀。一壁廂議
傳哀詔,要曉諭天下,欲扶太子登基。時有魏徵在傍道:「列位且住,不可,不
可。假若驚動州縣,恐生不測。且再按候一日,我主必還魂也。」下邊閃上許敬
宗道:「魏丞相言之甚謬。自古云:『潑水難收,人逝不返。』你怎麼還說這等
虛言,惑亂人心,是何道理?」魏徵道:「不瞞許先生說,下官自幼得授仙術,
推算最明,管取陛下不死。」

正講處,只聽得棺中連聲大叫道:「渰殺我耶!渰殺我耶!」諕得個文官武將心
慌,皇后嬪妃膽戰。一個個:面如秋後黃桑葉,腰似春前嫩柳條。儲君腳軟,難
扶喪杖盡哀儀﹔侍長魂飛,怎戴梁冠遵孝禮。嬪妃打跌,彩女欹斜。嬪妃打跌,
卻如狂風吹倒敗芙蓉﹔彩女欹斜,好似驟雨衝歪嬌菡萏。眾臣悚懼,骨軟筋麻。
戰戰兢兢,痴痴啞啞。把一座白虎殿,卻像斷梁橋﹔鬧喪臺,就如倒塌寺。

此時眾宮人走得精光,那個敢近靈扶柩。多虧了正直的徐茂公、理烈的魏丞相、
有膽量的秦瓊、忒猛撞的敬德,上前來扶著棺材,叫道:「陛下有甚麼放不下心
處,說與我等,不要弄鬼,驚駭了眷族。」魏徵道:「不是弄鬼,此乃陛下還魂
也。快取器械來。」打開棺蓋,果見太宗坐在裏面,還叫:「渰死我了!是誰救
撈?」茂公等上前扶起道:「陛下甦醒,莫怕,臣等都在此護駕哩。」唐王方才
開眼道:「朕適才好苦:躲過陰司惡鬼難,又遭水面喪身災。」眾臣道:「陛下
寬心勿懼,有甚水災來?」唐王道:「朕騎著馬,正行至渭水河邊,見雙頭魚戲
。被朱太尉欺心,將朕推下馬來,跌落河中,幾乎渰死。」魏徵道:「陛下鬼氣
尚未解。」急著太醫院進安神定魄湯藥,又安排粥膳。連服一二次,方才反本還
原,知得人事。一計唐王死去,已三晝夜,復回陽間為君。有詩為證:
    萬古江山幾變更,歷來數代敗和成。
    周秦漢晉多奇事,誰似唐王死復生?

當日天色已晚,眾臣請王歸寢,各各散訖。

次早,脫卻孝衣,換了彩服,一個個紅袍烏帽,一個個紫綬金章,在那朝門外等
候宣召。

卻說太宗自服了安神定魄之劑,連進了數次粥湯,被眾臣扶入寢室,一夜穩睡,
保養精神,直至天明方起,抖擻威儀,你看他怎生打扮:
戴一頂衝天冠,穿一領赭黃袍。繫一條藍田碧玉帶,踏一對創業無憂履。貌堂
堂,賽過當朝﹔威烈烈,重興今日。好一個清平有道的大唐王,起死回生的李陛
下。

唐王上金鑾寶殿,聚集兩班文武,山呼已畢,依品分班。只聽得傳旨道:「有事
出班來奏,無事退朝。」那東廂閃過徐茂公、魏徵、王珪、杜如晦、房玄齡、袁
天罡、李淳風、許敬宗等﹔西廂閃過殷開山、劉洪基、馬三寶、段志玄、程咬金
、秦叔寶、胡敬德、薛仁貴等,一齊上前,在白玉階前俯伏啟奏道:「陛下前朝
一夢,如何許久方覺?」太宗道:「日前接得魏徵書,朕覺神魂出殿,只見羽林
軍請朕出獵。正行時,人馬無蹤,又見那先君父王與先兄弟爭嚷。正難解處,見
一人烏帽皂袍,乃是判官崔?,喝退先兄弟。朕將魏徵書傳遞與他。正看時,又
見青衣者執幢幡,引朕入內,到森羅殿上,與十代閻王敘坐。他說那涇河龍誣告
我許救轉殺之事,是朕將前言陳具一遍。他說已三曹對過案了,急命取生死文簿
,檢看我的陽壽。時有崔判官傳上簿子,閻王看了,道寡人有三十三年天祿,過
得一十三年,還該我二十年陽壽,即著朱太尉、崔判官送朕回來。朕與十王作別
,允了送他瓜果謝恩。自出了森羅殿,見那陰司裏不忠不孝、非禮非義、作踐五
穀、明欺暗騙、大斗小秤、姦盜詐偽、淫邪欺罔之徒,受那些磨燒舂剉之苦,煎
熬弔剝之刑,有千千萬萬,看之不足。又過著枉死城中,有無數的冤魂,盡都是
六十四處煙塵的草寇、七十二處叛賊的魂靈,擋住了朕之來路。幸虧崔判官作保
,借得河南相老兒的金銀一庫,買轉鬼魂,方得前行。崔判官教朕回陽世,千萬
作一場水陸大會,超度那無主的孤魂,將此言叮嚀。分別出了那六道輪迴之下,
有朱太尉請朕上馬,飛也相似,行到渭水河邊,我看見那水面上有雙頭魚戲。正
歡喜處,他將我撮著腳,推下水中,朕方得還魂也。」眾臣聞此言,無不稱賀。
遂此編行傳報天下,各府縣官員上表稱慶不題。

卻說太宗又傳旨赦天下罪人。又查獄中重犯。時有審官將刑部絞斬罪人,查有四
百餘名呈上。太宗放赦回家,拜辭父母兄弟,託產與親戚子姪,明年今日赴曹,
仍領應得之罪。眾犯謝恩而退。又出恤孤榜文。又查宮中老幼彩女共有三千人,
出旨配軍。自此,內外俱善。有詩為證。詩曰:
    大國唐王恩德洪,道過堯舜萬民豐。
    死囚四百皆離獄,怨女三千放出宮。
    天下多官稱上壽,朝中眾宰賀元龍。
    善心一念天應佑,福蔭應傳十七宗。

太宗既放宮女,出死囚已畢,又出御製榜文,遍傳天下。榜曰:
乾坤浩大,日月照鑒分明﹔宇宙寬洪,天地不容姦黨。使心用術,果報只在今生
﹔善布淺求,獲福休言後世。千般巧計,不如本分為人﹔萬種強徒,怎似隨緣節
儉。心行慈善,何須努力看經﹔意欲損人,空讀如來一藏!

自此時,蓋天下無一人不行善者。一壁廂又出招賢榜,招人進瓜果到陰司裏去﹔
一壁廂將寶藏庫金銀一庫,差尉遲恭、胡敬德上河南開封府,訪相良還債。

榜張數日,有一赴命進瓜果的賢者,本是均州人,姓劉名全,家有萬貫之資。只
因妻李翠蓮在門首拔金釵齋僧,劉全罵了他幾句,說他不遵婦道,擅出閨門。李
氏忍氣不過,自縊而死。撇下一雙兒女年幼,晝夜悲啼。劉全又不忍見,無奈,
遂捨了性命,棄了家緣,撇了兒女,情願以死進瓜,將皇榜揭了,來見唐王。王
傳旨意,教他去金亭館裏,頭頂一對南瓜,袖帶黃錢,口噙藥物。

那劉全果服毒而死,一點魂靈,頂著瓜果,早到鬼門關上。把門的鬼使喝道:
「你是甚人,敢來此處?」劉全道:「我奉大唐太宗皇帝欽差,特進瓜果與十代
閻王受用的。」那鬼使欣然接引。劉全徑至森羅寶殿,見了閻王,將瓜果進上道
:「奉唐王旨意,遠進瓜果,以謝十王寬宥之恩。」閻王大喜道:「好一個有信
有德的太宗皇帝!」遂此收了瓜果。便問那進瓜的人姓名,那方人氏。劉全道:
「小人是均州城民籍。姓劉名全。因妻李氏縊死,撇下兒女,無人看管,小人情
願捨家棄子,捐軀報國,特與我王進貢瓜果,謝眾大王厚恩。」十王聞言,即命
查勘劉全妻李氏。那鬼使速取來在森羅殿下,與劉全夫妻相會。訴罷前言,回謝
十王恩宥。那閻王卻檢生死簿子看時,他夫妻們都有登仙之壽,急差鬼使送回。
鬼使啟上道:「李翠蓮歸陰日久,屍首無存,魂將何附?」閻王道:「唐御妹李
玉英今該促死,你可借他屍首,教他還魂去也。」那鬼使領命,即將劉全夫妻二
人還魂,同出陰司而去。
  
    畢竟不知夫妻二人如何還魂,且聽下回分解。





}  \end{pinyinscope}\switchcolumn{\myfontc \section{第 一 一 回} 游 地 府 , 太 宗 还 魂 进 瓜 果 , 刘 全 续 配 诗 道 : 百 岁 光 阴 似 水 流 , 一 生 事 业 等 浮 。
昨 天 早 晨 脸 上 桃 花 色 , 今 日 头 边 雪 片 飘 浮 。
白 蚁 阵 残 正 是 幻 , 子 规 声 切 早 回 头 。
自 古 以 来 阴 能 延 长 寿 命 , 善 不 求 , 怜 悯 上 天 自 然 周 全 。
又 说 太 宗 渺 渺 茫 茫 , 魂 灵 径 直 走 到 五 凤 楼 前 , 只 见 那 御 林 军 马 , 请 大 驾 出 朝 采 猎 。
太 宗 高 兴 地 听 从 了 他 的 话 , 飘 飘 而 去 。
走 了 很 长 时 间 , 人 和 马 都 没 有 了 。
他 独 自 一 个 人 , 在 荒 郊 野 外 散 步 。
正 惊 慌 难 以 寻 找 道 路 , 只 见 那 一 边 , 有 一 个 人 高 声 大 叫 道 : 大 唐 皇 帝 , 往 此 里 来 , 往 此 里 来 。
太 宗 听 了 这 话 , 抬 头 看 , 只 见 那 个 人 , 头 顶 乌 纱 , 腰 围 犀 角 。
头 顶 乌 纱 飘 动 的 软 带 , 腰 围 犀 角 显 现 在 金 厢 。
手 捧 牙 笏 凝 祥 云 , 身 穿 罗 袍 隐 瑞 光 。
脚 踏 着 一 双 粉 底 靴 , 登 上 云 层 雾 气 , 怀 揣 着 一 本 生 死 簿 , 注 定 存 亡 。
鬓 发 蓬 松 飘 动 在 耳 朵 上 , 胡 须 飞 舞 着 绕 着 腮 旁 。
从 前 我 曾 经 做 过 唐 朝 的 国 相 , 现 在 掌 管 案 卷 侍 奉 阎 王 。
太 宗 走 到 他 的 地 方 , 只 见 他 跪 在 路 旁 , 口 口 说 : 陛 下 赦 免 我 的 罪 过 , 远 远 地 迎 接 我 的 罪 过 。
唐 太 宗 问 道 : 你 是 什 么 人 , 因 为 有 什 么 事 前 来 接 见 , 那 个 人 说 : 我 半 个 月 前 在 森 罗 殿 上 , 见 到 泾 河 鬼 龙 告 诉 陛 下 答 应 救 助 反 被 诛 杀 的 原 因 , 第 一 殿 秦 广 大 王 就 派 遣 鬼 使 催 请 陛 下 , 要 三 曹 对 案 。
我 已 经 知 道 了 , 所 以 来 此 地 等 候 接 待 。
不 料 今 天 来 得 晚 , 希 望 宽 恕 我 的 罪 过 , 宽 恕 我 的 罪 过 。
太 宗 说 : 你 姓 什 么 名 字 叫 什 么 , 是 什 么 官 职 ? 那 人 说 : 我 在 我 的 时 候 , 在 阳 曹 侍 奉 先 君 驾 车 前 , 任 兹 洲 县 令 , 后 来 又 任 礼 部 侍 郎 , 姓 崔 名?
现 在 在 阴 司 , 得 以 接 受 都 掌 案 判 官 。
太 宗 大 喜 , 立 即 走 近 前 , 御 手 忙 忙 搀 着 说 : 先 生 远 劳 。
我 驾 车 前 , 魏 徵 有 一 封 信 , 正 寄 给 先 生 , 却 好 好 相 遇 。
判 官 谢 恩 , 问 : 书 信 在 哪 里 ?
太 宗 立 即 从 袖 中 取 出 来 递 给 太 宗 。
崔 某 拜 接 了 , 拆 开 封 好 看 。
他 的 信 是 : 辱 爱 弟 魏 徵 叩 头 书 拜 大 都 , 考 察 契 兄 崔 老 先 生 在 台 下 , 回 忆 昔 日 交 游 , 音 容 如 在 。
转 眼 之 间 几 年 , 没 有 听 说 清 教 。
常 常 只 是 遇 到 节 令 , 设 置 蔬 菜 祭 祀 , 不 知 道 享 受 没 有 , 又 承 认 不 舍 弃 , 梦 中 亲 临 示 示 , 才 知 道 我 哥 哥 长 大 人 升 迁 。
为 什 么 阴 阳 两 隔 , 天 各 有 一 方 , 不 能 面 见 。
现 在 因 为 我 的 太 宗 文 皇 帝 忽 然 而 死 , 估 计 是 我 对 着 三 曹 , 一 定 能 够 与 兄 长 相 会 。
万 祈 陛 下 俯 念 生 日 交 情 , 方 便 一 二 , 放 我 陛 下 回 阳 , 非 常 喜 爱 。
王 容 再 次 上 书 谢 恩 。
不 尽 。
那 判 官 看 了 信 , 满 心 欢 喜 地 说 : 魏 人 曹 前 日 梦 见 斩 老 龙 一 事 , 我 已 经 早 就 知 道 了 , 很 是 夸 奖 不 尽 。
又 蒙 蒙 他 早 晚 看 望 我 的 子 孙 , 今 天 既 然 有 书 信 来 , 陛 下 宽 心 , 不 管 我 送 陛 下 回 阳 , 重 登 玉 阙 。
太 宗 称 谢 。
二 人 正 在 说 话 时 , 只 见 那 边 有 一 对 青 衣 童 子 , 手 里 拿 着 幢 幡 、 宝 盖 , 高 声 喊 道 : 阎 王 有 请 , 有 请 。
太 宗 就 和 崔 判 官 和 两 个 童 子 一 起 步 行 前 进 。
忽 然 看 见 一 座 城 , 城 门 上 挂 着 一 面 大 牌 , 上 面 写 着 : 幽 冥 地 府 鬼 门 关 七 个 大 金 字 。
那 青 衣 人 将 旗 幡 摇 动 , 引 太 宗 径 直 进 入 城 中 , 顺 街 而 走 。
只 见 街 旁 有 先 主 李 渊 、 先 兄 李 建 成 、 老 弟 李 元 吉 , 上 前 说 : 世 民 来 了 , 世 民 来 了 。
建 成 、 元 吉 就 来 绑 打 索 命 。
太 宗 逃 避 不 及 , 被 他 拽 住 。
幸 好 有 崔 判 官 叫 来 一 个 青 脸 獠 牙 的 鬼 使 , 喝 退 了 李 建 成 、 李 元 吉 , 太 宗 才 得 以 脱 身 而 去 。
走 了 不 到 几 里 路 , 看 见 一 座 碧 瓦 楼 台 , 真 是 壮 丽 。
只 见 飘 飘 万 叠 彩 霞 堆 , 隐 隐 千 条 红 雾 出 现 。
灯 火 荧 荧 的 屋 檐 上 飞 出 怪 兽 头 , 辉 煌 五 叠 鸳 鸯 片 。
门 口 钻 几 路 红 金 钉 , 门 槛 设 置 一 个 横 着 白 玉 段 。
近 来 的 光 辉 放 出 晓 烟 , 帘 幕 闪 闪 闪 闪 , 穿 穿 红 色 的 闪 电 。
楼 台 高 耸 接 接 青 霄 , 廊 平 排 连 接 宝 院 。
兽 鼎 香 云 袭 御 衣 , 绛 纱 灯 火 明 宫 扇 。
左 边 猛 烈 摆 开 牛 头 , 右 边 罗 马 面 。
接 着 死 者 送 鬼 转 金 牌 , 引 魂 招 魂 垂 素 练 。
叫 做 阴 司 总 会 门 , 下 方 是 阎 老 森 罗 殿 。
唐 太 宗 正 在 外 面 观 看 , 只 见 那 墙 厢 房 里 的 玉 环 十 分 响 响 , 仙 香 十 分 奇 异 , 外 面 有 两 对 提 着 蜡 烛 , 后 面 却 是 十 代 阎 王 从 台 阶 上 走 来 的 , 是 秦 广 王 、 初 江 王 、 宋 帝 王 、 官 王 、 阎 罗 王 、 平 等 王 、 泰 山 王 、 都 市 王 、 卞 城 王 、 转 轮 王 。
十 王 出 来 在 森 罗 宝 殿 , 拉 着 背 子 亲 身 , 迎 接 太 宗 。
太 宗 谦 虚 , 不 敢 前 行 。
十 王 说 : 陛 下 是 阳 间 人 王 , 我 们 都 是 阴 间 鬼 王 , 我 们 都 是 阴 间 鬼 王 , 你 们 的 职 责 应 当 如 此 , 何 必 过 分 谦 让 呢 太 宗 说 : 我 得 罪 了 我 的 部 下 , 怎 么 敢 谈 论 阴 阳 人 鬼 的 道 理 呢 ?
太 宗 前 行 , 径 直 进 入 森 罗 殿 上 , 与 十 王 行 礼 完 毕 , 分 别 宾 主 坐 定 。
大 约 过 了 一 会 儿 , 秦 广 王 拱 手 进 言 说 : 泾 河 鬼 龙 告 诉 陛 下 允 许 救 援 而 反 而 杀 了 他 , 这 是 为 什 么 呢 太 宗 说 : 我 曾 经 夜 里 梦 见 老 龙 求 救 , 实 在 是 允 许 他 无 事 。
不 料 他 犯 罪 , 应 当 处 斩 , 应 该 我 的 曹 官 魏 徵 处 斩 。
我 宣 传 魏 徵 在 殿 上 下 棋 , 不 知 道 他 做 了 一 个 梦 就 被 杀 了 。
这 是 那 人 曹 官 出 没 神 机 , 又 是 那 龙 王 犯 罪 应 当 处 死 , 难 道 是 我 的 过 错 吗 ? 十 王 听 了 这 话 , 伏 礼 说 : 自 那 龙 还 没 有 出 生 之 前 , 南 斗 星 死 簿 上 已 经 确 定 该 被 人 曹 所 杀 , 我 等 早 就 知 道 了 。
但 是 只 是 他 在 这 里 审 理 辩 解 , 一 定 要 陛 下 来 此 , 三 曹 对 案 。
是 我 们 把 他 送 入 轮 藏 , 转 世 去 了 。
现 在 又 有 劳 陛 下 降 临 , 希 望 宽 恕 我 催 促 之 罪 。
说 完 , 命 令 掌 管 生 死 簿 的 判 官 急 忙 拿 出 账 子 来 , 看 看 陛 下 的 阳 寿 天 禄 该 有 多 少 ?
崔 判 官 急 忙 转 任 司 房 , 将 天 下 万 国 国 王 天 禄 的 总 簿 , 先 一 一 检 阅 , 只 见 南 赡 部 洲 大 唐 太 宗 皇 帝 注 定 贞 观 十 三 年 。
判 官 吃 了 一 惊 , 急 忙 拿 出 浓 墨 大 笔 , 把 一 个 字 上 添 了 两 个 画 , 又 拿 出 簿 子 呈 上 。
十 王 从 头 一 看 , 见 太 宗 名 字 下 有 三 十 三 年 了 。 阎 王 惊 问 : 陛 下 登 基 多 少 年 了 , 太 宗 说 : 我 即 位 , 现 在 已 经 十 三 年 了 。
阎 王 说 : 陛 下 宽 心 不 忧 虑 , 还 有 二 十 年 的 阳 寿 。
这 一 次 来 , 已 是 对 案 明 白 , 请 让 我 回 本 复 阳 。
太 宗 听 到 这 话 , 亲 自 称 谢 。
十 个 阎 王 派 遣 崔 判 官 、 朱 太 尉 二 人 , 送 太 宗 还 魂 。
太 宗 走 出 森 罗 殿 , 又 起 身 问 十 王 道 : 我 宫 中 老 少 安 全 不 是 怎 么 样 ? 十 王 说 : 都 很 安 , 只 恐 怕 皇 帝 妹 妹 的 寿 命 似 乎 不 长 。
太 宗 又 再 次 拜 谢 说 : 我 回 到 阳 世 , 没 有 什 么 东 西 可 以 酬 谢 , 只 是 回 答 瓜 果 而 已 。
十 王 高 兴 地 说 : 我 处 处 有 东 瓜 、 西 瓜 , 只 少 南 瓜 。
太 宗 说 : 我 回 去 就 送 来 , 就 送 来 。
从 此 就 互 相 作 揖 而 别 。
那 太 尉 拿 着 一 首 引 魂 , 在 前 面 引 路 。
崔 判 官 随 后 保 护 太 宗 , 径 直 出 宫 监 狱 。
太 宗 抬 头 看 , 不 是 旧 路 , 问 判 官 说 : 这 路 差 错 了 , 判 官 说 : 不 错 。
阴 司 里 就 是 这 样 , 有 去 路 , 没 有 来 路 。
如 今 送 陛 下 自 己 转 转 轮 藏 出 身 , 一 是 请 陛 下 游 览 地 府 , 一 是 教 陛 下 转 托 出 生 。
太 宗 只 得 随 他 两 个 人 引 路 前 来 。
走 了 几 里 路 , 忽 然 看 见 一 座 高 山 , 阴 云 覆 盖 在 地 上 , 黑 雾 迷 漫 在 空 中 。
太 宗 说 : 崔 先 生 , 那 厢 是 什 么 山 ? 判 官 说 : 是 幽 冥 背 阴 山 。
太 宗 惶 恐 地 说 : 我 怎 么 回 去 呢 判 官 说 : 陛 下 宽 心 , 有 我 们 引 领 。
太 宗 战 战 兢 兢 , 跟 着 两 个 人 , 上 了 一 个 山 岩 , 抬 头 观 看 , 只 见 山 岩 很 多 凹 陷 , 气 势 更 加 崎 岖 。
高 峻 如 蜀 岭 , 高 大 如 庐 山 岩 。
不 是 阳 世 的 名 山 , 而 是 阴 司 的 险 地 。
荆 棘 丛 生 鬼 怪 , 石 崖 深 邃 隐 藏 邪 魔 。
耳 边 听 不 到 野 兽 和 野 鸟 的 喧 闹 声 音 , 眼 前 只 看 见 鬼 妖 的 行 动 。
阴 风 飒 飒 , 黑 雾 漫 漫 。
阴 风 飒 飒 , 是 神 兵 口 内 哨 来 的 烟 火 ; 黑 雾 漫 漫 , 是 鬼 魅 在 暗 中 喷 出 的 云 气 。
一 望 高 低 没 有 景 色 , 互 相 观 看 左 右 的 人 都 纷 纷 逃 亡 。
那 里 的 山 也 有 , 山 也 有 , 山 也 有 , 岭 也 有 , 洞 也 有 , 涧 也 有 , 只 是 山 不 生 草 , 山 不 插 天 , 岭 不 行 客 , 洞 不 纳 云 , 涧 不 流 水 。
岸 前 都 是 魍 , 岭 下 尽 是 神 魔 , 洞 中 收 罗 野 鬼 , 涧 底 隐 匿 邪 魂 。
山 前 山 后 , 牛 头 马 面 纷 纷 喧 呼 , 半 掩 半 藏 , 饿 鬼 穷 魂 时 常 相 对 哭 泣 。
催 命 的 判 官 , 急 急 忙 忙 传 递 信 票 ; 追 魂 的 太 尉 , 呵 呵 喝 公 文 。
急 脚 子 , 旋 风 滚 滚 ; 勾 司 人 , 黑 雾 纷 纷 。
太 宗 完 全 靠 着 那 判 官 的 保 护 , 过 了 阴 山 。
前 进 又 经 过 许 多 衙 门 , 一 处 地 都 是 悲 声 震 耳 , 恶 怪 惊 心 。
太 宗 又 说 : 这 是 什 么 地 方 ? 判 官 说 : 这 是 阴 山 背 后 的 十 八 层 地 狱 。
判 官 说 : 你 听 我 说 : 吊 筋 狱 、 幽 冤 狱 、 火 坑 狱 , 寂 寞 寥 寥 , 烦 恼 烦 恼 , 都 是 生 前 做 下 千 种 种 种 种 种 种 种 种 种 种 种 种 种 种 种 种 种 种 种 种 种 种 种 种 种 种 种 种 种 种 种 种 种 种 种 种 种 种 种 种 种 种 种 种 种 种 种 种 种 种 种 种 种 种 种 种 种 种 种 种 种 种 事 业 , 死 后 全 都 来 受 罪 的 名 号 。
都 狱 、 拔 舌 狱 、 剥 皮 狱 , 哭 哭 啼 啼 , 悲 哀 惨 惨 , 只 因 为 不 忠 不 孝 损 害 了 天 理 , 佛 口 蛇 心 堕 入 这 个 门 。
磨 擦 狱 、 碓 捣 狱 、 车 崩 狱 , 皮 开 肉 裂 , 抹 嘴 咨 牙 , 竟 是 欺 心 昧 己 , 不 公 道 , 巧 语 花 言 , 暗 损 人 。
寒 冰 狱 、 脱 壳 狱 、 抽 肠 狱 , 垢 面 蓬 头 , 愁 眉 憔 悴 , 都 是 大 斗 小 秤 欺 骗 痴 呆 , 致 使 灾 祸 拖 累 自 身 。
油 锅 狱 、 黑 暗 狱 、 刀 山 狱 , 他 们 战 战 兢 兢 , 悲 哀 切 切 , 都 是 因 为 强 暴 欺 骗 良 善 , 藏 头 缩 颈 苦 伶 仃 。
血 池 狱 、 阿 鼻 狱 、 秤 杆 狱 , 脱 皮 露 骨 , 折 臂 断 筋 , 也 只 是 为 了 谋 财 害 命 , 宰 畜 屠 生 , 堕 落 千 年 难 以 解 脱 , 沉 沦 永 世 不 能 翻 身 。
一 个 个 捆 紧 紧 的 绳 子 , 绳 子 缠 着 绳 索 。
还 有 一 个 红 头 鬼 、 黑 脸 鬼 , 长 枪 短 剑 ; 牛 头 鬼 、 马 面 鬼 , 铁 简 铜 锤 , 只 打 得 皱 眉 苦 面 血 淋 淋 , 叫 地 叫 天 没 有 救 应 。
正 是 说 : 人 生 却 不 要 把 心 欺 骗 , 神 鬼 昭 彰 , 放 过 谁 , 善 恶 到 头 终 有 报 , 只 争 得 早 与 来 迟 。
太 宗 听 了 , 心 中 十 分 惊 恐 。
进 前 又 走 了 不 多 时 , 看 见 一 伙 鬼 卒 各 自 手 持 幢 幡 , 在 路 旁 跪 下 说 : 桥 梁 使 者 来 接 你 。
判 官 喝 令 起 身 离 去 , 皇 上 上 前 领 着 太 宗 , 从 金 桥 过 去 。
太 宗 又 看 见 那 一 边 有 一 座 银 桥 , 桥 上 有 几 个 忠 孝 贤 良 之 辈 , 公 平 正 大 之 人 , 也 有 幢 幡 接 引 , 那 墙 边 又 有 一 座 桥 , 寒 风 滚 滚 , 血 浪 滔 滔 , 号 哭 之 声 不 绝 。
太 宗 问 道 : 那 座 桥 是 什 么 名 字 ? 判 官 说 : 陛 下 , 那 叫 奈 河 桥 。
如 果 到 了 阳 间 , 必 须 传 记 。
桥 下 都 是 : 奔 流 浩 浩 的 水 , 险 峻 狭 窄 的 路 。
俨 然 像 练 搭 在 长 江 , 却 像 火 坑 浮 在 上 界 。
阴 气 逼 人 , 寒 冷 透 骨 , 腥 风 扑 鼻 , 味 道 深 入 心 。
波 浪 翻 翻 浪 花 , 往 来 都 没 有 渡 人 的 船 , 赤 脚 蓬 头 蓬 头 , 出 入 都 是 作 业 鬼 。
桥 长 数 里 , 宽 只 有 三 尺 , 高 有 百 尺 , 深 有 千 重 。
上 面 没 有 扶 手 栏 杆 , 下 面 有 抢 人 恶 怪 。
枷 锁 缠 身 , 打 上 河 道 险 路 。
你 看 那 桥 边 的 神 将 很 凶 顽 , 河 内 的 妖 魂 真 苦 恼 。
树 上 挂 的 是 青 、 红 、 黄 、 紫 色 的 丝 绸 衣 服 ; 壁 斗 崖 前 蹲 的 是 毁 骂 公 婆 淫 耍 的 妇 女 。
铜 蛇 铁 狗 任 凭 争 抢 食 物 , 永 远 堕 入 伊 河 , 没 有 出 路 。
《 诗 经 》 说 : 时 闻 鬼 哭 与 神 号 , 血 水 浑 波 万 丈 高 。
无 数 牛 头 和 马 面 , 凶 狠 地 把 守 河 桥 。
正 说 之 间 , 那 几 个 桥 梁 使 者 早 已 回 去 了 。
太 宗 心 又 惊 慌 , 点 头 暗 叹 , 默 默 悲 伤 。
相 随 着 判 官 、 太 尉 , 很 早 就 过 了 了 奈 河 恶 水 , 血 盆 苦 境 。
先 前 又 到 枉 死 城 , 只 听 喧 闹 闹 闹 的 人 吵 闹 , 分 明 说 : 李 世 民 来 了 , 李 世 民 来 了 。
太 宗 听 到 他 的 叫 声 , 心 惊 胆 战 。
看 见 一 伙 拖 腰 折 臂 、 有 脚 无 头 的 鬼 魅 , 上 前 拦 住 他 们 , 都 喊 道 : 还 我 命 来 , 还 我 命 来 , 还 我 命 来 , 惊 得 那 太 宗 藏 匿 躲 避 , 只 叫 道 : 崔 先 生 救 我 , 崔 先 生 救 我 ! 判 官 说 : 陛 下 , 那 些 人 都 是 六 十 四 处 烟 尘 、 七 十 二 处 草 寇 众 王 子 、 众 头 目 的 鬼 魂 , 都 是 冤 枉 的 冤 业 , 无 收 无 管 , 不 得 超 生 , 又 没 钱 钞 盘 缠 。
陛 下 得 到 一 些 钱 钞 给 他 , 我 才 救 得 了 吗 ?
太 宗 说 : 我 空 身 来 到 这 里 , 哪 里 得 到 钱 钞 呢 判 官 说 : 陛 下 , 阳 间 有 一 个 人 , 金 银 若 干 , 在 我 这 个 阴 司 里 寄 放 。
陛 下 可 以 拿 出 名 字 立 下 一 个 盟 约 , 小 判 可 以 作 保 , 暂 且 借 给 他 一 个 仓 库 , 给 他 们 分 散 这 些 饿 鬼 , 这 样 才 能 够 回 去 。
判 官 说 : 他 是 河 南 开 封 府 人 , 姓 相 名 良 , 他 有 十 三 库 金 银 在 这 里 。
陛 下 如 果 借 用 过 他 的 , 到 阳 间 还 他 就 可 以 了 。
太 宗 很 高 兴 , 愿 意 出 名 借 用 。
于 是 就 写 了 文 书 给 判 官 , 借 给 他 一 库 金 银 , 让 太 尉 全 部 发 给 。
判 官 又 指 点 道 : 这 些 金 银 , 你 们 可 以 均 分 开 用 度 , 放 你 们 大 唐 爷 爷 过 去 , 他 的 寿 命 还 早 么 !
我 领 了 十 个 王 钧 的 话 , 送 他 还 魂 , 让 他 到 阳 间 做 一 个 水 陆 大 会 , 让 你 们 超 生 , 再 休 生 事 。
众 鬼 听 了 , 得 到 了 金 银 , 都 唯 唯 地 退 下 。
判 官 命 令 太 尉 摇 动 引 魂 幡 , 领 太 宗 出 去 , 冤 死 在 城 中 , 奔 上 平 阳 的 大 路 , 飘 飘 荡 荡 地 离 去 。
往 前 走 了 很 多 时 间 , 又 来 到 了 六 道 轮 回 的 地 方 。
又 见 那 腾 云 的 人 , 身 披 霞 ; 接 受 贿 赂 的 人 , 腰 挂 金 鱼 。
僧 尼 、 道 俗 、 走 兽 、 飞 禽 、 魅 、 魍 , 纷 纷 奔 走 在 那 轮 回 之 下 , 各 自 进 行 自 己 的 道 路 。
唐 王 问 他 说 : 这 个 意 思 是 怎 么 样 的 呢 ? 判 官 说 : 陛 下 明 心 见 性 , 这 一 定 要 记 住 了 , 传 给 阳 间 人 知 道 。
六 道 之 道 , 其 道 也 , 其 道 也 , 其 道 也 , 其 道 也 , 其 道 也 , 其 道 也 , 其 道 也 。
唐 王 听 了 后 , 点 头 叹 道 : 好 啊 真 是 善 啊 , 做 好 事 果 然 没 有 灾 祸 。
善 心 常 切 切 , 善 道 大 开 。
不 要 教 他 产 生 恶 念 , 这 一 定 会 少 有 刁 错 。
不 要 说 没 有 报 应 , 神 鬼 也 会 有 安 排 。
判 官 送 唐 王 直 到 那 超 生 贵 道 门 , 叩 头 呼 叫 唐 王 说 : 陛 下 呵 , 此 地 是 出 头 的 地 方 , 小 判 告 回 来 , 朱 太 尉 再 送 一 路 。
唐 王 谢 罪 说 : 有 劳 先 生 远 去 吧 。
判 官 说 : 陛 下 到 阳 间 , 千 万 要 做 水 陆 大 会 , 超 度 那 无 主 的 冤 魂 , 切 勿 忘 记 。
如 果 是 阴 司 里 没 有 报 仇 之 声 , 阳 世 间 才 能 享 受 太 平 之 庆 。
凡 是 百 种 不 善 的 地 方 , 都 可 以 一 一 改 过 。
普 告 世 人 为 善 , 管 教 你 们 的 后 代 绵 延 , 江 山 永 远 巩 固 。
唐 王 一 一 准 备 上 奏 , 辞 去 崔 判 官 , 随 着 朱 太 尉 一 起 进 门 来 。
那 太 尉 看 见 门 里 有 一 匹 海 马 , 马 鞍 和 马 鞍 齐 备 , 急 忙 请 唐 王 上 马 , 太 尉 左 右 扶 着 他 。
马 行 如 箭 , 早 就 到 了 渭 水 河 边 。
只 见 水 面 上 有 一 对 金 色 的 鲤 鱼 , 在 河 里 翻 波 跳 跃 。
唐 王 见 了 , 心 里 很 高 兴 , 兜 马 贪 看 不 舍 。
太 尉 说 : 陛 下 , 暂 且 动 一 点 , 趁 早 进 城 去 吧 。
唐 王 只 管 贪 看 , 不 肯 向 前 走 。
他 被 太 尉 撮 着 脚 , 高 喊 道 : 还 不 走 , 等 什 么 呀 扑 的 一 声 , 望 着 那 渭 河 推 下 马 去 。
后 来 , 他 就 脱 离 了 阴 司 , 径 直 回 到 了 阳 世 。
又 说 : 唐 朝 皇 帝 驾 驭 下 有 徐 茂 公 、 秦 叔 宝 、 胡 敬 德 、 段 志 玄 、 马 三 宝 、 程 咬 金 、 高 士 廉 、 李 世 、 房 玄 龄 、 杜 如 晦 、 萧 瑀 、 傅 奕 、 张 道 源 、 张 士 衡 、 王 珪 等 两 班 文 武 官 员 , 都 保 护 着 东 宫 太 子 与 皇 后 、 嫔 妃 、 宫 娥 、 侍 长 , 都 在 白 虎 殿 上 举 哀 。
一 壁 厢 议 传 达 哀 诏 , 要 晓 谕 天 下 , 想 扶 太 子 登 基 。
当 时 有 个 魏 徵 在 旁 边 说 : 列 位 暂 且 住 下 , 不 行 , 不 行 。
假 如 惊 动 州 县 , 恐 怕 会 发 生 不 测 之 事 。
再 按 候 一 天 , 我 的 主 人 一 定 会 还 魂 。
下 边 闪 上 去 , 许 敬 宗 说 : 魏 丞 相 说 得 很 荒 谬 。
自 古 以 来 说 : 洒 水 难 以 收 , 人 逝 不 返 。
你 为 什 么 还 说 这 些 虚 妄 的 话 , 蛊 惑 人 心 , 是 什 么 道 理 ? 魏 徵 说 : 不 骗 许 先 生 说 , 我 从 小 就 得 到 传 授 了 仙 术 , 推 算 最 明 , 只 要 陛 下 不 死 。
正 在 讲 话 的 地 方 , 只 听 到 棺 材 里 连 声 大 叫 道 : 杀 我 呀 , 杀 我 啊 ! 得 个 文 官 武 将 心 慌 , 皇 后 、 妃 嫔 、 妃 嫔 都 胆 战 。
一 个 个 人 说 : 脸 如 秋 后 黄 桑 叶 , 腰 似 春 前 嫩 柳 条 。
太 子 脚 软 , 难 以 扶 丧 , 拄 丧 杖 尽 哀 , 侍 长 魂 飞 , 何 必 戴 梁 冠 遵 孝 礼 。
妃 嫔 妃 子 打 坏 了 , 彩 女 女 儿 都 斜 着 了 。
嫔 妃 打 坏 了 , 却 好 像 狂 风 吹 倒 了 芙 蓉 ; 美 女 倾 斜 , 好 像 暴 雨 冲 倒 了 荷 花 。
众 臣 恐 惧 , 骨 肉 软 弱 , 筋 骨 麻 麻 。
战 战 兢 兢 , 痴 呆 哑 哑 。
把 一 座 白 虎 殿 , 却 像 断 梁 桥 一 样 , 闹 闹 丧 台 , 就 像 倒 塌 寺 。
这 时 宫 女 们 走 得 精 光 , 哪 个 胆 敢 靠 近 灵 柩 。
多 亏 是 正 直 的 徐 茂 公 , 善 于 理 事 的 魏 丞 相 , 有 胆 量 的 秦 琼 、 太 猛 撞 的 敬 德 , 上 前 来 扶 着 棺 材 , 喊 道 : 陛 下 有 什 么 放 不 下 心 的 地 方 , 说 与 我 等 人 , 不 要 弄 鬼 , 惊 骇 了 你 的 家 族 。
魏 徵 说 : 不 是 弄 鬼 , 这 是 陛 下 还 魂 。
快 拿 出 武 器 来 。
他 打 开 棺 盖 , 果 然 看 见 太 宗 坐 在 里 面 , 还 喊 道 : 死 我 了 , 是 谁 救 我 ? 茂 公 等 人 上 前 扶 起 他 说 : 陛 下 醒 过 来 , 不 要 害 怕 , 我 们 都 在 这 里 护 驾 啊 。
唐 王 方 才 睁 开 眼 睛 说 : 我 刚 才 好 苦 , 逃 避 了 阴 司 恶 鬼 的 灾 难 , 又 遭 到 水 面 丧 身 的 灾 难 。
大 臣 们 说 : 陛 下 宽 心 不 害 怕 , 有 什 么 水 灾 来 呢 唐 王 说 : 我 骑 着 马 , 正 走 到 渭 水 河 边 , 看 见 双 头 鱼 游 戏 。
我 被 朱 太 尉 欺 骗 我 , 把 我 推 下 马 来 , 跌 落 到 河 中 , 几 乎 饿 死 。
魏 徵 说 : 陛 下 的 鬼 气 还 没 有 消 除 。
他 急 忙 到 太 医 院 进 奉 安 神 定 魄 汤 药 , 又 安 排 粥 膳 。
连 服 一 两 次 , 才 返 本 复 原 , 知 道 了 人 事 。
一 想 , 唐 王 已 经 死 了 , 已 经 三 天 三 夜 了 , 又 回 到 阳 间 做 了 君 主 。
有 诗 作 证 : 万 古 江 山 几 变 更 , 历 来 数 代 败 和 成 。
周 、 秦 、 汉 、 晋 多 奇 事 , 谁 能 比 唐 王 死 又 复 生 ? 当 天 天 色 已 晚 , 众 臣 请 大 王 回 到 寝 宫 , 各 自 散 开 。
第 二 天 早 晨 , 脱 去 孝 服 , 换 上 彩 服 , 一 个 个 红 袍 乌 帽 , 一 个 个 紫 绶 金 章 , 在 朝 门 外 等 候 宣 召 。
又 说 : 太 宗 自 己 服 用 了 安 神 定 魄 的 药 , 接 连 进 了 几 次 粥 汤 , 被 众 臣 扶 着 进 寝 室 , 一 夜 安 睡 , 保 养 精 神 , 直 到 天 亮 才 起 来 , 振 奋 威 仪 , 你 看 他 怎 么 样 的 打 扮 ? 戴 一 顶 冲 天 冠 , 穿 一 领 赭 黄 袍 。
系 一 条 蓝 田 碧 玉 带 , 踏 一 对 创 业 无 忧 履 。
仪 表 堂 堂 , 超 过 当 朝 ; 威 严 赫 赫 , 重 兴 今 日 。
好 一 个 清 平 有 道 的 大 唐 王 , 起 死 回 生 的 李 陛 下 。
唐 王 登 上 金 銮 宝 殿 , 聚 集 了 两 班 文 武 官 员 , 山 呼 已 经 完 毕 , 按 照 品 级 分 班 。
只 听 到 传 旨 说 : 有 事 出 班 来 奏 , 无 事 退 朝 。
那 东 厢 闪 过 徐 茂 公 、 魏 征 、 王 珪 、 杜 如 晦 、 房 玄 龄 、 袁 天 雹 、 李 淳 风 、 许 敬 宗 等 人 ; 西 厢 闪 过 殷 开 山 、 刘 洪 基 、 马 三 宝 、 段 志 玄 、 程 咬 金 、 秦 叔 宝 、 胡 敬 德 、 薛 仁 贵 等 人 , 一 齐 上 前 , 在 白 玉 阶 前 俯 伏 启 奏 道 : 陛 下 前 朝 一 梦 , 为 什 么 许 久 才 醒 过 太 宗 说 : 昨 日 接 到 魏 徵 的 信 , 朕 觉 得 神 魂 出 殿 。
正 走 的 时 候 , 人 马 没 有 踪 迹 , 又 看 见 那 先 君 父 王 和 先 兄 弟 争 吵 。
正 在 难 解 的 地 方 , 忽 然 看 见 一 个 穿 着 黑 色 帽 子 穿 着 黑 色 袍 子 , 原 来 是 判 官 崔 , 喝 退 了 先 兄 弟 。
我 将 魏 徵 的 书 信 传 递 给 别 人 。
正 在 看 时 , 又 看 见 一 个 穿 青 衣 服 的 人 手 持 幢 幡 , 领 我 进 入 内 室 , 到 森 罗 殿 上 , 和 十 代 阎 王 叙 坐 。
又 曰 : 泾 河 龙 诬 告 我 许 救 转 杀 之 事 , 是 我 将 前 言 陈 述 了 一 遍 。
他 说 : 已 经 三 曹 对 过 了 , 急 忙 命 令 取 来 生 死 文 簿 , 检 看 我 的 阳 寿 。
当 时 有 个 崔 判 官 送 上 簿 子 , 阎 王 看 了 , 说 : 我 有 三 十 三 年 的 天 禄 , 过 了 十 三 年 , 还 给 我 二 十 年 的 阳 寿 , 立 即 写 着 朱 太 尉 、 崔 判 官 送 我 回 来 。
我 与 十 王 告 别 , 允 许 送 他 瓜 果 谢 恩 。
自 从 出 了 森 罗 殿 , 见 到 那 些 阴 司 里 的 不 忠 不 孝 、 非 礼 不 义 、 作 践 五 谷 、 明 欺 暗 欺 、 大 斗 、 小 秤 、 奸 盗 、 狡 诈 、 淫 邪 欺 骗 、 淫 邪 欺 骗 之 徒 , 受 到 那 些 磨 打 、 舂 打 、 拷 打 、 拷 打 、 拷 打 、 拷 打 的 刑 罚 , 有 千 千 万 万 , 看 不 够 。
又 过 于 城 中 , 有 无 数 的 冤 魂 , 全 都 是 六 十 四 处 烟 尘 的 草 寇 , 七 十 二 处 叛 贼 的 魂 魄 , 阻 住 我 的 来 路 。
幸 亏 崔 判 官 作 为 保 护 , 借 了 河 南 相 老 儿 的 金 银 一 库 , 买 转 了 鬼 魂 , 才 得 以 前 行 。
崔 判 官 教 我 回 到 阳 世 , 千 万 作 一 场 水 陆 大 会 , 超 度 那 无 主 的 孤 魂 , 将 此 话 告 诫 。
分 别 在 六 道 轮 回 之 下 , 有 个 叫 朱 太 尉 请 我 上 马 , 飞 也 很 相 似 , 走 到 渭 水 河 边 , 我 看 见 水 面 上 有 双 头 鱼 戏 。
正 在 欢 喜 之 处 , 他 将 我 撮 着 脚 , 推 下 水 中 , 我 才 得 以 还 魂 。
大 臣 们 听 了 这 话 , 无 不 称 贺 。
于 是 这 些 编 行 , 传 报 天 下 , 各 府 县 官 员 上 表 称 庆 不 题 。
又 劝 说 太 宗 , 又 传 旨 赦 免 天 下 罪 人 。
又 查 办 狱 中 重 罪 犯 人 。
当 时 有 审 官 将 刑 部 斩 首 的 罪 人 , 查 出 四 百 多 名 呈 上 来 。
太 宗 放 赦 回 家 , 拜 辞 父 母 兄 弟 , 托 产 给 亲 戚 子 侄 , 第 二 年 今 天 赴 曹 , 仍 然 承 认 应 得 的 罪 行 。
众 犯 谢 恩 而 退 。
又 发 出 抚 恤 孤 儿 的 榜 文 。
又 查 核 宫 中 老 幼 彩 女 共 有 三 千 人 , 皇 帝 下 旨 让 他 们 配 给 军 队 。
自 此 以 后 , 内 外 都 很 好 。
有 诗 作 证 。
《 诗 经 》 上 说 : 大 国 唐 王 恩 德 广 大 , 道 路 超 过 尧 舜 , 万 民 丰 富 。
死 囚 四 百 人 都 离 开 监 狱 , 怨 女 三 千 人 被 放 出 宫 中 。
天 下 多 官 称 上 寿 , 朝 中 众 宰 祝 贺 元 龙 。
善 心 一 念 , 上 天 就 会 保 佑 , 福 荫 应 该 传 给 十 七 宗 。
太 宗 已 经 放 出 宫 女 , 释 放 死 囚 已 经 完 毕 , 又 拿 出 御 制 的 榜 文 , 遍 布 天 下 。
榜 文 说 : 天 地 浩 大 , 日 月 照 耀 分 明 , 宇 宙 宽 宏 , 天 地 不 容 奸 党 。
如 果 用 心 用 法 术 , 果 报 只 在 今 生 ; 善 施 浅 求 , 获 得 福 报 , 不 要 再 说 后 世 。
千 般 巧 计 , 不 如 本 分 为 人 , 万 种 强 徒 , 怎 么 似 随 缘 节 俭 。
心 里 行 慈 善 , 何 必 要 努 力 看 经 呢 ? 想 要 损 害 人 , 就 空 读 如 来 一 藏 菩 萨 。 从 这 个 时 候 , 全 国 没 有 一 个 人 不 做 善 的 。
一 个 墙 厢 又 发 出 招 贤 榜 , 招 人 进 瓜 果 到 阴 司 里 去 , 另 一 个 墙 厢 将 宝 藏 库 金 银 一 库 , 派 尉 迟 恭 、 胡 敬 德 到 河 南 开 封 府 去 , 访 问 相 良 还 债 。
张 榜 几 天 , 有 一 个 奉 命 进 瓜 果 的 贤 人 , 原 是 均 州 人 , 姓 刘 名 全 , 家 中 有 万 贯 的 资 产 。
只 因 妻 子 李 翠 莲 在 门 口 拔 金 钗 斋 僧 , 刘 全 骂 了 他 几 句 , 说 他 不 遵 妇 道 , 擅 自 出 家 门 。
李 氏 忍 气 不 过 , 自 刎 而 死 。
他 们 把 一 双 儿 女 都 很 小 , 日 夜 悲 啼 。
刘 全 又 不 忍 心 看 见 , 无 可 奈 何 , 就 舍 弃 性 命 , 抛 弃 家 眷 , 抛 弃 儿 女 , 愿 意 以 死 进 瓜 , 将 皇 榜 揭 开 , 来 见 唐 王 。
阎 王 传 旨 , 让 他 去 金 亭 馆 里 , 头 顶 有 一 对 南 瓜 , 袖 中 带 着 黄 钱 , 口 中 含 着 药 物 。
那 刘 全 果 然 服 毒 而 死 , 一 点 魂 灵 , 顶 着 瓜 果 , 早 到 鬼 门 关 上 。
把 门 的 鬼 使 喝 道 : 你 是 个 什 么 人 , 竟 敢 来 此 地 ? 刘 全 说 : 我 奉 大 唐 太 宗 皇 帝 的 钦 诏 , 特 进 瓜 果 与 十 代 阎 王 受 用 的 。
那 个 鬼 使 欣 然 接 见 。
刘 全 径 直 到 森 罗 宝 殿 , 见 阎 王 , 把 瓜 果 送 给 皇 上 , 说 : 奉 唐 王 的 旨 意 , 远 送 瓜 果 , 以 谢 十 王 宽 恕 之 恩 。
阎 王 非 常 高 兴 地 说 : 好 一 个 人 有 信 用 有 德 行 的 太 宗 皇 帝 于 是 收 拾 了 瓜 果 。
便 问 那 进 瓜 的 人 姓 名 , 那 方 人 姓 。
刘 全 道 说 : 小 人 是 均 州 城 民 籍 。
姓 刘 名 全 。
因 为 妻 子 李 氏 死 了 , 抛 下 儿 女 , 没 有 人 看 管 , 小 人 情 愿 舍 家 弃 子 , 捐 躯 报 国 , 特 地 与 我 王 进 贡 瓜 果 , 感 谢 大 王 的 厚 恩 。
十 王 听 到 这 话 , 立 即 命 令 查 核 刘 全 的 妻 子 李 氏 。
那 个 鬼 让 他 快 点 拿 来 , 在 森 罗 殿 下 , 与 刘 全 夫 妻 相 会 。
他 说 罢 以 前 的 话 , 回 头 谢 十 王 的 恩 赦 。
阎 王 又 检 查 生 死 簿 , 看 看 他 的 夫 妻 都 有 登 仙 的 寿 命 , 急 忙 派 遣 鬼 使 送 回 去 。
鬼 使 向 皇 上 报 告 说 : 李 翠 莲 回 到 阴 间 已 经 很 久 了 , 尸 首 无 存 , 魂 将 依 附 于 什 么 ? 阎 王 说 : 唐 御 妹 李 玉 英 现 在 该 快 死 了 , 你 可 以 借 他 的 尸 首 , 让 她 还 魂 去 。
那 个 鬼 让 他 领 命 , 就 把 刘 全 夫 妻 二 人 送 回 灵 魂 , 一 同 出 了 阴 司 而 去 。
我 最 终 不 知 道 夫 妻 二 人 如 何 还 魂 , 暂 且 听 我 下 回 分 析 。
}\switchcolumn\flushpage  \begin{pinyinscope}{\myfontt \section{第一二回}     唐王秉誠修大會 觀音顯聖化金蟬

卻說鬼使同劉全夫妻二人出了陰司,那陰風遶遶,徑到了長安大國,將劉全的魂
靈推入金亭館裏,將翠蓮的靈魂帶進皇宮內院。只見那玉英宮主正在花陰下,徐
步綠苔而行,被鬼使撲個滿懷,推倒在地,活捉了他魂,卻將翠蓮的魂靈推入玉
英身內。鬼使回轉陰司不題。

卻說宮院中的大小侍婢見玉英跌死,急走金鑾殿,報與三宮皇后道:「宮主娘娘
跌死也。」皇后大驚,隨報太宗。太宗聞言,點頭嘆曰:「此事信有之也。朕曾
問十代閻君:『老幼安乎?』他道:『俱安,但恐御妹壽促。』果中其言。」合
宮人都來悲切,盡到花陰下看時,只見那宮主微微有氣。唐王道:「莫哭!莫哭
!休驚了他。」遂上前將御手扶起頭來,叫道:「御妹甦醒甦醒。」那宮主忽的
翻身,叫:「丈夫慢行,等我一等。」太宗道:「御妹,是我等在此。」宮主抬
頭睜眼觀看道:「你是誰人,敢來扯我?」太宗道:「是你皇兄、皇嫂。」宮主
道:「我那裏得個甚麼皇兄、皇嫂?我娘家姓李,我的乳名喚做李翠蓮,我丈夫
姓劉名全,兩口兒都是均州人氏。因為我三個月前拔金釵在門首齋僧,我丈夫怪
我擅出內門,不遵婦道,罵了我幾句,是我氣塞胸堂,將白綾帶懸梁縊死,撇下
一雙兒女,晝夜悲啼。今因我V夫被唐王欽差,赴陰司進瓜果,閻王憐憫,放我
夫妻回來。他在前走,因我來遲,趕不上他,我絆了一跌。你等無禮!不知姓名
,怎敢扯我?」太宗聞言,與眾宮人道:「想是御妹跌昏了,胡說哩。」傳旨教
太醫院進湯藥,將玉英扶入宮中。

唐王當殿,忽有當駕官奏道:「萬歲,今有進瓜果人劉全還魂,在朝門外等旨。」
唐王大驚,急傳旨,將劉全召進,俯伏丹墀。太宗問道:「進瓜果之事何如?」
劉全道:「臣頂瓜果,徑至鬼門關,引上森羅殿,見了那十代閻君,將瓜果奉上
,備言我王慇懃致謝之意。閻君甚喜,多多拜上我王道:『真是個有信有德的太
宗皇帝!』」唐王道:「你在陰司見些甚麼來?」劉全道:「臣不曾遠行,沒見
甚的,只聞得閻王問臣鄉貫、姓名。臣將棄家捨子,因妻縊死,願來進瓜之事,
說了一遍。他急差鬼使,引過我妻,就在森羅殿下相會。一壁廂又檢看死生文簿
,說我夫妻都有登仙之壽,便差鬼使送回。臣在前走,我妻後行,幸得還魂。但
不知妻投何所。」唐王驚問道:「那閻王可曾說你妻甚麼?」劉全道:「閻王不
曾說甚麼,只聽得鬼使說:『李翠蓮歸陰日久,屍首無存。』閻王道:『唐御妹
李玉英今該促死,教翠蓮即借玉英屍還魂去罷。』臣不知『唐御妹』是甚地方,
家居何處,我還未曾得去找尋哩。」

唐王聞奏,滿心歡喜,當對多官道:「朕別閻君,曾問宮中之事。他言:『老幼
俱安,但恐御妹壽促。』卻才御妹玉英花陰下跌死,朕急扶看,須臾甦醒,口叫
:『丈夫慢行,等我一等。』朕只道是他跌昏了胡言。又問他詳細,他說的話,
與劉全一般。」魏徵奏道:「御妹偶爾壽促,少甦醒即說此言,此是劉全妻借屍
還魂之事。此事也有,可請宮主出來,看他有甚話說。」唐王道:「朕才命太醫
院去進藥,不知何如。」便教妃嬪入宮去請。那宮主在裏面亂嚷道:「我吃甚麼
藥?這裏那是我家?我家是清涼瓦屋,不像這個害黃病的房子,花狸狐哨的門扇
,放我出去!放我出去!」

正嚷處,只見四五個女官、兩三個太監扶著他,直至殿上。唐王道:「你可認得
你丈夫麼?」玉英道:「說那裏話,我兩個從小兒的結髮夫妻,與他生男育女,
怎的不認得?」唐王叫內官攙他下去。那宮主下了寶殿,直至白玉階前,見了劉
全,一把扯住道:「丈夫,你往那裏去,就不等我一等?我跌了一跌,被那些沒
道理的人圍住我嚷,這是怎的說?」那劉全聽他說的話是妻之言,觀其人非妻之
面,不敢相認。唐王道:
「這正是山崩地裂有人見,捉生替死卻難逢。」好一個有道的君王,即將御妹的
妝奩、衣物、首飾,盡賞賜了劉全,就如陪嫁一般。又賜與他永免差徭的御旨,
著他帶領御妹回去。他夫妻兩個便在階前謝了恩,歡歡喜喜還鄉。有詩為證:
    人生人死是前緣,短短長長各有年。
    劉全進瓜回陽世,借屍還魂李翠蓮。

他兩個辭了君王,徑來均州城裏,見舊家業、兒女俱好,兩口兒宣揚善果不題。

卻說那尉遲恭將金銀一庫,上河南開封府訪看,相良原來賣水為活,同妻張氏在
門首販賣烏盆瓦器營生,但賺得些錢兒,只以盤纏為足,其多少齋僧布施,買金
銀紙錠,記庫焚燒,故有此善果臻身。陽世間是一條好善的窮漢,那世裏卻是個
積玉堆金的長者。尉遲恭將金銀送上他門,諕得那相公、相婆魂飛魄散。又兼有
本府官員,茅舍外車馬駢集。那老兩口子如痴如啞,跪在地下,只是磕頭禮拜。
尉遲恭道:「老人家請起。我雖是個欽差官,卻齎著我王的金銀送來還你。」他
戰兢兢的答道:「小的沒有甚麼金銀放債,如何敢受這不明之財?」尉遲恭道:
「我也訪得你是個窮漢,只是你齋僧布施,儘其所用,就買辦金銀紙錠,燒記陰
司,陰司裏有你積下的錢鈔。是我太宗皇帝死去三日,還魂復生,曾在那陰司裏
借了你一庫金銀,今此照數送還與你。你可一一收下,等我好去回旨。」那相良
兩口兒只是朝天禮拜,那裏敢受。道:「小的若受了這些金銀,就死得快了。雖
然是燒紙記庫,此乃冥冥之事﹔況萬歲爺爺那世裏借了金銀,有何憑據?我決不
敢受。」尉遲恭道:「陛下說,借你的東西,有崔判官作保可證。你收下罷。」
相良道:「就死也是不敢受的。」

尉遲恭見他苦苦推辭,只得具本差人啟奏。太宗見了本,知相良不受金銀,道:
「此誠為善良長者。」即傳旨教胡敬德將金銀與他修理寺院,起蓋生祠,請僧作
善,就當還他一般。旨意到日,敬德望闕謝恩宣旨,眾皆知之。遂將金銀買到城
裏軍民無礙的地基一段,周圍有五十畝寬闊,在上興工,起蓋寺院,名「敕建相
國寺」,左有相公、相婆的生祠,鐫碑刻石,上寫著「尉遲恭監造」,即今「大
相國寺」是也。

工完回奏,太宗甚喜。卻又聚集多官,出榜招僧,修建水陸大會,超度冥府孤魂
。榜行天下,著各處官員推選有道的高僧,上長安做會。那消個月之期,天下多
僧俱到。唐王傳旨,著太史丞傅奕選舉高僧,修建佛事。傅奕聞旨,即上疏止浮
圖,以言無佛。表曰:
西域之法,無君臣父子,以三塗六道,蒙誘愚蠢。追既往之罪,窺將來之福,口
誦梵言,以圖偷免。且生死壽夭,本諸自然﹔刑德威福,係之人主。今聞俗徒然
託,皆云由佛。自五帝三王,未有佛法,君明臣忠,年祚長久。至漢明帝始立胡
神,然惟西域桑門自傳其教。實乃夷犯中國,不足為信。

太宗聞言,遂將此表擲付群臣議之。時有宰相蕭瑀,出班俯?奏曰:「佛法興自
屢朝,弘善遏惡,冥助國家,理無廢棄。佛,聖人也。非聖者無法,請寘嚴刑。」
傅奕與蕭瑀論辨,言:「禮本於事親事君,而佛背親出家,以匹夫抗天子,以繼
體悖所親。蕭瑀不生於空桑,乃遵無父之教,正所謂非孝者無親。」蕭瑀但合掌
曰:「地獄之設,正為是人。」太宗召太僕卿張道源、中書令張士衡,問佛事營
福,其應何如。二臣對曰:「佛在清淨仁恕,果正佛空。周武帝以三教分次﹔大
慧禪師有贊幽遠,歷眾供養而無不顯﹔五祖投胎,達摩現像。自古以來,皆云三
教至尊而不可毀,不可廢。伏乞陛下聖鑒明裁。」太宗甚喜道:「卿之言合理。
再有所陳者,罪之。」遂著魏徵與蕭瑀、張道源邀請諸佛,選舉一名有大德行者
作壇主,設建道場。眾皆頓首謝恩而退。自此時出了法律:但有毀僧謗佛者,斷
其臂。

次日三位朝臣,聚眾僧,在那山川壇裏,逐一從頭查選,內中選得一名有德行的
高僧。你道他是誰人?
    靈通本諱號金蟬,只為無心聽佛講。
    轉托塵凡苦受磨,降生世俗遭羅網。
    投胎落地就逢兇,未出之前臨惡黨。
    父是海州陳狀元,外公總管當朝長。
    出身命犯落江星,順水隨波逐浪泱。
    海島金山有大緣,遷安和尚將他養。
    年方十八認親娘,特赴京都求外長。
    總管開山調大軍,洪州剿寇誅兇黨。
    狀元光蕊脫天羅,子父相逢堪賀獎。
    復謁當今受主恩,凌煙閣上賢名響。
    恩官不受願為僧,洪福沙門將道訪。
    小字江流古佛兒,法名喚做陳玄奘。

當日對眾舉出玄奘法師。這個人自幼為僧,出娘胎,就持齋受戒。他外公見是當
朝一路總管殷開山。他父親陳光蕊中狀元,官拜文淵殿大學士。一心不愛榮華,
只喜修持寂滅。查得他根源又好,德行又高﹔千經萬典,無所不通﹔佛號仙音,
無般不會。

當時三位引至御前,揚塵舞蹈。拜罷奏曰:「臣瑀等蒙聖旨,選得高僧一名陳玄
奘。」太宗聞其名,沉思良久道:「可是學士陳光蕊之兒玄奘否?」江流兒叩頭
曰:「臣正是。」太宗喜道:「果然舉之不錯,誠為有德行有禪心的和尚。朕賜
你左僧綱,右僧綱,天下大闡都僧綱之職。」玄奘頓首謝恩,受了大闡官爵。又
賜五彩織金袈裟一件、毘盧帽一頂。教他用心再拜明僧,排次闍黎班首,書辦旨
意,前赴化生寺,擇定吉日良時,開演經法。

玄奘再拜領旨而出,遂到化生寺裏,聚集多僧,打造禪榻,裝修功德,整理音樂
。選得大小明僧共計一千二百名,分派上中下三堂。諸所佛前,物件皆齊,頭頭
有次。選到本年九月初三日黃道良辰,開啟做七七四十九日水陸大會。即具表申
奏。太宗及文武國戚皇親,俱至期赴會,拈香聽講。有詩為證。詩曰:
    龍集貞觀正十三,王宣大眾把經談。
    道場開演無量法,雲霧光乘大願龕。
    御敕垂恩修上剎,金蟬脫殼化西涵。
    普施善果超沉沒,秉教宣揚前後三。

貞觀十三年,歲次己巳,九月甲戌,初三日,癸卯良辰,陳玄奘大闡法師聚集一
千二百名高僧,都在長安城化生寺開演諸品妙經。那皇帝早朝已畢,帥文武多官
,乘鳳輦龍車,出離金鑾寶殿,徑上寺來拈香。怎見那鑾駕?真個是:
一天瑞氣,萬道祥光。仁風輕淡蕩,化日麗非常。千官環佩分前後,五衛旌旗列
兩旁。執金瓜,擎斧鉞,雙雙對對﹔絳紗燭,御爐香,靄靄堂堂。龍飛鳳舞,鶚
薦鷹揚。聖明天子正,忠義大臣良。介福千年過舜禹,昇平萬代賽堯湯。又見那
曲柄傘,滾龍袍,輝光相射﹔玉連環,彩鳳扇,瑞靄飄揚。珠冠玉帶,紫綬金章
。護駕千隊,扶輿將兩行。這皇帝沐浴虔誠尊敬佛,皈依善果喜拈香。

唐王大駕早到寺前,吩咐住了音樂響器。下了車輦,引著多官,拜佛拈香。三匝
已畢,抬頭觀看,果然好座道場。但見:
幢幡飄舞,寶蓋飛輝。幢幡飄舞,凝空道道彩霞搖﹔寶蓋飛輝,映日翩翩紅電徹。
世尊金像貌臻臻,羅漢玉容威烈烈。瓶插仙花,爐焚檀降。瓶插仙花,錦樹輝輝
漫寶剎﹔爐焚檀降,香雲靄靄透清霄。時新果品砌朱盤,奇樣糖酥堆彩案。高僧
羅列誦真經,願拔孤魂離苦難。

太宗文武俱各拈香,拜了佛祖金身,參了羅漢。又見那大闡都綱陳玄奘法師引眾
僧羅拜唐王。禮畢,分班各安禪位。法師獻上濟孤榜文與太宗看。榜曰:
至德渺茫,禪宗寂滅。清淨靈通,周流三界。千變萬化,統攝陰陽。體用真常,
無窮極矣。觀彼孤魂,深宜哀愍。此奉太宗聖命:選集諸僧,參禪講法。大開方
便門庭,廣運慈悲舟楫,普濟苦海群生,脫免沉痾六趣。引歸真路,普玩鴻濛﹔
動止無為,混成純素。仗此良因,邀賞清都絳闕﹔乘吾勝會,脫離地獄凡籠。早
登極樂任逍遙,來往西方隨自在。
  詩曰:
    一爐永壽香,幾卷超生籙。
    無邊妙法宣,無際天恩沐。
    冤孽盡消除,孤魂皆出獄。
    願保我邦家,清平萬咸福。

一宿晚景題過。次早,法師又昇坐,聚眾誦經不題。

卻說南海普陀山觀世音菩薩,自領了如來佛旨,在長安城訪察取經的善人,日久
未逢真實有德行者。忽聞得太宗宣揚善果,選舉高僧,開建大會。又見得法師壇
主,乃是江流兒和尚,正是極樂中降來的佛子,又是他原引送投胎的長老。菩薩
十分歡喜,就將佛賜的寶貝捧上長街,與木叉貨賣。你道他是何寶貝?有一件錦
襴異寶袈裟、九環錫杖。還有那金緊禁三個箍兒,密密藏收,以俟後用。只將袈
裟、錫杖出賣。

長安城裏,有那選不中的愚僧,倒有幾貫村鈔。見菩薩變化個疥癩形容,身穿破
衲,赤腳光頭,將袈裟捧定,豔豔生光,他上前問道:「那癩和尚,你的袈裟要
賣多少價錢?」菩薩道:「袈裟價值五千兩,錫杖價值二千兩。」那愚僧笑道:
「這兩個癩和尚是瘋子!是傻子!這兩件粗物,就賣得七千兩銀子?只是除非穿
上身長生不老,就得成佛作祖,也值不得這許多!拿了去!賣不成!」

那菩薩更不爭吵,與木叉往前又走。行勾多時,來到東華門前,正撞著宰相蕭瑀
散朝而回,眾頭踏喝開街道。那菩薩公然不避,當街上拿著袈裟,徑迎著宰相。
宰相勒馬觀看,見袈裟豔豔生光,著手下人問那賣袈裟的要價幾何,菩薩道:
「袈裟要五千兩,錫杖要二千兩。」蕭瑀道:「有何好處,值這般高價?」菩薩
道:「袈裟有好處,有不好處﹔有要錢處,有不要錢處。」蕭瑀道:「何為好?
何為不好?」菩薩道:「著了我袈裟,不入沉淪,不墮地獄,不遭惡毒之難,不
遇虎狼之災,便是好處﹔若貪淫樂禍的愚僧,不齋不戒的和尚,毀經謗佛的凡夫
,難見我袈裟之面,這便是不好處。」又問道:「何為要錢,不要錢?」菩薩道
:「不遵佛法,不敬三寶,強買袈裟、錫杖,定要賣他七千兩,這便是要錢﹔若
敬重三寶,見善隨喜,皈依我佛,承受得起,我將袈裟、錫杖情願送他,與我結
個善緣,這便是不要錢。」蕭瑀聞言,倍添春色,知他是個好人。即便下馬,與
菩薩以禮相見,口稱:「大法長老,恕我蕭瑀之罪。我大唐皇帝十分好善,滿朝
的文武無不奉行。即今起建水陸大會,這袈裟正好與大都闡陳玄奘法師穿用。我
和你入朝見駕去來。」

菩薩欣然從之,拽轉步,徑進東華門裏。黃門官轉奏,蒙旨宣至寶殿。見蕭瑀引
著兩個疥癩僧人,立於階下,唐王問曰:「蕭瑀來奏何事?」蕭瑀俯伏階前道:
「臣出了東華門前,偶遇二僧,乃賣袈裟與錫杖者。臣思法師玄奘可著此服,故
領僧人啟見。」太宗大喜,便問那袈裟價值幾何。菩薩與木叉侍立階下,更不行
禮,因問袈裟之價,答道:「袈裟五千兩,錫杖二千兩。」太宗道:「那袈裟有
何好處,就值許多?」菩薩道:這袈裟,龍披一縷,免大鵬吞噬之災﹔鶴掛一絲
,得超凡入聖之妙。但坐處,有萬神朝禮﹔凡舉動,有七佛隨身。這袈裟,是冰
蠶造練抽絲,巧匠翻騰為線,仙娥織就,神女機成,方方簇幅繡花縫。片片相幫
堆錦簆。玲瓏散碎鬥妝花,色亮飄光噴寶豔。穿上滿身紅霧遶,脫來一段彩雲飛
。三天門外透元光,五岳山前生寶氣。重重嵌就西番蓮,灼灼懸珠星斗象。四角
上有夜明珠,攢頂間一顆祖母綠。雖無全照原本體,也有生光八寶攢。這袈裟,
閑時折疊,遇聖才穿。閑時折疊,千層包裹透虹霓﹔遇聖才穿,驚動諸天神鬼怕
。上邊有如意珠、摩尼珠、辟塵珠、定風珠﹔又有那紅瑪瑙、紫珊瑚、夜明珠、
舍利子。偷月沁白,與日爭紅。條條仙氣盈空,朵朵祥光捧聖。條條仙氣盈空,
照徹了天關﹔朵朵祥光捧聖,影遍了世界。照山川,驚虎豹﹔影海島,動魚龍。
沿邊兩道銷金鎖,叩領連環白玉琮。
  詩曰:
    三寶巍巍道可尊,四生六道盡評論。
    明心解養人天法,見性能傳智慧燈。
    護體莊嚴金世界,身心清淨玉壺冰。
    自從佛製袈裟後,萬劫誰能敢斷僧?」

唐王聞言,即命展開袈裟,從頭細看,果然是件好物。道:「大法長老,實不瞞
你。朕今大開善教,廣種福田,見在那化生寺聚集多僧,敷演經法。內中有一個
大有德行者,法名玄奘。朕買你這兩件寶物,賜他受用。你端的要價幾何?」菩
薩聞言,與木叉合掌皈依,道聲佛號,躬身上啟道:「既有德行,貧僧情願送他
,決不要錢。」說罷,抽身便走。唐王急著蕭瑀扯住,欠身立於殿上,問曰:
「你原說袈裟五千兩,錫杖二千兩,你見朕要買,就不要錢,敢是說朕心倚恃君
位,強要你的物件?更無此理。朕照你原價奉償,卻不可推避。」菩薩起手道:
「貧僧有願在前,原說果有敬重三寶,見善隨喜,皈依我佛,不要錢,願送與他
。今見陛下明德止善,敬我佛門﹔況又高僧有德有行,宣揚大法,理當奉上,決
不要錢。貧僧願留下此物告回。」唐王見他這等懃懇,甚喜。隨命光祿寺,大排
素宴酬謝。菩薩又堅辭不受,暢然而去,依舊望都土地廟中隱避不題。

卻說太宗設午朝,著魏徵?旨,宣玄奘入朝。那法師正聚眾登壇,諷經誦偈,一
聞有旨,隨下壇整衣,與魏徵同往見駕。太宗道:「求證善事,有勞法師,無物
酬謝。早間蕭瑀迎著二僧,願送錦襴異寶袈裟一件,九環錫杖一條。今特召法師
領去受用。」玄奘叩頭謝恩。太宗道:「法師如不棄,可穿上與朕看看。」長老
遂將袈裟抖開,披在身上,手持錫杖,侍立階前。君臣個個忻然。誠為如來佛子
。你看他:
    凜凜威顏多雅秀,佛衣可體如裁就。
    暉光豔豔滿乾坤,結綵紛紛凝宇宙。
    朗朗明珠上下排,層層金線穿前後。
    兜羅四面錦沿邊,萬樣稀奇鋪綺繡。
    八寶妝花縛鈕絲,金環束領攀絨扣。
    佛天大小列高低,星象尊卑分左右。
    玄奘法師大有緣,現前此物堪承受。
    渾如極樂活阿羅,賽過西方真覺秀。
    錫杖叮噹鬥九環,毘盧帽映多豐厚。
    誠為佛子不虛傳,勝似菩提無詐謬。

當時文武階前喝采。太宗喜之不勝,即著法師穿了袈裟,持了寶杖﹔又賜兩隊儀
從,著多官送出朝門,教他上大街行道,往寺裏去,就如中狀元誇官的一般。這
去玄奘再拜謝恩,在那大街上,烈烈轟轟,搖搖擺擺。你看那長安城裏,行商坐
賈、公子王孫、墨客文人、大男小女,無不爭看誇獎,俱道:「好個法師,真是
個活羅漢下降,活菩薩臨凡。」

玄奘直至寺裏,僧人下榻來迎。一見他披此袈裟,執此錫杖,都道是地藏王來了
,各各歸依,侍於左右。玄奘上殿,炷香禮佛。又對眾感述聖恩已畢,各歸禪座
。又不覺紅輪西墜。正是那:
日落煙迷草樹,帝都鐘鼓初鳴。叮叮三響斷人行。前後街前寂靜。上剎輝煌燈火
,孤村冷落無聲。禪僧入定理殘經。正好煉魔養性。

光陰撚指,卻當七日正會。玄奘又具表,請唐王拈香。此時善聲遍滿天下。太宗
即排駕,率文武多官、后妃國戚,早赴寺裏。那一城人,無論大小尊卑,俱詣寺
聽講。

當有菩薩與木叉道:「今日是水陸正會,以一七繼七七,可矣了。我和你雜在眾
人叢中,一則看他那會何如,二則看金蟬子可有福穿我的寶貝,三則也聽他講的
是那一門經法。」兩人隨投寺裏。正是有緣得遇舊相識,般若還歸本道場。入到
寺裏觀看,真個是:天朝大國,果勝裟婆。賽過祇園舍衛,也不亞上剎招提。那
一派仙音響喨,佛號喧嘩。
這菩薩直至多寶臺邊,果然是明智金蟬之相。詩曰:
    萬象澄明絕點埃,大典玄奘坐高臺。
    超生孤魂暗中到,聽法高流市上來。
    施物應機心路遠,出生隨意藏門開。
    對看講出無量法,老幼人人放喜懷。
  又詩曰:
    因遊法界講堂中,逢見相知不俗同。
    盡說目前千萬事,又談塵劫許多功。
    法雲容曳舒群岳,教網張羅滿太空。
    檢點人生歸善念,紛紛天雨落花紅。

 那法師在臺上念一會《受生度亡經》,談一會《安邦天寶篆》,又宣一會《勸
修功卷》。這菩薩近前來,拍著寶臺,厲聲高叫道:「那和尚,你只會談小乘教
法,可會談大乘麼?」玄奘聞言,心中大喜,翻身跳下臺來,對菩薩起手道:
「老師父,弟子失瞻多罪。見前的蓋眾僧人,都講的是小乘教法,卻不知大乘教
法如何。」菩薩道:「你這小乘教法,度不得亡者超昇,只可渾俗和光而已。我
有大乘佛法三藏,能超亡者昇天,能度難人脫苦,能修無量壽身,能作無來無去。」

正講處,有那司香巡堂官急奏唐王道:「法師正講談妙法,被兩個疥癩遊僧扯下
來亂說胡話。」王令擒來。只見許多人將二僧推擁進後法堂,見了太宗,那僧人
手也不起,拜也不拜,仰面道:「陛下問我何事?」唐王卻認得他,道:「你是
前日送袈裟的和尚?」菩薩道:「正是。」太宗道:「你既來此處聽講,只該吃
些齋便了,為何與我法師亂講,擾亂經堂,誤我佛事?」菩薩道:「你那法師講
的是小乘教法,度不得亡者昇天。我有大乘佛法三藏,可以度亡脫苦,壽身無壞
。」太宗正色喜問道:「你那大乘佛法在於何處?」菩薩道:「在大西天天竺國
大雷音寺我佛如來處,能解百冤之結,能消無妄之災。」太宗道:「你可記得麼
?」菩薩道:「我記得。」太宗大喜道:「教法師引去,請上臺開講。」

那菩薩帶了木叉,飛上高臺,遂踏祥雲,直至九霄,現出救苦原身,托了淨瓶楊
柳。左邊是木叉惠岸,執著棍,抖搜精神。喜的個唐王朝天禮拜,眾文武跪地焚
香。滿寺中僧尼道俗、士人工賈,無一人不拜禱道:「好菩薩!好菩薩!」有讚
為證。但見那:瑞靄散繽紛,祥光護法身。九霄華漢裏,現出女真人。那菩薩,
頭上戴一頂金葉紐、翠花鋪、放金光、生瑞氣的垂珠纓絡﹔身上穿一領淡淡色、
淺淺妝、盤金龍、飛綵鳳的結素藍袍﹔胸前掛一面對月明、舞清風、雜寶珠、攢
翠玉的砌香環珮﹔腰間繫一條冰蠶絲、織金邊、登彩雲、促瑤海的錦繡絨裙﹔面
前又領一個飛東洋、遊普世、感恩行孝、黃毛紅嘴白鸚哥。手內托著一個施恩濟
世的寶瓶,瓶內插著一枝灑青霄、撒大惡、掃開殘霧垂楊柳。玉環穿繡扣,金蓮
足下深。三天許出入。這才是救苦救難觀世音。

喜的個唐太宗忘了江山,愛的那文武官失卻朝禮,蓋眾多人都念「南無觀世音菩
薩」。太宗即傳旨,教巧手丹青描下菩薩真像。旨意一聲,選出個圖神寫聖、遠
見高明的吳道子(此人即後圖功臣於凌煙閣者)。當時展開妙筆,圖寫真形。那
菩薩祥雲漸遠,霎時間不見了金光。只見那半空中滴溜溜落下一張簡帖,上有幾
句頌子,寫得明白。頌曰:禮上大唐君,西方有妙文。程途十萬八千里,大乘進
慇懃。此經回上國,能超鬼出群。若有肯去者,求正果金身。

太宗見了頌子,即命眾僧:「且收勝會,待我差人取得大乘經來,再秉丹誠,重
修善果。」眾官無不遵依。當時在寺中問曰:「誰肯領朕旨意,上西天拜佛求經?」
問不了,傍邊閃過法師,帝前施禮道:「貧僧不才,願效犬馬之勞,與陛下求取
真經,祈保我王江山永固。」唐王大喜,上前將御手扶起道:「法師果能盡此忠
賢,不怕程途遙遠,跋涉山川,朕情願與你拜為兄弟。」玄奘頓首謝恩。唐王果
是十分賢德,就去那寺裏佛前,與玄奘拜了四拜,口稱「御弟聖僧」。玄奘感謝
不盡道:「陛下,貧僧有何德何能,敢蒙天恩眷顧如此?我這一去,定要捐軀努
力,直至西天﹔如不到西天,不得真經,即死也不敢回國,永墮沉淪地獄。」隨
在佛前拈香,以此為誓。唐王甚喜,即命回鑾,待選良利日辰,發牒出行,遂此
駕回各散。

玄奘亦回洪福寺裏。那本寺多僧與幾個徒弟,早聞取經之事,都來相見,因問:
「發誓願上西天,實否?」玄奘道:「是實。」他徒弟道:「師父呵,嘗聞人言
,西天路遠,更多虎豹妖魔。只怕有去無回,難保身命。」玄奘道:「我已發了
洪誓大願,不取真經,永墮沉淪地獄。大抵是受王恩寵,不得不盡忠以報國耳。
我此去真是渺渺茫茫,吉凶難定。」又道:「徒弟們,我去之後,或三二年,或
五七年,但看那山門裏松枝頭向東,我即回來﹔不然,斷不回矣。」眾徒將此言
切切而記。

次早,太宗設朝,聚集文武,寫了取經文牒,用了通行寶印。有欽天監奏曰:
「今日是人專吉星,堪宜出行遠路。」唐王大喜。又見黃門官奏道:「御弟法師
朝門外候旨。」隨即宣上寶殿道:「御弟,今日是出行吉日。這是通關文牒。朕
又有一個紫金缽盂,送你途中化齋而用。再選兩個長行的從者。又銀騔的馬一匹
,送為遠行腳力。你可就此行程。」玄奘大喜,即便謝了恩,領了物事,更無留
滯之意。唐王排駕,與多官同送至關外。只見那洪福寺僧與諸徒將玄奘的冬夏衣
服,俱送在關外相等。唐王見了,先教收拾行囊、馬匹,然後著官人執壺酌酒。
太宗舉爵,又問曰:「御弟雅號甚稱?」玄奘道:「貧僧出家人,未敢稱號。」
太宗道:「當時菩薩說,西天有經三藏。御弟可指經取號,號作三藏何如?」玄
奘又謝恩,接了御酒道:「陛下,酒乃僧家頭一戒,貧僧自為人,不會飲酒。」
太宗道:「今日之行,比他事不同,此乃素酒,只飲此一杯,以盡朕奉餞之意。」
三藏不敢不受,接了酒,方待要飲,只見太宗低頭,將御指拾一撮塵土,彈入酒
中。三藏不解其意,太宗笑道:「御弟呵,這一去,到西天,幾時可回?」三藏
道:「只在三年,徑回上國。」太宗道:「日久年深,山遙路遠,御弟可進此酒
:寧戀本鄉一捻土,莫愛他鄉萬兩金。」三藏方悟捻土之意,復謝恩飲盡,辭謝
出關而去。唐王駕回。

    畢竟不知此去何如,且聽下回分解。





}  \end{pinyinscope}\switchcolumn{\myfontc \section{第 一 二 回} , 唐 王 秉 诚 修 大 会 , 观 音 显 圣 化 , 金 蝉 又 说 : 鬼 使 与 刘 全 夫 妻 二 人 出 阴 宫 , 其 阴 风 夹 路 , 径 直 到 长 安 大 国 , 把 刘 全 的 魂 魄 推 进 金 亭 馆 里 , 把 翠 莲 的 魂 魄 带 进 皇 宫 内 院 。
只 见 那 玉 英 宫 的 神 主 正 在 花 阴 下 , 慢 慢 地 走 着 绿 苔 走 , 被 鬼 使 扑 得 满 怀 , 被 推 倒 在 地 上 , 活 捉 了 他 的 魂 魄 , 又 把 翠 莲 的 魂 灵 推 进 玉 英 的 身 子 里 。
鬼 使 回 转 , 阴 司 没 有 问 题 。
又 说 宫 院 中 的 大 小 侍 婢 见 玉 英 跌 死 , 急 忙 跑 到 金 銮 殿 , 报 告 给 三 宫 皇 后 说 : 宫 主 娘 娘 摔 死 了 。
皇 后 大 惊 , 随 即 报 告 太 宗 。
太 宗 听 到 这 话 , 点 头 叹 息 说 : 这 事 确 实 有 这 事 。
我 曾 经 问 十 代 阎 君 : 老 幼 安 全 吗 ? 他 说 : 都 安 , 只 恐 怕 御 妹 的 寿 命 太 短 了 。
果 然 符 合 了 他 的 话 。
这 些 宫 女 都 来 悲 伤 痛 切 , 全 都 到 花 阴 下 看 , 只 见 那 宫 主 微 微 有 气 。
唐 王 说 : 不 要 哭 , 不 要 哭 , 不 要 惊 吓 了 他 。
于 是 上 前 把 御 手 扶 起 头 来 , 喊 道 : 御 妹 苏 醒 。
其 宫 主 忽 然 翻 身 大 叫 : 大 丈 夫 慢 慢 走 , 等 我 一 等 。
太 宗 说 : 御 妹 , 是 我 们 在 这 里 。
太 宗 抬 头 睁 着 眼 睛 看 着 说 : 你 是 谁 , 敢 来 搅 我 ? 太 宗 说 : 是 你 的 皇 兄 、 皇 嫂 。
宫 主 说 : 我 哪 里 得 到 什 么 皇 兄 、 皇 嫂 , 我 娘 家 姓 李 , 我 的 乳 名 叫 做 李 翠 莲 , 我 的 丈 夫 姓 刘 名 全 , 两 个 孩 子 都 是 均 州 人 。
因 为 我 三 个 月 前 拔 金 钗 在 门 口 斋 戒 和 尚 , 我 的 丈 夫 怪 我 擅 自 出 入 内 门 , 不 遵 妇 道 , 骂 了 我 几 句 , 是 我 的 气 塞 胸 堂 , 将 白 绫 带 悬 挂 在 梁 上 , 抛 下 一 双 儿 女 , 日 夜 悲 啼 。
现 在 因 为 我 的 妻 子 被 唐 王 钦 差 , 到 阴 司 进 献 瓜 果 , 阎 王 怜 悯 怜 悯 , 放 我 夫 妻 回 来 。
他 在 前 走 , 因 为 我 来 得 晚 , 赶 不 上 他 , 我 被 绊 了 一 跌 。
你 们 无 礼 , 不 知 你 们 的 姓 名 , 怎 么 敢 说 我 呢 ? 太 宗 听 到 这 话 , 便 和 众 宫 人 说 : 想 是 御 妹 跌 昏 了 , 怎 么 说 呢 ?
皇 帝 传 旨 让 太 医 院 进 汤 药 , 让 玉 英 扶 进 宫 中 。
唐 王 当 殿 时 , 忽 然 有 位 当 驾 的 官 员 上 奏 说 : 万 岁 , 现 在 有 进 献 瓜 果 的 人 刘 全 还 魂 , 在 朝 门 外 等 候 圣 旨 。
唐 王 大 惊 , 急 忙 传 旨 , 将 刘 全 召 进 , 俯 伏 在 宫 殿 台 阶 上 。
太 宗 问 道 : 进 献 瓜 果 的 事 情 是 怎 样 的 ? 刘 全 说 : 我 顶 着 瓜 果 , 径 直 到 鬼 门 关 , 引 上 森 罗 殿 , 见 到 那 十 代 阎 君 , 把 瓜 果 送 给 皇 上 , 详 细 地 说 了 我 王 致 谢 的 意 思 。
阎 君 非 常 高 兴 , 多 多 拜 上 我 王 说 : 真 是 个 有 信 有 德 的 太 宗 皇 帝 啊 唐 王 说 : 你 在 阴 司 见 到 什 么 ? 刘 全 说 : 我 不 曾 远 行 , 没 有 见 到 什 么 , 只 听 到 阎 王 问 我 的 乡 贯 、 姓 名 。
我 将 要 弃 家 舍 子 , 因 为 妻 子 死 了 , 我 愿 来 进 瓜 的 事 , 说 了 一 遍 。
他 赶 快 派 遣 鬼 使 , 带 我 的 妻 子 去 , 就 在 森 罗 殿 下 相 会 。
一 个 墙 壁 的 厢 房 里 又 检 查 看 死 生 文 簿 , 说 我 夫 妻 都 有 登 仙 的 寿 命 , 就 派 鬼 使 送 回 去 。
我 在 前 走 , 我 的 妻 子 在 后 走 , 希 望 能 够 还 魂 。
但 不 知 妻 子 投 奔 到 什 么 地 方 。
唐 王 惊 讶 地 问 道 : 那 阎 王 可 以 说 你 的 妻 子 什 么 ? 刘 全 说 : 阎 王 不 曾 说 过 什 么 , 只 听 到 鬼 使 说 : 李 翠 莲 回 到 阴 间 已 久 , 尸 首 没 有 存 在 。
阎 王 说 : 唐 御 妹 李 玉 英 现 在 该 快 死 了 , 让 翠 莲 立 即 借 李 玉 英 的 尸 体 还 魂 去 吧 。
我 不 知 道 唐 御 妹 是 什 么 地 方 , 家 住 在 什 么 地 方 , 我 还 不 曾 去 寻 找 呀 。
唐 王 听 了 奏 章 , 满 心 欢 喜 , 应 当 面 对 许 多 官 员 说 : 我 告 别 阎 君 , 曾 经 问 过 宫 中 的 事 情 。
他 说 : 老 幼 都 安 逸 , 只 恐 怕 皇 帝 妹 妹 寿 命 短 促 。
不 一 会 儿 , 皇 上 的 妹 妹 玉 英 花 阴 下 跌 死 了 , 朕 急 忙 扶 着 她 看 她 , 不 一 会 儿 就 醒 了 , 口 里 大 叫 : 男 子 慢 慢 走 , 等 我 一 等 。
我 只 说 是 他 失 误 了 胡 言 。
又 问 他 详 细 , 他 说 的 话 , 与 刘 全 一 样 。
魏 徵 上 奏 说 : 御 妹 偶 尔 寿 命 很 短 , 刚 刚 醒 过 来 就 说 了 这 话 , 这 是 刘 全 妻 借 尸 还 魂 的 事 。
此 事 也 有 , 可 请 宫 主 出 来 , 看 他 有 什 么 话 说 。
唐 王 说 : 我 才 让 太 医 院 去 进 药 , 不 知 怎 么 样 ?
于 是 就 教 妃 嫔 进 入 宫 中 去 请 求 。
其 宫 主 在 里 面 乱 骂 道 : 我 吃 什 么 药 , 此 里 还 是 我 家 , 我 家 是 清 凉 瓦 屋 , 不 像 这 个 害 黄 病 的 房 子 , 花 狸 狐 哨 的 门 扇 , 放 我 出 去 , 放 我 出 去 。 正 在 吵 闹 的 时 候 , 只 见 四 五 个 女 官 , 两 三 个 太 监 扶 着 他 , 直 到 殿 上 。
唐 王 说 : 你 可 以 认 得 你 的 丈 夫 吗 ? 玉 英 说 : 你 说 那 里 的 话 , 我 两 个 从 小 儿 子 的 结 发 夫 妻 , 与 他 生 男 育 女 , 怎 么 不 认 得 ? 唐 王 叫 宦 官 搀 着 他 下 去 。
其 宫 主 下 了 宝 殿 , 径 直 到 白 玉 阶 前 , 见 到 刘 全 , 一 把 握 住 说 : 男 子 , 你 去 那 里 去 , 就 不 等 我 一 等 , 我 跌 了 一 跌 , 被 那 些 没 道 理 的 人 围 住 我 吵 , 这 是 什 么 说 法 ?
唐 王 说 : 这 正 是 山 崩 地 裂 有 人 见 到 , 捉 生 替 死 却 难 逢 。
君 王 于 是 将 妹 妹 的 妆 、 衣 物 、 首 饰 全 都 赏 赐 给 刘 全 , 就 好 像 陪 嫁 一 样 。
又 赐 给 他 永 远 免 除 差 的 御 旨 , 让 他 带 着 御 妹 回 去 。
夫 妻 两 个 , 便 在 阶 前 谢 恩 , 欢 喜 喜 欢 回 到 家 乡 。
有 诗 作 证 : 人 生 人 死 是 前 缘 , 短 短 长 长 各 有 年 。
刘 全 进 瓜 回 阳 世 , 借 尸 还 魂 李 翠 莲 。
其 他 两 个 人 辞 别 君 王 , 径 直 来 到 均 州 城 里 , 见 到 旧 时 的 家 业 、 儿 女 都 很 好 , 两 个 儿 子 宣 扬 善 果 果 没 有 题 写 。
又 说 : 那 尉 迟 恭 带 着 一 库 金 银 , 到 河 南 开 封 府 去 探 看 , 相 良 原 来 是 卖 水 为 生 , 同 妻 张 氏 在 门 口 贩 卖 乌 盆 瓦 器 营 生 , 只 是 赢 得 一 点 钱 , 只 以 盘 缠 为 满 足 , 多 少 斋 僧 布 施 , 买 金 银 纸 , 记 库 焚 烧 , 所 以 有 这 种 善 果 。
阳 世 间 是 一 个 好 善 的 穷 汉 , 那 世 上 却 是 一 个 积 玉 堆 金 的 长 者 。
尉 迟 恭 带 着 金 银 送 到 他 家 的 门 口 , 把 那 相 公 、 相 婆 的 尸 体 送 上 了 魂 飞 魄 散 。
又 兼 有 本 府 的 官 员 , 茅 屋 外 的 车 马 聚 集 在 一 起 。
其 老 两 口 子 , 如 痴 如 哑 , 跪 在 地 下 , 只 是 磕 头 礼 拜 。
尉 迟 恭 说 : 老 人 家 请 让 我 起 来 。
我 虽 然 是 个 钦 差 官 , 却 拿 着 我 王 的 金 银 送 来 还 给 你 。
尉 迟 恭 回 答 说 : 小 的 没 有 什 么 金 银 放 债 , 怎 么 敢 接 受 这 种 不 明 的 财 物 呢 ? 尉 迟 恭 说 : 我 也 找 到 你 是 个 穷 汉 , 只 是 你 吃 斋 僧 的 布 施 , 尽 用 他 的 用 处 , 就 买 出 办 金 银 纸 , 烧 掉 记 录 在 阴 司 里 , 阴 司 里 有 你 积 聚 下 来 的 钱 钞 。
是 我 太 宗 皇 帝 死 去 三 天 , 还 魂 复 生 , 曾 在 那 阴 司 里 借 了 你 一 库 金 银 , 现 在 我 照 数 送 还 给 你 。
你 可 以 一 一 收 下 , 等 我 好 好 回 去 。
相 良 两 个 口 儿 , 只 是 朝 天 礼 拜 , 哪 里 敢 接 受 。
说 : 小 的 如 果 接 受 了 这 些 金 银 , 就 快 死 得 快 了 。
即 使 是 烧 纸 记 库 , 也 是 冥 冥 之 事 , 何 况 万 岁 老 爷 那 世 代 里 借 了 金 银 , 又 有 什 么 凭 据 呢 我 决 不 敢 接 受 。
尉 迟 恭 说 : 陛 下 说 , 借 你 的 东 西 , 有 崔 判 官 作 保 证 可 证 。
你 收 下 来 吧 。
相 良 说 : 就 是 死 也 是 不 敢 接 受 的 。
尉 迟 恭 见 他 苦 苦 推 辞 , 只 得 写 好 本 书 , 派 人 上 奏 。
太 宗 见 了 魏 了 本 , 知 道 魏 相 良 不 接 受 金 银 , 便 说 : 我 确 实 是 善 良 长 者 。
于 是 就 传 圣 旨 教 胡 敬 德 拿 金 银 给 他 修 理 寺 院 , 兴 建 盖 生 祠 , 请 僧 人 做 善 事 , 就 应 当 还 给 他 一 般 。
皇 帝 的 旨 意 到 达 那 天 , 尉 迟 敬 德 望 阙 谢 恩 宣 读 圣 旨 , 大 家 都 知 道 了 。
于 是 就 用 金 银 买 到 城 里 军 民 无 碍 的 地 基 一 段 , 周 围 有 五 十 亩 宽 广 , 在 上 面 兴 工 , 建 造 寺 院 , 名 叫 敕 建 相 国 寺 , 左 边 有 相 公 、 相 婆 的 生 祠 , 碑 刻 石 碑 , 上 面 写 着 尉 迟 恭 监 造 , 就 是 今 天 的 大 相 国 寺 。
工 程 完 成 后 回 来 上 奏 , 太 宗 非 常 高 兴 。
又 聚 集 了 很 多 官 员 , 出 榜 招 请 和 尚 , 修 建 水 陆 大 会 , 超 度 冥 府 孤 魂 。
榜 于 天 下 , 各 地 官 员 推 选 有 道 的 高 僧 , 到 长 安 去 做 会 。
那 么 一 个 月 的 时 间 , 天 下 的 许 多 和 尚 都 来 了 。
唐 王 传 达 圣 旨 , 著 有 太 史 丞 傅 奕 选 拔 高 僧 , 修 建 佛 事 。
傅 奕 听 到 圣 旨 后 , 立 即 上 疏 请 求 停 止 佛 塔 , 说 没 有 佛 教 。
刘 表 说 : 西 域 的 法 令 , 没 有 君 臣 父 子 , 用 三 涂 六 道 来 诱 惑 愚 蠢 的 人 。
追 究 已 往 的 罪 过 , 窥 视 将 来 的 福 佑 , 口 诵 佛 言 , 以 图 苟 且 免 罪 。
况 且 生 死 寿 夭 , 根 源 于 自 然 , 刑 罚 和 德 行 、 威 福 , 关 系 于 君 主 。
现 在 听 说 世 俗 徒 然 寄 托 , 都 说 是 由 于 佛 教 。
自 从 五 帝 三 王 以 来 , 没 有 佛 法 , 君 主 英 明 臣 子 忠 诚 , 国 运 长 久 。
到 汉 明 帝 开 始 设 立 胡 神 , 然 而 只 有 西 域 的 僧 人 自 己 传 授 佛 教 。
实 际 上 是 夷 人 侵 犯 中 原 , 不 足 为 信 。
太 宗 听 到 这 话 , 就 把 这 个 奏 表 扔 给 群 臣 商 议 。
当 时 有 宰 相 萧 瑀 , 出 班 俯 上 奏 说 : 佛 法 兴 起 于 多 朝 , 弘 扬 善 行 遏 制 恶 行 , 冥 助 国 家 , 理 应 不 废 弃 。
佛 , 是 圣 人 。
不 是 圣 人 无 法 , 请 赦 免 严 刑 。
傅 奕 与 萧 瑀 辩 论 , 说 : 礼 本 于 侍 奉 父 母 , 事 奉 君 王 , 而 佛 背 弃 父 母 出 家 , 以 匹 夫 抗 拒 天 子 , 以 继 承 父 母 的 身 份 违 背 父 母 的 亲 属 。
萧 瑀 不 出 生 在 空 桑 , 却 遵 从 无 父 之 教 , 正 是 所 谓 不 孝 者 无 亲 。
萧 瑀 只 是 合 着 手 说 : 地 狱 的 设 置 , 正 是 为 了 这 个 人 。
太 宗 召 见 太 仆 卿 张 道 源 、 中 书 令 张 士 衡 , 问 道 : 佛 事 营 福 , 其 应 兆 是 怎 样 的 ?
两 位 大 臣 回 答 说 : 佛 在 清 净 仁 恕 , 果 真 正 佛 空 。
周 武 帝 以 三 教 分 为 三 教 , 大 慧 禅 师 赞 颂 幽 远 之 处 , 经 过 众 多 的 供 养 , 没 有 不 显 现 的 ; 五 祖 投 胎 , 达 摩 显 现 的 佛 像 。
自 古 以 来 , 都 说 三 教 至 尊 而 不 可 毁 , 不 可 废 弃 。
请 陛 下 圣 明 鉴 别 裁 定 。
太 宗 很 高 兴 地 说 : 你 的 话 合 乎 道 理 。
再 次 陈 述 的 , 就 治 罪 。
于 是 《 魏 徵 》 与 萧 瑀 、 张 道 源 邀 请 诸 佛 , 选 择 一 名 有 大 德 行 的 人 作 为 坛 主 , 设 立 道 场 。
众 人 都 叩 头 谢 恩 而 退 。
从 此 以 后 , 出 了 法 律 , 只 要 有 毁 谤 僧 人 , 毁 谤 佛 祖 的 , 就 要 砍 断 他 的 手 臂 。
第 二 天 , 三 位 朝 臣 , 聚 集 众 僧 , 在 山 川 坛 里 , 一 个 人 从 头 挑 选 , 从 里 面 选 出 一 个 有 德 行 的 高 僧 。
你 说 他 是 谁 人 ? 灵 通 本 来 名 字 叫 金 蝉 , 只 是 因 为 没 有 心 听 佛 讲 。
转 转 托 身 尘 世 受 苦 受 磨 , 降 生 于 世 俗 , 遭 受 罗 网 。
投 胎 落 地 就 遇 到 凶 险 , 未 出 门 前 就 遭 到 恶 党 。
父 亲 是 海 州 陈 状 元 , 外 公 总 管 当 朝 长 官 。
出 身 命 犯 落 江 星 , 顺 水 随 波 逐 浪 。
海 岛 金 山 有 大 缘 , 迁 安 和 尚 将 他 饲 养 。
年 方 十 八 , 认 亲 娘 , 特 意 赴 京 都 求 外 长 。
总 管 开 山 调 大 军 , 洪 州 剿 寇 , 诛 杀 凶 党 。
状 元 光 蕊 脱 天 罗 , 子 父 相 逢 , 可 以 庆 贺 奖 赏 。
又 拜 谒 当 今 受 皇 上 的 恩 典 , 凌 烟 阁 上 的 贤 人 名 声 响 亮 。
恩 官 不 肯 接 受 , 愿 意 做 和 尚 , 洪 福 和 尚 将 要 道 去 拜 访 。
他 的 小 名 叫 江 流 古 佛 儿 , 法 名 叫 做 陈 玄 奘 。
当 天 对 众 人 推 举 出 玄 奘 法 师 。
从 小 就 是 和 尚 , 出 娘 胎 , 就 持 斋 受 戒 。
外 公 见 , 是 当 朝 一 路 总 管 殷 开 山 。
其 父 陈 光 萼 中 状 元 , 授 文 渊 殿 大 学 士 。
一 心 不 爱 荣 华 , 只 喜 欢 修 持 寂 灭 。
他 的 根 源 又 好 , 德 行 又 高 , 千 经 万 典 , 无 所 不 通 ; 佛 号 仙 音 , 无 所 不 会 。
当 时 三 位 官 员 引 导 到 皇 帝 面 前 , 扬 起 尘 土 舞 蹈 。
行 礼 完 毕 上 奏 说 : 臣 王 禹 等 承 蒙 圣 旨 , 选 拔 得 到 一 个 高 僧 , 一 个 名 叫 陈 玄 奘 。
太 宗 听 到 他 的 名 字 , 沉 思 了 很 久 , 说 : 你 可 以 是 学 士 陈 光 蕊 的 儿 子 玄 奘 吗 ? 江 流 儿 叩 头 说 : 我 正 是 。
太 宗 高 兴 地 说 : 果 然 举 荐 不 错 , 确 实 是 有 德 行 有 禅 心 的 和 尚 。
我 赐 你 左 僧 纲 , 右 僧 纲 , 是 天 下 大 阐 都 僧 纲 的 职 责 。
玄 奘 叩 头 谢 恩 , 接 受 了 大 阐 的 官 爵 。
又 赐 给 他 五 彩 织 金 袈 裟 一 件 、 毗 卢 帽 一 顶 。
教 他 用 心 再 拜 明 僧 , 排 列 次 黎 班 首 , 书 写 旨 意 , 前 往 化 生 寺 , 选 择 吉 日 良 时 , 开 讲 经 法 。
玄 奘 再 拜 后 领 旨 出 来 , 就 到 化 生 寺 里 , 聚 集 了 很 多 僧 人 , 打 造 禅 榻 , 修 建 功 德 , 整 理 音 乐 。
选 得 大 小 明 僧 共 一 千 二 百 名 , 分 派 上 中 下 三 堂 。
各 位 佛 像 前 面 , 各 种 物 品 都 齐 整 , 头 头 有 次 序 。
选 择 到 本 年 九 月 初 三 日 黄 道 吉 日 , 开 启 了 七 七 四 十 九 日 水 陆 大 会 。
于 是 就 上 表 申 奏 。
太 宗 和 文 武 国 戚 皇 亲 , 都 到 约 定 的 日 期 赴 会 , 焚 香 听 讲 。
有 诗 作 证 。
《 诗 经 》 说 : 龙 集 贞 观 正 十 三 , 王 宣 大 众 把 经 谈 。
道 场 开 演 无 量 法 , 云 雾 光 乘 大 愿 龛 。
皇 上 的 敕 令 垂 恩 修 建 上 宇 , 金 蝉 脱 壳 化 西 方 。
普 施 善 果 超 出 沉 没 , 秉 持 教 化 宣 扬 , 前 后 三 次 。
贞 观 十 三 年 , 岁 次 己 巳 , 九 月 甲 戌 , 初 三 日 , 癸 卯 良 辰 , 陈 玄 奘 大 阐 法 师 聚 集 一 千 二 百 名 高 僧 , 都 在 长 安 城 化 生 寺 开 演 各 种 妙 经 。
那 皇 帝 早 朝 已 经 完 毕 , 率 领 文 武 官 员 , 乘 着 凤 龙 车 , 出 离 金 銮 宝 殿 , 径 直 上 寺 来 焚 香 。
哪 里 见 到 那 些 神 灵 的 车 驾 , 真 的 是 一 天 瑞 气 , 万 道 祥 光 。
仁 风 轻 淡 荡 荡 , 化 日 丽 非 常 。
千 官 环 佩 分 开 前 后 , 五 卫 旌 旗 排 列 在 两 旁 。
手 执 金 瓜 , 手 持 斧 , 双 双 对 对 , 绛 纱 烛 , 御 炉 香 , 云 气 飘 扬 。
龙 飞 凤 舞 , 荐 鹰 扬 。
圣 明 天 子 正 , 忠 义 大 臣 良 。
福 佑 千 年 超 过 舜 禹 , 升 平 万 代 比 尧 汤 。
又 看 见 那 曲 柄 伞 , 滚 龙 袍 , 光 辉 照 耀 , 玉 连 环 , 彩 凤 扇 , 瑞 气 飘 扬 。
珠 冠 玉 带 , 紫 色 绶 带 金 章 。
护 驾 千 队 , 扶 舆 两 行 。
这 位 皇 帝 沐 浴 , 虔 诚 地 敬 奉 佛 , 虔 诚 地 祈 祷 佛 , 喜 欢 捧 香 。
唐 王 大 驾 早 到 寺 前 , 吩 咐 住 下 来 。
下 了 车 辆 , 引 着 很 多 官 员 , 拜 佛 、 拈 香 。
三 匝 已 经 完 毕 , 抬 头 观 看 , 果 然 是 个 好 地 坐 在 道 场 上 。
只 见 幢 幡 飘 舞 , 宝 盖 飞 光 。
幢 幡 飘 舞 , 凝 空 道 道 , 彩 霞 摇 动 ; 宝 盖 飞 辉 , 映 日 翩 翩 , 红 电 闪 烁 。
世 尊 金 像 的 形 貌 非 常 美 好 , 罗 汉 玉 容 的 声 威 威 武 赫 赫 。
瓶 插 仙 花 , 炉 焚 檀 香 。
瓶 插 着 仙 花 , 锦 树 辉 煌 , 飘 满 宝 座 ; 炉 焚 檀 香 , 香 云 霭 霭 穿 透 清 霄 。
当 时 新 的 果 品 砌 成 红 色 的 盘 子 , 奇 异 的 糖 糖 堆 在 彩 案 上 。
高 僧 罗 列 诵 读 真 经 , 愿 意 解 脱 孤 魂 离 开 苦 难 。
太 宗 文 武 官 员 都 各 自 捧 香 , 拜 了 佛 祖 金 身 , 参 拜 了 罗 汉 。
又 见 那 大 阐 都 纲 、 陈 玄 奘 法 师 , 带 领 众 僧 人 罗 拜 唐 王 。
礼 仪 结 束 , 分 班 各 安 置 禅 位 。
法 师 献 上 济 孤 的 榜 文 给 太 宗 看 。
榜 文 说 : 至 德 渺 茫 , 禅 宗 寂 灭 。
清 净 灵 通 , 遍 及 三 界 。
千 变 万 化 , 统 领 阴 阳 。
体 用 真 常 , 无 穷 无 尽 。
看 到 他 的 孤 魂 , 深 深 地 应 该 哀 怜 怜 悯 。
这 是 奉 太 宗 的 圣 命 : 选 集 众 僧 , 参 禅 讲 法 。
大 开 方 便 门 庭 , 广 运 慈 悲 舟 楫 , 普 济 苦 海 众 生 , 免 除 沉 六 道 。
引 导 归 真 路 , 普 遍 欣 赏 鸿 濛 , 行 止 无 为 , 混 成 纯 朴 。
凭 借 着 这 些 好 机 会 , 邀 赏 清 都 绛 阙 , 乘 我 胜 会 , 脱 离 地 狱 的 大 笼 子 。
早 日 登 上 极 乐 , 任 凭 逍 遥 , 来 往 西 方 随 自 在 。
诗 说 : 一 炉 永 寿 香 , 几 卷 超 生 。
无 边 妙 法 宣 扬 , 无 际 天 恩 沐 浴 。
冤 孽 全 部 消 除 , 孤 魂 都 出 狱 。
希 望 保 全 我 们 的 国 家 , 清 平 万 民 福 。
一 宿 晚 上 , 景 色 已 经 过 了 。
第 二 天 早 上 , 法 师 又 登 上 座 位 , 聚 集 众 人 诵 读 经 书 , 没 有 题 字 。
又 说 : 南 海 普 陀 山 观 世 音 菩 萨 , 自 己 领 悟 了 如 来 佛 的 旨 意 , 在 长 安 城 去 访 求 佛 经 的 善 人 , 时 间 久 了 , 没 有 遇 到 真 正 有 德 行 的 人 。
忽 然 听 说 太 宗 宣 扬 善 果 , 选 拔 高 僧 , 举 行 大 会 。
又 见 到 法 师 的 坛 主 , 就 是 江 流 儿 和 尚 , 正 是 从 极 乐 中 降 下 来 的 佛 子 , 又 是 他 原 来 引 导 送 他 投 胎 的 长 老 。
于 是 , 大 喜 , 大 喜 , 就 把 佛 教 的 宝 贝 捧 上 大 街 上 , 与 木 叉 买 卖 。
你 说 他 是 什 么 宝 贝 ? 还 有 一 件 锦 异 宝 袈 裟 、 九 环 锡 杖 。
还 有 那 金 紧 禁 的 三 个 钳 儿 , 秘 密 收 藏 起 来 , 等 待 以 后 再 使 用 。
他 只 把 袈 裟 、 锡 杖 出 卖 。
长 安 城 里 , 有 个 挑 选 不 中 的 愚 僧 , 倒 有 几 贯 村 里 的 钱 钞 。
他 看 见 菩 萨 变 了 一 个 疮 的 形 状 , 身 穿 破 鞋 , 红 脚 光 着 头 , 把 袈 裟 捧 定 , 艳 丽 发 光 。 他 上 前 问 道 : 那 和 尚 , 你 的 袈 裟 要 卖 得 多 少 价 钱 ? 菩 萨 说 : 袈 裟 价 值 五 千 两 , 锡 杖 价 值 二 千 两 。
那 个 愚 蠢 的 和 尚 笑 着 说 : 这 两 个 和 尚 是 个 疯 子 , 是 个 痴 子 , 这 两 件 粗 糙 的 东 西 , 就 卖 得 七 千 两 银 子 , 只 是 除 非 穿 上 身 , 长 生 不 老 , 就 可 以 成 佛 作 祖 , 也 可 以 得 不 到 这 许 多 , 拿 了 去 , 卖 不 成 。 那 菩 萨 更 不 争 吵 , 和 木 叉 往 前 又 走 。
过 了 很 长 时 间 , 来 到 东 华 门 前 , 正 撞 着 宰 相 萧 王 禹 散 朝 回 来 。
那 菩 萨 公 然 不 回 避 , 在 街 上 拿 着 袈 裟 , 径 直 迎 接 宰 相 。
宰 相 勒 马 观 看 , 见 袈 裟 艳 丽 发 光 , 手 下 人 问 那 些 卖 袈 裟 的 要 价 是 多 少 ? 菩 萨 说 : 袈 裟 要 五 千 两 , 锡 杖 要 二 千 两 。
萧 瑀 说 : 有 什 么 好 的 地 方 , 可 以 得 到 这 样 的 高 价 啊 菩 萨 说 : 袈 裟 有 好 的 地 方 , 有 不 好 的 地 方 ; 有 要 钱 的 地 方 , 有 不 要 钱 的 地 方 。
萧 瑀 说 : 什 么 好 , 什 么 不 好 啊 菩 萨 说 : 穿 上 了 我 袈 裟 , 不 堕 入 地 狱 , 不 遭 恶 毒 的 灾 难 , 不 遭 虎 狼 的 灾 难 , 就 是 好 处 ; 如 果 贪 淫 乐 祸 的 愚 僧 , 不 斋 戒 的 和 尚 , 毁 坏 经 书 诽 谤 佛 祖 的 凡 夫 , 难 以 见 到 我 袈 裟 的 面 孔 , 这 就 是 不 好 处 。
又 问 道 : 为 什 么 要 钱 , 不 要 钱 ? 菩 萨 说 : 不 遵 佛 法 , 不 敬 三 宝 , 强 买 袈 裟 、 锡 杖 , 一 定 要 卖 它 七 千 两 , 这 就 是 要 钱 ; 如 果 敬 重 三 宝 , 见 善 就 随 喜 , 就 可 以 承 受 佛 法 , 承 受 了 。 我 将 袈 裟 、 锡 杖 送 给 他 , 与 我 结 成 善 缘 , 这 就 是 不 要 钱 。
萧 瑀 听 了 他 的 话 , 更 加 增 添 了 春 色 , 知 道 他 是 个 好 人 。
于 是 就 下 马 , 和 菩 萨 以 礼 相 见 , 并 口 说 : 大 法 长 老 , 宽 恕 我 萧 王 禹 的 罪 过 。
大 唐 皇 帝 十 分 喜 好 善 , 满 朝 文 武 官 员 , 没 有 不 奉 行 的 。
现 在 起 建 水 陆 大 会 , 这 袈 裟 正 好 与 大 都 阐 陈 玄 奘 法 师 穿 用 。
吾 与 你 一 起 入 朝 拜 见 皇 帝 , 看 见 皇 帝 的 去 来 。
菩 萨 欣 然 跟 着 他 , 拽 着 他 转 步 , 径 直 走 进 东 华 门 里 。
黄 门 官 转 奏 , 承 蒙 圣 旨 宣 到 宝 殿 。
看 见 萧 瑀 带 着 两 个 螃 疮 和 尚 站 在 台 阶 下 , 唐 王 问 道 : 萧 瑀 来 奏 报 的 是 什 么 事 ? 萧 瑀 俯 伏 在 台 阶 前 说 : 我 走 出 了 东 华 门 前 , 偶 然 遇 到 两 个 僧 人 , 是 卖 袈 裟 和 锡 杖 的 。
我 想 法 师 玄 奘 可 以 穿 上 这 种 衣 服 , 所 以 领 着 僧 人 的 启 奏 。
太 宗 大 喜 , 便 问 : 那 袈 裟 的 价 值 有 多 少 ?
菩 萨 和 木 叉 站 在 台 阶 下 , 更 不 行 礼 , 于 是 询 问 袈 裟 的 价 格 , 木 叉 回 答 说 : 袈 裟 五 千 两 , 锡 杖 二 千 两 。
太 宗 说 : 那 袈 裟 有 什 么 好 处 , 就 可 以 值 得 许 多 钱 啊 菩 萨 说 : 这 袈 裟 , 龙 穿 一 缕 , 免 除 大 鹏 吞 食 的 灾 难 ; 鹤 挂 一 丝 , 得 到 超 凡 入 圣 的 妙 境 。
只 要 坐 的 地 方 , 有 万 神 朝 拜 的 礼 仪 , 凡 是 举 动 , 有 七 个 佛 随 身 随 身 。
这 袈 裟 , 是 用 冰 蚕 织 成 的 , 抽 丝 抽 丝 , 巧 匠 翻 腾 成 线 , 仙 娥 织 成 , 神 女 织 成 , 方 方 簇 幅 绣 花 缝 。
片 片 相 互 帮 助 , 堆 满 锦 。
玲 珑 散 碎 斗 妆 花 , 颜 色 明 亮 飘 扬 光 彩 , 喷 出 的 珍 贵 艳 丽 。
穿 上 满 身 红 雾 缭 绕 , 脱 下 一 段 彩 云 飞 。
三 天 门 外 透 出 元 光 , 五 岳 山 前 生 出 宝 气 。
重 重 地 镶 嵌 着 西 番 的 莲 花 , 灼 灼 悬 挂 着 的 珠 星 、 斗 星 的 形 象 。
四 角 上 面 有 夜 明 珠 , 堆 在 顶 上 有 一 颗 祖 母 绿 。
虽 然 没 有 完 全 照 照 原 来 的 本 体 , 但 也 有 生 光 八 宝 聚 集 。
这 件 袈 裟 , 闲 时 折 合 , 遇 圣 才 穿 。
他 闲 暇 时 , 就 把 它 折 成 , 千 层 包 裹 , 穿 透 了 虹 霓 ; 遇 到 圣 人 才 能 挖 开 , 惊 动 了 天 神 鬼 怪 。
上 边 有 如 意 珠 、 摩 尼 珠 、 辟 尘 珠 、 定 风 珠 , 又 有 那 红 玛 瑙 、 紫 珊 瑚 、 夜 明 珠 、 舍 利 子 。
偷 偷 月 光 沁 沁 白 色 , 与 太 阳 争 红 色 。
一 股 仙 气 充 满 空 中 , 一 朵 祥 光 捧 着 圣 上 。
一 股 仙 气 充 满 空 中 , 照 彻 了 天 关 ; 一 朵 祥 光 捧 着 圣 人 , 影 子 遍 布 了 世 界 。
照 山 川 , 惊 虎 豹 , 影 海 岛 , 动 鱼 龙 。
沿 边 两 道 销 金 锁 , 叩 领 连 环 白 玉 琮 。
《 诗 经 》 说 : 三 宝 巍 巍 道 可 尊 , 四 生 六 道 尽 评 论 。
明 心 明 理 修 养 人 天 的 法 则 , 明 了 自 己 的 本 性 就 能 传 授 智 慧 灯 。
保 护 身 体 庄 严 金 世 界 , 身 心 清 净 玉 壶 冰 。
自 从 佛 祖 制 袈 裟 以 后 , 万 劫 之 中 谁 敢 断 绝 和 尚 ? 唐 王 听 了 这 话 , 立 即 命 人 打 开 袈 裟 , 从 头 细 看 , 果 然 是 一 件 好 东 西 。
他 说 : 大 法 长 老 , 实 在 不 骗 你 。
我 现 在 大 开 善 教 , 广 种 福 田 , 现 在 在 那 化 生 寺 聚 集 了 很 多 僧 人 , 弘 扬 佛 法 。
其 中 有 一 个 很 有 德 行 的 人 , 法 名 叫 玄 奘 。
朕 买 了 你 的 两 件 宝 物 , 赐 给 他 们 受 用 。
菩 萨 听 了 这 话 , 便 与 木 叉 合 掌 皈 依 , 声 音 佛 号 , 亲 自 上 启 说 : 既 然 有 德 行 , 贫 僧 愿 送 他 , 决 不 要 钱 。
说 完 , 抽 出 身 子 就 走 了 。
唐 王 急 着 把 萧 瑀 拉 住 , 怏 怏 地 站 在 殿 上 , 问 道 : 你 原 来 说 袈 裟 五 千 两 , 锡 杖 二 千 两 , 你 见 朕 要 买 , 就 不 要 钱 , 岂 敢 说 朕 心 里 倚 仗 你 的 地 位 , 强 要 你 的 物 品 , 再 没 有 这 种 道 理 。
我 照 你 原 来 的 价 钱 来 赔 偿 , 却 不 可 以 推 避 。
菩 萨 起 手 说 : 贫 僧 有 愿 望 在 前 面 , 希 望 说 果 真 有 敬 重 三 宝 , 见 善 就 随 喜 , 愿 意 皈 依 我 佛 , 不 要 钱 , 愿 送 给 他 。
现 在 看 到 陛 下 明 德 止 善 , 敬 重 我 佛 门 , 何 况 高 僧 有 德 行 , 宣 扬 佛 法 , 理 应 奉 上 , 决 不 要 钱 。
贫 僧 愿 留 下 这 些 东 西 告 诉 我 回 去 。
唐 王 见 他 如 此 诚 恳 , 非 常 高 兴 。
随 即 任 命 光 禄 寺 , 大 摆 素 宴 酬 谢 。
尉 迟 菩 萨 又 坚 决 辞 谢 , 不 肯 接 受 , 随 即 离 去 , 依 旧 在 望 都 土 地 庙 中 隐 居 不 再 题 写 。
又 说 太 宗 设 午 朝 , 著 述 魏 徵 的 旨 意 , 宣 布 玄 奘 入 朝 。
那 法 师 正 聚 集 众 人 登 上 佛 坛 , 诵 读 经 书 诵 诵 偈 语 , 一 听 就 有 旨 意 , 随 即 下 到 佛 坛 整 理 衣 服 , 与 魏 征 一 同 前 去 拜 见 皇 帝 。
太 宗 说 : 求 证 善 事 , 有 劳 法 师 , 无 物 酬 谢 。
早 些 时 候 , 萧 瑀 迎 接 二 位 和 尚 , 希 望 送 给 他 一 套 锦 缎 、 异 宝 袈 裟 、 九 环 锡 杖 一 支 。
现 在 特 地 召 见 法 师 , 领 他 去 受 用 。
玄 奘 叩 头 谢 恩 。
太 宗 说 : 法 师 如 果 不 抛 弃 , 可 以 穿 上 去 给 我 看 看 。
于 是 长 老 就 把 袈 裟 抖 开 , 披 在 身 上 , 手 里 拿 着 锡 杖 , 侍 立 在 阶 前 。
君 臣 之 间 , 各 自 欢 喜 。
诚 然 是 如 来 佛 子 。
你 看 他 , 凛 凛 威 容 , 多 雅 秀 , 佛 衣 可 以 身 体 如 裁 剪 。
光 辉 艳 丽 充 满 乾 坤 , 彩 虹 缤 纷 凝 结 宇 宙 。
明 珠 上 下 排 列 , 层 层 金 线 穿 透 前 后 。
兜 罗 四 面 锦 绣 , 沿 边 的 锦 缎 , 万 种 稀 奇 , 铺 上 了 绮 绣 。
八 宝 妆 花 , 缚 钮 丝 , 金 环 束 领 攀 绒 。
佛 天 的 大 小 分 别 高 低 , 星 象 的 尊 卑 分 别 左 右 。
玄 奘 法 师 非 常 有 缘 缘 , 现 在 我 现 在 的 这 些 东 西 可 以 承 受 。
浑 如 极 乐 活 阿 罗 , 赛 过 西 方 真 觉 秀 。
锡 杖 铿 锵 斗 九 环 , 毗 卢 帽 映 照 得 丰 厚 。
如 果 真 是 佛 子 不 虚 传 , 胜 似 菩 提 , 没 有 欺 诈 。
当 时 文 武 官 员 都 在 阶 前 都 很 高 兴 。
太 宗 非 常 高 兴 , 就 穿 着 法 师 袈 裟 , 拿 着 宝 杖 , 又 赐 给 两 队 仪 从 , 让 他 们 上 大 街 行 道 , 往 寺 里 去 , 就 像 中 状 元 夸 官 一 样 。
去 玄 奘 再 拜 谢 恩 , 在 那 大 街 上 , 发 出 激 烈 的 声 音 , 摇 摇 摇 动 。
你 看 那 些 长 安 城 里 的 商 人 、 公 子 王 孙 、 墨 客 、 文 人 、 大 男 小 女 , 没 有 不 争 相 夸 耀 , 都 说 : 好 的 法 师 , 真 是 个 活 罗 汉 下 降 , 活 菩 萨 临 世 。
玄 奘 径 直 来 到 寺 里 , 僧 人 下 榻 来 迎 接 。
一 见 到 他 披 着 袈 裟 , 拿 着 锡 杖 , 都 说 是 地 藏 王 来 了 , 各 自 归 依 , 在 左 右 侍 奉 。
玄 奘 上 殿 , 点 燃 香 礼 佛 。
又 对 众 人 感 激 述 说 圣 恩 已 经 完 毕 , 各 自 回 到 禅 座 。
又 不 觉 红 色 的 车 轮 向 西 坠 落 。
正 是 那 样 , 日 落 烟 迷 草 树 , 帝 都 钟 鼓 初 鸣 。
呜 呜 三 声 , 断 人 行 。
前 后 街 前 寂 静 无 声 。
上 层 楼 阁 辉 煌 灯 火 , 孤 村 寂 静 无 声 。
禅 僧 入 定 , 研 究 残 经 。
正 好 炼 魔 修 身 养 性 。
光 阴 凝 聚 , 却 应 当 七 天 正 会 。
玄 奘 又 上 表 请 唐 王 焚 香 。
这 时 , 善 声 遍 布 天 下 。
太 宗 立 即 排 驾 , 率 领 文 武 多 官 、 后 妃 国 戚 , 早 早 赶 到 寺 里 。
那 一 城 的 人 , 不 论 大 小 尊 卑 , 都 到 寺 院 听 讲 。
当 有 位 菩 萨 和 木 叉 说 : 今 天 是 水 陆 交 会 , 用 一 七 个 接 着 七 七 个 , 就 可 以 了 。
我 和 你 杂 在 众 人 之 中 , 一 是 看 他 的 那 样 , 二 是 看 金 蝉 子 可 有 福 , 穿 我 的 宝 贝 , 三 是 听 他 讲 的 是 那 一 门 经 法 。
两 个 人 随 即 投 到 寺 里 。
正 是 因 为 有 缘 得 以 遇 到 旧 相 识 , 般 若 还 归 本 道 场 。
进 入 寺 里 观 看 , 真 的 是 天 朝 大 国 , 果 胜 袈 裟 婆 。
赛 过 祇 园 舍 卫 , 也 不 亚 于 上 刹 招 提 。
那 一 派 仙 人 的 声 音 响 亮 , 佛 号 喧 哗 。
这 位 菩 萨 直 到 多 宝 台 边 , 果 然 是 明 智 金 蝉 的 神 像 。
《 诗 经 》 说 : 万 象 澄 明 绝 点 尘 埃 , 大 典 玄 奘 坐 高 台 。
超 生 孤 魂 暗 中 到 来 , 听 法 高 流 市 上 来 。
施 物 应 机 , 心 路 遥 远 , 出 生 随 意 , 藏 门 开 。
对 着 观 看 , 讲 出 无 量 的 法 则 , 老 幼 人 人 都 心 怀 喜 悦 。
又 有 诗 道 : 因 游 法 界 讲 堂 中 , 逢 见 相 知 不 俗 同 。
尽 说 眼 前 的 千 万 事 , 又 谈 论 尘 世 的 许 多 功 德 。
法 云 的 容 貌 舒 展 群 岳 , 教 网 张 罗 满 太 空 。
检 点 人 生 归 善 念 , 纷 纷 天 雨 落 花 红 。
那 法 师 在 台 上 念 一 会 《 受 生 度 亡 经 》 , 谈 一 遍 《 安 邦 天 宝 篆 》 , 又 宣 读 一 卷 《 劝 修 功 卷 》 。
这 位 菩 萨 从 近 前 来 , 拍 着 宝 台 , 厉 声 大 叫 道 : 那 和 尚 , 你 只 会 谈 小 乘 教 法 , 难 道 可 以 谈 大 乘 吗 ? 玄 奘 听 了 , 心 中 非 常 高 兴 , 翻 身 跳 下 台 来 , 对 菩 萨 伸 手 说 : 老 师 父 , 弟 子 失 去 瞻 仰 多 罪 。
见 前 面 的 大 众 和 尚 , 都 讲 的 是 小 乘 法 , 却 不 知 道 大 乘 教 法 是 怎 样 的 ?
菩 萨 说 : 你 这 个 小 乘 教 法 , 度 不 得 灭 亡 的 人 超 升 , 只 可 以 浑 俗 和 光 而 已 。
我 有 大 乘 佛 法 三 藏 , 能 超 脱 亡 者 升 天 , 能 度 脱 苦 难 的 人 , 能 修 炼 无 量 寿 身 , 能 做 无 来 无 去 。
正 在 讲 经 的 地 方 , 有 个 叫 那 司 香 巡 堂 的 官 急 忙 上 奏 唐 王 说 : 法 师 正 在 讲 论 妙 法 , 被 两 个 疥 癣 的 游 僧 拽 下 来 , 乱 说 胡 话 。
阎 王 命 令 把 他 抓 来 。
只 见 有 许 多 人 把 两 个 和 尚 推 进 后 法 堂 , 见 了 太 宗 , 那 个 僧 人 手 也 不 起 , 拜 也 不 拜 , 仰 面 说 : 陛 下 问 我 什 么 事 , 唐 王 却 认 得 他 , 说 : 你 是 前 日 送 袈 裟 的 和 尚 ? 菩 萨 说 : 正 是 。
太 宗 说 : 你 既 然 来 此 处 听 讲 , 只 应 吃 些 斋 就 了 , 为 什 么 和 我 法 师 乱 讲 , 扰 乱 经 堂 , 误 了 我 佛 事 呢 ?
我 有 大 乘 佛 法 三 藏 , 可 以 度 脱 苦 难 , 寿 命 永 远 不 坏 。
太 宗 神 色 高 兴 地 问 道 : 你 们 大 乘 佛 法 在 哪 里 ? 菩 萨 说 : 在 大 西 天 竺 国 大 雷 音 寺 , 能 解 除 百 冤 的 冤 情 , 能 消 除 无 妄 的 灾 难 。
太 宗 说 : 你 可 以 记 得 吗 ? 菩 萨 说 : 我 记 得 。
太 宗 非 常 高 兴 地 说 : 教 法 师 引 导 我 去 , 请 上 台 开 讲 。
那 菩 萨 带 着 木 叉 , 飞 上 高 台 , 就 踏 着 祥 云 , 直 到 九 霄 , 现 出 救 苦 的 本 身 , 托 付 了 净 瓶 杨 柳 。
左 边 是 个 木 叉 惠 岸 , 拿 着 棍 子 , 抖 擞 精 神 。
喜 的 唐 王 朝 天 行 礼 拜 礼 , 众 文 武 官 员 跪 地 焚 香 。
满 寺 中 的 僧 尼 、 尼 姑 、 道 俗 、 士 人 、 工 匠 、 商 人 , 没 有 一 个 人 不 跪 拜 祈 祷 说 : 好 菩 萨 , 好 菩 萨 啊 。
只 见 那 些 祥 瑞 的 云 气 散 开 缤 纷 , 祥 光 护 护 法 身 。
九 霄 华 汉 里 , 现 出 女 真 人 。
那 菩 萨 , 头 上 戴 着 一 顶 金 叶 纽 、 翠 花 铺 、 放 出 金 光 , 生 祥 瑞 气 的 垂 珠 缨 络 ; 身 上 穿 着 淡 淡 色 、 浅 浅 妆 、 盘 金 龙 、 飞 彩 凤 的 结 素 袍 袍 ; 胸 前 挂 着 一 面 对 月 明 、 舞 清 风 、 杂 宝 珠 、 攒 翠 玉 的 砌 香 环 佩 ; 腰 间 系 着 一 条 冰 蚕 丝 、 织 金 边 、 织 彩 云 、 催 促 瑶 海 的 锦 绣 丝 裙 ; 面 前 又 领 着 一 个 飞 东 洋 、 遨 游 遍 世 、 感 恩 德 行 孝 道 。
手 里 有 一 个 施 恩 救 世 的 宝 瓶 , 瓶 里 插 着 一 枝 洒 青 霄 , 撒 大 恶 , 扫 开 残 雾 垂 杨 柳 。
玉 环 穿 穿 绣 带 , 金 莲 脚 下 深 。
三 天 后 才 出 入 。
其 所 以 是 , 乃 是 救 苦 救 难 的 观 世 音 。
喜 欢 的 唐 太 宗 忘 了 江 山 , 喜 爱 的 那 文 武 官 员 失 去 朝 廷 礼 仪 , 大 概 是 许 多 人 都 念 南 无 观 世 音 菩 萨 。
太 宗 立 即 传 旨 , 教 巧 手 绘 画 菩 萨 的 真 像 。
圣 旨 一 声 , 选 出 一 个 画 神 画 圣 、 远 见 高 明 的 吴 道 子 , 这 个 人 就 是 后 来 在 凌 烟 阁 画 功 臣 的 人 。
当 时 展 开 妙 笔 , 画 出 真 形 。
那 菩 萨 的 祥 云 渐 渐 离 开 了 , 一 会 儿 就 看 不 见 了 金 光 。
只 见 在 半 空 中 滴 滴 滴 下 一 张 简 帖 , 上 面 有 几 句 颂 子 , 写 得 十 分 明 白 。
颂 道 : 礼 上 大 唐 君 , 西 方 有 美 妙 的 文 章 。
行 程 十 万 八 千 里 , 大 乘 进 军 。
此 经 回 到 上 国 , 能 超 出 鬼 群 。
如 果 有 人 肯 离 开 的 , 请 求 正 果 金 身 。
太 宗 见 了 颂 子 , 立 即 命 令 众 位 僧 人 说 : 暂 且 收 集 胜 会 , 等 我 派 人 取 得 《 大 乘 经 》 来 , 再 次 秉 持 丹 诚 , 重 修 善 果 。
众 官 员 无 不 遵 从 。
当 时 在 寺 中 问 道 : 谁 肯 领 我 的 旨 意 , 上 西 天 拜 佛 求 经 ? 问 不 了 , 旁 边 闪 过 法 师 , 皇 帝 上 前 施 礼 说 : 贫 僧 没 有 才 能 , 愿 效 犬 马 之 劳 , 与 陛 下 求 取 真 经 , 祈 求 保 证 我 王 的 江 山 永 远 巩 固 。
唐 王 大 喜 , 皇 上 上 前 将 御 手 扶 起 来 说 : 法 师 果 真 能 把 忠 贤 尽 忠 , 不 怕 路 途 遥 远 , 跋 涉 山 川 , 我 愿 意 与 你 拜 为 兄 弟 。
玄 奘 叩 头 谢 恩 。
唐 王 果 然 是 十 分 贤 德 , 就 走 到 那 寺 里 的 佛 像 前 , 和 玄 奘 拜 了 四 拜 , 口 称 : 御 弟 圣 僧 。
玄 奘 感 谢 不 尽 地 说 道 : 陛 下 , 贫 僧 有 什 么 德 行 , 怎 么 敢 承 蒙 天 恩 眷 顾 我 这 样 , 我 这 一 去 , 必 定 要 捐 躯 竭 力 , 直 到 西 天 , 得 不 到 西 天 , 得 不 到 真 经 , 即 使 死 也 不 敢 回 国 , 永 远 堕 入 地 狱 。
随 即 在 佛 前 点 香 , 以 此 为 誓 。
唐 王 非 常 高 兴 , 立 即 命 令 他 们 回 去 , 等 到 选 择 好 的 日 子 , 发 出 文 书 出 发 , 于 是 驾 回 各 自 散 去 。
玄 奘 也 回 到 洪 福 寺 里 。
那 本 寺 里 有 很 多 僧 人 和 几 个 弟 弟 , 早 就 听 到 佛 经 的 事 , 都 来 相 见 , 于 是 问 他 : 发 誓 愿 上 西 天 , 确 实 吗 ? 玄 奘 说 : 是 真 实 的 。
他 的 弟 弟 说 : 师 父 呵 , 曾 经 听 人 说 , 西 天 路 远 , 更 多 虎 豹 妖 魔 。
只 怕 有 去 无 回 , 难 保 身 命 。
玄 奘 说 : 我 已 经 发 了 大 誓 大 愿 , 不 取 真 经 , 永 堕 沉 沦 地 狱 。
大 抵 是 受 王 的 恩 宠 , 不 得 不 尽 忠 报 国 罢 了 。
我 此 去 , 真 是 渺 茫 茫 茫 , 吉 凶 难 定 。
又 说 : 徒 弟 们 , 我 去 之 后 , 有 时 三 二 年 , 有 时 五 七 年 , 只 看 那 山 门 里 的 松 树 枝 头 向 东 , 我 立 即 回 来 , 不 然 , 就 绝 不 回 去 了 。
众 位 徒 弟 将 这 些 话 切 切 地 记 下 来 。
次 早 , 太 宗 设 朝 , 召 集 文 武 官 员 , 写 了 取 经 文 牒 , 使 用 通 行 宝 印 。
有 个 钦 天 监 上 奏 说 : 今 天 是 人 专 占 吉 星 , 可 以 出 去 远 道 。
唐 王 大 喜 。
又 见 黄 门 官 上 奏 说 : 御 弟 法 师 在 门 外 听 候 圣 旨 。
随 即 在 宝 殿 宣 读 道 : 御 弟 , 今 天 是 出 行 的 吉 日 。
其 所 以 不 得 之 , 则 不 得 之 。
朕 又 有 一 个 紫 金 钵 盂 , 送 你 途 中 化 斋 使 用 。
再 选 两 个 长 行 的 随 从 。
又 有 一 匹 银 马 , 送 给 我 远 行 的 脚 力 。
你 可 以 就 此 行 。
玄 奘 非 常 高 兴 , 立 即 谢 了 恩 , 领 会 了 事 情 , 再 也 没 有 留 滞 的 意 思 。
唐 王 排 着 车 驾 , 与 多 官 一 同 送 到 关 外 。
只 见 那 洪 福 寺 的 僧 人 和 众 位 僧 人 带 着 玄 奘 的 冬 夏 衣 服 , 都 送 到 关 外 。
唐 王 见 了 , 先 教 他 收 拾 行 李 、 马 匹 , 然 后 让 官 员 拿 着 酒 壶 斟 酒 。
太 宗 举 起 酒 杯 , 又 问 道 : 御 弟 的 雅 号 很 称 呼 , 玄 奘 说 : 我 是 个 贫 僧 出 家 人 , 不 敢 称 号 。
太 宗 说 : 当 时 的 菩 萨 说 , 西 天 有 经 过 三 藏 。
御 弟 可 指 着 经 书 取 号 , 号 作 三 藏 , 怎 么 样 ? 玄 奘 又 谢 恩 , 接 过 御 酒 说 : 陛 下 , 酒 是 僧 家 的 一 戒 , 贫 僧 自 己 是 人 , 不 会 喝 酒 。
太 宗 说 : 今 天 的 行 动 , 与 其 他 的 事 情 不 同 , 这 是 素 酒 , 只 喝 这 一 杯 , 以 尽 我 奉 送 的 心 意 。
三 藏 不 敢 不 接 受 , 接 过 酒 , 正 等 要 喝 酒 , 只 见 太 宗 低 着 头 , 用 御 手 拾 起 一 撮 尘 土 , 弹 进 酒 中 。
三 藏 不 理 解 他 的 意 思 , 太 宗 笑 着 说 : 御 弟 呵 , 这 一 去 , 到 西 天 , 多 时 可 以 回 来 ? 三 藏 说 : 只 在 三 年 , 径 直 回 上 国 ?
太 宗 说 : 我 时 间 久 了 , 年 龄 已 经 很 久 了 , 山 路 遥 远 , 你 可 以 进 献 此 酒 , 宁 愿 恋 本 乡 一 捻 土 地 , 也 不 要 爱 别 乡 万 两 金 子 。
三 藏 这 才 明 白 捻 土 的 意 思 , 又 谢 恩 喝 完 了 , 辞 谢 出 关 后 离 开 了 。
唐 王 驾 车 回 京 。
我 最 终 不 知 道 此 去 如 何 , 暂 且 听 下 回 分 解 。
}\switchcolumn\flushpage  \begin{pinyinscope}{\myfontt \section{第一三回}     陷虎穴金星解厄  雙叉嶺伯欽留僧

詩曰:
    大有唐王降敕封,欽差玄奘問禪宗。
    堅心磨琢尋龍穴,著意修持上鷲峰。
    邊界遠遊多少國,雲山前度萬千重。
    自今別駕投西去,秉教迦持悟大空。

卻說三藏自貞觀十三年九月望前三日,蒙唐王與多官送出長安關外。一二日馬不
停蹄,早至法門寺。本寺住持上房長老,帶領眾僧有五百餘人,兩邊羅列,接至
裏面,相見獻茶。茶罷進齋,齋後不覺天晚。正是那:
    影動星河近,月明無點塵。
    雁聲鳴遠漢,砧韻響西鄰。
    歸鳥棲枯樹,禪僧講梵音。
    蒲團一榻上,坐到夜將分。

眾僧們燈下議論佛門定旨,上西天取經的原由:有的說水遠山高,有的說路多虎
豹﹔有的說峻嶺陡崖難度,有的說毒魔惡怪難降。三藏箝口不言,但以手指自心
,點頭幾度。眾僧們莫解其意,合掌請問道:「法師指心點頭者,何也?」三藏
答曰:「心生,種種魔生﹔心滅,種種魔滅。我弟子曾在化生寺對佛說下洪誓大
願,不由我不盡此心。這一去,定要到西天,見佛求經,使我們法輪回轉,願聖
主皇圖永固。」眾僧聞得此言,人人稱羨,個個宣揚,都叫一聲:「忠心赤膽大
闡法師!」誇讚不盡,請師入榻安寐。

早又是竹敲殘月落,雞唱曉雲生。那眾僧起來,收拾茶水、早齋。玄奘遂穿了袈
裟,上正殿,佛前禮拜道:「弟子陳玄奘,前往西天取經,但肉眼愚迷,不識活
佛真形。今願立誓:路中逢廟燒香,遇佛拜佛,遇塔掃塔。但願我佛慈悲,早現
丈六金身,賜真經,留傳東土。」祝罷,回方丈進齋。齋畢,那二從者整頓了鞍
馬,促趲行程。三藏出了山門,辭別眾僧。眾僧不忍分別,直送有十里之遙,噙
淚而返。三藏遂直西前進。正是那季秋天氣,但見:
數村木落蘆花碎,幾樹楓楊紅葉墜。路途煙雨故人稀,黃菊麗,山骨細,水寒荷
破人憔悴。白蘋紅蓼霜天雪,落霞孤鶩長空墜。依稀黯淡野雲飛,玄鳥去,賓鴻
至,嘹嘹嚦嚦聲宵碎。

師徒們行了數日,到了鞏州城,早有鞏州合屬官吏人等迎接入城中。安歇一夜,
次早出城前去。一路饑餐渴飲,夜住曉行,兩三日,又至河州衛。此乃是大唐的
山河邊界。早有鎮邊的總兵與本處僧道,聞得是欽差御弟法師上西方見佛,無不
恭敬。接至裏面供給了,著僧綱請往福原寺安歇。本寺僧人,一一參見,安排晚
齋。齋畢,吩咐二從者飽喂馬匹,天不明就行。

及雞方鳴,隨喚從者,卻又驚動寺僧,整治茶湯齋供。齋罷,出離邊界。這長老
心忙,太起早了。原來此時秋深時節,雞鳴得早,只好有四更天氣。一行三人,
連馬四口,迎著清霜,看著明月,行有數十里遠近,見一山嶺,只得撥草尋路,
說不盡崎嶇難走,又恐怕錯了路徑。正疑思之間,忽然失足,三人連馬都跌落坑
坎之中。三藏心慌,從者膽戰。卻才悚懼,又聞得裏面哮吼高呼,叫:「拿將來
!拿將來!」只見狂風滾滾,擁出五六十個妖邪,將三藏、從者揪了上去。這法
師戰戰兢兢的偷眼觀看,上面坐的那魔王十分兇惡。真個是:
雄威身凜凜,猛氣貌堂堂。電目飛光艷,雷聲振四方。鋸牙舒口外,鑿齒露腮旁。
錦繡圍身體,文斑裹脊梁。鋼鬍稀見肉,鉤爪利如霜。東海黃公懼,南山白額王。

諕得個三藏魂飛魄散,二從者骨軟筋麻。魔王喝令綁了。眾妖一齊將三人用繩索
綁縛。正要安排吞食,只聽得外面喧嘩,有人來報:「熊山君與特處士二位來也。」
三藏聞言,抬頭觀看,前走的是一條黑漢。你道他是怎生模樣:
    雄豪多膽量,輕健夯身軀。
    涉水惟兇力,跑林逞怒威。
    向來符吉夢,今獨露英姿。
    綠樹能攀折,知寒善諭時。
    准靈惟顯處,故此號山君。
  又見那後邊來的是一條胖漢。你道怎生模樣:
    嵯峨雙角冠,端肅聳肩背。
    性服青衣穩,蹄步多遲滯。
    宗名父作牯,原號母稱牸。
    能為田者功,因名特處士。

這兩個搖搖擺擺,走入裏面,慌得那魔王奔出迎接。熊山君道:「寅將軍一向得
意,可賀,可賀。」特處士道:「寅將軍丰姿勝常,真可喜,真可喜。」魔王道
:「二公連日如何?」山君道:「惟守素耳。」處士道:「惟隨時耳。」三個敘
罷,各坐談笑。

只見那從者綁得痛切悲啼。那黑漢道:「此三者何來?」魔王道:「自送上門來
者。」處士笑云:「可能待客否?」魔王道:「奉承,奉承。」山君道:「不可
盡用,食其二,留其一可也。」魔王領諾,即呼左右,將二從者剖腹剜心,剁碎
其屍:將首級與心肝奉獻二客,將四肢自食,其餘骨肉分給各妖。只聽得嘓啅之
聲,真似虎啖羊羔,霎時食盡。把一個長老幾乎諕死。這才是初出長安第一場苦
難。

正愴慌之間,漸漸的東方發白。那二怪至天曉方散,俱道:「今日厚擾,容日竭
誠奉酬。」方一擁而退。

不一時,紅日高昇,三藏昏昏沉沉,也辨不得東西南北。正在那不得命處,忽然
見一老叟,手持拄杖而來。走上前,用手一拂,繩索皆斷。對面吹了一口氣,三
藏方甦,跪拜於地道:「多謝老公公,搭救貧僧性命。」老叟答禮道:「你起來
。你可曾疏失了甚麼東西?」三藏道:「貧僧的從人已是被怪食了。只不知行李
、馬匹在於何處?」老叟用杖指道:「那廂不是一匹馬、兩個包袱?」三藏回頭
看時,果是他的物件,並不曾失落,心才略放下些。問老叟曰:「老公公,此處
是甚所在?公公何由在此?」老叟道:「此是雙叉嶺,乃虎狼巢穴處。你為何墮
此?」三藏道:「貧僧雞鳴時,出河州衛界,不料起得早了,冒霜撥露,忽失落
此地。見一魔王,兇頑太甚,將貧僧與二從者綁了。又見一條黑漢,稱是熊山君﹔
一條胖漢,稱是特處士:走進來,稱那魔王是寅將軍。他三個把我二從者吃了,
天光才散。不想我是那裏有這大緣大分,感得老公公來此救我?」老叟道:「處
士者,是個野牛精;山君者,是個熊羆精;寅將軍者,是個老虎精。左右妖邪,
盡都是山精樹鬼、怪獸蒼狼。只因你的本性元明,所以吃不得你。你跟我來,引
你上路。」

三藏不勝感激,將包袱捎在馬上,牽著韁繩,相隨老叟徑出了坑坎之中,走上大
路。卻將馬拴在道旁草頭上,轉身拜謝那公公,那公公遂化作一陣清風,跨一隻
硃頂白鶴,騰空而去。只見風飄飄遺下一張簡帖,書上四句頌子。頌子云:
    吾乃西天太白星,特來搭救汝生靈。
    前行自有神徒助,莫為艱難報怨經。

三藏看了,對天禮拜道:「多謝金星,度脫此難。」拜畢,牽了馬匹,獨自個孤
孤悽悽,往前苦進。這嶺上,真個是:
寒颯颯雨林風,響潺潺澗下水。香馥馥野花開,密叢叢亂石磊。鬧嚷嚷鹿與猿,
一隊隊獐和麂。喧雜雜鳥聲多,靜悄悄人事靡。那長老,戰兢兢心不寧;這馬兒
,力怯怯蹄難舉。

三藏捨身拚命,上了那峻嶺之間。行經半日,更不見個人煙村舍。一則腹中饑了,
二則路又不平。正在危急之際,只見前面有兩隻猛虎咆哮,後邊有幾條長蛇盤繞。
左有毒蟲,右有怪獸。三藏孤身無策,只得放下身心,聽天所命。又無奈那馬腰
軟蹄彎,即便跪下,伏倒在地,打又打不起,牽又牽不動。苦得個法師襯身無地
,真個有萬分悽楚,已自分必死,莫可奈何。

卻說他雖有災迍,卻有救應。正在那不得命處,忽然見毒蟲奔走,妖獸飛逃,猛
虎潛蹤,長蛇隱跡。三藏抬頭看時,只見一人,手執鋼叉,腰懸弓箭,自那山坡
前轉出,果然是一條好漢。你看他:
頭上戴一頂艾葉花斑豹皮帽,身上穿一領羊絨織錦叵羅衣,腰間束一條獅蠻帶,
腳下屣一對麂皮靴。環眼圓睛如弔客,圈鬚亂擾似河奎。懸一囊毒藥弓矢,拿一
桿點鋼大叉。雷聲震破山蟲膽,勇猛驚殘野雉魂。

三藏見他來得漸近,跪在路傍,合掌高叫道:「大王救命!大王救命!」那條漢
到邊前,放下鋼叉,用手攙起道:「長老休怕。我不是歹人,我是這山中的獵戶
,姓劉名伯欽,綽號鎮山太保。我才自來,要尋兩隻山蟲食用。不期遇著你,多
有沖撞。」三藏道:「貧僧是大唐駕下欽差,往西天拜佛求經的和尚。適間來到
此處,遇著些狼虎蛇蟲,四邊圍繞,不能前進。忽見太保來,眾獸皆走,救了貧
僧性命,多謝,多謝。」伯欽道:「我在這裏住人,專倚打些狼虎為生,捉些蛇
蟲過活,故此眾獸怕我走了。你既是唐朝來的,與我都是鄉里。此間還是大唐的
地界,我也是唐朝的百姓,我和你同食皇王的水土,誠然是一國之人。你休怕,
跟我來,到我舍下歇馬,明朝我送你上路。」三藏聞言,滿心歡喜,謝了伯欽,
牽馬隨行。

過了山坡,又聽得呼呼風響。伯欽道:「長老休走,坐在此間。風響處,是個山
貓來了,等我拿他家去管待你。」三藏見說,又膽戰心驚,不敢舉步。那太保執
了鋼叉,拽開步,迎將上去。只見一隻斑斕虎,對面撞見,他看見伯欽,急回頭
就走。這太保霹靂一聲,咄道:「那業畜那裏走!」那虎見趕得急,轉身掄爪撲
來。這太保三股叉舉手迎敵。諕得個三藏軟癱在草地。這和尚自出娘肚皮,那曾
見這樣凶險的勾當。太保與那虎在那山坡下,人虎相持,果是一場好鬥。但見:
怒氣紛紛,狂風滾滾。怒氣紛紛,太保衝冠多膂力﹔狂風滾滾,斑彪逞勢噴紅塵。
那一個張牙舞爪,這一個轉步回身。三股叉擎天幌日,千花尾擾霧雲飛。這一個
當胸亂刺,那一個劈面來吞。閃過的再生人道,撞著的定見閻君。只聽得那斑彪
哮吼,太保聲狠。斑彪哮吼,振裂山川驚鳥獸﹔太保聲狠,喝開天府現星辰。那
一個金睛怒出,這一個壯膽生嗔。可愛鎮山劉太保,堪誇據地獸之君。人虎貪生
爭勝負,些兒有慢喪三魂。

他兩個鬥了有一個時辰,只見那虎爪慢腰鬆,被太保舉叉平胸刺倒。可憐呵,鋼
叉尖穿透心肝,霎時間血流滿地。揪著耳朵,拖上路來。好男子,氣不連喘,面
不改色,對三藏道:「造化,造化。這隻山貓,勾長老食用一日。」三藏誇讚不
盡道:「太保真山神也!」伯欽道:「有何本事,敢勞過獎?這個是長老的洪福
。去來,趕早兒剝了皮,煮些肉,管待你也。」

他一隻手執著叉,一隻手拖著虎,在前引路。三藏牽著馬,隨後而行。迤行過山
坡,忽見一座山莊。那門前真個是:
參天古樹,漫路荒籐。萬壑風塵冷,千崖氣象奇。一徑野花香襲體,數竿幽竹綠
依依。草門樓,籬笆院,堪描堪畫﹔石板橋,白土壁,真樂真稀。秋容蕭索,爽
氣孤高。道傍黃葉落,嶺上白雲飄。疏林內山禽聒聒,莊門外細犬嘹嘹。

伯欽到了門首,將死虎擲下,叫:「小的們何在?」只見走出三四個家僮,都是
怪形惡相之類,上前拖拖拉拉,把隻虎扛將進去。伯欽吩咐教趕早剝了皮,安排
將來待客。復回頭迎接三藏進內,彼此相見,三藏又拜謝伯欽厚恩憐憫救命。伯
欽道:「同鄉之人,何勞致謝。」坐定茶罷,有一老嫗領著一個媳婦,對三藏進
禮。伯欽道:「此是家母、小妻。」三藏道:「請令堂上坐,貧僧奉拜。」老嫗
道:「長老遠客,各請自珍,不勞拜罷。」伯欽道:「母親呵,他是唐王駕下,
差往西天見佛求經者。適間在嶺頭上遇著孩兒,孩兒念一國之人,請他來家歇馬
,明日送他上路。」老嫗聞言,十分懽喜道:「好,好,好。就是請他,不得這
般恰好。明日你父親週忌,就浼長老做些好事,念卷經文,到後日送他去罷。」
這劉伯欽雖是一個殺虎手,鎮山的太保,他卻有些孝順之心。聞得母言,就要安
排香紙,留住三藏。

說話間,不覺的天色將晚。小的們排開桌凳,拿幾盤爛熟虎肉,熱騰騰的放在上
面。伯欽請三藏權用,再另辦飯。三藏合掌當胸道:「善哉!貧僧不瞞太保說,
自出娘胎,就做和尚,更不曉得吃葷。」伯欽聞得此說,沉吟了半晌道:「長老
,寒家歷代以來,不曉得吃素。就是有些竹筍,採些木耳,尋些乾菜,做些豆腐
,也都是獐鹿虎豹的油煎,卻無甚素處。有兩眼鍋灶,也都是油膩透了。這等奈
何?反是我請長老的不是。」三藏道:「太保不必多心,請自受用。我貧僧就是
三五日不吃飯,也可忍餓,只是不敢破了齋戒。」伯欽道:「倘或餓死,卻如之
何?」三藏道:「感得太保天恩,搭救出虎狼叢裏,就是餓死,也強如喂虎。」

伯欽的母親聞說,叫道:「孩兒不要與長老閑講,我自有素物,可以管待。」伯
欽道:「素物何來?」母親道:「你莫管我,我自有素的。」叫媳婦將小鍋取下
,著火燒了油膩,刷了又刷,洗了又洗,卻仍安在灶上。先燒半鍋滾水,別用。
卻又將些山地榆葉子,著水煎作茶湯。然後將些黃粱粟米,煮起飯來。又把些乾
菜煮熟。盛了兩碗,拿出來鋪在桌上。老母對著三藏道:「長老請齋。這是老身
與兒婦,親自動手整理的些極潔極淨的茶飯。」三藏下來謝了,方才上坐。

那伯欽另設一處,鋪排些沒鹽沒醬的老虎肉、香獐肉、蟒蛇肉、狐狸肉、兔肉,
點剁鹿肉乾巴,滿盤滿碗的陪著三藏吃齋。方坐下,心欲舉箸,只見三藏合掌誦
經,諕得個伯欽不敢動箸,急起身立在旁邊。三藏念不數句,卻教請齋。伯欽道
:「你是個念短頭經的和尚?」三藏道:「此非是經,乃是一卷揭齋之咒。」伯
欽道:「你們出家人,偏有許多計較,吃飯便也念誦念誦。」

吃了齋飯,收了盤碗,漸漸天晚。伯欽引著三藏出中宅,到後邊走走。穿過夾道,
有一座草亭。推開門,入到裏面。只見那四壁上掛幾張強弓硬弩,插幾壺箭﹔過
梁上搭兩塊血腥的虎皮﹔牆根頭插著許多槍刀叉棒﹔正中間設兩張坐器。伯欽請
三藏坐坐。三藏見這般兇險腌臢,不敢久坐,遂出了草亭。又往後再行,是一座
大園藏,卻看不盡那叢叢菊蕊堆黃,樹樹楓楊掛赤。又見呼的一聲,跑出十來隻
肥鹿,一大陣黃獐,見了人,呢呢痴痴,更不恐懼。三藏道:「這獐鹿想是太保
養家了的?」伯欽道:「似你那長安城中人家,有錢的集財寶,有莊的集聚稻糧
。我們這打獵的,只得聚養些野獸,備天陰耳。」他兩個說話閑行,不覺黃昏,
復轉前宅安歇。

次早,那合家老小都起來,就整素齋,管待長老,請開啟念經。這長老淨了手,
同太保家堂前拈了香,拜了家堂。三藏方敲響木魚,先念了淨口業的真言,又念
了淨身心的神咒然後開《度亡經》一卷。誦畢,伯欽又請寫薦亡疏一道,再開念
《金剛經》、《觀音經》。一一朗音高誦。誦畢,吃了午齋,又念《法華經》、
《彌陀經》,各誦幾卷,又念一卷《孔雀經》,及談苾?洗業的故事,早又天晚。
獻過了種種香火,化了眾神紙馬,燒了薦亡文疏。佛事已畢,又各安寢。

卻說那伯欽的父親之靈,超薦得脫沉淪,鬼魂兒早來到自家宅內,托一夢與合宅
長幼道:「我在陰司裏苦難難脫,日久不得超生。今幸得聖僧念了經卷,消了我
的罪業,閻王差人送我上中華富地,長者人家托生去了。你們可好生謝送長老,
不要怠慢,不要怠慢。我去也。」這才是:萬法莊嚴端有意,薦亡離苦出沉淪。
那合家兒夢醒,又早太陽東上。伯欽的娘子道:「太保,我今夜夢見公公來家,
說他在陰司苦難難脫,日久不得超生。今幸得聖僧念了經卷,消了他的罪業,閻
王差人送他上中華富地,長者人家托生去,教我們好生謝那長老,不得怠慢他。
說罷,徑出門,徉徜去了。我們叫他不應,留他不住。醒來卻是一夢。」伯欽道
:「我也是那等一夢,與你一般。我們起去對母親說去。」他兩口子正欲去說,
只見老母叫道:「伯欽孩兒,你來,我與你說話。」二人至前,老母坐在床上道
:「兒呵,我今夜得了個喜夢,夢見你父親來家,說多虧了長老超度,已消了罪
業,上中華富地,長者家去托生。」夫妻們俱呵呵大笑道:「我與媳婦皆有此夢
,正來告稟,不期母親呼喚,也是此夢。」遂叫一家大小起來,安排謝意,替他
收拾馬匹,都至前拜謝道:「多謝長老超薦我亡父脫難超生,報答不盡。」三藏
道:「貧僧有何能處,敢勞致謝?」

伯欽把三口兒的夢話對三藏陳訴一遍,三藏也喜。早供給了素齋,又具白銀一兩
為謝。三藏分文不受。一家兒又懇懇拜央,三藏畢竟分文未受。但道:「是你肯
發慈悲送我一程,足感至愛。」伯欽與母妻無奈,急做了些粗麵燒餅乾糧,叫伯
欽遠送。三藏歡喜收納。太保領了母命,又喚兩三個家僮,各帶捕獵的器械,同
上大路。看不盡那山中野景,嶺上風光。
  
行經半日,只見對面處有一座大山,真個是高接青霄,崔巍險峻。三藏不一時到
了邊前。那太保登此山如行平地,正走到半山之中,伯欽回身,立於路下道:
「長老,請自前進,我卻告回。」三藏聞言,滾鞍下馬道:「千萬敢勞太保再送
一程。」伯欽道:「長老不知。此山喚做兩界山,東半邊屬我大唐所管,西半邊
乃是韃靼的地界。那廂狼虎不伏我降,我卻也不能過界,你自去罷。」三藏心驚
,掄開手,牽衣執袂,滴淚難分。正在那叮嚀拜別之際,只聽得山腳下叫喊如雷
道:「我師父來也!我師父來也!」諕得個三藏痴呆,伯欽打掙。
  
    畢竟不知是甚人叫喊,且聽下回分解。





}  \end{pinyinscope}\switchcolumn{\myfontc \section{第 一 三 回} 陷 虎 穴 、 金 星 解 厄 双 叉 岭 , 伯 钦 留 僧 诗 说 : 大 有 唐 王 降 敕 封 , 钦 差 玄 奘 问 禅 宗 。
坚 心 磨 炼 , 寻 找 龙 穴 , 注 意 修 行 , 登 上 峰 顶 。
边 界 远 游 多 少 国 家 , 云 山 前 度 万 千 重 。
从 今 天 起 别 驾 投 奔 西 去 , 秉 持 教 诲 迦 持 悟 大 空 。
又 说 : 三 藏 从 贞 观 十 三 年 九 月 十 五 日 前 三 天 , 承 蒙 唐 王 和 多 官 送 出 长 安 关 外 。
一 两 天 马 不 停 , 早 晨 到 法 门 寺 。
本 寺 的 住 持 上 房 长 老 , 带 领 众 僧 有 五 百 多 人 , 两 边 排 列 着 , 接 到 里 面 , 相 见 进 献 茶 叶 。
茶 罢 , 进 斋 斋 , 斋 后 不 觉 天 晚 。
正 是 那 么 , 影 子 摇 动 星 河 近 , 月 亮 明 亮 无 点 尘 。
雁 声 传 远 汉 , 砧 声 响 西 邻 。
归 鸟 栖 息 在 枯 树 上 , 禅 僧 讲 梵 音 。
蒲 团 在 床 上 , 坐 到 半 夜 时 分 。
众 僧 人 在 灯 下 议 论 佛 门 的 定 旨 , 上 天 取 经 的 原 因 : 有 的 说 水 远 山 高 , 有 的 说 路 上 虎 豹 多 , 有 的 说 岭 上 陡 崖 难 以 越 过 , 有 的 说 毒 魔 恶 怪 难 以 降 临 。
三 藏 闭 口 不 说 话 , 只 是 用 手 指 自 己 的 心 , 点 头 几 度 。
众 僧 人 都 不 理 解 他 的 意 思 , 合 掌 请 教 道 : 法 师 指 心 点 头 , 是 什 么 意 思 ? 三 藏 回 答 说 : 心 生 , 种 种 魔 生 ; 心 灭 , 种 种 魔 灭 。
我 的 弟 子 曾 经 在 化 生 寺 对 佛 说 下 大 誓 大 愿 , 不 是 因 为 我 不 能 尽 心 尽 力 。
这 一 去 , 一 定 要 到 西 天 去 , 见 佛 求 经 , 使 我 们 的 法 轮 回 转 , 希 望 圣 主 的 图 谋 永 远 牢 固 。
众 僧 听 到 了 这 话 , 人 人 称 赞 羡 慕 , 每 个 个 都 宣 扬 , 都 大 叫 一 声 : 忠 心 赤 胆 大 阐 法 师 啊 夸 赞 不 尽 , 请 大 师 入 床 安 睡 。
早 晨 又 是 竹 敲 残 月 落 , 鸡 唱 晓 云 生 。
众 僧 起 来 , 收 拾 茶 水 、 早 斋 。
玄 奘 于 是 穿 上 袈 裟 , 登 上 正 殿 , 佛 在 前 面 礼 拜 道 : 弟 子 陈 玄 奘 , 前 往 西 天 取 经 , 只 是 肉 眼 愚 昧 , 不 认 识 活 佛 的 真 形 。
现 在 我 愿 意 立 誓 : 路 上 遇 到 庙 烧 香 , 遇 到 佛 拜 佛 , 遇 到 塔 扫 塔 。
只 希 望 我 佛 慈 悲 , 早 日 显 现 一 丈 六 寸 金 身 , 赐 给 真 经 , 留 传 东 方 。
祝 祷 完 毕 , 回 到 方 丈 进 斋 。
斋 戒 完 毕 , 那 两 个 随 从 整 顿 好 马 鞍 , 催 促 行 程 。
三 藏 出 了 山 门 , 辞 别 众 僧 。
众 僧 人 不 忍 心 分 别 , 径 直 送 他 走 了 十 里 路 程 , 他 含 着 眼 泪 返 回 。
三 藏 便 直 向 西 进 军 。
正 是 那 个 季 秋 的 天 气 , 只 看 见 几 个 村 庄 的 树 木 落 下 , 芦 花 碎 落 , 几 棵 树 上 的 枫 树 和 杨 柳 红 叶 坠 落 。
路 上 的 烟 雨 故 人 稀 少 , 黄 菊 艳 丽 , 山 骨 细 细 , 水 寒 荷 破 人 憔 悴 。
白 苹 红 蓼 霜 天 雪 天 , 落 霞 孤 鹜 长 空 坠 落 。
依 稀 黯 淡 , 野 云 飞 , 玄 鸟 离 去 , 宾 鸿 来 到 , 鸣 叫 的 声 音 夜 晚 碎 碎 。
师 傅 之 子 , 其 弟 弟 之 子 , 其 弟 子 之 子 , 其 兄 弟 之 子 , 其 兄 弟 之 子 , 其 兄 弟 之 子 , 其 兄 弟 之 子 , 其 兄 弟 之 子 , 其 兄 弟 之 子 , 其 兄 弟 之 子 。
安 顿 一 夜 , 第 二 天 早 晨 出 城 前 去 。
一 路 上 饥 饿 吃 饭 , 口 渴 喝 水 , 夜 晚 住 宿 , 清 晨 出 发 , 走 了 两 三 天 , 又 到 了 河 州 卫 。
此 乃 是 大 唐 的 山 河 边 界 。
早 有 镇 守 边 疆 的 总 兵 和 本 地 的 僧 道 , 听 说 他 是 钦 差 御 弟 法 师 , 上 西 方 见 佛 , 没 有 不 恭 敬 的 。
接 到 里 面 供 给 了 , 僧 纲 请 求 到 福 原 寺 去 休 息 。
本 寺 的 僧 人 , 一 一 参 见 , 安 排 晚 斋 。
斋 戒 完 毕 , 他 吩 咐 两 个 随 从 吃 饱 喂 马 匹 , 天 不 亮 就 走 了 。
到 鸡 刚 鸣 叫 时 , 随 即 召 唤 随 从 的 人 , 却 又 惊 动 了 寺 中 的 僧 人 , 整 理 茶 汤 斋 供 。
斋 戒 完 毕 , 离 开 边 界 。
这 个 长 老 心 忙 , 太 早 了 。
原 来 此 时 是 秋 深 时 节 , 鸡 叫 得 早 , 只 好 有 四 更 天 气 。
一 行 三 人 , 连 马 四 口 , 迎 着 清 霜 , 看 着 明 月 , 走 了 几 十 里 远 近 , 看 见 一 座 山 岭 , 只 能 拨 草 寻 路 , 说 不 尽 是 崎 岖 难 走 , 又 恐 怕 错 了 路 。
正 在 疑 虑 之 间 , 忽 然 失 脚 , 三 个 人 连 马 都 跌 落 在 坑 坑 中 。
三 脏 心 惊 , 随 从 的 人 胆 战 。
又 听 到 里 面 怒 吼 , 大 叫 道 : 拿 将 来 , 拿 将 来 , 只 见 狂 风 滚 滚 , 拥 出 五 六 十 个 妖 邪 , 把 三 藏 、 随 从 的 人 把 他 们 都 绑 上 去 。
法 师 之 所 以 不 得 也 。
真 的 是 : 雄 威 身 上 凛 凛 , 猛 气 相 貌 堂 堂 。
闪 闪 闪 闪 闪 闪 闪 闪 闪 闪 , 雷 声 震 动 四 方 。
锯 牙 伸 开 口 外 , 凿 齿 露 出 嘴 唇 旁 边 。
锦 绣 围 住 身 体 , 文 彩 斑 斓 裹 着 脊 梁 。
钢 的 胡 子 很 少 见 到 肉 , 钩 爪 尖 利 得 像 霜 一 样 锋 利 。
东 海 黄 公 惧 怕 , 南 山 白 额 王 。
如 果 得 到 三 藏 , 就 会 魂 飞 魄 散 , 两 个 随 从 的 人 就 会 骨 瘦 筋 麻 。
王 曰 : 王 曰 : 王 曰 : 王 曰 : 我 不 能 为 之 。
众 妖 一 齐 把 三 个 人 用 绳 索 捆 绑 起 来 。
正 要 安 排 吞 食 , 只 听 到 外 面 喧 哗 , 有 人 来 报 告 说 : 熊 山 君 与 特 处 士 二 位 来 了 。
三 藏 听 了 这 话 , 抬 头 看 , 先 前 走 的 是 一 个 黑 色 的 汉 人 。
卿 曰 : 雄 豪 多 胆 量 , 身 体 轻 健 , 身 体 健 壮 。
涉 水 凭 借 凶 猛 的 力 量 , 跑 到 林 中 逞 怒 的 威 势 。
以 前 都 是 吉 梦 , 现 在 却 独 自 显 露 出 英 姿 。
绿 树 能 攀 折 , 知 寒 善 晓 时 。
淮 南 王 准 是 神 灵 的 地 方 , 所 以 这 里 号 山 君 。
又 见 那 后 边 来 的 , 是 一 条 胖 汉 。
你 说 什 么 样 子 ? 巍 峨 双 角 冠 , 端 庄 肃 穆 , 高 耸 肩 背 。
他 生 性 穿 着 青 色 衣 服 , 稳 稳 稳 稳 , 步 行 多 有 迟 缓 。
宗 族 的 名 字 是 父 亲 的 作 , 原 来 的 号 称 是 母 亲 的 名 字 。
能 够 为 田 的 人 建 立 功 勋 , 因 而 名 为 特 处 士 。
两 个 人 摇 摇 地 走 , 走 进 里 面 , 惊 慌 地 跑 出 来 迎 接 。
熊 山 君 说 : 寅 将 军 一 向 得 意 , 可 以 祝 贺 , 可 以 祝 贺 。
特 处 士 说 : 寅 将 军 丰 姿 超 过 平 常 , 真 是 可 喜 , 真 是 可 喜 。
魔 王 说 : 二 公 连 日 怎 么 样 ? 山 君 说 : 只 有 守 素 罢 了 。
处 士 说 : 只 是 随 着 时 机 而 已 。
三 个 人 议 论 完 毕 , 各 自 坐 着 谈 笑 。
只 见 那 从 者 被 绑 得 痛 哭 。
那 黑 汉 说 : 这 三 个 人 是 从 哪 里 来 的 ? 魔 王 说 : 自 己 送 上 门 来 的 。
处 士 笑 着 说 : 可 以 等 待 客 人 吗 ? 魔 王 说 : 奉 承 , 奉 承 。
山 君 说 : 不 可 以 全 部 使 用 , 吃 掉 两 个 , 留 下 一 个 就 可 以 了 。
这 时 , 魔 王 领 着 他 们 的 话 , 就 叫 来 左 右 随 从 , 把 两 个 随 从 的 人 剖 腹 剖 心 , 把 他 们 的 尸 体 剁 碎 了 。 把 脑 袋 和 心 肝 献 给 了 两 个 客 人 , 把 自 己 的 四 肢 自 食 , 其 余 的 骨 肉 分 给 各 位 妖 怪 。
只 听 到 的 声 音 , 真 像 虎 吃 羊 羔 , 顷 刻 之 间 就 吃 完 了 。
使 一 个 长 老 , 几 乎 饿 死 。
初 出 长 安 , 第 一 场 苦 难 。
正 在 惊 恐 之 中 , 渐 渐 变 白 色 。
那 两 个 怪 物 到 天 亮 才 散 去 , 一 齐 说 : 今 天 我 们 这 么 多 的 烦 恼 , 我 们 今 天 我 们 尽 心 竭 诚 报 答 。
李 方 一 拥 而 退 。
不 到 一 个 时 候 , 红 日 升 起 , 三 脏 昏 昏 沉 沉 , 分 辨 不 出 东 西 南 北 。
正 在 那 个 不 得 命 的 地 方 , 忽 然 看 见 一 个 老 头 , 手 里 拿 着 拐 杖 走 来 。
走 上 前 , 用 手 一 拂 , 绳 索 都 断 了 。
对 面 吹 了 一 口 气 , 三 藏 才 苏 醒 过 来 , 跪 在 地 上 说 : 多 谢 老 公 公 , 给 我 帮 助 贫 僧 的 性 命 。
老 头 答 礼 说 : 你 起 来 。
三 藏 说 : 贫 僧 的 随 从 已 经 被 怪 吃 了 。
只 不 知 道 行 李 、 马 匹 在 什 么 地 方 , 老 头 用 拐 杖 指 着 说 : 那 里 不 是 一 匹 马 、 两 个 包 袱 ? 三 藏 回 头 看 时 , 果 然 是 他 的 物 品 , 并 不 曾 失 落 , 心 才 略 略 放 下 一 点 。
又 问 老 翁 说 : 老 公 公 , 此 处 是 什 么 地 方 , 公 公 为 什 么 在 这 里 ? 老 翁 说 : 这 是 双 叉 岭 , 是 虎 狼 巢 穴 的 地 方 。
三 藏 菩 萨 说 : 我 在 鸡 鸣 时 , 出 了 河 州 卫 境 , 不 料 起 得 早 了 , 冒 着 霜 打 开 露 水 , 忽 然 失 落 在 这 个 地 方 。
他 看 见 一 个 魔 王 , 凶 狠 凶 狠 太 过 分 , 把 贫 僧 和 两 个 随 从 的 人 绑 了 。
又 看 见 一 条 黑 色 的 汉 人 , 声 称 是 熊 山 君 ; 一 条 胖 汉 人 , 声 称 是 特 处 士 。
其 他 三 个 人 把 我 两 个 随 从 吃 了 , 天 光 才 散 去 。
不 想 我 是 哪 里 有 这 样 的 大 缘 和 大 分 , 感 动 得 老 公 公 来 救 我 , 老 叟 说 : 处 士 是 个 野 牛 精 , 山 君 是 个 熊 精 , 寅 将 军 是 个 老 虎 精 。
左 右 的 妖 邪 , 全 都 是 山 精 、 树 鬼 、 怪 兽 、 苍 狼 。
因 为 你 的 本 性 元 明 , 所 以 吃 不 到 你 。
你 跟 我 来 , 引 你 上 路 。
三 藏 不 胜 感 激 , 将 包 袱 送 到 马 上 , 牵 着 绳 , 跟 着 老 叟 径 直 走 出 坑 坎 之 中 , 走 上 大 路 。
又 把 马 拴 在 道 旁 的 草 头 上 , 转 身 向 那 公 公 道 歉 , 那 公 公 就 变 成 一 阵 清 风 , 跨 上 一 只 顶 白 鹤 , 腾 空 而 去 。
只 见 风 飘 下 一 张 简 帖 , 书 上 四 句 颂 子 。
颂 子 说 : 我 是 西 天 的 太 白 星 , 特 来 搭 救 你 的 生 灵 。
前 行 自 有 神 灵 帮 助 , 不 要 为 艰 难 报 仇 。
三 藏 看 了 , 对 着 天 礼 拜 道 : 多 谢 金 星 , 度 脱 此 难 。
拜 完 , 牵 着 马 匹 , 独 自 一 个 孤 独 凄 凄 地 向 前 苦 苦 地 向 前 走 。
此 岭 上 真 是 : 寒 冷 飒 飒 雨 林 风 , 潺 潺 涧 下 水 。
香 气 馥 郁 , 野 花 开 , 密 密 的 丛 林 乱 石 磊 磊 。
喧 闹 闹 闹 , 鹿 和 猿 , 一 队 一 队 獐 和 麂 。
喧 杂 杂 乱 的 鸟 声 很 多 , 静 悄 悄 悄 的 人 事 都 没 有 了 。
那 些 长 辈 , 战 战 兢 兢 , 心 里 不 安 ; 这 个 马 儿 , 力 量 怯 懦 , 蹄 子 难 以 举 动 。
三 藏 舍 身 拼 命 , 上 了 那 陡 峻 的 山 岭 之 间 。
走 了 半 天 , 更 不 见 一 个 人 烟 村 舍 。
一 则 腹 中 饥 饿 了 , 二 则 路 又 不 平 。
正 在 危 急 的 时 候 , 只 见 前 面 有 两 只 猛 虎 , 后 边 有 几 条 长 蛇 盘 绕 着 。
左 边 有 毒 虫 , 右 边 有 怪 兽 。
三 藏 孤 身 无 策 , 只 得 放 下 自 己 的 身 心 , 听 从 上 天 的 命 令 。
又 无 可 奈 那 匹 马 , 腰 柔 蹄 弯 , 立 即 跪 下 , 伏 在 地 上 , 打 又 打 不 起 , 牵 又 牵 不 动 。
苦 得 一 个 法 师 身 体 没 有 地 方 , 真 的 也 有 万 分 悲 痛 , 已 经 自 己 分 析 必 死 , 无 可 奈 何 。
又 曰 : 我 虽 有 灾 祸 , 却 有 救 应 。
正 在 那 个 不 得 命 的 地 方 , 忽 然 看 见 毒 虫 奔 走 , 妖 兽 飞 逃 , 猛 虎 潜 伏 , 长 蛇 隐 藏 踪 迹 。
三 藏 抬 头 看 时 , 只 见 一 个 人 , 手 里 拿 着 钢 叉 , 腰 上 挂 着 弓 箭 , 从 那 个 山 坡 前 转 出 来 , 果 然 是 一 条 好 汉 。
你 看 他 , 头 上 戴 一 顶 艾 叶 花 斑 豹 皮 帽 , 身 上 穿 一 领 羊 绒 织 锦 罗 衣 , 腰 间 系 着 一 条 狮 子 带 , 脚 下 系 着 一 对 犀 皮 靴 。
环 眼 圆 睛 如 同 吊 客 , 环 眉 乱 发 如 同 河 魁 。
悬 挂 一 袋 毒 药 弓 箭 , 拿 一 根 点 钢 大 叉 。
雷 声 震 破 山 中 的 虫 胆 , 勇 猛 惊 吓 野 鸡 的 魂 魄 。
三 藏 见 他 渐 渐 走 近 , 跪 在 路 旁 , 合 掌 大 叫 道 : 大 王 救 命 , 大 王 救 命 , 那 个 汉 人 来 到 边 前 , 放 下 铁 叉 , 用 手 扶 起 说 : 长 老 不 要 害 怕 。
我 不 是 凶 恶 的 人 , 我 是 山 中 的 猎 户 , 姓 刘 名 伯 钦 , 号 镇 山 太 保 。
我 刚 刚 来 , 要 找 两 只 山 虫 吃 。
不 料 遇 到 你 , 多 有 冲 撞 。
三 藏 菩 萨 说 : 我 是 大 唐 皇 帝 驾 下 的 钦 差 , 去 西 天 拜 佛 求 经 的 和 尚 。
偶 尔 来 到 此 地 , 遇 到 一 些 狼 虎 蛇 虫 , 四 周 围 绕 , 不 能 前 进 。
忽 然 看 见 太 保 来 了 , 众 兽 都 逃 走 了 , 救 了 贫 僧 的 性 命 。
伯 钦 说 : 我 在 这 里 住 的 人 , 专 门 靠 着 打 狼 虎 来 生 , 捉 蛇 虫 来 活 命 , 所 以 这 些 野 兽 怕 我 逃 走 了 。
你 既 然 是 唐 朝 来 的 , 与 我 都 是 乡 里 。
今 之 所 以 为 之 , 而 不 可 以 为 之 。
你 不 要 害 怕 , 跟 我 来 , 到 我 家 下 歇 马 , 第 二 天 早 晨 我 送 你 上 路 。
三 藏 听 了 这 话 , 满 心 欢 喜 , 向 伯 钦 道 歉 , 牵 着 马 随 行 。
过 了 山 坡 , 又 听 到 一 声 呼 叫 的 风 声 。
伯 钦 说 : 长 老 不 要 走 , 坐 在 这 里 。
风 吹 响 的 地 方 , 是 个 山 猫 来 了 , 等 我 拿 他 家 去 管 你 。
三 藏 听 了 这 话 , 又 胆 战 心 惊 , 不 敢 动 步 。
太 保 拿 着 铁 叉 , 拽 开 步 子 , 迎 上 去 。
只 见 一 只 斑 老 虎 , 对 面 撞 , 他 看 见 伯 钦 , 急 忙 回 头 就 走 了 。
这 个 太 保 一 声 霹 雳 , 呵 斥 道 : 那 业 畜 从 哪 里 逃 走 。 那 老 虎 见 这 只 老 虎 急 忙 跑 了 , 转 身 伸 爪 扑 来 。
这 个 太 保 有 三 股 叉 , 举 手 迎 敌 。
得 了 三 藏 , 软 软 地 躺 在 草 地 上 。
这 个 和 尚 自 从 出 娘 的 肚 皮 , 哪 曾 见 到 这 样 凶 险 的 行 动 ?
太 保 和 那 虎 在 那 山 坡 下 , 人 虎 相 持 , 果 然 是 一 场 好 斗 。
只 见 一 股 怒 气 纷 纷 , 狂 风 滚 滚 。
怒 气 纷 纷 , 太 保 冲 上 帽 子 , 多 有 力 气 ; 狂 风 滚 滚 , 斑 彪 逞 势 冲 击 红 尘 。
那 一 个 张 牙 舞 爪 , 另 一 个 转 步 回 身 。
三 股 叉 擎 着 天 空 , 飘 着 太 阳 , 千 花 尾 搅 动 雾 云 飞 。
其 一 个 当 胸 乱 刺 , 其 一 个 劈 面 来 吞 食 。
闪 过 的 再 生 人 道 , 撞 着 的 一 定 会 见 阎 君 。
只 听 到 那 斑 彪 咆 哮 , 太 保 的 声 音 很 狠 。
斑 彪 怒 吼 , 震 裂 山 川 惊 动 鸟 兽 ; 太 保 声 音 狠 毒 , 喝 开 天 府 显 现 星 辰 。
那 一 个 金 睛 怒 出 , 这 一 个 壮 胆 生 气 。
可 爱 的 镇 山 刘 太 保 , 可 以 夸 耀 占 据 地 兽 的 君 主 。
人 虎 贪 生 争 夺 胜 负 , 小 孩 有 轻 慢 丧 失 三 魂 。
他 两 个 人 斗 了 一 个 时 辰 , 只 见 虎 爪 慢 , 腰 肢 松 弛 , 被 太 保 举 叉 平 胸 刺 倒 。
可 怜 啊 , 钢 叉 尖 穿 透 心 肝 , 顷 刻 之 间 血 流 满 地 。
把 他 们 的 耳 朵 搓 住 , 把 他 们 拖 上 路 来 。
好 男 子 , 气 不 喘 , 面 不 改 色 , 对 三 藏 说 : 造 化 造 化 , 造 化 造 化 。
此 时 山 猫 , 勾 引 长 老 吃 了 一 天 。
三 藏 夸 奖 不 尽 说 道 : 太 保 真 是 山 神 呀 伯 钦 说 : 有 什 么 本 事 , 我 敢 劳 你 过 奖 , 这 是 长 老 的 大 福 。
去 来 , 赶 早 儿 剥 掉 皮 , 煮 些 肉 , 只 管 等 着 你 。
其 一 只 手 拿 叉 , 一 只 手 拖 着 虎 , 在 前 面 引 路 。
三 藏 拉 着 马 , 随 后 而 行 。
往 前 走 过 山 坡 , 忽 然 看 见 一 座 山 庄 。
那 门 前 真 的 是 : 参 天 古 树 , 漫 路 荒 藤 。
万 壑 风 尘 冷 , 千 崖 气 象 奇 。
一 条 小 路 上 的 野 花 香 气 袭 人 身 体 , 几 竿 幽 竹 绿 荫 依 依 。
草 门 楼 , 篱 笆 院 , 可 以 绘 画 ; 石 板 桥 , 白 土 墙 , 真 乐 真 稀 。
秋 容 萧 索 , 爽 气 孤 高 。
道 旁 黄 叶 落 , 岭 上 白 云 飘 。
疏 林 里 的 山 禽 鸣 叫 , 庄 园 门 外 的 小 狗 鸣 叫 。
伯 钦 到 了 门 口 , 把 死 虎 扔 下 去 , 大 叫 道 : 小 的 人 在 哪 里 ? 只 见 走 出 来 的 三 四 个 家 僮 , 都 是 怪 形 恶 相 之 类 , 上 前 拖 着 , 扯 着 , 把 老 虎 扛 着 进 去 。
伯 钦 命 令 他 早 早 剥 掉 皮 , 安 排 来 接 待 客 人 。
又 回 头 迎 接 三 藏 进 入 内 室 , 彼 此 相 见 , 三 藏 又 拜 谢 伯 钦 的 厚 恩 , 怜 悯 救 命 。
伯 钦 说 : 我 是 同 乡 的 人 , 何 必 致 谢 ?
喝 完 茶 , 有 一 个 老 人 领 着 一 个 媳 妇 , 对 着 三 藏 进 献 礼 物 。
伯 钦 说 : 这 是 我 家 的 母 亲 和 小 妻 子 。
三 藏 菩 萨 说 : 请 让 我 在 堂 上 坐 下 , 贫 僧 奉 行 拜 礼 。
老 说 : 长 老 远 客 , 各 自 请 求 自 己 珍 惜 , 不 用 拜 谢 。
伯 钦 说 : 母 亲 呵 , 他 是 唐 王 驾 下 的 , 派 我 到 西 天 去 见 佛 求 经 的 人 。
恰 巧 在 岭 头 上 遇 见 孩 子 , 孩 子 想 到 一 国 的 人 , 请 他 来 家 里 歇 马 , 第 二 天 送 他 上 路 。
老 听 了 这 话 , 十 分 高 兴 地 说 : 好 , 好 , 好 。
就 是 请 求 他 , 不 得 这 样 恰 好 。
明 天 你 父 亲 去 世 , 就 让 长 老 做 些 好 事 , 念 一 卷 经 文 , 到 后 天 送 他 去 吧 。
刘 伯 钦 虽 然 是 一 个 杀 虎 手 , 镇 山 的 太 保 , 他 却 有 孝 顺 之 心 。
听 到 母 亲 的 话 , 就 要 安 排 香 纸 , 留 住 三 藏 。
说 话 之 间 , 不 觉 天 色 将 晚 。
小 的 人 挤 开 桌 子 , 拿 出 几 盘 烂 熟 的 虎 肉 , 放 在 上 面 。
伯 钦 请 三 藏 权 且 使 用 , 再 另 外 办 饭 。
三 藏 合 掌 当 胸 说 : 好 啊 我 不 骗 太 保 说 , 自 从 出 娘 胎 , 就 做 和 尚 , 更 不 知 道 吃 荤 。
伯 钦 听 到 这 话 , 沉 吟 半 晌 说 : 长 老 , 我 家 历 代 以 来 , 不 知 道 吃 素 。
就 是 有 些 竹 笋 , 采 些 木 耳 , 找 些 干 菜 , 做 些 豆 腐 , 也 都 是 獐 鹿 虎 豹 的 油 煎 , 却 没 有 什 么 特 点 。
还 有 两 眼 锅 灶 , 也 都 是 油 腻 透 了 。
这 种 情 况 怎 么 办 ? 反 而 是 我 请 长 老 的 不 是 。
三 藏 说 : 太 保 不 必 多 心 , 请 您 自 己 接 受 任 用 。
我 是 个 贫 僧 , 三 五 天 不 吃 饭 , 也 可 以 忍 受 饥 饿 , 只 是 不 敢 破 坏 斋 戒 。
伯 钦 说 : 如 果 有 人 饿 死 , 又 怎 么 办 呢 ? 三 藏 说 : 感 谢 太 保 天 恩 , 把 他 救 出 虎 狼 丛 中 , 就 是 饿 死 , 也 强 得 像 喂 虎 一 样 。
伯 钦 的 母 亲 听 了 , 大 叫 道 : 孩 子 不 要 和 长 老 闲 谈 , 我 自 有 素 来 的 东 西 , 可 以 管 它 。
伯 钦 说 : 白 色 的 东 西 从 哪 里 来 ? 母 亲 说 : 你 不 管 我 , 我 自 有 素 色 的 东 西 。
又 叫 媳 妇 把 小 锅 取 出 来 , 放 在 火 里 烧 得 油 滑 , 洗 了 又 洗 , 仍 然 放 在 灶 上 。
先 烧 半 锅 , 然 后 再 用 。
又 拿 了 一 些 山 地 榆 叶 子 , 放 在 水 里 煎 成 茶 汤 。
然 后 将 黄 粱 、 粟 米 煮 起 来 。
又 把 干 菜 煮 熟 。
又 取 之 于 酒 , 以 酒 为 食 。
老 母 对 着 三 藏 菩 萨 说 : 长 老 请 斋 戒 。
这 是 老 人 和 儿 媳 , 亲 自 动 手 整 理 的 , 极 其 洁 净 的 茶 饭 。
三 藏 下 来 谢 罪 , 才 上 座 。
伯 钦 另 外 设 置 一 个 地 方 , 铺 排 着 没 盐 没 酱 的 老 虎 肉 、 香 獐 肉 、 蟒 蛇 肉 、 狐 狸 肉 、 兔 肉 , 点 切 鹿 肉 、 干 巴 , 满 盘 满 碗 , 陪 着 三 藏 吃 斋 。
刚 刚 坐 下 , 心 想 举 起 筷 子 , 只 见 三 藏 合 掌 诵 经 , 只 见 一 个 伯 钦 不 敢 动 筷 子 , 急 忙 起 身 站 在 旁 边 。
三 藏 念 不 出 几 句 , 就 教 他 吃 斋 。
伯 钦 说 : 你 是 念 短 头 经 的 和 尚 , 三 藏 说 : 这 不 是 经 , 而 是 一 卷 揭 斋 的 咒 语 。
伯 钦 说 : 你 们 是 出 家 人 , 偏 偏 有 许 多 计 较 , 吃 饭 便 是 念 诵 念 诵 。
吃 了 斋 饭 , 收 了 盘 碗 , 渐 渐 天 晚 了 。
伯 钦 带 着 三 藏 出 自 中 宅 , 到 后 边 走 。
穿 过 夹 道 , 有 一 座 草 亭 。
推 开 门 , 进 入 里 面 。
只 见 四 面 墙 壁 上 挂 着 几 张 强 弓 硬 弩 , 插 着 几 壶 箭 ; 过 桥 上 搭 了 两 块 血 腥 的 虎 皮 , 墙 根 上 插 着 许 多 枪 刀 叉 棒 , 正 中 间 设 置 了 两 张 坐 具 。
伯 钦 请 三 藏 坐 。
三 藏 看 见 这 样 凶 险 腥 臭 , 不 敢 久 坐 , 于 是 出 了 草 亭 。
又 往 后 再 走 , 是 一 座 大 的 园 子 , 却 看 不 到 那 丛 菊 蕊 堆 满 黄 色 的 地 方 , 树 上 的 枫 树 和 杨 树 都 是 红 色 的 。
又 听 到 一 声 呼 叫 , 跑 出 十 来 只 肥 鹿 , 一 大 片 黄 獐 , 见 了 人 , 这 个 人 又 痴 呆 , 更 不 害 怕 。
三 藏 说 : 这 獐 鹿 想 是 太 保 养 家 了 , 伯 钦 说 : 像 你 那 长 安 城 中 的 人 家 , 有 钱 的 集 财 宝 , 有 庄 的 集 聚 粮 食 。
我 们 这 个 打 猎 的 , 只 能 聚 集 一 些 野 兽 , 防 备 天 阴 罢 了 。
其 他 两 个 人 说 话 闲 暇 地 走 , 不 知 不 觉 天 色 昏 暗 , 又 转 到 前 宅 安 歇 。
第 二 天 早 晨 , 全 家 老 小 都 起 来 , 准 备 斋 戒 , 等 待 长 老 , 请 开 启 念 经 。
这 位 长 老 洁 净 了 手 , 和 太 保 家 堂 前 一 起 点 香 , 拜 了 家 堂 。
三 藏 刚 敲 响 木 鱼 , 先 念 净 口 业 的 真 言 , 又 念 净 身 心 的 神 咒 , 然 后 开 《 度 亡 经 》 一 卷 。
诵 读 完 毕 , 伯 钦 又 请 求 抄 写 《 荐 亡 疏 》 一 道 , 又 开 始 念 《 金 刚 经 》 、 《 观 音 经 》 。
一 一 朗 朗 的 声 音 朗 朗 地 朗 朗 地 朗 朗 地 诵 读 。
诵 读 完 毕 , 吃 了 午 斋 , 又 念 《 法 华 经 》 、 《 弥 陀 经 》 , 各 念 几 卷 , 又 念 一 卷 《 孔 雀 经 》 , 还 谈 洗 业 的 故 事 , 早 又 晚 。
进 献 过 了 种 种 香 火 , 化 化 了 众 神 的 纸 马 , 烧 掉 了 荐 亡 的 文 书 。
佛 事 已 经 完 毕 , 又 各 自 安 寝 。
又 说 那 伯 钦 的 父 亲 之 灵 , 越 级 推 荐 得 以 免 除 沉 沦 。 鬼 魂 儿 早 就 来 到 自 己 家 的 宅 子 里 , 假 托 一 个 梦 与 他 家 的 长 幼 说 : 我 在 阴 司 里 , 苦 难 难 以 脱 身 , 时 间 太 久 不 能 超 生 。
现 在 有 幸 圣 僧 念 了 经 卷 , 消 除 了 我 的 罪 业 , 阎 王 派 人 送 我 上 中 华 富 地 , 长 者 人 家 托 生 去 了 。
你 们 可 以 好 好 地 向 长 老 道 谢 , 不 要 懈 怠 , 不 要 怠 慢 。
我 离 开 了 。
夫 子 曰 : 万 法 之 道 , 端 有 意 , 引 导 亡 国 之 苦 , 出 于 沉 沦 之 中 。
那 合 家 儿 梦 醒 了 , 又 早 晨 太 阳 东 上 。
伯 钦 的 娘 子 说 : 太 保 , 我 今 夜 梦 见 公 公 公 公 来 我 家 , 说 他 在 阴 司 苦 难 , 很 久 不 能 超 生 。
今 天 有 幸 圣 僧 念 经 卷 , 消 除 他 的 罪 业 , 阎 王 派 人 送 他 到 中 华 富 地 , 长 者 人 家 托 生 去 , 教 我 们 好 好 地 向 那 位 长 老 道 歉 , 不 要 怠 慢 他 。
说 完 , 径 直 走 出 门 , 悠 悠 地 走 了 。
吾 闻 之 曰 : 我 不 应 , 留 下 他 不 住 。
醒 来 后 , 又 是 一 个 梦 。
伯 钦 说 : 我 也 是 那 样 的 一 个 梦 , 和 你 一 样 。
吾 闻 之 。
他 两 口 子 正 要 去 说 , 只 见 老 母 大 叫 道 : 伯 钦 孩 子 , 你 来 , 我 和 你 说 话 。
二 人 来 到 面 前 , 老 母 坐 在 床 上 说 : 儿 子 呵 , 我 今 晚 得 了 一 个 喜 梦 , 梦 见 你 父 亲 来 家 , 说 多 亏 了 长 老 超 度 , 已 经 消 除 了 罪 业 , 上 了 中 华 富 地 , 长 者 家 去 托 生 。
夫 妻 们 都 笑 着 说 : 我 和 媳 妇 都 有 这 样 的 梦 , 正 是 来 告 诉 我 , 不 料 到 母 亲 呼 唤 , 也 是 这 个 梦 。
于 是 就 叫 一 家 大 小 起 来 , 安 排 谢 恩 , 替 他 收 拾 马 匹 。 郅 都 来 到 面 前 拜 谢 说 : 多 谢 长 老 , 我 的 亡 父 脱 难 超 生 , 报 答 不 尽 。
三 藏 说 : 贫 僧 有 什 么 能 力 , 敢 劳 我 致 谢 。 伯 钦 把 三 个 口 儿 的 梦 话 对 三 藏 , 陈 述 了 一 遍 , 三 藏 也 很 高 兴 。
早 晨 供 给 了 素 斋 , 又 准 备 了 一 两 白 银 。
三 藏 分 文 不 接 受 。
一 家 的 儿 子 又 恳 切 地 叩 拜 , 三 藏 完 毕 , 分 文 没 有 接 受 。
只 是 说 : 是 你 肯 发 慈 悲 送 我 一 路 , 足 感 至 爱 。
伯 钦 和 母 亲 、 妻 子 无 可 奈 何 , 急 忙 做 了 些 粗 面 、 烧 饼 、 干 粮 , 叫 伯 钦 远 送 。
三 藏 很 高 兴 地 收 纳 了 。
太 保 领 着 母 亲 的 命 令 , 又 叫 来 两 三 个 家 僮 , 各 带 着 打 猎 的 器 械 , 一 同 上 了 大 路 。
看 不 到 那 山 中 的 野 景 , 岭 上 的 风 光 。
走 了 半 天 , 只 见 对 面 处 有 一 座 大 山 , 真 是 高 高 连 接 青 霄 , 崔 嵬 、 巍 峨 、 险 峻 。
三 藏 不 一 时 , 到 了 边 境 之 前 。
那 太 保 登 上 这 座 山 , 好 像 走 在 平 地 上 , 正 走 到 半 山 之 中 , 伯 钦 回 身 , 站 在 路 边 说 : 长 老 , 请 您 自 己 前 进 , 我 又 告 诉 我 回 去 。
三 藏 听 到 这 话 后 , 跃 马 下 马 说 : 千 万 敢 劳 太 保 再 送 我 一 路 。
伯 钦 说 : 长 老 不 知 道 。
此 山 称 为 两 界 山 , 东 半 边 属 我 大 唐 所 管 辖 , 西 半 边 是 靼 的 地 界 。
那 里 狼 虎 不 让 我 投 降 , 我 退 却 也 不 能 越 过 边 界 , 你 自 己 去 吧 。
三 藏 心 中 惊 慌 , 拉 开 手 , 拉 着 衣 服 , 握 着 袖 子 , 眼 泪 难 分 。
正 在 那 个 地 方 的 人 叮 嘱 拜 别 的 时 候 , 只 听 见 山 脚 下 的 人 喊 叫 声 如 雷 , 说 : 我 师 父 来 了 , 我 师 父 来 了 。
最 终 不 知 道 是 什 么 人 喊 叫 , 暂 且 听 下 回 的 分 析 解 释 。
}\switchcolumn\flushpage  \begin{pinyinscope}{\myfontt \section{第十四回}     心猿歸正 六賊無蹤

詩曰:
    佛即心兮心即佛,心佛從來皆要物。
    若知無物又無心,便是真心法身佛。
    法身佛,沒模樣,一顆圓光涵萬象。
    無體之體即真體,無相之相即實相。
    非色非空非不空,不來不向不回向。
    無異無同無有無,難捨難取難聽望。
    內外靈光到處同,一佛國在一沙中。
    一粒沙含大千界,一個身心萬法同。
    知之須會無心訣,不染不滯為淨業。
    善惡千端無所為,便是南無釋迦葉。

卻說那劉伯欽與唐三藏驚驚慌慌,又聞得叫聲「師父來也」。眾家僮道:「這叫
的必是那山腳下石匣中老猿。」太保道:「是他,是他。」三藏問:「是甚麼老
猿?」太保道:「這山舊名五行山,因我大唐王征西定國,改名兩界山。先年間
曾聞得老人家說:王莽篡漢之時,天降此山,下壓著一個神猴,不怕寒暑,不吃
飲食,自有土神監押,教他饑餐鐵丸,渴飲銅汁。自昔到今,凍餓不死。這叫必
定是他。長老莫怕,我們下山去看來。」三藏只得依從,牽馬下山。行不數里,
只見那石匣之間果有一猴,露著頭,伸著手,亂招手道:「師父,你怎麼此時才
來?來得好,來得好。救我出來,我保你上西天去也。」這長老近前細看,你道
他是怎生模樣:
尖嘴縮腮,金睛火眼。頭上堆苔蘚,耳中生薜蘿。鬢邊少髮多青草,頷下無鬚有
綠莎。眉間土,鼻凹泥,十分狼狽;指頭粗,手掌厚,塵垢餘多。還喜得眼睛轉
動,喉舌聲和。語言雖利便,身體莫能那。正是五百年前孫大聖,今朝難滿脫天
羅。

劉太保誠然膽大,走上前來,與他拔去了鬢邊草,頷下莎,問道:「你有甚麼說
話?」那猴道:「我沒話說,教那個師父上來,我問他一問。」三藏道:「你問
我甚麼?」那猴道:「你可是東土大王差往西天取經去的麼?」三藏道:「我正
是,你問怎麼?」那猴道:「我是五百年前大鬧天宮的齊天大聖,只因犯了誑上
之罪,被佛祖壓於此處。前者有個觀音菩薩,領佛旨意,上東土尋取經人。我教
他救我一救,他勸我再莫行兇,歸依佛法,盡慇懃保護取經人,往西方拜佛,功
成後自有好處。故此晝夜提心,晨昏弔膽,只等師父來救我脫身。我願保你取經
,與你做個徒弟。」三藏聞言,滿心歡喜道:「你雖有此善心,又蒙菩薩教誨,
願入沙門,只是我又沒斧鑿,如何救得你出?」那猴道:「不用斧鑿,你但肯救
我,我自出來也。」三藏道:「我自救你,你怎得出來?」那猴道:「這山頂上
有我佛如來的金字壓帖,你只上山去將帖兒揭起,我就出來了。」三藏依言,回
頭央浼劉伯欽道:「太保呵,我與你上山走一遭。」伯欽道:「不知真假何如?」
那猴高叫道:「是真,決不敢虛謬。」
  
伯欽只得呼喚家僮,牽了馬匹。他卻扶著三藏,復上高山。攀籐附葛,只行到那
極巔之處,果然見金光萬道,瑞氣千條,有塊四方大石,石上貼著一封皮,卻是
「唵嘛呢叭吽」六個金字。三藏近前跪下,朝石頭看著金字,拜了幾拜,望西禱
祝道:「弟子陳玄奘,特奉旨意求經。果有徒弟之分,揭得金字,救出神猴,同
證靈山;若無徒弟之分,此輩是個兇頑怪物,哄賺弟子,不成吉慶,便揭不得起
。」祝罷又拜。拜畢,上前將六個金字輕輕揭下。只聞得一陣香風,劈手把壓帖
兒刮在空中,叫道:「吾乃監押大聖者。今日他的難滿,吾等回見如來,繳此封
皮去也。」嚇得個三藏與伯欽一行人望空禮拜。徑下高山,又至石匣邊,對那猴
道:「揭了壓帖矣,你出來罷。」那猴歡喜,叫道:「師父,你請走開些,我好
出來,莫驚了你。」
  
伯欽聽說,領著三藏,一行人回東即走。走了五七里遠近,又聽得那猴高叫道:
「再走,再走。」三藏又行了許遠,下了山,只聞得一聲響喨,真個是地裂山崩
。眾人盡皆悚懼。只見那猴早到了三藏的馬前,赤淋淋跪下,道聲:「師父,我
出來也。」對三藏拜了四拜,急起身,與伯欽唱個大喏道:「有勞大哥送我師父
,又承大哥替我臉上薅草。」謝畢,就去收拾行李,扣背馬匹。那馬見了他,腰
軟蹄矬,戰兢兢的立站不住。蓋因那猴原是弼馬溫,在天上看養龍馬的,有些法
則,故此凡馬見他害怕。
  
三藏見他意思,實有好心,真個像沙門中的人物,便叫:「徒弟呵,你姓甚麼?」
猴王道:「我姓孫。」三藏道:「我與你起個法名,卻好呼喚。」猴王道:「不
勞師父盛意,我原有個法名,叫做孫悟空。」三藏歡喜道:「也正合我們的宗派
。你這個模樣,就像那小頭陀一般,我與你起個混名,稱為行者,好麼?」悟空
道:「好,好,好。」自此時又稱為孫行者。
  
那伯欽見孫行者一心收拾要行,卻轉身對三藏唱個喏道:「長老,你幸此間收得
個好徒,甚喜,甚喜。此人果然去得。我卻告回。」三藏躬身作禮相謝道:「多
有拖步,感激不勝。回府多多致意令堂老夫人、令荊夫人,貧僧在府多擾,容回
時踵謝。」伯欽回禮,遂此兩下分別。
  
卻說那孫行者請三藏上馬,他在前邊背著行李,赤條條,拐步而行。不多時,過
了兩界山,忽然見一隻猛虎,咆哮剪尾而來。三藏在馬上驚心。行者在路傍歡喜
道:「師父莫怕他,他是送衣服與我的。」放下行李,耳朵裏拔出一個針兒,迎
著風,幌一幌,原來是個碗來粗細一條鐵棒。他拿在手中,笑道:「這寶貝,五
百餘年不曾用著他,今日拿出來掙件衣服兒穿穿。」你看他拽開步,迎著猛虎,
道聲:「業畜!那裏去!」那隻虎蹲著身,伏在塵埃,動也不敢動動。卻被他照
頭一棒,就打的腦漿迸萬點桃紅,牙齒噴幾珠玉塊。諕得那陳玄奘滾鞍落馬,咬
指道聲:「天那!天那!劉太保前日打的斑斕虎,還與他鬥了半日;今日孫悟空
不用爭持,把這虎一棒打得稀爛。正是強中更有強中手!」
  
行者拖將虎來道:「師父略坐一坐,等我脫下他的衣服來,穿了走路。」三藏道
:「他那裏有甚衣服?」行者道:「師父莫管我,我自有處置。」好猴王,把毫
毛拔下一根,吹口仙氣,叫:「變!」變作一把牛耳尖刀,從那虎腹上挑開皮,
往下一剝,剝下個囫圇皮來。剁去了爪甲,割下頭來,割個四四方方一塊虎皮。
提起來,量了一量道:「闊了些兒,一幅可作兩幅。」拿過刀來,又裁為兩幅。
收起一幅,把一幅圍在腰間。路傍揪了一條葛籐,緊緊束定,遮了下體道:「師
父,且去,且去。到了人家,借些針線,再縫不遲。」他把條鐵棒捻一捻,依舊
像個針兒,收在耳裏。背著行李,請師父上馬。
  
兩個前進,長老在馬上問道:「悟空,你才打虎的鐵棒,如何不見?」行者笑道
:「師父,你不曉得。我這棍,本是東洋大海龍宮裏得來的,喚做天河鎮底神珍
鐵,又喚做如意金箍棒。當年大反天宮,甚是虧他。隨身變化,要大就大,要小
就小。剛才變做一個繡花針兒模樣,收在耳內矣。但用時,方可取出。」三藏聞
言暗喜。又問道:「方才那虎見了你,怎麼就不動動?讓你自在打他,何說?」
悟空道:「不瞞師父說,莫道是隻虎,就是一條龍,見了我也不敢無禮。我老孫
頗有降龍伏虎的手段,翻江攪海的神通;見貌辨色,聆音察理;大之則量於宇宙
,小之則攝於毫毛;變化無端,隱顯莫測。剝這個虎皮,何為稀罕?若到那疑難
處,看展本事麼。」三藏聞得此言,愈加放懷無慮,策馬前行。
  
師徒兩個走著路,說著話,不覺得太陽西墜。但見:
燄燄斜暉返照,天涯海角歸雲。千山鳥雀噪聲頻,覓宿投林成陣。野獸雙雙對對
,回窩族族群群。一鉤新月破黃昏。萬點明星光暈。
行者道:「師父走動些,天色晚了。那壁廂樹木森森,想必是人家莊院,我們趕
早投宿去來。」三藏果策馬而行,徑奔人家,到了莊院前下馬。行者撇了行李,
走上前,叫聲:「開門!開門!」那裏面有一老者扶筇而出,?喇的開了門。看
見行者這般惡相,腰繫著一塊虎皮,好似個雷公模樣,諕得腳軟身麻,口出譫語
道:「鬼來了!鬼來了!」三藏近前攙住,叫道:「老施主休怕。他是我貧僧的
徒弟,不是鬼怪。」老者抬頭,見了三藏的面貌清奇,方才立定,問道:「你是
那寺裏來的和尚,帶這惡人上我門來?」三藏道:「我貧僧是唐朝來的,往西天
拜佛求經。適路過此間,天晚,特造檀府借宿一宵,明早不犯天光就行。萬望方
便一二。」老者道:「你雖是個唐人,那個惡的卻非唐人。」悟空厲聲高呼道:
「你這個老兒全沒眼色!唐人是我師父,我是他徒弟。我也不是甚糖人,蜜人,
我是齊天大聖!你們這裏人家,也有認得我的。我也曾見你來。」那老者道:
「你在那裏見我?」悟空道:「你小時不曾在我面前扒柴?不曾在我臉上挑菜?」
老者道:「這廝胡說!你在那裏住?我在那裏住?我來你面前扒柴、挑菜?」悟
空道:「我兒子便胡說。你是認不得我了,我本是這兩界山石匣中的大聖,你再
認認看。」老者方才省悟道:「你倒有些像他。但你是怎麼得出來的?」悟空將
菩薩勸善,令他等待唐僧揭帖脫身之事,對那老者細說了一遍。
  
老者卻才下拜,將唐僧請到裏面,即喚老妻與兒女都來相見,具言前事,個個忻
喜。又命看茶。茶罷,問悟空道:「大聖呵,你也有年紀了?」悟空道:「你今
年幾歲了?」老者道:「我痴長一百三十歲了。」行者道:「還是我重子重孫哩
。我那生身的年紀,我不記得是幾時;但只在這山腳下,已五百餘年了。」老者
道:「是有,是有。我曾記得祖公公說,此山乃從天降下,就壓了一個神猴。只
到如今,你才脫體。我那小時見你時,你頭上有草,臉上有泥,還不怕你;如今
臉上無了泥,頭上無了草,卻像瘦了些,腰間又苫了一塊大虎皮,與鬼怪能差多
少?」一家兒聽得這般話說,都呵呵大笑。
  
這老兒頗賢,即令安排齋飯。飯後,悟空道:「你家姓甚?」老者道:「舍下姓
陳。」三藏聞言,即下來起手道:「老施主與貧僧是華宗。」行者道:「師父,
你是唐姓,怎的和他是華宗?」三藏道:「我俗家也姓陳,乃是唐朝海州弘農郡
聚賢莊人氏。我的法名叫做陳玄奘。只因我大唐太宗皇帝賜我做御弟三藏,指唐
為姓,故名唐僧也。」那老者見說同姓,又十分歡喜。
  
行者道:「老陳,左右打攪你家,我有五百多年不洗澡了,你可去燒些湯來,與
我師徒們洗浴洗浴,一發臨行謝你。」那老兒即令燒湯拿盆,掌上燈火。師徒浴
罷,坐在燈前。行者道:「老陳,還有一事累你:有針線借我用用。」那老兒道
:「有,有,有。」即教媽媽取針線來,遞與行者。行者又有眼色,見師父洗浴
,脫下一件白布短小直裰未穿,他即扯過來披在身上。卻將那虎皮脫下,聯接一
處。打一個馬面樣的摺子,圍在腰間,勒了籐條,走到師父面前道:「老孫今日
這等打扮,比昨日如何?」三藏道:「好,好,好。這等樣,才像個行者。」三
藏道:「徒弟,你不嫌殘舊,那件直裰兒,你就穿了罷。」悟空唱個喏道:「承
賜,承賜。」他又去尋些草料喂了馬。此時各各事畢,師徒與那老兒亦各歸寢。
  
次早,悟空起來,請師父走路。三藏著衣,教行者收拾鋪蓋行李。正欲告辭,只
見那老兒早具臉湯,又具齋飯。齋罷,方才起身。三藏上馬,行者引路。不覺饑
餐渴飲,夜宿曉行。又值初冬時候,但見那:
霜凋紅葉千林瘦,嶺上幾株松柏秀。未開梅蕊散香幽,暖短晝,小春候,菊殘荷
盡山茶茂,寒橋古樹爭枝鬥。曲澗涓涓泉水溜,淡雲欲雪滿天浮。朔風驟,牽衣
袖,向晚寒威人怎受?
  
師徒們正走多時,忽見路傍?哨一聲,闖出六個人來,各執長槍短劍,利刃強弓
,大?一聲道:「那和尚那裏走!趕早留下馬匹,放下行李,饒你性命過去。」
諕得那三藏魂飛魄散,跌下馬來,不能言語。行者用手扶起道:「師父放心,沒
些兒事,這都是送衣服送盤纏與我們的。」三藏道:「悟空,你想有些耳閉。他
說教我們留馬匹、行李,你倒問他要甚麼衣服、盤纏。」行者道:「你管守著衣
服、行李、馬匹,待老孫與他爭持一場,看是何如。」三藏道:「好手不敵雙拳
,雙拳不如四手。他那裏六條大漢,你這般小小的一個人兒,怎麼敢與他爭持?」
  行者的膽量原大,那容分說,走上前來,叉手當胸,對那六個人施禮道:
「列位有甚麼緣故,阻我貧僧的去路?」那人道:「我等是剪徑的大王,行好心
的山主。大名久播,你量不知。早早的留下東西,放你過去;若道半個『不』字
,教你碎屍粉骨。」行者道:「我也是祖傳的大王,積年的山主,卻不曾聞得列
位有甚大名。」那人道:「你是不知,我說與你聽:一個喚做眼看喜,一個喚做
耳聽怒,一個喚做鼻嗅愛,一個喚作舌嘗思,一個喚作意見慾,一個喚作身本憂
。」悟空笑道:「原來是六個毛賊。你卻不認得我這出家人是你的主人公,你倒
來擋路。把那打劫的珍寶拿出來,我與你作七分兒均分,饒了你罷。」那賊聞言
,喜的喜,怒的怒,愛的愛,思的思,慾的慾,憂的憂,一齊上前亂嚷道:「這
和尚無禮。你的東西全然沒有,轉來和我等要分東西。」他掄槍舞劍,一擁前來
,照行者劈頭亂砍,乒乒乓乓,砍有七八十下。悟空停立中間,只當不知。那賊
道:「好和尚,真個的頭硬。」行者笑道:「將就看得過罷了。你們也打得手困
了,卻該老孫取出個針兒來耍耍。」那賊道:「這和尚是一個行針灸的郎中變的
。我們又無病症,說甚麼動針的話?」
  
行者伸手去耳朵裏拔出一根繡花針兒,迎風一幌,卻是一條鐵棒,足有碗來粗細
。拿在手中道:「不要走,也讓老孫打一棍兒試試手。」諕得這六個賊四散逃走
。被他拽開步,團團趕上,一個個盡皆打死。剝了他的衣服,奪了他的盤纏,笑
吟吟走將來道:「師父請行,那賊已被老孫剿了。」三藏道:「你十分撞禍。他
雖是剪徑的強徒,就是拿到官司,也不該死罪;你縱有手段,只可退他去便了,
怎麼就都打死?這卻是無故傷人的性命,如何做得和尚?出家人掃地恐傷螻蟻命
,愛惜飛蛾紗罩燈。你怎麼不分皂白,一頓打死?全無一點慈悲好善之心。早還
是山野中無人查考;若到城市,倘有人一時沖撞了你,你也行兇,執著棍子亂打
傷人,我可做得白客,怎能脫身?」悟空道:「師父,我若不打死他,他卻要打
死你哩。」三藏道:「我這出家人寧死,決不敢行兇。我就死,也只是一身,你
卻殺了他六人,如何理說?此事若告到官,就是你老子做官,也說不過去。」行
者道:「不瞞師父說,我老孫五百年前,據花果山稱王為怪的時節,也不知打死
多少人。假似你說這般到官,倒也得些狀告是。」三藏道:「只因你沒收沒管,
暴橫人間,欺天誑上,才受這五百年前之難。今既入了沙門,若是還像當時行兇
,一味傷生,去不得西天,做不得和尚。忒惡!忒惡!」
  
原來這猴子一生受不得人氣。他見三藏只管緒緒叨叨,按不住心頭火發道:「你
既是這等,說我做不得和尚,上不得西天,不必恁般絮聒惡我,我回去便了。」
那三藏卻不曾答應。他就使一個性子,將身一縱,說一聲:「老孫去也!」三藏
急抬頭,早已不見。只聞得呼的一聲,回東而去。撇得那長老孤孤零零,點頭自
歎,悲怨不已道:「這廝這等不受教誨,我但說他幾句,他怎麼就無形無影的徑
回去了?罷罷罷,也是我命裏不該招徒弟,進人口。如今欲尋他無處尋,欲叫他
叫不應,去來,去來。」正是:捨身拚命歸西去,莫倚傍人自主張。
  
那長老只得收拾行李,捎在馬上,也不騎馬,一隻手拄著錫杖,一隻手揪著韁繩
,淒淒涼涼,往西前進。行不多時,只見山路前面有一個年高的老母,捧一件綿
衣,綿衣上有一頂花帽。三藏見他來得至近,慌忙牽馬,立於右側讓行。那老母
問道:「你是那裏來的長老,孤孤恓恓獨行於此?」三藏道:「弟子乃東土大唐
奉差往西天拜活佛求真經者。」老母道:「西方佛乃大雷音寺天竺國界,此去有
十萬八千里路。你這等單人獨馬,又無個伴侶,又無個徒弟,你如何去得?」三
藏道:「弟子日前收得一個徒弟,他性潑兇頑,是我說了他幾句,他不受教,遂
渺然而去也。」老母道:「我有這一領綿布直裰、一頂嵌金花帽,原是我兒子用
的,他只做了三日和尚,不幸命短身亡。我才去他寺裏哭了一場,辭了他師父,
將這兩件衣、帽拿來,做個憶念。長老呵,你既有徒弟,我把這衣帽送了你罷。」
三藏道:「承老母盛賜,但只是我徒弟已走了,不敢領受。」老母道:「他那廂
去了?」三藏道:「我聽得呼的一聲,他回東去了。」老母道:「東邊不遠,就
是我家,想必往我家去了。我那裏還有一篇咒兒,喚做『定心真言』,又名做
『緊箍兒咒』。你可暗暗的念熟,牢記心頭,再莫洩漏一人知道。我去趕上他,
叫他還來跟你,你卻將此衣帽與他穿戴。他若不服你使喚,你就默念此咒,他再
不敢行兇,也再不敢去了。」三藏聞言,低頭拜謝。那老母化一道金光,回東而
去。三藏情知是觀音菩薩授此真言,急忙撮土焚香,望東懇懇禮拜。拜罷,收了
衣帽,藏在包袱中間。卻坐於路傍,誦習那定心真言,來回念了幾遍,念得爛熟
,牢記心胸不題。
  
卻說那悟空別了師父,一觔斗雲,徑轉東洋大海。按住雲頭,分開水道,徑至水
晶宮前。早驚動龍王出來迎接,接至宮裏坐下。禮畢,龍王道:「近聞得大聖難
滿,失賀!想必是重整仙山,復歸古洞矣?」悟空道:「我也有此心性,只是又
做了和尚了。」龍王道:「做甚和尚?」行者道:「我虧了南海菩薩勸善,教我
正果,隨東土唐僧上西方拜佛,皈依沙門,又喚為行者了。」龍王道:「這等真
是可賀,可賀。這才叫做改邪歸正,懲創善心。既如此,怎麼不西去,復東回何
也?」行者笑道:「那是唐僧不識人性。有幾個毛賊剪徑,是我將他打死,唐僧
就緒緒叨叨,說了我若干的不是。你想老孫可是受得悶氣的?是我撇了他,欲回
本山,故此先來望你一望,求鍾茶吃。」龍王道:「承降,承降。」當時龍子、
龍孫即捧香茶來獻。
  
茶畢,行者回頭一看,見後壁上掛著一幅「圯橋進履」的畫兒。行者道:「這是
甚麼景致?」龍王道:「大聖在先,此事在後,故你不認得。這叫做『圯橋三進
履』。」行者道:「怎的是『三進履』?」龍王道:「此仙乃是黃石公,此子乃
是漢世張良,石公坐在圯橋上,忽然失履於橋下,遂喚張良取來。此子即忙取來
,跪獻於前。如此三度,張良略無一毫倨傲怠慢之心,石公遂愛他勤謹,夜授天
書,著他扶漢。後果然運籌帷幄之中,決勝千里之外。太平後,棄職歸山,從赤
松子遊,悟成仙道。大聖,你若不保唐僧,不盡勤勞,不受教誨,到底是個妖仙
,休想得成正果。」悟空聞言,沉吟半晌不語。龍王道:「大聖自當裁處,不可
圖自在,誤了前程。」悟空道:「莫多話,老孫還去保他便了。」龍王忻喜道:
「既如此,不敢久留,請大聖早發慈悲,莫要疏久了你師父。」
  
行者見他催促請行,急縱身,出離海藏,駕著雲,別了龍王。正走,卻遇著南海
菩薩。菩薩道:「孫悟空,你怎麼不受教誨,不保唐僧,來此處何幹?」慌得個
行者在雲端裏施禮道:「向蒙菩薩善言,果有唐朝僧到,揭了壓帖,救了我命,
跟他做了徒弟。他卻怪我兇頑,我才閃他一閃,如今就去保他也。」菩薩道:
「趕早去,莫錯過了念頭。」言畢各回。
  
這行者須臾間看見唐僧在路傍悶坐。他上前道:「師父,怎麼不走路?還在此做
甚?」三藏抬頭道:「你往那裏去來?教我行又不敢行,動又不敢動,只管在此
等你。」行者道:「我往東洋大海老龍王家討茶吃吃。」三藏道:「徒弟呵,出
家人不要說謊。你離了我沒多一個時辰,就說到龍王家吃茶?」行者笑道:「不
瞞師父說,我會駕觔斗雲,一個觔斗有十萬八千里路,故此得即去即來。」三藏
道:「我略略的言語重了些兒,你就怪我,使個性子丟了我去。像你這有本事的
,討得茶吃;像我這去不得的,只管在此忍餓。你也過意不去呀。」行者道:
「師父,你若餓了,我便去與你化些齋吃。」三藏道:「不用化齋,我那包袱裏
還有些乾糧,是劉太保母親送的。你去拿缽盂尋些水來,等我吃些兒走路罷。」
  
行者去解開包袱,在那包裹中間見有幾個粗麵燒餅,拿出來遞與師父。又見那光
豔豔的一領綿布直裰、一頂嵌金花帽。行者道:「這衣帽是東土帶來的?」三藏
就順口兒答應道:「是我小時穿戴的。這帽子若戴了,不用教經,就會念經;這
衣服若穿了,不用演禮,就會行禮。」行者道:「好師父,把與我穿戴了罷。」
三藏道:「只怕長短不一,你若穿得,就穿了罷。」行者遂脫下舊白布直裰,將
綿布直裰穿上,也就是比量著身體裁的一般。把帽兒戴上。三藏見他戴上帽子,
就不吃乾糧,卻默默的念那緊箍咒一遍。行者叫道:「頭痛,頭痛。」那師父不
住的又念了幾遍,把個行者痛得打滾,抓破了嵌金的花帽。三藏又恐怕扯斷金箍
,住了口不念。不念時,他就不痛了。伸手去頭上摸摸,似一條金線兒模樣,緊
緊的勒在上面,取不下,揪不斷,已此生了根了。他就耳裏取出針兒來,插入箍
裏,往外亂捎。三藏又恐怕他捎斷了,口中又念起來。他依舊生痛,痛得豎蜻蜓
,翻觔斗,耳紅面赤,眼脹身麻。那師父見他這等,又不忍不捨,復住了口。他
的頭又不痛了。行者道:「我這頭,原來是師父咒我的?」三藏道:「我念得是
緊箍經,何曾咒你?」行者道:「你再念念看。」三藏真個又念。行者真個又痛
,只教:「莫念,莫念。念動我就痛了。這是怎麼說?」三藏道:「你今番可聽
我教誨了?」行者道:「聽教了。」「你再可無禮了?」行者道:「不敢了。」
  
他口裏雖然答應,心上還懷不善,把那針兒幌一幌,碗來粗細,望唐僧就欲下手
。慌得長老口中又念了兩三遍。這猴子跌倒在地,丟了鐵棒,不能舉手,只教:
「師父,我曉得了。再莫念,再莫念。」三藏道:「你怎麼欺心,就敢打我?」
行者道:「我不曾敢打。我問師父,你這法兒是誰教你的?」三藏道:「是適間
一個老母傳授我的。」行者大怒道:「不消講了,這個老母,坐定是那個觀世音
。他怎麼那等害我?等我上南海打他去。」三藏道:「此法既是他授與我,他必
然先曉得了。你若尋他,他念起來,你卻不是死了?」行者見說得有理,真個不
敢動身,只得回心,跪下哀告道:「師父,這是他奈何我的法兒,教我隨你西去
。我也不去惹他,你也莫當常言,只管念誦。我願保你,再無退悔之意了。」三
藏道:「既如此,伏侍我上馬去也。」那行者才死心塌地,抖擻精神,束一束綿
布直裰,扣背馬匹,收拾行李,奔西而進。
  
    畢竟這一去,後面又有甚話說,且聽下回分解。




}  \end{pinyinscope}\switchcolumn{\myfontc \section{第 十 四 回} 心 猿 归 正 六 贼 无 踪 诗 说 : 佛 就 是 心 , 心 就 是 佛 , 心 佛 从 来 都 是 要 物 。
如 果 知 道 无 物 又 无 心 , 就 是 真 心 法 身 佛 。
法 身 佛 , 没 有 模 样 , 一 颗 圆 光 包 含 万 物 。
没 有 形 体 的 形 体 就 是 真 实 的 形 体 , 没 有 形 相 的 形 象 就 是 实 相 。
不 是 色 、 非 是 空 , 不 是 不 是 空 , 不 来 不 去 , 不 回 向 不 回 向 。
无 异 无 同 , 没 有 无 , 难 以 舍 弃 , 难 以 取 取 , 难 以 听 取 。
内 外 灵 光 到 处 都 是 一 样 的 , 一 个 佛 国 就 在 一 个 沙 子 里 。
一 粒 沙 包 含 大 千 界 , 一 个 身 心 万 法 都 一 样 。
知 道 , 必 须 领 会 无 心 的 秘 诀 , 不 染 染 不 滞 留 , 这 就 是 净 业 。
善 恶 千 端 无 所 作 为 , 就 是 南 无 释 迦 叶 。
又 说 : 那 刘 伯 钦 与 唐 三 藏 惊 慌 , 又 听 到 叫 声 : 师 父 来 了 。
众 家 僮 说 : 这 个 叫 的 一 定 是 那 山 脚 下 石 匣 中 的 老 猿 。
太 保 说 : 是 他 , 是 他 。
三 藏 问 : 是 什 么 老 猿 ? 太 保 说 : 这 座 山 原 来 名 叫 五 行 山 , 因 为 我 大 唐 王 征 西 定 国 , 改 名 为 两 界 山 。
先 前 曾 听 老 人 家 说 : 王 莽 篡 夺 汉 朝 的 时 候 , 上 天 降 下 此 山 , 下 面 压 着 一 个 神 猴 , 不 怕 寒 暑 , 不 吃 饮 食 , 自 有 土 神 监 押 , 教 他 饥 饿 吃 铁 丸 , 渴 渴 喝 铜 汁 。
从 古 到 今 , 冻 饿 不 死 。
这 个 叫 , 必 定 是 他 。
长 老 不 要 害 怕 , 我 们 下 山 去 看 看 。
三 藏 只 得 依 从 , 牵 着 马 下 山 。
走 了 不 到 几 里 , 只 见 那 石 匣 中 果 然 有 一 个 猴 子 , 露 着 头 , 伸 着 手 , 乱 招 手 说 : 师 父 , 你 为 什 么 这 时 才 来 , 来 得 好 , 来 得 得 好 。
救 我 出 来 , 我 保 你 上 西 天 去 。
这 位 长 老 靠 近 前 仔 细 看 , 你 说 他 是 什 么 样 子 ? 尖 嘴 缩 腮 , 金 睛 火 眼 。
头 上 堆 满 了 苔 藓 , 耳 朵 里 生 出 了 萝 。
鬓 边 少 发 多 青 草 , 颧 下 没 有 胡 须 , 有 绿 莎 。
眉 间 土 , 鼻 子 凹 陷 泥 土 , 十 分 狼 狈 ; 指 头 粗 , 手 掌 厚 , 尘 垢 多 。
还 喜 得 眼 睛 转 动 , 喉 舌 声 音 和 谐 。
语 言 虽 然 有 利 , 身 体 也 无 法 阻 挡 。
正 是 五 百 年 前 的 孙 子 大 圣 人 , 今 朝 难 满 脱 天 网 。
刘 太 保 果 然 胆 大 , 走 上 前 来 , 与 他 拔 去 鬓 边 草 , 捋 下 莎 草 , 问 道 : 你 有 什 么 说 话 ? 那 猴 说 : 我 没 有 话 说 , 教 那 个 师 父 上 来 , 我 问 他 一 问 。
三 藏 说 : 你 问 我 什 么 ? 那 猴 说 : 你 可 是 东 土 大 王 派 我 去 西 天 取 经 去 的 吗 ? 三 藏 说 : 我 正 是 , 你 问 怎 么 样 ? 那 猴 说 : 我 是 五 百 年 前 大 闹 天 宫 的 齐 天 大 圣 , 只 因 为 犯 了 上 的 罪 , 被 佛 祖 压 在 这 个 地 方 。
以 前 有 个 观 音 菩 萨 , 领 悟 佛 祖 的 旨 意 , 到 东 方 去 寻 找 经 书 的 人 。
吾 教 他 救 我 一 救 , 他 劝 我 再 不 要 行 凶 , 归 依 佛 法 , 尽 力 保 护 佛 经 的 人 , 去 西 方 拜 佛 , 功 成 之 后 , 自 有 好 处 。
所 以 我 日 夜 提 心 , 早 晚 祈 祷 , 只 等 师 父 来 救 我 脱 身 。
我 愿 保 你 取 经 , 给 你 做 个 弟 弟 。
三 藏 听 了 这 话 , 满 心 欢 喜 地 说 : 你 虽 然 有 这 样 的 善 心 , 又 承 蒙 菩 萨 的 教 诲 , 愿 意 进 入 沙 门 , 只 是 我 又 没 有 斧 凿 , 怎 么 能 救 得 你 出 来 呢 ? 那 猴 说 : 不 用 斧 凿 , 你 只 要 肯 救 我 , 我 自 己 出 来 。
三 藏 菩 萨 说 : 我 自 己 救 你 , 你 怎 么 能 出 来 ? 那 猴 说 : 这 山 顶 上 有 我 佛 如 来 的 金 字 贴 帖 , 你 只 要 上 山 去 把 帖 子 揭 起 来 , 我 就 出 来 了 。
三 藏 依 照 他 的 话 , 回 头 向 刘 伯 钦 说 : 太 保 呵 , 我 和 你 上 山 走 一 遍 。
伯 钦 说 : 不 知 道 真 假 怎 么 样 ? 那 猴 高 声 大 叫 道 : 真 , 决 不 敢 虚 假 。
伯 钦 只 得 呼 唤 家 僮 , 牵 着 马 匹 。
其 人 曰 : 吾 不 知 之 , 则 不 知 之 。
攀 藤 附 葛 , 只 走 到 那 最 高 顶 的 地 方 , 果 然 看 见 金 光 万 道 , 瑞 气 千 条 。 有 块 四 方 的 大 石 头 , 石 上 贴 着 一 封 皮 , 却 是 喇 喇 喇 喇 喇 喇 嘛 六 个 金 字 。
三 藏 靠 近 前 跪 下 , 朝 着 石 头 看 着 金 字 , 拜 了 几 拜 , 望 着 西 边 祷 告 说 : 弟 子 陈 玄 奘 , 特 奉 圣 旨 求 经 。
果 真 有 弟 弟 的 分 别 , 揭 出 金 字 , 救 出 神 猴 , 同 证 灵 山 。 如 果 没 有 弟 弟 的 份 份 , 那 么 , 这 些 人 就 是 个 凶 顽 怪 物 , 哄 骗 弟 子 , 不 成 吉 庆 , 就 揭 不 起 来 。
祝 辞 谢 , 又 拜 谢 。
拜 完 , 皇 上 上 前 把 六 个 金 字 轻 轻 地 揭 下 。
只 听 得 到 一 股 香 风 , 他 劈 开 手 把 压 帖 儿 刮 在 空 中 , 喊 道 : 我 是 监 押 大 圣 的 人 。
今 天 他 的 话 难 以 满 足 , 我 们 回 去 见 如 来 , 就 把 这 封 皮 送 回 去 了 。
他 惊 吓 得 个 三 藏 和 伯 钦 一 行 人 望 见 空 中 礼 拜 。
径 直 下 到 高 山 , 又 到 石 匣 边 , 对 那 个 猴 子 说 : 揭 了 压 贴 了 , 你 出 来 吧 。
那 猴 欢 喜 地 喊 道 : 师 父 , 你 请 你 走 开 些 , 我 好 出 来 , 不 要 惊 吓 了 你 。
伯 钦 听 了 , 带 着 三 藏 , 一 个 人 回 到 东 边 就 走 了 。
走 了 五 七 里 远 近 , 又 听 到 那 个 猴 子 高 高 地 叫 道 : 再 走 , 再 走 。
三 藏 又 走 了 很 远 , 下 了 山 , 只 听 到 一 声 响 响 , 真 是 地 裂 山 崩 。
众 人 都 很 害 怕 。
只 见 那 个 猴 子 早 上 来 到 三 藏 马 前 , 赤 色 淋 漓 地 跪 下 , 说 道 : 师 父 , 我 出 来 了 。
对 三 藏 拜 了 四 拜 , 急 忙 起 身 , 和 伯 钦 一 唱 大 锣 说 : 有 劳 大 哥 送 我 师 父 , 又 承 大 哥 替 我 的 脸 上 刈 草 。
说 完 , 就 走 了 , 收 拾 行 李 , 敲 着 背 上 的 马 匹 。
其 马 见 到 了 他 , 腰 子 软 , 蹄 子 颤 颤 , 立 刻 站 立 不 住 。
因 为 那 猴 原 来 是 弼 马 温 , 在 天 上 看 养 龙 马 的 , 有 一 点 法 则 , 所 以 凡 马 见 到 它 就 害 怕 。
三 藏 看 见 他 的 意 思 , 确 实 有 好 心 , 真 是 和 尚 中 的 人 物 , 就 喊 道 : 徒 弟 呵 , 你 姓 什 么 ? 猴 王 说 : 我 姓 孙 。
三 藏 说 : 我 给 你 起 个 法 名 , 却 好 呼 唤 。
猴 王 说 : 不 劳 师 父 大 意 , 我 愿 有 个 法 名 , 叫 做 孙 悟 空 。
三 藏 欢 喜 地 说 : 也 正 合 我 们 的 宗 族 。
你 的 模 样 , 就 像 那 个 小 头 陀 一 样 , 我 给 你 起 个 混 名 , 称 为 行 者 , 好 吗 ?
从 此 以 后 , 又 被 称 为 孙 行 。
那 伯 钦 见 孙 行 者 一 心 收 拾 着 想 要 走 , 又 转 身 对 三 藏 大 声 喊 道 : 长 老 , 你 幸 亏 这 里 收 到 一 个 好 徒 弟 , 非 常 高 兴 , 很 高 兴 。
这 个 人 果 然 离 开 了 。
我 又 告 诉 我 回 去 。
三 藏 亲 自 行 礼 , 互 相 感 谢 说 : 我 多 有 拖 步 , 我 感 激 不 胜 。
回 府 多 次 致 意 令 堂 老 夫 人 、 令 荆 夫 人 , 贫 僧 在 府 中 多 受 扰 扰 , 容 回 时 常 跟 着 谢 恩 。
伯 钦 回 礼 , 从 此 两 下 分 别 。
又 说 : 那 个 孙 子 请 三 藏 上 马 , 他 在 前 边 背 着 行 李 , 红 色 丝 带 , 拐 着 步 行 走 。
不 多 时 , 经 过 两 列 山 , 忽 然 看 见 一 只 猛 虎 , 吼 叫 着 剪 断 尾 巴 来 。
三 藏 在 马 上 惊 心 。
行 人 在 路 旁 欢 喜 地 说 : 师 父 不 要 怕 他 , 他 是 送 衣 服 给 我 的 。
他 放 下 行 李 , 从 耳 朵 里 拔 出 一 个 针 子 , 迎 着 风 , 打 了 一 个 账 子 , 原 来 是 个 碗 , 粗 细 的 一 条 铁 棒 。
他 笑 着 说 : 这 个 宝 贝 , 五 百 多 年 来 不 曾 用 它 , 今 天 拿 出 来 , 拿 一 件 衣 服 儿 穿 。
你 看 他 拉 开 步 , 迎 着 猛 虎 , 喊 道 : 业 畜 , 往 往 去 , 那 只 老 虎 蹲 在 尘 埃 里 , 动 也 不 敢 动 。
又 被 他 照 头 一 棒 , 就 打 了 他 的 脑 浆 , 迸 出 万 点 桃 花 , 牙 齿 喷 出 几 颗 珠 玉 块 。
得 那 陈 玄 奘 翻 鞍 落 马 , 咬 着 指 头 说 道 : 天 那 ! 天 那 , 刘 太 保 前 日 打 的 斑 虎 , 还 与 他 打 了 半 天 ; 今 天 孙 悟 空 不 用 争 夺 , 就 把 这 个 老 虎 一 棒 打 得 稀 烂 。
行 人 拖 着 虎 来 说 : 师 父 略 坐 一 下 , 等 我 脱 下 他 的 衣 服 来 , 穿 上 走 路 。
三 藏 说 : 他 哪 里 有 什 么 衣 服 ? 行 者 说 : 师 父 不 要 管 我 , 我 自 有 处 置 。
猴 王 , 把 毫 毛 拔 下 一 根 , 吹 出 一 根 仙 气 , 喊 道 : 变 变 了 一 把 牛 耳 尖 刀 , 从 那 虎 腹 上 挑 开 皮 , 往 下 去 一 掉 , 剥 下 一 个 痈 皮 来 。
割 去 爪 甲 , 割 下 头 来 , 割 下 四 四 方 方 , 一 块 虎 皮 。
提 起 来 , 量 了 一 量 , 说 : 宽 了 些 儿 , 一 幅 可 作 两 幅 。
他 拿 过 刀 来 , 又 裁 成 两 幅 。
收 起 一 幅 , 把 一 幅 包 在 腰 间 。
在 路 旁 又 扎 了 一 根 葛 藤 , 紧 紧 紧 紧 紧 紧 地 束 住 , 遮 住 了 下 肢 说 : 师 父 , 暂 且 去 吧 , 暂 且 去 吧 。
到 了 别 人 家 , 借 一 点 针 线 , 再 缝 也 不 迟 。
其 人 曰 : 吾 不 知 之 , 不 知 之 。
背 着 行 李 , 请 师 父 上 马 。
两 个 人 向 前 走 , 长 老 在 马 上 问 道 : 我 才 打 虎 的 铁 棒 , 为 什 么 不 见 了 ? 行 人 笑 着 说 : 师 父 , 你 不 知 道 。
我 这 个 棍 子 , 本 是 东 洋 大 海 龙 宫 里 得 来 的 , 叫 做 天 河 镇 底 神 珍 铁 , 又 叫 做 如 意 金 纽 棒 。
当 年 大 反 天 宫 , 非 常 亏 损 其 他 。
随 身 而 变 化 , 要 大 就 大 , 要 小 就 小 。
刚 才 变 成 一 个 绣 花 针 子 的 模 样 , 收 在 耳 里 。
只 要 用 时 , 才 可 以 取 出 。
三 藏 听 了 这 话 , 暗 自 高 兴 。
又 问 道 : 方 才 那 老 虎 见 了 你 , 为 什 么 不 动 动 , 让 你 自 己 在 打 他 , 有 什 么 说 的 ? 悟 空 说 : 不 骗 师 父 说 , 不 要 说 是 一 只 虎 , 就 是 一 条 龙 , 见 了 我 也 不 敢 无 礼 。
我 的 老 孙 子 很 有 投 降 龙 伏 虎 的 手 段 , 翻 江 搅 海 的 神 通 , 看 见 面 貌 , 辨 别 颜 色 , 听 到 声 音 , 观 察 事 理 ; 大 的 东 西 就 能 够 衡 量 宇 宙 , 小 的 东 西 就 能 够 从 毫 毛 上 看 出 来 。
剥 这 个 虎 皮 , 为 什 么 稀 罕 呢 ? 如 果 到 那 疑 难 处 , 看 你 的 本 事 吗 ?
三 藏 听 到 了 这 话 , 更 加 放 荡 无 虑 , 策 马 前 行 。
师 徒 两 个 人 在 路 上 走 , 说 话 , 不 觉 得 太 阳 向 西 坠 落 。
只 见 火 焰 斜 晖 返 照 , 天 涯 海 角 归 云 。
千 山 鸟 雀 噪 声 频 繁 , 找 宿 投 林 成 阵 。
野 兽 双 双 对 对 , 回 窝 族 人 群 群 聚 集 。
一 钩 新 月 破 黄 昏 。
万 点 明 星 光 晕 。
行 人 说 : 师 父 跑 一 点 , 天 色 晚 了 。
那 里 墙 边 树 木 森 森 , 想 必 是 人 家 的 庄 院 , 我 们 赶 早 投 宿 去 来 。
三 藏 果 然 策 马 而 行 , 径 直 奔 到 别 人 家 , 到 了 庄 院 前 下 马 。
行 人 摘 下 行 李 , 走 上 前 , 大 叫 道 : 开 门 , 开 门 , 那 里 面 有 一 个 老 人 扶 着 鞋 子 出 来 。
看 见 行 人 这 样 恶 劣 的 样 子 , 腰 上 系 着 一 块 虎 皮 , 好 像 是 个 雷 公 的 样 子 , 得 脚 软 身 麻 , 口 里 说 出 语 说 : 鬼 来 了 , 鬼 来 了 。 三 藏 靠 近 前 来 搀 住 , 喊 道 : 老 施 主 不 要 害 怕 。
他 是 贫 僧 的 徒 弟 , 不 是 鬼 怪 。
老 人 抬 头 , 见 到 三 藏 的 面 貌 清 新 奇 异 , 刚 刚 立 定 , 问 道 : 你 是 那 寺 里 来 的 和 尚 , 带 着 这 个 恶 人 上 我 的 门 来 , 三 藏 说 : 我 是 个 贫 僧 , 是 唐 朝 来 的 , 去 西 天 去 拜 佛 求 经 。
恰 巧 路 过 这 里 , 天 色 已 晚 , 特 地 到 檀 府 借 宿 一 夜 , 第 二 天 早 晨 不 冒 犯 天 光 就 走 了 。
万 望 方 便 一 二 。
老 人 说 : 你 虽 然 是 个 唐 人 , 那 个 恶 的 却 不 是 唐 人 。
悟 空 厉 声 高 呼 道 : 你 这 个 老 子 完 全 没 有 眼 色 , 唐 人 是 我 的 师 父 , 我 是 他 的 弟 弟 。
我 不 是 甚 糖 人 , 蜜 人 , 我 是 齐 天 大 圣 , 你 们 这 里 的 人 , 也 有 认 得 我 的 。
我 曾 见 过 你 来 。
那 个 老 人 说 : 你 在 哪 里 见 我 ? 悟 空 说 : 你 小 时 候 不 曾 在 我 脸 前 抓 柴 , 不 曾 在 我 脸 上 挑 菜 ? 老 人 说 : 这 个 胡 说 , 你 在 那 里 住 , 我 在 那 里 住 , 我 来 你 面 前 抓 柴 、 挑 菜 ? 悟 空 说 : 我 儿 子 就 胡 说 ?
你 是 认 不 到 我 了 , 我 本 来 就 是 这 两 界 山 石 匣 中 的 大 圣 , 你 再 认 认 看 看 。
老 人 方 才 醒 悟 道 : 你 倒 有 些 像 他 。
但 是 你 是 怎 么 能 出 来 的 ? 悟 空 拿 着 菩 萨 劝 善 , 让 他 等 待 唐 僧 揭 帖 脱 身 的 事 , 对 那 位 老 人 详 细 地 说 了 一 遍 。
老 人 刚 下 拜 , 把 唐 僧 请 到 里 面 , 立 即 叫 来 老 婆 和 儿 女 都 来 相 见 , 详 细 说 了 以 前 的 事 。
又 命 他 看 茶 。
茶 罢 , 问 悟 空 说 : 大 圣 呵 , 你 也 有 年 纪 了 , 悟 空 说 : 你 今 年 多 了 , 老 人 说 : 我 痴 子 长 一 百 三 十 岁 了 。
行 人 说 : 还 是 我 重 子 重 孙 啊 !
我 活 着 的 年 龄 , 我 不 知 道 是 什 么 时 候 , 只 在 山 脚 下 , 已 经 五 百 多 年 了 。
老 人 说 : 是 有 , 是 有 。
我 曾 经 记 得 祖 公 , 公 说 : 此 山 是 从 天 上 降 下 来 的 , 就 压 下 了 一 个 神 猴 。
只 到 今 天 , 你 才 脱 身 。
我 小 时 见 你 的 时 候 , 你 的 头 上 有 草 , 脸 上 有 泥 , 还 不 怕 你 ; 如 今 我 的 脸 上 没 了 泥 , 头 上 没 了 草 , 却 好 像 是 瘦 了 一 点 , 腰 间 又 盖 了 一 块 大 虎 皮 , 与 鬼 怪 能 有 多 少 吗 ?
这 个 老 子 很 贤 能 , 就 让 他 安 排 斋 饭 。
饭 后 , 悟 空 说 : 你 家 姓 什 么 ? 老 人 说 : 舍 下 姓 陈 。
三 藏 菩 萨 听 了 这 话 , 立 刻 下 来 , 伸 手 说 : 老 施 主 与 贫 僧 是 华 宗 。
行 者 说 : 师 父 , 你 是 唐 姓 , 怎 么 和 他 是 华 宗 ? 三 藏 说 : 我 是 俗 家 也 姓 陈 , 是 唐 朝 海 州 弘 农 郡 聚 贤 庄 人 。
吾 闻 之 。
只 因 为 我 大 唐 太 宗 皇 帝 赐 我 作 御 弟 三 藏 , 以 唐 为 姓 , 所 以 叫 唐 僧 。
那 个 老 人 见 他 们 说 是 同 姓 的 人 , 又 十 分 欢 喜 。
行 人 说 : 老 陈 , 左 右 打 搅 你 家 , 我 有 五 百 多 年 没 洗 澡 了 , 你 可 去 烧 些 汤 来 , 与 我 师 徒 们 洗 浴 洗 澡 , 一 发 临 行 前 谢 你 们 。
那 个 老 人 立 刻 叫 他 烧 汤 拿 盆 , 手 掌 上 点 灯 火 。
师 徒 洗 浴 完 毕 , 坐 在 灯 前 。
行 人 说 : 老 陈 , 还 有 一 件 事 拖 累 你 , 有 针 线 借 我 用 。
那 个 老 人 说 : 有 , 有 , 有 。
于 是 就 教 奶 妈 拿 来 针 线 , 递 给 行 人 。
行 者 又 有 眼 色 , 见 师 父 洗 浴 , 脱 下 一 件 白 布 , 短 小 直 未 穿 , 他 就 扯 过 来 , 披 在 身 上 。
又 把 那 虎 皮 脱 下 来 , 连 接 到 一 处 。
他 打 了 一 个 马 面 的 折 子 , 包 在 腰 间 , 勒 上 藤 条 , 走 到 师 父 面 前 说 : 老 孙 今 天 这 样 的 打 扮 , 比 昨 天 怎 么 样 ? 三 藏 说 : 好 , 好 , 好 , 好 , 好 , 好 。
这 样 的 样 子 , 才 能 像 个 行 人 。
三 藏 说 : 徒 弟 , 你 不 嫌 残 旧 , 那 件 直 儿 , 你 就 穿 了 吧 。
悟 空 高 声 喊 道 : 承 赐 , 承 赐 。
他 又 去 寻 找 草 料 喂 马 。
这 时 各 自 做 完 事 , 师 徒 和 那 老 子 也 各 自 回 去 睡 觉 。
第 二 天 早 晨 , 悟 空 起 来 , 请 师 父 走 路 。
三 藏 穿 上 衣 服 , 教 行 人 收 拾 铺 盖 行 李 。
正 想 告 别 , 只 见 那 个 老 人 早 备 了 脸 汤 , 又 准 备 了 斋 饭 。
斋 戒 完 毕 , 方 才 起 身 。
三 藏 上 马 , 走 路 的 人 引 路 。
不 知 不 觉 饿 了 吃 , 渴 了 喝 , 夜 晚 住 宿 在 天 亮 的 时 候 。
又 遇 上 初 冬 的 时 候 , 只 见 到 那 里 的 树 叶 , 霜 凋 红 叶 千 林 瘦 , 岭 上 有 几 棵 松 柏 秀 丽 的 树 木 。
还 没 有 开 花 时 , 梅 花 的 花 蕊 散 发 出 香 气 , 天 气 温 暖 , 白 天 暖 暖 , 小 春 候 , 菊 花 残 留 的 荷 叶 尽 , 山 茶 茂 盛 , 寒 桥 古 树 争 枝 争 斗 。
曲 涧 涓 涓 泉 水 流 , 淡 云 欲 雪 满 天 浮 。
北 风 刮 起 衣 袖 , 到 晚 上 寒 冷 的 时 候 , 威 严 的 人 怎 么 接 受 ? 师 徒 们 正 走 了 很 多 时 间 , 忽 然 看 见 路 旁 有 哨 一 声 , 闯 出 六 个 人 来 , 各 持 长 枪 短 剑 , 利 刃 强 弓 , 大 声 说 道 : 那 和 尚 到 哪 里 走 , 快 早 留 下 马 匹 , 放 下 行 李 , 饶 你 的 性 命 回 去 。
说 完 这 三 藏 魂 飞 魄 散 , 跌 下 马 来 , 不 能 说 话 。
行 者 用 手 扶 着 他 说 : 师 父 放 心 , 没 有 什 么 事 , 这 都 是 送 衣 服 送 盘 缠 给 我 们 的 。
三 藏 菩 萨 说 : 悟 空 , 你 想 有 些 耳 朵 闭 着 。
其 人 曰 : 我 要 什 么 衣 服 、 盘 缠 。
行 人 说 : 你 管 你 守 着 衣 服 、 行 李 、 马 匹 , 等 我 与 他 争 持 一 场 , 看 看 你 们 是 怎 么 样 的 ?
三 藏 说 : 好 手 不 如 双 拳 , 双 拳 不 如 四 手 。
他 那 里 六 条 大 汉 , 你 们 这 样 小 小 的 一 个 人 , 怎 么 敢 与 他 争 持 ? 行 人 的 胆 量 原 来 大 , 哪 能 分 别 说 ? 走 上 前 来 , 叉 手 当 胸 , 对 那 六 个 人 施 礼 说 : 列 位 有 什 么 缘 故 , 阻 我 贫 僧 的 去 路 ? 那 人 说 : 我 们 是 剪 径 的 大 王 , 行 走 好 心 的 山 主 。
大 名 久 已 传 播 , 你 估 量 不 知 道 。
早 早 留 下 东 西 , 放 你 回 去 ; 如 果 说 半 个 不 字 , 让 你 碎 骨 碎 骨 。
行 人 说 : 我 也 是 祖 传 的 大 王 , 多 年 的 山 主 , 却 没 有 听 说 你 在 位 上 有 很 大 的 名 声 。
那 人 说 : 你 是 不 知 道 , 我 说 给 你 听 , 一 个 叫 做 眼 睛 看 喜 , 一 个 叫 做 耳 朵 听 怒 , 一 个 叫 做 鼻 子 嗅 爱 , 一 个 叫 做 舌 尝 思 , 一 个 叫 做 意 见 欲 望 , 一 个 叫 做 身 本 忧 患 。
悟 空 笑 着 说 : 原 来 是 六 个 毛 贼 。
你 却 不 认 为 我 这 个 出 家 人 是 你 的 主 人 公 , 你 倒 来 阻 挡 道 路 。
把 打 劫 的 珍 宝 拿 出 来 , 我 与 你 作 七 分 儿 均 分 , 饶 你 罢 了 。
那 贼 听 了 , 喜 的 喜 , 怒 的 怒 , 爱 的 爱 , 思 的 思 , 欲 的 欲 , 忧 的 忧 , 一 齐 上 前 乱 吵 道 : 这 和 尚 无 礼 。
你 的 东 西 全 都 没 有 , 转 来 和 我 等 要 分 东 西 。
他 挥 刀 舞 剑 , 一 拥 而 来 , 照 着 行 人 砍 头 乱 砍 , 擂 得 滚 滚 , 砍 了 七 八 十 下 。
悟 空 停 留 在 中 间 , 只 是 不 知 道 。
那 个 盗 贼 说 : 好 和 尚 , 真 个 的 头 硬 。
行 人 笑 着 说 : 将 就 看 得 过 罢 了 。
你 们 也 打 得 手 困 了 , 还 是 老 孙 取 出 个 针 儿 来 玩 耍 。
那 个 盗 贼 说 : 这 和 尚 是 一 个 行 针 灸 的 郎 中 变 的 。
我 们 又 没 有 病 症 , 说 什 么 动 针 的 话 , 走 路 的 人 伸 手 从 耳 朵 里 拔 出 一 根 绣 花 针 , 迎 风 一 搭 , 却 是 一 条 铁 棒 , 脚 上 有 碗 来 的 粗 细 。
他 拿 在 手 中 说 : 不 要 走 , 也 让 老 孙 子 打 一 个 棍 子 试 试 。
得 了 六 个 贼 人 , 四 处 逃 走 。
因 为 他 被 拽 开 步 , 一 个 人 赶 上 去 , 一 个 个 都 被 打 死 了 。
剥 掉 他 的 衣 服 , 夺 去 他 的 盘 子 , 笑 着 走 来 说 : 师 父 请 我 去 , 那 贼 已 被 老 孙 剿 灭 了 。
三 藏 说 : 你 十 分 之 一 撞 祸 。
他 虽 然 是 剪 除 小 路 的 强 徒 , 即 使 是 拿 到 官 府 , 也 不 应 该 死 罪 ; 你 纵 然 有 手 段 , 只 可 让 他 去 就 了 , 何 必 就 都 打 死 , 这 又 是 无 缘 无 故 伤 害 人 的 性 命 , 怎 么 能 做 到 和 尚 呢 ? 出 家 人 扫 地 恐 怕 伤 害 蚁 的 性 命 , 爱 惜 飞 蛾 纱 罩 灯 。
你 何 不 分 黑 白 , 一 顿 打 死 , 完 全 没 有 一 点 慈 悲 好 善 之 心 。
如 果 到 了 城 市 , 倘 若 有 人 一 时 冲 撞 了 你 , 你 也 行 凶 , 拿 着 棍 子 乱 打 伤 人 , 我 可 以 做 白 客 , 怎 么 能 脱 身 ? 悟 空 说 : 师 父 , 我 如 果 不 打 死 他 , 他 却 要 打 死 你 吗 ?
三 藏 说 : 我 这 个 出 家 人 宁 愿 死 , 决 不 敢 行 凶 。
我 就 死 了 , 也 只 是 一 个 人 , 你 却 杀 了 六 个 人 , 又 有 什 么 道 理 说 呢 ? 这 事 如 果 告 到 官 府 , 就 是 你 老 子 做 官 , 也 说 不 过 去 。
行 人 说 : 不 骗 师 父 说 , 我 老 孙 在 五 百 年 前 , 占 据 花 果 山 称 王 为 怪 的 时 候 , 也 不 知 道 打 死 了 多 少 人 。
假 如 你 说 这 样 到 官 , 倒 也 得 到 一 些 状 告 。
三 藏 说 : 只 因 你 没 收 没 管 , 横 行 人 间 , 欺 骗 上 天 , 才 遭 受 这 五 百 年 前 的 灾 难 。
现 在 既 然 已 经 成 了 和 尚 , 如 果 还 像 当 时 行 凶 , 一 味 伤 害 生 灵 , 去 不 得 西 天 , 做 不 到 和 尚 。
太 恶 啊 , 太 恶 , 原 来 这 个 猴 子 一 生 都 得 不 到 人 的 气 味 。
他 看 见 三 藏 菩 萨 只 管 绪 不 停 , 心 头 火 发 , 说 : 你 既 然 是 这 样 的 人 , 说 我 做 不 到 和 尚 , 上 不 得 西 天 , 不 必 像 这 样 的 人 一 样 纷 纷 恶 我 , 我 回 去 就 可 以 了 。
三 藏 却 没 有 回 答 。
使 者 一 个 性 子 , 把 自 己 的 身 子 放 了 , 说 一 声 道 : 老 孙 去 了 。 三 藏 急 忙 抬 头 , 早 已 不 见 了 。
只 听 到 一 声 呼 叫 的 声 音 , 回 东 而 去 。
把 那 长 老 孤 独 零 零 , 点 头 自 叹 , 悲 怨 不 已 , 说 : 这 个 人 这 样 不 接 受 教 诲 , 我 只 说 几 句 , 他 为 什 么 就 无 形 无 影 的 径 直 回 去 了 , 罢 罢 , 也 是 我 命 里 不 该 招 徒 弟 , 进 人 口 。
如 今 想 要 寻 找 他 没 有 地 方 去 寻 找 , 想 要 叫 他 也 不 应 该 , 去 来 去 来 , 去 来 去 来 。
这 正 是 说 : 舍 身 拼 命 回 去 , 不 要 依 靠 旁 人 自 己 主 张 。
其 长 老 只 得 收 拾 行 李 , 送 到 马 上 , 也 不 骑 马 , 一 只 手 拄 着 锡 杖 , 一 只 手 握 着 绳 , 凄 凄 凉 爽 , 往 西 向 前 走 。
走 了 不 多 时 , 只 见 山 路 前 面 有 个 年 纪 高 的 老 母 , 捧 着 一 件 绵 衣 , 绵 衣 上 有 一 顶 花 帽 。
三 藏 看 见 他 来 到 附 近 , 慌 忙 牵 着 马 , 站 在 右 边 让 他 走 。
那 位 老 母 问 道 : 你 是 从 哪 里 来 的 长 老 , 孤 独 独 自 走 到 这 里 , 三 藏 说 : 弟 子 是 东 土 大 唐 奉 命 到 西 天 去 拜 谒 活 佛 求 真 经 的 。
老 母 说 : 西 方 佛 是 大 雷 音 寺 、 天 竺 国 的 边 界 , 这 里 去 有 十 万 八 千 里 的 路 。
你 这 些 人 单 人 独 马 , 又 没 有 一 个 伴 侣 , 又 没 有 一 个 徒 弟 , 你 怎 么 去 得 到 ? 三 藏 说 : 弟 子 昨 日 前 收 到 一 个 徒 弟 , 他 性 情 泼 泼 凶 顽 , 是 我 说 了 他 几 句 , 他 不 接 受 教 诲 , 于 是 就 渺 然 而 去 。
老 母 说 : 我 有 这 一 领 绵 布 直 、 一 顶 嵌 金 花 帽 , 原 来 是 我 儿 子 使 用 的 , 他 只 做 了 三 天 和 尚 , 不 幸 命 短 身 亡 。
吾 闻 之 。
长 老 呵 , 你 既 然 有 弟 弟 , 我 把 这 件 衣 帽 送 给 你 吧 。
三 藏 说 : 承 蒙 老 母 的 厚 赐 , 只 是 我 的 弟 弟 已 经 走 了 , 我 不 敢 接 受 。
老 母 说 : 他 那 厢 去 了 , 三 藏 说 : 我 听 到 呼 叫 的 一 声 , 他 回 东 去 了 。
老 母 说 : 东 边 不 远 , 就 是 我 家 , 想 必 到 我 家 去 了 。
我 的 那 里 还 有 一 篇 咒 语 , 叫 做 定 心 真 言 , 又 叫 做 紧 紧 儿 咒 。
君 子 之 所 以 为 之 , 不 可 以 为 之 , 不 可 以 为 之 , 不 可 以 为 之 , 不 可 以 为 之 。
我 去 赶 上 他 , 叫 他 还 来 跟 你 , 你 却 拿 这 个 衣 帽 给 他 穿 戴 。
他 若 不 服 从 你 的 使 唤 , 你 就 默 念 此 咒 , 他 再 不 敢 行 凶 , 也 再 不 敢 去 。
三 藏 听 了 这 话 , 低 头 拜 谢 。
那 老 母 变 成 一 道 金 光 , 回 向 东 方 而 去 。
三 藏 菩 萨 知 道 是 观 音 菩 萨 传 授 的 真 言 , 急 忙 撮 土 焚 香 , 望 东 诚 恳 地 礼 拜 。
拜 谢 完 毕 , 收 拾 了 衣 帽 , 藏 在 包 袱 中 间 。
又 坐 在 路 旁 , 诵 读 念 诵 那 定 心 真 言 , 来 回 念 了 几 遍 , 念 得 烂 熟 , 牢 记 心 胸 , 不 再 问 题 。
又 说 : 那 悟 空 告 别 了 师 父 , 一 斗 云 , 径 直 转 到 东 洋 大 海 。
按 住 云 头 , 分 开 水 道 , 径 直 到 水 晶 宫 前 。
早 早 惊 动 龙 王 出 来 迎 接 , 接 到 宫 中 坐 下 。
礼 仪 结 束 后 , 龙 王 说 : 近 来 听 说 得 到 大 圣 难 满 , 失 喜 , 想 必 是 重 新 整 顿 仙 山 , 再 回 到 古 洞 了 。 悟 空 说 : 我 也 有 这 种 心 性 , 只 是 又 做 了 和 尚 了 。
龙 王 说 : 你 做 什 么 和 尚 ? 行 者 说 : 我 亏 了 南 海 菩 萨 劝 善 , 教 我 正 果 , 跟 随 东 土 唐 僧 到 西 方 拜 佛 , 皈 依 和 尚 , 又 叫 他 为 行 者 了 。
龙 王 说 : 这 些 人 真 是 可 庆 贺 , 可 庆 贺 。
其 所 以 为 善 之 心 , 以 为 善 之 心 , 以 为 善 之 心 。
既 然 如 此 , 为 什 么 不 向 西 去 , 又 往 东 回 去 呢 ? 行 人 笑 着 说 : 这 是 唐 僧 不 认 识 人 性 。
有 几 个 毛 贼 剪 除 道 路 , 是 我 把 他 打 死 , 唐 僧 就 绪 纷 , 说 了 我 若 干 的 不 是 。
你 想 老 孙 子 可 是 受 得 闷 气 的 , 是 我 抛 弃 他 , 想 回 到 本 山 去 , 所 以 先 来 望 你 一 望 , 请 你 一 点 茶 吃 。
龙 王 说 : 承 投 降 , 承 投 降 。
当 时 龙 子 、 龙 孙 就 捧 着 香 茶 来 进 献 。
茶 水 喝 完 , 走 路 的 人 回 头 一 看 , 见 后 墙 上 挂 着 一 幅 桥 进 鞋 子 。
行 人 说 : 这 是 什 么 景 致 ? 龙 王 说 : 大 圣 在 先 , 此 事 在 后 , 所 以 你 不 认 得 。
夫 子 曰 : 桥 三 进 履 。
龙 王 说 : 这 个 仙 人 就 是 黄 石 公 , 这 个 孩 子 就 是 汉 代 的 张 良 。 石 公 坐 在 桥 上 , 忽 然 丢 失 鞋 子 在 桥 下 , 于 是 叫 张 良 拿 来 。
这 个 孩 子 立 即 拿 来 , 跪 在 面 前 献 上 。
如 此 三 度 , 张 良 一 点 也 没 有 一 丝 一 毫 骄 傲 傲 慢 的 念 头 。 石 公 于 是 喜 爱 他 勤 勉 谨 慎 , 夜 间 传 授 天 书 , 写 着 他 扶 汉 。
后 来 果 然 在 帷 幄 之 中 运 筹 , 决 胜 于 千 里 之 外 。
太 平 以 后 , 他 离 开 官 职 回 到 山 中 , 跟 着 赤 松 子 游 览 , 觉 得 成 仙 之 道 。
大 圣 , 你 如 果 不 保 证 唐 僧 , 不 竭 尽 勤 劳 , 不 接 受 教 诲 , 到 底 是 个 妖 仙 , 也 不 想 得 到 成 就 正 果 。
悟 空 听 了 他 的 话 , 沉 吟 半 晌 , 不 说 话 。
龙 王 说 : 大 圣 自 然 应 当 裁 决 , 不 可 以 图 谋 自 在 , 误 了 前 途 。
悟 空 说 : 不 要 多 说 , 老 子 还 去 保 他 就 是 了 。
龙 王 欣 喜 地 说 : 既 然 这 样 , 不 敢 久 留 , 请 大 圣 早 发 慈 悲 , 不 要 太 久 了 你 师 父 。
行 人 见 他 催 促 请 求 出 发 , 急 忙 放 下 身 子 , 离 开 海 藏 , 驾 着 云 , 告 别 了 龙 王 。
正 走 , 又 遇 到 南 海 菩 萨 。
菩 萨 说 : 孙 悟 空 , 你 怎 么 不 受 教 诲 , 不 保 唐 僧 , 来 此 地 干 什 么 ? 忽 然 有 个 行 人 在 云 端 里 施 礼 说 : 刚 才 承 蒙 菩 萨 善 言 , 果 然 有 唐 朝 的 僧 人 到 了 , 揭 了 贴 帖 , 救 了 我 的 生 命 , 跟 他 做 了 徒 弟 。
其 他 人 又 怪 我 凶 顽 , 我 才 闪 他 一 闪 , 现 在 就 去 保 护 他 。
菩 萨 说 : 快 点 去 吧 , 不 要 错 过 了 念 头 。
说 完 就 各 自 回 来 了 。
这 个 行 人 不 一 会 儿 , 看 见 唐 僧 在 路 旁 闷 闷 地 坐 着 。
三 藏 抬 头 说 : 你 到 哪 里 去 , 教 我 走 又 不 敢 走 , 动 又 不 敢 动 , 只 管 你 在 这 里 等 你 。
行 人 说 : 我 去 东 洋 大 海 老 龙 王 家 讨 茶 吃 吃 。
三 藏 说 : 徒 弟 呵 , 出 家 人 不 要 说 谎 。
你 离 开 我 不 多 一 个 时 辰 , 就 说 到 龙 王 家 吃 茶 , 走 的 人 笑 着 说 : 不 骗 师 父 说 , 我 会 驾 着 斗 云 , 一 个 斗 有 十 万 八 千 里 路 , 所 以 我 能 立 即 去 就 来 。
三 藏 说 : 我 略 略 的 话 语 多 了 些 , 你 就 责 怪 我 , 让 我 的 性 子 丢 掉 我 去 。
像 你 这 样 有 本 事 的 , 讨 得 茶 吃 ; 像 我 这 样 去 得 不 到 的 , 只 管 在 这 里 忍 耐 挨 饿 。
卿 亦 不 知 其 意 , 不 知 其 意 。
行 者 说 : 师 父 , 你 如 果 饿 了 , 我 就 去 给 你 做 斋 饭 吃 。
三 藏 说 : 不 用 化 斋 , 我 的 包 袱 里 还 有 些 干 粮 , 是 刘 太 保 母 亲 送 的 。
你 去 拿 钵 盂 去 找 些 水 来 , 等 我 吃 些 儿 子 走 路 罢 了 。
行 者 去 解 开 包 袱 , 在 包 裹 中 间 看 见 有 几 个 粗 米 和 烧 饼 , 拿 出 来 送 给 师 父 。
又 见 那 光 艳 的 一 领 绵 布 直 , 一 顶 镶 嵌 金 花 帽 。
行 者 说 : 这 衣 帽 是 东 土 人 带 来 的 , 三 藏 便 顺 口 答 道 : 是 我 小 时 候 戴 的 。
帽 子 如 果 戴 了 , 不 用 教 经 , 就 会 念 经 ; 衣 服 如 果 穿 了 , 不 用 演 礼 , 就 会 行 礼 。
行 人 说 : 好 师 父 , 把 我 给 我 穿 戴 了 。
三 藏 说 : 只 怕 长 短 不 一 , 你 如 果 穿 得 , 就 穿 了 。
行 人 于 是 脱 下 旧 的 白 布 直 , 把 绵 布 直 穿 上 去 , 这 就 是 比 量 着 身 体 裁 裁 的 样 子 。
以 此 言 之 。
三 藏 之 人 , 皆 不 知 之 。
走 路 的 人 喊 道 : 头 痛 , 头 痛 。
师 父 不 住 , 又 念 了 几 遍 , 把 行 者 痛 得 打 滚 , 把 钉 在 金 的 花 帽 上 。
三 藏 又 怕 扯 断 金 箍 , 住 在 嘴 里 不 念 。
不 念 及 时 , 则 不 觉 痛 。
伸 手 到 头 上 去 摸 摸 , 好 像 一 条 金 线 儿 的 样 子 , 紧 紧 地 勒 在 上 面 , 取 不 下 , 抽 不 断 , 已 经 这 里 生 了 根 了 。
他 就 从 耳 里 取 出 针 子 来 , 把 它 插 入 锁 子 里 , 往 外 乱 捎 。
三 藏 又 恐 怕 他 捎 断 了 , 口 里 又 念 起 来 。
其 他 人 也 。
那 师 父 见 到 他 这 样 的 人 , 又 不 忍 心 不 舍 , 又 住 了 口 。
他 的 头 , 又 不 痛 了 。
行 者 说 : 我 的 这 头 , 原 来 是 师 父 咒 我 的 , 三 藏 说 : 我 念 得 是 紧 紧 经 , 何 曾 诅 咒 你 呢 ? 行 者 说 : 你 再 念 念 看 。
三 藏 真 个 又 念 念 。
行 者 真 的 又 痛 心 , 只 教 他 说 : 不 要 念 , 不 要 念 。
念 念 一 动 , 我 就 痛 了 。
三 藏 说 : 你 今 天 可 以 听 我 的 教 诲 了 。 行 者 说 : 听 我 的 教 诲 了 。
那 人 说 : 你 再 可 以 无 礼 了 。 行 人 说 : 不 敢 了 。
他 口 里 虽 然 答 应 , 但 心 里 还 有 不 善 之 心 。
忽 然 长 老 口 中 又 念 了 两 三 遍 。
这 个 猴 子 跌 倒 在 地 上 , 丢 了 铁 棒 , 不 能 举 手 , 只 教 他 说 : 师 父 , 我 知 道 了 。
再 不 要 念 念 , 再 不 要 念 念 。
三 藏 说 : 你 怎 么 欺 心 , 竟 敢 打 我 ? 行 者 说 : 我 不 曾 敢 打 我 。
我 问 师 父 , 你 这 个 法 子 是 谁 教 你 的 ? 三 藏 说 : 是 刚 才 有 一 位 老 母 传 授 我 的 。
行 者 大 怒 道 : 不 用 讲 了 , 这 个 老 母 , 坐 定 是 那 个 观 世 音 。
他 何 以 害 我 ? 等 我 上 南 海 打 他 去 。
三 藏 说 : 此 法 既 然 是 他 授 给 我 , 他 必 然 先 知 道 了 。
你 如 果 去 寻 找 他 , 他 念 起 来 , 你 却 不 是 死 了 。 行 者 见 说 得 有 理 , 真 的 不 敢 动 身 , 只 得 回 心 , 跪 下 哀 告 说 : 师 父 , 这 是 他 怎 么 我 的 法 儿 , 教 我 随 你 西 去 。
我 也 不 去 惹 他 , 你 也 不 要 常 说 , 只 管 念 诵 。
我 愿 保 住 你 , 再 也 没 有 后 悔 的 意 思 了 。
三 藏 说 : 既 然 这 样 , 伏 等 我 上 马 去 吧 。
那 个 行 人 才 死 心 塌 地 , 精 神 抖 擞 , 捆 好 一 束 绵 布 直 接 撞 开 , 敲 着 背 上 的 马 匹 , 收 拾 行 李 , 奔 向 西 边 走 。
毕 竟 这 一 去 , 后 面 又 有 什 么 话 说 , 暂 且 听 下 回 的 分 析 解 释 。
}\switchcolumn\flushpage  \begin{pinyinscope}{\myfontt \section{第一五回}     蛇盤山諸神暗佑 鷹愁澗意馬收韁

卻說行者伏侍唐僧西進,行經數日,正是那臘月寒天,朔風凜凜,滑凍凌凌。走
的是些懸崖峭壁崎嶇路,疊嶺層巒險峻山。三藏在馬上,遙聞?喇喇水聲聒耳,
回頭叫:「悟空,是那裏水響?」行者道:「我記得此處叫做蛇盤山鷹愁澗,想
必是澗裏水響。」說不了,馬到澗邊,三藏勒韁觀看。但見:
    涓涓寒脈穿雲過,湛湛清波映日紅。
    聲搖夜雨聞幽谷,彩發朝霞眩太空。
    千仞浪飛噴碎玉,一泓水響吼清風。
    流歸萬頃煙波去,鷗鷺相忘沒釣逢。
  
師徒兩個正然看處,只見那澗當中響一聲,鑽出一條龍來,推波掀浪,攛出崖山
,就搶長老。慌得個行者丟了行李,把師父抱下馬來,回頭便走。那條龍就趕不
上,把他的白馬連鞍轡一口吞下肚去,依然伏水潛蹤。行者把師父送在那高阜上
坐了,卻來牽馬挑擔,止存得一擔行李,不見了馬匹。他將行李擔送到師父面前
道:「師父,那孽龍也不見蹤影,只是驚走我的馬了。」三藏道:「徒弟呵,卻
怎生尋得馬著麼?」行者道:「放心,放心,等我去看來。」
  
他打個?哨,跳在空中,火眼金睛,用手搭涼篷,四下裏觀看,更不見馬的蹤跡
。按落雲頭,報道:「師父,我們的馬斷乎O那龍吃了,四下裏再看不見。」三
藏道:「徒弟呀,那廝能有多大口,卻將那匹大馬連鞍轡都吃了?想是驚張溜韁
,走在那山凹之中。你再仔細看看。」行者道:「你也不知我的本事。我這雙眼
,白日裏常看一千里路的吉凶。像那千里之內,蜻蜓兒展翅,我也看見,何期那
匹大馬,我就不見?」三藏道:「既是他吃了,我如何前進?可憐呵,這千山萬
水,怎生走得?」說著話,淚如雨落。行者見他哭將起來,他那裏忍得住暴燥,
發聲喊道:「師父莫要這等膿包形麼,你坐著,坐著,等老孫去尋著那廝,教他
還我馬匹便了。」三藏卻才扯住道:「徒弟呵,你那裏去尋他?只怕他暗地裏攛
將出來,卻不又連我都害了?那時節人馬兩亡,怎生是好?」行者聞得這話,越
加嗔怒,就叫喊如雷道:「你忒不濟,不濟!又要馬騎,又不放我去,似這般看
著行李,坐到老罷。」
  
哏哏的吆喝,正難息怒,只聽得空中有人言語,叫道:「孫大聖莫惱,唐御弟休
哭。我等是觀音菩薩差來的一路神祗,特來暗中保取經者。」那長老聞言,慌忙
禮拜。行者道:「你等是那幾個,可報名來,我好點卯。」眾神道:「我等是六
丁六甲、五方揭諦、四值功曹、一十八位護教伽藍,各各輪流值日聽候。」行者
道:「今日先從誰起?」眾揭諦道:「丁甲、功曹、伽藍輪次。我五方揭諦,惟
金頭揭諦晝夜不離左右。」行者道:「既如此,不當值者且退,留下六丁神將與
日值功曹和眾揭諦保守著我師父。等老孫尋那澗中的孽龍,教他還我馬來。」眾
神遵令。三藏才放下心,坐在石崖之上,吩咐行者仔細。行者道:「只管寬心。」
好猴王,束一束綿布直裰,撩起虎皮裙子,揝著金箍鐵棒,抖擻精神,徑臨澗壑
,半雲半霧的,在那水面上高叫道:「潑泥鰍,還我馬來!還我馬來!」
  
卻說那龍吃了三藏的白馬,伏在那澗底中間,潛靈養性,只聽得有人叫罵索馬。
他按不住心中火發,急縱身躍浪翻波,跳將上來道:「是那個敢在這裏海口傷吾
?」行者見了他,大?一聲「休走,還我馬來!」掄著棍,劈頭就打。那條龍張
牙舞爪來抓。他兩個在澗邊前這一場賭鬥,果是驍雄。但見那:
龍舒利爪,猴舉金箍。那個鬚垂白玉線,這個眼幌赤金燈。那個鬚下明珠噴彩霧
,這個手中鐵棒舞狂風。那個是迷爺娘的業子,這個是欺天將的妖精。他兩個都
因有難遭磨折,今要成功各顯能。
  
來來往往,戰勾多時,盤旋良久,那條龍力軟筋麻,不能抵敵,打一個轉身,又
攛於水內,深潛澗底,再不出頭。被猴王罵詈不絕,他也只推耳聾。
  
行者沒及奈何,只得回見三藏道:「師父,這個怪被老孫罵將出來,他與我賭鬥
多時,怯戰而走,只躲在水中間,再不出來了。」三藏道:「不知端的可是他吃
了我馬?」行者道:「你看你說的話,不是他吃了,他還肯出來招聲,與老孫犯
對?」三藏道:「你前日打虎時,曾說有降龍伏虎的手段,今日如何便不能降他
?」原來那猴子吃不得人急他。見三藏搶白了他這一句,他就發起神威道:「不
要說,不要說,等我與他再見個上下。」
  
這猴王拽開步,跳到澗邊,使出那翻江攪海的神通,把一條鷹愁陡澗徹底澄清的
水,攪得似那九曲黃河泛漲的波。那孽龍在於深澗中坐臥不寧,心中思想道:
「這才是福無雙降,禍不單行。我才脫了天條死難,不上一年,在此隨緣度日,
又撞著這般個潑魔,他來害我。」你看他越思越惱,受不得屈氣,咬著牙,跳將
出去,罵道:「你是那裏來的潑魔,這等欺我?」行者道:「你莫管我那裏不那
裏,你只還了馬,我就饒你性命。」那龍道:「你的馬是我吞下肚去,如何吐得
出來?不還你,便待怎的?」行者道「不還馬時看棍,只打殺你,償了我馬的性
命便罷。」他兩個又在那山崖下苦鬥。鬥不數合,小龍委實難搪,將身一幌,變
作一條水蛇兒,鑽入草科中去了。
  
猴王拿著棍,趕上前來,撥草尋蛇,那裏得些影響。急得他三尸神咋,七竅煙生
,念了一聲「唵」字咒語,即喚出當坊土地、本處山神,一齊來跪下道:「山神
、土地來見。」行者道:「伸過孤拐來,各打五棍見面,與老孫散散心。」二神
叩頭哀告道:「望大聖方便,容小神訴告。」行者道:「你說甚麼?」二神道:
「大聖一向久困,小神不知幾時出來,所以不曾接得,萬望恕罪。」行者道:
「既如此,我且不打你。我問你:鷹愁澗裏,是那方來的怪龍?他怎麼搶了我師
父的白馬吃了?」二神道:「大聖自來不曾有師父,原來是個不伏天不伏地混元
上真,如何得有甚麼師父的馬來?」行者道:「你等是也不知。我只為那誑上的
勾當,整受了這五百年的苦難。今蒙觀音菩薩勸善,著唐朝駕下真僧救出我來,
教我跟他做徒弟,往西天去拜佛求經。因路過此處,失了我師父的白馬。」二神
道:「原來是如此。這澗中自來無邪,只是深陡寬闊,水光徹底澄清,鴉鵲不敢
飛過﹔因水清照見自己的形影,便認做同群之鳥,往往身擲於水內:故名『鷹愁
陡澗』。只是向年間,觀音菩薩因為尋訪取經人去,救了一條玉龍,送他在此,
教他等候那取經人,不許為非作歹。他只是饑了時,上岸來撲些鳥鵲吃,或是捉
些獐鹿食用。不知他怎麼無知,今日沖撞了大聖。」行者道:「先一次,他還與
老孫侮手,盤旋了幾合﹔後一次,是老孫叫罵,他再不出。因此使了一個翻江攪
海的法兒,攪混了他澗水,他就攛將上來,還要爭持。不知老孫的棍重,他遮架
不住,就變做一條水蛇,鑽在草裏。我趕來尋他,卻無蹤跡。」土地道:「大聖
不知。這條澗千萬個孔竅相通,故此這波瀾深遠。想是此間也有一孔,他鑽將下
去。也不須大聖發怒,在此找尋﹔要擒此物,只消請將觀世音來,自然伏了。」
  
行者見說,喚山神、土地,同來見了三藏,具言前事。三藏道:「若要去請菩薩
,幾時才得回來?我貧僧饑寒怎忍?」說不了,只聽得暗空中有金頭揭諦叫道:
「大聖,你不須動身,小神去請菩薩來也。」行者大喜,道聲:「有累,有累。
快行,快行。」那揭諦急縱雲頭,徑上南海。行者吩咐山神、土地守護師父,日
值功曹去尋齋供,他又去澗邊巡遶不題。
  
卻說金頭揭諦一駕雲,早到了南海。按祥光,直至落伽山紫竹林中,託那金甲諸
天與木叉惠岸轉達,得見菩薩。菩薩道:「汝來何幹?」揭諦道:「唐僧在蛇盤
山鷹愁陡澗失了馬,急得孫大聖進退兩難。及問本處土神,說是菩薩送在澗裏的
孽龍吞了。那大聖著小神來告請菩薩降這孽龍,還他馬匹。」菩薩聞言道:「這
廝本是西海敖閏之子,他為縱火燒了殿上明珠,他父告他忤逆,天庭上犯了死罪
。是我親見玉帝,討他下來,教他與唐僧做個腳力。他怎麼反吃了唐僧的馬?這
等說,等我去來。」那菩薩降蓮臺,徑離仙洞,與揭諦駕著祥光,過了南海而來
。有詩為證。詩曰:
    佛說蜜多三藏經,菩薩揚善滿長城。
    摩訶妙語通天地,般若真言救鬼靈。
    致使金蟬重脫殼,故令玄奘再修行。
    只因路阻鷹愁澗,龍子歸真化馬形。

  
那菩薩與揭諦不多時到了蛇盤山,卻在那半空裏留住祥雲,低頭觀看,只見孫行
者正在澗邊叫罵。菩薩著揭諦喚他來。那揭諦按落雲頭,不經由三藏,直至澗邊
,對行者道:「菩薩來也。」行者聞得,急縱雲跳到空中,對他大叫道:「你這
個七佛之師,慈悲的教主,你怎麼生方法兒害我?」菩薩道:「我把你這個大膽
的馬流,村愚的赤尻。我倒再三盡意,度得個取經人來,叮嚀教他救你性命,你
怎麼不來謝我活命之恩,反來與我嚷鬧?」行者道:「你弄得我好哩。你既放我
出來,讓我逍遙自在耍子便了。你前日在海上迎著我,傷了我幾句,教我來盡心
竭力,伏侍唐僧便罷了,你怎麼送他一頂花帽,哄我戴在頭上受苦?把這個箍子
長在老孫頭上,又教他念一卷甚麼『緊箍兒咒』,著那老和尚念了又念,教我這
頭上疼了又疼,這不是你害我也?」菩薩笑道:「你這猴子,你不遵教令,不受
正果,若不如此拘係你,你又誑上欺天,知甚好歹?再似從前撞出禍來,有誰收
管?須是得這個魔頭,你才肯入我瑜伽之門路哩。」行者道:「這樁事,作做是
我的魔頭罷。你怎麼又把那有罪的孽龍,送在此處成精,教他吃了我師父的馬匹
?此又是縱放歹人為惡,太不善也。」菩薩道:「那條龍,是我親奏玉帝,討他
在此,專為求經人做個腳力。你想那東土來的凡馬,怎歷得這萬水千山?怎到得
那靈山佛地?須是得這個龍馬,方才去得。」行者道:「像他這般懼怕老孫,潛
躲不出,如之奈何?」菩薩叫揭諦道:「你去澗邊叫一聲『敖閏龍王玉龍三太子
,你出來,有南海菩薩在此。』他就出來了。」
  
那揭諦果去澗邊叫了兩遍。那小龍翻波跳浪,跳出水來,變作一個人像,踏了雲
頭,到空中對菩薩禮拜道:「向蒙菩薩解脫活命之恩,在此久等,更不聞取經人
的音信。」菩薩指著行者道:「這不是取經人的大徒弟?」小龍見了道:「菩薩
,這是我的對頭。我昨日腹中饑餒,果然吃了他的馬匹。他倚著有些力量,將我
鬥得力怯而回,又罵得我閉門不敢出來。他更不曾提著一個『取經』的字樣。」
行者道:「你又不曾問我姓甚名誰,我怎麼就說?」小龍道:「我不曾問你是那
裏來的潑魔?你嚷道:『管甚麼那裏不那裏,只還我馬來。』何曾說出半個『唐』
字?」菩薩道:「那猴頭專倚自強,那肯稱讚別人?今番前去,還有歸順的哩。
若問時,先提起『取經』的字來,卻也不用勞心,自然拱伏。」
  
行者歡喜領教。菩薩上前,把那小龍的項下明珠摘了,將楊柳枝蘸出甘露,往他
身上拂了一拂,吹口仙氣,喝聲叫:「變!」那龍即變做他原來的馬匹毛片。又
將言語吩咐道:「你須用心還了業障,功成後超越凡龍,還你個金身正果。」那
小龍口啣著橫骨,心心領諾。菩薩教悟空領他去見三藏。「我回海上去也。」行
者扯住菩薩不放道:「我不去了,我不去了。西方路這等崎嶇,保這個凡僧,幾
時得到?似這等多磨多折,老孫的性命也難全,如何成得甚麼功果?我不去了,
我不去了。」菩薩道:「你當年未成人道,且肯盡心修悟﹔你今日脫了天災,怎
麼倒生懶惰?我門中以寂滅成真,須是要信心正果。假若到了那傷身苦磨之處,
我許你叫天天應,叫地地靈﹔十分再到那難脫之際,我也親來救你。你過來,我
再贈你一般本事。」菩薩將楊柳葉兒摘下三個,放在行者的腦後,喝聲:「變!」
即變做三根救命的毫毛。教他:「若到那無濟無主的時節,可以隨機應變,救得
你急苦之災。」行者聞了這許多好言,才謝了大慈大悲的菩薩。那菩薩香風繞繞
,彩霧飄飄,徑轉普陀而去。
  
這行者才按落雲頭,揪著那龍馬的頂鬃,來見三藏道:「師父,馬有了也。」三
藏一見,大喜道:「徒弟,這馬怎麼比前反肥盛了些?在何處尋著的?」行者道
:「師父,你還做夢哩。卻才是金頭揭諦請了菩薩來,把那澗裏龍化作我們的白
馬,其毛片相同,只是少了鞍轡。著老孫揪將來也。」三藏大驚道:「菩薩何在
?待我去拜謝他。」行者道:「菩薩此時已到南海,不耐煩矣。」三藏就撮土焚
香,望南禮拜。拜罷,起身即與行者收拾前進。行者喝退了山神、土地,吩咐了
揭諦、功曹,卻請師父上馬。三藏道:「那無鞍轡的馬,怎生騎得?且待尋船渡
過澗去,再作區處。」行者道:「這個師父好不知時務!這個曠野山中,船從何
來?這匹馬,他在此久住,必知水勢,就騎著他做個船兒過去罷。」
  
三藏無奈,只得依言,跨了產馬。行者挑著行囊。到了澗邊。只見那上流頭,有
一個漁翁,撐著一個枯木的?子,順流而下。行者見了,用手招呼道:「那老漁
,你來,你來。我是東土取經去的,我師父到此難過,你來渡他一渡。」漁翁聞
言,即忙撐攏。行者請師父下了馬,扶持左右。三藏上了?子,揪上馬匹,安了
行李。那老漁撐開?子,如風似箭,不覺的過了鷹愁陡澗,上了西岸。三藏教行
者解開包袱,取出大唐的幾文錢鈔,送與老漁。老漁把?子一篙撐開道:「不要
錢,不要錢。」向中流渺渺茫茫而去。三藏甚不過意,只管合掌稱謝。行者道:
「師父休致意了,你不認得他?他是此澗裏的水神。不曾來接得我老孫,老孫還
要打他哩。只如今免打就勾了他的,怎敢要錢!」那師父也似信不信,只得又跨
著產馬,隨著行者,徑投大路,奔西而去。這正是:廣大真如登彼岸,誠心了性
上靈山。
  
同師前進,不覺的紅日沉西,天光漸晚。但見:
淡雲撩亂,山月昏蒙。滿天霜色生寒,四面風聲透體。孤鳥去時蒼渚闊,落霞明
處遠山低。疏林千樹吼,空嶺獨猿啼。長途不見行人跡,萬里歸舟入夜時。
  
三藏在馬上遙觀,忽見路傍一座莊院。三藏道:「悟空,前面人家,可以借宿,
明早再行。」行者抬頭看見道:「師父,不是人家莊院。」三藏道:「如何不是
?」行者道:「人家莊院,卻沒飛魚穩獸之脊,這斷是個廟宇庵院。」
  
師徒們說著話,早已到了門首。三藏下了馬,只見那門上有三個大字,乃「里社
祠」,遂入門裏。那裏邊有一個老者,項掛著數珠兒,合掌來迎,叫聲:「師父
請坐。」三藏慌忙答禮,上殿去參拜了聖像。那老者即呼童子獻茶。茶罷,三藏
問老者道:「此廟何為『里社』?」老者道:「敝處乃西番哈咇國界。這廟後有
一莊人家,共發虔心,立此廟宇。里者,乃一鄉里地﹔社者,乃一社土神。每遇
春耕、夏耘、秋收、冬藏之日,各辦三牲花果,來此祭社,以保四時清吉、五穀
豐登、六畜茂盛故也。」三藏聞言,點頭誇讚:「正是『離家三里遠,別是一鄉
風』。我那裏人家,更無此善。」老者卻問:「師父仙鄉是何處?」三藏道:
「貧僧是東土大唐國,奉旨意,上西天拜佛求經的。路過寶坊,天色將晚,特投
聖祠,告宿一宵,天光即行。」那老者十分歡喜,道了幾聲「失迎」,又叫童子
辦飯。三藏吃畢,謝了。
  
行者的眼乖,見他房簷下有一條搭衣的繩子,走將去,一把扯斷,將馬腳繫住。
那老者笑道:「這馬是那裏偷來的?」行者怒道:「你那老頭子,說話不知高低
。我們是拜佛的聖僧,又會偷馬?」老兒笑道:「不是偷的,如何沒有鞍轡韁繩
,卻來扯斷我晒衣的索子?」三藏陪禮道:「這個頑皮,只是性燥。──你要拴
馬,好生問老人家討條繩子,如何就扯斷他的衣索?──老先生,休怪,休怪。
我這馬,實不瞞你說,不是偷的。昨日東來,至鷹愁陡澗,原有騎的一匹白馬,
鞍轡俱全。不期那澗裏有條孽龍,在彼成精,他把我的馬連鞍轡一口吞之。幸虧
我徒弟有些本事,又感得觀音菩薩來澗邊擒住那龍,教他就變做我原騎的白馬,
毛片俱同,馱我上西天拜佛。今此過澗,未經一日,卻到了老先的聖祠,還不曾
置得鞍轡哩。」那老者道:「師父休怪,我老漢作笑耍子,誰知你高徒認真。我
小時也有幾個村錢,也好騎匹駿馬。只因累歲迍邅,遭喪失火,到此沒了下梢,
故充為廟祝,侍奉香火。幸虧這後莊施主家募化度日。我那裏倒還有一副鞍轡,
是我平日心愛之物,就是這等貧窮,也不曾捨得賣了。才聽老師父之言,菩薩尚
且救護神龍,教他化馬馱你,我老漢卻不能少有周濟。明日將那鞍轡取來,願送
老師父,扣背前去,乞為笑納。」三藏聞言,稱謝不盡。早又見童子拿出晚齋。
齋罷,掌上燈,安了鋪,各各寢歇。
  
至次早,行者起來道:「師父,那廟祝老兒昨晚許我們鞍轡,問他要,不要饒他
。」說未了,只見那老兒果擎著一副鞍轡、襯屜、韁籠之類,凡馬上一切用的,
無不全備,放在廊下道:「師父,鞍轡奉上。」三藏見了,歡喜領受。教行者拿
了,背上馬看,可相稱否。行者走上前,一件件的取起看了,果然是些好物。有
詩為證。詩曰:
    雕鞍彩晃柬銀星,寶鐙光飛金線明。
    襯屜幾層絨苫疊,牽韁三股紫絲繩。
    轡頭皮劄團花粲,雲扇描金舞獸形。
    環嚼叩成磨煉鐵,兩垂蘸水結毛纓。

  
行者心中暗喜,將鞍轡背在馬上,就似量著做的一般。三藏拜謝那老,那老慌忙
攙起道:「惶恐,惶恐。何勞致謝?」那老者也不再留,請三藏上馬。那長老出
得門來,攀鞍上馬。行者擔著行李。那老兒復袖中取出一條鞭兒來,卻是皮丁兒
寸劄的香籐柄子,虎筋絲穿結的梢兒,在路傍拱手奉上道:「聖僧,我還有一條
挽手兒,一發送了你罷。」那三藏在馬上接了道:「多承布施,多承布施。」
  
正打問訊,卻早不見了那老兒。及回看那里社祠,是一片光地。只聽得半空中有
人言語道:「聖僧,多簡慢你。我是落伽山山神、土地,蒙菩薩差送鞍轡與汝等
的。汝等可努力西行,卻莫一時怠慢。」慌得個三藏滾鞍下馬,望空禮拜道:
「弟子肉眼凡胎,不識尊神尊面,望乞恕罪。煩轉達菩薩,深蒙恩佑。」你看他
只管朝天磕頭,也不計其數。路傍邊活活的笑倒個孫大聖,孜孜的喜壞個美猴王
,上前來扯住唐僧道:「師父,你起來罷,他已去得遠了,聽不見你禱祝,看不
見你磕頭,只管拜怎的?」長老道:「徒弟呀,我這等磕頭,你也就不拜他一拜
,且立在傍邊,只管哂笑,是何道理?」行者道:「你那裏知道,像他這個藏頭
露尾的,本該打他一頓﹔只為看菩薩面上,饒他打,儘勾了,他還敢受我老孫之
拜?老孫自小兒做好漢,不曉得拜人,就是見了玉皇大帝、太上老君,我也只是
唱個喏便罷了。」三藏道:「不當人子,莫說這空頭話。快起來,莫誤了走路。」
那師父才起來收拾,投西而去。
  
此去行有兩個月太平之路,相遇的都是些羅羅、回回、狼蟲虎豹。光陰迅速,又
值早春時候。但見山林錦翠色,草木發青芽﹔梅英落盡,柳眼初開。師徒們行玩
春光,又見太陽西墜。三藏勒馬遙觀,山凹裏有樓臺影影,殿閣沉沉。三藏道:
「悟空,你看那裏是甚麼去處?」行者抬頭看了道:「不是殿宇,定是寺院。我
們趕起些,那裏借宿去。」三藏欣然從之,放開龍馬,徑奔前來。
  
    畢竟不知此去是甚麼去處,且聽下回分解。





}  \end{pinyinscope}\switchcolumn{\myfontc \section{第 一 五 回} 蛇 盘 山 诸 神 暗 中 保 佑 , 鹰 愁 涧 意 马 收 拾 , 又 说 行 人 伏 在 唐 僧 的 身 上 , 西 进 , 走 了 几 天 , 正 是 那 个 腊 月 寒 冷 的 天 气 , 朔 风 凛 凛 , 冰 冻 凌 凌 。
走 的 是 悬 崖 峭 壁 、 崎 岖 的 路 , 叠 岭 层 峦 险 峻 的 山 。
三 藏 在 马 上 , 远 远 听 到 水 声 喧 耳 , 回 头 大 叫 : 悟 空 , 是 那 里 的 水 声 ? 行 人 说 : 我 记 得 此 处 叫 蛇 盘 山 鹰 愁 涧 , 想 必 是 涧 里 的 水 声 。
说 不 完 , 马 跑 到 涧 边 , 三 藏 勒 着 马 来 观 看 。
只 见 一 股 细 细 的 寒 脉 穿 过 云 彩 过 去 , 湛 湛 的 清 波 映 日 红 。
声 音 摇 动 夜 间 的 雨 声 传 到 幽 谷 中 , 彩 云 飘 扬 , 朝 霞 闪 耀 太 空 。
千 仞 浪 飞 溅 碎 玉 石 , 一 泓 水 鸣 叫 清 风 。
流 归 万 顷 烟 波 去 , 鸥 鹭 相 忘 没 有 钓 鱼 。
师 徒 两 个 人 正 在 观 看 的 地 方 , 只 见 那 涧 水 当 中 响 一 声 , 钻 出 一 条 龙 来 , 推 波 掀 浪 , 冲 出 崖 山 , 就 去 抢 捉 老 人 。
忽 然 有 个 行 人 丢 了 行 李 , 把 师 父 抱 下 马 来 , 回 头 就 走 了 。
其 所 以 不 得 也 , 其 所 以 不 得 也 。
行 者 送 我 到 高 阜 上 坐 , 又 来 牵 马 挑 担 , 只 留 下 一 担 行 李 , 不 见 马 匹 。
他 将 行 李 担 送 到 师 父 面 前 说 : 师 父 , 那 妖 龙 也 不 见 踪 影 , 只 是 惊 走 我 的 马 了 。
三 藏 说 : 徒 弟 呵 , 你 怎 么 还 能 找 到 马 着 呢 ? 行 人 说 : 放 心 , 放 心 , 等 我 去 看 看 。
他 打 了 哨 , 跳 在 空 中 , 火 眼 金 睛 , 用 手 搭 着 凉 篷 , 四 下 围 观 , 更 不 见 马 的 踪 迹 。
按 着 落 云 头 , 报 告 说 : 师 父 , 我 们 的 马 断 了 吗 ? 那 龙 吃 了 , 四 下 里 再 看 不 见 。
三 藏 说 : 徒 弟 啊 , 那 个 人 能 有 多 大 口 , 竟 然 把 那 匹 大 马 连 鞍 都 吃 了 , 想 是 惊 慌 张 开 , 跑 在 那 山 凹 之 中 。
你 再 仔 细 看 看 。
行 者 说 : 你 也 不 知 道 我 的 本 事 。
我 的 双 眼 睛 , 白 日 里 常 常 看 一 千 里 路 的 吉 凶 。
像 那 千 里 之 内 , 蜻 蜓 儿 展 翅 , 我 也 看 见 , 何 期 那 匹 大 马 , 我 就 不 见 了 ? 三 藏 说 : 既 然 是 他 吃 了 , 我 又 怎 么 往 前 走 呢 ? 可 怜 呀 , 这 千 山 万 水 , 怎 么 能 逃 走 呢 ?
行 者 见 他 痛 哭 将 要 起 来 , 他 又 怎 么 忍 得 住 得 干 燥 , 发 声 喊 道 : 师 父 不 要 这 样 的 痈 疽 , 你 坐 着 , 坐 着 , 等 老 孙 去 找 他 , 教 他 还 我 马 匹 就 行 了 。
三 藏 听 到 这 话 , 更 加 恼 怒 , 就 大 声 喊 叫 , 说 : 你 呀 , 你 到 哪 里 去 找 他 , 只 怕 他 在 黑 暗 地 里 把 我 带 出 来 , 还 不 是 连 我 都 害 了 , 那 时 节 人 马 两 亡 , 怎 么 能 够 这 样 好 呢 ? 行 人 听 到 这 话 , 更 加 恼 怒 , 就 大 声 呼 叫 如 雷 , 说 道 : 你 太 不 济 , 不 济 , 又 要 马 骑 , 又 不 放 我 去 , 好 像 这 样 看 着 行 李 , 坐 到 老 罢 。
大 声 喊 叫 , 正 难 以 息 怒 , 只 听 到 空 中 有 人 说 话 , 大 叫 道 : 孙 大 圣 莫 恼 , 唐 御 弟 休 哭 。
我 们 是 观 音 菩 萨 派 来 的 一 路 神 祗 , 特 来 暗 中 保 护 佛 经 。
那 位 长 老 听 了 这 话 , 慌 忙 礼 拜 。
行 人 说 : 你 们 是 那 几 个 人 , 可 以 报 名 来 , 我 好 点 卯 。
众 神 说 : 我 们 是 六 丁 六 甲 、 五 方 揭 谛 、 四 值 功 曹 、 十 八 位 护 教 伽 蓝 , 各 自 轮 流 轮 流 等 候 。
行 者 说 : 今 天 先 从 谁 开 始 ? 众 人 揭 着 话 说 : 丁 甲 、 功 曹 、 伽 蓝 轮 次 轮 次 。
我 五 方 的 揭 谛 , 只 有 金 头 的 揭 谛 , 日 夜 不 离 左 右 。
行 者 说 : 既 然 这 样 , 不 应 当 赶 上 的 人 暂 且 退 回 去 , 留 下 六 丁 神 将 与 日 遇 功 曹 和 众 揭 谛 保 护 着 我 的 师 父 。
等 老 孙 去 找 那 涧 中 的 孽 龙 , 让 他 还 我 的 马 来 。
众 神 听 从 了 他 的 命 令 。
三 藏 这 才 放 下 心 , 坐 在 石 崖 上 , 吩 咐 行 人 仔 细 。
行 人 说 : 只 管 宽 心 。
猴 王 , 捆 了 一 束 绵 布 直 , 撩 起 虎 皮 裙 子 , 扯 着 金 钏 铁 棒 , 精 神 抖 擞 , 径 直 前 临 涧 壑 , 半 云 半 雾 , 在 水 面 上 高 呼 道 : 泼 泥 鳅 , 还 我 马 来 , 还 我 马 来 。 又 说 那 龙 吃 了 三 藏 的 白 马 , 埋 伏 在 那 涧 底 里 , 潜 心 养 性 , 只 听 见 有 人 喊 骂 索 马 。
他 心 中 火 发 , 急 忙 纵 身 跃 浪 翻 波 , 跳 上 来 说 : 是 那 个 胆 敢 在 这 里 海 口 伤 害 我 ? 行 人 见 了 他 , 大 骂 一 声 : 休 走 , 还 我 马 来 !
那 条 龙 , 张 牙 舞 爪 来 捉 。
其 两 个 人 在 涧 边 , 前 面 的 一 场 赌 斗 , 果 然 是 勇 猛 雄 猛 的 。
只 见 那 个 人 说 : 龙 伸 开 利 爪 , 猴 举 起 金 箍 。
那 个 胡 须 垂 白 玉 线 , 这 个 眼 睛 罩 红 金 灯 。
其 胡 须 下 有 明 珠 , 喷 出 彩 雾 , 这 个 手 中 有 铁 棒 , 舞 起 狂 风 。
其 子 , 其 子 也 , 其 子 也 。
其 他 两 个 都 因 为 有 难 而 遭 受 挫 折 , 现 在 要 成 功 , 各 自 显 耀 自 己 的 才 能 。
来 来 往 往 , 交 战 很 长 时 间 , 盘 旋 了 很 久 , 那 条 龙 力 软 筋 麻 , 不 能 抵 挡 敌 人 , 打 一 个 转 身 , 又 潜 入 水 里 , 深 潜 到 涧 底 , 再 也 不 出 头 。
猴 王 骂 不 绝 , 他 也 只 推 耳 聋 。
行 人 没 有 办 法 , 只 得 回 头 看 见 三 藏 菩 萨 说 : 师 父 , 这 个 怪 物 被 老 孙 骂 将 出 来 , 他 与 我 赌 斗 多 时 , 胆 战 而 逃 , 只 藏 在 水 中 间 , 再 也 不 出 来 了 。
行 者 说 : 你 看 你 说 的 话 , 不 是 他 吃 了 , 他 还 肯 出 来 呼 喊 , 与 老 孙 犯 对 答 , 三 藏 说 : 你 前 日 打 虎 时 曾 说 有 投 龙 伏 虎 的 手 段 , 今 天 为 什 么 不 能 降 服 他 呢 ? 原 来 那 猴 子 吃 不 到 人 急 着 他 ?
见 三 藏 之 人 , 先 生 曰 : 不 要 说 , 不 要 说 , 等 我 与 他 再 见 一 个 上 下 。
这 个 猴 子 拽 开 步 , 跳 到 涧 边 , 使 出 那 翻 江 搅 海 的 神 通 , 把 一 条 鹰 愁 山 涧 彻 底 澄 清 的 水 , 搅 得 好 像 那 九 曲 黄 河 泛 滥 的 波 涛 。
那 妖 龙 在 深 涧 中 坐 卧 不 安 , 心 中 想 着 说 : 这 才 是 福 无 双 降 , 祸 不 单 行 。
我 才 脱 脱 了 天 性 死 难 , 不 上 一 年 , 在 这 里 随 缘 度 日 , 又 撞 着 这 个 泼 魔 , 他 来 害 我 。
你 看 他 越 思 越 恼 , 受 不 得 屈 气 , 咬 着 牙 , 跳 出 去 , 骂 道 : 你 是 哪 里 来 的 泼 魔 , 这 样 欺 我 ? 行 者 说 : 你 不 管 我 那 里 不 是 你 , 你 只 要 还 马 , 我 就 饶 你 的 性 命 。
那 龙 说 : 你 的 马 是 我 吞 下 肚 子 , 为 什 么 吐 出 来 , 不 还 你 , 你 就 等 什 么 呢 ? 行 人 说 : 不 还 马 时 看 棍 子 , 只 打 死 你 , 偿 了 我 马 的 性 命 就 罢 了 。
其 两 个 人 又 在 山 崖 下 苦 战 。
打 不 了 几 合 , 小 龙 就 是 难 以 阻 挡 , 他 把 自 己 的 身 子 拿 出 来 , 变 成 一 条 水 蛇 , 钻 进 草 科 中 去 了 。
猴 王 拿 着 棍 子 , 赶 上 前 来 , 拨 草 寻 蛇 , 那 里 得 到 一 点 影 子 。
急 忙 找 到 三 个 尸 神 , 七 窍 烟 生 , 念 了 一 声 字 的 咒 语 , 立 即 把 当 坊 的 土 地 、 本 处 的 山 神 叫 出 来 , 一 齐 来 跪 下 说 : 山 神 、 土 地 来 见 。
行 人 说 : 伸 过 孤 拐 来 , 各 打 五 个 棍 子 见 面 , 与 老 孙 子 散 散 心 。
二 位 神 叩 头 哀 告 道 : 希 望 大 圣 有 方 便 , 容 许 小 神 告 诉 。
行 者 说 : 你 们 说 什 么 ? 二 位 神 人 说 : 大 圣 一 向 长 久 困 苦 , 小 神 不 知 道 什 么 时 候 出 来 , 所 以 我 们 不 曾 接 过 , 万 望 饶 恕 我 们 的 罪 过 。
行 人 说 : 既 然 这 样 , 我 暂 且 不 打 你 。
我 问 你 们 : 鹰 愁 涧 里 , 是 那 方 来 的 怪 龙 , 他 为 什 么 夺 了 我 师 父 的 白 马 吃 了 ? 二 位 神 人 说 : 大 圣 自 来 就 不 曾 有 过 师 父 , 原 来 是 个 不 伏 天 不 伏 地 混 元 上 真 , 怎 么 会 有 什 么 师 父 的 马 ?
我 只 是 为 了 那 上 的 勾 当 , 完 全 受 到 了 五 百 年 的 苦 难 。
现 在 承 蒙 观 音 菩 萨 劝 善 , 现 在 唐 朝 皇 帝 驾 下 的 真 僧 救 出 我 来 , 教 我 和 他 做 徒 弟 , 去 西 天 去 拜 佛 求 经 。
因 为 路 过 此 处 , 失 去 了 我 师 父 的 白 马 。
二 位 神 人 说 : 原 来 就 是 这 样 。
这 条 涧 中 自 来 就 没 有 邪 气 , 只 是 深 又 陡 又 宽 又 宽 , 水 光 彻 底 澄 清 , 乌 鸦 、 鹊 雀 不 敢 飞 过 去 ; 因 为 水 清 照 见 自 己 的 形 影 , 便 认 为 是 同 群 的 鸟 , 往 往 投 入 水 中 , 所 以 叫 做 鹰 愁 陡 涧 。
只 是 前 年 前 , 观 音 菩 萨 因 为 去 寻 找 取 经 的 人 , 救 了 一 条 玉 龙 , 送 他 到 这 里 , 教 他 等 待 那 些 取 经 的 人 , 不 许 做 非 作 歹 。
其 实 是 饥 饿 了 的 时 候 , 上 岸 来 吃 鸟 鹊 , 或 者 是 捉 獐 鹿 来 吃 。
不 知 其 他 无 知 , 今 日 冲 撞 了 大 圣 。
行 人 说 : 先 前 一 次 , 他 还 与 老 孙 侮 辱 手 , 盘 旋 了 几 合 , 后 一 次 , 是 老 孙 喊 骂 , 他 再 也 不 出 来 。
因 此 使 得 一 个 翻 江 翻 海 的 法 子 , 搅 混 了 他 的 涧 水 , 他 就 要 上 来 , 还 要 争 持 。
不 知 道 老 孙 的 棍 重 , 他 挡 住 架 不 住 , 就 变 成 一 条 水 蛇 , 钻 到 草 里 。
我 赶 来 找 他 , 却 没 有 踪 迹 。
土 地 说 : 大 圣 不 知 道 。
这 条 涧 水 千 万 个 孔 窍 相 通 , 所 以 这 里 水 流 深 远 。
想 来 是 这 里 也 有 一 个 孔 , 他 钻 就 要 下 去 。
也 不 必 大 圣 发 怒 , 在 这 里 寻 找 , 如 果 捉 住 这 个 东 西 , 只 要 请 让 我 带 着 观 世 音 来 , 自 然 就 会 埋 伏 了 。
行 者 见 了 , 就 叫 来 山 神 、 土 地 , 一 同 来 见 了 三 藏 , 详 细 说 了 以 前 的 事 。
三 藏 菩 萨 说 : 如 果 要 去 请 菩 萨 , 几 时 才 回 来 , 我 贫 僧 饥 寒 , 怎 么 忍 耐 呢 ? 说 不 了 , 只 听 到 暗 空 中 有 金 头 揭 谛 叫 道 : 大 圣 , 你 不 必 动 身 , 小 神 去 请 菩 萨 来 。
行 人 非 常 高 兴 , 说 道 : 有 拖 累 , 有 拖 累 。
快 行 , 快 行 。
那 揭 谛 急 忙 放 纵 云 头 , 径 直 上 南 海 。
行 者 告 诫 山 神 、 土 地 守 护 师 父 , 每 天 赶 上 功 曹 去 寻 找 斋 供 , 他 又 去 山 涧 边 巡 逻 不 问 题 。
又 说 金 头 揭 谛 一 驾 云 , 早 到 南 海 。
按 照 祥 光 , 直 到 落 伽 山 紫 竹 林 中 , 托 付 那 金 甲 诸 天 和 木 叉 惠 岸 转 达 , 得 以 见 到 菩 萨 。
菩 萨 说 : 你 来 干 什 么 ? 揭 谛 说 : 唐 僧 在 蛇 盘 山 鹰 愁 陡 涧 失 了 马 , 急 得 孙 大 圣 , 进 退 两 难 。
又 问 原 来 的 土 神 , 他 说 是 菩 萨 送 我 在 涧 中 的 孽 龙 吞 了 。
那 大 圣 着 小 神 来 告 诉 请 求 菩 萨 降 下 这 个 孽 龙 , 还 给 他 的 马 匹 。
菩 萨 听 了 , 说 道 : 这 个 人 本 是 西 海 敖 的 儿 子 , 他 纵 火 烧 了 殿 上 的 明 珠 , 他 父 亲 告 发 他 忤 逆 , 在 天 庭 上 犯 了 死 罪 。
是 我 亲 见 玉 帝 , 讨 他 下 来 , 让 他 与 唐 僧 做 脚 力 。
他 何 以 反 而 吃 了 唐 僧 的 马 , 这 样 的 说 法 , 等 我 去 来 。
那 菩 萨 降 临 莲 台 , 径 直 离 仙 洞 , 与 揭 谛 一 起 驾 着 祥 光 , 经 过 南 海 而 来 。
有 诗 作 证 。
《 诗 经 》 说 : 佛 说 蜜 多 三 藏 经 , 菩 萨 扬 善 满 长 城 。
摩 诃 妙 语 通 天 地 , 般 若 真 言 救 鬼 灵 。
致 使 金 蝉 重 新 脱 壳 , 所 以 让 玄 奘 再 次 修 行 。
只 因 道 路 阻 隔 , 鹰 愁 涧 , 龙 子 归 真 化 马 形 。
那 菩 萨 和 揭 谛 不 多 时 到 了 蛇 盘 山 , 却 在 那 半 空 中 留 住 祥 云 , 低 头 观 看 , 只 见 孙 行 者 正 在 涧 边 大 叫 骂 。
菩 萨 在 揭 谛 里 , 唤 他 来 。
那 揭 谛 按 着 云 头 , 不 经 过 三 藏 , 直 到 涧 边 , 对 行 人 说 : 菩 萨 来 了 。
行 者 听 了 , 急 忙 放 着 云 跳 到 空 中 , 对 着 他 大 叫 道 : 你 这 个 七 佛 之 师 , 慈 悲 的 教 主 , 你 为 什 么 生 方 法 儿 害 我 ? 菩 萨 说 : 我 把 你 这 个 大 胆 的 马 流 , 村 愚 的 赤 尻 ?
我 再 三 尽 意 , 估 计 得 到 一 个 取 经 人 来 , 吩 咐 他 救 你 的 性 命 , 你 为 什 么 不 来 感 谢 我 活 命 之 恩 , 反 而 来 和 我 吵 闹 ? 行 人 说 : 你 弄 得 我 好 吗 ?
你 既 然 放 我 出 来 , 让 我 逍 遥 自 在 , 你 就 可 以 了 。
你 前 日 在 海 上 迎 着 我 , 伤 了 我 几 句 , 教 我 来 尽 心 竭 力 , 伏 侍 唐 僧 就 罢 了 , 你 为 何 赠 给 他 一 顶 花 帽 , 哄 我 戴 在 头 上 受 苦 , 把 这 个 钳 子 长 在 老 孙 头 上 , 又 教 他 念 一 卷 紧 紧 儿 咒 , 让 那 老 和 尚 念 了 又 念 , 教 我 的 头 上 疼 了 又 疼 , 这 不 是 你 害 我 吗 ? 菩 萨 笑 着 说 : 你 这 个 猴 子 , 你 不 遵 从 教 令 , 不 接 受 。
行 人 说 : 这 件 事 , 做 就 是 我 的 魔 头 罢 了 。
你 为 何 又 把 有 罪 的 孽 龙 送 到 这 里 成 精 , 教 他 吃 了 我 师 父 的 马 匹 , 这 又 是 放 纵 凶 恶 的 人 作 恶 , 太 不 好 。
菩 萨 说 : 那 条 龙 , 是 我 亲 自 上 奏 玉 帝 , 讨 他 就 在 这 里 , 专 门 为 求 经 人 做 脚 力 。
你 想 那 东 土 来 的 凡 马 , 哪 里 能 得 到 它 万 水 千 山 , 哪 里 能 得 到 那 灵 山 佛 地 , 必 须 是 得 到 这 个 龙 马 , 才 能 去 掉 。
行 者 说 : 像 他 这 样 害 怕 老 孙 子 , 潜 藏 不 出 , 怎 么 办 呢 ? 菩 萨 叫 揭 谛 说 : 你 去 涧 边 叫 一 声 叫 叫 敖 逢 龙 王 玉 龙 三 太 子 , 你 出 来 , 有 南 海 菩 萨 在 这 里 。
其 所 以 不 得 之 。
那 揭 谛 果 然 在 涧 边 喊 了 两 遍 。
那 小 龙 翻 波 跳 浪 , 跳 出 水 来 , 变 成 一 个 人 像 , 踏 了 云 头 , 到 空 中 对 菩 萨 礼 拜 道 : 刚 才 承 蒙 菩 萨 解 脱 生 命 的 恩 德 , 在 这 里 久 等 , 更 没 有 听 到 取 经 人 的 音 信 。
菩 萨 指 着 行 人 说 : 这 不 是 取 经 人 的 大 弟 弟 , 小 龙 见 了 说 : 菩 萨 , 这 是 我 的 对 头 。
我 昨 日 腹 中 饥 饿 , 果 然 吃 了 他 的 马 匹 。
其 所 以 不 得 之 , 其 所 以 不 得 之 。
他 更 不 曾 提 出 一 个 《 取 经 》 的 字 样 。
行 人 说 : 你 又 不 曾 问 我 姓 什 么 名 字 , 我 怎 么 就 说 ? 小 龙 说 : 我 不 曾 问 你 是 哪 里 来 的 泼 魔 , 你 吵 道 : 管 什 么 那 里 不 那 里 , 只 还 我 马 来 。
你 何 曾 说 出 半 个 唐 字 呢 ? 菩 萨 说 : 那 猴 头 专 门 倚 仗 自 强 , 怎 么 会 称 赞 别 人 呢 ? 今 天 前 去 , 还 有 归 顺 的 呢 ?
如 果 问 话 时 , 先 提 起 《 取 经 》 的 字 来 , 又 不 用 劳 心 , 自 然 拱 手 伏 地 。
行 路 的 人 欢 喜 地 领 教 。
菩 萨 上 前 , 把 小 龙 脖 子 下 的 明 珠 摘 出 来 , 用 杨 柳 枝 蘸 出 甘 露 , 在 他 身 上 拂 去 一 拂 , 吹 了 一 口 仙 气 , 喝 声 叫 道 : 变 了 , 那 龙 就 变 成 了 他 原 来 的 马 匹 毛 片 。
又 用 言 语 嘱 咐 道 : 你 必 须 用 心 回 去 了 业 障 , 功 成 之 后 , 超 越 凡 龙 , 还 给 你 金 身 正 果 。
那 小 龙 口 咬 着 横 骨 , 心 里 领 诺 。
菩 萨 教 导 悟 空 , 领 他 去 见 三 藏 。
说 : 我 回 到 海 上 去 吧 。
行 者 拽 住 菩 萨 不 放 , 说 : 我 不 去 了 , 我 不 去 了 。
西 方 的 道 路 这 样 崎 岖 , 保 护 这 个 凡 僧 , 多 时 得 到 , 似 乎 这 样 多 磨 多 折 , 老 孙 的 性 命 也 难 以 保 全 , 怎 么 能 成 就 什 么 功 果 呢 我 不 去 了 , 我 不 去 了 , 我 不 去 了 , 我 不 去 了 。
菩 萨 说 : 你 当 年 未 成 人 道 , 而 且 肯 尽 心 修 行 觉 悟 , 你 今 天 脱 了 天 灾 , 为 什 么 还 会 倒 生 懒 惰 , 我 门 中 以 寂 灭 成 真 , 必 须 要 信 心 正 果 。
假 如 到 了 那 伤 身 苦 磨 的 地 方 , 我 答 应 你 叫 天 应 , 叫 地 地 神 , 十 分 之 二 到 那 难 以 脱 脱 的 时 候 , 我 也 亲 自 来 救 你 。
你 回 来 , 我 再 赠 给 你 一 件 本 事 。
菩 萨 把 杨 柳 叶 子 摘 下 三 个 , 放 在 行 人 的 脑 后 , 喝 道 : 变 了 就 变 成 了 三 根 救 命 的 毫 毛 。
教 他 说 : 如 果 到 了 那 些 无 济 于 无 主 的 时 候 , 可 以 随 机 应 变 , 救 得 了 你 们 的 急 难 。
行 者 听 了 这 些 好 话 , 才 感 谢 了 大 慈 大 悲 的 菩 萨 。
那 菩 萨 香 风 绕 绕 , 彩 雾 飘 飘 , 径 转 普 陀 寺 而 去 。
这 个 行 人 刚 刚 抓 住 云 头 , 扯 着 那 龙 马 的 头 鬃 , 来 见 三 藏 菩 萨 说 : 师 父 , 马 有 了 。
三 藏 一 看 , 非 常 高 兴 地 说 : 弟 子 , 这 匹 马 怎 么 比 先 前 反 而 肥 壮 了 , 在 哪 里 找 着 呢 ? 行 人 说 : 师 父 , 你 还 做 梦 吗 ?
这 才 是 金 头 揭 谛 请 了 菩 萨 来 , 把 那 涧 里 的 龙 变 成 我 们 的 白 马 , 它 们 的 毛 片 相 同 , 只 是 少 了 鞍 。
着 老 孙 , 是 将 来 的 。
三 藏 大 吃 一 惊 说 : 菩 萨 在 哪 里 , 等 我 去 拜 谢 他 。
行 者 说 : 菩 萨 此 时 已 经 到 了 南 海 , 不 用 麻 烦 了 。
三 藏 就 摘 土 焚 香 , 望 着 南 边 礼 拜 。
行 礼 完 毕 , 起 身 就 与 行 人 收 拾 起 身 往 前 走 。
行 者 喝 退 了 山 神 、 土 地 , 吩 咐 揭 谛 、 功 曹 , 又 请 师 父 上 马 。
三 藏 说 : 那 没 有 鞍 的 马 , 怎 么 能 骑 得 到 呢 ? 暂 且 等 我 找 船 渡 过 山 涧 去 , 再 作 个 区 区 的 地 方 。
行 人 说 : 这 个 师 父 好 不 知 时 务 , 这 个 旷 野 山 中 , 船 从 哪 里 来 ? 这 匹 马 , 他 在 这 里 久 住 , 一 定 知 道 水 势 , 就 骑 着 他 做 一 个 船 儿 过 去 吧 。
三 藏 无 可 奈 何 , 只 得 依 照 他 的 话 , 跨 了 产 马 。
行 人 挑 着 行 李 。
到 了 山 涧 边 。
只 见 那 上 游 的 头 , 有 一 个 渔 翁 , 撑 着 一 个 枯 木 的 水 , 顺 流 而 下 。
行 者 见 了 , 用 手 招 呼 道 : 那 老 渔 , 你 来 , 你 来 。
我 是 东 土 求 经 的 , 我 师 父 到 此 难 过 , 你 来 渡 他 一 渡 。
渔 翁 听 了 这 话 , 就 忙 忙 把 他 拉 拢 起 来 。
行 者 请 师 父 下 了 马 , 扶 持 左 右 的 人 。
三 藏 送 上 了 子 , 抽 上 马 匹 , 安 置 了 行 李 。
那 老 渔 人 把 它 撑 开 , 像 风 似 箭 , 不 知 不 觉 过 了 鹰 愁 山 的 陡 涧 , 上 了 西 岸 。
三 藏 教 行 者 解 开 包 袱 , 取 出 大 唐 的 几 文 钱 钞 , 送 给 老 渔 。
老 渔 人 把 一 根 竹 篙 撑 开 道 : 不 要 钱 , 不 要 钱 。
向 着 中 流 , 茫 茫 茫 茫 地 走 了 。
三 藏 很 不 满 意 , 只 管 合 掌 称 谢 。
行 人 说 : 师 父 休 致 意 了 , 你 不 认 得 他 , 他 是 这 涧 里 的 水 神 。
不 曾 来 接 我 的 老 孙 , 老 孙 还 要 打 他 吗 ?
只 是 现 在 免 打 就 勾 了 他 的 , 哪 敢 要 钱 呢 ? 那 师 父 似 乎 相 信 不 信 , 只 得 又 跨 着 产 马 , 随 着 行 人 , 径 直 奔 赴 大 路 , 奔 向 西 方 而 去 。
这 正 是 : 广 大 真 如 登 彼 岸 , 诚 心 了 性 上 灵 山 。
同 师 往 前 走 , 不 知 不 觉 地 变 红 色 的 太 阳 沉 下 西 边 , 天 光 渐 渐 晚 了 。
只 见 淡 云 撩 乱 , 山 月 昏 暗 。
满 天 霜 色 生 寒 冷 , 四 面 风 声 吹 透 身 体 。
孤 鸟 离 开 的 时 候 , 苍 渚 空 阔 , 落 霞 明 亮 的 时 候 , 远 山 低 。
疏 林 千 棵 大 树 吼 叫 , 空 岭 独 猿 啼 。
长 途 不 见 行 人 的 踪 迹 , 万 里 归 船 入 夜 时 。
三 藏 在 马 上 远 远 地 观 看 , 忽 然 看 见 路 旁 有 一 座 庄 院 。
三 藏 菩 萨 说 : 悟 空 , 前 面 的 人 家 , 可 以 借 宿 , 明 早 再 去 。
行 人 抬 头 看 见 道 : 师 父 , 不 是 人 家 的 庄 院 。
三 藏 说 : 为 什 么 不 是 ? 行 人 说 : 人 家 的 庄 院 , 却 没 有 飞 鱼 稳 兽 的 脊 梁 , 这 断 是 个 庙 宇 庵 院 。
师 傅 之 子 , 其 兄 弟 之 子 , 其 兄 弟 之 子 , 其 兄 弟 之 子 , 其 兄 弟 之 子 , 其 兄 弟 之 子 , 其 兄 弟 之 子 。
三 藏 下 了 马 , 只 见 那 门 上 有 三 个 大 字 , 是 里 社 祠 。
其 中 有 一 个 老 人 , 脖 子 挂 着 几 个 珠 子 , 合 掌 来 迎 接 , 大 叫 道 : 师 父 请 坐 。
三 藏 急 忙 回 礼 , 上 殿 去 参 拜 圣 像 。
那 个 老 人 就 叫 来 童 子 进 献 茶 叶 。
茶 罢 , 三 藏 问 老 人 说 : 此 庙 为 什 么 叫 里 社 ? 老 人 说 : 我 这 里 是 西 番 哈 国 的 边 界 。
此 庙 后 面 有 一 个 庄 人 家 , 一 起 发 了 虔 诚 的 心 , 建 造 了 这 个 庙 宇 。
里 是 指 一 乡 里 的 土 地 , 社 是 指 一 社 土 神 。
每 逢 春 耕 、 夏 耘 、 秋 收 、 冬 藏 的 日 子 , 各 备 三 牲 、 花 果 , 来 此 祭 社 , 以 保 四 时 清 吉 , 五 谷 丰 登 , 六 畜 繁 盛 。
三 藏 听 了 这 话 , 点 头 夸 赞 说 : 正 是 离 家 三 里 远 , 别 是 一 乡 风 。
我 是 那 里 人 家 , 再 没 有 这 样 的 好 处 。
老 人 又 问 : 师 父 仙 乡 是 什 么 地 方 ? 三 藏 说 : 贫 僧 是 东 土 大 唐 国 , 奉 圣 旨 , 上 西 天 拜 佛 求 经 的 。
路 过 宝 坊 , 天 色 将 晚 , 特 意 投 奔 圣 祠 , 告 诉 他 住 宿 一 夜 , 天 光 就 走 了 。
那 个 老 人 十 分 欢 喜 , 说 了 几 声 失 迎 , 又 叫 童 子 办 饭 。
三 藏 吃 完 , 谢 了 。
行 人 眼 睛 不 明 , 看 见 他 的 房 檐 下 有 一 条 绳 子 , 走 了 将 要 走 , 一 把 把 绳 子 拽 断 , 把 马 脚 系 住 。
那 个 老 头 子 笑 着 说 : 这 马 是 从 哪 里 偷 来 的 ? 行 人 发 怒 说 : 你 这 个 老 头 子 , 说 话 不 知 道 高 低 。
我 们 是 拜 佛 的 圣 僧 , 又 会 偷 马 , 老 儿 笑 着 说 : 不 是 偷 的 , 为 什 么 没 有 鞍 绳 , 却 来 扯 断 我 晒 衣 的 绳 子 ? 三 藏 陪 礼 说 : 这 个 顽 皮 , 只 是 性 干 干 燥 。
你 要 拴 马 , 好 好 问 老 人 家 讨 一 条 绳 子 , 为 什 么 就 拽 断 他 的 衣 服 绳 索 ? 老 先 生 , 不 要 怪 , 不 要 怪 。
我 的 这 匹 马 , 实 在 不 骗 你 说 , 不 是 偷 的 。
昨 天 东 边 来 , 到 了 鹰 愁 山 的 陡 涧 , 原 来 有 一 匹 白 马 , 马 鞍 和 马 鞍 都 完 了 。
不 料 那 涧 里 有 一 条 孽 龙 , 在 那 里 成 精 , 他 把 我 的 马 连 鞍 一 口 吞 掉 。
幸 亏 我 的 弟 弟 有 些 本 事 , 又 感 动 了 观 音 菩 萨 来 到 涧 边 捉 住 那 龙 , 教 他 就 变 成 我 原 来 骑 的 白 马 , 毛 片 都 是 一 样 的 , 让 我 到 西 天 去 拜 佛 。
今 天 此 处 过 山 涧 , 不 到 一 天 , 却 到 了 老 先 的 圣 祠 , 还 不 曾 安 置 过 鞍 了 。
那 个 老 人 说 : 师 父 不 要 怪 怪 , 我 是 个 老 汉 戏 弄 孩 子 , 谁 知 道 你 高 徒 认 真 。
我 小 时 候 有 几 个 村 子 的 钱 , 也 喜 欢 骑 一 匹 骏 马 。
只 因 多 年 丧 失 火 灾 , 到 现 在 没 了 下 梢 , 所 以 充 当 了 庙 祝 , 侍 奉 香 火 。
幸 亏 这 以 后 的 庄 园 施 舍 给 主 人 家 , 招 募 教 化 度 日 。
我 的 那 里 倒 还 有 一 副 鞍 , 是 我 平 日 心 喜 爱 的 东 西 , 就 是 这 样 的 贫 穷 , 也 不 曾 舍 得 卖 掉 。
刚 听 老 师 父 的 话 , 菩 萨 尚 且 救 护 神 龙 , 教 他 变 马 你 , 我 是 个 老 汉 子 , 却 不 能 稍 有 周 济 。
第 二 天 我 把 那 个 马 鞍 拿 来 , 愿 意 送 老 师 父 , 我 就 敲 着 他 的 背 走 了 , 请 给 我 笑 纳 。
三 藏 听 了 这 话 , 称 谢 不 尽 。
早 晨 又 见 童 子 拿 出 晚 斋 。
斋 戒 完 毕 , 掌 上 灯 火 , 安 放 了 铺 , 各 自 睡 觉 休 息 。
到 第 二 天 早 晨 , 行 人 起 来 说 : 师 父 , 那 庙 祝 老 儿 昨 晚 答 应 我 们 马 鞍 , 问 他 要 , 不 要 饶 他 。
还 没 说 完 , 只 见 那 个 老 人 果 然 拿 着 一 副 鞍 、 衬 网 、 笼 之 类 的 东 西 , 凡 马 上 所 用 的 东 西 , 没 有 不 完 备 的 , 放 在 廊 下 说 : 师 父 , 鞍 奉 上 。
三 藏 见 了 , 欢 喜 接 受 。
教 导 行 人 拿 了 , 背 上 马 看 , 可 以 互 相 称 呼 吗 ?
行 者 走 上 前 , 一 个 一 个 一 个 地 方 都 拿 起 来 看 , 果 然 是 个 好 东 西 。
有 诗 作 证 。
诗 说 : 雕 鞍 彩 晃 柬 银 星 , 宝 光 飞 金 线 明 。
有 几 层 毯 子 , 毯 子 叠 叠 , 有 三 股 紫 色 丝 绳 。
头 皮 团 花 , 云 扇 雕 金 舞 兽 形 。
环 绕 敲 击 成 磨 炼 铁 , 两 边 下 垂 蘸 水 , 结 成 毛 缨 。
行 人 心 中 暗 自 高 兴 , 把 鞍 背 在 马 上 , 就 好 像 量 着 做 的 一 样 。
三 藏 拜 谢 那 老 , 那 老 惊 慌 地 搀 着 起 来 说 : 惶 恐 , 惶 恐 。
何 劳 致 谢 ? 那 老 人 不 再 留 下 , 请 三 藏 上 马 。
那 个 长 老 从 门 口 出 来 , 攀 着 马 鞍 上 马 。
行 人 担 着 行 李 。
那 老 子 从 袖 中 取 出 一 条 鞭 子 来 , 却 是 皮 丁 儿 寸 的 香 藤 柄 子 , 虎 筋 丝 穿 结 的 梢 子 , 在 路 旁 拱 手 奉 上 , 说 : 圣 僧 , 我 还 有 一 条 挽 手 儿 , 一 发 送 给 你 罢 了 。
三 藏 在 马 上 接 着 说 : 多 承 布 施 , 多 承 布 施 。
他 正 打 开 问 讯 , 却 早 早 不 见 了 那 个 老 孩 子 。
回 头 看 那 里 的 社 祠 , 是 一 片 光 亮 的 地 方 。
只 听 到 半 空 中 有 人 说 话 , 说 道 : 圣 僧 , 多 简 慢 你 。
我 是 落 伽 山 的 山 神 、 土 地 , 承 蒙 菩 萨 派 我 送 鞍 给 你 们 。
你 们 可 以 努 力 西 行 , 不 要 一 时 懈 怠 。
他 惊 慌 得 到 了 三 藏 马 鞍 下 马 , 望 空 礼 拜 道 : 弟 子 肉 眼 凡 胎 , 不 认 识 尊 神 尊 面 , 希 望 饶 恕 我 的 罪 过 。
请 转 达 菩 萨 , 深 深 地 承 蒙 恩 德 保 佑 。
你 看 他 只 管 朝 天 叩 头 , 也 不 计 其 数 。
在 路 旁 边 活 活 地 笑 倒 个 孙 大 圣 , 喜 欢 坏 个 美 猴 王 , 上 前 来 拽 住 唐 僧 说 : 师 父 , 你 起 来 罢 , 他 已 经 离 开 很 远 了 , 听 不 见 你 祷 祝 , 看 不 见 你 叩 头 , 只 管 拜 什 么 ? 长 老 说 : 弟 子 啊 , 我 这 样 的 槌 头 , 你 就 不 拜 他 一 拜 , 而 且 立 在 旁 边 , 只 管 笑 , 是 什 么 道 理 ?
三 藏 说 : 不 当 人 子 , 不 要 说 这 种 空 话 。
快 点 起 来 , 不 要 误 了 走 路 。
那 个 师 父 刚 起 来 收 拾 , 投 向 西 边 走 了 。
此 去 有 两 个 月 , 太 平 之 路 , 相 遇 的 都 是 罗 罗 、 回 回 、 狼 虫 虎 豹 。
光 阴 迅 速 , 又 适 逢 早 春 时 节 。
只 见 山 林 锦 绣 的 翠 色 , 草 木 发 出 青 芽 ; 梅 花 落 尽 , 柳 眼 初 开 。
师 徒 们 在 春 光 中 游 玩 , 又 看 见 太 阳 西 坠 。
三 藏 勒 马 远 远 地 观 看 , 山 凹 里 有 楼 台 影 影 , 殿 阁 沉 沉 。
三 藏 菩 萨 说 : 悟 空 , 你 看 那 里 是 什 么 地 方 ? 行 人 抬 头 看 了 , 说 : 不 是 殿 宇 , 定 是 寺 院 。
我 们 赶 快 起 来 , 到 那 里 借 宿 去 。
三 藏 欣 然 听 从 了 他 的 话 , 放 开 龙 马 , 径 直 奔 向 前 来 。
我 最 终 不 知 道 此 去 是 什 么 去 的 地 方 , 暂 且 听 下 回 的 分 析 。
}\switchcolumn\flushpage  \begin{pinyinscope}{\myfontt \section{第十六回}     觀音院僧謀寶貝 黑風山怪竊袈裟

卻說他師徒兩個策馬前來,直至山門首觀看,果然是一座寺院。但見那:
層層殿閣,疊疊廊房。三山門外,巍巍萬道彩雲遮;五福堂前,豔豔千條紅霧遶
。兩路松篁,一林檜柏。兩路松篁,無年無紀自清幽;一林檜柏,有色有顏隨傲
麗。又見那鐘鼓樓高,浮屠塔峻。安禪僧定性,啼樹鳥音閑。寂寞無塵真寂寞,
清虛有道果清虛。
  詩曰:
    上剎祇園隱翠窩,招提勝景賽娑婆。
    果然淨土人間少,天下名山僧占多。

  
長老下了馬,行者歇了擔,正欲進門,只見那門裏走出一眾僧來。你看他怎生模
樣:
    頭戴左笄帽,身穿無垢衣。
    銅環雙墜耳,絹帶束腰圍。
    草履行來穩,木魚手內提。
    口中常作念,般若總皈依。

  
三藏見了,侍立門傍,道個問訊。那和尚連忙答禮,笑道:「失瞻。」問:「是
那裏來的?請入方丈獻茶。」三藏道:「我弟子乃東土欽差,上雷音寺拜佛求經
。至此處天色將晚,欲借上剎一宵。」那和尚道:「請進裏坐,請進裏坐。」三
藏方喚行者牽馬進來。那和尚忽見行者相貌,有些害怕,便問:「那牽馬的是個
甚麼東西?」三藏道:「悄言,悄言。他的性急,若聽見你說是甚麼東西,他就
惱了。他是我的徒弟。」那和尚打了個寒噤,咬著指頭道:「這般一個醜頭怪腦
的,好招他做徒弟?」三藏道:「你看不出來哩,醜自醜,甚是有用。」
  
那和尚只得同三藏與行者進了山門。山門裏,又見那正殿上書四個大字,是「觀
音禪院」。三藏又大喜道:「弟子屢感菩薩聖恩,未及叩謝。今遇禪院,就如見
菩薩一般,甚好拜謝。」那和尚聞言,即命道人開了殿門,請三藏朝拜。那行者
拴了馬,丟了行李,同三藏上殿。三藏展背舒身,鋪胸納地,望金像叩頭。那和
尚便去打鼓。行者就去撞鐘。三藏俯伏臺前,傾心禱祝。祝拜已畢,那和尚住了
鼓,行者還只管撞鐘不歇,或緊或慢,撞了許久。那道人道:「拜已畢了,還撞
鐘怎麼?」行者方丟了鐘杵,笑道:「你那裏曉得!我這是『做一日和尚撞一日
鐘』的。」此時卻驚動那寺裏大小僧人、上下房長老,聽得鐘聲亂響,一齊擁出
道:「那個野人在這裏亂敲鐘鼓?」行者跳將出來,咄的一聲道:「是你孫外公
撞了耍子的。」那些和尚一見了,諕得跌跌滾滾,都爬在地下道:「雷公爺爺!」
行者道:「雷公是我的重孫兒哩。起來,起來,不要怕,我們是東土大唐來的老
爺。」眾僧方才禮拜。見了三藏,都才放心不怕。內有本寺院主請道:「老爺們
到後方丈中奉茶。」遂而解韁牽馬,抬了行李,轉過正殿,徑入後房,序了坐次。
  
那院主獻了茶,又安排齋供。天光尚早,三藏稱謝未畢,只見那後面有兩個小童
,攙著一個老僧出來。看他怎生打扮:
頭上戴一頂毘盧方帽,貓睛石的寶頂光輝;身上穿一領錦絨褊衫,翡翠毛的金邊
晃亮。一對僧鞋攢八寶,一根拄杖嵌雲星。滿面皺痕,好似驪山老母;一雙昏眼
,卻如東海龍君。口不關風因齒落,腰駝背屈為筋攣。
  
眾僧道:「師祖來了。」三藏躬身施禮迎接道:「老院主,弟子拜揖。」那老僧
還了禮,又各敘坐。老僧道:「適間小的們說,東土唐朝來的老爺,我才出來奉
見。」三藏道:「輕造寶山,不知好歹,恕罪,恕罪。」老僧道:「不敢,不敢
。」因問:「老爺,東土到此,有多少路程?」三藏道:「出長安邊界,有五千
餘里。過兩界山,收了一眾小徒,一路來,行過西番哈咇國,經兩個月,又有五
六千里,才到了貴處。」老僧道:「也有萬里之遙了。我弟子虛度一生,山門也
不曾出去,誠所謂『坐井觀天』,樗朽之輩。」三藏又問:「老院主高壽幾何?」
老僧道:「痴長二百七十歲了。」行者聽見道:「這還是我萬代孫兒哩。」三藏
瞅了他一眼道:「謹言,莫要不識高低,沖撞人。」那和尚便問:「老爺,你有
多少年紀了?」行者道:「不敢說。」
  
那老僧也只當一句瘋話,便不介意,也不再問,只叫獻茶。有一個小幸童,拿出
一個羊脂玉的盤兒,有三個法藍鑲金的茶鍾。又一童,提一把白銅壺兒,斟了三
杯香茶。真個是色欺榴蕊豔,味勝桂花香。三藏見了,誇愛不盡道:「好物件,
好物件,真是美食美器。」那老僧道:「污眼,污眼。老爺乃天朝上國,廣覽奇
珍,似這般器具,何足過獎?老爺自上邦來,可有甚麼寶貝,借與弟子一觀?」
三藏道:「可憐,我那東土無甚寶貝;就有時,路程遙遠,也不能帶得。」行者
在傍道:「師父,我前日在包袱裏,曾見那領袈裟,不是件寶貝?拿與他看看如
何?」眾僧聽說袈裟,一個個冷笑。行者道:「你笑怎的?」院主道:「老爺才
說袈裟是件寶貝,言實可笑。若說袈裟,似我等輩者,不止二三十件;若論我師
祖,在此處做了二百五六十年和尚,足有七八百件。」叫:「拿出來看看。」那
老和尚也是他一時賣弄,便叫道人開庫房,頭陀抬櫃子,就抬出十二櫃,放在天
井中,開了鎖。兩邊設下衣架,四圍牽了繩子,將袈裟一件件抖開掛起,請三藏
觀看。果然是滿堂綺繡,四壁綾羅。
  
行者一一觀之,都是些穿花納錦,刺繡銷金之物,笑道:「好,好,好。收起,
收起。把我們的也取出來看看。」三藏把行者扯住,悄悄的道:「徒弟,莫要與
人鬥富。你我是單身在外,只恐有錯。」行者道:「看看袈裟,有何差錯?」三
藏道:「你不曾理會得。古人有云:『珍奇玩好之物,不可使見貪婪奸偽之人。』
倘若一經入目,必動其心;既動其心,必生其計。汝是個畏禍的,索之而必應其
求,可也;不然,則殞身滅命,皆起於此,事不小矣。」行者道:「放心,放心
,都在老孫身上。」你看他不由分說,急急的走了去,把個包袱解開,早有霞光
迸迸,尚有兩層油紙裹定。去了紙,取出袈裟,抖開時,紅光滿室,彩氣盈庭。
眾僧見了,無一個不心歡口讚,真個好袈裟。上頭有:
    千般巧妙明珠墜,萬樣稀奇佛寶攢。
    上下龍鬚鋪綵綺,兜羅四面錦沿邊。
    體掛魍魎從此滅,身披魑魅入黃泉。
    托化天仙親手製,不是真僧不敢穿。

  
那老和尚見了這般寶貝,果然動了奸心,走上前,對三藏跪下,眼中垂淚道:
「我弟子真是沒緣。」三藏攙起道:「老院師有何話說?」他道:「老爺這件寶
貝方才展開,天色晚了,奈何眼目昏花,不能看得明白,豈不是無緣?」三藏教
:「掌上燈來,讓你再看。」那老僧道:「爺爺的寶貝已是光亮,再點了燈,一
發晃眼,莫想看得仔細。」行者道:「你要怎的看才好?」老僧道:「老爺若是
寬恩放心,教弟子拿到後房,細細的看一夜,明早送還老爺西去,不知尊意何如
?」三藏聽說,吃了一驚,埋怨行者道:「都是你,都是你。」行者笑道:「怕
他怎的?等我包起來,教他拿了去看。但有疏虞,盡是老孫管整。」那三藏阻當
不住,他把袈裟遞與老僧道:「憑你看去。只是明早照舊還我,不得損污些須。」
老僧喜喜歡歡,著幸童將袈裟拿進去。卻吩咐眾僧,將前面禪堂掃淨,取兩張籐
床,安設鋪蓋,請二位老爺安歇;一壁廂又教安排明早齋送行。遂而各散,師徒
們關了禪堂,睡下不題。
  
卻說那和尚把袈裟騙到手,拿在後房燈下,對袈裟號咷痛哭。慌得那本寺僧不敢
先睡。小幸童也不知為何,卻去報與眾僧道:「公公哭到二更時候,還不歇聲。」
有兩個徒孫是他心愛之人,上前問道:「師公,你哭怎的?」老僧道:「我哭無
緣,看不得唐僧寶貝。」小和尚道:「公公年紀高大,發過了。他的袈裟放在你
面前,你只消解開看便罷了,何須痛哭?」老僧道:「看的不長久。我今年二百
七十歲,空掙了幾百件袈裟。怎麼得有他這一件?怎麼得做個唐僧?」小和尚道
:「師公差了。唐僧乃是離鄉背井的一個行腳僧。你這等年高享用,也勾了,倒
要像他做行腳僧,何也?」老僧道:「我雖是坐家自在,樂乎晚景,卻不得他這
袈裟穿穿。若教我穿得一日兒,就死也閉眼,也是我來陽世間為僧一場。」眾僧
道:「好沒正經。你要穿他的,有何難處?我們明日留他住一日,你就穿他一日
;留他住十日,你就穿他十日;便罷了,何苦這般痛哭?」老僧道:「縱然留他
住了年載,也只穿得年載,到底也不得氣長。他要去時,只得與他去,怎生留得
長遠?」
  
正說話處,有一個小和尚,名喚廣智,出頭道:「公公要得長遠,也容易。」老
僧聞言,就歡喜起來道:「我兒,你有甚麼高見?」廣智道:「那唐僧兩個是走
路的人,辛苦之甚,如今已睡著了。我們想幾個有力量的,拿了槍刀,打開禪堂
,將他殺了,把屍首埋在後園,只我一家知道,卻又謀了他的白馬、行囊,卻把
那袈裟留下,以為傳家之寶,豈非子孫長久之計耶?」老和尚見說,滿心歡喜,
卻才揩了眼淚道:「好,好,好,此計絕妙。」即便收拾槍刀。
  
內中又有一個小和尚,名喚廣謀,就是那廣智的師弟,上前來道:「此計不妙。
若要殺他,須要看看動靜。那個白臉的似易,那個毛臉的似難,萬一殺他不得,
卻不反招己禍?我有一個不動刀槍之法,不知你尊意如何?」老僧道:「我兒,
你有何法?」廣謀道:「依小孫之見,如今喚聚東山大小房頭,每人要乾柴一束
,捨了那三間禪堂,放起火來,教他欲走無門,連馬一火焚之。就是山前山後人
家看見,只說是他自不小心,走了火,將我禪堂都燒了。那兩個和尚,卻不都燒
死?又好掩人耳目。袈裟豈不是我們傳家之寶?」那些和尚聞言,無不歡喜,都
道:「強,強,強,此計更妙,更妙。」遂教各房頭搬柴來。唉!這一計,正是
:弄得個高壽老僧該命盡,觀音禪院化為塵。原來他那寺裏有七八十個房頭,大
小有二百餘眾。當夜一擁搬柴,把個禪堂前前後後,四面圍繞不通,安排放火不
題。
  
卻說三藏師徒安歇已定。那行者卻是個靈猴,雖然睡下,只是存神煉氣,朦朧著
醒眼。忽聽得外面不住的人走,查查的柴響風生。他心疑惑道:「此時夜靜,如
何有人行得腳步之聲?莫敢是賊盜,謀害我們的?」他就一骨魯跳起,欲要開門
出看,又恐驚醒師父。你看他弄個精神,搖身一變,變做一個蜜蜂兒。真個是:
口甜尾毒,腰細身輕。穿花度柳飛如箭,粘絮尋香似落星。小小微軀能負重,囂
囂薄翅會乘風。卻自椽棱下,鑽出看分明。
  
只見那眾僧們搬柴運草,已圍住禪堂放火哩。行者暗笑道:「果依我師父之言,
他要害我們性命,謀我的袈裟,故起這等毒心。我待要拿棍打他呵,可憐又不禁
打,一頓棍都打死了,師父又怪我行兇。罷,罷,罷,與他個順手牽羊,將計就
計,教他住不成罷!」
  
好行者,一觔斗跳上南天門裏。諕得個龐、劉、苟、畢躬身,馬、趙、溫、關控
背,俱道:「不好了,不好了!那鬧天宮的主子又來了。」行者搖著手道:「列
位免禮,休驚。我來尋廣目天王的。」說不了,卻遇天王早到,迎著行者道:
「久闊,久闊。前聞得觀音菩薩來見玉帝,借了四值功曹、六丁六甲並揭諦等,
保護唐僧往西天取經去,說你與他做了徒弟,今日怎麼得閑到此?」行者道:
「且休敘闊。唐僧路遇歹人,放火燒他,事在萬分緊急,特來尋你借辟火罩兒,
救他一救。快些拿來使使,即刻返上。」天王道:「你差了。既是歹人放火,只
該借水救他,如何要辟火罩?」行者道:「你那裏曉得就裏。借水救之,卻燒不
起來,倒相應了他;只是借此罩,護住了唐僧無傷,其餘管他,盡他燒去。快些
,快些,此時恐已無及,莫誤了我下邊幹事。」那天王笑道:「這猴子還是這等
起不善之心,只顧了自家,就不管別人。」行者道:「快著,快著,莫要調嘴,
害了大事。」那天王不敢不借,遂將罩兒遞與行者。
  
行者拿了,按著雲頭,徑到禪堂房脊上,罩住了唐僧與白馬、行李。他卻去那後
面老和尚住的方丈房上頭坐著,保護那袈裟。看那些人放起火來,他轉捻訣念咒
,望巽地上吸一口氣吹將去,一陣風起,把那火轉吹得烘烘亂發。好火,好火!
但見:
黑煙漠漠,紅燄騰騰。黑煙漠漠,長空不見一天星;紅燄騰騰,大地有光千里赤
。起初時,灼灼金蛇;次後來,威威血馬。南方三?逞英雄,回祿大神施法力。
燥乾柴燒烈火性,說甚麼燧人鑽木;熱油門前飄彩燄,賽過了老祖開爐。正是那
無情火發,怎禁這有意行兇。不去弭災,反行助虐。風隨火勢,燄飛有千丈餘高
;火逞風威,灰迸上九霄雲外。乒乒乓乓,好便似殘年爆竹;潑潑喇喇,卻就如
軍中炮聲。燒得那當場佛像莫能逃,東院伽藍無處躲。勝如赤壁夜鏖兵,賽過阿
房宮內火。
這正是星星之火,能燒萬頃之田。須臾間,風狂火盛,把一座觀音院,處處通紅
。你看那眾和尚,搬箱抬籠,搶桌端鍋,滿院裏叫苦連天。孫行者護住了後邊方
丈,辟火罩罩住了前面禪堂,其餘前後火光大發,真個是照天紅燄輝煌,透壁金
光照耀。
  
不期火起之時,驚動了一山獸怪。這觀音院正南二十里遠近,有座黑風山,山中
有一個黑風洞,洞中有一個妖精,正在睡醒翻身。只見那窗間透亮,只道是天明
。起來看時,卻是正北下的火光晃亮。妖精大驚道:「呀!這必是觀音院裏失了
火。這些和尚好不小心。我看時,與他救一救來。」好妖精,縱起雲頭,即至煙
火之下,果然沖天之火,前面殿宇皆空,兩廊煙火方灼。他大拽步,撞將進去,
正呼喚叫取水來,只見那後房無火,房脊上有一人放風。他卻情知如此,急入裏
面看時,見那方丈中間有些霞光彩氣,臺案上有一個青氈包袱。他解開一看,見
是一領錦襴袈裟,乃佛門之異寶。正是財動人心,他也不救火,他也不叫水,拿
著那袈裟,趁鬨打劫,拽回雲步,徑轉東山而去。
  
那場火只燒到五更天明,方才滅息。你看那眾僧們赤赤精精,啼啼哭哭,都去那
灰內尋銅鐵,撥腐炭,撲金銀。有的在牆筐裏,苫搭窩棚;有的赤壁根頭,支鍋
造飯。叫冤叫屈,亂嚷亂鬧不題。
  
卻說行者取了辟火罩,一觔斗送上南天門,交與廣目天王道:「謝借,謝借。」
天王收了道:「大聖至誠了。我正愁你不還我的寶貝,無處尋討,且喜就送來也
。」行者道:「老孫可是那當面騙物之人?這叫做『好借好還,再借不難』。」
天王道:「許久不面,請到宮少坐一時,何如?」行者道:「老孫比在前不同,
爛板凳,高談闊論了;如今保唐僧,不得身閑。容敘,容敘。」急辭別墜雲,又
見那太陽星上。徑來到禪堂前,搖身一變,變做個蜜蜂兒,飛將進去,現了本像
看時,那師父還沉睡哩。
  
行者叫道:「師父,天亮了,起來罷。」三藏才醒覺,翻身道:「正是。」穿了
衣服,開門出來,忽抬頭,只見些倒壁紅牆,不見了樓臺殿宇。大驚道:「呀!
怎麼這殿宇俱無,都是紅牆,何也?」行者道:「你還做夢哩,今夜走了火的。」
三藏道:「我怎不知?」行者道:「是老孫護了禪堂,見師父濃睡,不曾驚動。」
三藏道:「你有本事護了禪堂,如何就不救別房之火?」行者笑道:「好教師父
得知:果然依你昨日之言,他愛上我們的袈裟,算計要燒殺我們。若不是老孫知
覺,到如今皆成灰骨矣。」三藏聞言,害怕道:「是他們放的火麼?」行者道:
「不是他是誰?」三藏道:「莫不是怠慢了你,你幹的這個勾當?」行者道:
「老孫是這等憊懶之人,幹這等不良之事?實實是他家放的。老孫見他心毒,果
是不曾與他救火,只是與他略略助些風的。」三藏道:「天那,天那!火起時,
只該助水,怎轉助風?」行者道:「你可知古人云:『人沒傷虎心,虎沒傷人意
。』他不弄火,我怎肯弄風?」三藏道:「袈裟何在?敢莫是燒壞了也?」行者
道:「沒事,沒事,燒不壞,那放袈裟的方丈無火。」三藏恨道:「我不管你,
但是有些兒傷損,我只把那話兒念動念動,你就是死了。」行者慌了道:「師父
莫念,莫念,管尋還你袈裟就是了。等我去拿來走路。」三藏才牽著馬,行者挑
了擔,出了禪堂,徑往後方丈去。
  
卻說那些和尚正悲切間,忽的看見他師徒牽馬挑擔而來,諕得一個個魂飛魄散道
:「冤魂索命來了。」行者喝道:「甚麼冤魂索命?快還我袈裟來。」眾僧一齊
跪倒,叩頭道:「爺爺呀,冤有冤家,債有債主。要索命不干我們事,都是廣謀
與老和尚定計害你的,莫問我們討命。」行者咄的一聲道:「我把你這些該死的
畜生,那個問你討甚麼命。只拿袈裟來還我走路!」其間有兩個膽量大的和尚道
:「老爺,你們在禪堂裏已燒死了,如今又來討袈裟,端的還是人,是鬼?」行
者笑道:「這夥孽畜,那裏有甚麼火來?你去前面看看禪堂,再來說話。」眾僧
們爬起來往前觀看,那禪堂外面的門窗?扇,更不曾燎灼了半分。眾人悚懼,才
認得三藏是種神僧,行者是尊護法。一齊上前叩頭道:「我等有眼無珠,不識真
人下界。你的袈裟在後面方丈中老師祖處哩。」三藏行過了三五層敗壁破牆,嗟
嘆不已。只見方丈果然無火,眾僧搶入裏面,叫道:「公公,唐僧乃是神人,未
曾燒死,如今反害了自己家當。趁早拿出袈裟,還他去也。」
  
原來這老和尚尋不見袈裟,又燒了本寺的房屋,正在萬分煩惱焦燥之處,一聞此
言,怎敢答應。因尋思無計,進退無方,拽開步,躬著腰,往那牆上著實撞了一
頭,可憐只撞得腦破血流魂魄散,咽喉氣斷染紅沙。有詩為證。詩曰:
    堪嘆老衲性愚蒙,枉作人間一壽翁。
    欲得袈裟傳遠世,豈知佛寶不凡同。
    但將容易為長久,定是蕭條取敗功。
    廣智廣謀成甚用?損人利己一場空。

  
慌得個眾僧哭道:「師公已撞殺了,又不見袈裟,怎生是好?」行者道:「想是
汝等盜藏起也。都出來,開具花名手本,等老孫逐一查點。」那上下房的院主,
將本寺和尚、頭陀、幸童、道人盡行開具手本二張,大小人等共計二百三十名。
行者請師父高坐,他卻一一從頭唱名搜檢,都要解放衣襟,分明點過,更無袈裟
。又將那各房頭搬搶出去的箱籠物件,從頭細細尋遍,那裏得有蹤跡。三藏心中
煩惱,懊恨行者不盡,卻坐在上面念動那咒。行者撲的跌倒在地,抱著頭,十分
難禁,只教:「莫念,莫念,管尋還了袈裟。」那眾僧見了,一個個戰兢兢的,
上前跪下勸解,三藏才合口不念。行者一骨魯跳起來,耳朵裏掣出鐵棒,要打那
些和尚,被三藏喝住道:「這猴頭,你頭痛還不怕,還要無禮?休動手,且莫傷
人,再與我審問一問。」眾僧們磕頭禮拜,哀告三藏道:「老爺饒命。我等委實
的不曾看見。這都是那老死鬼的不是。他昨晚看著你的袈裟,只哭到更深時候,
看也不曾敢看,思量要圖長久,做個傳家之寶,設計定策,要燒殺老爺。自火起
之候,狂風大作,各人只顧救火,搬搶物件,更不知袈裟去向。」
  
行者大怒,走進方丈屋裏,把那觸死鬼屍首抬出,選剝了細看,渾身更無那件寶
貝。就把個方丈掘地三尺,也無蹤影。行者忖量半晌,問道:「你這裏可有甚麼
妖怪成精麼?」院主道:「老爺不問,莫想得知。我這裏正東南有座黑風山,黑
風洞內有一個黑大王,我這老死鬼常與他講道,他便是個妖精。別無甚物。」行
者道:「那山離此有多遠近?」院主道:「只有二十里,那望見山頭的就是。」
行者笑道:「師父放心,不須講了,一定是那黑怪偷去無疑。」三藏道:「他那
廂離此有二十里,如何就斷得是他?」行者道:「你不曾見夜間那火,光騰萬里
,亮透三天,且休說二十里,就是二百里也照見了。坐定是他見火光焜耀,趁著
機會,暗暗的來到這裏,看見我們袈裟是件寶貝,必然趁鬨擄去也。等老孫去尋
他一尋。」三藏道:「你去了時,我卻何倚?」行者道:「這個放心,暗中自有
神靈保護,明中等我叫那些和尚伏侍。」即喚眾和尚過來,道:「汝等著幾個去
埋那老鬼;著幾個伏侍我師父,看守我白馬。」眾僧領諾。行者又道:「汝等莫
順口兒答應,等我去了,你就不來奉承。看師父的,要怡顏悅色;養白馬的,要
水草調勻。假有一毫兒差了,照依這個樣棍,與你們看看。」他掣出棍子,照那
火燒的磚牆上,撲的一下,把那牆打得粉碎,又震倒了有七八層牆。眾僧見了,
個個骨軟身麻,跪著磕頭滴淚道:「爺爺寬心前去,我等竭力虔心,供奉老爺,
決不敢一毫怠慢。」
  
好行者,急縱觔斗雲,徑上黑風山,尋找這袈裟。正是那:
    金禪求正出京畿,仗錫投西涉翠微。
    虎豹狼蟲行處有,工商士客見時稀。
    路逢異國愚僧妒,全仗齊天大聖威。
    火發風生禪院廢,黑熊夜盜錦襴衣。

    畢竟此去不知袈裟有無,吉凶如何,且聽下回分解。





}  \end{pinyinscope}\switchcolumn{\myfontc \section{第 十 六 回} 观 音 院 的 僧 人 谋 宝 贝 , 黑 风 山 奇 怪 地 窃 取 袈 裟 , 又 说 他 的 师 徒 两 个 人 策 马 前 来 , 径 直 到 山 门 口 观 看 , 果 然 是 一 座 寺 院 。
只 见 那 里 , 层 层 的 殿 阁 , 层 层 叠 叠 的 廊 房 。
三 山 门 外 , 巍 巍 万 道 彩 云 遮 住 ; 五 福 堂 前 , 艳 丽 千 条 红 雾 缭 绕 。
两 路 松 竹 , 一 林 松 柏 。
两 路 松 竹 , 无 年 无 纪 , 自 然 清 幽 ; 一 林 松 柏 , 有 色 有 颜 , 随 着 傲 慢 华 丽 。
又 见 那 钟 鼓 楼 高 , 佛 塔 高 峻 。
安 禅 僧 定 性 , 啼 树 鸟 声 闲 。
寂 寞 无 尘 真 寂 寞 , 清 虚 有 道 果 清 虚 。
诗 说 : 上 刹 祇 园 隐 翠 窝 , 招 提 胜 景 赛 娑 婆 。
果 然 , 净 土 人 间 少 , 天 下 名 山 僧 人 占 据 多 。
长 老 下 了 马 , 行 人 歇 了 担 子 , 正 想 进 门 , 只 见 那 门 里 走 出 一 群 和 尚 。
你 看 他 怎 么 样 子 ? 他 头 戴 左 帽 , 身 穿 无 垢 衣 。
铜 环 双 坠 耳 , 绢 带 束 腰 围 。
草 履 走 来 稳 稳 , 木 鱼 手 里 提 着 。
口 中 常 常 念 念 , 般 若 总 是 皈 依 。
三 藏 见 了 , 就 站 在 门 旁 , 向 他 讲 了 一 个 问 讯 。
那 和 尚 连 忙 回 礼 , 笑 着 说 : 失 望 。
又 问 : 是 从 哪 里 来 的 ? 请 进 去 方 丈 进 献 茶 叶 。
三 藏 菩 萨 说 : 我 的 弟 子 是 东 土 的 钦 差 , 到 雷 音 寺 拜 佛 求 经 。
到 此 处 天 色 将 晚 , 想 借 上 刹 一 夜 。
那 和 尚 说 : 请 进 里 坐 , 请 进 里 坐 。
三 藏 正 叫 来 行 人 , 牵 着 马 进 来 。
那 和 尚 忽 然 看 见 行 人 的 相 貌 , 有 些 害 怕 , 就 问 : 那 牵 马 的 是 什 么 东 西 ?
其 性 急 躁 , 如 果 听 见 你 说 的 是 什 么 东 西 , 他 就 很 恼 怒 。
他 是 我 的 弟 弟 。
那 和 尚 打 了 个 寒 冷 , 咬 着 指 头 说 : 这 样 一 个 丑 头 怪 脑 , 好 招 他 做 徒 弟 , 三 藏 说 : 你 看 不 出 来 哩 , 丑 自 己 丑 , 很 是 有 用 。
和 尚 只 得 与 三 藏 和 行 人 一 起 走 进 山 门 。
山 门 里 , 又 见 那 正 殿 上 写 着 四 个 大 字 , 是 观 音 禅 院 。
三 藏 菩 萨 又 大 喜 说 : 弟 子 多 次 感 激 菩 萨 的 圣 恩 , 没 来 得 及 叩 谢 。
现 在 遇 到 禅 院 , 就 好 像 见 到 菩 萨 一 样 , 很 好 地 拜 谢 。
那 和 尚 听 了 , 立 即 命 令 道 人 打 开 殿 门 , 请 三 藏 朝 拜 。
那 个 行 人 系 了 马 , 丢 了 行 李 , 和 三 藏 一 起 上 殿 。
三 藏 展 开 背 后 舒 展 身 体 , 铺 满 胸 襟 , 放 在 地 上 , 望 着 金 像 叩 头 。
那 和 尚 就 去 打 鼓 。
行 人 就 去 撞 钟 。
三 藏 俯 伏 在 台 前 , 倾 心 祈 祷 。
祝 祷 完 毕 , 那 和 尚 停 下 了 鼓 , 走 路 的 人 还 只 管 敲 钟 不 停 , 有 时 紧 有 时 慢 , 敲 了 很 久 。
那 个 道 人 说 : 拜 已 完 了 , 还 撞 钟 又 怎 么 样 ? 行 人 刚 丢 了 钟 杵 , 笑 着 说 : 你 哪 里 知 道 , 我 是 做 一 天 和 尚 撞 一 天 钟 的 。
这 时 又 惊 动 了 那 个 寺 里 的 大 小 和 尚 、 上 下 房 的 长 老 , 听 到 钟 声 乱 响 , 一 齐 拥 出 来 说 : 那 个 野 人 在 这 里 乱 敲 钟 鼓 , 行 人 跳 出 来 , 呵 斥 一 声 说 : 是 你 孙 外 公 撞 了 弄 子 的 。
那 些 和 尚 一 见 了 , 就 吓 得 跌 跌 跳 跃 , 都 爬 在 地 下 , 说 : 雷 公 爷 爷 !
起 来 , 起 来 , 不 要 害 怕 , 我 们 是 从 东 方 大 唐 来 的 老 爷 。
众 位 和 尚 方 才 行 礼 拜 。
见 了 三 藏 , 都 放 心 不 怕 。
寺 内 有 个 寺 院 主 人 请 求 说 : 老 爷 们 到 后 方 丈 中 奉 茶 。
于 是 解 开 马 缰 牵 着 马 , 抬 起 行 李 , 转 过 正 殿 , 径 直 进 入 后 房 , 依 次 排 列 坐 次 。
院 主 献 茶 , 又 安 排 斋 供 。
天 光 还 早 , 三 藏 还 没 说 完 , 只 见 他 的 后 面 有 两 个 小 孩 , 扶 着 一 个 老 和 尚 出 来 。
看 他 的 儿 子 , 头 上 戴 着 一 顶 袈 裟 , 顶 上 戴 着 一 顶 宝 镜 , 头 上 戴 着 一 个 宝 顶 光 亮 , 身 上 穿 着 锦 绒 衫 , 翡 翠 毛 的 金 边 光 亮 。
一 对 和 尚 的 鞋 子 堆 满 八 宝 , 一 根 拄 杖 镶 嵌 着 云 星 。
满 脸 皱 着 的 痕 迹 , 好 像 山 老 母 , 一 双 眼 睛 , 却 像 东 海 龙 君 。
口 不 关 风 , 因 为 牙 齿 落 落 , 腰 驼 背 屈 , 筋 挛 。
众 僧 说 : 师 祖 来 了 。
三 藏 亲 自 施 礼 迎 接 道 : 老 院 主 , 弟 子 拜 揖 。
那 个 老 和 尚 回 去 了 礼 , 又 各 自 叙 说 坐 下 。
老 和 尚 说 : 刚 才 小 的 人 说 , 东 土 唐 朝 来 的 老 爷 , 我 才 出 来 奉 见 。
三 藏 说 : 轻 率 地 造 宝 山 , 不 知 好 歹 , 恕 罪 , 恕 罪 。
老 和 尚 说 : 不 敢 , 不 敢 。
于 是 问 : 老 爷 , 东 土 到 这 里 , 有 多 少 路 程 ? 三 藏 说 : 从 长 安 边 界 出 来 , 有 五 千 多 里 。
经 过 两 界 山 , 收 拢 了 一 群 小 徒 , 一 路 来 , 走 过 西 番 哈 国 , 经 过 两 个 月 , 又 走 了 五 六 千 里 , 才 到 达 贵 州 。
老 和 尚 说 : 也 有 万 里 之 遥 了 。
我 弟 子 虚 度 一 生 , 在 山 门 里 也 不 曾 出 去 , 真 是 所 谓 坐 井 观 天 , 朽 朽 之 辈 。
三 藏 又 问 : 老 院 主 的 寿 命 有 多 少 ? 老 僧 说 : 痴 子 长 二 百 七 十 岁 了 。
行 人 听 见 后 说 : 这 还 是 我 万 代 的 孙 子 啊 !
三 藏 看 了 他 一 眼 , 说 : 谨 慎 地 说 , 不 要 不 认 识 高 低 , 冲 撞 别 人 。
那 和 尚 便 问 道 : 老 爷 , 你 有 多 少 年 纪 了 ? 行 人 说 : 不 敢 说 。
那 个 老 僧 也 只 是 一 句 狂 话 , 便 不 介 意 , 也 不 再 问 , 只 叫 献 茶 。
又 有 一 个 小 幸 童 , 拿 出 一 个 羊 脂 玉 的 盘 子 , 还 有 三 个 法 蓝 钉 金 的 茶 钟 。
又 有 一 个 童 子 , 提 着 一 把 白 铜 瓶 , 喝 了 三 杯 香 茶 。
真 的 是 颜 色 比 荔 花 艳 丽 , 味 道 胜 过 桂 花 香 。
三 藏 见 了 , 夸 耀 喜 爱 不 尽 说 : 好 物 品 , 好 物 品 , 真 是 美 食 美 器 。
那 个 老 和 尚 说 : 污 眼 , 玷 眼 。
老 爷 是 天 朝 上 国 , 广 泛 地 览 览 奇 珍 异 宝 , 像 这 样 的 器 具 , 哪 里 值 得 夸 奖 老 爷 从 上 邦 来 , 能 有 什 么 宝 贝 , 借 给 弟 子 看 一 看 吧 。 三 藏 说 : 可 怜 , 我 那 东 土 没 有 什 么 宝 贝 ; 即 使 有 时 , 路 程 遥 远 , 也 不 能 带 得 。
行 者 在 旁 边 说 : 师 父 , 我 前 日 在 包 袱 里 , 曾 见 过 那 件 袈 裟 , 不 是 一 件 宝 贝 , 拿 给 他 看 看 怎 么 样 ? 众 僧 听 了 袈 裟 , 一 个 个 都 冷 笑 。
行 者 说 : 你 笑 什 么 ? 院 主 说 : 老 爷 才 说 袈 裟 是 一 件 宝 贝 , 说 的 确 可 笑 。
如 果 说 袈 裟 , 像 我 等 人 的 , 不 止 二 三 十 件 ; 如 果 说 我 师 祖 , 在 此 地 做 了 二 百 五 六 十 年 和 尚 , 足 有 七 八 百 件 。
又 叫 道 : 拿 出 来 看 看 。
又 叫 道 人 打 开 库 房 , 头 陀 抬 了 柜 子 , 就 抬 出 十 二 个 柜 子 , 放 在 天 井 里 , 打 开 锁 链 。
两 边 摆 下 衣 架 , 四 周 牵 着 绳 子 , 将 袈 裟 一 一 点 抖 开 挂 起 来 , 请 三 藏 观 看 。
果 然 是 满 堂 绮 绣 , 四 壁 绫 罗 。
行 人 一 一 看 了 , 都 是 穿 花 纳 锦 , 刺 绣 销 金 的 东 西 。 笑 着 说 : 好 , 好 , 好 , 好 , 好 , 好 。
收 起 来 , 收 起 来 。
把 我 们 的 也 取 出 来 看 。
三 藏 把 行 人 拉 住 , 悄 悄 地 说 : 你 , 不 要 和 别 人 争 富 。
你 我 是 单 身 在 外 , 只 恐 怕 有 错 误 。
行 者 说 : 看 看 袈 裟 , 有 什 么 差 错 ? 三 藏 说 : 你 不 曾 理 会 得 到 。
古 人 有 这 样 说 : 珍 奇 玩 好 的 东 西 , 不 能 让 人 见 到 贪 婪 奸 伪 的 人 。
如 果 一 旦 入 眼 , 就 必 定 动 摇 他 的 心 , 既 已 动 摇 他 的 心 , 就 必 定 产 生 计 谋 。
你 是 害 怕 祸 患 的 人 , 索 要 他 们 一 定 要 求 , 可 以 ; 不 然 , 我 们 就 会 丢 掉 自 己 的 性 命 , 都 是 由 于 这 里 产 生 的 , 事 情 不 小 了 。
行 人 说 : 放 心 , 放 心 , 都 在 老 孙 子 身 上 。
你 看 他 不 由 分 说 , 急 急 忙 忙 地 走 了 , 把 包 袱 解 开 , 早 晨 有 霞 光 迸 裂 , 还 有 两 层 油 纸 包 住 。
拿 出 袈 裟 , 抖 开 时 , 红 光 满 屋 , 彩 气 满 院 。
众 僧 见 了 , 没 有 一 个 不 心 欢 口 赞 , 真 是 好 袈 裟 。
上 头 有 : 千 般 巧 妙 明 珠 坠 , 万 种 稀 奇 佛 宝 聚 集 。
上 下 龙 须 铺 上 彩 色 的 绸 缎 , 兜 罗 四 面 锦 缎 。
身 上 挂 着 魍 从 此 灭 亡 , 身 上 披 着 妖 魅 入 黄 泉 。
假 托 化 天 仙 亲 手 制 作 , 不 是 真 僧 , 不 敢 穿 。
那 老 和 尚 见 了 这 样 的 宝 贝 , 果 然 动 了 奸 心 , 跑 上 前 , 对 着 三 藏 跪 下 , 眼 中 流 泪 说 : 我 的 弟 子 真 是 没 有 缘 份 。
三 藏 搀 着 起 来 说 : 老 院 师 有 什 么 话 说 ? 他 说 : 老 爷 这 件 宝 贝 刚 刚 展 开 , 天 色 晚 了 , 为 什 么 眼 睛 昏 花 , 不 能 看 得 明 白 , 难 道 不 是 没 有 缘 缘 吗 ? 三 藏 教 他 说 : 掌 上 灯 来 , 让 你 再 看 。
那 个 老 和 尚 说 : 你 的 宝 贝 已 经 光 亮 了 , 再 点 点 点 点 点 点 点 点 了 灯 , 一 发 动 了 眼 睛 , 不 要 想 看 得 仔 细 。
老 僧 说 : 老 爷 如 果 是 宽 恩 放 心 , 让 弟 子 拿 到 后 房 , 细 细 地 看 一 夜 , 明 早 送 还 老 爷 西 去 , 不 知 你 的 意 思 怎 么 样 ? 三 藏 听 了 , 吃 了 一 惊 , 埋 怨 和 尚 说 : 都 是 你 , 都 是 你 。
行 者 笑 着 说 : 怕 他 怎 么 样 , 等 我 包 起 来 , 让 他 拿 了 去 看 。
只 要 有 疏 远 忧 虑 , 全 是 老 孙 管 整 。
三 藏 阻 挡 不 住 , 他 把 袈 裟 送 给 老 僧 说 : 凭 你 看 去 。
只 是 明 早 照 旧 还 我 , 不 要 损 害 我 的 胡 须 。
老 和 尚 喜 欢 , 穿 着 宠 幸 的 小 孩 , 把 袈 裟 拿 进 去 。
又 叫 众 位 僧 人 , 把 前 面 的 禅 堂 扫 净 , 拿 出 两 张 藤 床 , 安 排 铺 盖 , 请 二 位 老 人 安 歇 。 另 一 个 墙 厢 , 又 教 他 们 安 排 明 早 斋 戒 送 行 。
于 是 就 各 自 散 去 了 , 师 徒 们 都 关 上 了 禅 堂 , 睡 下 不 再 问 题 。
又 说 : 那 和 尚 把 袈 裟 骗 到 手 , 拿 在 后 房 灯 下 , 对 袈 裟 大 声 痛 哭 。
惊 恐 得 那 本 寺 的 僧 人 , 不 敢 先 睡 。
小 幸 童 不 知 道 是 什 么 , 便 去 报 告 给 众 僧 说 : 公 公 哭 到 二 更 时 候 , 还 没 有 停 声 。
有 两 个 弟 子 , 是 他 心 爱 的 人 , 上 前 问 道 : 师 公 , 你 哭 什 么 呀 老 和 尚 说 : 我 哭 没 缘 由 , 看 不 到 唐 僧 的 宝 贝 。
小 和 尚 说 : 公 公 您 年 纪 高 大 , 发 已 经 出 发 了 。
他 的 袈 裟 放 在 你 面 前 , 你 只 需 解 开 看 就 罢 了 , 何 必 痛 哭 ? 老 僧 说 : 看 的 不 长 久 。
我 今 年 二 百 七 十 岁 , 空 得 几 百 件 袈 裟 。
为 什 么 还 有 这 样 的 事 , 怎 么 能 做 个 唐 僧 呢 ? 小 和 尚 说 : 师 公 好 了 。
唐 僧 , 是 离 乡 背 井 的 一 个 行 脚 僧 。
你 们 这 些 年 龄 已 高 享 用 , 也 勾 了 , 倒 要 像 他 做 行 脚 僧 , 为 什 么 ? 老 和 尚 说 : 我 虽 然 是 坐 在 家 里 自 在 , 享 受 晚 景 快 乐 , 却 得 不 到 他 的 袈 裟 穿 穿 。
如 果 教 我 穿 得 一 天 儿 子 , 就 死 也 闭 眼 , 也 是 我 来 到 阳 世 间 作 和 尚 一 场 。
众 僧 说 : 好 没 正 经 。
你 要 穿 他 的 , 又 有 什 么 难 处 呢 我 们 明 天 留 他 住 一 天 , 你 就 穿 他 一 天 ; 留 他 住 十 天 , 你 就 穿 他 十 天 ; 就 算 罢 了 , 何 苦 这 样 痛 哭 ? 老 和 尚 说 : 纵 然 留 他 住 了 年 载 , 也 只 穿 得 年 载 , 到 底 也 不 得 气 长 。
他 要 离 开 时 , 只 得 与 他 去 , 怎 么 能 留 得 长 远 呢 ? 正 在 说 话 的 地 方 , 有 一 个 叫 广 智 , 出 头 说 : 公 公 要 长 远 , 也 很 容 易 。
老 僧 听 了 这 话 , 欢 喜 起 来 说 : 我 儿 子 , 你 有 什 么 高 见 啊 广 智 说 : 那 两 个 是 走 路 的 人 , 十 分 辛 苦 , 现 在 已 经 睡 着 了 。
我 们 想 几 个 有 力 量 的 人 , 拿 着 刀 剑 , 打 开 禅 堂 , 把 他 杀 了 , 把 尸 首 埋 在 后 园 , 只 有 我 一 家 人 知 道 , 却 又 谋 划 他 的 白 马 行 囊 , 却 把 那 袈 裟 留 下 来 , 作 为 传 家 的 宝 贝 , 难 道 不 是 子 孙 长 久 的 计 策 吗 ? 老 和 尚 见 了 , 满 心 欢 喜 , 只 是 揩 了 眼 泪 说 : 好 , 好 , 这 个 计 策 是 最 妙 的 。
于 是 就 收 拾 了 刀 枪 。
其 中 又 有 一 个 小 和 尚 , 名 叫 广 谋 , 就 是 那 广 智 的 师 弟 。
如 果 杀 了 他 , 必 须 看 看 他 的 动 静 。
那 个 白 面 的 人 似 乎 容 易 , 那 个 毛 面 的 人 似 乎 难 , 万 一 杀 他 不 得 , 却 不 反 而 招 来 自 己 的 灾 祸 , 我 有 一 个 不 动 刀 枪 之 法 , 不 知 你 的 尊 严 意 思 是 怎 么 样 的 ? 老 和 尚 说 : 我 儿 子 , 你 有 什 么 办 法 ? 广 谋 说 : 依 照 我 的 看 法 , 今 天 我 召 集 在 东 山 大 小 房 头 , 每 人 要 干 柴 一 捆 , 舍 下 了 那 三 间 禅 堂 , 放 火 来 , 教 他 想 走 无 门 , 连 马 一 火 烧 掉 。
就 是 山 前 山 后 人 家 看 见 , 只 说 是 他 自 己 不 小 心 , 走 了 火 , 把 我 的 禅 堂 都 烧 掉 了 。
那 两 个 和 尚 , 不 都 烧 死 , 又 好 掩 盖 别 人 的 耳 目 。
袈 裟 难 道 不 是 我 们 传 家 的 宝 贝 吗 ? 那 些 和 尚 听 了 , 无 不 欢 喜 , 都 说 : 努 力 , 强 , 这 个 计 策 更 妙 , 更 妙 。
于 是 就 让 各 房 头 搬 柴 来 。
唉 , 这 一 个 计 策 , 正 是 说 : 弄 得 个 高 寿 老 僧 该 命 尽 , 观 音 禅 院 化 为 尘 土 。
原 来 他 那 寺 里 有 七 八 十 个 房 子 , 大 小 有 二 百 多 人 。
当 天 晚 上 一 个 人 簇 拥 着 搬 柴 , 把 那 个 禅 堂 前 前 后 后 , 四 面 围 绕 不 通 , 安 排 放 火 不 题 。
回 答 说 : 三 藏 师 徒 安 歇 已 经 定 下 来 。
那 个 行 人 却 是 个 灵 猴 , 虽 然 睡 了 , 只 是 保 存 精 神 炼 气 , 糊 涂 着 醒 了 眼 睛 。
忽 然 听 到 外 面 不 住 的 人 走 , 一 声 一 声 一 响 。
他 心 怀 疑 惑 , 说 : 此 时 夜 静 , 为 什 么 有 人 走 路 的 声 音 , 没 有 人 敢 说 是 盗 贼 , 谋 害 我 们 的 。
你 看 他 玩 弄 精 神 , 摇 动 身 体 一 变 , 变 成 一 个 蜜 蜂 儿 。
真 的 是 : 口 味 尾 毒 , 腰 细 身 轻 。
穿 花 度 柳 飞 如 箭 , 粘 絮 寻 香 似 落 星 。
小 小 的 小 小 的 小 小 的 身 躯 能 负 重 , 浩 浩 荡 荡 的 翅 膀 会 乘 风 。
又 从 椽 子 的 棱 角 下 面 , 钻 出 来 看 , 分 明 。
只 见 那 些 和 尚 们 搬 柴 运 草 , 已 围 住 禅 堂 放 火 了 。
行 者 暗 笑 着 说 : 果 然 依 照 我 师 父 的 话 , 他 要 害 我 们 的 性 命 , 因 为 谋 害 我 的 袈 裟 , 所 以 才 产 生 这 种 毒 心 。
我 要 拿 棍 子 打 他 呵 , 可 怜 又 不 禁 打 , 一 顿 棍 子 都 打 死 了 , 师 父 又 怪 我 行 凶 。
罢 , 罢 , 罢 , 和 他 个 顺 手 牵 羊 , 将 计 就 计 , 教 他 住 不 成 罢 了 。 好 行 的 人 , 一 个 斗 跳 上 南 天 门 里 。
何 况 得 到 庞 、 刘 、 苟 、 毕 躬 身 , 马 、 赵 、 温 、 关 、 关 、 控 都 说 : 不 好 了 , 不 好 了 , 那 闹 天 宫 的 主 子 又 来 了 。
行 者 摇 着 手 说 : 列 位 免 去 礼 仪 , 不 要 惊 慌 。
我 来 寻 找 广 目 天 王 的 。
说 不 完 , 忽 然 遇 到 天 王 早 到 , 迎 着 行 人 说 : 久 远 , 久 阔 。
以 前 听 说 有 个 观 音 菩 萨 来 见 玉 帝 , 借 了 四 位 功 曹 、 六 丁 六 甲 和 揭 谛 等 人 , 保 护 唐 僧 到 西 天 去 取 经 , 说 你 与 他 做 了 徒 弟 , 今 天 怎 么 能 闲 到 这 里 呢 ? 行 人 说 : 暂 且 不 要 谈 话 。
唐 僧 在 路 上 遇 到 一 个 坏 人 , 放 火 烧 他 , 事 情 在 万 分 紧 急 , 特 来 找 你 借 用 辟 火 罩 儿 , 救 他 一 个 救 。
快 点 拿 来 使 者 , 立 刻 返 回 上 去 。
天 王 说 : 你 好 了 。
既 然 是 坏 人 放 火 , 只 应 借 水 救 他 , 为 什 么 要 避 开 火 罩 呢 ? 行 人 说 : 你 哪 里 知 道 就 在 里 面 。
借 水 救 火 , 火 烧 不 起 来 , 倒 相 呼 应 了 他 , 只 是 借 此 罩 住 了 唐 僧 没 有 伤 害 , 其 余 的 管 它 , 全 都 烧 掉 了 。
快 些 , 快 些 , 此 时 恐 怕 已 来 不 及 了 , 别 错 了 我 下 边 干 事 。
那 天 王 笑 着 说 : 这 个 猴 子 还 是 这 样 产 生 不 好 的 念 头 , 只 顾 自 己 家 , 就 不 管 别 人 了 。
行 人 说 : 快 着 , 快 着 , 不 要 调 舌 , 害 了 大 事 。
那 天 王 不 敢 不 借 , 就 把 罩 儿 送 给 了 行 人 。
行 人 拿 了 , 按 着 云 头 , 径 直 走 到 禅 堂 房 脊 上 , 把 唐 僧 和 白 马 、 行 李 笼 住 。
其 后 有 一 个 老 和 尚 住 在 方 丈 房 里 , 还 有 一 个 人 保 护 他 的 袈 裟 。
其 他 人 都 不 知 道 , 不 知 其 他 人 也 。
喜 好 火 , 喜 好 火 , 只 见 到 一 股 黑 烟 漠 漠 , 红 焰 腾 腾 。
黑 烟 漠 漠 , 长 空 看 不 见 一 颗 天 星 ; 红 色 的 火 焰 腾 腾 , 大 地 上 有 光 芒 , 千 里 之 外 就 会 发 出 红 色 。
起 初 的 时 候 , 金 蛇 闪 烁 , 后 来 的 时 候 , 血 马 威 猛 。
南 方 三 逞 英 雄 , 回 禄 大 神 施 法 力 。
干 燥 干 柴 烧 烈 火 的 本 性 , 说 什 么 燧 人 钻 木 , 烧 油 门 前 飘 彩 焰 , 比 赛 过 了 老 祖 开 炉 。
正 是 那 无 情 火 发 , 怎 么 能 禁 止 这 种 有 意 行 凶 。
不 去 消 灾 , 反 而 助 虐 。
风 随 着 火 势 , 火 焰 飞 到 千 丈 多 高 的 地 方 ; 火 逞 风 威 , 灰 烬 飞 上 九 霄 云 外 。
声 音 嘈 嘈 , 好 像 残 年 的 爆 竹 一 样 ; 声 音 浩 浩 浩 浩 浩 浩 浩 浩 荡 , 却 像 军 中 炮 声 一 样 。
烧 得 那 当 场 的 佛 像 无 法 逃 走 , 东 院 的 伽 蓝 无 处 躲 避 。
胜 如 在 赤 壁 夜 鏖 兵 , 战 过 阿 房 宫 内 发 生 火 灾 。
这 正 是 星 星 之 火 , 能 烧 万 顷 的 田 地 。
不 一 会 儿 , 风 狂 火 猛 , 把 一 座 观 音 院 里 的 观 音 院 里 都 变 得 红 了 。
你 看 那 些 和 尚 , 搬 着 箱 子 , 抬 着 笼 子 , 抢 着 桌 子 端 着 锅 子 , 满 院 子 里 叫 苦 连 天 。
孙 行 者 护 住 了 后 边 方 丈 , 辟 火 笼 罩 住 了 前 面 的 禅 堂 , 其 余 前 后 的 火 光 大 发 , 真 是 照 天 红 焰 辉 煌 , 穿 透 墙 壁 金 光 照 耀 。
不 料 火 烧 起 的 时 候 , 就 惊 动 了 一 座 山 兽 怪 。
观 音 院 正 南 二 十 里 远 近 , 有 座 黑 风 山 , 山 中 有 个 黑 风 洞 , 洞 里 有 个 妖 精 , 正 在 睡 醒 时 翻 身 。
只 见 那 窗 间 透 出 亮 光 , 只 说 是 天 明 。
起 来 看 时 , 却 是 正 北 下 的 火 光 。
妖 精 大 惊 说 : 啊 , 这 一 定 是 观 音 院 里 失 火 。
和 尚 好 不 小 心 。
我 看 他 的 时 候 , 与 他 救 一 次 。
他 喜 好 妖 精 , 纵 手 起 云 头 , 立 即 到 了 烟 火 的 下 面 , 果 然 是 冲 天 的 火 , 前 面 的 殿 宇 都 是 空 的 , 两 廊 的 烟 火 正 烧 。
他 大 拽 步 , 撞 着 将 要 进 去 , 正 喊 叫 着 取 水 来 , 只 见 那 后 房 没 有 火 , 房 脊 上 有 一 个 人 放 风 。
他 就 知 道 这 个 情 况 , 急 忙 进 入 里 面 去 看 时 , 看 见 那 方 丈 的 中 间 有 些 霞 光 彩 气 , 台 案 上 有 一 个 青 毡 包 袱 。
他 解 开 一 看 , 见 是 一 领 锦 袈 裟 , 是 佛 门 中 的 珍 宝 。
正 是 财 物 感 动 人 心 , 他 也 不 救 火 , 他 也 不 叫 水 , 拿 着 那 袈 裟 , 趁 着 打 劫 , 拽 回 云 步 , 径 直 转 到 东 山 去 。
大 火 烧 到 五 更 天 亮 , 才 熄 灭 。
你 看 那 些 和 尚 们 都 是 赤 红 色 的 , 哭 泣 哭 泣 , 都 去 那 灰 里 寻 找 铜 铁 , 拨 开 腐 烂 的 炭 火 , 扑 击 金 银 。
有 的 在 墙 子 里 , 用 草 苫 盖 成 窝 棚 ; 有 的 在 红 色 的 墙 头 , 拿 着 锅 做 饭 。
叫 冤 叫 屈 , 乱 喧 乱 闹 不 题 。
又 说 行 人 取 了 辟 火 罩 , 一 个 斗 送 上 南 天 门 , 交 给 广 目 天 王 说 : 谢 借 , 谢 借 。
天 王 收 了 , 说 : 大 圣 至 诚 了 。
我 正 愁 你 不 还 我 的 宝 贝 , 无 处 寻 找 , 而 且 喜 就 送 来 。
行 人 说 : 老 孙 可 是 那 当 面 骗 物 的 人 , 这 叫 做 好 借 好 还 , 再 借 不 难 。
天 王 说 : 许 久 不 见 面 , 请 让 我 到 宫 中 稍 坐 一 时 , 怎 么 样 ? 行 人 说 : 老 孙 近 来 在 前 不 同 , 烂 板 凳 , 高 谈 阔 论 了 , 如 今 保 住 唐 僧 , 不 得 身 闲 。
容 叙 , 容 叙 。
急 忙 告 别 , 坠 落 在 云 间 , 又 出 现 在 那 太 阳 星 上 。
径 直 来 到 禅 堂 前 , 转 身 一 变 , 变 成 了 个 蜜 蜂 儿 , 飞 进 去 , 现 在 佛 像 看 时 , 那 师 父 还 沉 睡 着 了 。
行 者 喊 道 : 师 父 , 天 亮 了 , 起 来 吧 。
三 藏 这 才 醒 过 来 , 翻 身 说 道 : 正 是 。
于 是 , 忽 然 抬 起 头 来 , 只 见 倒 壁 红 墙 , 不 见 楼 台 殿 宇 。
大 吃 一 惊 说 : 啊 , 怎 么 这 殿 宇 都 没 有 , 都 是 红 墙 , 为 什 么 ? 行 人 说 : 你 还 做 梦 吗 ? 今 夜 走 了 火 的 。
三 藏 菩 萨 说 : 我 怎 么 不 知 道 ? 行 者 说 : 是 老 孙 护 了 禅 堂 , 见 到 师 父 睡 得 很 浓 , 不 曾 惊 动 。
三 藏 说 : 你 有 本 事 保 护 了 禅 堂 , 为 什 么 不 救 别 房 的 火 呢 ? 行 者 笑 着 说 : 好 好 教 师 父 得 知 , 果 然 如 你 昨 天 说 的 话 , 他 爱 上 我 们 的 袈 裟 , 算 计 想 要 烧 死 我 们 。
如 果 不 是 老 孙 子 的 知 觉 , 到 现 在 都 变 成 灰 骨 了 。
三 藏 听 了 , 十 分 害 怕 , 说 : 是 他 们 放 的 火 吗 ? 行 者 说 : 不 是 他 是 谁 ? 三 藏 说 : 莫 不 是 怠 慢 了 你 , 你 干 的 这 个 勾 当 吗 ? 行 人 说 : 老 孙 是 这 样 懒 惰 的 人 , 干 这 些 不 好 的 事 , 实 际 上 是 他 家 放 的 。
老 孙 见 他 心 毒 , 果 然 是 不 曾 给 他 救 火 , 只 是 给 他 略 略 助 风 。
三 藏 说 : 天 啊 , 天 啊 ! 火 烧 起 来 的 时 候 , 只 该 帮 助 水 , 怎 么 会 转 助 风 呢 ? 行 人 说 : 你 可 以 知 道 古 人 说 : 人 没 有 伤 害 虎 的 心 , 虎 没 有 伤 害 人 的 意 思 。
三 藏 说 : 袈 裟 在 哪 里 , 岂 敢 不 是 烧 坏 了 ? 行 者 说 : 没 有 事 , 烧 不 坏 , 那 放 袈 裟 的 方 丈 没 有 火 。
三 藏 怨 恨 地 说 : 我 不 管 你 , 只 是 有 些 伤 害 , 我 只 把 那 话 儿 念 动 , 念 动 你 就 是 死 了 。
行 人 惊 慌 地 说 : 师 父 不 要 念 , 不 要 念 , 管 寻 还 你 袈 裟 就 是 了 。
等 我 去 , 拿 来 走 路 。
三 藏 刚 牵 着 马 , 行 人 挑 了 担 子 , 走 出 了 禅 堂 , 径 直 往 后 方 丈 远 去 。
又 说 : 有 一 个 和 尚 正 在 悲 哀 痛 苦 的 时 候 , 忽 然 看 见 他 的 师 徒 , 牵 着 马 挑 着 担 子 来 , 吓 得 一 个 人 魂 飞 魄 散 , 说 : 冤 魂 来 了 。
行 人 喝 道 : 你 为 什 么 冤 魂 要 命 , 快 把 袈 裟 还 给 我 。
众 僧 一 齐 跪 下 , 叩 头 说 : 爷 爷 啊 , 冤 有 冤 家 , 债 有 债 主 。
要 求 命 不 干 我 们 的 事 , 都 是 广 谋 与 老 和 尚 定 计 谋 害 你 的 , 不 要 问 我 们 讨 伐 命 令 。
行 者 呵 斥 一 声 说 : 我 把 你 们 这 些 该 死 的 畜 生 , 那 个 问 你 们 讨 得 什 么 命 ?
其 中 有 两 个 胆 量 大 的 和 尚 说 : 老 爷 , 你 们 在 禅 堂 里 已 经 烧 死 了 , 现 在 又 来 讨 伐 袈 裟 , 端 的 还 是 人 , 是 鬼 吗 ? 行 人 笑 着 说 : 这 些 孽 畜 , 那 里 有 什 么 火 来 , 你 们 去 前 面 看 看 禅 堂 , 再 来 说 话 。
众 僧 人 爬 起 来 , 往 前 观 看 , 那 禅 堂 外 面 的 门 窗 , 更 不 曾 烧 烧 过 半 分 。
众 人 害 怕 , 才 认 出 三 藏 是 神 僧 , 行 走 的 人 是 尊 护 法 。
一 齐 上 前 叩 头 说 : 我 们 有 眼 无 珠 , 不 认 识 真 人 下 界 。
你 的 袈 裟 在 后 面 方 丈 中 老 师 祖 的 地 方 吗 ?
三 藏 走 了 三 五 层 , 败 壁 破 墙 , 叹 息 不 已 。
只 见 方 丈 果 然 没 有 火 , 众 僧 人 抢 进 里 面 , 大 叫 道 : 公 公 , 唐 僧 是 神 人 , 未 曾 烧 死 , 现 在 反 而 害 了 自 己 的 家 。
及 早 拿 出 袈 裟 , 还 给 他 去 。
原 来 这 个 老 和 尚 找 不 到 袈 裟 , 又 烧 了 本 寺 的 房 屋 , 正 在 万 分 烦 恼 焦 燥 的 地 方 , 一 听 到 这 样 的 话 , 怎 么 敢 答 应 。
因 为 想 想 没 有 办 法 , 进 退 没 有 办 法 , 拽 开 步 子 , 亲 自 穿 着 腰 , 往 那 墙 上 实 际 撞 了 一 个 头 , 可 怜 只 是 撞 破 了 脑 袋 , 血 流 了 , 魂 魄 散 开 , 咽 喉 气 断 , 染 红 沙 。
有 诗 作 证 。
诗 说 : 堪 叹 老 衲 性 愚 蒙 , 枉 作 人 间 一 寿 翁 。
想 得 到 袈 裟 传 给 远 世 , 哪 知 佛 宝 不 同 。
但 将 领 容 易 长 久 , 一 定 会 使 国 家 萧 条 得 到 失 败 。
广 泛 的 智 慧 广 泛 的 谋 略 , 能 够 有 什 么 用 处 呢 ? 损 害 别 人 利 己 , 一 场 空 虚 。
忽 然 有 个 众 僧 哭 着 说 : 师 公 已 经 被 撞 死 了 , 又 没 有 见 袈 裟 , 怎 么 能 产 生 这 么 好 呢 ? 行 者 说 : 我 想 是 你 们 偷 藏 起 来 的 。
都 出 来 , 打 开 了 花 名 手 本 , 等 老 孙 逐 一 查 点 。
那 上 下 房 的 院 主 , 把 本 寺 的 和 尚 、 头 陀 、 幸 童 、 道 人 全 都 打 开 了 两 张 手 本 , 大 小 人 等 共 计 二 百 三 十 名 。
行 者 请 师 父 高 坐 , 他 却 一 一 从 头 喊 名 检 查 , 都 要 解 开 衣 襟 , 分 明 点 过 , 再 没 有 袈 裟 。
又 将 各 房 子 里 的 人 都 拿 出 去 的 箱 子 , 从 头 细 细 地 寻 遍 , 那 里 得 到 了 踪 迹 。
三 藏 心 中 烦 恼 , 心 中 怨 恨 , 念 念 念 念 念 念 念 念 念 念 念 念 念 念 念 念 念 念 念 念 念 念 念 念 念 念 念 念 念 念 念 念 念 念 念 念 念 念 念 念 念 念 念 念 念 念 念 念 念 念 念 念 念 念 念 念 念 念 念 念 念 念 念 念 念 念 念 念 念 念 念 念 念 念 念 念 念 念 念 念 念 念 念 念 念 念 念 念 念 念 念 念 念 念 念 念 念 念 念 念 念 念 念 念 念 念 念 念 念 念 念 念 念 念 念 念 念 念 念 念 念 念 念 念 念 念 念 念 念 念 念 念 念 念 念 念 念 念 念 念 念 念 念 念 念 念 念 念 念 念 念 念 念 念 念 念 念 念 念 念 念 念 念 念 念 念 念 念 念 念 念 念 念 念 念 念 念 念 念 念 念 念 念 念 念 念 念 念 念 念 念 念 念 念 念 念 念 念 念 念 念 念 念 念 念 念 念 念 念 念 念 念 念 念 念 念 念 念 念 念 念 念 念 念 念 念 念 念 念 念 念 念 念 念 念 念 念 念 念 念 念 念 念
行 者 扑 倒 在 地 , 抱 着 头 , 十 分 难 以 禁 止 , 只 教 他 说 : 不 要 念 , 不 要 念 , 管 寻 还 了 袈 裟 。
众 僧 见 了 , 一 个 个 人 都 很 惊 慌 , 上 前 跪 下 劝 解 , 三 藏 菩 萨 才 合 口 不 念 。
行 人 一 个 骨 鲁 跳 起 来 , 从 耳 朵 里 拿 出 铁 棒 , 想 要 打 那 些 和 尚 , 被 三 藏 大 骂 , 说 : 这 个 猴 头 , 你 头 痛 还 不 怕 , 还 要 无 礼 , 不 要 动 手 , 暂 且 不 要 伤 害 人 , 再 和 我 审 问 一 问 。
众 僧 人 都 叩 头 礼 拜 , 哀 告 三 藏 说 : 老 爷 饶 命 。
吾 所 谓 之 , 吾 所 谓 之 。
这 都 是 那 老 死 鬼 的 不 是 。
他 昨 夜 看 着 你 的 袈 裟 , 只 哭 得 到 更 深 的 时 候 , 看 也 不 曾 敢 看 , 想 想 要 长 久 , 做 出 传 家 的 宝 物 , 设 计 谋 划 , 要 烧 死 老 爷 。
自 从 火 烧 起 的 时 候 , 狂 风 大 作 , 各 人 只 顾 救 火 , 抢 夺 物 品 , 更 不 知 道 袈 裟 去 向 。
行 人 大 怒 , 走 进 方 丈 的 屋 子 里 , 把 那 触 死 鬼 的 尸 首 抬 出 来 。
就 把 一 个 方 丈 挖 地 三 尺 , 也 没 有 踪 影 。
行 人 估 计 半 晌 , 问 道 : 你 这 里 可 以 有 什 么 妖 怪 成 精 吗 ? 院 主 说 : 老 爷 不 问 , 没 想 得 知 。
我 的 东 南 方 有 一 座 黑 风 山 , 黑 风 洞 里 有 一 个 黑 大 王 , 我 这 个 老 死 鬼 经 常 和 他 讲 道 , 他 就 是 个 妖 精 。
别 的 东 西 没 有 什 么 东 西 。
行 人 说 : 那 山 离 此 地 有 多 远 ? 院 主 说 : 只 有 二 十 里 , 那 看 见 山 头 就 是 。
行 人 笑 着 说 : 师 父 放 心 , 不 必 讲 了 , 一 定 是 那 黑 怪 偷 去 无 疑 。
三 藏 说 : 他 那 厢 离 此 地 有 二 十 里 , 为 什 么 就 断 得 到 它 呢 ? 行 人 说 : 你 不 曾 见 过 夜 间 那 火 , 光 芒 万 里 , 亮 光 透 彻 三 天 , 暂 且 不 说 二 十 里 , 就 是 二 百 里 也 照 得 到 了 。
以 此 为 是 , 必 然 不 知 其 所 以 为 是 , 必 然 不 知 其 所 以 为 是 也 。
等 老 孙 去 找 他 一 寻 。
三 藏 说 : 你 走 了 , 我 又 依 靠 什 么 ? 行 者 说 : 这 个 放 心 , 暗 中 自 有 神 灵 保 护 , 明 中 等 我 叫 那 些 和 尚 伏 侍 。
于 是 就 把 和 尚 叫 过 来 , 说 : 你 们 带 几 个 去 埋 那 个 老 鬼 , 带 几 个 伏 在 我 的 师 父 身 边 , 看 守 我 的 白 马 。
众 位 和 尚 领 着 他 们 的 诺 言 。
行 者 又 说 : 你 们 不 要 顺 口 儿 答 应 , 等 我 去 了 , 你 们 就 不 来 奉 承 。
看 师 父 的 , 要 怡 悦 颜 悦 色 ; 养 白 马 的 , 要 水 草 调 匀 。
假 如 有 一 点 点 差 错 , 就 依 照 这 个 样 子 棍 子 , 给 你 们 看 看 。
他 拿 出 木 棍 子 , 把 火 烧 的 砖 墙 上 , 扑 得 一 下 , 把 那 墙 打 得 粉 碎 , 又 打 倒 了 七 八 层 墙 。
众 僧 见 了 , 全 都 骨 瘦 身 麻 , 跪 着 叩 头 滴 泪 说 : 爷 爷 放 宽 心 去 , 我 等 竭 力 虔 诚 , 供 奉 老 爷 , 决 不 敢 一 丝 一 丝 一 丝 。
好 行 的 人 急 忙 放 出 斗 云 , 径 直 上 了 黑 风 山 , 寻 找 这 袈 裟 。
正 是 那 : 金 禅 求 正 出 京 畿 , 仗 锡 投 西 渡 翠 微 。
虎 豹 狼 虫 行 走 的 地 方 有 虎 豹 狼 虫 , 工 商 士 客 见 到 的 时 候 很 少 。
途 中 遇 到 异 国 愚 僧 嫉 妒 , 全 仗 齐 天 大 圣 的 威 严 。
火 发 风 生 , 禅 院 废 弃 , 黑 熊 夜 偷 锦 衣 。
再 说 这 里 去 , 不 知 道 袈 裟 有 没 有 , 吉 凶 怎 么 样 , 姑 且 听 下 回 分 解 。
}\switchcolumn\flushpage  \begin{pinyinscope}{\myfontt \section{第一七回}     孫行者大鬧黑風山 觀世音收伏熊羆怪

話說孫行者一觔斗跳將起去,諕得那觀音院大小和尚並頭陀、幸童、道人等一個
個朝天禮拜道:「爺爺呀,原來是騰雲駕霧的神聖下界,怪道火不能傷。恨我那
個不識人的老剝皮使心用心,今日反害了自己。」三藏道:「列位請起,不須恨
了。這去尋著袈裟,萬事皆休﹔但恐找尋不著,我那徒弟性子有些不好,汝等性
命不知如何,恐一人不能脫也。」眾僧聞得此言,一個個提心弔膽,告天許願,
只要尋得袈裟,各全性命不題。
  
卻說孫大聖到空中,把腰兒扭了一扭,早來到黑風山上。住了雲頭,仔細看,果
然是座好山,況正值春光時節,但見:
萬壑爭流,千崖競秀。鳥啼人不見,花落樹猶香。雨過天連青壁潤,風來松捲翠
屏張。山草發,野花開,懸崖峭嶂﹔薜蘿生,佳木麗,峻嶺平崗。不遇幽人,那
尋樵子?澗邊雙鶴飲,石上野猿狂。矗矗堆螺排黛色,巍巍擁翠弄嵐光。
  
那行者正觀山景,忽聽得芳草坡前,有人言語。他卻輕步潛蹤,閃在那石崖之下
,偷睛觀看。原來是三個妖魔,席地而坐:上首的是一條黑漢,左首下是一個道
人,右首下是一個白衣秀士。都在那裏高談闊論,講的是立鼎安爐,摶砂煉汞,
白雪黃芽,傍門外道。正說中間,那黑漢笑道:「後日是我母難之日,二公可光
顧光顧。」白衣秀士道:「年年與大王上壽,今年豈有不來之理?」黑漢道:
「我夜來得了一件寶貝,名喚錦襴佛衣,誠然是件玩好之物。我明日就以他為壽
,大開筵宴,邀請各山道官,慶賀佛衣,就稱為佛衣會如何?」道人笑道:「妙
,妙,妙。我明日先來拜壽,後日再來赴宴。」
  
行者聞得佛衣之言,定以為是他寶貝。他就忍不住怒氣,跳出石崖,雙手舉起金
箍棒,高叫道:「我把你這夥賊怪!你偷了我的袈裟,要做甚麼佛衣會?趁早兒
將來還我。」喝一聲「休走!」掄起棒,照頭一下。慌得那黑漢化風而逃,道人
駕雲而走,只把個白衣秀士一棒打死。拖將過來看處,卻是一條白花蛇怪。索性
提起來,捽做五七斷。徑入深山,找尋那個黑漢。轉過尖峰,抹過峻嶺,又見那
壁陡崖前,聳出一座洞府。但見那:
煙霞渺渺,松柏森森。煙霞渺渺采盈門,松柏森森青遶戶。橋踏枯槎木,峰巔繞
薜蘿。鳥啣紅蕊來雲壑,鹿踐芳叢上石臺。那門前時催花發,風送花香。臨堤綠
柳轉 黃鸝,傍岸夭桃翻粉蝶。雖然曠野不堪誇 ,卻賽蓬萊山下景。
  
行者到於門首,又見那兩扇石門關得甚緊。門上有一橫石板,明書六個大字,乃
「黑風山黑風洞」。即便掄棒,叫聲:「開門!」那裏面有把門的小妖,開了門
出來,問道:「你是何人,敢來擊吾仙洞?」行者罵道:「你個作死的孽畜!甚
麼個去處,敢稱仙洞?『仙』字是你稱的?快進去報與你那黑漢,教他快送老爺
的袈裟出來,饒你一窩性命。」小妖急急跑到裏面,報道:「大王,佛衣會做不
成了,門外有一個毛臉雷公嘴的和尚來討袈裟哩。」那黑漢被行者在芳草坡前趕
將來,卻才關了門,坐還未穩,又聽得那話,心中暗想道:「這廝不知是那裏來
的,這般無禮,他敢嚷上我的門來。」教取披掛,隨結束了,綽一桿黑纓槍,走
出門來。這行者閃在門外,執著鐵棒,睜睛觀看,只見那怪果生得兇險:
    碗子鐵盔火漆光,烏金鎧甲亮輝煌。
    皂羅袍罩風兜袖,黑綠絲絛穗長。
    手執黑纓槍一桿,足 踏烏皮靴一雙。
    眼晃金睛如掣電,正是山中黑風王。

  
行者暗笑道:「這廝真個如燒?的一般,築煤的無二,想必是在此處刷炭為生,
怎麼這等一身烏黑?」那怪厲聲高叫道:「你是個甚麼和尚,敢在我這裏大膽?」
行者執鐵棒,撞至面前,大?一聲道:「不要閑講,快還你老外公的袈裟來。」
那怪道:「你是那寺裏和尚?你的袈裟在那裏失落了,敢來我這裏索取?」行者
道:「我的袈裟在直北觀音院後方丈裏放著,只因那院裏失了火,你這廝趁鬨擄
掠,盜了來,要做佛衣會慶壽,怎敢抵賴?快快還我,饒你性命﹔若牙迸半個
『不』字,我推倒了黑風山,屣平了黑風洞,把你這一洞妖邪都碾為齏粉。」
  
那怪聞言,呵呵冷笑道:「你這個潑物,原來昨夜那火就是你放的。你在那方丈
屋上行兇招風,是我把一件袈裟拿來了,你待怎麼?你是那裏來的?姓甚名誰?
有多大手段,敢那等海口浪言。」行者道:「是你也認不得你老外公哩。你老外
公 乃大唐上國駕前御弟三藏法師之徒弟,姓孫,名悟空行者。若問老孫的手段
,說出來,教你魂飛魄散,死在眼前。」那怪道:「我不曾會,你有甚麼手段,
說來我聽。」行者笑道:「我兒子,你站穩著,仔細聽之。我:
    自小神通手段高,隨風變化逞英豪。
    養性修真熬日月,跳出輪迴把命逃。
    一點誠心曾訪道,靈臺山上採藥苗。
    那山有個老仙長,壽年十萬八千高。
    老孫拜他為師父,指我長生路一條。
    他說身內有丹藥,外邊採取枉徒勞。
    得傳大品天仙訣,若無根本實難熬。
    回光內照寧心坐,身中日月坎離交。
    萬事不思全寡慾,六根清淨體堅牢。
    返老還童容易得,超凡入聖路非遙。
    三年無漏成仙體,不同俗輩受煎熬。
    十洲三島還遊戲,海角天涯轉一遭。
    活該三百多餘歲,不得飛昇上九霄。
    下海降龍真寶貝,才有金箍棒一條。
    花果山前為帥首,水簾洞裏聚群妖。
    玉皇大帝傳宣詔,封我齊天極品高。
    幾番大鬧靈霄殿,數次曾偷王母桃。
    天兵十萬來降我,層層密密布槍刀。
    戰退天王歸上界,哪吒負痛領兵逃。
    顯聖真君能變化,老孫硬賭跌平交。
    道祖觀音同玉帝,南天門上看降妖。
    卻被老君助一陣,二郎擒我到天曹。
    將身綁在降妖柱,即命神兵把首梟。
    刀砍鎚敲不得壞,又教雷打火來燒。
    老孫其實有手段,全然不怕半分毫。
    送在老君爐裏煉,六丁神火慢煎熬。
    日滿開爐我跳出,手持鐵棒遶天跑。
    縱橫到處無遮擋,三十三天鬧一遭。
    我佛如來施法力,五行山壓老孫腰。
    整整壓該五百載,幸逢三藏出唐朝。
    吾今皈正西方去,轉上雷音見玉毫。
        你去乾坤四海問一問,我是歷代馳名第一妖。」

那怪聞言笑道:「你原來是那鬧天宮的弼馬溫麼?」行者最惱的是人叫他弼馬溫
,聽見這一聲,心中大怒,罵道:「你這賊怪!偷了袈裟不還,倒傷老爺。不要
走,看棍。」那黑漢側身躲過,綽長槍,劈手來迎。兩家這場好殺:
如意棒,黑纓槍,二人洞口逞剛強。分心劈臉刺,著臂照頭傷。這個橫丟陰棍手
,那個直撚急三槍。白虎爬山來探爪,黃龍臥道轉身忙。噴彩霧,吐毫光,兩個
妖仙不可量。一個是修正齊天聖,一個是成精黑大王。這場山裏相爭處,只為袈
裟各不良。
  
那怪與行者鬥了十數回合,不分勝負,漸漸紅日當午。那黑漢舉槍架住鐵棒道:
「孫行者,咱兩個且收兵,等我進了膳來,再與你賭鬥。」行者道:「你這個孽
畜,教做漢子?好漢子,半日兒就要吃飯?似老孫在山根下,整壓了五百餘年,
也未曾嘗些湯水,那裏便餓哩?莫推故,休走,還我袈裟來,方讓你去吃飯。」
那怪虛幌一槍,撤身入洞,關了石門,收回小怪,且安排筵宴,書寫請帖,邀請
各山魔王慶會不題。
  
卻說行者攻門不開,也只得回觀音院。那本寺僧人已葬埋了那老和尚,都在方丈
裏伏侍唐僧。早齋已畢,又擺上午齋。正那裏添湯換水,只見行者從空降下,眾
僧禮拜,接入方丈,見了三藏。三藏道:「悟空,你來了?袈裟如何?」行者道
:「已有了根由。早是不曾冤了這些和尚,原來是那黑風山妖怪偷了。老孫 去
暗暗的尋他,只見他與一個白衣秀士、一個老道人,坐在那芳草坡前講話。也是
個不打自招的怪物,他忽然說出道:後日是他母難之日,邀請諸邪來做生日﹔夜
來得了一件錦襴佛衣,要以此為壽,作一大宴,喚做慶賞佛衣會。是老孫搶到面
前,打了一棍,那黑漢化風而走,道人也不見了,只把個白衣秀士打死,乃是一
條白花蛇成精。我又急急趕到他洞口,叫他出來與他賭鬥。他已承認了,是他拿
回。戰勾這半日,不分勝負。那怪回洞,卻要吃飯,關了石門,懼戰不出。老孫
卻來回看師父,先報此信。已是有了袈裟的下落,不怕他不還我。」
  
眾僧聞言,合掌的合掌,磕頭的磕頭,都念聲:「南無阿彌陀佛!今日尋著下落
,我等方有了性命矣。」行者道:「你且休喜歡暢快,我還未曾到手,師父還未
曾出門哩。只等有了袈裟,打發得我師父好好的出門,才是你們的安樂處﹔若稍
有些須不虞,老孫可是好惹的主子!可曾有好茶飯與我師父吃?可曾有好草料喂
馬?」眾僧俱滿口答應道:「有,有,有,更不曾一毫待怠慢了老爺。」三藏道
:「自你去了這半日,我已吃過了三次茶湯,兩餐齋供了,他俱不曾敢慢我。但
只是你還盡心竭力去尋取袈裟回來。」行者道:「莫忙,既有下落,管情拿住這
廝,還你原物。放心,放心。」
  
正說處,那上房院主又整治素供,請孫老爺吃齋。行者卻吃了些須,復駕祥雲,
又去找尋。正行間,只 見一個小妖,左脅下夾著一個花梨木匣兒,從大路而來。
行者度他匣內必有甚麼柬札,舉起棒,劈頭一下,可憐不禁打,就打得似個肉餅
一般。卻拖在路傍,揭開匣兒觀看,果然是一封請帖。帖上寫著:
侍生熊羆頓首拜,啟上大闡金池老上人丹房:屢承佳惠,感激淵深。夜觀回祿之
難,有失救護,諒仙機必無他害。生偶得佛衣一件,欲作雅會,謹具花酌,奉扳
清賞。至期,千乞仙駕過臨一敘。是荷。先二日具。
  
行者見了,呵呵大笑道:「那個老剝皮,死得他一毫兒也不虧,他原來與妖精結
黨。怪道他也活了二百七十歲,想是那個妖精傳他些甚麼服氣的小法兒,故有此
壽。老孫還記得他的模樣,等我就變做那和尚,往他洞裏走走,看我那袈裟放在
何處。假若得手,即便拿回,卻也省力。」
  
好大聖,念動咒語,迎著風一變,果然就像那老和尚一般。藏了鐵棒,拽開步,
徑來洞口,叫聲:「開門!」那小妖開了門,見是這般模樣,急轉身報道:「大
王,金池長老來了。」那怪大驚道:「剛才差了小的去下簡帖請他,這時候還未
到那裏哩,如何他就來得這等迅速?想是小的不曾撞著他,斷是孫行者呼他來討
袈裟的。管事的,可把佛衣藏了,莫教他看見。」
  
行者進了前門,但見那天井中松篁交翠,桃李爭妍,叢叢花發,簇簇蘭香,卻也
是個洞天之處。又見那二門上有一聯對子,寫著:「靜隱深山無俗慮﹔幽居仙洞
樂天真。」行者暗道:「這廝也是個脫垢離塵,知命的怪物。」入門裏,往前又
進,到於三層門裏,都是些畫棟雕梁,明窗彩戶。只見那黑漢子穿的是黑綠紵絲
袢襖,罩一領鴉青花綾披風,戴一頂烏角軟巾,穿一雙麂皮皂靴。見行者進來,
整頓衣巾,降階迎接道:「金池老友,連日欠親。請坐,請坐。」行者以禮相見
。見畢而坐,坐定而茶。茶罷,妖精欠身道:「適有小簡奉啟,後日一敘,何老
友今日就下顧也?」行者道:「正來進拜,不期路遇華翰,見有佛衣雅會,故此
急急奔來,願求見見。」那怪笑道:「老友差矣。這袈裟本是唐僧的,他在你處
住錫,你豈不曾看見,反來就我看看?」行者道:「貧僧借來,因夜晚還不曾展
看,不期被大王取來。又被火燒了荒山,失落了家私。那唐僧的徒弟又有些驍勇
,亂忙中,四下裏都尋覓不見。原來是大王的洪福收來,故特來一見。」
  
正講處,只見有一個巡山的小妖來報道:「大王,禍事了,下請書的小校被孫行
者打死在大路傍邊,他綽著經兒,變化做金池長老,來騙佛衣也。」那怪聞言,
暗道:「我說那長老怎麼今日就來,又來得迅速,果然是他。」急縱身,拿過槍
來,就刺行者。行者耳朵裏急掣出棍子,現了本相,架住槍尖,就在他那中廳裏
跳出,自天井中鬥到前門外。諕得那洞裏群魔都喪膽,家間老幼盡無魂。這場在
山頭好賭鬥,比前番更是不同。好殺:
那猴王膽大充和尚,這黑漢心靈隱佛衣。語去言來機會巧,隨機應變不差池。袈
裟欲見無由見,寶貝玄微真妙微。小怪巡山言禍事,老妖發怒顯神威。翻身打出
黑風洞,槍棒爭持辨是非。棒架長槍聲響喨,槍迎鐵棒放光輝。悟空變化人間少
,妖怪神通世上稀。這個要把佛衣來慶壽,那個不得袈裟肯善歸?這番苦戰難分
手,就是活佛臨凡也解不 得圍。
  
他兩個從洞口打上山頭,自山頭殺在雲外,吐霧噴風,飛砂走石,只鬥到紅日沉
西,不分勝敗。那怪道:「姓孫的,你且住了手,今日天晚,不好相持。你去,
你去,待明早來,與你定個死活。」行者叫道:「兒子莫走,要戰便像個戰的,
不可以天晚相推。」看他沒頭沒臉的,只情使棍子打來。這黑漢又化陣清風,轉
回本洞,緊閉石門不出。
  
行者卻無計策奈何,只得也回觀音院裏,按落雲頭,道聲:「師父。」那三藏眼
兒巴巴的正望他哩,忽見到了面前,甚喜﹔又見他手裏沒有袈裟,又懼。問道:
「怎麼這番還不曾有袈裟來?」行者袖中取出個簡帖兒來,遞與三藏道:「師父
,那怪物與這死的老剝皮原是朋友。他著一個小妖送此帖來,還請他去赴佛衣會
。是老孫就把那小妖打死,變做那老和尚,進他洞去,騙了一鍾茶吃。欲問他討
袈裟看看,他不肯拿出。正坐間,忽被一個甚麼巡山的走了風信,他就與我打將
起來。只鬥到這早晚,不分上下。他見天晚,閃回洞去,緊閉石門。老孫無奈,
也暫回來。」三藏道:「你手段比他何如?」行者道:「我也硬不多兒,只戰個
手平。」
  
三藏才看了簡帖,又遞與那院主道:「你師父敢莫也是妖精麼?」那院主慌忙跪
下道:「老爺,我師父是人。只因那黑大王修成人道,常來寺裏與我師父講經,
他傳了我師父些養神服氣之術,故以朋友相稱。」行者道:「這夥和尚沒甚妖氣
,他一個個頭圓頂天,足方履地,但比老孫肥胖長大些兒,非妖精也。你看那帖
兒上寫著『侍生熊羆』,此物必定是個黑熊成精。」三藏道:「我聞得古人云:
『熊與猩猩相類。』都是獸類。他卻怎麼成精?」行者笑道:「老孫是獸類,見
做了齊天大聖,與他何異?大抵世間之物,凡有九竅者,皆可以修行成仙。」三
藏又道:「你才說他本事與你手平,你卻怎生得勝,取我袈裟回來?」行者道:
「莫管,莫管,我有處治。」
  
商議間,眾僧擺上晚齋,請他師徒們吃了。三藏教掌燈,仍去前面禪堂安歇。眾
僧都挨牆倚壁,苫搭窩棚 ,各各睡下,只把後方丈讓與那上下院主安 身。此時
夜靜,但見:
銀河現影,玉宇無塵。滿天星燦爛,一水浪收痕。萬籟聲寧,千山鳥絕。溪邊漁
火息,塔上佛燈昏。昨夜闍黎鐘鼓響,今宵一遍哭聲聞。
  
是夜在禪堂歇宿。那三藏想著袈裟,那裏得穩睡?忽翻身見窗外透白,急起叫道
:「悟空,天明了,快尋袈裟去。」行者一骨魯跳將起來,一見眾僧侍立,供奉
湯水,行者道:「你等用心伏侍我師父,老孫去也。」三藏下床,扯住道:「你
往那裏去?」行者道:「我想這樁事都是觀音菩薩沒理,他有這個禪院在此,受
了這裏人家香火,又容那妖精鄰住。我去南海尋他,與他講一講,教他親來問妖
精討袈裟還我。」三藏道:「你這去,幾時回來?」行者道:「時少只在飯罷,
時多只在晌午,就成功了。那些和尚可好伏侍,老孫去也。」
  
說聲去,早已無蹤。須臾間到了南海,停雲觀看。但見那:
汪洋海遠,水勢連天。祥光籠宇宙,瑞氣照山川。千層雪浪吼青霄,萬疊煙波滔
白晝。水飛四野,浪滾週遭。水飛四野振轟雷,浪滾週遭鳴霹靂。休言水勢,且
看中間。五色朦朧寶疊山,紅黃紫皂綠和藍。才見觀音真勝境,試看南海落伽山
。好去處,山峰高聳,頂透虛空。中間有千樣奇花,百般瑞草。風搖寶樹,日映
金蓮。觀音殿,瓦蓋琉璃﹔潮音洞,門鋪玳瑁。綠楊影裏語鸚哥,紫竹林中啼孔
雀。羅紋石上,護法威嚴﹔瑪瑙灘前,木叉雄壯。這行者觀不盡那異景非常,徑
直按雲頭,到竹林之下。早有諸天迎接道:「菩薩前者對眾言大聖歸善,甚是宣
揚。今保唐僧,如何得暇到此?」行者道:「因保唐僧,路逢一事,特見菩薩,
煩為通報。」諸天遂來洞口報知,菩薩喚入。行者遵法而行,至寶蓮臺下拜了。
菩薩問曰:「你來何幹?」行者道:「我師父路遇你的禪院,你受了人間香火,
容一個黑熊精在那裏鄰住,著他偷了我師父袈裟,屢次取討不與,今特來問你要
的。」菩薩道:「這猴子說話,這等無狀。既是熊精偷了你的袈裟,你怎來問我
取討?都是你這個孽猴大膽,將寶貝賣弄,拿與小人看見,你卻又行兇,喚風發
火,燒了我的留雲下院,反來我處放刁。」行者見菩薩說出這話,知他曉得過去
未來之事,慌忙禮拜道:「菩薩,乞恕弟子之罪,果是這般這等。但恨那怪物不
肯與我袈裟,師父又要念那話兒咒語,老孫忍不得頭疼,故此來拜煩菩薩。望菩
薩慈悲之心,助我去拿那妖精,取衣西進也。」菩薩道:「那怪物有許多神通,
卻也不亞於你。也罷,我看唐僧面上,和你去走一遭。」行者聞言,謝恩再拜。
即請菩薩出門,遂同駕祥雲,早到黑風山,墜落雲頭,依路找洞。
  
正行處,只見那山坡前走出一個道人,手拿著一個玻璃盤兒,盤內安著兩粒仙丹
,往前正走。被行者撞個滿懷,掣出棒,就照頭一下,打得腦裏漿流出,腔中血
迸攛。菩薩大驚道:「你這個猴子,還是這等放潑。他又不曾偷你袈裟,又不與
你相識,又無甚冤仇,你怎麼就將他打死?」行者道:「菩薩,你認他不得,他
是那黑熊精的朋友。他昨日和一個白衣秀士,都在芳草坡前坐講。後日是黑精的
生日,請他們來慶佛衣會。今日他先來拜壽,明日來慶佛衣會。所以我認得,定
是今日替那妖去上壽。」菩薩說:「既是這等說來,也罷。」行者才去把那道人
提起來看,卻是一隻蒼狼。傍邊那個盤兒底下卻 有字,刻道「凌虛子製」。
  
行者見了,笑道:「造化,造化,老孫也是便益,菩薩也是省力。這怪叫做不打
自招,那怪教他今日了劣。」菩薩說道:「悟空,這教怎麼說?」行者道:「菩
薩,我悟空有一句話兒,叫做將計就計,不知菩薩可肯依我 ?」菩薩道:「你
說。」行者說道:「菩薩,你看這盤兒中是兩粒仙丹,便是我們與那妖魔的贄見
﹔這盤兒後面刻的四個字,說『凌虛子製』,便是我們與那妖魔的勾頭。菩薩若
要依得我時,我好替你作個計較,也就不須動得干戈,也不須勞得征戰,妖魔眼
下遭瘟,佛衣眼下出現﹔菩薩要不依我時,菩薩往西,我悟空往東,佛衣只當相
送,唐三藏只當落空。」菩薩笑道:「這猴熟嘴。」行者道:「不敢,倒是一個
計較。」菩薩說:「你這計較怎說?」行者道:「這盤上刻那『凌虛子製』,想
這道人就叫做凌虛子。菩薩,你要依我時,可就變做這個道人,我把這丹吃了一
粒,變上一粒,略大些兒。菩薩,你就捧了這個盤兒、兩粒仙丹,去與那妖上壽
,把這丸大些的讓與那妖。待那妖一口吞之,老孫便於中取事:他若不肯獻出佛
衣,老孫將他肚腸就也織將一件出來。」菩薩沒法,只得也點點頭兒依他。行者
笑道:「如何?」
  
爾時菩薩迺以廣大慈悲,無邊法力,億萬化身,以心會意,以意會身,恍惚之間
,變作凌虛仙子:
    鶴氅仙風颯,飄颻欲步虛。
    蒼顏松柏老,秀色古今無。
    去去還無住,如如自有殊。
    總來歸一法,只是隔郛軀。
  
行者看道:「妙呵,妙呵!還是妖精菩薩,還是菩薩妖精?」菩薩笑道:「悟空
,菩薩、妖精,總是一念﹔若論本來,皆屬無有。」行者心下頓悟,轉身卻就變
做一粒仙丹:
    走盤無不定,圓明未有方。
    三三勾漏合,六六少翁商。
    瓦鑠黃金焰,牟尼白晝光。
    外邊鉛與汞,未許易論量。

行者變了那顆丹,終是略大些兒。菩薩認定,拿了那個玻璃盤兒,徑到妖洞門口
看時,果然是:
崖深岫險,雲生嶺上﹔柏蒼松翠,風颯林間。崖深岫險,果是妖邪出沒人煙少;
柏蒼松翠,也可仙真修隱道情多。山有澗,澗有泉,潺潺流水咽鳴琴,便堪洗耳
﹔崖有鹿,林有鶴,幽幽仙籟動間岑,亦可賞心。這是妖仙有分降菩提,弘誓無
邊垂惻隱。
  
菩薩看了,心中暗喜道:「這孽畜占了這座山洞,卻是也有些道分。」因此心中
已是有個慈悲。
  
走到洞口,只見守洞小妖都有些認得道:「凌虛仙長來了。」一邊傳報,一邊接
引。那妖早已迎出門道:「凌虛,有勞仙駕珍顧,蓬蓽有輝。」菩薩道:「小道
敬獻一粒仙丹,敢稱千壽。」他二人拜畢,方才坐定,又敘起他昨日之事。菩薩
不答,連忙拿丹盤道:「大王,且見小道鄙意。」覷定一粒大的,推與那妖道:
「願大王千壽。」那妖亦推一粒,遞與菩薩道:「願與凌虛子同之。」讓畢,那
妖才待要咽,那藥順口兒一直滾下。現了本相,理起四平。那妖滾倒在地。菩薩
現相,問妖取了佛衣。行者早已從鼻孔中出去。菩薩又怕那妖無禮,卻把一個箍
兒丟在那妖頭上。那妖起來,提槍要刺,行者、菩薩早已起在空中,將真言念起
。那怪依舊頭疼,丟了槍,滿地亂滾。半空裏笑倒個美猴王,平地下滾壞個黑熊
怪。
  
菩薩道:「孽畜,你如今可皈依麼?」那怪滿口道:「心願皈依,只望饒命。」
行者恐耽擱了工夫,意欲就打。菩薩急止住道:「休傷他命,我有用他處哩。」
行者道:「這樣怪物,不打死他,反留他在何處用哩?」菩薩道:「我那落伽
山後無人看管,我要帶他去做個守山大神。」行者笑道:「誠然是個救苦慈尊,
一靈不損。若是老孫有這樣咒語,就念上他娘千遍。這回兒就有許多黑熊,都
教他了帳。」卻說那怪甦醒多時,公道難禁疼痛,只得跪在地下哀告道:「但
饒性命,願皈正果。」菩薩方墜落祥光,又與他摩頂受戒,教他執了長槍,跟
隨左右。那黑熊才一片野心今日定,無窮頑性此時收。
菩薩吩咐道:「悟空,你回去罷,好生伏侍唐僧,以後再休懈惰生事。」行者
道:「深感菩薩遠來,弟子還當回送回送。」菩薩道:「免送。」行者才捧著
袈裟,叩頭而別。菩薩亦帶了熊羆,徑回大海。有詩為證。詩曰:
    祥光靄靄凝金像,萬道繽紛實可誇。
    普濟世人垂憫恤,遍觀法界現金蓮。
    今來多為傳經意,此去原無落點瑕。
    降怪成真歸大海,空門復得錦袈裟。

    畢竟不知向後事情如何,且聽下回分解。





}  \end{pinyinscope}\switchcolumn{\myfontc \section{第 一 七 回} , 孙 行 者 大 闹 黑 风 山 , 观 世 音 收 伏 熊 怪 话 , 说 孙 行 者 一 斗 跳 将 要 起 身 走 , 看 到 那 观 音 院 大 小 和 尚 及 头 陀 、 幸 童 、 道 人 等 一 个 个 朝 天 礼 拜 道 : 爷 爷 啊 , 原 来 是 腾 云 驾 雾 的 神 圣 下 界 , 怪 道 火 不 能 伤 。
恨 我 不 识 人 的 老 人 剥 皮 使 心 , 今 天 反 而 伤 害 了 自 己 。
三 藏 说 : 列 位 请 您 起 来 , 不 必 怨 恨 了 。
这 里 去 寻 找 袈 裟 , 万 事 都 可 以 休 息 , 但 恐 怕 找 不 到 , 我 那 弟 弟 的 性 子 有 些 不 好 , 你 们 的 性 命 不 知 怎 么 样 , 恐 怕 一 个 人 不 能 脱 脱 。
众 僧 听 了 这 话 , 一 个 个 提 心 吊 胆 , 告 诉 上 天 , 愿 意 说 : 只 要 找 到 袈 裟 , 就 可 以 保 全 性 命 。
又 说 孙 大 圣 到 了 空 中 , 把 腰 儿 扭 了 一 圈 , 早 上 来 到 黑 风 山 上 。
住 在 云 头 , 仔 细 看 , 果 然 是 一 座 好 山 , 何 况 正 值 春 光 的 时 节 , 只 见 到 万 壑 争 流 , 千 崖 竞 秀 。
鸟 啼 人 不 见 , 花 落 树 还 有 香 。
雨 过 天 空 连 接 , 青 色 的 墙 壁 都 变 得 润 泽 , 风 来 松 树 卷 起 翠 色 的 屏 风 张 开 。
山 上 的 草 丛 生 长 , 野 外 的 花 开 放 , 悬 崖 峭 壁 , 草 丛 生 , 佳 木 茂 盛 , 峻 岭 平 冈 。
不 遇 到 幽 人 , 何 必 寻 找 樵 夫 ? 涧 边 的 双 鹤 饮 , 石 上 的 野 猿 狂 。
巍 巍 的 石 螺 堆 列 着 碧 绿 的 颜 色 , 巍 巍 的 山 峰 簇 拥 着 翠 色 映 映 着 山 光 。
那 个 行 人 正 在 观 赏 山 景 , 忽 然 听 到 芳 草 坡 前 , 有 人 说 话 。
他 轻 步 行 走 , 在 那 石 崖 之 下 , 偷 眼 观 看 。
原 来 是 三 个 妖 魔 , 坐 在 地 上 , 上 首 的 是 一 条 黑 色 的 汉 人 , 左 头 下 面 的 是 一 个 道 人 , 右 头 下 面 的 是 一 个 白 衣 秀 士 。
以 此 为 之 , 以 此 为 之 , 以 此 为 之 。
正 在 说 中 间 , 那 黑 汉 笑 着 说 : 今 天 是 我 母 亲 去 世 的 日 子 , 二 公 可 以 光 顾 光 顾 。
白 衣 秀 士 说 : 年 年 给 大 王 祝 寿 , 今 年 岂 有 不 来 的 道 理 ? 黑 汉 说 : 我 昨 晚 得 到 一 件 宝 贝 , 名 叫 锦 佛 衣 , 实 在 是 一 件 玩 好 的 东 西 。
我 明 天 就 以 他 为 寿 , 大 开 筵 席 , 邀 请 各 山 道 官 , 庆 贺 佛 衣 , 就 称 为 佛 衣 会 怎 么 样 ? 道 人 笑 着 说 : 妙 , 妙 , 妙 , 妙 , 妙 。
我 第 二 天 先 来 祝 寿 , 以 后 再 来 参 加 宴 会 。
行 者 听 到 了 佛 衣 的 话 , 一 定 认 为 是 他 的 宝 贝 。
其 他 人 之 所 以 为 之 , 不 可 以 为 之 。
他 喝 了 一 声 , 喝 了 一 声 , 喝 了 一 声 , 喝 了 一 声 , 喝 了 一 声 , 休 走 吧 。 王 起 来 棒 子 , 照 着 头 一 下 。
惊 得 那 个 黑 汉 化 风 而 逃 , 道 人 驾 着 云 而 走 , 只 把 那 个 白 衣 秀 士 一 棒 打 死 。
把 它 拖 过 来 看 的 地 方 , 却 是 一 条 白 花 蛇 怪 。
于 是 就 把 他 提 起 来 , 打 了 五 七 断 。
径 直 进 入 深 山 , 寻 找 那 个 黑 汉 。
转 过 尖 峰 , 抹 过 陡 岭 , 又 见 那 墙 壁 陡 崖 前 , 耸 出 一 座 洞 府 。
只 见 那 里 , 烟 霞 缥 缈 , 松 柏 森 森 。
烟 霞 缥 缈 , 采 采 盈 门 , 松 柏 森 森 , 青 色 环 绕 门 户 。
桥 上 踏 枯 木 , 峰 顶 绕 萝 。
鸟 红 蕊 飞 来 云 壑 , 鹿 践 芳 草 上 石 台 。
那 门 前 时 催 促 花 开 , 风 送 花 香 。
临 堤 绿 柳 转 黄 , 傍 岸 夭 桃 翻 粉 蝶 。
虽 然 旷 野 不 堪 夸 耀 , 却 比 蓬 莱 山 下 的 景 色 更 美 。
走 的 人 到 了 门 口 , 又 看 见 那 两 扇 石 门 关 得 很 紧 。
洞 口 上 有 一 块 横 着 的 石 板 , 上 面 写 着 六 个 大 字 , 是 黑 风 山 黑 风 洞 。
于 是 就 拿 着 棒 子 , 大 声 叫 道 : 开 门 , 那 里 面 有 个 把 门 的 小 妖 , 打 开 门 出 来 , 问 道 : 你 是 什 么 人 , 竟 敢 来 攻 打 我 仙 洞 ? 行 人 骂 道 : 你 这 个 作 死 的 孽 畜 , 怎 么 样 的 去 处 , 敢 称 仙 洞 ? 仙 字 是 你 自 称 的 , 快 进 去 报 给 你 那 黑 汉 , 教 他 快 送 老 爷 的 袈 裟 出 来 , 饶 你 一 窝 性 命 。
小 妖 急 忙 跑 到 里 面 , 报 告 说 : 大 王 , 佛 衣 会 做 不 成 了 , 门 外 有 一 个 毛 面 雷 公 嘴 的 和 尚 来 讨 菩 裟 哩 。
那 个 黑 汉 被 行 者 在 芳 草 坡 前 赶 快 来 , 他 才 关 上 门 , 坐 还 没 稳 , 又 听 到 那 个 话 , 心 中 暗 想 道 : 这 个 人 不 知 道 是 从 哪 里 来 的 , 这 样 无 礼 , 他 胆 敢 喧 闹 上 我 的 门 来 。
教 他 拿 出 披 挂 , 随 即 结 束 了 , 苏 绰 一 支 黑 色 的 帽 子 , 走 出 门 来 。
这 个 行 人 在 门 外 闪 闪 , 拿 着 铁 棒 , 睁 着 眼 睛 观 看 , 只 见 那 怪 物 果 然 生 出 凶 险 的 怪 物 。 碗 子 铁 盔 火 漆 光 , 乌 金 铠 甲 光 亮 辉 煌 。
黑 色 绫 罗 袍 , 罩 着 风 兜 袖 , 黑 色 绿 色 丝 , 穗 长 。
他 手 里 拿 着 一 根 黑 色 帽 子 , 脚 踏 着 乌 皮 靴 子 一 双 。
眼 睛 闪 闪 金 睛 , 好 像 闪 电 一 样 , 正 是 山 中 的 黑 风 王 。
行 人 暗 笑 着 说 : 这 个 人 真 的 如 烧 柴 的 一 样 , 筑 煤 的 没 有 两 个 , 想 必 是 在 此 处 淘 炭 为 生 , 何 必 这 样 一 身 黑 黑 , 那 怪 怪 厉 声 高 叫 道 : 你 是 什 么 和 尚 , 敢 在 我 这 里 大 胆 吗 ? 行 人 拿 着 铁 棒 , 撞 到 面 前 , 大 声 说 : 不 要 闲 说 , 快 还 你 老 外 公 的 袈 裟 来 。
那 人 怪 道 : 你 是 那 寺 里 的 和 尚 , 你 的 袈 裟 在 那 里 失 落 了 , 敢 来 我 这 里 索 取 ? 行 人 说 : 我 的 袈 裟 在 直 北 观 音 院 后 方 丈 里 放 着 , 只 因 为 那 院 里 失 火 , 你 这 个 人 趁 着 抢 劫 , 偷 了 来 , 要 做 佛 衣 会 庆 寿 , 何 敢 抵 抗 , 快 快 还 我 , 饶 你 性 命 ; 如 果 牙 裂 半 个 不 字 , 我 推 倒 了 黑 风 山 , 把 黑 风 洞 。
那 怪 听 了 这 话 , 就 呵 斥 冷 笑 道 : 你 这 个 泼 物 , 原 来 昨 夜 那 火 就 是 你 放 的 。
你 在 那 方 丈 屋 子 上 行 凶 招 风 , 是 我 把 一 件 袈 裟 拿 来 了 , 你 等 什 么 , 你 是 从 哪 里 来 的 , 姓 什 么 名 字 是 谁 , 有 多 大 的 手 段 , 怎 么 敢 等 海 口 浪 语 ?
行 人 说 : 是 你 , 认 不 到 你 老 外 公 吗 ?
你 老 外 公 是 大 唐 上 国 驾 前 御 弟 三 藏 法 师 的 弟 弟 , 姓 孙 , 名 叫 悟 空 行 者 。
如 果 问 老 子 的 手 段 , 说 出 来 , 使 你 魂 飞 魄 散 , 死 在 眼 前 。
李 那 奇 怪 地 说 : 我 不 曾 参 加 过 , 你 有 什 么 手 段 , 说 来 我 听 。
行 人 笑 着 说 : 我 的 儿 子 , 你 站 稳 着 , 仔 细 听 。
我 说 : 自 小 就 神 通 , 手 段 高 , 随 风 变 化 , 逞 英 豪 。
养 性 修 真 , 熬 炼 日 月 , 跳 出 轮 回 , 把 命 运 逃 走 。
一 点 诚 心 曾 经 拜 访 道 士 , 在 灵 台 山 上 采 药 苗 。
山 上 有 个 老 仙 长 , 寿 命 十 万 八 千 高 。
老 子 拜 他 为 师 父 , 指 点 我 长 生 路 一 条 。
其 人 曰 : 身 内 有 丹 药 , 外 边 采 取 , 枉 枉 徒 劳 。
得 以 传 授 《 大 品 天 仙 诀 》 , 如 果 没 有 根 本 , 实 在 难 以 煎 熬 。
回 光 内 照 , 宁 心 坐 , 身 中 日 月 坎 离 交 。
万 事 不 考 虑 , 保 全 寡 欲 , 六 根 清 净 身 体 坚 固 。
返 老 还 童 是 容 易 得 到 的 , 超 凡 入 圣 的 途 径 并 非 遥 远 。
三 年 没 有 漏 洞 成 仙 体 , 不 同 俗 人 一 样 受 到 煎 熬 。
十 洲 三 岛 还 游 戏 , 海 角 天 涯 转 一 遭 。
活 了 三 百 多 年 , 不 能 飞 升 上 九 霄 。
下 海 降 龙 真 是 宝 贝 , 才 有 金 钳 棒 一 支 。
花 果 山 前 是 帅 首 , 水 帘 洞 里 聚 集 了 群 妖 。
玉 皇 大 帝 传 宣 诏 书 , 封 我 齐 天 极 品 高 。
几 次 大 闹 灵 霄 殿 , 多 次 曾 偷 王 母 桃 。
天 兵 十 万 来 投 降 我 , 层 层 密 密 地 布 满 刀 枪 。
战 败 后 , 天 王 回 到 上 界 , 何 吒 负 痛 领 兵 逃 走 。
显 圣 真 君 能 变 化 , 老 孙 硬 打 赌 , 跌 平 交 。
道 祖 观 音 像 玉 帝 , 在 南 天 门 上 看 降 妖 。
却 被 老 君 帮 助 一 阵 , 二 郎 擒 拿 我 到 天 曹 。
将 自 己 的 身 子 绑 在 降 妖 柱 上 , 立 即 命 令 神 兵 把 头 割 下 来 。
他 又 叫 雷 打 火 来 烧 。
老 孙 子 其 实 很 有 才 干 , 完 全 不 怕 半 分 毫 。
送 到 老 君 炉 里 炼 , 六 丁 神 火 慢 熬 。
日 落 时 分 开 火 炉 , 我 跳 出 来 , 手 里 拿 着 铁 棒 绕 着 天 空 逃 跑 。
纵 横 到 处 没 有 遮 挡 , 三 十 三 天 闹 一 场 。
我 佛 如 来 施 法 力 , 五 行 山 压 老 孙 腰 。
整 整 地 压 倒 了 五 百 年 , 幸 逢 三 藏 出 自 唐 朝 。
我 现 在 皈 依 佛 法 皈 正 西 方 去 , 转 上 雷 声 显 现 玉 毫 。
你 去 天 地 四 海 问 一 问 , 我 是 历 代 流 传 名 声 的 第 一 妖 孽 。
那 怪 听 到 话 笑 着 说 : 你 原 来 是 那 闹 天 宫 的 弼 马 温 吗 ? 行 人 最 恼 恨 的 是 人 叫 他 弼 马 温 , 听 到 这 一 声 , 心 中 大 怒 , 骂 道 : 你 这 个 贼 怪 , 偷 了 袈 裟 不 还 , 倒 伤 老 爷 。
不 要 走 , 看 棍 子 。
那 个 黑 汉 侧 着 身 子 逃 过 去 , 挥 动 长 枪 , 劈 开 手 来 迎 接 。
两 家 喜 欢 杀 人 : 如 意 棒 , 黑 缨 枪 , 二 人 洞 口 逞 强 。
分 心 劈 面 刺 , 刺 臂 照 头 伤 。
这 个 横 抛 阴 棍 手 , 那 个 直 握 急 三 枪 。
白 虎 爬 山 来 探 爪 , 黄 龙 躺 在 道 上 转 身 忙 。
喷 出 彩 雾 , 吐 出 毫 光 , 两 个 妖 仙 不 可 估 量 。
一 个 是 修 正 齐 天 圣 , 一 个 是 成 精 黑 大 王 。
这 是 山 里 互 相 争 夺 的 地 方 , 只 是 因 为 袈 裟 各 不 好 。
那 怪 与 行 人 打 了 十 几 回 合 , 不 分 胜 负 , 渐 渐 变 红 , 太 阳 正 当 中 午 。
那 个 黑 汉 举 起 刀 把 铁 棒 , 说 道 : 孙 行 者 , 你 们 两 个 暂 且 收 兵 , 等 我 进 上 膳 食 来 , 再 与 你 们 赌 斗 。
行 人 说 : 你 这 个 孽 畜 , 教 我 做 汉 子 , 好 汉 子 , 半 天 儿 就 要 吃 饭 , 好 像 老 孙 子 在 山 根 下 , 全 被 压 了 五 百 多 年 , 也 不 曾 尝 过 过 汤 水 , 那 里 就 会 挨 饿 了 , 不 要 推 究 原 因 , 休 走 吧 , 还 给 我 袈 裟 来 , 才 让 你 去 吃 饭 。
那 怪 虚 张 一 张 箭 , 撤 身 入 洞 , 关 上 石 门 , 收 回 小 怪 , 并 且 安 排 筵 席 , 书 写 请 帖 , 邀 请 各 山 魔 王 举 行 庆 贺 。
又 说 行 人 攻 城 门 不 开 , 也 只 得 回 观 音 院 。
那 本 寺 的 僧 人 已 经 埋 葬 了 那 个 老 和 尚 , 都 在 方 丈 里 , 伏 地 侍 奉 唐 僧 。
早 斋 已 经 吃 完 , 又 摆 上 午 斋 。
只 见 行 人 从 空 中 降 下 来 , 众 僧 礼 拜 , 接 进 方 丈 , 见 了 三 藏 菩 萨 。
三 藏 菩 萨 说 : 悟 空 , 你 来 了 , 袈 裟 怎 么 样 ? 行 者 说 : 已 经 有 了 根 由 。
早 就 不 曾 冤 枉 了 这 些 和 尚 , 原 来 是 那 个 黑 风 山 的 妖 怪 偷 了 。
去 之 , 不 见 之 , 不 见 之 , 不 见 之 , 不 见 之 。
他 忽 然 说 出 道 : 今 天 是 他 母 亲 去 世 的 日 子 , 邀 请 各 位 邪 恶 来 做 生 日 , 夜 里 得 到 了 一 件 锦 佛 衣 , 要 用 它 作 为 寿 命 , 举 行 一 次 宴 会 , 叫 做 庆 赏 佛 衣 会 。
是 老 孙 子 夺 到 面 前 , 打 了 一 个 棍 子 , 那 个 黑 汉 化 风 而 走 , 道 人 也 不 见 了 , 只 是 把 那 个 白 衣 秀 士 打 死 了 , 原 来 是 一 条 白 花 蛇 变 成 了 精 。
吾 又 急 忙 赶 到 他 的 洞 口 , 叫 他 出 来 与 他 赌 斗 。
其 人 已 经 承 认 了 , 是 他 拿 回 来 的 。
双 方 交 战 半 天 , 不 分 胜 负 。
怪 怪 回 洞 , 却 要 吃 饭 , 关 上 石 门 , 害 怕 战 斗 不 出 来 。
老 子 又 来 回 去 看 看 师 父 , 先 报 了 这 个 信 。
已 经 有 了 袈 裟 的 下 落 , 不 怕 他 不 还 我 。
众 僧 听 了 , 合 掌 的 合 掌 , 叩 头 的 叩 头 , 都 念 道 : 南 无 阿 弥 陀 佛 , 今 天 寻 着 下 落 , 我 们 才 有 了 性 命 。
行 人 说 : 你 暂 且 不 要 喜 欢 畅 快 , 我 还 没 有 到 手 , 师 父 还 没 有 出 门 呀 !
只 等 到 有 了 袈 裟 , 打 发 我 师 父 好 好 出 门 , 才 是 你 们 的 安 乐 之 地 ; 如 果 稍 有 不 测 之 事 , 老 孙 可 以 是 好 惹 的 主 子 , 可 曾 有 好 茶 饭 给 我 师 父 吃 , 还 有 好 草 料 喂 马 吗 ? 众 僧 都 满 口 答 应 道 : 有 , 有 , 更 不 曾 一 丝 一 丝 一 丝 等 待 老 爷 。
三 藏 说 : 自 从 你 离 开 这 半 天 , 我 已 经 吃 过 三 次 茶 汤 , 两 餐 斋 供 了 , 其 他 都 不 敢 轻 慢 我 。
只 是 你 还 尽 心 竭 力 去 寻 找 袈 裟 回 来 。
行 人 说 : 不 要 忙 , 既 有 下 落 , 管 理 情 况 , 拿 住 这 个 人 , 还 给 你 原 来 的 东 西 。
放 心 , 放 心 。
正 在 说 完 , 那 上 房 院 主 又 整 顿 素 餐 , 请 孙 老 爷 吃 斋 。
行 人 又 吃 了 些 胡 须 , 又 驾 着 祥 云 , 又 去 找 寻 找 。
正 走 时 , 只 见 一 个 小 妖 , 左 胁 下 夹 着 一 个 花 梨 木 匣 子 , 从 大 路 上 走 来 。
行 人 估 计 他 的 匣 子 里 一 定 有 什 么 柬 札 , 举 起 棍 子 , 劈 开 头 一 下 , 可 怜 不 禁 打 , 就 打 得 像 个 肉 饼 一 样 。
又 拖 在 路 旁 , 揭 开 匣 子 观 看 , 果 然 是 一 封 请 帖 。
帖 子 上 写 着 : 侍 生 熊 叩 头 叩 拜 , 启 奏 皇 上 大 阐 金 池 老 上 人 丹 房 : 多 次 承 蒙 好 好 的 恩 惠 , 感 激 深 深 。
夜 里 观 察 回 禄 的 祸 难 , 如 果 有 失 救 护 , 就 可 以 明 白 仙 机 一 定 没 有 其 他 灾 害 。
先 生 偶 然 得 到 一 件 佛 衣 , 想 要 举 行 雅 会 , 谨 准 备 好 花 斟 酒 , 奉 送 给 我 清 赏 。
到 了 约 定 的 日 期 , 千 请 求 仙 驾 过 去 , 亲 自 叙 述 一 次 。
这 是 荷 花 。
先 前 两 天 准 备 好 。
行 者 见 了 , 哈 哈 大 笑 道 : 那 个 老 人 剥 皮 , 死 得 他 一 点 也 不 亏 , 他 原 来 与 妖 精 结 党 。
奇 怪 说 他 活 了 二 百 七 十 岁 , 想 是 那 个 妖 精 传 给 他 那 么 服 气 的 小 法 儿 , 所 以 才 有 这 样 的 寿 命 。
老 孙 还 记 得 他 的 样 子 , 等 我 就 变 成 了 和 尚 , 去 他 洞 里 走 , 看 我 的 袈 裟 放 在 什 么 地 方 ?
假 使 得 到 了 手 , 就 马 上 拿 回 去 , 退 却 也 很 省 力 。
大 圣 人 , 念 咒 语 , 迎 风 一 变 , 果 然 像 那 个 老 和 尚 一 样 。
把 铁 棒 藏 起 来 , 拽 开 步 子 , 径 直 来 到 洞 口 , 喊 叫 道 : 开 门 , 那 个 小 妖 人 打 开 门 , 见 是 这 样 的 模 样 , 急 忙 转 身 报 告 说 : 大 王 , 金 池 长 老 来 了 。
那 怪 大 吃 一 惊 说 : 刚 才 差 遣 小 的 去 下 简 帖 请 他 , 这 时 还 没 到 那 里 了 , 为 什 么 他 就 来 得 这 样 快 , 想 是 小 的 不 曾 撞 着 他 , 断 定 是 孙 行 者 叫 他 来 讨 袈 裟 的 。
管 事 的 , 可 以 把 佛 衣 藏 起 来 , 不 要 让 他 看 见 。
走 的 人 走 进 前 门 , 只 见 那 天 井 中 松 竹 交 错 , 桃 李 争 艳 , 丛 生 的 花 发 , 簇 拥 的 兰 香 , 却 也 是 个 个 洞 天 的 地 方 。
又 看 见 那 两 个 门 上 有 一 个 对 子 , 写 着 : 静 隐 深 山 无 俗 虑 , 幽 居 仙 洞 乐 天 真 。
行 人 暗 中 说 : 这 个 人 也 是 个 脱 尘 脱 尘 , 知 命 的 怪 物 。
进 入 门 里 , 往 前 又 进 去 , 走 到 三 层 门 里 , 都 是 画 栋 雕 梁 , 明 窗 彩 户 。
只 见 那 个 黑 色 的 汉 子 穿 的 是 黑 色 绿 色 的 丝 , 罩 着 一 领 乌 青 色 的 绫 罗 衣 , 戴 着 一 顶 乌 角 的 软 巾 , 穿 着 一 双 鹿 皮 黑 色 的 靴 子 。
看 见 行 人 进 来 , 整 顿 衣 服 , 走 下 台 阶 迎 接 道 : 金 池 老 友 , 连 日 不 亲 。
请 坐 , 请 坐 。
行 者 以 礼 相 见 。
见 完 就 坐 下 , 坐 定 就 喝 茶 。
茶 罢 , 妖 精 拖 着 身 子 说 : 刚 才 有 个 小 书 信 送 给 我 , 以 后 再 叙 说 , 何 老 友 今 天 就 下 来 看 望 我 呢 ? 行 人 说 : 正 来 进 拜 , 不 料 路 上 遇 到 华 翰 , 见 到 有 一 个 穿 着 佛 衣 雅 会 , 所 以 急 急 奔 来 , 希 望 能 见 到 我 。
那 人 怪 笑 着 说 : 老 朋 友 好 了 。
这 袈 裟 本 是 唐 僧 的 , 他 在 你 住 锡 , 你 难 道 不 曾 看 见 , 反 而 来 到 我 那 里 去 看 看 , 行 人 说 : 我 借 来 , 因 为 夜 晚 还 不 曾 展 开 看 , 不 料 被 大 王 取 来 。
又 被 火 烧 了 荒 山 , 失 去 了 家 人 。
唐 僧 的 徒 弟 又 有 些 勇 猛 的 人 , 在 乱 忙 之 中 , 四 下 里 都 寻 找 不 见 。
原 来 是 大 王 的 洪 福 收 来 , 所 以 特 来 见 一 面 。
正 在 讲 经 的 地 方 , 只 见 有 一 个 巡 山 的 小 妖 来 报 告 说 : 大 王 , 祸 事 了 , 下 来 请 书 的 小 校 , 被 孙 行 者 打 死 在 大 路 旁 边 , 他 绰 着 经 儿 , 变 化 成 金 池 长 老 , 来 欺 骗 佛 衣 。
那 怪 听 了 他 的 话 , 暗 中 说 : 我 说 那 长 老 , 今 天 就 来 了 , 又 来 得 得 很 快 , 果 然 是 他 。
他 急 忙 放 下 身 子 , 拿 过 来 的 箭 来 , 就 刺 了 行 人 。
行 者 急 忙 抽 出 棍 子 , 现 出 本 来 的 相 貌 , 架 住 枪 尖 , 就 在 他 的 中 厅 里 跳 出 来 , 从 天 井 中 斗 到 前 门 外 。
等 到 那 个 洞 里 的 群 魔 都 丧 胆 , 家 里 的 老 幼 都 没 有 魂 魄 。
此 场 在 山 头 好 赌 , 与 前 番 更 是 不 同 。
好 杀 : 那 猴 王 胆 大 充 和 尚 , 这 个 黑 汉 心 灵 隐 藏 佛 衣 。
语 言 去 去 言 语 来 , 机 会 巧 妙 , 随 机 应 变 不 差 池 。
袈 裟 欲 见 无 从 见 , 宝 贝 玄 微 真 妙 微 。
小 的 怪 物 巡 视 山 上 说 出 祸 事 , 老 的 妖 怪 发 怒 显 示 神 威 。
翻 身 打 出 黑 风 洞 , 枪 棒 争 相 争 辩 是 非 。
棒 架 长 枪 声 响 , 枪 迎 铁 棒 放 光 辉 。
悟 空 变 化 人 间 很 少 , 妖 怪 神 通 世 上 很 少 。
这 个 人 要 拿 佛 衣 来 祝 寿 , 那 个 人 不 得 袈 裟 肯 善 归 , 这 次 苦 战 难 分 手 , 就 是 活 佛 临 凡 也 解 不 出 围 。
其 他 两 个 人 从 洞 口 打 上 山 头 , 从 山 头 杀 死 在 云 外 , 吐 出 雾 气 , 喷 出 风 来 , 飞 砂 走 石 , 只 能 斗 到 红 日 沉 西 , 不 分 胜 负 。
那 人 奇 怪 地 说 : 姓 孙 的 , 你 暂 且 拉 住 他 的 手 , 今 天 天 晚 了 , 不 好 相 持 。
你 去 , 你 去 , 等 到 明 早 来 , 与 你 决 定 死 生 。
行 人 大 叫 道 : 儿 子 不 要 走 , 要 打 就 像 个 打 仗 的 人 , 不 能 因 为 天 晚 相 互 推 举 。
看 他 没 头 没 脸 的 , 只 是 让 棍 子 打 来 。
这 个 黑 汉 又 变 成 一 阵 清 风 , 转 回 本 洞 , 紧 闭 石 门 不 出 来 。
行 走 的 人 却 没 有 办 法 , 只 好 回 到 观 音 院 里 , 按 着 云 头 , 说 道 : 师 父 。
其 三 藏 之 眼 睛 , 正 在 望 他 , 忽 然 见 到 他 , 非 常 高 兴 , 又 见 他 手 里 没 有 袈 裟 , 又 害 怕 。
又 问 道 : 为 什 么 这 次 还 不 曾 有 袈 裟 来 ? 行 人 从 袖 中 取 出 一 个 简 帖 子 , 递 给 三 藏 说 : 那 怪 物 与 这 个 死 的 老 剥 皮 原 是 朋 友 。
他 是 一 个 小 妖 , 送 此 帖 子 来 , 还 请 他 去 参 加 佛 衣 会 。
因 为 这 个 老 孙 子 就 把 他 打 死 , 变 成 了 那 个 老 和 尚 , 进 入 洞 里 去 , 骗 了 一 钟 茶 吃 。
想 问 他 去 讨 袈 裟 , 他 不 肯 拿 出 来 。
正 坐 之 间 , 忽 然 被 一 个 巡 山 的 人 走 了 , 他 就 和 我 打 将 起 来 。
只 是 争 斗 到 这 个 早 晚 , 不 分 上 下 。
他 看 见 天 色 已 晚 , 就 闪 回 洞 中 去 了 , 紧 紧 关 上 石 门 。
老 孙 无 可 奈 何 , 也 暂 且 回 来 。
三 藏 说 : 你 的 手 段 比 他 怎 么 样 ? 行 人 说 : 我 也 硬 不 多 儿 , 只 打 一 个 手 平 。
三 藏 才 看 了 简 帖 , 又 递 给 那 院 主 说 : 你 师 父 敢 不 是 妖 精 吗 ? 那 院 主 忙 忙 跪 下 说 : 老 爷 , 我 师 父 是 人 。
因 为 那 黑 大 王 修 炼 成 人 之 道 , 经 常 来 到 寺 里 和 我 师 父 讲 经 , 他 传 给 我 师 父 的 一 点 养 神 服 气 的 方 法 , 所 以 以 朋 友 相 称 。
行 人 说 : 这 伙 和 尚 没 有 什 么 妖 气 , 他 们 一 个 个 头 圆 顶 天 , 脚 方 履 地 , 只 是 比 老 孙 子 肥 胖 , 长 大 些 小 孩 , 不 是 妖 精 。
你 看 那 帖 子 上 写 着 侍 生 熊 , 此 物 必 定 是 个 黑 熊 成 精 。
三 藏 说 : 我 听 古 人 说 : 熊 与 猩 猩 相 类 似 。
这 些 都 是 兽 类 。
行 人 笑 着 说 : 老 子 是 兽 类 , 现 做 了 齐 天 大 圣 , 与 他 有 什 么 不 同 呢 大 概 世 间 的 东 西 , 凡 有 九 窍 的 , 都 可 以 修 行 成 仙 。
三 藏 又 说 : 你 才 说 他 的 本 事 与 你 的 手 平 , 你 怎 么 能 得 到 胜 利 , 拿 我 袈 裟 回 来 ? 行 者 说 : 不 管 , 不 管 , 我 有 处 治 。
商 议 之 间 , 众 僧 摆 上 晚 斋 , 请 他 师 徒 们 吃 了 。
三 藏 教 掌 管 灯 火 , 仍 然 去 前 面 的 禅 堂 安 歇 。
众 僧 人 都 靠 墙 靠 壁 , 苫 草 盖 屋 , 各 自 睡 觉 , 只 把 后 方 丈 让 给 那 上 下 院 主 安 身 。
此 时 夜 静 , 只 看 见 一 片 银 河 出 现 的 影 子 , 玉 宇 没 有 尘 土 。
满 天 星 星 灿 烂 , 一 条 水 浪 都 收 起 了 痕 迹 。
万 籁 声 宁 , 千 山 鸟 绝 。
溪 边 的 渔 火 熄 灭 了 , 塔 上 的 佛 灯 也 昏 暗 。
昨 天 夜 里 , 我 们 的 钟 鼓 响 了 , 今 天 夜 里 一 遍 , 哭 声 就 听 到 了 。
这 天 夜 里 在 禅 堂 歇 宿 。
那 三 藏 菩 萨 想 穿 袈 裟 , 怎 么 能 稳 睡 呢 忽 然 翻 身 反 身 , 看 见 窗 外 透 出 白 色 , 急 忙 起 来 大 叫 道 : 悟 空 , 天 明 了 , 快 去 找 袈 裟 去 。
行 人 一 个 骨 鲁 跳 将 要 起 来 , 一 见 众 僧 站 在 那 里 , 供 奉 汤 水 , 行 人 说 : 你 们 尽 心 服 侍 我 的 师 父 , 老 孙 就 去 了 。
三 藏 下 床 , 拽 住 道 : 你 去 哪 里 去 ? 行 者 说 : 我 想 这 些 事 都 是 观 音 菩 萨 没 理 , 他 有 此 处 禅 院 在 这 里 , 接 受 了 这 里 人 家 的 香 火 , 又 容 许 那 妖 精 邻 住 。
我 去 南 海 寻 找 他 , 和 他 讲 一 讲 , 教 他 亲 自 来 问 妖 精 , 讨 菩 裟 还 给 我 。
三 藏 说 : 你 这 样 去 , 几 时 回 来 ? 行 人 说 : 时 少 只 在 饭 罢 , 时 多 只 在 晌 午 , 就 成 功 了 。
那 些 和 尚 可 好 伏 地 侍 候 , 老 孙 子 去 吧 。
说 完 就 走 了 , 早 已 没 有 了 踪 影 。
不 一 会 儿 , 到 了 南 海 , 停 下 云 来 观 看 。
只 见 那 个 人 说 : 汪 洋 海 远 , 水 势 连 天 。
祥 光 笼 罩 宇 宙 , 瑞 气 照 山 川 。
千 层 雪 浪 吼 叫 青 霄 , 万 叠 烟 波 滔 滔 白 昼 。
水 飞 四 野 , 浪 翻 四 周 。
水 飞 四 野 震 雷 , 波 浪 汹 涌 周 围 , 吼 吼 霹 雳 。
不 要 说 水 势 , 姑 且 看 中 间 。
五 色 朦 胧 的 宝 贝 堆 积 在 山 上 , 红 、 黄 、 紫 、 皂 、 绿 、 和 、 蓝 。
才 看 到 观 音 真 是 胜 境 , 试 看 南 海 落 伽 山 。
是 个 好 的 地 方 , 山 峰 高 耸 , 峰 顶 透 透 空 中 。
中 间 有 千 种 奇 花 , 百 种 瑞 草 。
风 吹 动 宝 树 , 日 映 金 莲 。
观 音 殿 , 瓦 盖 琉 璃 ; 潮 音 洞 , 门 铺 上 玳 瑁 。
绿 杨 影 里 说 鹦 鹉 , 紫 竹 林 中 啼 孔 雀 。
罗 纹 石 上 , 护 法 威 严 ; 玛 瑙 滩 前 , 木 叉 雄 壮 。
此 行 的 人 看 不 尽 那 奇 异 的 景 色 , 径 直 按 着 云 头 , 来 到 竹 林 下 面 。
早 有 诸 天 迎 接 道 : 菩 萨 先 前 对 众 人 说 大 圣 归 善 , 非 常 宣 扬 。
现 在 保 护 唐 僧 , 为 什 么 还 有 空 闲 到 这 里 呢 ? 行 者 说 : 因 为 保 护 唐 僧 , 路 上 遇 到 一 件 事 , 特 地 见 到 菩 萨 , 请 给 我 通 报 。
于 是 诸 天 就 来 到 洞 口 报 告 , 菩 萨 把 他 叫 进 去 。
行 走 的 人 遵 照 法 律 行 事 , 到 宝 莲 台 下 拜 了 。
菩 萨 问 道 : 你 来 干 什 么 ? 行 者 说 : 我 师 父 路 上 遇 到 你 的 禅 院 , 你 接 受 了 人 间 香 火 , 容 下 一 个 黑 熊 精 在 那 里 住 , 他 偷 了 我 师 父 袈 裟 , 多 次 索 讨 不 给 , 现 在 特 来 问 你 要 的 。
菩 萨 说 : 这 个 猴 子 说 话 , 这 些 人 没 有 什 么 样 子 。
既 然 是 熊 精 偷 了 你 的 袈 裟 , 你 为 什 么 来 问 我 讨 伐 , 都 是 你 这 个 孽 猴 大 胆 , 把 宝 贝 卖 弄 , 拿 给 小 人 看 见 , 你 又 行 凶 , 呼 风 发 火 , 烧 我 的 留 云 下 院 , 反 而 来 我 处 放 刁 。
行 者 见 菩 萨 说 出 这 样 的 话 , 知 道 他 知 道 过 去 和 未 来 的 事 情 , 慌 忙 礼 拜 道 : 菩 萨 , 请 恕 弟 子 的 罪 过 , 果 然 是 这 样 的 。
只 恨 那 怪 物 不 肯 给 我 袈 裟 , 师 父 又 要 念 那 话 儿 的 咒 语 , 老 孙 忍 不 得 头 疼 , 所 以 来 拜 麻 烦 菩 萨 。
希 望 菩 萨 慈 悲 之 心 , 帮 助 我 去 捉 那 妖 精 , 取 出 衣 服 西 进 。
菩 萨 说 : 那 怪 物 有 许 多 神 通 , 却 不 亚 于 你 。
也 罢 , 我 看 唐 僧 的 脸 上 , 和 你 去 走 一 遭 。
行 路 的 人 听 到 了 这 话 , 感 谢 恩 德 , 再 次 拜 谢 。
就 请 菩 萨 出 门 , 于 是 一 同 驾 着 祥 云 , 早 晨 到 了 黑 风 山 , 坠 落 在 云 头 上 , 依 着 路 去 找 洞 。
正 走 的 地 方 , 只 见 那 山 坡 前 走 出 一 个 道 人 , 手 里 拿 着 一 个 玻 璃 盘 子 , 盘 里 放 着 两 粒 仙 丹 , 往 前 走 。
被 行 人 打 得 满 怀 , 拿 出 棒 子 , 就 照 着 头 一 下 , 打 得 脑 袋 里 的 水 流 出 来 , 口 中 的 血 迸 出 来 。
菩 萨 大 惊 道 : 你 这 个 猴 子 , 还 是 这 样 放 尿 。
他 又 不 曾 偷 你 袈 裟 , 又 不 与 你 认 识 , 又 没 有 什 么 冤 仇 , 你 为 什 么 就 把 他 打 死 ? 行 者 说 : 你 认 不 到 , 他 是 那 黑 熊 精 的 朋 友 。
其 余 皆 不 知 其 所 以 知 之 。
后 日 是 黑 精 的 生 日 , 请 他 们 来 庆 贺 佛 衣 会 。
今 天 他 先 来 祝 寿 , 明 天 来 庆 贺 佛 衣 会 。
所 以 我 认 得 , 必 定 是 今 天 替 那 妖 去 祝 寿 。
菩 萨 说 : 既 然 是 这 样 说 来 的 , 也 就 算 了 。
行 人 刚 去 , 把 那 个 道 人 提 起 来 看 , 却 是 一 只 苍 狼 。
旁 边 的 盘 子 底 下 有 字 , 刻 道 : 凌 虚 子 制 。
行 者 见 了 , 笑 着 说 : 造 化 , 造 化 , 老 孙 是 便 利 , 菩 萨 是 节 省 精 力 。
这 怪 叫 做 不 打 自 招 , 那 怪 教 他 今 天 的 丑 恶 。
菩 萨 说 : 菩 萨 , 我 觉 得 空 有 一 句 话 , 叫 做 将 计 就 计 , 不 知 道 菩 萨 可 以 依 靠 我 ?
行 者 说 : 菩 萨 , 你 看 这 盘 子 中 是 两 粒 仙 丹 , 就 是 我 们 与 那 妖 魔 的 见 ; 这 盘 子 后 面 刻 的 四 个 字 , 说 是 凌 虚 子 制 , 就 是 我 们 与 那 妖 魔 的 勾 头 。
菩 萨 如 果 要 依 从 我 的 时 候 , 我 就 好 替 你 做 计 较 , 就 不 必 动 用 干 戈 , 也 不 必 劳 动 战 争 , 妖 魔 眼 下 遭 到 瘟 疫 , 佛 衣 眼 下 就 会 出 现 ; 如 果 菩 萨 如 果 不 依 从 我 的 时 候 , 菩 萨 往 西 去 , 我 悟 空 去 东 去 , 佛 衣 只 会 相 送 , 唐 三 藏 只 会 落 在 空 中 。
菩 萨 笑 着 说 : 这 个 猴 子 熟 嘴 。
行 人 说 : 不 敢 , 倒 是 一 个 计 较 。
菩 萨 说 : 你 这 个 计 较 怎 么 说 ? 行 者 说 : 这 盘 上 刻 着 那 个 凌 虚 子 的 制 作 , 想 来 这 个 道 人 就 叫 做 凌 虚 子 。
菩 萨 , 你 要 依 从 我 的 时 候 , 可 以 变 成 这 个 道 人 , 我 把 这 丹 吃 了 一 粒 , 再 换 上 一 粒 , 略 大 些 小 些 。
菩 萨 , 你 就 捧 了 这 个 盘 子 、 两 粒 仙 丹 , 去 给 那 妖 祝 寿 , 把 那 丸 大 的 让 给 那 妖 。
等 到 那 妖 人 一 口 吞 下 它 , 老 孙 就 从 里 面 取 出 事 情 , 说 : 他 如 果 不 肯 献 出 佛 衣 , 老 孙 就 把 它 的 肚 子 吃 掉 了 , 织 了 一 件 出 来 。
菩 萨 没 法 , 只 得 点 点 头 儿 依 他 。
行 者 笑 着 说 : 怎 么 样 啊 这 时 , 菩 萨 就 以 广 大 慈 悲 , 无 边 法 力 , 千 万 化 身 , 以 心 来 与 意 相 合 , 以 意 来 与 身 相 会 , 恍 恍 之 间 , 变 成 凌 虚 仙 子 。
苍 颜 松 柏 老 了 , 秀 色 古 今 没 有 。
去 去 还 没 有 固 定 , 如 同 自 有 不 同 。
总 来 归 一 个 方 法 , 只 是 隔 着 我 的 身 躯 。
行 者 看 了 看 后 说 : 妙 呵 , 妙 呵 , 还 是 妖 精 菩 萨 , 还 是 菩 萨 妖 精 啊 菩 萨 笑 着 说 : 悟 空 , 菩 萨 、 妖 精 , 总 是 一 个 念 头 , 如 果 论 说 自 己 的 本 来 , 都 属 于 没 有 。
行 人 顿 悟 , 转 身 就 变 成 了 一 粒 仙 丹 。 走 盘 没 有 不 定 的 , 圆 明 没 有 方 。
三 三 勾 漏 合 , 六 六 少 翁 商 。
瓦 黄 金 焰 , 牟 尼 白 昼 光 。
外 边 的 铅 和 磁 , 不 许 容 易 商 量 。
行 人 改 变 了 那 个 丹 , 始 终 还 是 大 了 些 。
菩 萨 认 定 了 , 拿 了 那 个 玻 璃 盘 子 , 径 直 到 妖 洞 门 口 去 看 时 , 果 然 是 : 山 崖 深 邃 险 峻 , 云 彩 生 于 岭 上 ; 柏 树 苍 苍 , 松 树 翠 绿 , 风 吹 到 林 间 。
山 崖 深 邃 险 峻 , 果 然 是 妖 邪 出 没 , 人 烟 稀 少 ; 柏 树 苍 苍 松 树 翠 , 也 可 以 说 是 仙 人 修 炼 修 炼 修 炼 修 炼 隐 藏 的 道 理 情 趣 很 多 。
山 上 有 涧 水 , 涧 中 有 泉 水 , 潺 潺 流 水 , 呜 鸣 的 琴 声 , 就 可 以 洗 耳 ; 山 崖 上 有 鹿 , 林 林 中 有 鹤 , 幽 深 的 仙 籁 动 入 空 中 , 也 可 以 欣 赏 心 境 。
这 是 妖 仙 有 分 别 降 菩 提 , 弘 誓 无 边 垂 隐 隐 。
菩 萨 看 了 , 心 里 暗 喜 道 : 这 个 孽 畜 占 了 这 座 山 洞 , 却 是 也 有 道 分 。
因 此 心 中 已 经 有 了 慈 悲 。
走 到 洞 口 , 只 见 守 洞 的 小 妖 都 认 得 道 : 凌 虚 仙 长 来 了 。
一 边 传 报 , 另 一 边 接 引 。
那 妖 人 早 已 出 门 迎 接 , 说 道 : 凌 虚 , 有 劳 仙 驾 珍 惜 , 蓬 有 辉 。
菩 萨 说 : 小 道 敬 献 一 粒 仙 丹 , 敢 称 千 寿 。
其 他 两 个 人 拜 谢 完 毕 , 方 才 坐 定 , 又 叙 述 起 他 昨 天 的 事 。
菩 萨 不 回 答 , 连 忙 拿 着 丹 盘 说 : 大 王 , 暂 且 看 见 小 道 浅 陋 的 意 思 。
看 定 一 粒 大 的 , 推 给 那 妖 说 : 希 望 大 王 千 寿 。
那 妖 也 推 了 一 粒 米 , 递 给 菩 萨 说 : 我 愿 意 和 凌 虚 子 一 起 吃 。
说 完 , 那 妖 人 刚 要 咽 , 那 药 顺 口 儿 一 直 往 下 。
显 现 了 本 相 , 道 理 兴 起 四 方 。
那 妖 怪 就 倒 在 地 上 。
菩 萨 现 出 相 貌 , 问 妖 怪 , 取 了 佛 衣 。
行 人 早 已 从 鼻 孔 中 出 去 。
菩 萨 又 怕 那 妖 人 无 礼 , 却 把 一 个 钳 子 扔 在 那 妖 人 头 上 。
那 妖 魔 起 来 , 提 枪 要 刺 , 行 者 、 菩 萨 早 已 起 身 在 空 中 , 将 真 言 念 起 来 。
其 他 怪 物 仍 然 头 痛 , 丢 了 枪 , 满 地 乱 撒 。
半 空 里 笑 倒 了 个 美 猴 王 , 平 地 下 翻 倒 了 一 个 黑 熊 怪 。
菩 萨 说 : 孽 畜 , 你 现 在 可 以 皈 依 吗 ? 那 怪 满 口 说 : 我 愿 意 念 念 皈 依 , 只 希 望 饶 命 。
行 人 恐 怕 耽 误 了 工 夫 , 想 去 打 。
菩 萨 急 忙 止 住 说 : 不 要 伤 害 他 人 的 性 命 , 我 还 有 用 别 的 地 方 吗 ?
行 者 说 : 这 样 的 怪 物 , 不 打 死 他 , 反 而 留 他 在 哪 里 用 呢 ? 菩 萨 说 : 我 那 落 伽 山 后 面 没 有 人 看 管 , 我 要 带 他 去 做 个 守 山 大 神 。
行 者 笑 着 说 : 果 真 是 这 个 救 苦 慈 尊 , 一 个 灵 魂 都 不 会 受 到 损 害 。
如 果 是 老 孙 子 有 这 样 的 咒 语 , 就 念 上 他 娘 千 遍 。
这 个 儿 子 就 有 许 多 黑 熊 , 都 教 他 了 帐 。
又 说 : 那 怪 已 经 醒 了 好 多 时 , 公 道 难 以 禁 止 痛 痛 , 只 得 跪 在 地 下 哀 告 说 : 只 要 饶 你 的 性 命 , 愿 意 皈 依 佛 法 。
菩 萨 刚 刚 落 入 祥 光 , 又 与 他 摩 顶 受 戒 , 教 他 手 持 长 枪 , 随 从 左 右 。
那 黑 熊 才 是 一 片 野 心 今 天 定 下 来 , 无 穷 的 顽 顽 的 性 情 就 在 这 时 收 拾 。
菩 萨 吩 咐 道 : 悟 空 , 你 回 去 吧 , 好 生 伏 侍 唐 僧 , 以 后 再 休 息 , 懈 怠 生 事 。
行 者 说 : 我 深 深 感 念 菩 萨 远 道 而 来 , 弟 子 回 去 当 回 送 回 去 。
菩 萨 说 : 不 要 送 你 。
行 人 才 捧 着 袈 裟 , 叩 头 告 别 。
菩 萨 也 带 着 熊 , 径 直 回 到 大 海 。
有 诗 作 证 。
诗 说 : 祥 光 霭 霭 凝 金 像 , 万 道 缤 纷 实 可 夸 。
普 济 世 人 , 慈 爱 抚 恤 , 遍 观 法 界 现 金 莲 。
现 在 来 的 多 是 为 了 传 经 的 意 思 , 此 去 的 原 本 没 有 点 点 瑕 疵 。
降 鬼 成 真 归 大 海 , 空 门 又 得 到 锦 袈 裟 。
最 终 不 知 道 以 后 事 情 的 情 况 如 何 , 暂 且 听 下 回 分 析 解 释 。
}\switchcolumn\flushpage  \begin{pinyinscope}{\myfontt \section{第十八回}     觀音院唐僧脫難 高老莊大聖除魔

行者辭了菩薩,按落雲頭,將袈裟掛在香柟樹上,掣出棒來,打入黑風洞裏,那
洞裏那得一個小妖。原來是他見菩薩出現,降得那老怪就地打滾,急急都散走了
。行者一發行兇,將他那幾層門上都積了乾柴,前前後後,一齊發火,把個黑風
洞燒做個紅風洞,卻拿了袈裟,駕祥光,轉回直北。
  
話說那三藏望行者急忙不來,心甚疑惑:不知是請菩薩不至,不知是行者託故而
逃。正在那胡猜亂 想之中,只見半空中彩霧燦燦,行者忽墜階前跪道:「師父,
袈裟來了。」三藏大喜。眾僧亦無不歡悅道:「好了,好了,我等性命今日方才
得全了。」三藏接了袈裟道:「悟空,你早間去時,原約到飯罷晌午,如何此時
日西方回?」行者將那請菩薩施變化降妖的事情,備陳了一遍。三藏聞言,遂設
香案,朝南禮拜罷,道:「徒弟呵,既然有了佛衣,可快收拾包裹去也。」行者
道:「莫忙,莫忙。今日將晚,不是走路的時候,且待明日早行。」眾僧們一齊
跪下道:「孫老爺說得是。一則天晚,二來我等有些願心兒,今幸平安,有了寶
貝,待我還了願,請老爺散了福,明早再送西行。」行者道:「正是,正是。」
你看那些和尚都傾囊倒底,把那火裏搶出的餘資,各出所有,整頓了些齋供,燒
了些平安無事的紙,念了幾卷消災解厄的經。當晚事畢。
  
次早,方刷扮了馬匹,包裹了行囊出門,眾僧遠送方回。行者引路而去,正是那
春融時節,但見那:
    草襯玉驄蹄跡軟,柳搖金線露華新。
    桃杏滿林爭豔麗,薜蘿遶徑放精神。
    沙堤日暖鴛鴦睡,山澗花香蛺蝶馴。
    這般秋去冬殘春過半,不知何年行滿得真文。

  
師徒們行了五七日荒路,忽一日天色將晚,遠遠的望見一村人家。三藏道:「悟
空,你看那壁廂有座山莊相近,我們去告宿一宵,明日再行何如?」行者道:
「且等老孫去看看吉凶,再作區處。」那師父挽住絲韁,這行者定睛觀看,真個
是:
竹籬密密,茅屋重重。參天野樹迎門,曲水溪橋映戶。道傍楊柳綠依依,園內花
開香馥馥。此時那夕照沉西,處處山林喧鳥雀;晚煙出爨,條條道徑轉牛羊。又
見那食飽雞豚眠屋角,醉酣鄰叟唱歌來。
  
行者看罷道:「師父請行,定是一村好人家,正可借宿。」那長老催動白馬,早
到街衢之口。又見一個少年,頭裹綿布,身穿藍襖,持傘背包,斂褌劄褲,腳踏
著一雙三耳草鞋,雄糾糾的,出街忙走。行者順手一把扯住道:「那裏去?我問
你一個信兒:此間是甚麼地方?」那個人只管苦掙,口裏嚷道:「我莊上沒人,
只是我好問信?」行者陪著笑道:「施主莫惱。『與人方便,自己方便。』你就
與我說說地名何害?我也可解得你的煩惱。」那人掙不脫手,氣得亂跳道:「蹭
蹬,蹭蹬。家長的屈氣受不了,又撞著這個光頭,受他的清氣。」行者道:「你
有本事,劈開我的手,你便就去了也罷。」那人左扭右扭,那裏扭得動,卻似一
把鐵鈐拑住一般。氣得他丟了包袱,撇了傘,兩隻手雨點似來抓行者。行者把一
隻手扶著行李,一隻手抵住那人,憑他怎麼支吾,只是不能抓著。行者愈加不放
,急得爆燥如雷。三藏道:「悟空,那裏不有人來了?你再問那人就是,只管扯
住他怎的?放他去罷。」行者笑道:「師父不知,若是問了別人沒趣,須是問他
,才有買賣。」那人被行者扯住不放,只得說出道:「此處乃是烏斯藏國界之地
,喚做高老莊。一莊人家有大半姓高,故此喚做高老莊。你放了我去罷。」行者
又道:「你這樣行裝,不是個走近路的。你實與我說,你要往那裏去,端的所幹
何事,我才放你。」
  
這人無奈,只得以實情告訴道:「我是高太公的家人,名叫高才。我那太公有個
老女兒,年方二十歲,更不曾配人。三年前被一個妖精占了,那妖整做了這三年
女婿。我太公不悅,說道:『女兒招了妖精,不是長法:一則敗壞家門,二則沒
個親家來往。』一向要退這妖精。那妖精那裏肯退,轉把女兒關在他後宅,將有
半年,再不放出與家內人相見。我太公與了我幾兩銀子,教我尋訪法師,拿那妖
怪。我這些時不曾住腳,前前後後,請了有三四個人,都是不濟的和尚,膿包的
道士,降不得那妖精。剛才罵了我一場,說我不會幹事。又與了我五錢銀子做盤
纏,教我再去請好法師降他。不期撞著你這個紇刺星扯住,誤了我走路,故此裏
外受氣,我無奈,才與你叫喊。不想你又有些拿法,我掙不過你,所以說此實情
。你放我走罷。」行者道:「你的造化,我有營生,這才是湊四合六的勾當。你
也不須遠行,莫要花費了銀子。我們不是那不濟的和尚,膿包的道士,其實有些
手段,慣會拿妖。這正是:『一來照顧郎中,二來又醫得眼好。』煩你回去上覆
你那家主,說我們是東土駕下差來的御弟聖僧,往西天拜佛求經者,善能降妖縛
怪。」高才道:「你莫誤了我。我是一肚子氣的人,你錯哄了我,沒甚手段,拿
不住那妖精,卻不又帶累我來受氣?」行者道:「管教不誤了你,你引我到你家
門首去來。」那人也無計奈何,真個提著包袱,拿了傘,轉步回身,領他師徒到
於門首道:「二位長老,你且在馬臺上略坐坐,等我進去報主人知道。」行者才
放了手,落擔牽馬,師徒們坐立門傍等候。
  
那高才入了大門,徑往中堂上走,可可的撞見高太公。太公罵道:「你那個蠻皮
畜生!怎麼不去尋人,又回來做甚 ?」高才放下包、傘道:「上告主人公得知:
小人才行出街口,忽撞見兩個和尚:一個騎馬,一個挑擔。他扯住我不放,問我
那裏去。我再三不曾與他說及,他纏得沒奈何,不得脫手,遂將主人公的事情,
一一說與他知。他卻十分歡喜,要與我們拿那妖怪哩。」高老道:「是那裏來的
?」高才道:「他說是東土駕下差來的御弟聖僧,前往西天拜佛求經的。」太公
道:「既是遠來的和尚,怕不真有些手段。他如今在那裏?」高才道:「現在門
外等候。」
  
那太公即忙換了衣服,與高才出來迎接,叫聲:「長老。」三藏聽見,急轉身,
早已到了面前。那老者戴一頂烏綾巾,穿一領蔥白蜀錦衣,踏一雙糙米皮的犢子
靴,繫一條黑綠絛子,出來笑語相迎,便叫:「二位長老,作揖了。」三藏還了
禮,行者站著不動。那老者見他相貌兇醜,便就不敢與他作揖。行者道:「怎麼
不唱老孫喏?」那老兒有幾分害怕,叫高才道:「你這小廝卻不弄殺我也?家裏
現有一個醜頭怪腦的女婿打發不開,怎麼又引這個雷公來害我?」行者道:「老
高,你空長了許大年紀,還不省事。若專以相貌取人,乾淨錯了。我老孫醜自醜
,卻有些本事。替你家擒得妖精,捉得鬼魅,拿住你那女婿,還了你女兒,便是
好事,何必諄諄以相貌為言?」太公見說,戰兢兢的,只得強打精神,叫聲:
「請進。」這行者見請,才牽了白馬,教高才挑著行李,與三藏進去。他也不管
好歹,就把馬拴在敞廳柱上,扯過一張退光漆交椅,叫師父坐下。他又扯過一張
椅子,坐在傍邊。那高老道:「這個小長老,倒也家懷。」行者道:「你若肯留
我住得半年,還家懷哩。」
  
坐定,高老問道:「適間小价說,二位長老是東土來的?」三藏道:「便是。貧
僧奉朝命往西天拜佛求經,因過寶莊,特借一宿,明日早行。」高老道:「二位
原是借宿的,怎麼說會拿怪?」行者道:「因是借宿,順便拿幾個妖怪兒耍耍的
。動問府上有多少妖怪?」高老道:「天哪!還吃得有多少哩,只這一個妖怪女
婿,已被他磨慌了。」行者道:「你把那妖怪的始末,有多大手段,從頭兒說說
我聽,我好替你拿他。」高老道:「我們這莊上,自古至今,也不曉得有甚麼鬼
祟魍魎,邪魔作耗。只是老拙不幸,不 曾有子,止生三個女兒:大的喚名香蘭,
第二的名玉蘭,第三的名翠蘭。那兩個從小兒配與本莊人家。止有小的個要招個
女婿,指望他與我同家過活,做個養老女婿,撐門抵戶,做活當差。不期三年前
,有 一個漢子,模樣兒倒也精緻。他說是福陵山上人家,姓豬,上無父母,下
無兄弟,願與人家做個女婿。我老拙見是這般一個無根無絆的人,就招了他。一
進門時,倒也勤謹:耕田耙地,不用牛具;收割田禾,不用刀杖;昏去明來,其
實也好。只是一件,有些會變嘴臉。」行者道:「怎麼樣變?」高老道:「初來
時是一條黑胖漢,後來就變做一個長嘴大耳朵的獃子,腦後又有一溜鬃毛,身體
粗糙怕人,頭臉就像個豬的模樣。食腸卻又甚大:一頓要吃三五斗米飯,早間點
心也得百十個燒餅才勾。喜得還吃齋素;若再吃葷酒,便是老拙這些家業田產之
類,不上半年,就吃個罄淨。」三藏道:「只因他做得,所以吃得。」高老道:
「吃還是件小事。他如今又會弄風,雲來霧去,走石飛砂,諕得我一家並左鄰右
舍,俱不得安生。又把那翠蘭小女關在後宅子裏,一發半年也不曾見面,更不知
死活如何。因此知他是個妖怪,要請個法師與他去退去退。」
  
行者道:「這個何難?老兒你管放心,今夜管情與你拿住,教他寫了退親文書,
還你女兒如何?」高老大喜道:「我為招了他不打緊,壞了我多少清名,疏了我
多少親眷。但得拿住他,要甚麼文書?就煩與我除了根罷。」行者道:「容易,
容易。入夜之時,就見好歹。」
  
老兒十分歡喜,才教展抹桌椅,擺列齋供。齋罷將晚,老兒問道:「要甚兵器?
要多少人隨?趁早好備。」行者道:「兵器我自有。」老兒道:「二位只是那根
錫杖, 錫杖怎麼打得那個妖精?」行者隨於耳內取出一個繡花針來,捻在手中,
迎 風幌了一幌,就是碗來粗細的一根金箍鐵棒,對著高老道:「你看這條棍子,
比你家兵器如何?可打得這怪否?」高老又道:「既有兵器,可要人跟?」行者
道:「我不用人,只是要幾個年高有德的老兒,陪我師父清坐閑敘,我好撇他而
去。等我把那妖精拿來,對眾取供,替你除了根罷。」那老兒即喚家僮,請了幾
個親故朋友。一時都到,相見已畢,行者道:「師父, 你放心穩坐,老孫去也。」
  
你看他揝著鐵棒,扯著高老道:「你引我去後宅子裏妖精的住處看看。」高老遂
引他到後宅門首。行者道:「你去取鑰匙來。」高老道:「你且看看,若是用得
鑰匙,卻不請你了。」行者笑道:「你那老兒年紀雖大,卻不識耍。我把這話兒
哄你一哄,你就當真。」走上前,摸了一摸,原來是銅汁灌的鎖子。狠得他將金
箍棒一搗,搗開門扇,裏面卻黑洞洞的。行者道:「老高,你去叫你女兒一聲,
看他可在裏面?」那老兒硬著膽叫道:「三姐姐!」那女兒認得是他父親的聲音
,才少氣無力的應了一聲道:「爹爹,我在這裏哩。」行者閃金睛,向黑影裏仔
細看時,你道他怎生模樣?但見那:
雲鬢亂堆無掠,玉容未洗塵淄。一片蘭心依舊,十分嬌態傾頹。櫻唇全無氣血,
腰肢屈屈偎偎。愁蹙蹙,蛾眉淡;瘦怯怯,語聲低。
  
他走來看見高老,一把扯住,抱頭大哭。行者道:「且莫哭,且莫哭。我問你,
妖怪往那裏去了?」女子道:「不知往那裏去。這些時,天明就去,入夜方來。
雲雲霧霧,往回不知何所。因是曉得父親要祛退他,他也常常防備,故此昏來朝
去。」行者道:「不消說了。老兒,你帶令愛往前邊宅裏,慢慢的敘闊,讓老孫
在此等他。他若不來,你卻莫怪;他若來了,定與你剪草除根。」那老高歡歡喜
喜的把女兒帶將前去。
  
行者卻弄神通,搖身一變,變得就如那女子一般,獨自個坐在房裏等那妖精。不
多時,一陣風來,真個是走石飛砂。好風:
    起初時微微蕩蕩,向後來渺渺茫茫。
    微微蕩蕩乾坤大,渺渺茫茫無阻礙。
    凋花折柳勝揌麻,倒樹摧林如拔菜。
    翻江攪海鬼神愁,裂石崩山天地怪。
    啣花糜鹿失來蹤,摘果猿猴迷在外。
    七層鐵塔侵佛頭,八面幢幡傷寶蓋。
    金梁玉柱起根搖,房上瓦飛如燕塊。
    舉棹梢公許願心,開船忙把豬羊賽。
    當坊土地棄祠堂,四海龍王朝上拜。
    海邊撞損夜叉船,長城刮倒半邊塞。

  
那陣狂風過處,只見半空裏來了一個妖精,果然生得醜陋:黑臉短毛,長喙大耳
;穿一領青不青、藍不藍的梭布直裰,繫一條花布手巾。行者暗笑道:「原來是
這個買賣。」好行者,卻不迎他,也不問他,且睡在床上推病,口裏哼哼嘖嘖的
不絕。那怪不識真假,走進房,一把摟住,就要親嘴。行者暗笑道:「真個要來
弄老孫哩。」即使個拿法,托著那怪的長嘴,叫做個小跌。漫頭一料,撲的摜下
床來。那怪爬起來,扶著床邊道:「姐姐,你怎麼今日有些怪我?想是我來得遲
了?」行者道:「不怪,不怪。」那妖道:「既不怪我,怎麼就丟我這一跌?」
行者道:「你怎麼就這等樣小家子,就摟我親嘴?我因今日有些不自在;若每常
好時,便起來開門等你了。你可脫了衣服睡是。」那怪不解其意,真個就去脫衣
。行者跳起來,坐在淨桶上。那怪依舊復來床上摸一把,摸不著人,叫道:「姐
姐,你往那裏去了?請脫衣服睡罷。」行者道:「你先睡,等我出個恭來。」那
怪果先解衣上床。
  
行者忽然嘆口氣,道聲:「造化低了。」那怪道:「你惱怎的?造化怎麼得低的
?我得到了你家,雖是吃了些茶飯,卻也不曾白吃你的:我也曾替你家掃地通溝
、搬磚運瓦、築土打牆、耕田耙地、種麥插秧、創家立業。如今你身上穿的錦,
戴的金,四時有花果享用,八節有蔬菜烹煎,你還有那些兒不趁心處,這般短嘆
長吁,說甚麼造化低了?」行者道:「不是這等說。今日我的父母隔著牆,丟磚
料瓦的,甚是打我罵我哩。」那怪道:「他打罵你怎的?」行者道:「他說我和
你做了夫妻,你是他門下一個女婿,全沒些兒禮體。這樣個醜嘴臉的人,又會不
得姨夫,又見不得親戚,又不知你雲來霧去,端的是那裏人家,姓甚名誰,敗壞
他清德,玷辱他門風,故此這般打罵,所以煩惱。」那怪道:「我雖是有些兒醜
陋,若要俊,卻也不難。我一來時,曾與他講過,他願意方才招我。今日怎麼又
說起這話?我家住在福陵山雲棧洞。我以相貌為姓,故姓豬,官名叫做豬剛鬣。
他若再來問你,你就以此話與他說便了。」
  
行者暗喜道:「那怪卻也老實,不用動刑,就供得這等明白。既有了地方、姓名
,不管怎的也拿住他。」行者道:「他要請法師來拿你哩。」那怪笑道:「睡著
, 睡著,莫睬他。我有天罡數的變化,九齒的釘鈀,怕甚麼法師、和尚、道士
?就是你老子有虔心,請下九天蕩魔祖師下界,我也曾與他做過相識,他也不敢
怎的我 。」行者道:「他說請一個五百年前大鬧天宮姓孫的齊天大聖,要來拿
你哩。」那怪聞得這個名頭,就有三分害怕道:「既是這等說,我去了罷,兩口
子做不成了。」行者道:「你怎的就去?」那怪道:「你不知道,那鬧天宮的弼
馬溫有些本事,只恐我弄他不過,低了名頭,不像模樣。」
  
說罷,套上衣服,開了門,往外就走。被行者一把扯住,將自己臉上抹了一抹,
現出原身,喝道:「好妖怪,那裏走!你抬頭看看我是那個?」那怪轉過眼來,
看見行者咨牙?嘴,火眼金睛,磕頭毛臉,就是個活雷公相似。慌得他手麻腳軟
,劃剌的一聲,掙破了衣服,化狂風脫身而去。行者急上前,掣鐵棒,望風打了
一下。那怪化萬道火光,徑轉本山而去。行者駕雲,隨後趕來,叫聲:「那裏走
!你若上天,我就趕到斗牛宮;你若入地,我就追至枉死獄。」
  
    咦!畢竟不知這一去趕至何方,有何勝敗,且聽下回分解。





}  \end{pinyinscope}\switchcolumn{\myfontc \section{第 十 八 回} 观 音 院 , 唐 僧 脱 难 高 老 庄 大 圣 除 魔 行 者 辞 别 菩 萨 , 打 落 在 云 头 上 , 把 袈 裟 挂 在 香 树 上 , 拿 出 棒 来 , 打 进 黑 风 洞 里 , 那 个 洞 里 怎 么 得 到 一 个 小 妖 。
原 来 是 他 见 到 菩 萨 出 现 , 投 降 了 那 个 老 怪 , 就 地 打 滚 , 急 忙 都 逃 走 了 。
行 人 一 发 , 行 凶 , 将 其 他 几 层 门 上 都 堆 了 干 柴 , 前 前 后 后 后 , 一 齐 发 火 , 把 黑 风 洞 烧 成 红 风 洞 , 又 拿 出 袈 裟 , 驾 着 祥 光 , 转 回 直 北 。
话 说 : 那 三 藏 菩 萨 望 见 行 者 急 忙 不 来 , 心 里 很 疑 惑 , 说 : 不 知 道 是 请 菩 萨 不 来 , 不 知 道 是 行 者 托 故 逃 走 了 。
正 在 那 胡 猜 乱 想 之 中 , 只 见 半 空 中 彩 雾 闪 闪 , 走 路 的 人 忽 然 掉 下 台 阶 前 跪 着 说 : 师 父 , 袈 裟 来 了 。
三 藏 大 喜 。
众 僧 人 也 无 不 欢 喜 地 说 : 好 了 好 了 , 好 了 , 我 们 的 性 命 今 天 才 得 以 保 全 。
三 藏 接 了 袈 裟 , 说 : 悟 空 , 你 早 早 离 去 时 , 原 约 到 饭 罢 中 午 , 为 什 么 这 时 日 西 方 回 来 ? 行 者 将 请 菩 萨 施 变 化 降 妖 的 事 情 , 全 都 陈 述 了 一 遍 。
三 藏 听 了 , 就 摆 好 香 案 , 朝 南 礼 拜 完 毕 , 说 : 徒 弟 呵 , 既 然 有 了 佛 衣 , 可 以 快 点 收 拾 包 裹 去 。
行 人 说 : 不 要 忙 , 不 要 忙 。
今 天 快 晚 , 不 是 走 路 的 时 候 , 暂 且 等 到 明 天 早 早 上 路 。
众 僧 人 一 齐 跪 下 说 : 孙 老 爷 说 得 对 。
一 则 天 晚 , 二 来 我 们 有 些 愿 意 的 儿 子 , 现 在 幸 好 平 安 , 有 了 宝 贝 , 等 我 回 去 了 愿 望 , 请 老 爷 分 散 福 分 , 明 早 再 送 我 西 行 。
行 人 说 : 正 是 , 正 是 。
你 看 那 些 和 尚 都 倾 心 倒 腹 , 把 火 里 夺 出 的 余 财 , 各 自 拿 出 所 有 , 整 顿 了 斋 供 , 烧 了 些 平 安 无 事 的 纸 , 念 了 几 卷 消 灾 解 厄 的 经 。
当 晚 事 情 结 束 。
第 二 天 早 晨 , 方 洗 装 了 马 匹 , 包 袱 着 行 李 出 门 , 众 僧 远 送 他 回 来 。
行 人 引 路 而 去 , 正 是 那 春 天 融 融 的 时 节 , 只 见 那 些 草 色 映 着 玉 , 蹄 迹 软 , 柳 枝 摇 动 着 金 线 露 出 的 光 彩 新 。
桃 、 杏 满 林 竞 相 艳 丽 , 、 、 、 、 、 、 、 、 、 、 、 、 、 、 、 、 、 、 、 、 、 、 、 、 、 、 、 、 、 、 、 、 、 、 、 、 、 、 、 、 、 、 、 、 、 、 、 、 、 、 、 、 、 、 、 、 、 、 、 游 、 放 精 神 。
沙 堤 日 暖 , 鸳 鸯 睡 着 , 山 涧 中 的 花 香 , 蝴 蝶 飞 舞 。
这 样 秋 去 冬 残 春 过 半 , 不 知 何 年 行 满 得 到 真 文 。
师 徒 们 走 了 五 七 天 荒 路 , 忽 然 有 一 天 天 色 将 晚 , 远 远 望 见 一 村 人 家 。
三 藏 说 : 悟 空 , 你 看 那 墙 边 有 座 山 庄 , 我 们 去 告 诉 我 们 住 宿 一 夜 , 明 天 再 去 , 怎 么 样 ? 行 人 说 : 暂 且 等 老 孙 去 看 看 吉 凶 , 再 作 区 区 的 地 方 。
那 个 师 父 挽 住 丝 , 这 个 行 人 定 眼 观 看 , 真 是 竹 篱 密 密 , 茅 屋 重 重 。
参 天 野 树 迎 门 , 曲 水 溪 桥 映 户 。
道 旁 的 杨 柳 绿 荫 依 依 , 园 里 的 花 开 , 香 气 馥 郁 。
此 时 那 晚 上 的 夕 阳 照 得 沉 西 , 处 处 山 林 喧 噪 鸟 雀 , 晚 上 的 烟 雾 飘 出 , 道 路 绕 过 牛 羊 。
又 看 见 那 些 吃 饱 鸡 豚 睡 在 屋 角 上 , 喝 得 醉 了 , 邻 叟 唱 歌 来 。
行 人 看 完 后 说 : 师 父 请 我 去 , 一 定 是 一 村 好 人 家 , 正 可 借 宿 。
那 位 长 老 催 促 着 白 马 , 早 早 到 了 街 道 的 口 。
又 见 一 个 少 年 , 头 裹 着 棉 布 , 身 穿 着 红 色 的 鞋 子 , 拿 着 伞 子 , 背 裹 着 裤 子 , 脚 踏 着 三 耳 草 鞋 , 雄 健 崛 起 , 出 了 街 上 忙 走 。
行 者 顺 手 一 把 握 住 道 : 哪 里 去 ? 我 问 你 一 个 信 儿 , 此 间 是 什 么 地 方 ? 那 个 人 只 管 苦 苦 地 挣 扎 , 口 里 喧 闹 道 : 我 家 庄 上 没 有 人 , 只 是 我 喜 欢 问 信 吗 ? 行 人 陪 着 笑 着 说 : 施 主 莫 恼 。
给 别 人 方 便 , 自 己 方 便 。
卿 以 此 为 是 , 我 以 此 为 是 , 我 也 可 以 解 除 你 的 烦 恼 。
那 个 人 挣 不 脱 手 , 气 得 乱 跳 道 : 踢 , 踢 踢 。
家 长 的 气 息 不 能 承 受 , 又 撞 着 他 的 光 头 , 接 受 他 的 清 气 。
行 人 说 : 你 有 本 事 , 劈 开 我 的 手 , 你 就 去 了 。
那 个 人 左 边 扭 着 右 边 扭 , 那 里 的 人 扭 得 动 了 , 却 好 像 一 把 铁 住 了 一 样 。
其 中 有 一 只 手 , 好 像 是 来 抓 走 走 的 人 。
行 者 扶 着 行 李 , 一 只 手 抵 住 那 个 人 , 凭 他 怎 么 支 撑 我 , 只 是 不 能 捉 住 。
走 路 的 人 更 加 不 放 , 急 忙 得 像 雷 鸣 一 样 迅 猛 。
三 藏 说 : 悟 空 , 那 里 不 有 人 来 了 , 你 再 问 那 人 就 是 , 只 管 拽 住 他 怎 么 样 , 放 他 去 罢 了 。
行 者 笑 着 说 : 师 父 不 知 道 , 如 果 是 问 了 别 人 没 有 理 由 , 必 须 问 他 , 才 有 买 卖 。
那 个 人 被 行 人 拽 住 不 放 , 只 得 说 出 道 : 这 里 是 乌 斯 藏 国 的 边 界 , 叫 做 高 老 庄 。
一 个 庄 人 家 中 有 一 个 姓 高 , 所 以 叫 它 高 老 庄 。
你 放 了 我 去 吧 !
行 人 又 说 : 你 这 样 行 装 , 不 是 个 走 近 路 的 。
你 实 在 跟 我 说 , 你 要 去 那 里 去 , 端 的 干 什 么 事 , 我 才 放 你 。
这 个 人 无 可 奈 何 , 只 得 以 实 情 告 诉 他 说 : 我 是 高 太 公 的 家 人 , 名 叫 高 才 。
我 的 太 公 有 个 老 女 儿 , 年 方 二 十 岁 , 更 不 曾 嫁 人 。
三 年 前 被 一 个 妖 精 占 了 , 那 妖 怪 完 全 做 了 这 三 年 的 女 婿 。
我 太 公 不 高 兴 , 说 道 : 女 儿 招 来 妖 精 , 不 是 长 久 的 法 则 : 一 是 败 坏 家 门 , 二 是 没 有 亲 家 来 往 。
我 一 向 要 斥 退 这 妖 精 。
那 妖 精 哪 里 肯 退 , 转 而 把 女 儿 关 在 他 的 后 宅 , 将 有 半 年 , 再 也 不 放 出 去 与 家 里 的 人 相 见 。
我 太 公 给 我 几 两 银 子 , 教 我 寻 找 法 师 , 拿 那 妖 怪 。
我 这 些 时 候 不 曾 停 下 脚 , 前 前 后 后 , 请 求 了 三 四 个 人 , 都 是 不 能 成 功 的 和 尚 , 口 中 的 道 士 , 投 降 得 不 到 那 妖 精 。
先 生 曰 : 我 不 会 做 事 。
又 给 我 五 钱 银 子 做 盘 缠 , 教 我 再 去 , 请 好 法 师 投 降 。
没 想 到 你 这 个 刺 星 拽 住 , 误 了 我 走 路 , 所 以 里 外 受 气 , 我 无 可 奈 何 , 才 和 你 一 道 叫 喊 。
不 想 你 又 有 些 拿 法 , 我 不 过 你 , 所 以 说 出 这 个 实 情 。
你 放 我 走 吧 !
行 者 说 : 你 的 造 化 , 我 有 营 生 , 这 才 是 凑 合 四 合 六 的 勾 当 。
君 子 不 必 要 远 行 , 不 要 浪 费 银 子 。
吾 不 能 救 , 而 不 能 救 , 而 不 能 救 。
这 正 是 说 : 一 来 照 顾 郎 中 , 二 来 又 医 得 眼 好 。
你 回 去 上 报 你 的 家 主 , 说 我 们 是 东 土 皇 帝 派 来 的 御 弟 圣 僧 , 去 西 天 拜 佛 求 经 的 人 , 善 于 降 魔 绑 鬼 。
高 才 说 : 你 不 要 误 了 我 。
我 是 一 肚 子 气 的 人 , 你 错 骗 了 我 , 没 有 什 么 手 段 , 拿 不 住 那 妖 精 , 还 不 是 又 带 累 我 来 受 气 吗 ? 行 者 说 : 管 教 不 误 了 你 , 你 引 我 到 你 家 门 前 去 来 。
其 人 无 计 奈 何 , 真 的 提 着 包 袱 , 拿 着 伞 , 转 转 回 身 , 领 着 他 的 师 徒 来 到 门 前 说 : 二 位 长 老 , 你 暂 且 在 马 台 上 稍 稍 坐 下 , 等 我 进 去 报 告 主 人 知 道 。
行 人 才 放 下 手 , 掉 下 担 子 牵 着 马 , 师 徒 们 坐 在 门 旁 等 候 。
高 才 进 入 大 门 , 径 直 往 中 堂 上 走 , 可 能 撞 见 高 太 公 。
太 公 骂 道 : 你 那 个 蛮 皮 畜 生 , 为 什 么 不 去 找 人 , 又 回 来 做 什 么 ? 高 才 放 下 包 袱 和 伞 说 : 上 告 主 人 公 , 得 知 道 我 小 人 刚 走 出 街 口 , 忽 然 遇 见 两 个 和 尚 : 一 个 骑 马 , 一 个 挑 担 子 。
其 他 人 把 我 放 了 , 问 我 到 哪 里 去 ?
我 再 三 不 曾 与 他 谈 及 , 他 缠 得 没 有 办 法 , 不 得 脱 手 , 于 是 将 主 人 公 的 事 情 全 都 告 诉 他 。
其 他 人 皆 非 常 欢 喜 , 是 要 与 我 们 一 起 拿 妖 怪 。
高 老 说 : 是 从 哪 里 来 的 ? 高 才 说 : 他 说 是 东 土 驾 下 派 来 的 御 弟 圣 僧 , 前 往 西 天 拜 佛 求 经 的 。
太 公 说 : 既 然 是 远 道 来 的 和 尚 , 恐 怕 不 真 有 些 手 段 。
高 才 说 : 现 在 在 门 外 等 候 。
太 公 立 即 换 了 衣 服 , 与 高 才 出 来 迎 接 , 大 叫 道 : 长 老 。
三 藏 听 了 , 急 忙 转 身 , 早 已 到 了 面 前 。
那 个 老 人 戴 着 一 顶 黑 绫 巾 , 穿 着 一 领 葱 白 的 蜀 锦 衣 , 踏 着 一 双 糙 米 皮 的 犊 子 靴 , 系 着 一 条 黑 绿 的 裤 子 , 出 来 笑 语 相 迎 , 便 大 叫 : 二 位 长 老 , 作 揖 了 。
三 藏 回 去 后 , 行 人 站 在 那 里 不 动 。
那 个 老 人 见 他 面 貌 凶 狠 丑 陋 , 就 不 敢 与 他 作 揖 。
那 个 老 子 有 几 分 害 怕 , 叫 高 才 说 : 你 这 个 小 孩 竟 不 戏 弄 杀 我 呀 家 里 现 在 有 一 个 丑 头 怪 脑 的 女 婿 , 打 发 不 开 , 怎 么 又 引 来 这 个 雷 公 来 害 我 呢 ? 行 人 说 : 老 高 , 你 空 长 了 许 多 大 年 纪 , 还 不 省 事 ?
如 果 专 门 以 相 貌 取 人 , 那 么 干 净 就 错 了 。
我 的 老 孙 子 丑 自 己 丑 , 却 有 些 本 事 。
替 你 家 擒 获 妖 精 , 捉 住 鬼 魅 , 拿 住 你 的 女 婿 , 还 给 你 的 女 儿 , 就 是 好 事 , 何 必 谆 谆 地 说 出 自 己 的 相 貌 ? 太 公 见 了 , 就 是 战 战 兢 兢 , 只 得 勉 强 打 精 神 , 叫 喊 道 : 请 进 前 。
这 个 行 人 请 求 , 才 牵 着 白 马 , 让 高 才 挑 着 行 李 , 和 三 藏 进 去 。
其 人 皆 不 知 其 人 , 而 不 知 其 人 , 则 不 知 其 人 , 则 不 知 其 人 , 则 不 知 其 人 , 则 不 知 其 人 , 则 不 知 其 人 , 则 不 知 其 人 。
其 又 有 一 人 , 不 敢 言 , 不 敢 言 。
那 高 老 说 : 这 个 小 长 老 , 倒 也 怀 念 家 乡 。
行 人 说 : 你 如 果 肯 留 我 住 半 年 , 还 家 怀 里 吧 。
坐 定 后 , 高 老 问 道 : 刚 才 有 个 小 价 人 说 , 二 位 长 老 是 从 东 土 来 的 , 三 藏 说 : 就 是 。
贫 僧 奉 朝 廷 的 命 令 前 往 西 天 拜 佛 求 经 , 因 而 路 过 宝 庄 , 特 地 借 了 一 宿 , 第 二 天 早 上 出 发 。
高 老 说 : 二 位 原 来 是 借 宿 的 , 为 什 么 说 会 拿 怪 呢 ? 行 人 说 : 因 为 是 借 宿 , 我 就 拿 几 个 妖 怪 儿 戏 耍 。
李 动 问 府 中 有 多 少 妖 怪 , 高 老 说 : 天 呀 , 还 吃 得 有 多 少 啊 , 只 有 这 个 妖 怪 女 婿 , 已 经 被 他 打 得 惊 慌 了 。
行 者 说 : 你 把 妖 怪 的 始 末 , 有 多 大 的 手 段 , 从 头 儿 说 我 听 , 我 好 替 你 拿 它 。
高 老 说 : 我 们 这 个 庄 子 上 , 自 古 至 今 , 也 不 知 道 有 什 么 鬼 怪 、 魍 、 妖 魔 作 祟 。
只 是 老 拙 不 幸 , 不 曾 生 下 儿 子 , 只 生 了 三 个 女 儿 : 大 的 叫 香 兰 , 第 二 个 叫 玉 兰 , 第 三 个 叫 翠 兰 。
两 个 从 小 儿 发 配 给 本 庄 人 家 。
只 有 小 的 人 , 要 招 个 女 婿 , 指 望 他 与 我 同 家 过 活 , 做 个 养 老 女 婿 , 撑 门 抵 户 , 做 活 当 差 。
不 料 三 年 前 , 有 一 个 汉 子 , 模 样 子 倒 也 很 精 致 。
他 说 , 是 福 陵 山 上 的 人 家 , 姓 猪 , 上 没 有 父 母 , 下 没 有 兄 弟 , 愿 意 给 别 人 做 个 女 婿 。
我 老 拙 , 见 到 这 样 一 个 无 根 无 绊 的 人 , 就 招 来 了 他 。
一 旦 进 门 时 , 他 倒 在 地 上 很 勤 勉 谨 慎 ; 耕 田 打 地 , 不 用 牛 具 , 收 割 田 中 禾 苗 , 不 用 刀 杖 ; 昏 黄 离 去 , 明 亮 回 来 , 它 的 实 际 情 况 也 很 好 。
只 是 一 件 事 , 有 些 人 会 变 成 嘴 面 。
行 人 说 : 你 怎 么 样 变 ? 高 老 说 : 刚 来 时 是 一 个 黑 胖 子 , 后 来 就 变 成 一 个 长 嘴 大 耳 朵 的 子 , 脑 子 后 面 又 有 一 个 獐 毛 , 身 体 粗 糙 , 怕 人 , 头 脸 就 像 猪 的 样 子 。
吃 肠 又 很 大 , 一 顿 要 吃 三 五 斗 米 饭 , 早 晚 点 心 也 要 一 百 十 个 烧 饼 才 能 吃 。
喜 欢 还 要 吃 斋 素 , 如 果 再 吃 荤 酒 , 就 是 老 拙 这 些 家 业 田 产 之 类 , 不 上 半 年 , 就 吃 完 净 。
三 藏 说 : 只 因 为 他 做 得 , 所 以 吃 得 。
高 老 说 : 吃 饭 还 是 一 件 小 事 。
他 现 在 又 会 弄 风 , 云 来 雾 去 , 走 石 飞 砂 , 何 况 我 一 家 及 左 邻 右 舍 , 都 不 得 安 生 。
又 把 翠 兰 小 女 关 在 后 宅 子 里 , 一 发 半 年 也 没 有 见 面 , 更 不 知 道 死 活 如 何 。
因 此 而 知 其 是 也 。
行 人 说 : 这 个 有 什 么 难 , 老 儿 你 管 放 心 , 今 晚 把 情 与 你 拿 住 , 教 他 写 了 退 亲 的 文 书 , 还 给 你 的 女 儿 怎 么 样 ? 高 老 大 喜 道 : 我 为 招 唤 了 他 不 打 紧 , 坏 了 我 多 少 清 名 , 疏 远 我 多 少 亲 眷 。
只 要 拿 住 他 , 要 什 么 文 书 , 就 麻 烦 给 我 除 了 根 罢 了 。
行 人 说 : 容 易 , 容 易 。
入 夜 的 时 候 , 就 看 见 好 歹 。
老 儿 十 分 欢 喜 , 才 让 他 展 开 桌 子 , 摆 设 斋 供 。
斋 戒 完 毕 将 近 晚 时 , 老 儿 问 道 : 你 要 什 么 兵 器 , 要 多 少 人 随 从 , 赶 快 好 好 准 备 。
行 人 说 : 兵 器 我 自 己 有 。
老 儿 说 : 二 位 只 是 那 根 锡 杖 , 锡 杖 怎 么 打 得 那 个 妖 精 呢 ? 行 人 从 耳 里 取 出 一 个 绣 花 针 , 捻 在 手 中 , 迎 风 打 了 一 个 裤 子 , 就 是 碗 来 粗 细 的 一 根 金 钳 铁 棒 , 面 对 着 高 老 说 : 你 看 这 个 棍 子 , 比 你 家 的 兵 器 怎 么 样 , 可 以 打 得 这 样 怪 吗 ? 高 老 又 说 : 既 然 有 兵 器 , 可 要 人 跟 着 。
等 我 把 妖 精 拿 来 , 对 众 人 取 供 , 替 你 除 掉 根 罢 了 。
那 个 老 人 立 即 叫 来 家 僮 , 请 了 几 个 亲 朋 朋 友 。
一 会 儿 都 到 了 , 相 见 已 经 完 毕 , 行 人 说 : 师 父 , 你 放 心 稳 坐 , 老 孙 子 去 吧 。
你 看 他 着 铁 棒 , 拽 着 高 老 说 : 你 引 我 到 后 宅 子 里 的 妖 精 住 处 去 看 。
高 老 就 把 他 领 到 后 宅 的 门 口 。
行 人 说 : 你 去 取 锁 匙 来 。
高 老 说 : 你 暂 且 看 看 , 如 果 用 得 了 钥 匙 , 却 不 请 你 了 。
行 人 笑 着 说 : 你 那 个 老 子 年 纪 虽 大 , 却 不 认 识 戏 弄 。
我 把 这 话 儿 子 哄 你 一 哄 , 你 就 是 真 的 。
他 走 上 前 , 摸 了 一 遍 , 原 来 是 铜 汁 灌 的 锁 子 。
其 余 , 其 中 有 一 个 人 , 其 中 有 一 个 人 , 其 中 有 一 个 人 , 一 个 人 , 一 个 人 , 一 个 人 , 一 个 人 , 一 个 人 , 一 个 人 , 一 个 人 , 一 个 人 , 一 个 人 , 一 个 人 , 一 个 人 , 一 个 人 , 一 个 人 , 一 个 人 , 一 个 人 , 一 个 人 , 一 个 人 , 一 个 人 , 一 个 人 , 一 个 人 , 一 个 人 , 一 个 人 也 。
行 人 说 : 老 高 , 你 去 叫 你 的 女 儿 一 声 , 看 她 可 在 里 面 , 那 个 老 子 硬 着 胆 大 叫 道 : 三 姐 姐 , 那 个 女 儿 认 得 是 他 父 亲 的 声 音 , 才 少 气 无 力 应 了 一 声 说 : 爹 爹 , 我 在 这 里 吗 ?
走 路 的 人 闪 金 睛 , 到 黑 色 的 影 子 里 仔 细 看 时 , 你 说 他 是 什 么 样 子 , 只 看 见 那 个 人 : 云 鬓 乱 堆 没 有 掠 夺 , 玉 容 未 洗 尘 。
一 片 兰 心 依 旧 , 十 分 娇 态 倾 颓 。
桃 花 的 嘴 唇 全 无 气 血 , 腰 肢 弯 弯 曲 曲 , 依 傍 着 。
愁 绪 郁 闷 , 蛾 眉 就 淡 淡 ; 瘦 弱 怯 懦 , 语 声 就 低 。
他 走 来 , 看 见 高 老 , 一 把 把 他 拽 住 , 抱 着 头 大 哭 。
行 人 说 : 暂 且 不 要 哭 , 暂 且 不 要 哭 。
我 问 你 , 妖 怪 到 哪 里 去 了 , 女 子 说 : 不 知 道 到 哪 里 去 ?
这 些 时 候 , 天 亮 就 走 了 , 到 了 夜 晚 才 来 。
云 云 雾 雾 , 往 回 不 知 到 哪 里 去 了 。
因 此 而 知 其 父 亲 , 以 其 父 亲 , 以 其 父 亲 , 以 其 父 母 , 以 其 父 母 之 礼 , 以 其 父 亲 之 礼 , 以 其 父 亲 之 礼 , 以 其 父 亲 之 礼 , 以 其 父 亲 之 礼 , 以 其 父 母 之 礼 , 以 其 父 母 之 礼 , 以 其 父 母 之 礼 , 以 其 父 母 之 礼 , 以 其 父 母 之 礼 , 以 其 父 母 之 礼 , 以 其 父 母 之 礼 , 以 其 父 母 之 礼 , 以 其 父 母 之 礼 , 以 其 父 母 之 礼 。
行 人 说 : 不 必 说 了 。
老 儿 , 你 带 着 令 爱 去 前 边 的 宅 院 , 慢 慢 地 说 话 , 让 我 的 孙 子 在 这 里 等 着 他 。
他 如 果 不 来 , 你 就 不 要 责 怪 ; 他 如 果 来 了 , 我 就 会 给 你 剪 草 除 根 。
老 人 喜 欢 喜 喜 , 把 女 儿 带 上 去 。
行 者 忽 然 变 了 神 通 , 转 身 一 变 , 变 成 就 像 那 个 女 子 一 样 , 独 自 坐 在 房 里 等 着 妖 精 。
不 多 时 , 一 阵 风 来 , 真 是 走 石 飞 砂 。
好 风 : 起 初 时 微 微 飘 荡 , 到 后 来 茫 茫 茫 茫 。
微 微 荡 荡 , 天 地 大 , 茫 茫 无 所 阻 碍 。
凋 花 折 柳 胜 麻 , 倒 树 摧 林 如 拔 菜 。
翻 江 搅 海 , 鬼 神 愁 苦 , 裂 石 崩 山 , 天 地 怪 异 。
花 糜 鹿 失 去 了 来 踪 , 摘 果 猿 猴 迷 在 外 面 。
七 层 铁 塔 侵 入 佛 头 , 八 面 幢 幡 伤 坏 宝 盖 。
金 梁 玉 柱 起 根 摇 动 , 房 屋 上 的 瓦 片 飞 得 像 燕 子 的 块 块 。
举 桨 梢 公 许 愿 心 , 开 船 忙 忙 把 猪 羊 赛 赛 。
当 坊 土 地 抛 弃 祠 堂 , 四 海 龙 王 朝 上 拜 。
海 边 撞 损 夜 叉 船 , 长 城 被 刮 倒 半 边 塞 。
其 中 有 一 种 狂 风 吹 过 的 地 方 , 只 见 半 空 中 有 一 个 妖 精 , 果 然 生 得 很 丑 陋 。 黑 脸 短 毛 , 长 嘴 大 耳 , 穿 着 一 领 青 不 青 、 蓝 不 蓝 的 织 布 直 , 系 着 一 条 花 布 手 巾 。
行 人 暗 笑 着 说 : 原 来 是 这 个 买 卖 。
喜 欢 行 走 的 人 , 却 不 迎 接 他 , 也 不 问 他 , 而 且 睡 在 床 上 推 病 , 口 里 不 停 地 吵 吵 嚷 嚷 嚷 嚷 嚷 嚷 嚷 嚷 嚷 嚷 嚷 嚷 嚷 嚷 嚷 嚷 嚷 嚷 嚷 嚷 嚷 嚷 嚷 嚷 嚷 嚷 嚷 嚷 嚷 嚷 嚷 嚷 嚷 嚷 嚷 嚷 嚷 嚷 嚷 嚷 嚷 嚷 嚷 嚷 嚷 嚷 嚷 嚷 嚷 嚷 嚷 嚷 嚷 嚷 嚷 嚷 嚷 嚷 嚷 嚷 嚷 嚷 嚷 嚷 嚷 嚷 嚷 嚷 嚷 嚷 嚷 嚷 嚷 嚷 嚷 嚷 嚷 嚷 嚷 嚷 嚷 嚷 嚷 嚷 。
怪 物 不 知 道 真 假 , 走 进 房 中 , 一 把 握 住 , 就 要 亲 口 。
行 人 暗 笑 着 说 : 真 的 你 要 来 弄 弄 老 孙 子 啊 !
于 是 就 让 他 拿 法 , 把 那 怪 的 长 嘴 , 叫 做 个 小 跌 。
漫 头 一 料 料 , 扑 得 跳 下 床 来 。
那 怪 物 爬 起 来 , 扶 着 床 边 说 : 姐 姐 , 你 为 什 么 今 天 有 些 怪 我 , 想 是 我 来 得 得 迟 了 。
那 妖 说 : 既 然 不 怪 我 , 为 什 么 就 丢 掉 我 这 样 的 样 子 , 行 人 说 : 你 为 什 么 就 这 样 的 小 家 子 , 就 抱 着 我 亲 口 , 我 因 为 今 天 有 些 不 自 在 , 如 果 每 到 常 好 的 时 候 , 就 起 来 开 门 等 你 了 。
你 可 以 脱 掉 衣 服 睡 了 。
那 怪 物 不 理 解 他 的 意 思 , 真 的 就 去 脱 掉 衣 服 。
行 人 跳 起 来 , 坐 在 一 个 净 桶 上 。
那 怪 物 仍 旧 又 来 到 床 上 摸 一 把 , 摸 不 出 人 , 叫 道 : 姐 姐 , 你 到 哪 里 去 了 , 请 脱 去 衣 服 睡 觉 吧 。
行 人 说 : 你 先 睡 , 等 我 出 来 一 个 恭 来 。
那 怪 果 然 先 解 开 衣 服 上 床 。
行 人 忽 然 叹 气 , 说 道 : 造 化 低 了 。
那 怪 道 : 你 恼 得 怎 么 样 , 造 化 怎 么 能 低 下 呢 我 得 到 了 你 家 , 虽 然 是 吃 了 些 茶 饭 , 却 也 不 曾 白 吃 你 的 。 我 也 曾 经 替 你 家 扫 地 通 沟 , 搬 砖 运 瓦 , 筑 土 打 墙 , 耕 田 打 地 , 种 麦 插 秧 , 创 家 立 业 。
如 今 你 身 上 穿 的 锦 缎 , 戴 的 金 子 , 四 季 有 花 果 , 八 节 有 蔬 菜 烹 煎 , 你 还 有 那 些 儿 子 不 在 心 的 地 方 , 这 样 短 叹 长 吁 , 说 什 么 造 化 低 了 ?
今 日 我 的 父 母 隔 着 墙 , 丢 砖 料 瓦 , 是 打 我 骂 我 呀 !
那 人 怪 道 : 他 打 骂 你 怎 么 样 ? 行 者 说 : 我 和 你 做 了 夫 妻 , 你 是 他 门 下 的 一 个 女 婿 , 完 全 没 有 儿 子 的 礼 节 。
这 样 的 人 , 又 会 得 不 到 姨 夫 , 又 见 不 到 亲 戚 , 又 不 知 道 你 云 来 雾 去 , 端 的 是 那 里 人 家 , 姓 甚 名 谁 , 败 坏 他 的 品 德 , 玷 辱 他 的 门 风 , 所 以 这 样 的 打 骂 , 所 以 恼 恼 我 。
李 那 奇 怪 地 说 : 我 虽 然 是 有 些 儿 子 丑 陋 , 如 果 想 要 俊 杰 , 退 却 也 不 难 。
我 一 来 的 时 候 , 曾 经 和 他 讲 过 , 他 愿 意 方 才 招 我 。
今 天 又 说 : 我 家 住 在 福 陵 山 的 云 栈 洞 。
我 以 相 貌 为 姓 , 所 以 姓 猪 , 官 名 叫 猪 刚 鬣 。
他 如 果 再 来 问 你 , 你 就 把 这 个 话 告 诉 他 就 行 了 。
行 走 的 人 暗 自 高 兴 地 说 : 那 怪 怪 我 也 老 实 , 不 用 动 刑 , 就 能 这 样 明 白 。
既 有 地 方 、 姓 名 , 不 管 何 也 。
行 者 说 : 他 要 请 法 师 来 抓 你 吗 ?
那 怪 笑 着 说 : 睡 着 , 睡 着 , 不 要 理 睬 他 。
吾 有 天 地 之 数 , 九 齿 之 钉 , 恐 怕 什 么 法 师 、 和 尚 、 道 士 , 就 是 你 老 子 有 虔 诚 的 心 , 请 求 下 九 天 荡 魔 祖 师 下 界 , 我 也 曾 经 和 他 做 过 , 他 也 不 敢 说 我 。
行 人 说 : 他 说 请 一 个 五 百 年 前 大 闹 天 宫 , 姓 孙 的 齐 天 大 圣 , 要 来 抓 你 吗 ?
那 怪 听 到 这 个 名 字 , 就 有 三 分 害 怕 的 道 : 既 然 是 这 样 的 话 , 我 去 了 罢 , 两 口 子 做 不 成 了 。
行 人 说 : 你 怎 么 就 去 ? 那 怪 道 : 你 不 知 道 , 那 闹 天 宫 的 弼 马 温 有 些 本 事 , 只 怕 我 弄 他 不 过 , 低 了 名 头 , 不 像 模 样 。
说 完 , 把 衣 服 换 上 , 打 开 门 , 往 外 走 。
被 行 者 一 把 把 握 住 , 把 自 己 的 脸 上 抹 了 一 抹 , 出 现 出 原 来 的 身 子 , 喝 道 : 好 妖 怪 , 从 哪 里 走 , 你 抬 头 看 我 是 那 个 。 那 怪 怪 转 过 眼 来 , 看 见 行 人 的 蜇 口 , 火 眼 金 睛 , 槌 头 毛 面 , 就 是 个 活 雷 公 相 似 。
他 手 麻 脚 软 , 一 声 打 击 , 打 破 了 衣 服 , 变 成 狂 风 , 脱 身 而 去 。
行 人 急 忙 上 前 , 拉 着 铁 棒 , 望 风 打 了 一 下 。
那 怪 化 万 道 火 光 , 直 接 转 到 本 山 而 去 。
行 人 驾 着 云 车 , 随 后 赶 来 , 喊 叫 道 : 那 里 跑 , 你 若 上 天 , 我 就 赶 到 斗 牛 宫 ; 你 如 果 进 入 地 , 我 就 追 到 冤 死 狱 。
唉 , 我 竟 不 知 道 这 一 去 赶 到 什 么 地 方 , 有 什 么 胜 负 , 姑 且 听 我 下 回 分 析 。
}\switchcolumn\flushpage  \begin{pinyinscope}{\myfontt \section{第十九回}     雲棧洞悟空收八戒 浮屠山玄奘受心經

卻說那怪的火光前走,這大聖的彩霞隨後。正行處,忽見一座高山,那怪把紅光
結聚,現了本相,撞入洞內,取出一柄九齒釘鈀來戰。行者喝一聲道:「潑怪!
你是那裏來的邪魔?怎麼知道我老孫的名號?你有甚麼本事,實實供來,饒你性
命。」那怪道:「是你也不知我的手段,上前來站穩著,我說與你聽。我:
    自小生來心性拙,貪閑愛懶無休歇。
    不曾養性與修真,混沌迷心熬日月。
    忽然閑裏遇真仙,就把寒溫坐下說。
    勸我回心莫墮凡,傷生造下無邊孽。
    有朝大限命終時,八難三途悔不喋。
    聽言意轉要修行,聞語心回求妙訣。
    有緣立地拜為師,指示天關並地闕。
    得傳九轉大還丹,工夫晝夜無時輟。
    上至頂門泥丸宮,下至腳板湧泉穴。
    周流腎水入華池,丹田補得溫溫熱。
    嬰兒?女配陰陽,鉛汞相投分日月。
    離龍坎虎用調和,靈龜吸盡金烏血。
    三花聚頂得歸根,五氣朝元通透徹。
    功圓行滿卻飛昇,天仙對對來迎接。
    朗然足下彩雲生,身輕體健朝金闕。
    玉皇設宴會群仙,各分品級排班列。
    敕封元帥管天河,總督水兵稱憲節。
    只因王母會蟠桃,開宴瑤池邀眾客。
    那時酒醉意昏沉,東倒西歪亂撒潑。
    逞雄撞入廣寒宮,風流仙子來相接。
    見他容貌挾人魂,舊日凡心難得滅。
    全無上下失尊卑,扯住嫦娥要陪歇。
    再三再四不依從,東躲西藏心不悅。
    色膽如天叫似雷,險些震倒天關闕。
    糾察靈官奏玉皇,那日吾當命運拙。
    廣寒圍困不通風,進退無門難得脫。
    卻被諸神拿住我,酒在心頭還不怯。
    押赴靈霄見玉皇,依律問成該處決。
    多虧太白李金星,出班俯?親言說。
    改刑重責二千鎚,肉綻皮開骨將折。
    放生遭貶出天關,福陵山下圖家業。
    我因有罪錯投胎,俗名喚做豬剛鬣。」

行者聞言道:「你這廝原來是天蓬水神下界,怪道知我老孫名號。」那怪道聲:
「哏!你這誑上的弼馬溫,當年撞那禍時,不知帶累我等多少,今日又來此欺人
。不要無禮,吃我一鈀。」行者怎肯容情,舉起棒,當頭就打。他兩個在那半山
之中,黑夜裏賭鬥。好殺:
行者金睛似閃電,妖魔環眼似銀花。這一個口噴彩霧,那一個氣吐紅霞。氣吐紅
霞昏處亮,口噴彩霧夜光華。金箍棒,九齒鈀,兩個英雄實可誇:一個是大聖臨
凡世,一個是元帥降天涯。那個因失威儀成怪物,這個幸逃苦難拜僧家。鈀去好
似龍伸爪,棒迎渾若鳳穿花。那個道:「你破人親事如殺父!」這個道:「你強
姦幼女正該拿!」閑言語,亂喧嘩,往往來來棒架鈀。看看戰到天將曉,那妖精
兩膊覺酸麻。
  
他兩個自二更時分,直戰到東方發白。那怪不能迎敵,敗陣而逃,依然又化狂風
,徑回洞裏,把門緊閉,再不出頭。行者在這洞門外看有一座石碣,上書雲棧洞
三字。見那怪不出,天又大明,心卻思量:「恐師父等候,且回去見他一見,再
來捉此怪不遲。」隨踏雲點一點,早到高老莊。
  
卻說三藏與那諸老談今論古,一夜無眠。正想行者不來,只見天井裏忽然站下行
者。行者收藏鐵棒,整衣上廳。叫道:「師父,我來了。」慌得那諸老一齊下拜
,謝道:「多勞,多勞。」三藏問道:「悟空,你去這一夜,拿得妖精在那裏?」
行者道:「師父,那妖不是凡間的邪祟,也不是山間的怪獸。他本是天蓬元帥臨
凡,只因錯投了胎,嘴臉像一個野豬模樣,其實性靈尚存。他說以相為姓,喚名
豬剛鬣。是老孫從後宅裏掣棒就打,他化一陣狂風走了。被老孫著風一棒,他就
化道火光,徑轉他那本山洞裏,取出一柄九齒釘鈀,與老孫戰了一夜。適才天色
將明,他怯戰而走,把洞門緊閉不出。老孫還要打開那門,與他見個好歹,恐師
父在此疑慮盼望,故先來回個信息。」
  
說罷,那老高上前跪下道:「長老,沒及奈何,你雖趕得去了,他等你去後復來
,卻怎區處?索性累你與我拿住,除了根,才無後患。我老夫不敢怠慢,自有重
謝:將這家財田地,憑眾親友寫立文書,與長老平分。只是要剪草除根,莫教壞
了我高門清德。」行者笑道:「你這老兒不知分限。那怪也曾對我說,他雖是食
腸大,吃了你家些茶飯,也與你幹了許多好事,這幾年掙了許多家貲,皆是他之
力量。他不曾白吃了你東西,問你祛他怎的?據他說,他是一個天神下界,替你
巴家做活,又未曾害了你家女兒。想這等一個女婿,也門當戶對,不怎麼壞了家
聲,辱了行止,當真的留他也罷。」老高道:「長老,雖是不傷風化,但名聲不
甚好聽,動不動著人就說:『高家招了一個妖怪女婿。』這句話兒教人怎當?」
三藏道:「悟空,你既是與他做了一場,一發與他做個結局,才見始終。」行者
道:「我才試他一試耍子。此去一定拿來與你們看,且莫憂愁。」叫:「老高,
你還好生管待我師父,我去也。」
  
說聲去,就無形無影的,跳到他那山上,來到洞口,一頓鐵棍,把兩扇門打得粉
碎。口裏罵道:「那?糠的夯貨,快出來與老孫打麼。」那怪正喘噓噓的睡在洞
內,聽見打得門響,又聽見罵?糠的夯貨,他卻惱怒難禁,只得拖著鈀,抖擻精
神,跑將出來,厲聲罵道:「你這個弼馬溫,著實憊懶。與你有甚相干,你把我
大門打破?你且去看看律條,打進大門而入,該個雜犯死罪哩。」行者笑道:
「這個獃子!我就打了大門,還有個辨處。像你強占人家女子,又沒個三媒六證
,又無些茶紅酒禮,該問個真犯斬罪哩。」那怪道:「且休閑講,看老豬這鈀。」
行者使棒支住道:「你這鈀可是與高老家做長工築地種菜的?有何好處怕你?」
那怪道:「你錯認了,這鈀豈是凡間之物?你且聽我道來:
    此是鍛煉神冰鐵,磨琢成工光皎潔。
    老君自己動鈐鎚,熒親身添炭屑。
    五方五帝用心機,六丁六甲費周折。
    造成九齒玉垂牙,鑄就雙環金墜葉。
    身妝六曜排五星,體按四時依八節。
    短長上下定乾坤,左右陰陽分日月。
    六爻神將按天條,八卦星辰依斗列。
    名為上寶沁金鈀,進與玉皇鎮丹闕。
    因我修成大羅仙,為吾養就長生客。
    敕封元帥號天蓬,欽賜釘鈀為御節。
    舉起烈焰並毫光,落下猛風飄瑞雪。
    天曹神將盡皆驚,地府閻羅心膽怯。
    人間那有這般兵,世上更無此等鐵。
    隨身變化可心懷,任意翻騰依口訣。
    相攜數載未曾離,伴我幾年無日別。
    日食三餐並不丟,夜眠一宿渾無撇。
    也曾佩去赴蟠桃,也曾帶他朝帝闕。
    皆因仗酒卻行兇,只為倚強便撒潑。
    上天貶我降凡塵,下世儘我作罪孽。
    石洞心邪曾吃人,高莊情喜婚姻結。
    這鈀下海掀翻龍鼉窩,上山抓碎虎狼穴。
    諸般兵刃且休題,惟有吾當鈀最切。
    相持取勝有何難,賭鬥求功不用說。
    何怕你銅頭鐵腦一身鋼,鈀到魂消神氣泄。」

行者聞言,收了鐵棒道:「獃子不要說嘴,老孫把這頭伸在那裏,你且築一下兒
,看可能魂消氣泄?」那怪真個舉起鈀,著氣力築將來,撲的一下,鑽起鈀的火
光焰焰,更不曾築動一些兒頭皮。諕得他手麻腳軟,道聲:「好頭!好頭!」行
者道:「你是也不知。老孫因為鬧天宮,偷了仙丹,盜了蟠桃,竊了御酒,被小
聖二郎擒住,押在斗牛宮前,眾天神把老孫斧剁鎚敲,刀砍劍刺,火燒雷打,也
不曾損動分毫。又被那太上老君拿了我去,放在八卦爐中,將神火鍛煉,煉做個
火眼金睛,銅頭鐵臂。不信,你再築幾下,看看疼與不疼?」那怪道:「你這猴
子,我記得你鬧天宮時,家住在東勝神洲傲來國花果山水簾洞裏,到如今久不聞
名,你怎麼來到這裏,上門子欺我?莫敢是我丈人去那裏請你來的?」行者道:
「你丈人不曾去請我。因是老孫改邪歸正,棄道從僧,保護一個東土大唐駕下御
弟,叫做三藏法師,往西天拜佛求經,路過高莊借宿,那高老兒因話說起,就請
我救他女兒,拿你這?糠的夯貨。」
  
那怪一聞此言,丟了釘鈀,唱個大喏道:「那取經人在那裏?累煩你引見引見。」
行者道:「你要見他怎的?」那怪道:「我本是觀世音菩薩勸善,受了他的戒行
,這裏持齋把素,教我跟隨那取經人往西天拜佛求經,將功折罪,還得正果。教
我等他這幾年,不聞消息。今日既是你與他做了徒弟,何不早說取經之事,只倚
兇強,上門打我?」行者道:「你莫詭詐欺心軟我,欲為脫身之計。果然是要保
護唐僧,略無虛假,你可朝天發誓,我才帶你去見我師父。」那怪撲的跪下,望
空似搗碓的一般,只管磕頭道:「阿彌陀佛,南無佛,我若不是真心實意,還教
我犯了天條,劈屍萬段。」行者見他賭咒發願,道:「既然如此,你點把火來燒
了你這住處,我方帶你去。」那怪真個搬些蘆葦荊棘,點著一把火,將那雲棧洞
燒得像個破瓦?。對行者道:「我今已無罣礙了,你卻引我去罷。」行者道:
「你把釘鈀與我拿著。」那怪就把鈀遞與行者。行者又拔了一根毫毛,吹口仙氣
,叫:「變!」即變做一條三股麻繩,走過來,把手背綁剪了。那怪真個倒背著
手,憑他怎麼綁縛。卻又揪著耳朵,拉著他,叫:「快走,快走。」那怪道:
「輕著些兒,你的手重,揪得我耳根子疼。」行者道:「輕不成,顧你不得。常
言道:『善豬惡拿。』只等見了我師父,果有真心,方才放你。」他兩個半雲半
霧的,徑轉高家莊來。有詩為證:
    金性剛強能剋木,心猿降得木龍歸。
    金從木順皆為一,木戀金仁總發揮。
    一主一賓無間隔,三交三合有玄微。
    性情並喜貞元聚,同證西方話不違。

頃刻間到了莊前。行者拑著他的鈀,揪著他的耳道:「你看那廳堂上端坐的是誰
?乃吾師也。」那高氏諸親友與老高,忽見行者把那怪背綁揪耳而來,一個個忻
然迎到天井中,道聲:「長老,長老,他正是我家的女婿。」那怪走上前,雙膝
跪下,背著手,對三藏叩頭,高叫道:「師父,弟子失迎。早知是師父住在我丈
人家,我就來拜接,怎麼又受到許多周折?」三藏道:「悟空,你怎麼降得他來
拜我?」行者才放了手,拿釘鈀柄兒打著,喝道:「獃子,你說麼。」那怪把菩
薩勸善事情,細陳了一遍。
  
三藏大喜,便叫:「高太公,取個香案用用。」老高即忙抬出香案。三藏淨了手
焚香,望南禮拜道:「多蒙菩薩聖恩。」那幾個老兒也一齊添香禮拜。拜罷,三
藏上廳高坐,教悟空放了他繩。行者才把身抖了一抖,收上身來,其縛自解。那
怪從新禮拜三藏,願隨西去。又與行者拜了,以先進者為兄,遂稱行者為師兄。
三藏道:「既從吾善果,要做徒弟,我與你起個法名,早晚好呼喚。」他道:
「師父,我是菩薩已與我摩頂受戒,起了法名,叫做豬悟能也。」三藏笑道:
「好,好。你師兄叫做悟空,你叫做悟能,其實是我法門中的宗派。」悟能道:
「師父,我受了菩薩戒行,斷了五葷三厭,在我丈人家持齋把素,更不曾動葷。
今日見了師父,我開了齋罷。」三藏道:「不可,不可。你既是不吃五葷三厭,
我再與你起個別名,喚為八戒。」那獃子歡歡喜喜道:「謹遵師命。」因此又叫
做豬八戒。
  
高老見這等去邪歸正,更十分喜悅,遂命家僮安排筵宴,酬謝唐僧。八戒上前扯
住老高道:「爺,請我拙荊出來拜見公公、伯伯,如何?」行者笑道:「賢弟,
你既入了沙門,做了和尚,從今後,再莫題起那『拙荊』的話說。世間只有個火
居道士,那裏有個火居的和尚?我們且來敘了坐次,吃頓齋飯,趕早兒往西天走
路。」高老兒擺了桌席,請三藏上坐;行者與八戒坐於左右兩傍;諸親下坐。高
老把素酒開樽,滿斟一杯,奠了天地,然後奉與三藏。三藏道:「不瞞太公說,
貧僧是胎裏素,自幼兒不吃葷。」老高道:「因知老師清素,不曾敢動葷。此酒
也是素的,請一杯不妨。」三藏道:「也不敢用酒,酒是我僧家第一戒者。」悟
能慌了道:「師父,我自持齋,卻不曾斷酒。」悟空道:「老孫雖量窄,吃不上
罈把,卻也不曾斷酒。」三藏道:「既如此,你兄弟們吃些素酒也罷,只是不許
醉飲誤事。」遂而他兩個接了頭鍾。各人俱照舊坐下,擺下素齋。說不盡那杯盤
之盛,品物之豐。
  
師徒們宴罷,老高將一紅漆丹盤,拿出二百兩散碎金銀,奉三位長老為途中之費
;又將三領綿布褊衫為上蓋之衣。三藏道:「我們是行腳僧,遇莊化飯,逢處求
齋,怎敢受金銀財帛?」行者近前,掄開手抓了一把,叫:「高才,昨日累你引
我師父,今日招了一個徒弟,無物謝你,把這些碎金碎銀,權作帶領錢,拿了去
買草鞋穿。以後但有妖精,多作成我幾個,還有謝你處哩。」高才接了,叩頭謝
賞。老高又道:「師父們既不受金銀,望將這粗衣笑納,聊表寸心。」三藏又道
:「我出家人,若受了一絲之賄,千劫難修。只是把席上吃不了的餅果,帶些去
做乾糧足矣。」
  
八戒在傍邊道:「師父、師兄,你們不要便罷,我與他家做了這幾年女婿,就是
掛腳糧也該三石哩。──丈人呵,我的直裰,昨晚被師兄扯破了,與我一件青錦
袈裟;鞋子綻了,與我一雙好新鞋子。」高老聞言,不敢不與,隨買一雙新鞋,
將一領褊衫,換下舊時衣物。那八戒搖搖擺擺,對高老唱個喏道:「上覆丈母、
大姨、二姨並姨夫、姑舅諸親:我今日去做和尚了,不及面辭,休怪。丈人呵,
你還好生看待我渾家,只怕我們取不成經時,好來還俗,照舊與你做女婿過活。」
行者喝道:「夯貨,卻莫胡說。」八戒道:「不是胡說,只恐一時間有些兒差池
,卻不是和尚誤了做,老婆誤了娶,兩下裏都耽擱了?」
  
三藏道:「少題閑話,我們趕早兒去來。」遂此收拾了一擔行李,八戒擔著;背
了白馬,三藏騎著;行者肩擔鐵棒,前面引路。一行三眾,辭別高老及眾親友,
投西而去。有詩為證。詩曰:
    滿地煙霞樹色高,唐朝佛子苦勞勞。
    饑餐一缽千家飯,寒著千針一衲袍。
    意馬胸頭休放蕩,心猿乖劣莫教嚎。
    情和性定諸緣合,月滿金華是伐毛。

三眾進西路途,有個月平穩。行過了烏斯藏界,猛抬頭見一座高山。三藏停鞭勒
馬道:「悟空、悟能,前面山高,須索仔細仔細。」八戒道:「沒事。這山喚做
浮屠山,山中有一個烏巢禪師,在此修行,老豬也曾會他。」三藏道:「他有些
甚麼勾當?」八戒道:「他倒也有些道行。他曾勸我跟他修行,我不曾去罷了。」
師徒們說著話,不多時,到了山上。好山!但見那:
山南有青松碧檜,山北有綠柳紅桃。鬧聒聒,山禽對語;舞翩翩,仙鶴齊飛。香
馥馥,諸花千樣色;青冉冉,雜草萬般奇。澗下有滔滔綠水,崖前有朵朵祥雲。
真個是景致非常幽雅處,寂然不見往來人。
  
那師父在馬上遙觀,見香檜樹前有一柴草窩,左邊有麋鹿啣花,右邊有山猴獻果
,樹梢頭有青鸞、彩鳳齊鳴,玄鶴、錦雞咸集。八戒指道:「那不是烏巢禪師?」
三藏縱馬加鞭,直至樹下。
  
卻說那禪師見他三眾前來,即便離了巢穴,跳下樹來。三藏下馬奉拜,那禪師用
手攙道:「聖僧請起。失迎,失迎。」八戒道:「老禪師,作揖了。」禪師驚問
道:「你是福陵山豬剛鬣,怎麼有此大緣,得與聖僧同行?」八戒道:「前年蒙
觀音菩薩勸善,願隨他做個徒弟。」禪師大喜道:「好,好,好!」又指定行者
,問道:「此位是誰?」行者笑道:「這老禪怎麼認得他,倒不認得我?」禪師
道:「因少識耳。」三藏道:「他是我的大徒弟孫悟空。」禪師陪笑道:「欠禮
,欠禮。」
  
三藏再拜:「請問西天大雷音寺還在那裏?」禪師道:「遠哩,遠哩。只是路多
虎豹,難行。」三藏慇懃致意,再問:「路途果有多遠?」禪師道:「路途雖遠
,終須有到之日,卻只是魔瘴難消。我有《多心經》一卷,凡五十四句,共計二
百七十字。若遇魔瘴之處,但念此經,自無傷害。」三藏拜伏於地懇求,那禪師
遂口誦傳之。經云:
摩訶般若波羅蜜多心經》:觀自在菩薩,行深般若波羅蜜多,時照見五蘊皆空,
度一切苦厄。舍利子,色不異空,空不異色;色即是空,空即是色。受想行識,
亦復如是。舍利子,是諸法空相,不生不滅,不垢不淨,不增不減。是故空中無
色,無受想行識,無眼耳鼻舌身意,無色聲香味觸法,無眼界,乃至無意識界,
無無明,亦無無明盡。乃至無老死,亦無老死盡。無苦寂滅道,無智亦無得。以
無所得故,菩提薩埵。依般若波羅蜜多故,心無罣礙;無罣礙故,無有恐怖。遠
離顛倒夢想,究竟涅槃。三世諸佛,依般若波羅蜜多故,得阿耨多羅三藐三菩提
。故知般若波羅蜜多是大神咒,是大明咒,是無上咒,是無等等咒,能除一切苦
,真實不虛。故說般若波羅蜜多咒,即說咒曰:「揭諦揭諦,波羅揭諦,波羅僧
揭諦,菩提薩婆訶!」
  
此時唐朝法師本有根源,耳聞一遍《多心經》,即能記憶,至今傳世。此乃修真
之總經,作佛之會門也。
  
那禪師傳了經文,踏雲光,要上烏巢而去。被三藏又扯住奉告,定要問個西去的
路程端的。那禪師笑云:
    道路不難行,試聽我吩咐。
    千山千水深,多瘴多魔處。
    若遇接天崖,放心休恐怖。
    行來摩耳巖,側著腳蹤步。
    仔細黑松林,妖狐多截路。
    精靈滿國城,魔主盈山住。
    老虎坐琴堂,蒼狼為主簿。
    獅象盡稱王,虎豹皆作御。
    野豬挑擔子,水怪前頭遇。
    多年老石猴,那裏懷嗔怒。
    你問那相識,他知西去路。

行者聞言,冷笑道:「我們去,不必問他,問我便了。」三藏還不解其意。那禪
師化作金光,徑上烏巢而去。長老往上拜謝,行者心中大怒,舉鐵棒望上亂搗,
只見蓮花生萬朵,祥霧護千層。行者縱有攪海翻江力,莫想挽著烏巢一縷籐。三
藏見了,扯住行者道:「悟空,這樣一個菩薩,你搗他窩巢怎的?」行者道:
「他罵了我兄弟兩個一場去了。」三藏道:「他講的西天路徑,何嘗罵你?」行
者道:「你那裏曉得?他說『野豬挑擔子』是罵的八戒;『多年老石猴』是罵的
老孫。你怎麼解得此意?」八戒道:「師兄息怒。這禪師也曉得過去未來之事,
但看他『水怪前頭遇』這句話,不知驗否?饒他去罷。」行者見蓮花祥霧,近那
巢邊,只得請師父上馬,下山往西而去。那一去:
管教清福人間少,致使災魔山裏多。
  
    畢竟不知前程端的如何,且聽下回分解。





}  \end{pinyinscope}\switchcolumn{\myfontc \section{第 十 九 回} 云 栈 洞 悟 空 收 八 戒 浮 屠 山 , 玄 奘 接 受 《 心 经 》 , 又 说 : 那 怪 异 的 火 光 往 前 走 , 这 大 圣 的 彩 霞 随 后 走 。
正 走 的 地 方 , 忽 然 看 见 一 座 高 山 , 那 怪 物 把 红 光 聚 集 在 一 起 , 现 出 原 来 的 相 貌 , 撞 进 洞 内 , 取 出 一 柄 九 齿 钉 辕 来 作 战 。
行 者 喝 了 一 声 说 : 泼 怪 , 你 是 从 哪 里 来 的 邪 魔 , 怎 么 知 道 我 老 孙 的 名 号 ? 你 有 什 么 本 事 , 实 际 上 供 给 你 , 饶 你 的 性 命 。
那 人 怪 道 : 是 你 也 不 知 道 我 的 手 段 , 上 前 来 站 稳 , 我 说 给 你 听 。
我 说 : 自 小 生 以 来 , 心 性 拙 劣 , 贪 婪 闲 暇 , 爱 好 懒 惰 , 没 有 休 息 。
不 曾 修 养 自 己 的 本 性 和 修 炼 真 性 , 混 沌 迷 惑 之 心 , 煎 熬 日 月 。
忽 然 在 闲 暇 的 地 方 遇 到 了 真 仙 , 就 拿 着 寒 温 的 坐 下 说 。
劝 我 回 心 , 不 要 堕 入 凡 俗 之 中 , 伤 害 生 命 , 造 成 无 边 的 孽 孽 。
有 朝 廷 的 大 限 命 终 时 , 八 难 三 途 悔 恨 不 吐 。
听 到 言 语 的 意 思 转 变 , 要 修 行 , 听 到 语 言 的 心 思 回 头 寻 求 妙 诀 。
有 缘 可 以 立 地 拜 他 为 师 , 指 示 天 关 和 地 阙 。
得 以 传 授 九 转 大 还 丹 , 工 夫 昼 夜 不 停 。
往 上 到 顶 门 泥 丸 宫 , 往 下 到 脚 板 的 涌 泉 穴 。
周 围 的 肾 水 流 入 华 池 , 丹 田 补 得 温 温 热 热 。
婴 儿 、 女 儿 配 合 阴 阳 , 铅 汞 相 投 分 日 月 。
离 龙 坎 虎 用 来 调 和 , 灵 龟 吸 尽 金 乌 的 血 。
三 花 聚 集 在 山 顶 上 得 到 归 根 , 五 气 朝 元 通 透 彻 。
功 圆 行 满 却 飞 升 , 天 仙 相 对 来 迎 接 。
明 朗 的 脚 下 彩 云 生 , 身 轻 体 健 , 朝 觐 金 阙 。
玉 皇 设 宴 会 群 仙 , 各 自 分 别 品 级 排 列 队 列 。
敕 令 封 元 帅 管 理 天 河 , 总 督 水 兵 , 称 为 宪 节 。
只 因 为 王 母 会 蟠 桃 , 在 瑶 池 设 宴 邀 请 众 宾 客 。
当 时 酒 醉 意 昏 沉 , 东 倒 西 歪 乱 撒 泼 。
逞 雄 撞 入 广 寒 宫 , 风 流 仙 子 来 相 接 。
见 其 容 貌 挟 持 人 的 魂 魄 , 以 前 的 凡 心 难 以 消 灭 。
完 全 没 有 上 下 失 去 尊 卑 , 拉 住 嫦 娥 要 陪 歇 。
再 三 再 三 再 三 , 不 依 从 , 东 逃 西 藏 , 心 里 很 不 高 兴 。
他 的 脸 色 如 天 , 呼 叫 似 雷 , 险 恶 地 震 倒 了 天 门 阙 。
纠 察 灵 官 奏 报 玉 皇 , 那 天 我 应 当 命 运 拙 劣 。
被 围 困 不 能 通 风 , 进 退 无 门 , 难 以 脱 身 。
却 被 众 神 捉 住 我 , 酒 在 心 头 还 不 怕 。
押 送 到 灵 霄 去 见 玉 皇 , 依 据 法 律 审 问 完 成 该 判 决 。
多 亏 太 白 李 金 星 , 出 班 俯 , 亲 自 说 话 。
更 改 刑 罚 , 重 罚 二 千 斤 , 肉 裂 皮 裂 , 骨 头 将 要 折 断 。
刘 放 生 被 贬 出 天 关 , 在 福 陵 山 下 谋 划 家 业 。
我 因 为 有 罪 错 投 胎 , 俗 名 叫 做 猪 刚 鬣 。
行 人 听 到 后 说 道 : 你 这 个 人 原 来 是 天 蓬 水 神 下 界 , 怪 道 我 知 道 我 老 孙 的 名 字 。
那 怪 道 : 唉 呀 , 你 这 个 上 的 弼 马 温 , 当 年 撞 那 祸 的 时 候 , 不 知 道 带 累 我 们 多 少 , 今 天 又 来 这 样 欺 人 。
不 要 无 礼 , 吃 我 一 头 硬 硬 。
走 路 的 人 怎 么 肯 容 忍 , 举 起 棒 子 , 当 头 就 打 。
其 两 个 人 在 半 山 之 中 , 黑 夜 里 打 斗 。
喜 欢 杀 人 : 行 人 金 睛 象 闪 电 , 妖 魔 环 眼 似 银 花 。
其 一 个 口 喷 彩 雾 , 其 一 个 气 吐 红 霞 。
云 气 吐 出 红 霞 , 黄 昏 处 处 光 亮 , 口 中 喷 出 彩 雾 夜 光 华 。
金 钳 棒 , 九 齿 钢 , 两 个 英 雄 确 实 值 得 夸 耀 : 一 个 是 大 圣 临 世 , 一 个 是 元 帅 降 临 天 涯 。
其 所 以 不 得 也 。
钢 的 去 处 好 似 龙 伸 爪 , 棒 的 迎 接 浑 如 凤 穿 花 。
其 他 道 : 你 破 坏 人 家 的 事 情 就 像 杀 父 亲 一 样 。 这 个 道 , 你 强 奸 幼 女 , 正 应 当 拿 。
看 到 战 斗 到 天 将 亮 时 , 那 妖 精 两 臂 觉 得 酸 麻 。
其 他 两 个 , 从 二 更 时 分 , 一 直 打 到 东 方 发 白 。
那 怪 物 不 能 迎 击 敌 人 , 败 阵 逃 走 , 又 变 成 狂 风 , 径 直 回 到 洞 里 , 把 门 紧 闭 着 , 再 也 不 出 头 。
走 路 的 人 在 此 洞 门 外 看 , 有 一 座 石 碣 , 上 面 写 着 云 栈 洞 三 个 字 。
看 见 那 怪 物 不 出 来 , 天 又 大 亮 了 , 心 里 想 着 说 : 恐 怕 师 父 等 候 , 暂 且 回 去 见 他 一 见 , 再 来 捉 这 怪 物 不 迟 。
随 着 踏 云 点 一 点 , 早 到 高 老 庄 。
又 说 三 藏 和 那 些 老 人 谈 论 今 古 , 一 夜 都 没 有 睡 觉 。
正 想 行 人 不 来 , 只 见 天 井 里 忽 然 站 在 下 边 走 路 的 人 。
走 路 的 人 收 藏 了 铁 棒 , 整 理 好 衣 服 上 了 厅 堂 。
又 叫 道 : 师 父 , 我 来 了 。
这 些 老 人 一 齐 下 拜 , 道 谢 道 : 多 劳 , 多 劳 。
三 藏 问 道 : 悟 空 , 你 去 这 一 夜 , 拿 得 妖 精 在 哪 里 , 行 人 说 : 师 父 , 那 妖 不 是 凡 间 的 邪 魔 鬼 怪 , 也 不 是 山 间 的 怪 兽 。
其 本 来 是 天 上 的 元 帅 , 在 世 上 的 时 候 , 只 因 为 错 误 投 胎 , 嘴 上 就 好 像 一 个 野 猪 , 他 的 性 灵 还 存 在 。
其 说 , 以 相 为 姓 , 称 之 为 猪 刚 鬣 。
是 老 孙 子 从 后 宅 里 拿 棍 子 就 打 , 他 变 成 一 阵 狂 风 走 了 。
因 为 他 被 风 吹 一 棒 , 他 就 变 成 火 光 , 径 直 转 到 他 的 原 来 的 山 洞 里 , 取 出 一 柄 九 齿 钉 硬 的 钢 子 , 与 老 孙 子 交 战 了 一 夜 。
刚 才 天 色 将 亮 , 他 胆 怯 战 斗 而 逃 , 把 洞 口 紧 闭 不 出 来 。
老 子 还 要 打 开 那 个 门 , 与 他 见 一 个 好 歹 , 恐 怕 师 父 在 这 里 疑 惑 盼 望 , 所 以 先 来 回 信 。
说 完 , 那 老 人 上 前 跪 下 说 : 长 老 , 没 有 来 得 及 怎 么 办 , 你 虽 然 赶 得 去 了 , 他 等 你 去 后 再 来 , 你 又 怎 么 样 的 地 方 , 你 要 把 你 给 我 拿 住 , 除 掉 我 的 根 , 才 没 有 后 患 。
我 老 夫 不 敢 怠 慢 , 自 有 重 谢 : 将 这 些 家 财 、 田 地 , 凭 着 众 亲 友 写 成 文 书 , 与 长 老 平 分 。
只 是 要 剪 草 除 根 , 不 要 教 他 坏 了 我 高 门 清 德 。
行 人 笑 着 说 : 你 这 个 老 儿 , 不 知 道 分 界 。
何 怪 也 曾 经 对 我 说 : 他 虽 然 是 个 食 肠 大 , 吃 了 你 家 的 茶 饭 , 也 和 你 们 干 了 许 多 好 事 , 这 几 年 来 得 了 许 多 家 产 , 都 是 他 的 力 量 。
他 不 曾 白 吃 了 你 的 东 西 , 问 你 消 除 他 的 干 什 么 ? 根 据 他 的 说 法 , 他 是 一 个 天 神 下 界 , 替 你 巴 家 做 活 , 又 不 曾 害 你 家 的 女 儿 。
想 到 这 样 的 女 婿 , 也 门 当 户 对 , 不 是 坏 了 家 声 , 辱 辱 了 行 止 , 应 当 真 的 留 下 他 也 罢 了 。
老 高 说 : 长 老 , 虽 然 不 伤 风 化 , 但 名 声 不 很 好 听 , 动 不 动 就 让 人 说 : 高 家 招 来 了 一 个 妖 怪 的 女 婿 。
三 藏 说 : 你 既 然 是 与 他 做 了 一 场 , 一 发 给 他 做 个 结 局 , 才 能 见 到 始 终 。
行 人 说 : 我 才 试 他 一 试 玩 子 。
此 去 一 定 拿 来 给 你 们 看 , 暂 且 不 要 忧 愁 。
老 高 , 你 还 好 好 活 着 , 等 我 师 父 , 我 去 吧 。
声 音 之 后 , 无 形 无 影 , 跳 到 他 的 山 上 , 来 到 洞 口 , 一 顿 铁 棍 子 , 把 两 扇 门 打 得 粉 碎 。
口 里 骂 道 : 那 糠 的 矿 物 , 快 出 来 与 老 孙 打 吗 ?
其 怪 物 正 在 呼 啸 声 中 , 又 听 见 打 得 门 响 , 又 听 见 骂 糠 的 大 钱 , 他 却 恼 怒 难 以 制 止 , 只 得 拖 着 硬 硬 的 硬 硬 的 精 神 , 跑 出 来 , 厉 声 骂 道 : 你 这 个 弼 马 温 , 实 在 是 疲 惫 懒 惰 。
与 你 有 什 么 关 系 , 你 把 我 的 大 门 打 破 , 你 暂 且 去 看 法 律 条 文 , 打 进 大 门 进 去 , 该 是 杂 犯 死 罪 吗 ?
行 人 笑 着 说 : 这 个 子 , 我 就 打 了 大 门 , 还 有 辨 别 的 地 方 。
像 你 强 占 人 家 的 女 子 , 又 没 有 三 媒 六 证 , 又 没 有 茶 、 红 、 酒 、 礼 , 应 该 问 你 真 犯 斩 罪 吗 ?
那 人 奇 怪 地 说 : 暂 且 闲 谈 , 看 老 猪 这 锭 。
行 人 让 棒 子 支 住 道 : 你 们 这 些 硬 硬 的 硬 硬 的 硬 硬 的 硬 硬 , 是 与 高 老 家 做 长 工 筑 地 种 菜 的 , 有 什 么 好 处 害 怕 你 呢 ? 那 人 怪 道 : 你 错 认 了 , 这 锭 岂 是 世 间 的 东 西 , 你 暂 且 听 我 说 道 : 这 是 锻 炼 神 冰 铁 , 磨 琢 成 的 工 匠 光 亮 皎 洁 。
老 君 自 己 动 锤 , 火 星 亲 自 添 上 炭 屑 。
五 方 五 帝 用 心 机 , 六 丁 六 甲 费 尽 周 旋 。
制 成 九 齿 玉 垂 牙 , 铸 成 双 环 金 坠 叶 。
身 上 装 饰 六 曜 , 排 列 五 星 , 身 体 按 照 四 时 , 依 照 八 节 。
长 短 上 下 定 天 地 , 左 右 阴 阳 分 日 月 。
六 爻 神 将 按 照 天 道 的 规 律 , 八 卦 星 辰 依 据 斗 宿 排 列 。
名 为 上 宝 沁 金 钢 , 进 献 给 玉 皇 镇 守 丹 阙 。
因 为 我 修 炼 成 了 大 罗 仙 , 为 我 养 就 了 长 生 客 。
敕 令 封 元 帅 号 天 蓬 , 赐 给 他 钉 钢 作 为 御 节 。
举 起 烈 焰 和 毫 光 , 落 下 猛 风 飘 起 瑞 雪 。
天 曹 、 神 将 都 惊 慌 失 措 , 地 府 、 阎 罗 心 里 胆 怯 。
人 间 哪 有 这 样 的 兵 器 , 世 上 更 没 有 这 样 的 铁 器 。
随 身 的 变 化 可 以 心 怀 , 任 意 翻 腾 依 据 口 诀 。
相 携 几 年 未 曾 离 别 , 伴 我 几 年 没 有 什 么 日 子 告 别 。
每 天 吃 三 餐 都 不 放 , 夜 晚 睡 了 一 夜 浑 然 没 有 。
也 曾 佩 带 去 参 加 蟠 桃 , 也 曾 带 着 他 朝 拜 皇 宫 。
都 是 因 为 仗 酒 而 行 凶 , 只 是 因 为 仗 恃 强 大 就 撒 泼 。
上 天 把 我 贬 谪 到 尘 世 , 下 世 把 我 作 罪 孽 。
石 洞 的 心 邪 曾 经 吃 人 , 高 庄 的 感 情 喜 欢 结 成 婚 姻 。
这 样 的 钢 子 下 海 掀 翻 龙 窝 , 上 山 捉 碎 虎 狼 洞 穴 。
各 种 兵 刃 将 要 休 息 , 只 有 我 当 硬 硬 的 硬 硬 。
相 持 取 得 胜 利 有 什 么 困 难 , 争 斗 求 功 , 不 用 说 。
何 必 怕 你 们 钢 头 铁 脑 一 身 钢 子 , 钢 子 到 了 魂 消 神 气 泄 。
行 人 听 了 , 收 起 铁 棒 说 : 子 不 要 说 嘴 , 老 孙 子 把 这 头 伸 在 那 里 , 你 暂 且 筑 一 下 儿 , 看 可 能 魂 消 气 泄 , 那 怪 怪 真 的 举 起 鞘 子 , 气 力 筑 将 来 , 扑 的 一 下 , 钻 起 鞘 子 的 火 光 焰 焰 , 更 不 曾 筑 动 一 点 儿 头 皮 。
得 他 手 麻 脚 软 , 喊 道 : 好 头 , 好 头 啊 走 路 的 人 说 : 你 是 不 知 道 的 。
因 为 这 个 道 理 , 无 所 不 知 , 无 所 不 知 , 无 所 不 知 , 无 所 不 知 , 无 所 不 知 , 无 所 不 知 , 无 所 不 知 , 无 所 不 知 , 无 所 不 知 。
又 被 那 太 上 老 君 拿 了 我 去 , 放 在 八 卦 炉 里 , 将 神 火 炼 成 , 炼 成 火 眼 金 睛 , 铜 头 铁 臂 。
不 信 , 你 再 筑 几 下 , 看 看 疼 还 是 不 疼 , 那 人 怪 道 : 你 这 个 猴 子 , 我 记 得 你 闹 天 宫 时 , 家 住 在 东 胜 神 洲 傲 来 国 花 果 山 水 帘 洞 里 , 到 现 在 已 久 不 闻 名 , 你 为 何 来 到 这 里 , 上 门 子 欺 我 , 莫 敢 是 我 丈 人 去 那 里 请 你 来 的 ?
因 为 老 孙 改 邪 归 正 , 弃 道 从 和 尚 , 保 护 一 个 东 方 大 唐 皇 帝 的 御 弟 , 叫 做 三 藏 法 师 , 去 西 天 拜 佛 求 经 , 路 过 高 庄 借 宿 。
那 怪 一 听 到 这 句 话 , 就 丢 掉 钉 头 钢 子 , 大 声 喝 道 : 那 取 经 的 人 在 哪 里 , 烦 劳 你 引 见 引 见 。
那 人 怪 道 : 我 本 来 是 观 世 音 菩 萨 劝 善 , 接 受 了 他 的 戒 行 , 这 里 持 斋 持 素 , 教 我 随 那 取 经 的 人 到 西 天 拜 佛 求 经 , 以 功 赎 罪 , 还 得 到 正 果 。
教 我 等 他 几 年 , 没 听 到 消 息 。
今 天 既 然 是 你 与 他 做 了 弟 弟 , 为 什 么 不 早 说 取 经 的 事 , 只 依 仗 凶 强 , 上 门 打 我 呢 ? 行 人 说 : 你 们 不 要 欺 骗 我 , 想 要 做 脱 身 的 打 算 。
果 然 是 要 保 护 唐 僧 , 没 有 虚 假 , 你 可 以 朝 天 发 誓 , 我 才 带 你 去 见 我 师 父 。
那 怪 物 扑 在 地 上 跪 下 , 望 见 空 中 好 像 捣 碓 一 样 , 只 管 磕 头 说 : 阿 弥 陀 佛 , 南 无 佛 , 我 如 果 不 是 真 心 实 意 , 还 教 我 犯 了 天 条 , 劈 碎 尸 体 万 段 。
行 者 见 他 赌 咒 发 誓 , 说 : 既 然 这 样 , 你 点 点 火 来 烧 了 你 的 住 处 , 我 才 带 你 去 。
那 怪 真 的 搬 来 芦 苇 荆 棘 , 点 上 一 把 火 , 把 那 云 栈 洞 烧 得 像 破 瓦 。
他 对 行 人 说 : 我 现 在 已 经 没 有 障 碍 了 , 你 却 领 我 去 吧 !
行 人 说 : 你 拿 钉 铃 与 我 拿 着 。
于 是 就 把 钢 子 送 给 行 人 。
行 人 又 拔 出 一 根 毫 毛 , 吹 着 一 口 仙 气 , 喊 道 : 变 , 立 即 变 成 一 条 三 股 麻 绳 , 走 过 来 , 把 手 背 上 的 绑 剪 了 。
怪 是 真 的 倒 背 着 手 , 凭 他 怎 么 绑 ?
又 把 他 的 耳 朵 搓 住 , 叫 道 : 快 走 , 快 走 。
那 人 奇 怪 地 说 : 轻 着 些 儿 子 , 你 的 手 重 , 我 的 耳 根 疼 。
行 人 说 : 轻 不 成 , 只 顾 你 得 不 到 。
他 常 常 说 道 : 善 于 养 猪 , 恶 于 捉 拿 。
只 等 见 到 了 我 师 父 , 果 真 有 真 心 , 才 放 你 。
他 两 个 半 云 半 雾 , 径 直 转 到 高 家 庄 来 。
有 诗 作 证 : 金 性 刚 强 能 克 木 , 心 猿 降 得 木 龙 归 。
金 从 木 顺 都 为 一 , 木 恋 金 仁 总 是 发 挥 。
一 个 主 人 一 个 宾 客 没 有 隔 阂 , 三 个 交 会 三 个 合 在 一 起 , 都 有 玄 妙 的 秘 密 。
性 情 都 喜 欢 贞 元 聚 会 , 同 证 西 方 话 不 违 。
顷 刻 间 , 到 了 庄 子 的 前 面 。
行 者 拍 着 他 的 钢 子 , 握 着 他 的 耳 朵 说 : 你 看 那 厅 堂 上 端 坐 的 是 谁 , 就 是 我 的 老 师 。
那 高 氏 的 几 位 亲 友 和 老 高 , 忽 然 看 见 行 人 把 那 怪 物 的 背 子 绑 着 , 扯 着 耳 朵 来 了 , 一 个 个 欢 喜 地 迎 到 天 井 里 , 说 道 : 长 老 , 长 老 , 他 正 是 我 家 的 女 婿 。
那 怪 走 上 前 , 双 膝 跪 下 , 背 着 手 , 对 着 三 藏 磕 头 , 高 声 大 叫 道 : 师 父 , 弟 子 失 去 迎 接 。
早 知 是 师 父 住 在 我 父 亲 家 里 , 我 就 来 拜 接 , 何 又 受 到 许 多 周 礼 折 服 ? 三 藏 说 : 我 怎 么 投 降 得 到 他 来 拜 我 ? 行 者 才 放 下 手 , 拿 着 钉 铃 柄 子 打 着 , 喝 道 : 子 , 你 说 什 么 ?
怪 怪 地 把 菩 萨 劝 善 的 事 情 , 细 细 陈 述 了 一 遍 。
三 藏 大 喜 , 就 喊 道 : 高 太 公 , 拿 个 香 案 来 使 用 。
老 高 立 刻 就 抬 出 香 案 。
三 藏 净 了 手 焚 香 , 望 南 礼 拜 道 : 多 蒙 菩 萨 的 圣 恩 。
其 中 有 一 个 老 人 , 一 齐 添 香 礼 拜 。
拜 完 之 后 , 三 藏 上 堂 高 坐 , 教 悟 空 放 了 他 绳 子 。
行 人 刚 刚 把 自 己 的 身 子 抖 了 一 抖 , 把 自 己 的 身 子 收 起 来 , 那 个 捆 缚 自 然 解 开 了 。
那 怪 从 新 礼 拜 三 藏 菩 萨 , 愿 随 他 西 去 。
又 与 行 者 拜 了 , 把 先 进 的 人 当 作 哥 哥 , 于 是 称 行 者 为 师 兄 。
三 藏 菩 萨 说 : 既 然 跟 随 我 善 果 , 就 要 做 弟 弟 , 我 给 你 起 个 法 名 , 早 晚 好 呼 唤 。
他 说 : 师 父 , 我 是 菩 萨 已 经 给 我 摩 顶 受 戒 , 起 了 法 名 , 叫 做 猪 悟 能 。
三 藏 笑 着 说 : 好 , 好 。
师 曰 : 师 曰 : 师 曰 : 师 曰 : 师 曰 : 师 曰 : 师 曰 : 师 曰 : 吾 师 之 师 , 吾 师 之 师 曰 : 吾 师 之 师 , 吾 师 之 师 曰 : 吾 师 之 师 , 吾 师 之 师 曰 : 吾 师 之 师 , 吾 师 之 师 也 。
悟 能 说 : 师 父 , 我 受 了 菩 萨 戒 行 , 断 了 五 荤 三 厌 , 在 我 父 亲 家 里 持 斋 持 素 , 更 不 曾 动 过 荤 。
今 天 见 到 了 师 父 , 我 开 斋 罢 了 。
三 藏 说 : 不 行 , 不 行 。
你 既 然 不 吃 五 荤 三 厌 , 我 再 给 你 起 个 别 名 , 叫 它 八 戒 。
那 子 喜 欢 喜 喜 地 说 : 谨 遵 师 傅 的 命 令 。
因 此 又 称 为 猪 八 戒 。
高 老 看 见 这 些 人 去 邪 归 正 , 更 加 十 分 喜 悦 , 于 是 命 家 僮 安 排 筵 席 , 酬 谢 唐 僧 。
八 戒 上 前 拉 住 老 高 说 : 老 爷 , 请 我 拙 荆 出 来 拜 见 公 公 、 伯 伯 , 怎 么 样 ? 行 者 笑 着 说 : 贤 弟 , 你 既 然 进 了 沙 门 , 做 了 和 尚 , 从 今 以 后 , 再 不 要 题 写 这 些 拙 荆 的 话 说 。
世 间 只 有 个 火 居 道 士 , 哪 里 还 有 个 火 居 和 尚 , 我 们 暂 且 来 叙 谈 一 下 , 吃 一 顿 斋 饭 , 赶 早 儿 去 西 天 走 路 。
高 老 儿 摆 了 桌 子 , 请 三 藏 上 座 , 行 者 和 八 戒 坐 在 左 右 两 旁 , 众 亲 属 下 座 。
高 老 把 素 酒 打 开 酒 杯 , 斟 满 一 杯 , 把 天 地 放 在 一 起 , 然 后 奉 送 给 三 藏 。
三 藏 菩 萨 说 : 不 瞒 太 公 说 , 我 是 胎 里 素 , 从 小 儿 就 不 吃 荤 。
老 高 说 : 因 为 我 知 道 老 师 清 廉 朴 素 , 不 敢 吃 荤 。
此 酒 也 是 素 的 , 请 一 杯 不 妨 。
三 藏 说 : 也 不 敢 用 酒 , 酒 是 我 僧 人 家 的 第 一 戒 。
悟 能 惊 慌 地 说 : 师 父 , 我 自 己 念 斋 , 却 不 曾 喝 酒 。
悟 空 说 : 老 孙 虽 然 胸 量 狭 窄 , 吃 不 上 把 , 却 也 不 曾 喝 过 酒 。
三 藏 说 : 既 然 这 样 , 你 们 兄 弟 们 吃 些 白 酒 也 罢 了 , 只 是 不 许 喝 醉 酒 误 事 。
于 是 他 两 个 人 接 了 头 钟 。
各 人 都 照 旧 坐 下 , 摆 下 素 斋 。
说 不 尽 杯 盘 之 盛 , 品 物 之 丰 富 。
老 高 拿 出 一 个 红 漆 丹 盘 , 拿 出 二 百 两 散 碎 金 银 , 奉 送 三 位 长 老 作 为 途 中 的 费 用 , 又 拿 出 三 领 绵 布 裤 衫 做 上 盖 的 衣 服 。
三 藏 菩 萨 说 : 我 们 是 行 脚 僧 , 遇 到 庄 化 饭 , 遇 到 处 求 斋 , 怎 么 敢 接 受 金 银 财 帛 ? 行 人 走 近 前 , 张 开 手 抓 了 一 把 , 叫 道 : 高 才 , 昨 天 拖 累 你 引 我 师 父 , 今 天 招 了 一 个 徒 弟 , 无 物 谢 你 , 把 这 些 碎 金 碎 银 , 权 作 带 领 钱 , 拿 去 去 买 草 鞋 穿 。
以 后 只 要 有 妖 精 , 多 作 成 我 几 个 , 还 有 谢 你 的 地 方 吗 ?
高 才 接 过 , 叩 头 谢 赏 。
老 高 又 说 : 师 父 们 既 然 不 接 受 金 银 , 希 望 将 这 些 粗 衣 服 笑 纳 , 聊 表 一 寸 心 。
三 藏 又 说 : 我 是 出 家 人 , 如 果 接 受 了 一 丝 的 贿 赂 , 千 劫 难 以 修 行 。
只 是 把 席 上 吃 不 完 的 饼 和 果 , 带 去 做 干 粮 就 足 够 了 。
八 戒 在 旁 边 说 : 师 父 、 师 兄 , 你 们 不 要 就 罢 了 , 我 和 他 家 做 了 这 几 年 女 婿 , 就 是 挂 脚 粮 也 应 该 三 石 了 。
丈 人 呵 , 我 的 直 , 昨 晚 被 师 兄 扯 破 了 , 给 我 一 件 青 锦 袈 裟 ; 鞋 子 裂 了 , 给 我 一 双 好 的 新 鞋 子 。
高 老 听 了 这 话 , 不 敢 不 给 , 随 即 买 了 一 双 新 鞋 , 拿 一 领 衫 , 换 下 旧 时 的 衣 服 。
那 八 戒 摇 摇 摇 , 对 着 高 老 高 喊 道 : 上 报 丈 母 、 大 姨 、 二 姨 以 及 姨 夫 、 姑 舅 诸 亲 , 我 今 天 去 做 和 尚 了 , 还 来 不 及 当 面 辞 别 , 不 用 怪 怪 。
丈 人 呵 , 你 还 好 好 好 好 地 看 待 我 浑 家 , 只 怕 我 们 取 不 成 常 时 , 好 来 还 俗 , 照 旧 与 你 作 女 婿 过 活 。
走 路 的 人 喝 道 : 我 们 这 个 钱 , 你 们 却 不 要 说 什 么 。
八 戒 说 : 不 是 胡 说 , 只 怕 一 时 间 有 些 儿 差 池 , 却 不 是 和 尚 误 了 做 , 老 婆 误 了 娶 , 两 下 里 都 耽 误 了 。 三 藏 说 : 少 题 闲 话 , 我 们 赶 早 儿 去 来 。
于 是 收 拾 了 一 担 行 李 , 八 戒 担 着 , 背 着 白 马 , 三 藏 骑 着 ; 走 路 的 人 肩 扛 着 铁 棒 , 前 面 引 路 。
一 行 三 次 , 辞 别 高 老 及 众 亲 友 , 投 西 而 去 。
有 诗 作 证 。
诗 说 : 满 地 烟 霞 树 色 高 , 唐 朝 佛 子 苦 劳 劳 。
饥 饿 时 吃 一 钵 千 家 饭 , 寒 冷 时 穿 千 针 一 衲 袍 。
意 马 心 头 休 放 荡 , 心 猿 乖 劣 莫 教 啼 啼 。
情 和 性 定 各 缘 合 , 月 亮 满 金 华 就 是 伐 毛 。
三 军 进 军 西 进 , 路 途 上 有 一 个 月 平 安 稳 定 。
走 过 乌 斯 藏 县 境 , 王 猛 抬 头 看 见 一 座 高 山 。
三 藏 停 下 鞭 子 勒 马 说 : 悟 空 、 悟 能 , 前 面 的 山 高 , 必 须 索 要 仔 细 。
八 戒 说 : 没 事 。
此 山 叫 做 浮 屠 山 , 山 中 有 个 乌 巢 禅 师 , 在 此 修 行 , 老 猪 也 曾 经 会 见 过 他 。
三 藏 说 : 他 有 什 么 勾 当 ? 八 戒 说 : 他 倒 也 有 些 道 行 。
他 曾 经 劝 我 修 行 , 我 不 曾 离 开 罢 了 。
师 子 们 说 话 , 不 多 久 , 就 到 了 山 上 。
好 山 , 只 见 到 那 里 , 山 南 有 青 松 碧 梓 , 山 北 有 绿 柳 红 桃 。
喧 闹 喧 闹 , 山 禽 对 语 , 舞 翩 翩 , 仙 鹤 齐 飞 。
香 气 馥 郁 , 各 种 花 朵 千 种 颜 色 ; 青 冉 冉 , 杂 草 万 种 奇 异 。
涧 下 有 滔 滔 绿 水 , 崖 前 有 朵 朵 祥 云 。
真 是 景 致 非 常 幽 雅 的 地 方 , 寂 静 不 见 往 来 的 人 。
那 师 父 在 马 上 远 远 地 观 看 , 看 见 香 栎 树 前 有 一 个 柴 草 窝 , 左 边 有 麋 鹿 花 , 右 边 有 山 猴 献 果 , 树 梢 头 有 青 鸾 、 彩 凤 一 齐 鸣 叫 , 玄 鹤 、 锦 鸡 都 聚 集 在 这 里 。
八 戒 指 着 他 说 : 那 不 是 乌 巢 禅 师 ? 三 藏 纵 马 加 鞭 , 直 到 树 下 。
又 说 : 那 禅 师 见 他 三 个 人 前 来 , 就 离 开 了 巢 穴 , 跳 下 树 来 。
三 藏 下 马 拜 见 , 那 禅 师 用 手 搀 着 说 : 圣 僧 请 让 我 起 来 。
失 去 迎 接 , 失 去 迎 接 。
八 戒 说 : 老 禅 师 , 作 揖 了 。
禅 师 吃 惊 地 问 道 : 你 是 福 陵 山 的 猪 刚 鬣 , 为 什 么 有 这 样 大 的 缘 分 , 能 和 圣 僧 同 行 呢 ? 八 戒 说 : 前 年 受 观 音 菩 萨 劝 善 , 愿 随 他 做 个 徒 弟 。
禅 师 非 常 高 兴 地 说 : 好 , 好 , 好 啊 又 指 着 行 人 问 道 : 此 位 是 谁 ? 行 者 笑 着 说 : 这 位 老 禅 怎 么 认 得 他 , 倒 不 认 得 我 呢 ? 禅 师 说 : 因 为 我 很 少 认 识 罢 了 。
三 藏 说 : 他 是 我 的 大 弟 弟 孙 悟 空 。
禅 师 陪 着 笑 着 说 : 欠 礼 , 欠 礼 。
三 藏 再 拜 说 : 请 问 西 天 大 雷 音 寺 还 在 哪 里 ? 禅 师 说 : 远 哩 , 远 哩 。
只 是 路 上 有 很 多 虎 豹 , 难 以 走 。
三 藏 恭 敬 地 向 他 致 意 , 再 问 道 : 路 途 果 然 有 多 远 ? 禅 师 说 : 路 途 虽 然 远 , 最 终 必 须 有 到 的 日 子 , 却 只 是 魔 瘴 难 以 消 除 。
我 有 《 多 心 经 》 一 卷 , 共 五 十 四 句 , 共 计 二 百 七 十 字 。
如 果 遇 到 魔 瘴 的 地 方 , 只 要 念 诵 这 部 经 , 自 然 没 有 伤 害 。
三 藏 拜 伏 在 地 上 恳 求 , 那 禅 师 就 口 头 诵 读 传 授 。
《 摩 诃 般 若 波 罗 蜜 多 心 经 》 中 说 : 观 自 在 菩 萨 , 行 深 般 若 波 罗 蜜 多 , 时 常 照 见 五 蕴 都 是 空 , 度 过 一 切 苦 难 。
舍 利 子 的 颜 色 和 颜 色 没 有 区 别 于 空 中 , 空 中 没 有 区 别 于 颜 色 , 颜 色 就 是 空 , 空 就 是 色 。
受 、 想 、 行 、 知 、 认 , 也 是 这 样 的 。
舍 利 子 , 这 些 种 种 佛 法 的 空 相 , 不 生 不 灭 , 不 坏 不 净 , 不 增 不 减 。
因 此 , 空 中 没 有 色 , 没 有 受 想 、 行 识 , 没 有 眼 耳 、 鼻 舌 、 身 意 , 没 有 色 、 声 、 香 、 味 、 触 、 法 , 没 有 眼 界 , 至 于 没 有 意 识 界 , 没 有 无 明 , 也 没 有 无 明 。
至 于 没 有 老 死 的 , 也 没 有 老 死 的 。
无 苦 寂 灭 道 , 没 有 智 慧 也 不 会 得 到 。
因 为 没 有 什 么 得 到 的 缘 故 , 就 是 菩 提 萨 。
依 据 般 若 波 罗 蜜 多 的 缘 故 , 心 里 没 有 什 么 障 碍 , 心 里 没 有 什 么 障 碍 , 心 里 没 有 什 么 畏 惧 的 缘 故 , 就 没 有 什 么 恐 惧 。
远 离 颠 倒 的 梦 想 , 到 底 都 是 涅 槃 。
三 世 诸 佛 , 依 据 般 若 波 罗 蜜 多 的 缘 故 , 就 可 以 得 到 了 阿 耨 多 罗 三 昧 三 昧 三 菩 提 。
因 此 可 知 , 般 若 波 罗 蜜 多 是 大 神 咒 , 大 明 咒 , 无 上 咒 , 无 等 咒 , 能 除 一 切 苦 , 真 实 不 虚 。
所 以 说 般 若 波 罗 蜜 多 咒 , 就 说 咒 道 : 揭 谛 揭 谛 , 波 罗 揭 谛 , 波 罗 僧 揭 谛 , 菩 提 萨 婆 诃 。 这 时 唐 朝 法 师 本 来 就 有 根 源 , 耳 闻 一 遍 《 多 心 经 》 , 就 能 记 住 , 至 今 传 世 。
这 是 修 真 的 总 经 , 作 佛 的 会 门 。
那 禅 师 传 了 经 文 , 踏 着 云 光 , 要 上 到 乌 巢 去 。
被 三 藏 又 拽 住 奉 告 , 一 定 要 问 一 个 西 去 的 路 程 端 的 。
那 禅 师 笑 着 说 : 道 路 不 难 走 , 试 听 我 的 吩 咐 。
千 山 千 水 深 , 有 许 多 瘴 气 和 魔 魅 的 地 方 。
如 果 遇 到 接 天 崖 , 放 心 休 恐 惧 。
走 来 摩 耳 岩 , 侧 着 脚 步 行 走 。
仔 细 看 到 黑 松 林 , 妖 狐 多 截 断 道 路 。
精 灵 充 满 国 城 , 魔 主 满 山 住 。
老 虎 坐 在 琴 堂 上 , 苍 狼 做 主 簿 。
虎 、 象 都 称 王 , 虎 、 豹 都 作 为 驾 御 。
野 猪 挑 着 担 子 , 水 怪 在 前 头 遇 到 。
多 年 的 老 石 猴 , 那 里 心 怀 怨 怒 。
你 问 你 什 么 相 识 , 他 知 道 西 去 的 路 。
行 人 听 了 , 冷 笑 着 说 : 我 们 去 , 不 必 问 他 们 , 问 我 就 可 以 了 。
三 藏 回 来 后 , 不 明 白 他 的 意 思 。
那 禅 师 变 成 金 光 , 径 直 上 乌 巢 而 去 。
长 老 向 皇 上 拜 谢 , 行 人 心 中 大 怒 , 举 起 铁 棒 望 着 皇 上 乱 捣 , 只 见 莲 花 生 出 万 朵 , 祥 瑞 的 雾 气 护 护 着 千 层 。
行 路 的 人 即 使 有 搅 海 翻 江 的 力 气 , 也 不 想 挽 着 乌 巢 一 缕 藤 。
三 藏 见 了 , 拍 住 行 者 说 : 悟 空 , 这 样 一 个 菩 萨 , 你 捣 他 窝 窝 的 巢 子 怎 么 样 ? 行 者 说 : 他 骂 了 我 兄 弟 两 个 一 场 去 了 。
三 藏 说 : 他 讲 的 西 天 路 径 , 何 尝 骂 你 ? 行 者 说 : 你 哪 里 知 道 , 他 说 野 猪 挑 担 子 是 骂 的 八 戒 , 多 年 老 石 猴 是 骂 的 老 孙 。
你 怎 么 知 道 这 个 意 思 ? 八 戒 说 : 师 兄 息 怒 。
这 位 禅 师 也 知 道 过 去 未 来 的 事 情 , 只 看 他 的 水 怪 前 头 遇 这 句 话 , 不 知 道 应 验 了 , 饶 他 去 罢 了 。
走 路 的 人 看 见 莲 花 祥 雾 , 靠 近 那 个 巢 穴 的 边 境 , 只 得 请 师 父 上 马 , 下 山 往 西 走 。
那 一 去 : 管 教 清 廉 福 佑 人 间 少 , 致 使 灾 魔 山 里 多 。
最 终 不 知 道 前 程 的 意 思 是 怎 样 的 , 暂 且 听 下 回 的 分 析 。
}\switchcolumn\flushpage  \begin{pinyinscope}{\myfontt \section{第二○回}     黃風嶺唐僧有難 半山中八戒爭先

法本從心生,還是從心滅。
    生滅盡由誰?請君自辨別。
    既然皆己心,何用別人說?
    只須下苦功,扭出鐵中血。
    絨繩著鼻穿,挽定虛空結。
    拴在無為樹,不使他顛劣。
    莫認賊為子,心法都忘絕。
    休教他瞞我,一拳先打徹。
    現心亦無心,現法法也輟。
    人牛不見時,碧天光皎潔。
    秋月一般圓,彼此難分別。
這一篇偈子,乃是玄奘法師悟徹了《多心經》,打開了門戶,那長老常念常存,
一點靈光自透。
  
且說他三眾在路餐風宿水,帶月披星,早又至夏景炎天。但見那:
    花盡蝶無情敘,樹高蟬有聲喧。
    野蠶成繭火榴妍,沼內新荷出現。

  
那日正行時,忽然天晚,又見山路傍邊有一村舍。三藏道:「悟空,你看那日落
西山藏火鏡,月升東海現冰輪。幸而道傍有一人家,我們且借宿一宵,明日再走
。」八戒道:「說得是,我老豬也有些餓了,且到人家化些齋吃,有力氣,好挑
行李。」行者道:「這個戀家鬼,你離了家幾日,就生報怨。」八戒道:「哥呵
,比不得你這喝風呵煙的人。我從跟了師父這幾日,長忍半肚饑,你可曉得?」
三藏聞之道:「悟能,你若是在家心重呵,不是個出家的了,你還回去罷。」那
獃子慌得跪下道:「師父,你莫聽師兄之言,他有些贓埋人。我不曾報怨甚的,
他就說我報怨。我是個直腸的痴漢,我說道肚內饑了,好尋個人家化齋,他就罵
我是戀家鬼。師父呵,我受了菩薩的戒行,又承師父憐憫,情願要伏侍師父往西
天去,誓無退悔。這叫做『恨苦修行』。怎的說不是出家的話?」三藏道:「既
是如此,你且起來。」
  
那獃子縱身跳起,口裏絮絮叨叨的,挑著擔子,只得死心塌地,跟著前來。早到
了路傍人家門首。三藏下馬,行者接了韁繩,八戒歇了行李,都佇立綠蔭之下。
三藏拄著九環錫杖,按按藤纏篾織斗篷,先奔門前。只見一老者,斜倚竹床之上
口裏嚶嚶的念佛。三藏不敢高言,慢慢的叫一聲:「施主,問訊了。」那老者一
骨魯跳將起來,忙斂衣襟,出門還禮道:「長老,失迎。你自那方來的?到我寒
門何故?」三藏道:「貧僧是東土大唐和尚,奉聖旨,上雷音寺拜佛求經。適至
寶方天晚,意投檀府告借一宵,萬祈方便方便。」那老兒擺手搖頭道:「去不得
,西天難取經。要取經,往東天去罷。」三藏口中不語,意下沉吟:「菩薩指道
西去,怎麼此老說往東行?東邊那得有經?」靦腆難言,半晌不答。
  卻說行者素性兇頑,忍不住,上前高叫道:「那老兒,你這們大年紀,全不
曉事。我出家人遠來借宿,就把這厭鈍的話虎諕我。十分你家窄狹,沒處睡時,
我們在樹底下,好道也坐一夜,不打攪你。」那老者扯住三藏道:「師父,你倒
不言語,你那個徒弟,那般拐子臉別頦腮,雷公嘴,紅眼睛,一個癆病魔鬼,怎
麼反沖撞我這年老之人?」行者笑道:「你這個老兒,忒也沒眼色。似那俊刮些
兒的,叫做中看不中吃。想我老孫雖小,頗結實,皮裹一團筋哩。」那老者道:
「你想必有些手段。」行者道:「不敢誇言,也將就看得過。」老者道:「你家
居何處?因甚事削髮為僧?」行者道:「老孫祖貫東勝神洲海東傲來國花果山水
簾洞居住。自小兒學做妖怪,稱名悟空。憑本事,做了一個齊天大聖。只因不受
天錄,大反天宮,惹了一場災愆。如今脫難消災,轉拜沙門,前求正果。保我這
唐朝駕下的師父,上西天拜佛走遭,怕甚麼山高路險,水闊波狂?我老孫也捉得
怪,降得魔,伏虎擒龍,踢天弄井,都曉得些兒。倘若府上有甚麼丟磚打瓦、鍋
叫門開,老孫便能安鎮。」
  
那老兒聽得這篇言語,哈哈笑道:「原來是個撞頭化緣的熟嘴兒和尚。」行者道
:「你兒子便是熟嘴。我這些時,只因跟我師父走路辛苦,還懶說話哩。」那老
兒道:「若是你不辛苦,不懶說話,好道活活的聒殺我。你既有這樣手段,西方
也還去得,去得。你一行幾眾?請至茅舍裏安宿。」三藏道:「多蒙老施主不叱
之恩。我一行三眾。」老者道:「那一眾在那裏?」行者指著道:「這老兒眼花
,那綠蔭下站的不是?」老兒果然眼花,忽抬頭細看,一見八戒這般嘴臉,就諕
得一步一跌,往屋裏亂跑,只叫:「關門,關門,妖怪來了!」行者趕上扯住道
:「老兒莫怕,他不是妖怪,是我師弟。」老者戰兢兢的道:「好好好,一個醜
似一個的和尚。」八戒上前道:「老官兒,你若以相貌取人,乾淨差了。我們醜
自醜,卻都有用。」
  
那老者正在門前與三個和尚相講,只見那莊南邊有兩個少年人,帶著一個老媽媽
、三四個小男女,斂衣赤腳,插秧而回。他看見一匹白馬、一擔行李,都在他家
門首喧嘩,不知是甚來歷,都一擁上前問道:「做甚麼的?」八戒調過頭來,把
耳朵擺了幾擺,長嘴伸了一伸,嚇得那些人東倒西歪,亂蹡亂跌。慌得那三藏滿
口招呼道:「莫怕,莫怕。我們不是歹人,我們是取經的和尚。」那老兒才出了
門,攙著媽媽道:「婆婆起來,少要驚恐。這師父是唐朝來的,只是他徒弟臉嘴
醜些,卻也面惡人善。帶男女們家去。」那媽媽才扯著老兒,二少年領著兒女進
去。
  
三藏卻坐在他門樓裏竹床之上,埋怨道:「徒弟呀,你兩個相貌既醜,言語又粗
,把這一家兒嚇得七損八傷,都替我身造罪哩。」八戒道:「不瞞師父說,老豬
自從跟了你,這些時俊了許多哩。若像往常在高老莊時,把嘴朝前一掬,把耳兩
頭一擺,常嚇殺二三十人哩。」行者笑道:「獃子不要亂說,把那醜也收拾起些
。」三藏道:「你看悟空說的話,相貌是生成的,你教他怎麼收拾?」行者道:
「把那個耙子嘴揣在懷裏,莫拿出來﹔把那蒲扇耳貼在後面,不要搖動:這就是
收拾了。」那八戒真個把嘴揣了,把耳貼了,拱著頭,立於左右。行者將行李拿
入門裏,將白馬拴在樁上。
  
只見那老兒才引個少年,拿一個板盤兒,托三杯清茶來獻。茶罷,又吩咐辦齋。
那少年又拿一張有窟窿無漆水的舊桌,端兩條破頭折腳的凳子,放在天井中,請
三眾涼處坐下。三藏方問道:「老施主高姓?」老者道:「在下姓王。」「有幾
位令嗣?」道:「有兩個小兒,三個小孫。」三藏道:「恭喜,恭喜。」又問:
「年壽幾何?」道:「痴長六十一歲。」行者道:「好,好,好,花甲重逢矣。」
三藏復問道:「老施主,始初說西天經難取者,何也?」老者道:「經非難取,
只是道中艱澀難行。我們這向西去,只有三十里遠近,有一座山,叫做八百里黃
風嶺,那山中多有妖怪。故言難取者,此也。若論此位小長老,說有許多手段,
卻也去得。」行者道:「不妨,不妨。有了老孫與我這師弟,任他是甚麼妖怪,
不敢惹我。」
  
正說處,又見兒子拿將飯來,擺在桌上,道聲:「請齋。」三藏就合掌諷起齋經
。八戒早已吞了一碗。長老的幾句經還未了,那獃子又吃勾三碗。行者道:「這
個?糠的,好道撞著餓鬼了。」那老王倒也知趣,見他吃得快,道:「這個長老
,想著實餓了,快添飯來。」那獃子真個食腸大,看他不抬頭,一連就吃有十數
碗。三藏、行者俱各吃不上兩碗。獃子不住,便還吃哩。老王道:「倉卒無殽,
不敢苦勸,請再進一箸。」三藏、行者俱道:「勾了。」八戒道:「老兒滴答甚
麼,誰和你發課,說甚麼五爻六爻?有飯只管添將來就是。」獃子一頓,把他一
家子飯都吃得罄盡,還只說才得半飽。卻才收了家火,在那門樓下,安排了竹床
板鋪睡下。
  
次日天曉,行者去背馬,八戒去整擔。老王又教媽媽整治些點心湯水管待,三眾
方致謝告行。老者道:「此去倘路間有甚不虞,是必還來茅舍。」行者道:「老
兒,莫說哈話。我們出家人不走回頭路。」遂此策馬挑擔西行。
  
噫!這一去,果無好路朝西域,定有邪魔降大災。三眾前來,不上半日,果逢一
座高山,說起來十分險峻。三藏馬到臨崖,斜挑寶觀看, 果然那:
高的是山,峻的是嶺﹔陟的是崖,深的是壑﹔響的是泉,鮮的是花。那山高不高
,頂上接青霄﹔這澗深不深,底中見地府。山前面,有骨都都白雲,屹嶝嶝怪石
,說不盡千丈萬丈挾魂崖。崖後有彎彎曲曲藏龍洞,洞中有叮叮噹噹滴水巖。又
見些丫丫叉叉帶角鹿,泥泥痴痴看人獐,盤盤曲曲紅鱗蟒,耍耍頑頑白面猿。至
晚巴山尋穴虎,帶曉翻波出水龍,登的洞門?喇喇響。草裏飛禽撲轤轤起,林中
走獸掬行。猛然一陣狼蟲過,嚇得人心趷蹬蹬驚。正是那當倒洞當當倒洞,洞當
當倒洞當山。青岱染成千丈玉,碧紗籠罩萬堆煙。
  
那師父緩促銀驄,孫大聖停雲慢步,豬悟能磨擔徐行。正看那山,忽聞得一陣旋
風大作。三藏在馬上心驚,道:「悟空,風起了。」行者道:「風 卻怕他怎的?
此乃天家四時之氣,有何懼哉?」三藏道:「此風甚惡,比那天風不同。」行者
道:「怎見得不比天風?」三藏道:「你看這風:
    巍巍蕩蕩颯飄飄,渺渺茫茫出碧霄。
    過嶺只聞千樹吼,入林但見萬竿搖。
    岸邊擺柳連根動,園內吹花帶葉飄。
    收網漁舟皆緊纜,落篷客艇盡拋錨。
    途半征夫迷失路,山中樵子擔難挑。
    仙果林間猴子散,奇花叢內鹿兒逃。
    崖前檜柏顆顆倒,澗下松篁葉葉凋。
    播土揚塵沙迸迸,翻江攪海浪濤濤。」

八戒上前一把扯住行者道:「師兄,十分風大,我們且躲一躲兒乾淨。」行者笑
道:「兄弟不濟。風大時就躲,倘或親面撞見妖精,怎的是好?」八戒道:「哥
呵,你不曾聞得『避色如避仇,避風如避箭』哩?我們躲一躲,也不虧人。」行
者道:「且莫言語,等我把這風抓一把來聞一聞看。」八戒笑道:「師兄又扯空
頭謊了,風又好抓得過來聞?就是抓得來,便也漬了去了。」行者道:「兄弟,
你不知道老孫有個抓風之法。」好大聖,讓過風頭,把那風尾抓過來聞了一聞,
有些腥氣。道:「果然不是好風,這風的味道不是虎風,定是怪風,斷乎有些蹊
蹺。」
  
說不了,只見那山坡下剪尾跑蹄,跳出一隻斑斕猛虎。慌得那三藏坐不穩雕鞍,
翻根頭跌下白馬,斜倚在路傍,真個是魂飛魄散。八戒丟了行李,掣釘鈀,不讓
行者走上前,大喝一聲道:「孽畜,那裏走!」趕將去,劈頭就築。那隻虎直挺
挺站將起來,把那前左爪掄起,摳住自家的胸膛,往下一抓,滑剌的一聲,把個
皮剝將下來,站立道傍。你看他怎生惡相?咦!那模樣:
    血津津的赤剝身軀,紅媸媸的彎環腿足。
    火燄燄的兩鬢蓬鬆,硬搠搠的雙眉直豎。
    白森森的四個鋼牙,光耀耀的一雙金眼。
    氣昂昂的努力大哮,雄糾糾的厲聲高喊。

喊道:「慢來,慢來。吾當不是別人,乃是黃風大王部下的前路先鋒。今奉大王
嚴命,在山巡邏,要拿幾個凡夫去做案酒。你是那裏來的和尚,敢擅動兵器傷我
?」八戒罵道:「我把你這個孽畜!你是認不得我。我等不是那過路的凡夫,乃
東土大唐御弟三藏之弟子,奉旨上西方拜佛求經者。你早早的遠避他方,讓開大
路,休驚了我師父,饒你性命﹔若似前猖獗,鈀舉處,卻不留情。」那妖精那容
分說,急近步,丟一個架子,望八戒劈臉來抓﹔這八戒忙閃過,掄鈀就築。那怪
手無兵器,回身就走﹔八戒隨後趕來﹔那怪到了山坡下亂石叢中,取出兩口赤銅
刀,急掄起,轉身來迎。兩個在這坡前一往一來,一沖一撞的賭鬥。
  
那孫行者攙起唐僧道:「師父,你莫害怕。且坐住,等老孫去助助八戒,打倒那
怪好走。」三藏才坐將起來,戰兢兢的,口裏念著《多心經》不題。
  
那行者掣了鐵棒,喝聲叫:「拿了!」此時八戒抖搜精神,那怪敗下陣去。行者
道:「莫饒他,務要趕上。」他兩個掄起鈀,舉鐵棒,趕下山來。那怪慌了手腳
,使個金蟬脫殼計,打個滾,現了原身,依然是一隻猛虎。行者與八戒那裏肯捨
,趕著那虎,定要除根。那怪見他趕得至近,卻又摳著胸膛,剝下皮來,苫蓋在
那臥虎石上,脫真身,化一陣狂風,徑回路口。忽見著那師父正念《多心經》,
被他一把拿住,駕長風攝將去了。可憐那三藏呵,江流註定多磨折,寂滅門中功
行難。
  
那怪把唐僧擒來洞口,按住狂風,對把門的道:「你去報大王說,前路虎先鋒拿
了一個和尚,在門外聽令。」那洞主傳令,教拿進來。那虎先鋒腰插著兩口赤銅
刀,雙手捧著唐僧,上前跪下道:「大王,小將不才,蒙鈞令差往山上巡邏,忽
遇一個和尚,他是東土大唐駕下御弟三藏法師,上西方拜佛求經,被我擒來奉上
,聊具一饌。」
  
那洞主聞得此言,吃了一驚道:「我聞得前者有人傳說:三藏法師乃大唐奉旨意
取經的神僧﹔他手下有一個徒弟,名喚孫行者,神通廣大,智力高強。你怎麼能
勾捉得他來?」先鋒道:「他有兩個徒弟:先來的使一柄九齒釘鈀,他生得嘴長
耳大﹔又一個使一根金箍鐵棒,他生得火眼金睛。正趕著小將爭持,被小將使一
個金蟬脫殼之計,撤身得空,把這和尚拿來,奉獻大王,聊表一餐之敬。」洞主
道:「且莫吃他哩。」先鋒道:「大王,見食不食,呼為劣蹶?」洞主道:「你
不曉得。吃了他不打緊,只恐怕他那兩個徒弟上門吵鬧,未為穩便。且把他綁在
後園定風樁上,待三五日,他兩個不來攪擾,那時節,一則圖他身子乾淨,二來
不動口舌,卻不任我們心意?或煮或蒸,或煎或炒,慢慢的自在受用不遲。」先
鋒大喜道:「大王深謀遠慮,說得有理。」教:「小的們,拿了去。」
  
旁邊擁上七八個綁縛手,將唐僧拿去,好便似鷹拿燕雀,索綁繩纏。這的是苦命
江流思行者,遇難神僧想悟能。道聲:「徒弟呵!不知你在那山擒怪,何處降妖
,我卻被魔頭拿來,遭此毒害,幾時再得相見?好苦呵!你們若早些兒來,還救
得我命﹔若十分遲了,斷然不能保矣。」一邊嗟嘆,一邊淚落如雨。
  
卻說那行者、八戒趕那虎下山坡,只見那虎跑倒了,塌伏在崖前。行者舉棒儘力
一打,轉震得自己手疼。八戒復築了一鈀,亦將鈀齒迸起。原來是一張虎皮,蓋
著一塊臥虎石。行者大驚道:「不好了,不好了,中了他計也!」八戒道:「中
他甚計?」行者道:「這個叫做金蟬脫殼計:他將虎皮蓋在此,他卻走了。我們
且回去看看師父,莫遭毒手。」兩個急急轉來,早已不見了三藏。行者大叫如雷
道:「怎的好?師父已被他擒去了。」八戒即便牽著馬,眼中滴淚道:「天哪,
天哪!卻往那裏找尋?」行者抬著頭道:「莫哭,莫哭,一哭就挫了銳氣。橫豎
想只在此山,我們尋尋去來。」
  
他兩個果奔入山中,穿崗越嶺,行勾多時,只見那石崖之下聳出一座洞府。兩人
定步觀瞻,果然兇險。但見那:
疊障尖峰,迴巒古道。青松翠竹依依,綠柳碧梧冉冉。崖前有怪石雙雙,林內有
幽禽對對。澗水遠流沖石壁,山泉細滴漫沙堤。野雲片片,瑤草芊芊。妖狐狡兔
亂攛梭,角鹿香獐齊鬥勇。劈崖斜掛萬年籐,深壑半懸千歲柏。奕奕巍巍欺華嶽
,落花啼鳥賽天台。
  
行者道:「賢弟,你可將行李歇在藏風山凹之間,撒放馬匹,不要出頭。等老孫
去他門首與他賭鬥,必須拿住妖精,方才救得師父。」八戒道:「不消吩咐,請
快去。」
  
行者整一整直裰,束一束虎裙,掣了棒,撞至那門前,只見那門上有六個大字,
乃「黃風嶺黃風洞」。卻便丁字腳站定,執著棒,高叫道:「妖怪,趁早兒送我師
父出來,省得掀翻了你窩巢,屣平了你住處。」那小怪聞言,一個個害怕,戰兢
兢的跑入裏面報道:「大王,禍事了。」那黃風怪正坐間,問:「有何事?」小
妖道:「洞門外來了一個雷公嘴毛臉的和尚,手持著一根許大粗的鐵棒,要他師
父哩。」那洞主驚張,即喚虎先鋒道:「我教你去巡山,只該拿些山牛、野彘、
肥鹿、胡羊,怎麼拿那唐僧來,卻惹他那徒弟來此鬧吵,怎生區處?」先鋒道:
「大王放心穩便,高枕勿憂。小將不才,願帶領五十個小校出去,把那甚麼孫行
者拿來湊吃。」洞主道:「我這裏除了大小頭目,還有五七百名小校,憑你選擇
,領多少去。只要拿住那行者,我們才自自在在吃那和尚一塊肉,情願與你拜為
兄弟﹔但恐拿他不得,反傷了你,那時休得埋怨我也。」虎怪道:「放心,放心
。等我去來。」
  
果然點起五十名精壯小妖,擂鼓搖旗,纏兩口赤銅刀,騰出門來,厲聲高叫道:
「你是那裏來的個猴和尚,敢在此間大呼小叫的做甚?」行者罵道:「你這個剝
皮的畜生!你弄甚麼脫殼法兒,把我師父攝了,倒轉問我做甚。趁早好好送我師
父出來,還饒你這個性命。」虎怪道:「你師父是我拿了,要與我大王做頓下飯
。你識起倒,回去罷﹔不然,拿住你,一齊湊吃,卻不是買一個又饒一個?」行者
聞言,心中大怒,扢迸迸鋼牙錯嚙,滴流流火眼睜圓,掣鐵棒喝道:「你多大手
段,敢說這等大話?休走,看棍。」那先鋒急持刀接住。這一場果然不善,他兩
個各顯威能,好殺:
    那怪是個真鵝卵,悟空是個鵝卵石。
    赤銅刀架美猴王,渾如壘卵來擊石。
    鳥鵲怎與鳳凰爭,鵓鴿敢和鷹鷂敵。
    那怪噴風灰滿山,悟空吐霧雲迷日。
    來往不禁三五回,先鋒腰軟全無力。
    轉身敗了要逃生,卻被悟空抵死逼。

那虎怪抵架不住,回頭就走。他原來在那洞主面前說了嘴,不敢回洞,徑往山坡
上逃生。行者那裏肯放,執著棒,只情趕來,呼呼吼吼,喊聲不絕,卻趕到那藏
風山凹之間。正抬頭,見八戒在那裏放馬。八戒忽聽見呼呼聲喊,回頭觀看,乃
是行者趕敗的虎怪,就丟了馬,舉起鈀,刺斜著頭一築。可憐那先鋒,脫身要跳
黃絲網,豈知又遇罩魚人,卻被八戒一鈀,築得九個窟窿鮮血冒,一頭腦髓盡流
乾。有詩為證,詩曰:
    三五年前歸正宗,持齋把素悟真空。
    誠心要保唐三藏,初秉沙門立此功。

那獃子一腳屣住他的脊背,兩手掄鈀又築。行者見了,大喜道:「兄弟,正是這
等。他領了幾十個小妖,敢與老孫賭鬥,被我打敗了,他轉不往洞跑,卻跑來這
裏尋死。虧你接著,不然又走了。」八戒道:「弄風攝師父去的可是他?」行者
道:「正是,正是。」八戒道:「你可曾問他師父的下落麼?」行者道:「這怪
把師父拿在洞裏,要與他甚麼鳥大王做下飯。老孫惱了,就與他鬥將這裏來,卻
被你送了性命。兄弟呵,這個功勞算你的。你可還守著馬與行李,等我把這死怪
拖了去,再到那洞口索戰。須是拿得那老妖,方才救得師父。」八戒道:「哥哥
說得有理。你去,你去。若是打敗了這老妖,還趕將這裏來,等老豬截住殺他。」
  
好行者,一隻手提著鐵棒,一隻手拖著死虎,徑至他洞口。正是:
法師有難逢妖怪,情性相和伏亂魔。
  
    畢竟不知此去可降得妖怪,救得唐僧,且聽下回分解。





}  \end{pinyinscope}\switchcolumn{\myfontc \section{第 二 十 回} 黄 风 岭 , 唐 僧 在 半 山 中 发 难 , 八 戒 争 先 , 法 本 从 心 生 , 还 是 从 心 灭 。
生 与 死 都 是 由 谁 来 决 定 的 , 请 您 自 己 辨 别 。
既 然 都 是 自 己 的 心 , 又 何 必 别 人 说 , 只 需 要 下 苦 功 , 扭 出 铁 中 血 。
用 丝 织 成 的 丝 织 成 的 丝 织 品 , 用 丝 织 成 的 丝 织 品 , 用 绳 子 系 在 鼻 子 上 , 用 绳 子 拉 住 虚 空 的 绳 子 。
把 它 拴 在 无 为 树 上 , 不 让 它 颠 倒 。
不 要 认 为 贼 人 是 儿 子 , 心 法 都 忘 绝 了 。
不 要 让 他 欺 骗 我 , 一 拳 先 打 透 。
现 在 心 也 没 有 心 , 现 在 佛 法 也 没 有 了 。
人 和 牛 不 见 时 , 碧 天 光 亮 皎 洁 。
秋 月 一 样 圆 , 彼 此 难 分 。
这 一 篇 偈 子 , 是 玄 奘 法 师 觉 悟 了 《 多 心 经 》 , 打 开 了 门 户 , 那 么 长 老 的 念 头 常 常 存 在 , 一 点 灵 光 自 然 透 彻 。
并 且 说 他 们 三 个 人 在 路 上 吃 风 宿 水 , 带 着 月 亮 、 披 着 星 光 , 早 晨 又 到 了 夏 天 的 时 候 , 天 气 炎 热 的 时 候 。
只 见 那 个 人 说 : 花 尽 蝶 无 情 叙 述 , 树 高 蝉 有 声 音 喧 闹 。
野 蚕 织 成 茧 , 火 榴 艳 丽 , 沼 内 新 荷 出 现 。
那 天 正 走 的 时 候 , 忽 然 天 色 已 晚 , 又 看 见 山 路 旁 边 有 一 个 村 庄 。
三 藏 菩 萨 说 : 悟 空 , 你 看 那 太 阳 落 西 山 藏 火 镜 , 月 亮 升 东 海 现 冰 轮 。
幸 而 道 旁 有 一 个 人 家 , 我 们 暂 且 借 宿 一 夜 , 明 天 再 走 。
八 戒 说 : 说 得 对 , 我 的 老 猪 也 有 些 挨 饿 了 , 暂 且 到 别 人 家 做 饭 吃 , 有 力 气 , 好 挑 行 李 。
行 人 说 : 这 个 恋 家 鬼 , 你 离 开 家 几 天 , 就 生 出 报 仇 。
八 戒 说 : 哥 呵 , 比 不 得 你 这 个 喝 风 呵 烟 的 人 。
我 跟 随 了 师 父 这 几 天 , 长 期 忍 受 半 肚 饥 饿 , 你 可 以 知 道 了 三 藏 听 了 , 说 : 你 如 果 是 在 家 里 心 重 , 不 是 个 出 家 的 了 , 你 还 回 去 吧 。
那 个 子 惊 慌 地 跪 下 说 : 师 父 , 你 不 要 听 师 兄 的 话 , 他 有 些 赃 物 埋 人 。
吾 不 知 之 。
我 是 个 直 肠 的 痴 子 , 我 说 肚 里 饥 了 , 好 找 个 人 家 的 化 斋 , 他 就 骂 我 是 个 恋 家 鬼 。
师 父 呵 , 我 受 了 菩 萨 的 戒 行 , 又 承 蒙 师 父 的 怜 悯 怜 悯 , 愿 意 服 侍 师 父 去 西 天 去 , 发 誓 不 要 后 悔 。
其 所 以 为 之 , 其 所 以 为 之 , 其 所 以 为 之 。
三 藏 说 : 既 然 这 样 , 你 暂 且 起 来 。
子 纵 身 跳 起 , 口 里 纷 纷 , 挑 着 担 子 , 只 得 死 心 塌 地 , 随 着 前 来 。
早 早 到 了 路 旁 人 家 门 口 。
三 藏 下 马 , 走 路 的 人 拉 着 绳 子 , 八 戒 把 行 李 放 下 , 都 站 在 绿 荫 下 面 。
三 藏 拄 着 九 环 锡 杖 , 按 着 藤 藤 缠 着 的 绳 子 , 织 成 斗 篷 , 先 奔 到 门 前 。
只 见 一 个 老 人 , 斜 斜 地 站 在 竹 床 上 , 口 里 闷 闷 地 念 佛 。
三 藏 不 敢 说 话 , 慢 慢 地 叫 一 声 : 施 主 , 问 讯 了 。
那 个 老 人 一 个 骨 头 粗 鲁 地 跳 起 来 , 急 忙 收 拾 衣 襟 , 出 门 回 礼 说 : 长 老 , 不 要 迎 接 。
三 藏 说 : 贫 僧 是 东 土 大 唐 和 尚 , 奉 圣 旨 , 到 雷 音 寺 拜 佛 求 经 。
恰 好 到 了 萧 宝 方 天 色 已 晚 , 意 思 投 到 檀 府 请 求 借 用 一 夜 , 万 万 祈 求 方 便 方 便 。
那 个 老 人 摇 着 手 摇 着 头 说 : 去 不 得 , 西 天 难 取 经 。
要 取 经 书 , 往 东 天 去 吧 。
三 藏 口 中 不 说 话 , 心 中 沉 默 不 语 , 说 : 菩 萨 指 着 道 西 去 , 为 什 么 这 老 人 说 往 东 走 , 东 边 怎 么 能 有 经 呢 ?
又 说 行 者 一 向 性 情 凶 狠 , 忍 不 住 , 上 前 高 声 喊 道 : 那 老 子 , 你 这 些 大 年 纪 , 全 不 知 道 事 情 。
我 出 家 人 远 道 来 借 宿 , 就 用 这 些 厌 倦 的 话 虎 咬 咬 我 。
十 分 你 家 狭 窄 , 没 有 地 方 睡 觉 的 时 候 , 我 们 在 树 底 下 , 好 道 也 坐 一 夜 , 不 打 搅 你 们 。
那 个 老 人 拉 住 三 藏 说 : 师 父 , 你 倒 不 说 话 , 你 那 个 徒 弟 , 那 么 拐 子 脸 别 腮 , 雷 公 嘴 , 红 眼 睛 , 一 个 病 魔 鬼 , 怎 么 反 而 冲 撞 我 这 个 年 老 的 人 呢 ? 行 人 笑 着 说 : 你 这 个 老 子 , 太 太 没 有 眼 睛 。
好 像 那 俊 刮 些 儿 的 , 叫 做 中 看 不 中 吃 。
想 我 的 孙 子 虽 然 小 , 但 很 结 实 , 皮 裹 着 一 团 筋 。
那 个 老 人 说 : 你 想 一 定 有 些 手 段 。
行 人 说 : 我 不 敢 夸 大 说 , 也 要 看 得 过 。
老 人 说 : 你 家 住 在 什 么 地 方 , 因 为 有 什 么 事 要 削 发 为 和 尚 ? 行 人 说 : 老 孙 祖 贯 东 胜 神 洲 海 东 傲 来 国 花 果 山 水 帘 洞 居 住 。
从 小 就 学 了 妖 怪 , 称 他 为 悟 空 。
凭 借 自 己 的 本 事 , 做 了 齐 天 大 圣 。
只 因 为 没 有 接 受 上 天 的 记 录 , 大 反 天 宫 , 惹 了 一 场 灾 祸 。
如 今 他 们 脱 离 灾 难 消 除 灾 难 , 转 而 拜 为 僧 人 , 向 前 求 得 正 果 。
保 我 这 个 唐 朝 皇 帝 驾 下 的 师 父 , 上 到 西 天 拜 佛 , 走 上 了 , 怕 什 么 山 高 路 险 , 水 阔 波 狂 , 我 的 老 孙 子 也 捉 得 怪 , 投 降 了 魔 , 伏 虎 擒 龙 , 踢 天 弄 井 , 都 晓 得 些 儿 子 。
如 果 府 中 有 什 么 抛 砖 打 瓦 , 锅 叫 门 开 , 老 孙 子 就 能 安 稳 镇 守 。
那 老 儿 听 了 这 篇 话 , 哈 哈 笑 着 说 : 原 来 是 个 撞 头 化 缘 的 熟 嘴 儿 和 尚 。
行 人 说 : 你 儿 子 就 是 熟 嘴 。
我 这 些 时 候 , 只 因 为 跟 我 师 父 走 路 辛 苦 , 还 懒 说 话 。
那 个 老 子 说 : 如 果 是 你 不 辛 苦 , 不 懒 说 话 , 好 道 活 活 的 吵 杀 我 。
你 既 然 有 这 样 的 手 段 , 西 方 也 还 去 得 , 去 得 。
你 一 行 多 少 人 , 请 让 我 到 茅 屋 里 安 宿 。
三 藏 说 : 多 蒙 老 施 主 不 叱 之 恩 。
我 一 行 就 有 三 个 人 。
老 人 指 着 他 说 : 这 老 子 眼 花 , 那 绿 荫 下 站 的 不 是 老 儿 果 然 是 眼 花 , 忽 然 抬 头 细 看 , 一 见 八 戒 这 样 的 嘴 脸 , 就 吓 得 一 步 一 跌 , 往 屋 里 乱 跑 , 只 叫 : 关 门 , 妖 怪 来 了 。 行 人 赶 紧 上 去 扯 住 说 : 老 儿 不 是 妖 怪 , 是 我 师 弟 。
老 人 惊 慌 地 说 : 好 好 好 好 , 一 个 丑 似 一 个 和 尚 。
八 戒 上 前 说 : 老 官 儿 , 你 如 果 以 相 貌 取 人 , 干 净 就 好 了 。
我 们 自 己 丑 陋 , 却 都 有 用 。
那 个 老 人 正 在 门 前 与 三 个 和 尚 说 话 , 只 见 那 个 庄 子 南 边 有 两 个 少 年 人 , 带 着 一 个 老 婆 婆 , 三 四 个 小 男 女 , 抖 衣 赤 脚 , 插 秧 而 回 。
其 他 人 , 皆 在 其 家 门 口 喧 哗 , 不 知 道 是 什 么 缘 故 , 都 拥 上 前 问 道 : 做 什 么 样 的 ? 八 戒 调 过 头 来 , 把 耳 朵 放 了 几 个 , 长 嘴 伸 了 一 伸 , 吓 得 那 些 人 东 倒 西 歪 , 乱 跌 乱 跌 。
他 惊 慌 得 到 那 三 藏 满 口 招 呼 道 : 不 要 害 怕 , 不 要 害 怕 。
我 们 不 是 坏 人 , 我 们 是 取 经 的 和 尚 。
那 个 老 子 刚 刚 出 门 , 扶 着 母 亲 说 : 婆 婆 起 来 , 少 点 惊 恐 。
这 位 师 父 是 唐 朝 来 的 , 只 是 他 的 弟 弟 脸 嘴 很 丑 , 却 也 面 恶 人 善 。
带 着 男 女 们 家 去 。
其 母 刚 刚 拽 着 老 儿 , 两 个 少 年 领 着 儿 女 进 去 。
三 藏 又 坐 在 门 楼 里 的 竹 床 上 , 埋 怨 道 : 徒 弟 呀 , 你 们 两 个 相 貌 丑 陋 , 言 语 又 粗 糙 , 把 这 一 家 人 吓 得 七 损 八 伤 , 都 替 我 自 己 造 罪 呀 !
八 戒 说 : 不 骗 师 父 说 , 老 猪 自 从 跟 了 你 , 这 些 时 候 俊 杰 了 许 多 哩 !
像 以 前 在 高 老 庄 的 时 候 , 把 嘴 朝 前 一 捧 , 把 耳 朵 两 头 一 张 , 常 常 吓 死 二 三 十 个 人 。
行 人 笑 着 说 : 子 不 要 乱 说 , 把 那 丑 子 收 拾 起 来 。
三 藏 说 : 你 看 悟 空 说 的 话 , 相 貌 是 生 成 的 , 你 教 他 怎 么 收 拾 ? 行 者 说 : 把 那 个 碓 子 嘴 , 揣 在 怀 里 , 不 要 拿 出 来 ; 把 那 个 蒲 扇 耳 贴 在 后 面 , 不 要 摇 动 , 这 就 是 收 拾 了 。
八 戒 真 是 把 嘴 揣 住 , 把 耳 朵 贴 住 , 拱 着 头 , 立 在 左 右 。
行 人 把 行 李 拿 进 门 里 , 把 白 马 拴 在 木 桩 上 。
只 见 那 个 老 头 才 带 着 一 个 少 年 , 拿 着 一 个 板 盘 子 , 托 下 三 杯 清 茶 来 献 给 我 。
茶 罢 , 又 吩 咐 办 斋 。
又 取 一 张 破 头 折 脚 的 凳 子 , 放 在 天 井 中 , 请 三 个 人 在 凉 爽 的 地 方 坐 下 。
三 藏 正 问 道 : 老 施 主 姓 高 的 人 说 : 在 下 姓 王 。
他 说 : 我 有 两 个 小 儿 子 , 三 个 小 孙 子 。
三 藏 说 : 恭 喜 , 恭 喜 。
又 问 : 我 的 寿 命 有 多 少 ? 他 说 : 痴 子 长 六 十 一 岁 。
行 人 说 : 好 , 好 , 好 , 好 , 花 甲 重 新 遇 到 了 。
三 藏 又 问 道 : 老 施 主 , 开 始 说 西 天 经 难 取 的 原 因 , 是 什 么 原 因 呢 老 人 说 : 经 不 难 取 , 只 是 道 中 艰 涩 难 行 。
我 们 往 西 去 , 只 有 三 十 里 远 近 的 地 方 , 有 一 座 山 叫 八 百 里 黄 风 岭 , 那 山 中 有 许 多 妖 怪 。
所 以 说 难 以 取 得 的 原 因 就 是 这 个 原 因 。
如 果 谈 到 这 位 小 长 老 , 说 有 许 多 手 段 , 可 是 也 可 以 去 掉 。
行 人 说 : 不 妨 , 不 妨 。
有 了 老 孙 子 和 我 这 个 师 弟 , 任 凭 他 们 是 什 么 妖 怪 , 不 敢 惹 我 。
正 在 说 话 的 时 候 , 又 见 儿 子 拿 着 饭 来 , 放 在 桌 上 , 说 道 : 请 斋 饭 。
三 藏 就 合 起 手 来 讽 诵 《 斋 经 》 。
八 戒 早 已 吞 了 一 碗 。
长 老 的 几 句 经 还 没 有 完 , 那 子 又 吃 勾 三 碗 。
行 人 说 : 这 个 糠 的 , 好 道 撞 着 饿 鬼 了 。
老 王 倒 是 很 聪 明 , 见 他 吃 得 很 快 , 便 说 : 这 个 长 老 , 想 来 确 实 饿 了 , 快 点 点 点 点 点 点 点 点 点 点 点 点 点 点 点 点 点 点 点 点 点 点 点 点 点 点 点 吃 。
那 子 真 的 吃 肠 大 , 看 他 不 抬 头 , 一 连 就 吃 了 十 几 碗 。
三 藏 、 行 者 都 各 吃 不 上 两 碗 。
子 不 住 , 就 还 要 吃 哩 。
老 王 说 : 仓 促 之 间 没 有 饭 吃 , 不 敢 苦 苦 劝 说 , 请 再 给 我 一 个 筷 子 吧 。
三 藏 、 行 者 都 说 : 勾 了 。
八 戒 说 : 老 儿 滴 着 什 么 , 谁 和 你 发 课 , 说 什 么 五 爻 六 爻 , 有 饭 只 管 添 将 来 就 是 。
子 一 顿 , 把 他 一 家 的 饭 都 吃 光 了 , 还 只 说 才 吃 半 饱 。
后 来 又 收 了 家 里 的 火 , 在 那 个 门 楼 下 , 安 排 了 竹 床 板 铺 睡 下 。
第 二 天 天 亮 , 行 人 去 掉 背 马 , 八 戒 去 掉 整 担 子 。
老 王 又 教 我 整 理 点 心 汤 水 管 等 待 , 三 个 人 才 致 谢 告 别 。
老 人 说 : 此 去 如 果 路 上 有 什 么 不 测 , 这 一 定 会 回 到 茅 屋 。
行 人 说 : 老 儿 , 不 要 说 哈 话 。
我 们 出 家 人 , 不 走 回 头 路 。
于 是 就 策 马 挑 着 担 子 往 西 走 。
这 一 去 , 果 然 没 有 好 的 道 路 去 朝 拜 西 域 , 一 定 会 有 邪 魔 降 临 大 灾 。
三 众 前 来 , 不 到 半 天 , 果 然 遇 到 一 座 高 山 , 说 起 来 十 分 险 峻 。
三 藏 马 到 了 山 崖 上 , 斜 着 挑 着 宝 物 来 观 看 , 果 然 是 那 样 : 高 的 是 山 , 峻 的 是 岭 , 陡 峭 的 是 山 崖 , 深 的 是 壑 谷 , 响 的 是 泉 水 , 鲜 的 是 花 。
那 山 高 不 高 , 顶 上 连 接 着 青 霄 ; 此 涧 深 不 深 , 底 下 可 以 看 到 地 府 。
山 的 前 面 , 有 骨 都 都 白 云 , 屹 立 着 奇 异 的 岩 石 , 说 不 到 千 丈 万 丈 的 挟 魂 崖 。
山 崖 后 面 有 个 弯 弯 曲 曲 的 藏 龙 洞 , 洞 中 有 个 潺 潺 滴 水 岩 。
又 看 见 那 些 丫 丫 叉 叉 、 带 角 鹿 , 泥 泥 痴 呆 地 看 人 獐 , 盘 盘 曲 曲 , 红 鳞 蟒 , 戏 耍 耍 、 耍 耍 、 顽 蠢 的 白 面 猿 。
到 晚 上 , 巴 山 寻 找 洞 穴 虎 , 天 亮 时 翻 波 出 水 龙 , 登 上 洞 门 , 声 音 嘈 嘈 。
草 丛 中 飞 禽 扑 扑 而 起 , 林 中 走 兽 捧 着 行 走 。
猛 然 一 阵 狼 虫 飞 过 , 吓 得 人 心 惶 惶 不 安 。
正 是 那 当 倒 洞 , 当 当 倒 洞 , 洞 应 当 倒 洞 当 山 。
青 岱 被 染 成 千 丈 玉 石 , 碧 纱 笼 罩 万 堆 烟 。
那 师 父 慢 慢 地 催 促 银 , 孙 大 圣 停 下 来 说 : 慢 慢 行 走 , 猪 觉 得 能 磨 着 担 子 慢 慢 走 路 。
正 在 看 那 山 , 忽 然 听 到 一 阵 旋 风 大 作 。
三 藏 在 马 上 心 惊 , 说 : 悟 空 , 风 起 了 。
行 人 说 : 风 却 怕 它 怎 么 样 ? 这 是 天 家 四 时 之 气 , 有 什 么 害 怕 的 呢 ? 三 藏 说 : 这 风 很 恶 , 和 那 天 风 不 同 。
行 人 说 : 怎 么 见 得 不 比 天 风 ? 三 藏 说 : 你 看 这 风 , 巍 巍 荡 荡 , 飒 飒 飘 飘 , 渺 渺 茫 茫 出 碧 霄 。
过 岭 时 , 只 听 到 千 棵 大 树 吼 叫 , 进 入 树 林 , 只 见 万 竿 摇 动 。
岸 边 摆 开 的 柳 枝 连 根 动 摇 , 园 内 吹 花 带 叶 飘 。
收 网 捕 鱼 的 船 只 都 紧 紧 拉 绳 , 掉 下 船 篷 的 客 船 全 部 扔 掉 。
半 路 上 征 夫 迷 失 了 路 , 山 中 的 樵 夫 担 着 难 以 挑 起 来 。
仙 果 林 间 猴 子 散 , 奇 花 丛 中 鹿 儿 逃 走 。
石 崖 前 的 松 柏 树 颗 倒 下 , 涧 水 下 的 松 竹 树 叶 都 凋 落 了 。
播 土 扬 尘 沙 迸 裂 , 翻 江 搅 海 浪 涛 。
八 戒 上 前 一 把 把 行 人 拽 住 , 说 : 师 兄 , 十 分 风 大 , 我 们 暂 且 躲 避 一 个 儿 子 干 净 。
行 人 笑 着 说 : 兄 弟 不 能 成 功 。
风 大 时 就 躲 避 , 倘 若 亲 自 脸 面 撞 见 妖 精 , 怎 么 样 好 呢 ? 八 戒 说 : 哥 哥 呵 , 你 不 曾 听 说 过 避 色 如 避 仇 , 避 风 如 躲 箭 吗 ? 我 们 躲 避 一 次 躲 避 , 也 不 会 亏 损 人 。
行 人 说 : 暂 且 不 要 说 话 , 等 我 拿 这 风 一 把 来 听 一 听 。
八 戒 笑 着 说 : 师 兄 又 扯 空 头 谎 了 , 风 又 好 抓 得 过 来 闻 , 就 是 抓 得 来 , 就 也 泡 了 去 了 。
行 人 说 : 兄 弟 , 你 不 知 道 老 孙 子 有 抓 风 的 方 法 。
大 圣 人 , 让 他 过 风 头 , 把 那 风 尾 抓 过 来 , 听 了 一 遍 , 有 点 腥 气 。
说 : 果 然 不 是 好 风 , 这 风 的 味 道 不 是 虎 风 , 一 定 是 怪 风 , 断 然 有 些 歧 路 。
说 不 完 , 只 见 那 山 坡 下 剪 尾 跑 蹄 , 跳 出 一 只 斑 猛 虎 。
惊 得 那 三 藏 坐 不 稳 的 马 鞍 , 翻 翻 着 根 头 跌 下 白 马 , 斜 斜 地 背 靠 在 路 旁 , 真 是 魂 飞 魄 散 。
八 戒 把 行 李 扔 掉 , 拿 着 钉 硬 的 钢 子 , 不 让 行 人 走 上 前 , 大 声 喝 道 : 孽 畜 , 往 哪 里 走 啊 赶 快 走 , 劈 开 头 就 筑 起 来 。
其 一 只 老 虎 , 直 立 起 来 , 把 它 前 面 的 左 爪 起 来 , 咬 住 自 己 家 的 胸 膛 , 往 下 一 去 , 滑 刺 一 声 , 把 皮 剥 下 来 , 立 在 道 旁 边 。
你 看 他 怎 么 会 生 出 恶 相 呢 呀 , 那 个 模 样 , 血 鲜 赤 剥 身 体 , 红 的 弯 环 腿 脚 。
火 焰 焰 的 两 鬓 发 蓬 松 , 坚 硬 的 双 眉 直 竖 。
四 颗 钢 牙 , 光 闪 闪 , 一 双 金 眼 。
气 盛 而 气 盛 , 气 盛 而 气 盛 。
喊 道 : 慢 来 , 慢 来 。
我 不 是 别 人 , 而 是 黄 风 大 王 部 下 的 先 锋 。
现 在 奉 大 王 的 命 令 , 在 山 中 巡 逻 , 要 拿 几 个 凡 夫 去 做 案 酒 。
你 是 从 哪 里 来 的 和 尚 , 岂 敢 擅 自 动 兵 器 伤 害 我 ? 八 戒 骂 道 : 我 把 你 这 个 孽 畜 , 你 是 认 不 到 我 。
我 们 不 是 过 路 的 凡 夫 , 是 东 土 大 唐 御 弟 三 藏 的 弟 子 , 奉 旨 到 西 方 拜 佛 求 经 的 人 。
卿 早 早 避 开 其 他 地 方 , 让 我 开 大 路 , 不 要 惊 动 我 师 父 , 饶 你 性 命 ; 如 果 你 以 前 的 猖 獗 , 我 举 手 不 动 , 却 不 留 意 。
那 妖 精 哪 容 分 说 , 急 忙 走 近 步 , 扔 了 一 个 架 子 , 望 八 戒 劈 开 脸 来 捉 , 这 八 戒 忙 忙 跑 过 去 , 拿 着 铁 链 就 把 它 搭 起 来 。
那 怪 手 中 没 有 兵 器 , 回 身 就 走 了 , 八 戒 随 后 赶 来 。 那 怪 走 到 山 坡 下 的 乱 石 丛 中 , 取 出 两 口 红 铜 刀 , 急 忙 抬 起 来 , 转 身 来 迎 接 。
两 个 人 在 坡 前 , 一 往 一 来 , 一 冲 一 撞 。
那 个 孙 行 者 搀 着 唐 僧 说 : 师 父 , 你 不 要 害 怕 。
暂 且 坐 下 , 等 老 孙 去 帮 助 八 戒 , 打 倒 那 怪 就 好 走 了 。
三 藏 刚 坐 起 来 , 惊 慌 恐 惧 , 口 里 念 着 《 多 心 经 》 。
那 个 行 人 拿 着 铁 棒 , 大 声 喊 道 : 拿 了 。 这 时 八 戒 抖 了 精 神 , 那 怪 物 败 下 阵 去 。
行 人 说 : 不 要 饶 他 , 一 定 要 赶 上 去 。
其 两 个 起 鞘 , 举 铁 棒 , 快 下 山 来 。
那 怪 物 惊 慌 了 手 脚 , 让 个 金 蝉 脱 壳 计 , 打 个 滚 , 现 出 原 来 的 身 子 , 仍 然 是 一 只 猛 虎 。
行 者 和 八 戒 哪 里 肯 舍 弃 , 赶 着 那 虎 , 一 定 要 除 掉 根 。
那 怪 怪 见 他 赶 得 到 近 处 , 却 又 咬 着 他 的 胸 膛 , 剥 下 皮 来 , 用 草 苫 盖 在 那 卧 虎 石 上 , 脱 下 真 身 , 变 成 一 阵 狂 风 , 径 直 回 到 路 口 。
忽 然 看 见 他 的 师 父 正 念 《 多 心 经 》 , 被 他 一 把 拿 住 , 驾 着 长 风 驱 走 了 。
可 怜 那 三 藏 呵 , 江 流 注 定 多 磨 折 , 寂 灭 门 中 功 行 难 。
怪 怪 我 把 唐 僧 捉 到 洞 口 , 按 住 狂 风 , 对 着 门 的 人 说 : 你 去 报 告 大 王 说 , 前 路 虎 先 锋 捉 了 一 个 和 尚 , 在 门 外 听 从 命 令 。
那 个 洞 主 传 令 , 让 他 拿 进 去 。
那 虎 先 锋 腰 插 着 两 口 红 铜 刀 , 双 手 捧 着 唐 僧 , 上 前 跪 下 说 : 大 王 , 小 将 没 有 才 能 , 承 蒙 皇 上 的 命 令 派 我 去 山 上 巡 逻 , 忽 然 遇 见 一 个 和 尚 , 他 是 东 土 大 唐 皇 帝 的 御 弟 三 藏 法 师 , 上 西 方 拜 佛 求 经 , 被 我 捉 拿 来 奉 上 , 姑 且 备 了 一 餐 。
那 洞 主 听 到 了 这 句 话 , 吃 了 一 惊 说 : 我 听 说 前 面 有 人 传 说 : 三 藏 法 师 是 大 唐 奉 旨 求 经 的 神 僧 , 他 手 下 有 一 个 徒 弟 , 名 叫 孙 行 的 , 神 通 广 大 , 智 力 高 强 。
先 锋 说 : 他 有 两 个 弟 子 : 先 来 的 使 用 一 柄 九 齿 钉 钢 , 他 生 得 嘴 长 耳 大 ; 又 一 个 使 用 一 根 金 钳 铁 棒 , 他 生 得 火 眼 金 睛 。
正 赶 着 小 将 争 持 , 被 小 将 使 用 一 个 金 蝉 脱 壳 的 计 谋 , 撤 身 得 空 , 把 这 和 尚 拿 来 奉 献 给 大 王 , 姑 且 表 示 一 餐 之 敬 。
洞 主 说 : 暂 且 不 要 吃 他 哩 。
先 锋 说 : 大 王 , 你 看 见 食 不 食 , 叫 它 是 劣 踢 , 洞 主 说 : 你 不 知 道 。
吃 了 他 不 打 紧 , 只 恐 怕 他 的 两 个 弟 弟 上 门 喧 闹 , 不 算 稳 妥 。
况 且 把 它 绑 在 后 园 的 定 风 桩 上 , 等 三 五 天 后 , 他 两 个 不 来 搅 扰 , 那 时 节 , 一 则 是 图 他 身 子 干 净 , 二 来 是 不 动 口 舌 , 却 不 听 我 们 的 心 意 , 或 者 煮 、 或 者 煮 , 或 者 是 煮 , 或 者 是 炒 , 慢 慢 自 在 , 接 受 不 迟 。
先 锋 大 喜 说 : 大 王 深 谋 远 虑 , 说 得 有 理 。
教 诲 说 : 小 的 人 , 拿 了 去 吧 。
旁 边 抱 着 七 八 个 绑 着 的 手 , 把 唐 僧 拿 去 , 好 像 鹰 拿 燕 雀 , 用 绳 子 缠 住 。
这 是 苦 命 江 流 思 行 的 人 , 遇 难 神 僧 想 悟 能 。
道 士 说 : 徒 弟 呵 , 不 知 你 在 那 山 抓 住 怪 物 , 从 哪 里 招 来 妖 怪 , 我 却 被 魔 头 抓 来 , 遭 到 这 样 的 毒 害 , 何 时 再 能 见 到 你 们 呢 ? 你 们 如 果 早 些 儿 子 来 , 还 能 救 我 的 性 命 ; 如 果 十 分 迟 了 , 决 不 能 保 证 。
一 边 叹 息 , 一 边 泪 落 如 雨 。
又 说 : 那 行 者 、 八 戒 赶 着 那 老 虎 下 山 坡 , 只 见 那 老 虎 跑 倒 了 , 倒 伏 在 山 崖 前 。
走 路 的 人 举 起 棒 子 , 尽 力 打 一 下 , 转 动 得 自 己 的 手 疼 。
八 戒 又 筑 了 一 根 钢 子 , 也 把 钢 子 的 牙 齿 迸 开 了 起 来 。
原 来 是 一 张 虎 皮 , 盖 着 一 块 卧 虎 石 。
行 者 大 惊 道 : 不 好 了 , 不 好 了 , 中 了 他 的 计 策 。 八 戒 说 : 中 了 他 什 么 计 策 ? 行 者 说 : 这 个 叫 做 金 蝉 脱 壳 计 , 他 将 虎 皮 盖 在 这 里 , 他 却 逃 走 了 。
我 们 暂 且 回 去 看 看 师 父 , 不 要 遭 毒 手 。
两 个 人 急 忙 转 来 , 早 已 不 见 了 三 藏 。
行 人 大 叫 如 雷 , 说 : 你 怎 么 好 , 师 父 已 被 他 捉 去 了 。
八 戒 立 即 牵 着 马 , 眼 睛 流 着 泪 说 : 天 啊 , 天 啊 , 你 还 往 哪 里 去 寻 找 ? 行 人 抬 着 头 说 : 不 要 哭 , 不 要 哭 , 一 哭 就 挫 了 锐 气 。
横 竖 想 只 在 此 山 , 我 们 寻 找 去 来 。
两 个 人 果 然 奔 入 山 中 , 穿 山 越 岭 , 走 了 很 多 时 间 , 只 见 那 石 崖 之 下 耸 出 一 座 洞 府 。
两 人 定 步 观 看 , 果 然 是 凶 险 。
只 见 到 那 里 : 层 层 叠 叠 的 屏 障 尖 峰 , 回 绕 的 山 峰 古 道 。
青 松 翠 竹 依 依 , 绿 柳 碧 梧 冉 冉 。
崖 前 有 两 双 奇 石 , 林 中 有 幽 禽 对 对 。
涧 水 远 流 冲 击 石 壁 , 山 泉 细 流 漫 溢 沙 堤 。
野 云 片 片 , 瑶 草 芊 芊 。
妖 狐 狡 兔 乱 梭 , 角 鹿 香 獐 齐 斗 勇 。
劈 开 崖 壁 斜 挂 万 年 藤 条 , 深 壑 半 悬 千 年 柏 树 。
浩 浩 荡 荡 地 欺 负 华 岳 , 落 花 啼 鸟 赛 天 台 。
行 人 说 : 贤 弟 , 你 可 以 把 行 李 放 在 藏 风 山 凹 之 间 , 放 放 马 匹 , 不 要 出 头 。
等 到 老 子 去 掉 他 的 门 第 , 与 他 赌 斗 , 必 须 拿 住 妖 精 , 才 能 救 得 师 父 。
八 戒 说 : 不 必 给 我 吩 咐 , 请 你 快 点 去 吧 。
走 路 的 人 整 齐 整 齐 整 齐 一 个 直 , 捆 了 一 束 虎 裙 , 拿 着 棒 子 , 撞 到 那 门 前 , 只 见 那 门 上 有 六 个 大 字 , 是 黄 风 岭 黄 风 洞 。
忽 然 丁 字 脚 站 着 , 手 执 棒 子 , 高 声 喊 道 : 妖 怪 , 趁 早 儿 送 我 师 父 出 来 , 省 得 掀 翻 了 你 的 巢 窝 , 铲 平 了 你 的 住 处 。
其 小 怪 听 了 , 一 个 个 都 害 怕 , 战 兢 兢 兢 地 走 到 里 面 报 告 说 : 大 王 , 祸 事 了 。
那 个 黄 风 怪 正 在 坐 的 时 候 , 问 道 : 有 什 么 事 ? 小 妖 说 : 洞 门 外 来 了 一 个 雷 公 嘴 毛 面 的 和 尚 , 手 里 拿 着 一 根 大 粗 的 铁 棒 , 要 他 的 师 父 哩 。
那 洞 主 惊 慌 张 张 , 立 即 叫 来 虎 先 锋 说 : 我 教 你 去 巡 视 山 , 只 应 拿 些 山 牛 、 野 猪 、 肥 鹿 、 胡 羊 , 何 必 拿 那 唐 僧 来 , 却 让 他 那 个 徒 弟 来 这 里 闹 闹 , 哪 里 还 有 什 么 地 方 ? 先 锋 说 : 大 王 放 心 安 稳 , 高 枕 不 忧 虑 。
小 将 没 有 才 能 , 愿 意 带 领 五 十 个 小 校 出 去 , 把 那 些 孙 行 者 拿 来 吃 。
洞 主 说 : 我 们 这 里 除 了 大 小 头 目 , 还 有 五 七 百 名 小 校 , 凭 你 选 择 , 领 多 少 去 。
只 要 拿 住 那 行 人 , 我 们 才 自 己 在 吃 那 和 尚 一 块 肉 , 我 愿 意 与 你 拜 为 兄 弟 , 只 怕 你 们 不 得 , 反 而 伤 了 你 们 , 那 时 怎 么 能 埋 怨 我 呢 ?
石 虎 奇 怪 地 说 : 放 心 , 放 心 。
等 我 去 吧 。
果 然 点 起 五 十 名 精 壮 的 小 妖 , 擂 鼓 摇 旗 , 缠 着 两 口 红 铜 刀 , 腾 出 门 来 , 厉 声 大 叫 道 : 你 是 哪 里 来 的 个 猴 和 尚 , 胆 敢 在 这 里 大 叫 小 叫 做 吗 ? 行 人 骂 道 : 你 这 个 剥 皮 的 畜 生 , 你 弄 什 么 脱 壳 法 儿 , 把 我 师 父 抓 了 , 倒 转 问 我 做 什 么 ?
及 早 送 我 师 父 出 来 , 还 饶 你 的 性 命 。
石 虎 奇 怪 地 说 : 你 师 父 是 我 拿 的 , 要 给 我 大 王 做 下 饭 。
不 然 , 拿 住 你 , 一 齐 吃 , 却 不 是 买 一 个 又 饶 一 个 , 走 路 的 人 听 了 , 心 中 大 怒 , 裂 钢 牙 错 , 滴 流 流 火 , 眼 睛 睁 圆 , 拿 铁 棒 喝 道 : 你 多 大 的 手 段 , 敢 说 这 样 大 话 , 不 要 再 走 , 看 看 棍 子 。
先 锋 急 忙 拿 着 刀 来 接 他 。
这 一 场 果 然 不 好 , 其 他 两 个 人 都 显 耀 自 己 的 威 力 和 才 能 , 好 杀 他 们 : 那 怪 是 个 真 鹅 蛋 , 悟 空 是 个 鹅 蛋 石 。
红 铜 刀 架 着 美 猴 王 , 浑 如 块 蛋 来 打 石 头 。
鸟 鹊 怎 么 能 和 凤 凰 争 斗 呢 ? 鸽 敢 与 老 鹰 搏 击 敌 人 。
那 怪 喷 风 灰 满 山 , 悟 空 吐 雾 云 迷 日 。
来 往 不 止 三 五 回 , 先 锋 腰 肢 软 弱 , 完 全 没 有 力 气 。
转 身 败 , 想 逃 生 , 却 被 悟 空 抵 死 逼 迫 。
那 老 虎 奇 怪 地 拉 住 架 子 不 住 , 回 头 就 走 了 。
其 原 来 就 在 那 个 洞 主 面 前 说 了 嘴 , 不 敢 回 洞 去 , 径 直 往 山 坡 上 逃 生 。
行 人 怎 么 肯 放 你 , 拿 着 棒 子 , 只 想 赶 来 , 呼 叫 喊 叫 , 喊 声 不 断 , 又 赶 到 那 藏 风 山 凹 之 间 。
刚 抬 头 , 见 八 戒 在 那 里 放 马 。
八 戒 忽 然 听 见 呼 喊 声 呼 叫 , 回 头 观 看 , 原 来 是 行 人 赶 败 的 虎 怪 。
可 怜 那 先 锋 , 脱 身 要 跳 黄 丝 网 , 岂 知 又 遇 上 罩 鱼 人 , 却 被 八 戒 一 个 硬 硬 的 硬 硬 的 硬 硬 的 硬 硬 , 筑 起 了 九 个 洞 穴 , 鲜 血 冒 出 来 , 一 头 脑 髓 都 流 出 来 了 。
有 诗 作 证 , 诗 说 : 三 五 年 前 归 正 宗 , 持 斋 把 素 悟 真 空 。
诚 心 要 保 护 唐 三 藏 , 刚 刚 担 任 和 尚 立 下 这 样 的 功 劳 。
子 一 只 脚 咬 住 了 他 的 脊 背 , 两 只 手 拿 着 硬 硬 的 硬 硬 的 硬 硬 的 硬 硬 的 硬 硬 的 硬 硬 的 硬 硬 地 又 筑 起 来 。
行 人 见 了 , 非 常 高 兴 地 说 : 兄 弟 , 正 是 这 样 的 人 。
他 率 领 了 几 十 个 小 妖 , 敢 与 老 孙 子 赌 斗 , 被 我 打 败 了 , 他 转 而 不 往 洞 里 逃 , 又 跑 来 找 死 。
亏 你 接 着 , 不 然 又 走 了 。
八 戒 说 : 弄 风 摄 师 父 去 的 可 是 他 吗 ? 行 者 说 : 正 是 , 正 是 。
八 戒 说 : 你 可 以 曾 经 问 过 他 师 父 的 下 落 吗 ? 行 者 说 : 这 怪 物 把 师 父 拿 在 洞 里 , 要 给 他 什 么 鸟 大 王 做 下 饭 。
老 孙 恼 了 , 就 与 他 争 斗 , 拿 到 这 里 来 , 却 被 你 送 了 性 命 。
兄 弟 啊 , 这 个 功 劳 算 你 的 。
你 可 以 还 守 着 马 和 行 李 , 等 我 把 这 个 死 怪 拖 去 , 再 到 那 个 洞 口 去 找 战 。
必 须 是 拿 得 那 老 妖 , 才 能 救 得 师 父 。
八 戒 说 : 哥 哥 说 得 有 理 。
你 去 , 你 去 吧 。
如 果 是 打 败 了 这 个 老 妖 , 还 赶 快 拿 到 这 里 来 , 等 老 猪 截 住 杀 它 。
有 个 好 行 的 人 , 一 只 手 提 着 铁 棒 , 一 只 手 拖 着 死 虎 , 径 直 到 了 别 的 洞 口 。
正 是 法 师 有 难 逢 的 妖 怪 , 性 情 和 谐 , 就 会 埋 伏 了 乱 魔 。
我 竟 不 知 道 此 去 可 以 降 服 妖 怪 , 救 助 得 到 唐 僧 , 暂 且 听 从 下 回 分 解 。
}\switchcolumn\flushpage  \begin{pinyinscope}{\myfontt \section{第二一回}     護法設莊留大聖 須彌靈吉定風魔

卻說那五十個敗殘的小妖拿著些破旗、破鼓,撞入洞裏,報道:「大王,虎先鋒
戰不過那毛臉和尚,被他趕下東山坡去了。」老妖聞說,十分煩惱。正低頭不語
,默思計策,又有把前門的小妖道:「大王,虎先鋒被那毛臉和尚打殺了,拖在
門口罵戰哩。」那老妖聞言,愈加煩惱道:「這廝卻也無知。我倒不曾吃他師父
,他轉打殺我家先鋒,可恨!可恨!」叫:「取披掛來。我也只聞得講甚麼孫行
者,等我出去,看是個甚麼九頭八尾的和尚,拿他進來,與我虎先鋒對命。」眾
小妖急急抬出披掛。老妖結束齊整,綽一杆三股鋼叉,帥群妖跳出本洞。
  
那大聖停立門外,見那妖走將出來,著實驍勇。看他怎生打扮,但見那:
金盔晃日,金甲凝光。盔上纓飄山雉尾,羅袍罩甲淡鵝黃。勒甲絛盤龍耀彩,護
心鏡繞眼輝煌。鹿皮靴,槐花染色﹔錦圍裙,柳葉絨妝。手持三股鋼叉利,不亞
當年顯聖郎。
  
那老妖出得門來,厲聲高叫道:「那個是孫行者?」這行者腳屣著虎怪的皮囊,
手執著如意的鐵棒,答道:「你孫外公在此。送出我師父來。」那怪仔細觀看,
見行者身軀鄙猥,面容羸瘦,不滿四尺。笑道:「可憐,可憐。我只道是怎麼樣
扳翻不倒的好漢,原來是這般一個骷髏的病鬼。」行者笑道:「你這個兒子,忒
沒眼色。你外公雖是小小的,你若肯照頭打一叉柄,就長六尺。」那怪道:「你
硬著頭,吃吾一柄。」大聖公然不懼。那怪果打一下來。他把腰躬一躬,足長了
六尺,有一丈長短。慌得那妖把鋼叉按住,喝道:「孫行者,你怎麼把這護身的
變化法兒,拿來我門前使出?莫弄虛頭,走上來,我與你見見手段。」行者笑道
:「兒子呵,常言道:『留情不舉手,舉手不留情。』你外公手兒重重的,只怕
你捱不起這一棒。」那怪那容分說,撚轉鋼叉,望行者當胸就刺﹔這大聖正是會
家不忙,忙家不會,理開鐵棒,使一個「烏龍掠地勢」,撥開鋼叉,又照頭便打
。他二人在那黃風洞口,這一場好殺:
妖王發怒,大聖施威。妖王發怒,要拿行者抵先鋒﹔大聖施威,欲捉精靈救長老
。叉來棒架,棒去叉迎。一個是鎮山都總帥,一個是護法美猴王。初時還在塵埃
戰,後來各起在中央。點鋼叉,尖明銳利﹔如意棒,身黑箍黃。戳著的魂歸冥府
,打著的定見閻王。全憑著手疾眼快,必須要力壯身強。兩家捨死忘生戰,不知
那個平安那個傷。
  
那老妖與大聖鬥經三十回合,不分勝敗。這行者要見功績,使一個「身外身」的
手段:把毫毛揪下一把,用口嚼得粉碎,望上一噴,叫聲:「變!」變有百十個
行者,都是一樣打扮,各執一根鐵棒,把那怪圍在空中。那怪害怕,也使一般本
事:急回頭,望著巽地上,把口張了三張,呼的一口氣吹將出去,忽然間,一陣
黃風,從空刮起。好風,真個利害:
    冷冷颼颼天地變,無影無形黃沙旋。
    穿林折嶺倒松梅,播土揚塵崩嶺坫。
    黃河浪潑徹底渾,湘江水湧翻波轉。
    碧天振動斗牛宮,爭些刮倒森羅殿。
    五百羅漢鬧喧天,八大金剛齊嚷亂。
    文殊走了青毛獅,普賢白象難尋見。
    真武龜蛇失了群,梓橦騾子飄其韂。
    行商喊叫告蒼天,梢公拜許諸般願。
    煙波性命浪中流,名利殘生隨水辦。
    仙山洞府黑攸攸,海島蓬萊昏暗暗。
    老君難顧煉丹爐,壽星收了龍鬚扇。
    王母正去赴蟠桃,一風吹亂裙腰釧。
    二郎迷失灌州城,哪吒難取匣中劍。
    天王不見手心塔,魯班吊了金頭鑽。
    雷音寶闕倒三層,趙州石橋崩兩斷。
    一輪紅日蕩無光,滿天星斗皆昏亂。
    南山鳥往北山飛,東湖水向西湖漫。
    雌雄拆對不相呼,子母分離難叫喚。
    龍王遍海找夜叉,雷公到處尋閃電。
    十代閻王覓判官,地府牛頭追馬面。
    這風吹倒普陀山,捲起觀音經一卷。
    白蓮花卸海邊飛,吹倒菩薩十二院。
    盤古至今曾見風,不似這風來不善。
    唿喇喇,乾坤險不炸崩開,萬里江山都是顫。
  
那妖怪使出這陣狂風,就把孫大聖毫毛變的小行者刮得在那半空中卻似紡車兒一
般亂轉,莫想掄得棒,如何攏得身?慌得行者將毫毛一抖,收上身來。獨自個舉
著鐵棒,上前來打。又被那怪劈臉噴了一口黃風,把兩隻火眼金睛刮得緊緊閉合
,莫能睜開。因此難使鐵棒,遂敗下陣來。那妖收風回洞不題。
  
卻說豬八戒見那黃風大作,天地無光,牽著馬,守著擔,伏在山凹之間,也不敢
睜眼,不敢抬頭,口裏不住的念佛許願﹔又不知行者勝負何如,師父死活何如。
正在那疑思之時,卻早風定天晴。忽抬頭往那洞門前看處,卻也不見兵戈,不聞
鑼鼓。獃子又不敢上他門,又沒人看守馬匹、行李,果是進退兩難,愴惶不已。
憂慮間,只聽得孫大聖從西邊吆喝而來,他才欠身迎著道:「哥哥,好大風呵!
你從那裏走來?」行者擺手道:「利害,利害!我老孫自為人,不曾見這大風。
那老妖使一柄三股鋼叉,來與老孫交戰。戰到有三十餘合,是老孫使一個『身外
身』的本事。把他圍打,他甚著急,故弄出這陣風來。果是兇惡,刮得我站立不
住,收了本事,冒風而逃。──哏,好風!哏,好風!老孫也會呼風,也會喚雨
,不曾似這個妖精的風惡。」八戒道:「師兄,那妖精的武藝如何?」行者道:
「也看得過,叉法兒倒也齊整,與老孫也戰個手平。卻只是風惡了,難得贏他。」
八戒道:「似這般怎生救得師父?」行者道:「救師父且等再處。不知這裏可有
眼科先生,且教他把我眼醫治醫治。」八戒道:「你眼怎的來?」行者道:「我
被那怪一口風噴將來,吹得我眼珠酸痛,這會子冷淚常流。」八戒道:「哥呵,
這半山中,天色又晚,且莫說要甚麼眼科,連宿處也沒有了。」行者道:「要宿
處不難,我料著那妖精還不敢傷我師父,我們且找上大路,尋個人家住下,過此
一宵,明日天光,再來降妖罷。」八戒道:「正是,正是。」
  
他卻牽了馬,挑了擔,出山凹,行上路口。此時漸漸黃昏,只聽得路南山坡下有
犬吠之聲。二人停身觀看,乃是一家莊院,影影的有燈火光明。他兩個也不管有
路無路,漫草而行,直至那家門首。但見:
紫芝翳翳,白石蒼蒼。紫芝翳翳多青草,白石蒼蒼半綠苔。數點小螢光灼灼,一
林野樹密排排。香蘭馥郁,嫩竹新栽。清泉流曲澗,古柏倚深崖。地僻更無遊客
到,門前惟有野花開。
  
他兩個不敢擅入,只得叫一聲:「開門,開門!」那裏有一老者,帶幾個年幼的
農夫,叉鈀掃帚齊來,問道:「甚麼人?甚麼人?」行者躬身道:「我們是東土
大唐聖僧的徒弟。因往西方拜佛求經,路過此山,被黃風大王拿了我師父去了,
我們還未曾救得。天色已晚,特來府上告借一宵,萬望方便方便。」那老者答禮
道:「失迎,失迎。此間乃雲多人少之處,卻才聞得叫門,恐怕是妖狐、老虎及
山中強盜等類,故此小介愚頑,多有沖撞,不知是二位長老。請進,請進。」
  
他兄弟們牽馬挑擔而入,徑至裏邊,拴馬歇擔,與莊老拜見敘坐。又有蒼頭獻茶
。茶罷,捧出幾碗胡麻飯。飯畢,命設鋪就寢。行者道:「不睡還可,敢問善人
,貴地可有賣眼藥的?」老者道:「是那位長老害眼?」行者道:「不瞞你老人
家說,我們出家人自來無病,從不曉得害眼。」老人道:「既不害眼,如何討藥
?」行者道:「我們今日在黃風洞口救我師父,不期被那怪將一口風噴來,吹得
我眼珠酸痛,今有些眼淚汪汪,故此要尋眼藥。」那老者道:「善哉,善哉!你
這個長老,小小的年紀,怎麼說謊?那黃風大王,風最利害。他那風,比不得甚
麼春秋風、松竹風與那東西南北風。……」八戒道:「想必是夾腦風、羊耳風、
大麻風、偏正頭風?」長者道:「不是,不是。他叫做三昧神風。」行者道:
「怎見得?」老者道:「那風能吹天地暗,善刮鬼神愁,裂石崩崖惡,吹人命即
休。你們若遇著他那風吹了時,還想得活哩?只除是神仙,方可得無事。」行者
道:「果然,果然。我們雖不是神仙,神仙還是我的晚輩。這條命急切難休,卻
只是吹得我眼珠酸痛。」那老者道:「既如此說,也是個有來頭的人。我這敝處
卻無賣眼藥的。老漢也有些迎風冷淚,曾遇異人,傳了一方,名喚三花九子膏,
能治一切風眼。」行者聞言,低頭唱喏道:「願求些兒,點試點試。」那老者應
承,即走進去,取出一個瑪瑙石的小罐兒來,拔開塞口,用玉簪兒蘸出少許,與
行者點上,教他不得睜開,寧心睡覺,明早就好。點畢,收了石罐,徑領小介們
退於裏面。
  
八戒解包袱,展開鋪蓋,請行者安置。行者閉著眼亂摸。八戒笑道:「先生,你
的明杖兒呢?」行者道:「你這個饢糟的獃子,你照顧我做瞎子哩。」那獃子啞
啞的暗笑而睡。行者坐在鋪上,轉運神功,直到三更後方才睡下。
  
不覺又是五更將曉。行者抹抹臉,睜開眼道:「果然好藥,比常更有百分光明。」
卻轉頭後邊望望,呀!那裏得甚房舍窗門,但只見些老槐高柳,兄弟們都睡在那
綠莎茵上。那八戒醒來道:「哥哥,你嚷怎的?」行者道:「你睜開眼睛看看。」
獃子忽抬頭,見沒了人家,慌得一轂轆爬將起來道:「我的馬哩?」行者道:
「樹上拴的不是?」「行李呢?」行者道:「你頭邊放的不是?」八戒道:「這
家子也憊懶,他搬了,怎麼就不叫我們一聲?通得老豬知道,也好與你送些茶果
。想是躲門戶的,恐怕里長曉得,卻就連夜搬了。──噫!我們也忒睡得死,怎
麼他家拆房子,響也不聽見響響?」行者吸吸的笑道:「獃子,不要亂嚷。你看
那樹上是個甚麼紙帖兒?」八戒走上前,用手揭了,原來上面四句頌子云:
    莊居非是俗人居,護法伽藍點化廬。
    妙藥與君醫眼痛,盡心降怪莫躊躇。
  
行者道:「這夥強神,自換了龍馬,一向不曾點他,他倒又來弄虛頭。」八戒道
:「哥哥莫扯架子,他怎麼伏你點札?」行者道:「兄弟,你還不知哩。這護教
伽藍、六丁六甲、五方揭諦、四值功曹奉菩薩的法旨,暗保我師父者。自那日報
了名,只為這一向有了你,再不曾用他們,故不曾點札罷了。」八戒道:「哥哥
,他既奉法旨暗保師父,所以不能現身明顯,故此點化仙莊。你莫怪他,昨日也
虧他與你點眼,又虧他管了我們一頓齋飯,亦可謂盡心矣。你莫怪他,我們且去
救師父來。」行者道:「兄弟說得是。此處到那黃風洞口不遠,你且莫動身,只
在林子裏看馬守擔。等老孫去洞裏打聽打聽,看師父下落如何,再與他爭戰。」
八戒道:「正是這等,討一個死活的實信。假若師父死了,各人好尋頭幹事﹔若
是未死,我們好竭力盡心。」行者道:「莫亂談,我去也。」
  
他將身一縱,徑到他門首,門尚關著睡覺。行者不叫門,且不驚動妖怪,捻著訣
,念個咒語,搖身一變,變做一個花腳蚊蟲,真個小巧。有詩為證。詩曰:
    擾擾微形利喙,嚶嚶聲細如雷。
    蘭房紗帳善通隨,正愛炎天暖氣。
    只怕薰煙撲扇,偏憐燈火光輝。
    輕輕小小忒鑽疾,飛入妖精洞裏。

  
只見那把門的小妖正打鼾睡,行者往他臉上叮了一口,那小妖翻身醒了,道:
「我爺啞!好大蚊子,一口就叮了一個大疙疸。」忽睜眼道:「天亮了。」又聽
得支的一聲,二門開了。行者嚶嚶的飛將進去,只見那老妖吩咐各門上謹慎,一
壁廂收拾兵器:「只怕昨日那陣風不曾刮死孫行者,他今日必定還來,來時定教
他一命休矣。」
  
行者聽說,又飛過那廳堂,徑來後面,但見一層門關得甚緊。行者漫門縫兒鑽將
進去,原來是個大空園子,那壁廂定風樁上繩纏索綁著唐僧哩。那師父紛紛淚落
,心心只念著悟空、悟能,不知都在何處。行者停翅,叮在他光頭上,叫聲:
「師父。」那長老認得他的聲音,道:「悟空呵,想殺我也。你在那裏叫我哩?」
行者道:「師父,我在你頭上哩。你莫要心焦,少得煩惱。我們務必拿住妖精,
方才救得你的性命。」唐僧道:「徒弟呵,幾時才拿得妖精麼?」行者道:「拿
你的那虎怪,已被八戒打死了。只是老妖的風勢利害,料著只在今日,管取拿他
。你放心莫哭,我去啞。」
  
說聲去,嚶嚶的飛到前面。只見那老妖坐在上面,正點札各路頭目。又見那洞前
有一個小妖精,把個令字旗磨一磨,撞上廳來報道:「大王,小的巡山,才出門
,見一個長嘴大耳朵的和尚坐在林裏,若不是我跑得快些,幾乎被他捉住。卻不
見昨日那個毛臉和尚。」老妖道:「孫行者不在,想必是風吹死也﹔再不便去那
裏求救兵去了。」眾妖道:「大王,若果吹殺了他,是我們的造化﹔只恐吹不死
他,他去請些神兵來,卻怎生是好?」老妖道:「怕那甚麼神兵?若還定得我的
風勢,只除了靈吉菩薩來是,其餘何足懼也?」
  
行者在屋梁上,只聽得他這一句言語,不勝歡喜。即抽身飛出,現本相,來至林
中,叫聲:「兄弟。」八戒道:「哥,你往那裏去來?剛才一個打令字旗的妖精
,被我趕了去也。」行者笑道:「虧你,虧你。老孫變做蚊蟲兒,進他洞去探看
師父,原來師父被他綁在定風樁上哭哩。是老孫吩咐,教他莫哭。又飛在屋梁上
聽了一聽,只見那拿令字旗的喘噓噓的走進去報道:只是被你趕他,卻不見我。
老妖亂猜亂說,說老孫是風吹殺了,又說是請神兵去了。他卻自家供出一個人來
,甚妙,甚妙。」八戒道:「他供的是誰?」行者道:「他說怕甚麼神兵,那個
能定他的風勢,只除是靈吉菩薩來是。──但不知靈吉住在何處?」
  
正商議處,只見大路傍走出一個老公公來。你看他怎生模樣:
身健不扶拐杖,冰髯雪鬢蓬蓬。金花耀眼意朦朧,瘦骨衰筋強硬。  屈背低頭
緩步,龐眉赤臉如童。看他容貌是人稱,卻似壽星出洞。
  
八戒望見大喜道:「師兄,常言道:『要知山下路,須問去來人。』你上前問他
一聲,何如?」真個大聖藏了鐵棒,放下衣襟,上前叫道:「老公公,問訊了。」
那老者半答不答的還了個禮道:「你是那裏和尚?這曠野處,有何事幹?」行者
道:「我們是取經的聖僧。昨日在此失了師父,特來動問公公一聲:靈吉菩薩在
那裏住?」老者道:「靈吉在直南上,到那裏還有三千里路。有一山,呼名小須
彌山,山中有個道場,乃是菩薩講經禪院。汝等是取他的經去了?」行者道:
「不是取他的經,我有一事煩他,不知從那條路去。」老者用手向南指道:「這
條羊腸路就是了。」哄得那孫大聖回頭看路,那公公化作清風,寂然不見。只是
路傍留下一張簡帖,上有四句頌子云:
    上覆齊天大聖聽:老人乃是李長庚。
    須彌山有飛龍杖,靈吉當年受佛兵。

  
行者執了帖兒,轉身下路。八戒道:「哥呵,我們連日造化低了,這兩日白日裏
見鬼。那個化風去的老兒是誰?」行者把帖兒遞與八戒,念了一遍道:「李長庚
是那個?」行者道:「是西方太白金星的名號。」八戒慌得望空下拜道:「恩人
,恩人,老豬若不虧金星奏准玉帝呵,性命也不知化作甚的了。」行者道:「兄
弟,你卻也知感恩。但莫要出頭,只藏在這樹林深處,仔細看守行李、馬匹。等
老孫尋須彌山,請菩薩去耶。」八戒道:「曉得,曉得,你只管快快前去。老豬
學得個烏龜法,得縮頭時且縮頭。」
  
孫大聖跳在空中,縱觔斗雲,徑往直南上去,果然速快,他點頭經過三千里,扭
腰八百有餘程。須臾,見一座高山,半中間有祥雲出現,瑞藹紛紛。山凹裏果有
一座禪院,只聽得鐘磬悠揚,又見那香煙縹緲。大聖直至門前,見一道人,項掛
數珠,口中念佛。行者道:「道人作揖。」那道人躬身答禮道:「那裏來的老爺
?」行者道:「這可是靈吉菩薩講經處麼?」道人道:「此間正是,有何話說?」
行者道:「累煩你老人家與我傳答傳答:我是東土大唐駕下御弟三藏法師的徒弟
齊天大聖孫悟空行者,今有一事,要見菩薩。」道人笑道:「老爺字多話多,我
不能全記。」行者道:「你只說是唐僧徒弟孫悟空來了。」
  
道人依言,上講堂傳報。那菩薩即穿袈裟,添香迎接。這大聖才舉步入門,往裏
觀看,只見那:
滿堂錦繡,一屋威嚴。眾門人齊誦《法華經》,老班首輕敲金鑄磬。佛前供養,
盡是仙果仙花﹔案上安排,皆是素殽素品。輝煌寶燭,條條金燄射虹霓﹔馥郁真
香,道道玉煙飛彩霧。正是那講罷心閑方入定,白雲片片繞松梢。靜收慧劍魔頭
絕,般若波羅善會高。
  
那菩薩整衣出迓,行者登堂,坐了客位,隨命看茶。行者道:「茶不勞賜,但我
師父在黃風山有難,特請菩薩施大法力降怪救師。」菩薩道:「我受了如來法令
,在此鎮押黃風怪。如來賜了我一顆定風丹、一柄飛龍寶杖。當時被我拿住,饒
了他的性命,放他去隱性歸山,不許傷生造孽。不知他今日欲害令師,有違教令
,我之罪也。」那菩薩欲留行者,治齋相敘,行者懇辭,隨取了飛龍杖,與大聖
一齊駕雲。
  
不多時,至黃風山上。菩薩道:「大聖,這妖怪有些怕我,我只在雲端內住定,
你下去與他索戰,誘他出來,我好施法力。」行者依言,按落雲頭,不容分說,
掣鐵棒把他洞門打破。叫道:「妖怪,還我師父來也!」慌得那把門小妖急忙傳
報。那怪道:「這潑猴著實無禮,再不伏善,反打破我門。這一出去,使陣神風
,定要把他吹死。」仍前披掛,手綽鋼叉,又走出門來。見了行者,更不打話,
撚叉當胸就刺﹔大聖側身躲過。舉棒對面相還戰不數合,那怪吊回頭,望巽地上
,才待要張口呼風,只見那半空裏,靈吉菩薩將飛龍寶杖丟將下來,不知念了些
甚麼咒語,卻是一條八爪金龍,撥喇的掄開兩爪,一把抓住妖精,提著頭,兩三
捽,捽在山石崖邊,現了本相,卻是一個黃毛貂鼠。
  
行者趕上,舉棒就打,被菩薩攔住道:「大聖,莫傷他命我還要帶他去見如來。」
又對行者道:「他本是靈山腳下的得道老鼠,因為偷了琉璃盞內的清油,燈火昏
暗,恐怕金剛拿他,故此走了,卻在此處成精作怪。如來照見了他,不該死罪,
故著我轄押,但他傷生造孽,拿上靈山。今又沖撞大聖,陷害唐僧,我拿他去見
如來,明正其罪,才算這場功績哩。」行者聞言,卻謝了菩薩。菩薩西歸不題。
  
卻說豬八戒在那林內,正思量行者,只聽得山下叫聲:「悟能兄弟,牽馬挑擔來
耶。」那獃子認得是行者聲音,急收拾跑出林外,見了行者道:「哥哥,怎的幹
事來?」行者道:「請靈吉菩薩,使一條飛龍杖,拿住妖精,原來是個黃毛貂鼠
成精,被他帶去靈山見如來去了。我和你洞裏去救師父。」那獃子才歡歡喜喜。
  
二人撞入裏面,把那一窩狡兔、妖狐、香獐、角鹿,一頓釘鈀、鐵棒,盡情打死
,卻往後園拜救師父。師父出得門來,問道:「你兩人怎生捉得妖精?如何方救
得我?」行者將那請靈吉降妖的事情,陳了一遍。師父謝之不盡。他兄弟們把洞
中素物,安排些茶飯吃了,方才出門,找大路向西而去。
  
    畢竟不知向後如何,且聽下回分解。





}  \end{pinyinscope}\switchcolumn{\myfontc \section{第 二 一 回} , 护 法 设 庄 留 大 圣 须 弥 、 灵 吉 定 风 魔 , 又 说 : 那 五 十 个 败 人 , 拿 着 破 旗 、 破 鼓 , 撞 入 洞 中 , 报 告 说 : 大 王 , 虎 先 锋 打 不 过 那 毛 面 和 尚 , 被 他 赶 下 东 山 坡 去 了 。
老 妖 听 了 这 话 , 十 分 恼 怒 。
正 低 头 不 语 , 默 思 计 策 , 又 有 一 个 拿 前 门 的 小 妖 说 : 大 王 , 虎 先 锋 被 那 毛 面 和 尚 打 死 了 , 拖 在 门 口 骂 战 哩 。
那 个 老 妖 人 听 了 这 话 , 更 加 烦 恼 恼 怒 地 说 : 这 个 人 却 没 有 知 道 。
我 倒 不 曾 吃 他 师 父 , 他 转 而 打 杀 我 家 先 锋 , 可 恨 呀 !
吾 闻 之 , 不 知 之 , 不 知 之 , 不 知 之 。
那 些 小 妖 人 急 忙 抬 出 披 挂 。
老 妖 束 缚 整 齐 整 齐 , 一 根 三 股 钢 叉 , 率 领 群 妖 跳 出 本 洞 。
那 大 圣 在 门 外 停 下 来 , 看 见 那 妖 怪 走 了 将 要 出 来 , 显 得 非 常 骁 勇 。
看 他 怎 样 打 扮 , 只 看 见 那 个 人 说 : 金 铠 晃 日 , 金 甲 凝 光 。
盔 上 的 帽 缨 飘 动 着 山 雉 尾 巴 , 绫 罗 袍 袍 罩 着 铠 甲 , 淡 鹅 黄 色 。
刻 甲 盘 龙 耀 彩 , 护 心 镜 绕 眼 辉 煌 。
鹿 皮 靴 , 槐 花 染 色 , 锦 缎 围 裙 , 柳 叶 丝 织 品 。
他 手 里 拿 着 三 股 钢 叉 , 锋 利 , 不 亚 于 当 年 显 圣 郎 。
那 个 老 妖 出 门 来 , 厉 声 大 叫 道 : 那 个 是 孙 行 者 , 他 的 脚 踩 着 虎 怪 的 皮 袋 , 手 拿 着 如 意 的 铁 棒 , 回 答 说 : 你 的 孙 外 公 在 这 里 。
送 出 我 师 父 来 。
那 怪 仔 细 地 观 看 , 发 现 那 个 行 人 身 体 卑 鄙 , 面 容 瘦 弱 , 不 满 四 尺 。
笑 着 说 : 可 怜 , 可 怜 。
我 只 知 道 他 是 什 么 样 , 不 会 倒 的 好 汉 , 原 来 就 是 这 个 骷 髅 的 病 鬼 。
行 人 笑 着 说 : 你 这 个 儿 子 , 太 没 有 脸 色 。
你 的 外 公 虽 然 是 个 小 小 的 , 你 如 果 肯 照 头 打 一 叉 柄 , 就 可 以 长 六 尺 。
那 人 奇 怪 地 说 : 你 硬 着 头 , 吃 我 一 柄 。
大 圣 公 然 不 害 怕 。
那 怪 果 然 打 了 一 下 来 。
其 腰 一 身 , 足 有 六 尺 , 有 一 丈 长 。
忽 然 , 那 妖 人 把 钢 叉 按 住 , 喝 道 : 孙 行 者 , 你 为 什 么 把 这 个 护 身 法 儿 , 拿 到 我 门 前 使 出 来 , 不 要 弄 虚 头 , 跑 上 来 , 我 给 你 看 见 我 的 手 段 。
行 人 笑 着 说 : 儿 子 呵 , 常 说 道 : 留 情 不 举 手 , 举 手 不 留 情 。
你 的 外 公 手 儿 重 重 , 只 怕 你 能 挣 住 不 起 这 一 棒 。
那 怪 那 容 分 别 说 话 , 捻 转 钢 叉 , 望 着 行 人 当 胸 就 刺 , 这 位 大 圣 正 是 会 家 不 忙 , 忙 家 不 会 , 理 开 铁 棒 , 使 一 个 乌 龙 掠 地 势 , 拨 开 钢 叉 , 又 照 着 脑 袋 就 打 。
其 他 两 个 人 在 黄 风 洞 口 , 这 一 场 好 杀 , 妖 王 发 怒 , 大 圣 施 威 。
妖 王 发 怒 , 要 拿 走 的 人 抵 抗 先 锋 , 大 圣 施 威 , 想 抓 住 精 灵 救 助 长 老 。
叉 来 棒 架 , 棒 去 叉 迎 接 。
一 个 是 镇 山 都 统 帅 , 一 个 是 护 法 美 猴 王 。
初 时 还 在 尘 埃 之 中 , 后 来 各 自 起 来 在 中 央 。
点 钢 叉 , 尖 明 锐 利 ; 如 意 棒 , 身 子 黑 色 , 钳 黄 色 。
杀 你 的 魂 归 冥 府 , 打 你 的 魂 就 会 见 阎 王 。
全 靠 手 疾 眼 快 , 必 须 力 壮 身 强 。
两 家 舍 死 忘 生 , 不 知 道 那 个 平 安 , 那 个 伤 。
那 老 妖 与 大 圣 相 斗 经 三 十 回 合 , 不 分 胜 败 。
此 行 者 必 须 看 到 自 己 的 功 绩 , 使 用 一 个 身 外 身 的 手 段 , 把 毫 毛 扎 下 一 把 , 用 嘴 嚼 得 粉 碎 , 望 着 上 面 一 喷 , 叫 道 : 变 有 一 百 十 个 行 人 , 都 是 一 个 打 扮 的 , 各 拿 着 一 根 铁 棒 , 把 那 怪 物 包 在 空 中 。
使 者 急 忙 回 头 , 望 着 在 地 上 , 把 口 张 开 了 三 张 , 呼 得 一 口 气 就 要 吹 出 去 , 忽 然 间 有 一 股 黄 色 的 风 , 从 空 中 刮 起 来 。
好 风 , 真 是 利 害 关 系 。 冷 冷 冷 冷 冷 冷 , 天 地 变 化 , 无 影 无 形 , 黄 沙 旋 转 。
穿 林 折 岭 倒 松 梅 , 播 土 扬 尘 , 山 岭 崩 塌 。
黄 河 波 浪 冲 击 彻 底 浑 , 湘 江 水 涌 翻 波 转 。
碧 天 震 动 斗 牛 宫 , 争 相 刮 倒 森 罗 殿 。
五 百 罗 汉 喧 天 , 八 大 金 刚 齐 喧 闹 。
文 殊 走 了 青 毛 狮 子 , 普 贤 白 象 难 寻 见 。
真 武 王 龟 蛇 失 去 群 , 梓 子 飘 动 了 它 的 。
行 商 喊 叫 告 诉 苍 天 , 梢 公 叩 拜 , 答 应 了 各 种 愿 愿 。
烟 波 性 命 浪 中 流 , 名 利 残 生 随 水 而 已 。
仙 山 洞 府 黑 沉 冥 , 海 岛 蓬 莱 茫 茫 茫 茫 。
老 君 难 以 顾 念 炼 丹 炉 , 寿 星 收 了 龙 须 扇 。
王 母 正 去 参 加 蟠 桃 , 一 阵 风 吹 动 了 裙 子 和 腰 带 的 玉 带 。
二 郎 迷 失 灌 州 城 , 何 吒 难 取 匣 中 剑 。
天 王 不 见 手 心 塔 , 鲁 班 吊 唁 了 金 头 钻 。
雷 音 宝 阙 倒 塌 三 层 , 赵 州 石 桥 崩 塌 两 断 。
一 轮 红 日 飘 荡 无 光 , 满 天 星 斗 都 昏 暗 混 乱 。
南 山 的 鸟 飞 往 北 山 飞 , 东 湖 的 水 向 西 湖 漫 流 。
雌 雄 分 对 不 互 相 呼 唤 , 子 母 分 离 难 叫 唤 。
龙 王 遍 海 寻 找 夜 叉 , 雷 公 到 处 寻 找 闪 电 。
十 代 阎 王 寻 找 判 官 , 地 府 牛 头 追 马 面 。
这 风 吹 倒 了 普 陀 山 , 卷 起 了 一 卷 《 观 音 经 》 。
白 莲 花 卸 海 边 飞 , 吹 倒 菩 萨 十 二 院 。
自 古 至 今 还 曾 见 到 过 风 , 不 像 这 种 风 来 不 好 。
嘈 嘈 , 天 地 险 恶 , 不 会 发 生 崩 裂 , 万 里 江 山 都 会 震 动 。
其 妖 怪 使 者 出 现 了 狂 风 , 就 把 孙 大 圣 的 毫 毛 变 成 的 小 行 人 刮 得 在 半 空 中 , 就 好 像 纺 车 儿 一 样 乱 转 , 莫 想 得 到 棒 子 , 怎 么 能 伸 得 身 子 呢 ?
独 自 举 着 铁 棒 , 上 前 来 打 。
又 被 那 怪 物 劈 开 脸 , 喷 出 一 口 黄 风 , 把 两 只 火 眼 金 睛 刮 得 紧 紧 紧 紧 紧 紧 紧 紧 紧 紧 紧 紧 紧 紧 紧 紧 紧 紧 , 不 能 睁 开 。
因 此 难 以 使 用 铁 棒 , 于 是 败 下 阵 来 。
那 妖 收 风 , 回 洞 不 题 。
又 说 : 猪 八 戒 看 见 那 黄 风 大 作 , 天 地 无 光 , 牵 着 马 , 守 着 担 子 , 伏 在 山 凹 之 间 , 不 敢 抬 眼 , 不 敢 抬 头 , 口 里 不 住 念 佛 , 又 不 知 道 行 人 的 胜 负 如 何 , 师 父 的 生 死 如 何 ?
正 在 那 些 疑 虑 思 念 的 时 候 , 早 晨 风 定 天 晴 。
忽 然 抬 头 去 那 洞 门 前 看 的 地 方 , 却 又 看 不 见 兵 戈 , 听 不 到 鼓 声 。
子 又 不 敢 上 别 的 城 门 , 又 没 有 人 看 守 马 匹 、 行 李 , 果 然 是 进 退 两 难 , 恐 惧 不 已 。
在 忧 虑 之 中 , 只 听 到 孙 大 圣 从 西 边 喊 喝 着 来 , 他 才 拍 着 身 子 迎 着 说 : 哥 哥 , 好 大 风 呵 , 你 从 哪 里 走 来 ? 走 路 的 人 摆 着 手 说 : 利 害 , 利 害 , 我 老 孙 自 己 是 人 , 不 曾 见 过 这 样 大 风 。
那 个 老 妖 使 用 一 柄 三 股 钢 叉 , 来 和 老 孙 子 交 战 。
作 战 有 三 十 余 合 , 是 老 孙 子 使 用 一 个 身 外 身 的 本 事 。
以 此 为 之 , 以 此 为 之 , 以 此 为 之 。
果 然 是 凶 恶 , 刮 得 我 站 立 不 住 , 收 回 了 本 来 的 事 , 冒 着 风 逃 走 。
戏 , 喜 好 风 , 喜 好 风 , 老 孙 子 也 会 呼 风 , 也 会 呼 雨 , 不 曾 像 这 妖 精 的 风 恶 。
八 戒 说 : 师 兄 , 那 妖 精 的 武 艺 怎 么 样 ? 行 者 说 : 也 看 得 过 , 叉 法 儿 倒 也 整 齐 整 齐 , 与 老 孙 子 也 战 斗 得 手 平 。
但 是 风 恶 了 , 难 得 胜 他 。
八 戒 说 : 像 这 样 , 怎 么 能 救 得 到 师 父 呢 ? 行 者 说 : 救 师 父 , 暂 且 等 再 去 吧 。
不 知 道 这 里 可 有 眼 医 先 生 , 暂 且 教 他 用 我 的 眼 医 治 。
八 戒 说 : 你 的 眼 睛 怎 么 回 来 ? 行 者 说 : 我 被 那 怪 一 口 风 吹 来 , 吹 得 我 的 眼 珠 酸 痛 , 这 时 你 的 眼 泪 常 常 流 下 来 。
八 戒 说 : 哥 哥 呵 , 这 半 山 中 , 天 色 又 晚 , 况 且 不 说 要 什 么 眼 睛 , 连 宿 的 地 方 也 没 有 了 。
行 人 说 : 要 住 宿 的 地 方 不 难 , 我 估 计 着 那 妖 精 还 不 敢 伤 害 我 师 父 , 我 们 暂 且 找 上 大 路 , 找 个 人 家 住 下 , 过 此 一 夜 , 明 日 天 光 , 再 来 降 妖 罢 了 。
八 戒 说 : 正 是 , 正 是 。
他 又 牵 着 马 , 挑 了 担 子 , 走 出 山 凹 , 走 上 路 口 。
此 时 渐 渐 黄 昏 , 只 听 到 路 南 山 坡 下 有 狗 叫 的 声 音 。
二 人 停 下 来 观 看 , 原 来 是 一 家 庄 院 , 影 子 闪 闪 有 灯 火 光 明 。
其 他 两 个 人 , 不 管 有 路 无 路 , 漫 草 而 行 , 直 到 那 家 门 口 。
只 见 紫 芝 树 荫 蔽 , 白 石 头 苍 苍 。
紫 芝 树 密 蔽 , 多 青 草 , 白 石 苍 苍 半 绿 苔 。
几 点 点 点 点 点 小 的 萤 光 照 耀 , 一 片 树 林 间 的 野 树 密 密 排 列 。
香 兰 馥 郁 , 嫩 竹 新 栽 。
清 泉 流 过 曲 涧 , 古 柏 倚 在 深 崖 。
地 方 偏 僻 , 更 没 有 游 客 来 到 , 门 前 只 有 野 花 开 。
其 他 两 个 人 不 敢 擅 自 进 去 , 只 得 叫 一 声 : 开 门 , 开 门 。 那 里 有 一 个 老 年 人 , 带 着 几 个 年 幼 的 农 夫 , 叉 着 硬 棍 和 扫 帚 一 齐 来 , 问 道 : 什 么 人 , 什 么 人 ?
因 而 去 西 方 拜 佛 求 经 , 路 过 此 山 , 被 黄 风 大 王 捉 了 我 师 父 去 了 , 我 们 还 没 有 救 过 。
天 色 已 晚 , 我 特 来 府 上 请 求 借 用 一 夜 , 万 望 方 便 。
那 个 老 人 回 礼 说 : 不 迎 , 失 迎 。
这 里 是 云 多 人 少 的 地 方 , 却 才 听 到 叫 门 的 声 音 , 恐 怕 是 妖 狐 、 老 虎 以 及 山 中 强 盗 等 类 , 所 以 这 些 小 介 愚 顽 , 多 有 冲 撞 , 不 知 是 两 位 长 老 。
请 进 , 请 进 。
其 兄 弟 牵 着 马 挑 着 担 子 进 去 , 径 直 走 到 里 边 , 把 马 拴 在 担 子 上 , 与 庄 老 拜 见 并 叙 坐 。
又 有 苍 头 进 献 茶 叶 。
茶 罢 , 捧 出 几 碗 胡 麻 饭 。
吃 完 饭 , 命 令 在 铺 上 就 寝 。
行 者 说 : 不 睡 还 可 以 , 请 问 善 人 , 贵 地 能 有 卖 眼 药 的 老 人 说 : 是 那 位 长 老 害 眼 吗 ? 行 者 说 : 不 瞒 你 们 老 人 家 说 , 我 们 出 家 人 自 来 没 有 病 , 从 不 知 道 有 害 眼 。
老 人 说 : 既 然 不 害 眼 , 又 何 必 去 讨 药 ? 行 人 说 : 我 们 今 天 在 黄 风 洞 口 救 我 师 父 , 不 料 被 那 怪 将 一 口 风 喷 来 , 吹 得 我 的 眼 珠 酸 痛 , 现 在 有 些 眼 泪 汪 汪 , 所 以 我 要 寻 找 眼 药 。
那 个 老 人 说 : 好 啊 , 好 啊 , 你 这 个 老 人 , 年 纪 小 , 怎 么 说 谎 ? 那 黄 风 大 王 , 风 最 好 。
其 他 风 , 比 不 上 什 么 春 秋 风 、 松 竹 风 和 那 东 西 南 北 风 。
八 戒 说 : 想 必 是 夹 脑 风 、 羊 耳 风 、 大 麻 风 、 偏 正 头 风 , 长 者 说 : 不 是 , 不 是 。
其 所 谓 三 昧 神 风 。
老 人 说 : 那 风 能 吹 天 地 昏 暗 , 善 刮 鬼 神 愁 愁 , 裂 石 崩 崖 恶 , 吹 人 命 就 可 以 停 止 。
你 们 如 果 遇 到 他 们 , 那 风 吹 了 时 , 还 想 得 到 活 吗 ? 只 要 是 神 仙 , 才 可 以 无 事 。
行 人 说 : 果 然 , 果 然 。
我 们 虽 然 不 是 神 仙 , 但 神 仙 还 是 我 的 晚 辈 。
这 个 性 命 急 切 难 以 停 止 , 却 只 是 吹 得 我 的 眼 珠 疼 痛 。
那 个 老 人 说 : 既 然 这 样 说 , 也 是 个 有 头 的 人 。
我 这 个 敝 处 , 却 没 有 卖 眼 药 的 。
有 一 次 遇 到 奇 异 的 人 , 传 了 一 个 药 方 , 名 叫 三 花 九 子 膏 , 能 治 疗 一 切 病 。
行 人 听 到 这 话 , 低 头 大 喊 道 : 请 求 一 点 儿 子 , 点 试 一 试 。
那 个 老 人 答 应 了 , 立 即 走 进 去 , 取 出 一 个 玛 瑙 石 的 小 罐 子 , 拔 出 来 塞 口 , 用 玉 簪 儿 蘸 出 一 点 , 给 行 人 点 上 , 让 他 不 得 睁 开 , 宁 愿 睡 觉 , 明 早 就 好 了 。
点 完 , 收 了 石 罐 , 径 直 领 着 小 介 等 人 退 到 里 面 。
八 戒 解 开 包 袱 , 打 开 铺 盖 , 请 行 者 安 置 。
行 人 闭 着 眼 睛 乱 摸 。
八 戒 笑 着 说 : 先 生 , 你 的 明 杖 儿 吗 ? 行 者 说 : 你 这 个 糟 的 子 , 你 照 顾 我 做 瞎 子 哩 。
那 子 默 默 不 语 , 暗 笑 着 睡 觉 。
行 者 坐 在 铺 上 , 转 运 神 功 , 直 到 三 更 之 后 才 睡 下 。
不 知 不 觉 又 是 五 更 将 要 亮 的 时 候 。
行 者 抹 抹 脸 , 睁 开 眼 睛 说 : 果 然 是 好 药 , 比 平 常 更 有 百 分 光 明 。
又 转 过 头 后 边 去 望 , 望 见 , 呀 呀 呀 , 那 里 有 些 房 屋 和 门 户 , 只 见 到 一 棵 老 槐 、 高 柳 , 兄 弟 们 都 睡 在 绿 莎 草 茵 上 。
那 八 戒 醒 来 后 说 : 哥 哥 , 你 吵 什 么 ? 行 者 说 : 你 睁 开 眼 睛 看 看 。
子 忽 然 抬 头 , 看 见 没 了 人 家 , 慌 忙 得 到 一 只 袋 子 爬 起 来 , 说 : 我 的 马 啊 , 行 人 说 : 树 上 拴 的 不 是 , 行 李 吗 ? 行 人 说 : 你 头 边 放 的 不 是 , 八 戒 说 : 这 家 子 也 疲 惫 懒 惰 , 他 搬 了 , 为 什 么 不 叫 我 们 一 声 , 通 得 老 猪 知 道 , 也 好 给 你 送 些 茶 果 。
想 是 避 开 门 户 , 恐 怕 里 长 知 道 , 就 连 夜 搬 走 了 。
唉 , 我 们 也 差 得 睡 得 死 , 为 什 么 他 家 拆 房 子 , 响 也 听 不 见 响 响 , 走 路 的 人 吸 吸 地 笑 着 说 : 子 , 不 要 乱 吵 。
你 看 那 树 上 是 什 么 纸 帖 子 , 八 戒 走 上 前 , 用 手 揭 了 , 原 来 上 面 四 句 颂 子 说 : 庄 居 不 是 俗 人 居 , 护 法 伽 蓝 点 化 庐 。
妙 药 给 你 医 治 眼 痛 , 尽 心 降 下 怪 物 , 不 要 再 犹 豫 不 决 。
行 人 说 : 这 伙 强 神 , 自 己 换 了 龙 马 , 一 向 不 曾 点 他 , 他 倒 又 来 弄 虚 头 。
八 戒 说 : 哥 哥 不 要 扯 架 子 , 他 为 什 么 服 你 点 笔 ? 行 者 说 : 兄 弟 , 你 还 不 知 道 吗 ?
六 丁 六 甲 、 五 方 揭 谛 、 四 值 功 曹 奉 承 菩 萨 的 旨 意 , 暗 中 保 护 我 师 父 。
自 从 那 天 报 了 名 字 , 只 是 因 为 这 一 向 有 了 你 , 再 也 不 曾 任 用 他 们 , 所 以 不 曾 点 笔 罢 了 。
八 戒 说 : 哥 哥 , 他 既 奉 法 旨 , 暗 地 保 护 师 父 , 所 以 不 能 显 现 出 身 , 所 以 这 个 点 化 仙 庄 。
你 不 要 怪 他 , 昨 天 也 亏 他 给 你 点 眼 , 又 亏 他 管 我 们 一 顿 斋 饭 , 也 可 以 说 是 尽 心 尽 力 了 。
你 不 要 怪 他 , 我 们 暂 且 去 救 师 父 来 。
行 人 说 : 兄 弟 说 得 对 。
此 处 到 那 黄 风 洞 口 不 远 , 你 暂 且 不 要 动 身 , 只 在 林 子 里 看 马 守 担 。
等 老 子 去 洞 里 打 听 , 看 看 老 师 的 下 落 如 何 , 再 与 他 争 斗 。
八 戒 说 : 正 是 这 样 , 讨 伐 一 个 生 死 的 确 实 确 实 相 信 。
倘 若 师 父 死 了 , 各 人 都 喜 欢 寻 找 头 来 做 事 ; 如 果 还 没 有 死 , 我 们 就 要 尽 力 尽 心 。
行 人 说 : 不 要 乱 谈 , 我 去 吧 。
他 将 身 体 放 了 一 下 , 径 直 走 到 他 的 门 口 , 门 还 关 着 睡 觉 。
走 路 的 人 不 叫 门 , 也 不 惊 动 妖 怪 , 捻 着 诀 语 , 念 咒 语 , 动 身 一 变 , 变 成 一 个 花 脚 蚊 虫 , 真 是 小 巧 。
有 诗 作 证 。
《 诗 经 》 上 说 : 纷 纷 细 小 的 形 状 , 利 于 嘴 巴 , 声 音 细 细 如 雷 。
兰 房 的 纱 帐 , 善 于 随 从 , 正 喜 爱 炎 天 暖 气 。
只 怕 薰 烟 扑 扇 , 偏 怜 灯 火 光 辉 。
轻 轻 小 小 的 小 虫 , 巧 妙 钻 得 快 , 飞 进 妖 精 洞 里 。
只 见 那 把 门 的 小 妖 正 在 打 瞌 睡 , 走 路 的 人 在 他 的 脸 上 咬 了 一 口 , 那 个 小 妖 翻 身 醒 了 , 说 : 我 的 爹 爹 , 好 大 蚊 子 , 一 口 就 咬 了 一 个 大 琥 珀 。
忽 然 睁 开 眼 睛 说 : 天 亮 了 。
又 听 到 支 开 的 一 声 , 两 个 门 就 打 开 了 。
走 的 人 急 忙 飞 上 去 , 只 见 那 个 老 妖 人 在 门 口 叮 嘱 , 在 一 个 墙 上 收 拾 兵 器 , 说 : 只 怕 昨 天 的 风 不 曾 刮 死 孙 行 者 , 他 今 天 必 定 还 来 , 来 时 一 定 要 教 他 一 命 休 息 。
行 人 听 了 , 又 飞 过 那 厅 堂 , 径 直 来 到 后 面 , 只 见 一 层 门 关 得 很 紧 。
走 路 的 人 漫 在 门 缝 里 钻 进 去 , 原 来 是 个 大 空 园 子 , 那 墙 壁 边 的 定 风 桩 上 的 绳 子 缠 着 绳 索 绑 着 唐 僧 哩 。
师 父 涕 泪 流 满 地 , 心 里 只 念 念 着 悟 空 、 悟 能 , 不 知 道 都 在 什 么 地 方 。
行 人 停 下 翅 膀 , 在 他 光 头 上 咬 着 , 叫 道 : 师 父 。
那 个 长 老 认 得 他 的 声 音 , 说 道 : 我 想 杀 我 。
你 在 哪 里 叫 我 吗 ? 行 人 说 : 师 父 , 我 在 你 的 头 上 吗 ?
卿 莫 要 心 焦 , 少 得 烦 恼 。
吾 闻 之 , 必 得 之 。
唐 僧 说 : 徒 弟 呵 , 几 时 才 捉 得 妖 精 吗 ? 行 者 说 : 拿 你 的 那 虎 怪 , 已 经 被 八 戒 打 死 了 。
只 是 老 妖 的 风 势 利 害 , 估 计 只 在 今 天 , 管 取 拿 它 。
你 放 心 不 哭 , 我 去 掉 哑 巴 。
声 音 之 后 , 声 音 之 外 , 声 音 之 外 , 声 音 之 外 , 声 音 之 外 , 声 音 之 外 , 声 音 之 外 。
只 见 那 个 老 妖 人 坐 在 上 面 , 正 在 点 笔 各 路 的 头 目 。
又 见 那 洞 前 有 一 个 小 妖 精 , 拿 着 令 字 旗 磨 一 磨 , 撞 上 厅 来 报 告 说 : 大 王 , 小 的 巡 视 山 门 , 刚 出 门 , 就 见 一 个 长 嘴 大 耳 朵 的 和 尚 坐 在 树 林 里 , 如 果 不 是 我 跑 得 快 些 , 几 乎 被 他 捉 住 。
又 不 见 昨 天 那 个 毛 面 和 尚 。
老 妖 说 : 孙 行 者 不 在 , 想 必 是 被 风 吹 死 了 , 再 不 就 去 那 里 找 救 兵 去 了 。
众 妖 说 : 大 王 , 如 果 吹 杀 了 他 , 是 我 们 的 造 化 , 只 怕 吹 不 死 他 , 他 去 请 求 神 兵 来 , 又 怎 么 能 说 好 呢 ? 老 妖 说 : 怕 那 么 什 么 神 兵 , 如 果 还 能 得 到 我 的 风 势 , 只 除 了 灵 吉 菩 萨 来 , 其 余 什 么 值 得 害 怕 呢 ? 行 人 在 屋 梁 上 , 只 听 到 他 的 这 一 句 话 , 不 胜 欢 喜 。
立 刻 抽 出 身 子 飞 出 来 , 显 现 自 己 的 相 貌 , 来 到 树 林 中 , 大 叫 道 : 兄 弟 。
八 戒 说 : 哥 , 你 到 哪 里 去 来 , 刚 才 一 个 打 令 字 旗 的 妖 精 , 被 我 赶 去 。
行 人 笑 着 说 : 亏 你 , 亏 你 。
老 孙 变 成 蚊 虫 儿 , 进 入 其 洞 去 探 看 师 父 , 原 来 师 父 被 他 绑 在 定 风 桩 上 哭 啊 。
是 老 子 的 吩 咐 , 教 他 不 要 哭 。
又 飞 到 屋 梁 上 , 听 了 一 声 , 只 见 那 个 拿 着 令 字 旗 的 人 , 声 音 响 亮 地 走 进 去 报 告 说 : 只 是 被 你 赶 了 , 却 又 看 不 见 我 。
老 妖 乱 猜 乱 说 , 说 老 孙 是 被 风 吹 死 了 , 又 说 是 请 神 兵 去 了 。
又 从 自 己 家 中 供 出 一 个 人 。
八 戒 说 : 他 供 的 是 谁 ? 行 者 说 : 他 说 怕 什 么 神 兵 , 那 个 能 定 他 的 风 势 , 只 除 是 灵 吉 菩 萨 来 了 。
但 不 知 道 灵 吉 住 在 什 么 地 方 , 正 在 商 议 的 地 方 , 只 见 大 路 旁 边 走 出 一 个 老 公 公 。
你 看 他 怎 么 样 子 说 : 身 体 健 壮 , 不 拄 拐 杖 , 头 发 雪 白 , 鬓 发 蓬 蓬 。
金 花 耀 眼 , 意 思 恍 惚 , 瘦 骨 衰 筋 强 劲 。
屈 背 低 头 慢 步 , 眉 毛 赤 脸 如 童 童 。
看 他 的 容 貌 是 人 称 , 却 好 像 寿 星 出 洞 。
八 戒 望 见 后 非 常 高 兴 地 说 : 师 兄 , 常 常 说 道 : 要 知 道 山 下 的 路 , 须 问 去 来 的 人 。
你 上 前 问 他 一 声 , 怎 么 样 ? 真 个 大 圣 藏 了 铁 棒 , 放 下 衣 襟 , 上 前 大 叫 道 : 老 公 公 , 问 讯 了 。
那 个 老 人 半 答 不 答 , 回 礼 说 : 你 是 那 里 的 和 尚 , 这 个 旷 野 之 地 , 有 什 么 事 干 ? 行 者 说 : 我 们 是 取 经 的 圣 僧 。
昨 天 我 在 这 里 失 去 了 师 父 , 特 来 问 你 , 你 一 声 : 灵 吉 菩 萨 在 哪 里 住 ? 老 人 说 : 灵 吉 在 直 往 南 上 , 到 了 那 里 还 有 三 千 里 路 。
有 一 座 山 , 叫 小 须 弥 山 , 山 中 有 个 道 场 , 是 菩 萨 讲 经 的 禅 院 。
你 们 是 取 他 的 经 去 了 , 行 人 说 : 不 是 取 他 的 经 , 我 有 一 件 事 要 麻 烦 他 , 不 知 道 从 哪 条 路 去 。
老 人 用 手 向 南 指 着 道 : 这 条 羊 肠 路 就 是 了 。
大 圣 回 头 看 路 , 那 公 公 变 成 了 清 风 , 寂 然 不 见 了 。
只 是 路 旁 留 下 一 张 简 帖 , 上 面 有 四 句 颂 子 : 上 覆 齐 天 大 圣 听 , 老 人 乃 是 李 长 庚 。
须 弥 山 有 一 支 飞 龙 杖 , 灵 吉 当 年 曾 受 佛 教 的 军 队 。
行 人 拿 了 帖 子 , 转 身 下 路 。
八 戒 说 : 哥 呵 , 我 们 连 日 造 化 低 了 , 这 两 天 白 天 里 见 到 鬼 。
行 者 把 帖 子 递 给 八 戒 , 念 一 遍 说 : 李 长 庚 是 那 个 ? 行 者 说 : 是 西 方 太 白 金 星 的 名 号 。
八 戒 惊 慌 得 望 空 下 拜 说 : 恩 人 , 恩 人 , 老 猪 如 果 不 亏 金 星 奏 准 玉 帝 呵 , 我 的 性 命 也 不 知 道 变 成 了 什 么 了 。
行 人 说 : 兄 弟 , 你 却 知 道 感 恩 。
但 不 要 出 头 , 只 要 藏 在 这 树 林 深 处 , 仔 细 看 看 行 李 、 马 匹 。
等 老 孙 去 寻 找 须 弥 山 , 请 菩 萨 去 吗 ?
八 戒 说 : 晓 得 , 晓 得 , 你 只 管 快 快 往 前 去 。
老 猪 学 得 一 个 乌 龟 法 , 得 到 缩 头 时 暂 且 缩 头 。
孙 大 圣 跳 在 空 中 , 纵 手 搏 斗 , 径 直 往 直 往 南 上 去 , 果 然 快 快 , 他 点 头 走 了 三 千 里 , 腰 弯 了 八 百 多 里 。
不 一 会 儿 , 看 见 一 座 高 山 , 半 腰 中 间 有 祥 云 出 现 , 祥 瑞 和 祥 瑞 和 祥 瑞 纷 纷 。
山 坳 里 果 然 有 一 座 禅 院 , 只 听 到 钟 磬 悠 扬 , 又 见 到 那 香 烟 袅 袅 。
大 圣 一 直 来 到 门 前 , 看 见 一 个 道 人 , 脖 子 上 挂 着 几 颗 珠 子 , 口 里 念 佛 。
行 人 说 : 道 人 作 揖 。
那 个 道 人 亲 身 回 礼 说 : 从 哪 里 来 的 老 爷 ? 行 者 说 : 这 里 是 灵 吉 菩 萨 讲 经 的 地 方 吗 ? 道 人 说 : 这 里 正 是 , 有 什 么 话 说 呢 ? 行 者 说 : 烦 恼 你 老 人 家 给 我 传 答 , 回 答 说 : 我 是 东 土 大 唐 驾 下 御 弟 三 藏 法 师 的 弟 弟 齐 天 大 圣 孙 悟 空 行 , 现 在 有 一 件 事 , 要 见 菩 萨 。
道 人 笑 着 说 : 老 爷 字 多 , 话 多 , 我 不 能 全 记 。
行 人 说 : 你 只 说 是 唐 僧 的 弟 弟 孙 悟 空 来 了 。
道 人 依 照 他 的 话 , 上 讲 堂 传 报 。
那 菩 萨 就 穿 上 袈 裟 , 添 香 迎 接 。
这 个 大 圣 才 举 步 进 门 , 到 里 面 去 观 看 , 只 见 那 里 的 满 堂 锦 绣 , 一 屋 的 威 严 。
众 门 人 一 齐 诵 读 《 法 华 经 》 , 老 班 首 轻 轻 敲 金 磬 。
佛 前 供 奉 的 , 都 是 仙 果 仙 花 ; 案 上 安 排 的 , 都 是 素 素 品 。
辉 煌 的 宝 烛 , 条 条 金 线 闪 射 虹 霓 ; 馥 郁 的 真 香 , 道 路 玉 烟 飞 扬 彩 雾 。
正 是 那 讲 罢 心 闲 才 入 定 , 白 云 片 片 绕 松 梢 。
静 收 慧 剑 魔 头 绝 , 般 若 波 罗 善 会 高 。
那 菩 萨 整 理 衣 服 出 门 , 行 人 上 堂 , 坐 在 客 位 上 , 随 即 命 人 看 茶 。
行 者 说 : 茶 不 用 赐 给 , 但 我 师 父 在 黄 风 山 有 难 , 特 请 菩 萨 施 大 法 力 , 降 魔 救 法 师 。
菩 萨 说 : 我 接 受 了 如 来 法 令 , 在 这 里 镇 守 黄 风 怪 。
如 来 赐 给 我 一 颗 定 风 丹 、 一 柄 飞 龙 宝 杖 。
当 时 被 我 捉 住 , 饶 了 他 的 性 命 , 放 他 去 隐 藏 自 己 的 本 性 归 山 , 不 许 伤 生 造 孽 。
不 知 他 今 天 想 杀 害 令 师 , 违 背 教 令 , 是 我 的 罪 过 。
那 菩 萨 想 留 下 行 人 , 整 理 斋 戒 , 行 人 恳 切 推 辞 , 随 即 取 了 飞 龙 杖 , 与 大 圣 一 齐 驾 车 。
不 多 时 , 来 到 黄 风 山 上 。
菩 萨 说 : 大 圣 , 这 妖 怪 有 些 害 怕 我 , 我 只 在 云 端 里 住 住 , 你 下 去 和 他 交 战 , 引 诱 他 出 来 , 我 好 施 舍 法 力 。
走 路 的 人 按 照 他 的 话 , 把 他 打 落 在 云 头 上 , 不 能 分 说 , 拿 着 铁 棒 把 他 的 洞 口 打 破 。
又 叫 道 : 妖 怪 , 还 我 师 父 来 了 , 惊 得 那 把 门 的 小 妖 , 急 忙 传 来 报 告 。
那 人 奇 怪 地 说 : 这 个 泼 猴 , 实 在 无 礼 , 再 不 服 善 , 反 而 打 破 我 的 门 。
其 一 出 , 使 得 神 风 , 一 定 要 把 它 吹 死 。
又 往 前 披 挂 , 手 叉 着 铁 叉 , 又 走 出 门 来 。
见 了 行 人 , 更 不 打 话 , 把 叉 叉 当 胸 就 刺 , 大 圣 侧 身 避 过 。
举 起 棍 子 对 面 回 来 , 还 没 有 好 几 个 回 合 , 那 怪 怪 地 回 头 , 望 到 地 上 , 才 等 到 要 张 口 呼 风 , 只 见 那 半 空 中 , 灵 吉 菩 萨 将 飞 龙 宝 杖 扔 下 来 , 不 知 道 念 了 什 么 咒 语 , 却 是 一 条 八 爪 金 龙 , 拨 开 两 爪 , 一 把 捉 住 妖 精 , 提 着 头 , 两 三 个 子 , 在 山 石 崖 边 , 现 出 本 来 的 相 貌 , 却 是 一 个 黄 毛 貂 鼠 。
行 者 赶 上 去 , 举 棒 就 打 , 被 菩 萨 拦 住 说 : 大 圣 , 不 要 伤 害 他 的 性 命 , 我 还 要 带 他 去 见 如 来 。
又 对 行 人 说 : 他 本 来 是 灵 山 脚 下 的 得 道 老 鼠 , 因 为 偷 了 琉 璃 盏 里 的 清 油 , 灯 火 昏 暗 , 恐 怕 金 刚 抓 了 它 , 所 以 才 逃 走 了 , 却 在 这 里 成 精 作 怪 。
如 来 照 见 了 他 , 不 该 死 罪 , 所 以 把 我 押 送 , 只 是 他 伤 生 造 孽 , 把 他 拿 到 灵 山 去 。
现 在 又 冲 撞 大 圣 , 陷 害 唐 僧 , 我 拿 他 去 见 如 来 , 明 确 纠 正 他 的 罪 行 , 才 算 是 这 场 功 绩 吗 ?
行 者 听 了 这 话 , 就 向 菩 萨 道 歉 。
菩 萨 向 西 回 去 , 不 必 问 题 。
又 说 猪 八 戒 在 那 林 里 , 正 在 想 着 行 走 的 人 , 只 听 见 山 下 的 叫 声 说 : 悟 能 兄 弟 , 牵 着 马 挑 着 担 子 来 吗 ?
那 个 子 认 得 是 行 人 的 声 音 , 急 忙 把 它 收 拾 出 来 , 走 出 林 外 , 见 到 行 人 说 : 哥 哥 , 何 以 干 事 来 ? 行 人 说 : 请 灵 吉 菩 萨 , 让 我 一 支 飞 龙 杖 , 拿 住 妖 精 , 原 来 是 个 黄 毛 貂 鼠 成 精 , 被 他 带 去 灵 山 见 到 如 来 去 了 。
我 和 你 洞 里 去 救 师 父 。
那 子 才 喜 欢 喜 喜 。
两 个 人 在 里 面 撞 入 , 把 那 一 窝 狡 兔 、 妖 狐 、 角 鹿 , 一 顿 钉 硬 硬 的 钢 子 、 铁 棒 , 全 都 打 死 , 然 后 又 往 后 园 去 拜 救 师 父 。
师 父 出 门 来 , 问 道 : 你 们 两 个 人 为 什 么 捉 得 妖 精 , 为 什 么 方 才 救 我 ?
师 父 谢 罪 不 尽 。
其 兄 弟 之 中 , 安 排 茶 饭 吃 , 才 出 门 , 找 大 路 向 西 走 。
最 终 不 知 道 以 后 怎 么 样 , 暂 且 听 下 回 分 解 。
}\switchcolumn\flushpage  \begin{pinyinscope}{\myfontt \section{第二二回}     八戒大戰流沙河 木叉奉法收悟淨

話說唐僧師徒三眾脫難前來,不一日行過了黃風嶺,進西卻是一脈平陽之地。光
陰迅速,歷夏經秋,見了些寒蟬鳴敗柳,大火向西流。正行處,只見一道大水狂
瀾,渾波湧浪。三藏在馬上忙呼道:「徒弟,你看那前邊水勢寬闊,怎不見船隻
行走,我們從那裏過去?」八戒見了道:「果是狂瀾,無舟可渡。」那行者跳在
空中,用手搭涼篷而看,他也心驚道:「師父呵,真個是難,真個是難。這條河
若論老孫去呵,只消把腰兒扭一扭,就過去了﹔若師父,誠千分難渡,萬載難行
。」三藏道:「我這裏一望無邊,端的有多少寬闊?」行者道:「經過有八百里
遠近。」八戒道:「哥哥怎的定得個遠近之數?」行者道:「不瞞賢弟說,老孫
這雙眼,白日裏常看得千里路上的吉凶。卻才在空中看出,此河上下不知多遠,
但只見這經過足有八百里。」長老憂嗟煩惱,兜回馬,忽見岸上有一通石碑。三
眾齊來看時,見上有三個篆字,乃「流沙河」﹔腹上有小小的四行真字云:
    八百流沙界,三千弱水深。
    鵝毛飄不起,蘆花定底沉。

  
師徒們正看碑文,只聽得那浪湧如山,波翻若嶺,河當中滑辣的鑽出一個妖精,
十分兇醜:
    一頭紅燄髮蓬鬆,兩隻圓睛亮似燈。
    不黑不青藍靛臉,如雷如鼓老龍聲。
    身披一領鵝黃氅,腰束雙攢露白藤。
    項下骷髏懸九個,手持寶杖甚崢嶸。

  
那怪一個旋風,奔上岸來,徑搶唐僧。慌得行者把師父抱住,急登高岸,回身走
脫。那八戒放下擔子,掣出釘鈀﹔望妖精便築。那怪使寶杖架住。他兩個在流沙
河岸,各逞英雄。這一場好鬥:
九齒鈀,降妖杖,二人相敵河岸上。這個是總督大天蓬,那個是謫下捲簾將。昔
年曾會在靈霄,今日爭持賭猛壯。這一個鈀去探爪龍,那一個杖架磨牙象。伸開
大四平,鑽入迎風戧。這個沒頭沒臉抓,那個無亂無空放。一個是久占流沙界吃
人精,一個是秉教迦持修行將。
  
他兩個來來往往,戰經二十回合,不分勝負。
  
那大聖護了唐僧,牽著馬,守定行李。見八戒與那怪交戰,就恨得咬牙切齒,擦
掌磨拳,忍不住要去打他,掣出棒來道:「師父,你坐著,莫怕。等老孫和他耍
耍兒來。」那師父苦留不住。他打個唿哨,跳到前邊。原來那怪與八戒正戰到好
處,難解難分。被行者掄起鐵棒,望那怪著頭一下,那怪急轉身,慌忙躲過,徑
鑽入流沙河裏。氣得個八戒亂跳道:「哥呵,誰著你來的?那怪漸漸手慢,難架
我鈀,再不上三五合,我就擒住他了。他見你兇險,敗陣而逃,怎生是好?」行
者笑道:「兄弟,實不瞞你說,自從降了黃風怪,下山來,這個把月不曾耍棍,
我見你和他戰的甜美,我就忍不住腳癢,故就跳將來耍耍的。那知那怪不識耍,
就走了。」
  
他兩個攙著手,說說笑笑,轉回見了唐僧。唐僧道:「可曾捉得妖怪?」行者道
:「那妖怪不奈戰,敗回鑽入水去也。」三藏道:「徒弟,這怪久住于此,他知
道淺深。似這般無邊的弱水,又沒了舟楫,須是得個知水性的引領引領才好哩。」
行者道:「正是這等說。常言道:『近硃者赤,近墨者黑。』那怪在此,斷知水
性。我們如今拿住他,且不要打殺,只教他送師父過河,再做理會。」八戒道:
「哥哥不必遲疑,讓你先去拿他,等老豬看守師父。」行者笑道:「賢弟呀,這
樁兒我不敢說嘴,水裏勾當,老孫不大十分熟。若是空走,還要捻訣,又念念避
水咒,方才走得﹔不然,就要變化做甚麼魚蝦蟹鱉之類,我才去得。若論賭手段
,憑你在高山雲裏,幹甚麼蹊蹺異樣事兒,老孫都會﹔只是水裏的買賣,有些兒
榔杭。」八戒道:「老豬當年總督天河,掌管了八萬水兵大眾,倒學得知些水性
。卻只怕那水裏有甚麼眷族老小,七窩八代的都來,我就弄他不過,一時不被他
撈去耶?」行者道:「你若到他水中與他交戰,卻不要戀戰,許敗不許勝,把他
引將出來,等老孫下手助你。」八戒道:「言得是,我去耶。」說聲去,就剝了
青錦直裰,脫了鞋,雙手舞鈀,分開水路,使出那當年舊手段,躍浪翻波,撞將
進去,徑至水底之下,往前正走。
  
卻說那怪敗了陣回,方才喘定,又聽得有人推得水響。忽起身觀看,原來是八戒
執了鈀推水。那怪舉杖當面高呼道:「那和尚,那裏走?仔細看打。」八戒使鈀
架住道:「你是個甚麼妖精,敢在此間擋路?」那妖道:「你是也不認得我。我
不是那妖魔鬼怪,也不是少姓無名。」八戒道:「你既不是妖魔鬼怪,卻怎生在
此傷生?你端的甚麼姓名,實實說來,我饒你性命。」那怪道:「我:
    自小生來神氣壯,乾坤萬里曾遊蕩。
    英雄天下顯威名,豪傑人家做模樣。
    萬國九州任我行,五湖四海從吾撞。
    皆因學道蕩天涯,只為尋師遊地曠。
    常年衣缽謹隨身,每日心神不可放。
    沿地雲遊數十遭,到處閑行百餘趟。
    因此才得遇真人,引開大道金光亮。
    先將嬰兒姹女收,後把木母金公放。
    明堂腎水入華池,重樓肝火投心臟。
    三千功滿拜天顏,志心朝禮明華向。
    玉皇大帝便加陞,親口封為捲簾將。
    南天門裏我為尊,靈霄殿前吾稱上。
    腰間懸掛虎頭牌,手中執定降妖杖。
    頭頂金盔晃日光,身披鎧甲明霞亮。
    往來護駕我當先,出入隨朝予在上。
    只因王母降蟠桃,設宴瑤池邀眾將。
    失手打破玉玻璃,天神個個魂飛喪。
    玉皇即便怒生嗔,卻令掌朝左輔相:
    卸冠脫甲摘官銜,將身推在殺場上。
    多虧赤腳大天仙,越班啟奏將吾放。
    饒死回生不點刑,遭貶流沙東岸上。
    飽時困臥此山中,餓去翻波尋食餉。
    樵子逢吾命不存,漁翁見我身皆喪。
    來來往往吃人多,翻翻覆覆傷生瘴。
    你敢行兇到我門,今日肚皮有所望。
    莫言粗糙不堪嘗,拿住消停剁鮓醬。」

  
八戒聞言大怒,罵道:「你這潑物!全沒一些兒眼色。我老豬還掐出水沫兒來哩
,你怎敢說我粗糙,要剁鮓醬?看起來,你把我認做個老走硝哩。休得無禮,吃
你祖宗這一鈀。」那怪見鈀來,使一個「鳳點頭」躲過。兩個在水中打出水面,
各人踏浪登波。這一場賭鬥,比前不同,你看那:
捲簾將,天蓬帥,各顯神通真可愛。那個降妖寶杖著頭輪,這個九齒釘鈀隨手快
。躍浪振山川,推波昏世界。兇如太歲撞幢幡,惡似喪門掀寶蓋。這一個赤心凜
凜保唐僧,那一個犯罪滔滔為水怪。鈀抓一下九條痕,杖打之時魂魄敗。努力喜
相持,用心要賭賽。算來只為取經人,怒氣沖天不忍耐。攪得那鯁鮊鯉鱖退鮮鱗
,龜鱉黿鼉傷嫩蓋﹔紅蝦紫蟹命皆亡,水府諸神朝上拜。只聽得波翻浪滾似雷轟
,日月無光天地怪。
  
二人整鬥有兩個時辰,不分勝敗。這才是銅盆逢鐵帚,玉磬對金鐘。
  
卻說那大聖保著唐僧,立於左右,眼巴巴的望著他兩個在水上爭持,只是他不好
動手。只見那八戒虛幌一鈀,佯輸詐敗,轉回頭往東岸上走。那怪隨後趕來,將
近到了岸邊。這行者忍耐不住,撇了師父,掣鐵棒,跳到河邊,望妖精劈頭就打
。那妖物不敢相迎,颼的又鑽入河內。八戒嚷道:「你這弼馬溫,徹是個急猴子
!你再緩緩些兒,等我哄他到了高處,你卻阻住河邊,教他不能回首呵,卻不拿
住他也?他這進去,幾時又肯出來?」行者笑道:「獃子,莫嚷,莫嚷。我們且
回去見師父去來。」
  
八戒卻同行者到高岸上,見了三藏。三藏欠身道:「徒弟辛苦呀。」八戒道:
「且不說辛苦,只是降了妖精,送得你過河,方是萬全之策。」三藏道:「你才
與妖精交戰何如?」八戒道:「那妖的手段,與老豬是個對手。正戰處,使一個
詐敗,他才趕到岸上。見師兄舉著棍子,他就跑了。」三藏道:「如此怎生奈何
?」行者道:「師父放心,且莫焦惱。如今天色又晚,且坐在這崖岸之上,待老
孫去化些齋飯來,你吃了睡去,待明日再處。」八戒道:「說得是,你快去快來。」
  
行者急縱雲跳起去,正到直北下人家化了一缽素齋,回獻師父。師父見他來得甚
快,便叫:「悟空,我們去化齋的人家,求問他一個過河之策,不強似與這怪爭
持?」行者笑道:「這家子遠得狠哩,相去有五七千里之路,他那裏得知水性?
問他何益?」八戒道:「哥哥又來扯謊了,五七千里路,你怎麼這等去來得快?」
行者道:「你那裏曉得,老孫的觔斗雲,一縱有十萬八千里。像這五七千路,只
消把頭點上兩點,把腰躬上一躬,就是個往回,有何難哉?」八戒道:「哥呵,
既是這般容易,你把師父背著,只消點點頭,躬躬腰,跳過去罷了,何必苦苦的
與這怪廝戰?」行者道:「你不會駕雲?你把師父馱過去不是?」八戒道:「師
父的凡胎肉骨,重似泰山,我這駕雲的,怎稱得起?須是你的觔斗方可。」行者
道:「我的觔斗,好道也是駕雲,只是去的有遠近些兒。你是馱不動,我卻如何
馱得動?自古道:『遣泰山輕如芥子,攜凡夫難脫紅塵。』像這潑魔毒怪,使攝
法,弄風頭,卻是扯扯拉拉,就地而行,不能帶得空中而去。像那樣法兒,老孫
也會使會弄。還有那隱身法、縮地法,老孫件件皆知。但只是師父要窮歷異邦,
不能勾超脫苦海,所以寸步難行也。我和你只做得個擁護,保得他身在命在,替
不得這些苦惱,也取不得經來﹔就是有能先去見了佛,那佛也不肯把經善與你我
。正叫做『若將容易得,便作等閑看』。」那獃子聞言,喏喏聽受。遂吃了些無
菜的素食,師徒們歇在流沙河東崖次之下。
  
次早,三藏道:「悟空,今日怎生區處?」行者道:「沒甚區處,還須八戒下水
。」八戒道:「哥哥,你要圖乾淨,只作成我下水。」行者道:「賢弟,這番我
再不急性了,只讓你引他上來,我攔住河沿,不讓他回去,務要將他擒了。」
  
好八戒,抹抹臉,抖搜精神,雙手拿鈀,到河沿,分開水路,依然又下至窩巢。
那怪方才睡醒,忽聽推得水響,急回頭睜睛觀看,見八戒執鈀來至。他跳出來,
當頭阻住,喝道:「慢來,慢來,看杖。」八戒舉鈀架住道:「你是個甚麼哭喪
杖,斷叫你祖宗看杖?」那怪道:「你這廝甚不曉得哩。我這:
    寶杖原來名譽大,本是月裏梭羅派。
    吳剛伐下一枝來,魯班製造工夫蓋。
    裏邊一條金趁心,外邊萬道珠絲玠。
    名稱寶杖善降妖,永鎮靈霄能伏怪。
    只因官拜大將軍,玉皇賜我隨身帶。
    或長或短任吾心,要細要粗憑意態。
    也曾護駕宴蟠桃,也曾隨朝居上界。
    值殿曾經眾聖參,捲簾曾見諸仙拜。
    養成靈性一神兵,不是人間凡器械。
    自從遭貶下天門,任意縱橫遊海外。
    不當大膽自稱誇,天下槍刀難比賽。
    看你那個鏽釘鈀,只好鋤田與築菜。」

  
八戒笑道:「我把你少打的潑物,且莫管甚麼築菜,只怕蕩了一下兒,教你沒處
貼膏藥,九個眼子一齊流血。縱然不死,也是個到老的破傷風。」那怪丟開架手
,在那水底下,與八戒依然打出水面。這一番鬥,比前果更不同,你看他:
寶杖掄,釘鈀築,言語不通非眷屬。只因木母剋刀圭,致令兩下相戰觸。沒輸贏
,無反覆,翻波淘浪不和睦。這個怒氣怎含容,那個傷心難忍辱。鈀來杖架逞英
雄,水滾流沙能惡毒。氣昂昂,勞碌碌,多因三藏朝西域。釘鈀老大兇,寶杖十
分熟。這個揪住要往岸上拖,那個抓來就將水裏沃。聲如霹靂動魚龍,雲暗天昏
神鬼伏。
  
這一場,來來往往,鬥經三十回合,不見強弱。八戒又使個佯輸計,拖了鈀走。
那怪隨後又趕來,擁波捉浪,趕至崖邊。八戒罵道:「我把你這個潑怪,你上來
,這高處,腳踏實地好打。」那妖罵道:「你這廝哄我上去,又教那幫手來哩。
你下來,還在水裏相鬥。」原來那妖乖了,再不肯上岸,只在河沿與八戒鬧吵。
  卻說行者見他不肯上岸,急得他心焦性爆,恨不得一把捉來。行者道:「師
父,你自坐下,等我與他個『餓鷹叼食』。」就縱觔斗,跳在半空,刷的落下來
,要抓那妖。那妖正與八戒嚷鬧,忽聽得風響,急回頭,見是行者落下雲來,卻
又收了那杖,一頭淬下水,隱跡潛蹤,渺然不見。行者佇立岸上,對八戒說:
「兄弟啞,這妖也弄得滑了,他再不肯上岸,如之奈何?」八戒道:「難,難,
難,戰不勝他。就把吃奶的氣力也使盡了,只繃得個手平。」行者道:「且見師
父去。」
  
二人又到高岸,見了唐僧,備言難捉。那長老滿眼下淚道:「似此艱難,怎生得
渡?」行者道:「師父莫要煩惱。這怪深潛水底,其實難行。──八戒,你只在
此保守師父,再莫與他廝斗,等老孫往南海走走去來。」八戒道:「哥哥,你去
南海何幹?」行者道:「這取經的勾當,原是觀音菩薩﹔及脫解我等,也是觀音
菩薩。今日路阻流沙河,不能前進,不得他,怎生處治?等我去請他,還強如和
這妖精相鬥。」八戒道:「也是,也是。師兄,你去時,千萬與我上覆一聲:向
日多承指教。」三藏道:「悟空,若是去請菩薩,卻也不必遲疑,快去快來。」
  
行者即縱觔斗雲,徑上南海。咦!那消半個時辰,早望見普陀山境。須臾間,墜
下觔斗,到紫竹林外,又只見那二十四路諸天上前迎著道:「大聖何來?」行者
道:「我師有難,特來謁見菩薩。」諸天道:「請坐,容報。」那輪日的諸天徑
至潮音洞口報道:「孫悟空有事朝見。」菩薩正與捧珠龍女在寶蓮池畔扶欄看花
,聞報,即轉雲巖,開門喚入。大聖端肅皈依參拜。
  
菩薩問曰:「你怎麼不保唐僧,為甚事又來見我?」行者啟上道:「菩薩,我師
父前在高老莊,又收了一個徒弟,喚名豬八戒,多蒙菩薩又賜法諱悟能。才行過
黃風嶺,今至八百里流沙河,乃是弱水三千,師父已是難渡﹔河中又有個妖怪,
武藝高強,甚虧了悟能與他水面上大戰三次,只是不能取勝,被他攔阻,不能渡
河。因此,特告菩薩,望垂憐憫,濟渡他一濟渡。」菩薩道:「你這猴子,又逞
自滿,不肯說出保唐僧的話來麼?」行者道:「我們只是要拿住他,教他送我師
父渡河。水裏事,我又弄不得精細。只是悟能尋著他窩巢,與他打話,想是不曾
說出取經的勾當。」菩薩道:「那流沙河的妖怪,乃是捲簾大將臨凡,也是我勸
化的善信,教他保護取經之輩。你若肯說出是東土取經人時,他決不與你爭持,
斷然歸順矣。」行者道:「那怪如今怯戰,不肯上崖,只在水裏潛蹤,如何得他
歸順?我師如何得渡弱水?」菩薩即喚惠岸,袖中取出一個紅葫蘆兒,吩咐道:
「你可將此葫蘆,同孫悟空到流沙河水面上,只叫『悟淨』,他就出來了。先要
引他歸依了唐僧。然後把他那九個骷髏穿在一處,按九宮佈列,卻把這葫蘆安在
當中,就是法船一隻,能渡唐僧過流沙河界。」
  
惠岸聞言,謹遵師命,與大聖捧葫蘆出了潮音洞,奉法旨辭了紫竹林。有詩為證。
    五行匹配合天真,認得從前舊主人。
    煉已立基為妙用,辨明邪正見原因。
    金來歸性還同類,木去求情共復淪。
    二土全功成寂寞,調和水火沒纖塵。

  
他兩個不多時,按落雲頭,早來到流沙河岸。豬八戒認得是木叉行者,引師父上
前迎接。那木叉與三藏禮畢,又與八戒相見。八戒道:「向蒙尊者指示,得見菩
薩,我老豬果遵法教,今喜拜了沙門。這一向在途中奔碌,未及致謝,恕罪,恕
罪。」行者道:「且莫敘闊,我們叫喚那廝去來。」三藏道:「叫誰?」行者道
:「老孫見菩薩,備陳前事。菩薩說,這流沙河的妖怪,乃是捲簾大將臨凡,因
為在天有罪,墮落此河,忘形作怪。他曾被菩薩勸化,願歸師父往西天去的。但
是我們不曾說出取經的事情,故此苦苦爭鬥。菩薩今差木叉將此葫蘆,要與這廝
結作法船,渡你過去哩。」三藏聞言,頂禮不盡,對木叉作禮道:「萬望尊者作
速一行。」那木叉捧定葫蘆,半雲半霧,徑到了流沙河水面上,厲聲高叫道:
「悟淨,悟淨,取經人在此久矣,你怎麼還不歸順?」
  
卻說那怪懼怕猴王,回於水底,正在窩中歇息,只聽得叫他法名。情知是觀音菩
薩﹔又聞得說「取經人在此」:他也不懼斧鉞,急翻波伸出頭來,又認得是木叉
行者。你看他笑盈盈,上前作禮道:「尊者失迎。菩薩今在何處?」木叉道:
「我師未來,先差我來吩咐你早跟唐僧做個徒弟。叫把你項下掛的骷髏與這個葫
蘆,按九宮結做一隻法船,渡他過此弱水。」悟淨道:「取經人卻在那裏?」木
叉用手指道:「那東岸上坐的不是?」悟淨看見了八戒道:「他不知是那裏來的
個潑物,與我整鬥了這兩日,何曾言著一個取經的字兒?」又看見行者,道:
「這個主子,是他的幫手,好不利害,我不去了。」木叉道:「那是豬八戒,這
是孫行者,俱是唐僧的徒弟,俱是菩薩勸化的,怕他怎的?我且和你見唐僧去。」
  
那悟淨才收了寶杖,整一整黃錦直裰,跳上岸來,對唐僧雙膝跪下道:「師父,
弟子有眼無珠,不認得師父的尊容,多有沖撞,萬望恕罪。」八戒道:「你這膿
包,怎的早不皈依,只管要與我打?是何說話?」行者笑道:「兄弟,你莫怪他
,還是我們不曾說出取經的事情與姓名耳。」長老道:「你果肯誠心皈依吾教麼
?」悟淨道:「弟子向蒙菩薩教化,指沙為姓,與我起個法名,喚做沙悟淨,豈
有不從師父之理?」三藏道:「既如此,」叫:「悟空,取戒刀來,與他落了髮
。」大聖依言,即將戒刀與他剃了頭。又來拜了三藏,拜了行者與八戒,分了大
小。三藏見他行禮真像個和尚家風,故又叫他做沙和尚。木叉道:「既秉了迦持
,不必絮煩,早與作法船去來。」
  
那悟淨不敢怠慢,即將頸項下掛的骷髏取下,用索子結作九宮,把菩薩葫蘆安在
當中,請師父下岸。那長老遂登法船,坐於上面,果然穩似輕舟。左有八戒扶持
,右有悟淨捧托﹔孫行者在後面牽了龍馬,半雲半霧相跟﹔頭直上又有木叉擁護
。那師父才飄然穩渡流沙河界,浪靜風平過弱河。真個也如飛似箭,不多時,身
登彼岸,得脫洪波﹔又不拖泥帶水,幸喜腳乾手燥,清淨無為,師徒們腳踏實地
。那木叉按祥雲,收了葫蘆。又只見那骷髏一時解化作九股陰風,寂然不見。三
藏拜謝了木叉,頂禮了菩薩。正是:
木叉徑回東洋海,三藏上馬卻投西。
  
    畢竟不知幾時才得正果求經,且聽下回分解。


}  \end{pinyinscope}\switchcolumn{\myfontc \section{第 二 二 回} , 八 戒 大 战 流 沙 河 , 木 叉 奉 法 收 悟 净 , 说 : 唐 僧 师 徒 三 众 脱 难 前 来 , 不 一 天 就 走 过 黄 风 岭 , 向 西 后 退 却 是 一 条 山 脉 平 阳 的 地 方 。
光 阴 迅 速 , 经 历 夏 天 经 过 秋 天 , 看 见 了 一 些 寒 蝉 鸣 叫 败 柳 , 大 火 向 西 流 去 。
正 走 的 地 方 , 只 见 一 道 大 水 狂 澜 , 浑 波 涌 浪 。
三 藏 在 马 上 急 忙 喊 道 : 徒 弟 , 你 看 那 前 边 的 水 势 宽 阔 , 怎 么 不 见 船 只 行 走 , 我 们 从 哪 里 过 去 呢 ? 八 戒 见 了 , 说 : 果 然 是 狂 澜 , 没 有 船 可 渡 。
那 个 行 人 跳 在 空 中 , 用 手 搭 着 凉 篷 看 , 他 心 惊 地 说 : 师 父 呵 , 真 的 是 难 , 真 的 是 难 。
这 条 河 如 果 论 老 孙 去 呵 , 只 要 把 腰 儿 扭 一 圈 , 就 能 过 去 了 。 如 果 师 父 , 确 实 千 分 难 以 渡 过 , 万 年 难 以 渡 过 。
三 藏 菩 萨 说 : 我 这 里 一 望 无 边 , 哪 里 有 多 少 宽 广 ? 行 人 说 : 经 过 有 八 百 里 远 近 。
八 戒 说 : 哥 哥 怎 么 定 得 个 远 近 之 数 呢 ? 行 者 说 : 不 瞒 你 说 , 老 子 这 双 眼 , 白 日 里 常 常 看 得 千 里 路 上 的 吉 凶 。
这 条 河 上 下 不 知 道 多 远 , 只 见 此 河 经 过 足 有 八 百 里 。
长 老 们 忧 虑 叹 息 烦 恼 , 兜 着 马 回 去 , 忽 然 看 见 岸 上 有 一 道 石 碑 。
三 个 人 一 齐 来 看 , 看 见 上 面 有 三 个 篆 字 , 是 流 沙 河 , 腹 上 有 个 小 小 的 四 行 真 字 说 : 八 百 流 沙 界 , 三 千 弱 水 深 。
鹅 毛 飘 不 起 , 芦 花 定 底 沉 。
师 徒 们 正 在 看 碑 文 , 只 听 到 它 的 浪 涌 如 山 , 波 翻 如 岭 , 河 水 当 中 滑 辣 地 钻 出 一 个 妖 精 , 十 分 凶 恶 。
不 黑 不 青 , 蓝 靛 脸 , 如 雷 如 鼓 , 老 龙 的 声 音 。
身 披 一 领 鹅 黄 , 腰 束 双 攒 , 露 出 白 藤 。
脖 子 下 的 骷 髅 悬 挂 着 九 个 , 手 里 拿 着 宝 杖 , 非 常 肮 脏 。
那 怪 怪 一 个 旋 风 , 急 忙 上 岸 来 , 径 直 去 抢 唐 僧 。
忽 然 有 个 走 路 的 人 把 师 父 抱 住 , 急 忙 登 上 高 岸 , 回 身 逃 脱 。
那 八 戒 放 下 担 子 , 拿 出 钉 镣 , 希 望 妖 精 就 筑 起 来 。
那 怪 物 就 让 宝 杖 架 住 。
其 他 两 个 人 在 流 沙 河 岸 , 各 自 逞 雄 。
这 一 场 好 战 : 九 齿 钢 , 降 妖 杖 , 二 人 在 河 岸 上 相 对 。
其 所 谓 是 总 督 大 天 蓬 , 其 所 谓 是 下 卷 帘 将 。
昔 年 曾 经 在 灵 霄 聚 会 , 今 天 争 着 拿 着 赌 博 勇 猛 壮 健 壮 。
这 一 个 砒 去 探 爪 龙 , 那 一 个 杖 架 磨 牙 象 。
伸 开 大 四 平 , 钻 进 迎 风 。
这 个 没 头 没 脸 , 那 个 无 乱 无 空 放 。
一 个 是 长 久 占 据 流 沙 界 吃 人 精 , 一 个 是 秉 持 教 诲 迦 持 修 行 将 。
其 他 两 个 人 来 来 往 往 , 交 战 二 十 回 合 , 不 分 胜 负 。
那 大 圣 保 护 唐 僧 , 牵 着 马 , 守 定 行 李 。
见 八 戒 和 那 怪 交 战 , 就 恨 得 咬 牙 切 齿 , 忍 不 住 地 要 去 打 他 , 拿 出 棒 来 说 : 师 父 , 你 坐 在 这 里 , 不 要 害 怕 。
等 老 子 和 他 戏 , 戏 儿 来 。
那 师 父 苦 苦 挽 留 不 住 。
其 他 人 , 不 知 其 所 有 , 不 知 其 所 有 , 不 知 其 所 有 , 不 知 其 所 有 , 不 知 其 所 有 , 不 知 其 所 有 也 。
何 怪 与 八 戒 正 在 交 战 , 难 解 难 分 。
被 走 的 人 拿 起 铁 棒 , 望 着 那 怪 落 在 头 上 一 下 , 那 怪 急 忙 转 身 , 慌 忙 逃 过 , 径 直 钻 进 流 沙 河 里 。
他 的 气 得 了 八 戒 乱 跳 道 : 哥 呵 , 谁 带 着 你 来 的 , 那 怪 渐 渐 手 慢 , 难 以 架 住 我 的 硬 硬 的 硬 硬 , 再 不 上 三 五 合 , 我 就 捉 住 他 了 。
行 者 笑 着 说 : 兄 弟 , 实 在 不 骗 你 说 , 自 从 投 降 了 黄 风 怪 , 下 山 来 , 这 个 把 月 不 曾 弄 棍 子 , 我 见 你 与 他 交 战 的 甜 美 , 我 就 忍 不 住 脚 痒 , 所 以 跳 出 来 玩 弄 。
他 知 道 那 怪 不 认 识 , 就 走 了 。
其 两 个 人 扶 着 手 , 说 笑 笑 , 转 转 回 来 见 唐 僧 。
唐 僧 说 : 你 曾 经 捉 到 一 个 妖 怪 吗 ? 行 人 说 : 那 妖 怪 没 有 办 法 作 战 , 打 败 了 回 来 , 钻 进 水 中 去 。
三 藏 说 : 徒 弟 , 这 怪 物 长 久 住 在 这 里 , 他 知 道 道 的 深 浅 。
像 这 样 无 边 弱 水 , 又 没 有 舟 楫 , 必 须 得 到 一 个 懂 得 水 性 的 引 领 才 好 。
行 人 说 : 正 是 这 种 说 法 。
他 常 常 说 道 : 接 近 草 的 是 红 色 , 接 近 墨 线 的 是 黑 色 。
那 怪 物 就 在 这 里 了 解 水 的 本 性 。
吾 今 把 他 捉 住 , 暂 且 不 要 打 死 , 只 好 让 他 送 师 父 过 河 , 再 做 理 会 。
八 戒 说 : 哥 哥 不 必 迟 疑 , 让 你 先 去 捉 他 , 等 老 猪 看 守 师 父 。
行 人 笑 着 说 : 贤 弟 啊 , 这 个 小 孩 我 不 敢 说 嘴 , 水 里 勾 当 , 老 孙 不 大 十 分 熟 。
如 果 是 空 走 , 还 要 念 咒 咒 , 又 念 念 避 水 咒 , 方 才 能 走 得 , 不 然 , 就 要 变 化 做 什 么 鱼 虾 蟹 蟹 蟹 蟹 蟹 蟹 蟹 鳖 之 类 , 我 才 能 去 掉 。
如 果 是 论 赌 的 手 段 , 凭 你 在 高 山 云 雾 中 , 干 什 么 小 事 , 老 子 都 会 ; 只 是 水 里 的 买 卖 , 有 些 儿 榔 榔 。
八 戒 说 : 老 猪 当 年 总 督 天 河 , 掌 管 八 万 水 兵 大 众 , 倒 学 得 知 道 水 性 。
又 怕 那 水 里 有 什 么 家 族 老 小 , 七 窝 八 代 都 来 , 我 就 弄 他 不 过 , 一 时 不 被 他 捉 去 吗 ? 行 人 说 : 你 如 果 到 他 水 中 与 他 交 战 , 还 不 要 恋 战 , 许 诺 失 败 不 许 取 胜 , 把 他 引 出 来 , 等 老 孙 下 手 帮 助 你 。
八 戒 说 : 你 说 得 对 , 我 离 开 吗 ?
说 完 就 去 , 就 剥 下 青 锦 直 , 脱 掉 鞋 子 , 双 手 舞 动 钢 子 , 分 开 水 路 , 让 他 出 那 当 年 的 手 段 , 跃 浪 翻 波 , 撞 着 进 去 , 径 直 到 水 底 下 , 往 前 正 走 。
又 说 : 那 怪 物 败 了 阵 回 , 方 才 喘 息 , 又 听 到 有 人 推 得 水 声 。
忽 然 起 身 观 看 , 原 来 是 八 戒 手 里 拿 着 硬 硬 的 钢 子 推 水 。
那 怪 举 起 杖 子 当 面 高 喊 道 : 那 和 尚 , 往 哪 里 走 , 仔 细 看 打 。
八 戒 让 钢 链 搭 住 道 : 你 是 什 么 妖 精 , 竟 敢 在 这 里 阻 挡 道 路 , 那 妖 说 : 你 是 也 不 认 不 出 我 。
我 不 是 你 的 妖 魔 鬼 怪 , 也 不 是 你 的 少 姓 无 名 。
八 戒 说 : 你 既 然 不 是 妖 魔 鬼 怪 , 又 何 必 生 在 这 里 伤 害 生 命 , 你 端 的 姓 名 , 实 在 说 来 , 我 饶 你 的 性 命 。
李 那 奇 怪 地 说 : 我 从 小 生 以 来 , 神 气 旺 盛 , 天 地 万 里 曾 经 游 历 过 。
英 雄 天 下 显 赫 威 名 , 豪 杰 人 家 作 模 模 。
万 国 九 州 任 凭 我 行 , 五 湖 四 海 跟 着 我 去 。
都 是 因 为 学 道 而 荡 荡 天 涯 , 只 是 为 了 寻 找 老 师 游 览 地 旷 。
常 年 衣 钵 谨 慎 随 身 , 每 天 心 神 不 可 放 。
沿 着 地 方 云 游 数 十 遍 , 到 处 闲 行 一 百 多 次 。
因 此 才 得 以 遇 到 真 人 , 引 导 开 导 大 道 金 光 亮 。
先 把 婴 儿 、 嫣 女 收 起 来 , 然 后 把 木 母 金 公 放 了 。
明 堂 的 肾 水 流 入 华 池 , 重 楼 的 肝 火 投 入 心 脏 。
三 千 功 满 拜 天 颜 , 志 在 朝 礼 明 光 。
玉 皇 大 帝 就 加 封 他 , 亲 口 封 他 为 卷 帘 将 。
南 天 门 里 我 为 尊 , 灵 霄 殿 前 我 称 上 。
腰 间 挂 挂 虎 头 牌 , 手 中 拿 着 定 降 妖 杖 。
头 戴 金 盔 , 光 闪 日 光 , 身 披 铠 甲 , 光 辉 明 亮 。
往 来 护 驾 , 我 应 当 先 行 , 出 入 随 朝 , 我 在 上 。
只 因 为 王 母 降 蟠 桃 , 在 瑶 池 设 宴 邀 请 众 将 。
失 手 打 破 玉 琉 璃 , 天 神 各 个 魂 飞 而 死 。
玉 皇 立 即 发 怒 , 生 出 怨 恨 , 却 让 掌 朝 左 辅 相 说 : 脱 下 官 帽 , 摘 下 官 衔 , 将 自 己 推 到 杀 场 上 。
多 亏 赤 脚 大 天 仙 , 越 班 启 奏 , 将 我 放 了 。
饶 死 回 生 不 点 刑 , 遭 贬 流 沙 东 岸 上 。
吃 饱 了 时 候 困 倦 躺 在 这 座 山 中 , 饿 了 就 翻 波 寻 找 粮 饷 。
樵 夫 遇 到 我 的 命 不 存 , 渔 翁 见 到 我 的 身 体 都 丧 失 了 。
来 来 往 往 , 吃 人 多 , 翻 翻 覆 覆 伤 生 瘴 气 。
你 胆 敢 行 凶 到 我 家 , 今 天 我 的 肚 皮 有 所 希 望 。
不 要 说 粗 糙 不 堪 尝 , 拿 住 消 停 , 剜 酱 。
八 戒 听 了 大 怒 , 骂 道 : 你 这 泼 东 西 , 全 都 没 有 一 点 儿 的 眼 色 。
我 老 猪 还 掐 出 水 沫 来 哩 , 你 怎 么 敢 说 我 粗 糙 , 要 杀 酱 看 起 来 , 你 把 我 认 作 个 老 走 硝 哩 。
不 要 无 礼 , 吃 你 祖 宗 这 一 锭 。
那 怪 物 看 见 钢 子 来 了 , 让 一 个 凤 点 头 躲 过 去 。
两 个 人 在 水 中 打 出 水 面 , 各 自 踏 着 浪 子 上 波 。
这 一 场 赌 斗 , 与 以 前 不 同 , 你 看 那 卷 帘 将 , 天 蓬 帅 , 各 显 神 通 , 真 可 爱 。
其 降 妖 宝 杖 插 在 头 轮 上 , 这 个 九 齿 钉 钢 , 随 手 快 。
波 浪 振 动 山 川 , 推 波 荡 荡 世 界 。
凶 恶 如 太 岁 撞 打 幢 幡 , 凶 恶 似 丧 门 , 掀 起 宝 盖 。
这 一 个 忠 心 耿 耿 保 护 唐 僧 , 那 一 个 犯 罪 滔 滔 成 为 水 怪 。
硬 硬 地 用 棍 棒 打 他 , 他 的 魂 魄 就 会 败 坏 了 。
努 力 喜 欢 相 持 , 用 心 想 赌 赛 。
算 来 只 是 为 了 取 经 人 , 怒 气 冲 天 , 不 忍 耐 。
搅 得 那 、 鱼 、 鱼 、 鱼 、 鲜 鱼 、 鲜 鱼 、 鱼 、 鱼 、 鱼 、 鱼 、 鱼 、 鱼 、 鱼 、 鱼 、 鱼 、 蟹 都 死 了 , 水 府 诸 神 都 朝 上 拜 。
只 听 到 波 浪 翻 腾 , 似 雷 鸣 , 日 月 无 光 , 天 地 奇 怪 。
两 个 人 整 顿 战 斗 有 两 个 时 辰 , 不 分 胜 败 。
这 才 是 铜 盆 碰 上 铁 帚 , 玉 磬 对 着 金 钟 。
又 说 : 大 圣 保 护 着 唐 僧 , 站 在 左 右 , 眼 睛 巴 巴 , 望 着 他 两 个 人 在 水 上 争 持 , 只 是 他 不 好 动 手 。
只 见 那 八 戒 虚 张 一 根 铁 链 , 假 装 假 装 欺 骗 , 转 而 回 头 到 东 岸 上 走 。
其 怪 , 随 后 赶 来 , 快 到 岸 边 。
这 个 行 人 忍 耐 不 住 , 抛 了 师 父 , 拿 着 铁 棒 , 跳 到 河 边 , 望 妖 精 劈 开 头 就 打 。
其 妖 异 之 物 不 敢 迎 接 , 冷 冷 地 又 钻 进 河 内 。
八 戒 大 声 喊 道 : 你 这 个 弼 马 温 , 彻 是 个 急 猴 子 , 你 再 缓 缓 些 儿 , 等 我 把 他 们 哄 到 高 处 , 你 却 阻 住 河 边 , 教 他 不 能 回 头 呵 , 却 不 拿 住 他 , 他 这 进 去 , 几 时 又 肯 出 来 ? 行 人 笑 着 说 : 子 , 不 要 吵 , 不 要 吵 。
我 们 暂 且 回 去 , 见 到 师 父 去 来 。
八 戒 回 去 , 同 行 的 人 到 高 岸 上 , 见 到 了 三 藏 菩 萨 。
三 藏 拍 着 身 子 说 : 徒 弟 辛 苦 啊 。
八 戒 说 : 暂 且 不 说 辛 苦 , 只 是 降 了 妖 精 , 送 得 你 过 河 , 才 是 万 全 之 策 。
三 藏 说 : 你 刚 刚 与 妖 精 交 战 怎 么 样 ? 八 戒 说 : 那 妖 怪 的 手 段 , 与 老 猪 是 对 手 。
使 之 , 使 之 不 得 , 不 可 得 也 。
见 师 兄 举 棍 子 , 他 就 逃 走 了 。
三 藏 菩 萨 说 : 如 果 这 样 , 怎 么 办 呢 ? 行 者 说 : 师 父 放 心 , 暂 且 不 要 焦 虑 烦 恼 。
如 今 天 色 又 晚 , 暂 且 坐 在 这 里 的 崖 岸 上 , 等 老 孙 去 教 化 一 点 斋 饭 来 , 你 吃 了 睡 去 , 等 明 天 再 去 吧 。
八 戒 说 : 说 得 对 , 你 快 去 快 来 。
行 人 急 忙 放 着 云 跳 起 去 , 正 赶 到 直 北 下 人 家 化 了 一 钵 素 斋 , 回 来 献 给 师 父 。
师 父 见 他 来 得 得 很 快 , 便 叫 道 : 我 去 化 斋 的 人 家 , 请 问 他 一 个 过 河 的 计 策 , 不 勉 强 好 像 跟 这 怪 物 争 持 吗 ? 行 人 笑 着 说 : 这 家 子 远 得 狠 哩 , 相 距 有 五 七 千 里 路 , 他 哪 里 得 知 水 性 , 问 他 有 什 么 益 处 ? 八 戒 说 : 哥 哥 又 来 扯 谎 了 , 五 七 千 里 路 , 你 怎 么 这 样 去 来 得 快 吗 ? 行 人 说 : 你 哪 里 知 道 老 孙 ?
像 这 样 五 七 千 路 , 只 要 把 头 点 上 两 点 , 把 腰 上 一 个 身 , 就 是 个 往 回 , 有 什 么 难? 八 戒 说 : 哥 呵 , 既 然 是 这 样 容 易 , 你 把 师 父 背 着 , 只 需 点 点 头 , 躬 身 腰 , 跳 过 去 罢 了 , 何 必 苦 苦 地 与 这 个 怪 物 交 战 呢 ? 行 人 说 : 你 不 会 驾 云 , 你 把 师 父 过 去 不 是 吗 ? 八 戒 说 : 师 父 的 凡 胎 肉 骨 , 重 重 似 泰 山 , 我 。
行 人 说 : 我 的 斗 , 好 道 也 是 驾 云 , 只 是 去 的 有 远 近 。
你 是 不 动 , 我 又 怎 么 能 动 摇 呢 自 古 以 来 有 道 : 遣 泰 山 轻 如 芥 子 , 带 着 凡 夫 难 以 脱 离 尘 世 。
像 这 样 的 泼 魔 毒 怪 , 让 他 摄 法 , 玩 弄 风 头 , 却 是 摇 摆 摆 , 顺 着 地 走 , 不 能 把 它 带 到 空 中 去 。
像 这 样 的 儿 子 , 老 孙 子 也 会 让 他 们 做 。
还 有 那 隐 身 法 、 缩 地 法 , 老 孙 子 件 件 都 知 道 。
只 是 师 父 要 穷 途 异 国 , 不 能 超 脱 苦 海 , 所 以 寸 步 难 行 。
我 和 你 只 做 得 个 拥 护 , 保 得 他 的 身 在 命 中 , 替 不 得 这 些 苦 恼 , 也 取 不 到 经 书 来 , 就 是 有 能 先 去 见 佛 , 那 么 佛 也 不 肯 把 经 书 好 好 给 你 。
正 好 叫 做 : 如 果 将 容 易 得 到 , 就 作 等 闲 看 。
那 子 听 了 他 的 话 , 就 恭 敬 地 听 了 。
于 是 吃 了 些 无 菜 的 素 食 , 师 徒 们 在 流 沙 河 东 边 的 山 崖 下 歇 息 。
第 二 天 早 晨 , 三 藏 菩 萨 说 : 悟 空 , 今 天 哪 里 产 生 区 处 ? 行 者 说 : 没 有 什 么 区 处 , 还 需 要 八 戒 下 水 。
八 戒 说 : 哥 哥 , 你 要 想 干 净 , 只 做 成 我 下 水 。
行 人 说 : 贤 弟 , 这 次 我 再 不 急 性 了 , 只 让 你 引 他 上 来 , 我 拦 住 河 边 , 不 让 他 回 去 , 一 定 要 把 他 擒 获 。
喜 欢 八 戒 , 抹 抹 脸 , 抖 擞 精 神 , 双 手 拿 着 铜 器 , 到 河 边 , 分 开 水 路 , 依 然 又 下 到 窝 巢 。
那 怪 鬼 刚 刚 睡 醒 , 忽 然 听 到 推 出 水 的 声 音 , 急 忙 回 头 , 睁 着 眼 睛 看 , 看 见 八 戒 手 里 拿 着 钢 子 来 到 。
他 跳 出 来 , 当 头 挡 住 , 喝 道 : 慢 来 , 慢 来 , 看 看 杖 子 。
八 戒 举 起 硬 硬 的 硬 硬 的 硬 硬 的 硬 硬 的 硬 硬 的 硬 硬 地 说 : 你 是 什 么 哭 丧 杖 , 你 断 绝 叫 你 祖 宗 看 杖 , 那 怪 道 : 你 这 个 人 怎 么 不 晓 得 呀 !
我 说 : 宝 杖 原 来 名 誉 大 , 本 是 月 里 的 梭 罗 派 。
吴 刚 砍 下 一 枝 来 , 鲁 班 制 造 工 夫 盖 。
里 边 一 条 金 趁 心 , 外 边 万 道 珠 丝 。
他 的 名 字 叫 宝 杖 , 善 于 降 魔 , 永 远 镇 守 灵 霄 , 能 够 埋 伏 妖 怪 。
只 因 为 官 拜 大 将 军 , 玉 皇 赐 给 我 随 身 带 。
有 的 长 处 有 的 短 处 , 都 由 我 的 心 思 去 做 , 要 细 处 要 粗 的 , 要 根 据 自 己 的 意 思 来 表 达 。
也 曾 护 驾 宴 饮 蟠 桃 , 也 曾 随 朝 廷 居 住 在 上 界 。
正 赶 上 大 殿 曾 经 经 过 众 位 圣 人 参 拜 , 卷 起 帘 子 曾 见 到 众 位 仙 人 拜 见 。
养 成 灵 性 是 一 种 神 兵 , 不 是 人 间 的 一 般 器 械 。
自 从 遭 贬 下 天 门 , 任 意 纵 横 游 览 海 外 。
不 应 当 大 胆 自 夸 , 天 下 的 刀 剑 难 以 比 赛 。
看 你 的 那 个 锈 钉 钢 , 只 好 锄 田 和 筑 菜 。
八 戒 笑 着 说 : 我 把 你 少 打 的 泼 物 , 况 且 不 管 什 么 筑 菜 , 只 怕 荡 了 一 下 儿 , 教 你 没 地 方 贴 膏 药 , 九 个 眼 睛 一 齐 流 血 。
即 使 不 死 , 也 是 一 个 到 老 的 破 伤 风 。
那 怪 物 抛 开 了 架 子 的 手 , 放 在 水 底 下 , 和 八 戒 还 在 水 面 上 打 出 来 。
这 一 次 战 斗 , 比 以 前 果 然 更 不 同 , 你 看 他 说 : 宝 杖 , 钉 钢 筑 , 言 语 不 通 , 不 是 我 的 亲 属 。
只 因 为 木 母 克 制 刀 圭 , 以 致 使 两 下 相 战 。
没 有 赢 利 , 没 有 反 复 , 翻 波 滚 浪 , 不 和 睦 。
这 个 怒 气 怎 么 含 容 , 那 个 伤 心 难 以 忍 辱 。
钢 来 杖 架 逞 英 雄 , 水 滚 流 沙 能 恶 毒 。
气 势 昂 昂 , 劳 碌 碌 碌 , 多 凭 借 三 藏 去 西 域 。
钉 铃 老 大 凶 险 , 宝 杖 十 分 熟 。
其 中 有 一 个 人 , 一 个 人 , 一 个 人 , 一 个 人 , 一 个 人 , 一 个 人 , 一 个 人 , 一 个 人 , 一 个 人 , 一 个 人 , 一 个 人 , 一 个 人 , 一 个 人 , 一 个 人 , 一 个 人 , 一 个 人 , 一 个 人 , 一 个 人 , 一 个 人 , 一 个 人 , 一 个 人 , 一 个 人 , 一 个 人 , 一 个 人 , 一 个 人 , 一 个 人 , 一 个 人 , 一 个 人 , 一 个 人 , 一 个 人 , 一 个 人 , 一 个 人 , 一 个 子 一 个 人 , 一 个 人 都 能 把 它 也 。
声 如 霹 雳 震 动 鱼 龙 , 云 暗 天 昏 , 神 鬼 伏 伏 。
这 一 场 , 来 来 往 往 , 打 了 三 十 回 合 , 不 分 强 弱 。
八 戒 又 派 出 假 装 输 送 的 计 策 , 拖 着 钢 子 逃 走 。
那 怪 物 随 后 又 赶 来 , 拥 波 捉 浪 , 赶 到 崖 边 。
八 戒 骂 道 : 我 把 你 这 个 泼 怪 , 你 上 来 , 这 个 高 处 , 脚 踏 实 地 好 打 。
那 妖 人 骂 道 : 你 这 个 孩 子 哄 我 上 去 , 又 让 那 个 帮 手 来 吗 ?
你 下 来 , 还 在 水 里 互 相 争 斗 。
原 来 那 妖 怪 了 , 再 也 不 肯 上 岸 , 只 在 河 边 与 八 戒 闹 闹 。
又 说 : 行 人 见 他 不 肯 上 岸 , 急 得 他 心 焦 性 爆 , 恨 得 不 得 一 把 把 他 捉 来 。
行 人 说 : 师 父 , 你 自 己 坐 下 , 等 我 和 他 个 饿 鹰 吞 食 。
即 放 斗 , 跳 到 半 空 , 冲 刷 下 来 , 要 抓 那 妖 怪 。
那 妖 怪 正 在 和 八 戒 喧 闹 , 忽 然 听 到 风 声 , 急 忙 回 头 , 看 见 是 行 人 落 下 云 来 , 又 收 起 那 拐 杖 , 一 头 淬 下 水 , 隐 迹 潜 踪 , 渺 然 不 见 。
行 人 站 在 岸 上 , 对 八 戒 说 : 兄 弟 哑 , 这 妖 怪 也 弄 得 滑 了 , 他 再 不 肯 上 岸 , 怎 么 办 呢 ? 八 戒 说 : 难 , 难 , 难 , 战 不 胜 他 。
就 把 吃 奶 的 气 力 都 用 尽 了 , 只 扯 得 一 个 手 平 。
行 人 说 : 暂 且 见 到 师 父 去 吧 。
二 人 又 到 了 高 岸 , 见 到 唐 僧 , 都 说 难 以 捉 住 。
那 个 长 老 满 眼 泪 说 : 像 这 样 艰 难 , 怎 么 能 够 渡 过 去 呢 ? 行 人 说 : 师 父 , 不 要 烦 恼 。
这 怪 物 深 潜 于 水 底 , 其 实 难 以 行 走 。
八 戒 , 你 只 在 这 里 保 护 师 父 , 再 也 不 要 和 他 厮 打 , 等 老 孙 子 去 南 海 去 吧 。
八 戒 说 : 哥 哥 , 你 去 南 海 干 什 么 ? 行 者 说 : 这 是 取 经 的 , 原 来 是 观 音 菩 萨 , 等 到 解 脱 我 们 , 也 就 是 观 音 菩 萨 。
今 日 道 路 阻 隔 流 沙 河 , 不 能 前 进 , 得 不 到 其 他 地 方 , 怎 么 能 治 理 好 呢 等 我 去 请 他 , 还 强 如 和 这 妖 精 相 斗 。
八 戒 说 : 也 是 , 也 是 。
师 兄 , 你 去 的 时 候 , 千 万 给 我 上 覆 一 声 : 往 日 多 承 你 的 指 教 。
三 藏 菩 萨 说 : 悟 空 , 如 果 是 去 请 求 菩 萨 , 再 也 不 必 迟 疑 , 快 去 快 来 。
行 人 立 即 放 开 云 , 径 直 上 到 南 海 。
唉 , 那 么 半 个 时 辰 , 早 就 望 见 普 陀 山 的 境 界 。
不 一 会 儿 , 坠 下 一 斗 , 走 到 紫 竹 林 外 , 又 看 见 那 二 十 四 路 诸 天 上 前 迎 接 着 说 : 大 圣 从 哪 里 来 ? 行 者 说 : 我 师 父 有 难 , 特 来 拜 见 菩 萨 。
诸 天 说 : 请 坐 下 , 容 许 回 报 。
那 轮 日 的 诸 天 径 直 到 潮 音 洞 口 报 告 说 : 孙 慧 空 有 事 朝 见 。
菩 萨 正 和 捧 珠 的 龙 女 在 宝 莲 池 旁 扶 着 栏 杆 看 花 , 听 到 报 告 , 立 即 转 到 云 岩 , 打 开 门 把 他 叫 进 去 。
大 圣 端 庄 严 肃 , 虔 诚 地 参 拜 。
菩 萨 问 道 : 你 为 什 么 不 保 唐 僧 , 为 什 么 事 又 来 见 我 ? 行 者 启 奏 皇 上 说 : 菩 萨 , 我 师 父 前 在 高 老 庄 , 又 收 了 一 个 徒 弟 , 叫 名 叫 猪 八 戒 , 多 蒙 菩 萨 又 赐 法 名 叫 悟 能 。
刚 刚 走 过 黄 风 岭 , 今 天 到 了 八 百 里 的 流 沙 河 , 却 是 弱 水 三 千 里 , 师 父 已 经 难 以 渡 过 河 去 了 ; 河 中 又 有 个 妖 怪 , 武 艺 高 强 , 很 有 明 白 , 能 够 和 他 在 水 面 上 大 战 三 次 , 只 是 不 能 取 胜 , 被 他 拦 住 , 不 能 渡 河 。
因 此 , 特 地 告 诉 菩 萨 , 希 望 你 怜 悯 怜 悯 怜 悯 , 救 渡 他 一 渡 。
菩 萨 说 : 你 这 个 猴 子 , 又 逞 自 满 , 不 肯 说 出 保 护 唐 僧 的 话 来 吗 ? 行 者 说 : 我 们 只 是 要 抓 住 他 , 教 他 送 我 师 父 渡 河 。
水 里 的 事 , 我 又 弄 得 不 精 细 。
只 是 觉 得 能 够 寻 找 到 他 的 窝 窝 , 和 他 打 话 , 想 是 他 不 曾 说 出 自 己 的 经 书 。
菩 萨 说 : 那 么 流 沙 河 的 妖 怪 , 是 卷 帘 大 将 临 凡 , 也 是 我 劝 导 教 化 的 善 信 , 教 他 保 护 取 经 的 人 。
你 如 果 肯 说 出 自 东 土 取 经 人 的 时 候 , 他 决 不 会 与 你 争 执 , 决 然 归 顺 。
行 者 说 : 你 今 天 胆 怯 作 战 , 不 肯 上 山 崖 , 只 在 水 里 潜 踪 , 怎 么 能 够 他 归 顺 我 , 我 师 为 什 么 能 够 渡 过 弱 水 ? 菩 萨 立 即 叫 惠 岸 , 从 袖 中 取 出 一 个 红 葫 芦 , 吩 咐 说 : 你 可 以 把 这 个 葫 芦 , 和 孙 悟 空 一 起 到 流 沙 河 的 水 面 上 , 只 叫 悟 净 , 他 就 出 来 了 。
先 王 之 所 以 为 之 。
然 后 把 他 的 九 个 骷 髅 穿 在 一 个 地 方 , 按 照 九 宫 排 列 , 又 把 它 的 葫 芦 放 在 当 中 , 就 是 一 只 法 船 , 能 够 渡 过 唐 僧 经 过 流 沙 河 边 界 。
惠 岸 听 了 , 恭 敬 地 遵 从 师 傅 的 命 令 , 和 大 圣 捧 葫 芦 出 了 潮 音 洞 , 奉 法 旨 辞 去 紫 竹 林 。
有 诗 作 证 。
五 行 配 合 天 真 , 认 得 从 前 的 主 人 。
修 炼 自 己 的 根 基 , 作 为 妙 用 , 辨 明 邪 正 的 原 因 。
金 来 归 , 性 还 同 类 , 木 去 求 情 共 复 沦 。
两 土 完 全 成 就 了 寂 寞 , 调 和 水 火 , 消 除 纤 尘 。
其 他 两 个 人 不 多 时 , 按 落 在 云 头 上 , 早 早 来 到 流 沙 河 岸 。
猪 八 戒 认 为 是 木 叉 行 走 的 人 , 就 领 着 师 父 上 前 迎 接 。
那 木 叉 与 三 藏 菩 萨 礼 仪 完 毕 , 又 与 八 戒 相 见 。
八 戒 说 : 刚 才 承 蒙 尊 者 的 指 示 , 得 以 见 到 菩 萨 , 我 老 猪 果 然 遵 从 佛 法 的 教 导 , 现 在 我 喜 欢 拜 见 了 和 尚 。
这 一 向 在 途 中 奔 忙 , 还 没 来 得 及 致 谢 , 恕 罪 , 恕 罪 。
行 人 说 : 暂 且 不 要 叙 述 , 我 们 叫 唤 那 个 奴 仆 去 来 。
三 藏 菩 萨 说 : 你 叫 谁 ? 行 者 说 : 我 见 到 菩 萨 , 详 细 陈 述 以 前 的 事 情 。
菩 萨 说 : 这 个 流 沙 河 的 妖 怪 , 是 卷 帘 大 将 临 凡 , 因 为 在 天 上 有 罪 , 堕 落 在 这 个 河 里 , 忘 记 形 体 作 怪 。
他 曾 经 受 菩 萨 劝 化 , 愿 意 归 依 师 父 , 往 西 天 去 。
吾 闻 之 。
菩 萨 现 在 让 木 叉 带 着 这 个 葫 芦 , 要 和 这 个 人 结 成 法 船 , 渡 你 过 去 吧 。
三 藏 菩 萨 听 了 这 话 , 不 能 尽 礼 , 便 对 木 叉 作 礼 说 : 希 望 你 快 去 一 趟 。
那 木 叉 捧 定 葫 芦 , 半 云 半 雾 , 径 直 到 了 流 沙 河 的 水 面 上 , 厉 声 大 叫 道 : 悟 净 , 取 经 人 在 此 已 经 很 久 了 , 你 为 什 么 还 不 归 顺 呢 ? 又 说 那 怪 兽 害 怕 猴 王 , 回 到 水 底 , 正 在 窝 中 歇 息 , 只 听 到 叫 他 的 法 名 。
我 知 道 他 是 观 世 音 菩 萨 , 又 听 到 他 说 : 取 经 的 人 在 这 里 , 他 也 不 怕 斧 头 , 急 忙 翻 波 , 伸 出 头 来 , 又 认 出 他 是 木 叉 行 的 人 。
你 看 他 笑 得 满 盈 , 上 前 行 礼 说 : 尊 者 失 去 迎 接 。
木 叉 说 : 我 师 傅 还 没 有 来 , 先 让 我 来 , 吩 咐 你 早 早 跟 唐 僧 做 个 弟 弟 。
叫 你 脖 子 下 挂 的 骷 髅 和 葫 芦 , 按 九 宫 结 做 一 只 法 船 , 渡 过 它 的 弱 水 。
木 叉 用 手 指 着 说 : 那 东 岸 上 坐 的 不 是 ? 悟 净 看 见 了 八 戒 说 : 他 不 知 道 是

 }\switchcolumn\flushpage